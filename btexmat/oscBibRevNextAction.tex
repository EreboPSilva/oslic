% Telekom osCompendium English next action bibfile review file
%
% (c) Karsten Reincke, Deutsche Telekom AG, Darmstadt 2011
%
% This LaTeX-File is licensed under the Creative Commons Attribution-ShareAlike
% 3.0 Germany License (http://creativecommons.org/licenses/by-sa/3.0/de/): Feel
% free 'to share (to copy, distribute and transmit)' or 'to remix (to adapt)'
% it, if you '... distribute the resulting work under the same or similar
% license to this one' and if you respect how 'you must attribute the work in
% the manner specified by the author ...':
%
% In an internet based reuse please link the reused parts to www.telekom.com and
% mention the original authors and Deutsche Telekom AG in a suitable manner. In
% a paper-like reuse please insert a short hint to www.telekom.com and to the
% original authors and Deutsche Telekom AG into your preface. For normal
% quotations please use the scientific standard to cite.
%
% [ File structure derived from 'mind your Scholar Research Framework' 
%   mycsrf (c) K. Reincke CC BY 3.0  http://mycsrf.fodina.de/ ]
%
% select the document class
% S.26: [ 10pt|11pt|12pt onecolumn|twocolumn oneside|twoside notitlepage|titlepage final|draft
%         leqno fleqn openbib a4paper|a5paper|b5paper|letterpaper|legalpaper|executivepaper openrigth ]
% S.25: { article|report|book|letter ... }
%
% oder koma-skript S.10 + 16
\documentclass[DIV=calc,BCOR=5mm,11pt,headings=small,oneside,abstract=true,toc=bib]{scrartcl}

%%% (1) general configurations %%%
\usepackage[utf8]{inputenc}

%%% (2) language specific configurations %%%
\usepackage[]{a4,ngerman}
\usepackage[ngerman, german, english]{babel}
\selectlanguage{english}

%language specific quoting signs
%default for language emglish is american style of quotes
\usepackage{csquotes}

% jurabib configuration
\usepackage[see]{jurabib}
\bibliographystyle{jurabib}
% Telekom osCompendium English Jurabib Configuration Include Module 
%
% (c) Karsten Reincke, Deutsche Telekom AG, Darmstadt 2011
%
% This LaTeX-File is licensed under the Creative Commons Attribution-ShareAlike
% 3.0 Germany License (http://creativecommons.org/licenses/by-sa/3.0/de/): Feel
% free 'to share (to copy, distribute and transmit)' or 'to remix (to adapt)'
% it, if you '... distribute the resulting work under the same or similar
% license to this one' and if you respect how 'you must attribute the work in
% the manner specified by the author ...':
%
% In an internet based reuse please link the reused parts to www.telekom.com and
% mention the original authors and Deutsche Telekom AG in a suitable manner. In
% a paper-like reuse please insert a short hint to www.telekom.com and to the
% original authors and Deutsche Telekom AG into your preface. For normal
% quotations please use the scientific standard to cite.
%
% [ File structure derived from 'mind your Scholar Research Framework' 
%   mycsrf (c) K. Reincke CC BY 3.0  http://mycsrf.fodina.de/ ]

% the first time cite with all data, later with shorttitle
\jurabibsetup{citefull=first}

%%% (1) author / editor list configuration
%\jurabibsetup{authorformat=and} % uses 'und' instead of 'u.'
% therefore define your own abbreviated conjunction: 
% an 'and before last author explicetly written conjunction

% for authors in citations
\renewcommand*{\jbbtasep}{ a.\ } % bta = between two authors sep
\renewcommand*{\jbbfsasep}{, } % bfsa = between first and second author sep
\renewcommand*{\jbbstasep}{, a.\ }% bsta = between second and third author sep
% for editors in citations
\renewcommand*{\jbbtesep}{ a.\ } % bta = between two authors sep
\renewcommand*{\jbbfsesep}{, } % bfsa = between first and second author sep
\renewcommand*{\jbbstesep}{, a.\ }% bsta = between second and third author sep

% for authors in literature list
\renewcommand*{\bibbtasep}{ a.\ } % bta = between two authors sep
\renewcommand*{\bibbfsasep}{, } % bfsa = between first and second author sep
\renewcommand*{\bibbstasep}{, a.\ }% bsta = between second and third author sep
% for editors  in literature list
\renewcommand*{\bibbtesep}{ a.\ } % bte = between two editors sep
\renewcommand*{\bibbfsesep}{, } % bfse = between first and second editor sep
\renewcommand*{\bibbstesep}{, a.\ }% bste = between second and third editor sep

% use: name, forname, forname lastname u. forname lastname
\jurabibsetup{authorformat=firstnotreversed}
\jurabibsetup{authorformat=italic}

%%% (2) title configuration
% in every case print the title, let it be seperated from the 
% author by a colon and use the slanted font
\jurabibsetup{titleformat={all,colonsep}}
%\renewcommand*{\jbtitlefont}{\textit}

%%% (3) seperators in bib data
% separate bibliographical hints and page hints by a comma
\jurabibsetup{commabeforerest}

%%% (4) specific configuration of bibdata in quotes / footnote
% use a.a.O if possible
\jurabibsetup{ibidem=strict}
% replace ugly a.a.O. by translation of ders., a.a.O.
\AddTo\bibsgerman{
  \renewcommand*{\ibidemname}{Id., l.c.}
  \renewcommand*{\ibidemmidname}{id., l.c.}
}
\renewcommand*{\samepageibidemname}{Id., ibid.}
\renewcommand*{\samepageibidemmidname}{id., ibid.}


%%% (5) specific configuration of bibdata in bibliography
% ever an in: before journal and collection/book-tiltes 
\renewcommand*{\bibbtsep}{in: }
\renewcommand*{\bibjtsep}{in: }
% ever a colon after author names 
\renewcommand*{\bibansep}{: }
% ever a semi colon after the title
% \AddTo\bibsgerman{\renewcommand*{\urldatecomment}{Referenzdownload: }}
\renewcommand*{\bibatsep}{; }
% ever a comma before date/year
\renewcommand*{\bibbdsep}{, }

% let jurabib insert the S. and p. information
% no S. necessary in bib-files and in cites/footcites
\jurabibsetup{pages=format}

% use a compressed literature-list using a small line indent
\jurabibsetup{bibformat=compress}
\setlength{\jbbibhang}{1em}

% which follows the design of the cites and offers comments
\jurabibsetup{biblikecite}

% print annotations into bibliography
\jurabibsetup{annote}
\renewcommand*{\jbannoteformat}[1]{{ \itshape #1 }}

%refine the prefix of url download
\AddTo\bibsgerman{\renewcommand*{\urldatecomment}{reference download: }}

% we want to have the year of articles in brackets
\renewcommand*{\bibaldelim}{(}
\renewcommand*{\bibardelim}{)}

% in english version Nr. must be replaced by No.
\renewcommand*{\artnumberformat}[1]{\unskip,\space No.~#1}
\renewcommand*{\pernumberformat}[1]{\unskip\space No.~#1}%
\renewcommand*{\revnumberformat}[1]{\unskip\space No.~#1}%


%Reformatierung Seriestitels and Seriesnumber
\DeclareRobustCommand{\numberandseries}[2]{%
\unskip\unskip%,
\space\bibsnfont{(=~#2}%
\ifthenelse{\equal{#1}{}}{)}{, [Vol./No.]~#1)}%
}%


% Local Variables:
% mode: latex
% fill-column: 80
% End:


% language specific hyphenation
% Telekom osCompendium osHyphenation Include Module
%
% (c) Karsten Reincke, Deutsche Telekom AG, Darmstadt 2011
%
% This LaTeX-File is licensed under the Creative Commons Attribution-ShareAlike
% 3.0 Germany License (http://creativecommons.org/licenses/by-sa/3.0/de/): Feel
% free 'to share (to copy, distribute and transmit)' or 'to remix (to adapt)'
% it, if you '... distribute the resulting work under the same or similar
% license to this one' and if you respect how 'you must attribute the work in
% the manner specified by the author ...':
%
% In an internet based reuse please link the reused parts to www.telekom.com and
% mention the original authors and Deutsche Telekom AG in a suitable manner. In
% a paper-like reuse please insert a short hint to www.telekom.com and to the
% original authors and Deutsche Telekom AG into your preface. For normal
% quotations please use the scientific standard to cite.
%
% [ File structure derived from 'mind your Scholar Research Framework' 
%   mycsrf (c) K. Reincke CC BY 3.0  http://mycsrf.fodina.de/ ]
%


\hyphenation{rein-cke}
\hyphenation{OS-LiC}
\hyphenation{ori-gi-nal}


%%% (3) layout page configuration %%%

% select the visible parts of a page
% S.31: { plain|empty|headings|myheadings }
%\pagestyle{myheadings}
\pagestyle{headings}

% select the wished style of page-numbering
% S.32: { arabic,roman,Roman,alph,Alph }
\pagenumbering{arabic}
\setcounter{page}{1}

% select the wished distances using the general setlength order:
% S.34 { baselineskip| parskip | parindent }
% - general no indent for paragraphs
\setlength{\parindent}{0pt}
\setlength{\parskip}{1.2ex plus 0.2ex minus 0.2ex}


%%% (4) general package activation %%%
%\usepackage{utopia}
%\usepackage{courier}
%\usepackage{avant}
\usepackage[dvips]{epsfig}

% graphic
\usepackage{graphicx,color}
\usepackage{array}
\usepackage{shadow}
\usepackage{fancybox}

%- start(footnote-configuration)
%  flush the cite numbers out of the vertical line and let
%  the footnote text directly start in the left vertical line
\usepackage[marginal]{footmisc}
%- end(footnote-configuration)

\usepackage[intoc]{nomencl}
\let\abbr\nomenclature
% Modify Section Title of nomenclature
\renewcommand{\nomname}{Periodicals, Shortcuts, and Overlapping Abbreviations}
%\renewcommand{\nomname}{Periodika, ihre Kurzformen und generelle Abkürzungen}

% insert point between abbrewviation and explanation
\setlength{\nomlabelwidth}{.24\hsize}
\renewcommand{\nomlabel}[1]{#1 \dotfill}
% reduce the line distance
\setlength{\nomitemsep}{-\parsep}
\makenomenclature


\begin{document}

%% use all entries of the bliography
\nocite{*}

%%-- start(titlepage)
\subject{Open Source secondary literature, collected for the OSLiC}
\title{Unverified Literature (Data) in the file 'oscNextActions.bib'}
\author{K. Reincke% Telekom osCompendium License Include Module
%
% (c) Karsten Reincke, Deutsche Telekom AG, Darmstadt 2011
%
% This LaTeX-File is licensed under the Creative Commons Attribution-ShareAlike
% 3.0 Germany License (http://creativecommons.org/licenses/by-sa/3.0/de/): Feel
% free 'to share (to copy, distribute and transmit)' or 'to remix (to adapt)'
% it, if you '... distribute the resulting work under the same or similar
% license to this one' and if you respect how 'you must attribute the work in
% the manner specified by the author ...':
%
% In an internet based reuse please link the reused parts to www.telekom.com and
% mention the original authors and Deutsche Telekom AG in a suitable manner. In
% a paper-like reuse please insert a short hint to www.telekom.com and to the
% original authors and Deutsche Telekom AG into your preface. For normal
% quotations please use the scientific standard to cite.
%
% [ File structure derived from 'mind your Scholar Research Framework' 
%   mycsrf (c) K. Reincke CC BY 3.0  http://mycsrf.fodina.de/ ]
%
\footnote{
This text is licensed under the Creative Commons Attribution-ShareAlike 3.0 Germany
License (http://creativecommons.org/licenses/by-sa/3.0/de/): Feel free \enquote{to
share (to copy, distribute and transmit)} or \enquote{to remix (to
adapt)} it, if you \enquote{[\ldots] distribute the resulting work under the
same or similar license to this one} and if you respect how \enquote{you
must attribute the work in the manner specified by the author(s)
[\ldots]}):
\newline
In an internet based reuse please mention the initial authors in a suitable
manner, name their sponsor \textit{Deutsche Telekom AG} and link it to
\texttt{http://www.telekom.com}. In a paper-like reuse please insert a short
hint to \texttt{http://www.telekom.com}, to the initial authors, and to their
sponsor \textit{Deutsche Telekom AG} into your preface. For normal quotations
please use the scientific standard to cite.
\newline
{ \tiny \itshape [derived from myCsrf (= 'mind your Scholar Research Framework') 
\copyright K. Reincke CC BY 3.0  http://mycsrf.fodina.de/)] }}}
\maketitle
%%-- end(titlepage)
Release [0.98.1
]

\small
\bibliography{../bibfiles/oscNextActions}

\footnotesize
% Telekom osCompendium English Nomenclation Tokens Include Module 
%
% (c) Karsten Reincke, Deutsche Telekom AG, Darmstadt 2011
%
% This LaTeX-File is licensed under the Creative Commons Attribution-ShareAlike
% 3.0 Germany License (http://creativecommons.org/licenses/by-sa/3.0/de/): Feel
% free 'to share (to copy, distribute and transmit)' or 'to remix (to adapt)'
% it, if you '... distribute the resulting work under the same or similar
% license to this one' and if you respect how 'you must attribute the work in
% the manner specified by the author ...':
%
% In an internet based reuse please link the reused parts to www.telekom.com and
% mention the original authors and Deutsche Telekom AG in a suitable manner. In
% a paper-like reuse please insert a short hint to www.telekom.com and to the
% original authors and Deutsche Telekom AG into your preface. For normal
% quotations please use the scientific standard to cite.
%
% [ File structure derived from 'mind your Scholar Research Framework' 
%   mycsrf (c) K. Reincke CC BY 3.0  http://mycsrf.fodina.de/ ]


%\abbr[aaO]{a.a.O.}{am angegebenen Ort}
%\abbr[ds]{ds.}{kollektiv für ders., dies., \ldots}
\abbr[etseqq]{et seqq.}{and the following ones}
\abbr[id]{id.}{idem = latin for 'the same', be it a man, woman or a group\ldots}
\abbr[ibid]{ibid.}{ibidem = latin for 'at the same place'}
\abbr[ifross]{ifross}{Institut für Rechtsfragen der Freien und Open Source
Software}
\abbr[lc]{l.c.}{loco citato = latin for 'in the place cited'}
\abbr[np]{np.}{no page numbering}
\abbr[wp]{wp.}{webpage / webdocument without any internal (page)numbering}
\abbr[nst]{n.st.}{not stated}
\abbr[njear]{n.y.}{year not stated / no year}
\abbr[nlocation]{n.l.}{location not stated / no location}
\abbr[ub]{UB}{'Universitätsbibliothek' = library of university X}
\abbr[ulb]{ULB}{'Universitäts- \& Landesbibliothek' = library of university and state X}
\abbr[apl]{ApL}{Apache License}
\abbr[bsd]{BSD}{Berkeley Software Distrobution (License)}
\abbr[mit]{MIT}{Massachusetts Institute of Technology (License)}
\abbr[mspl]{Ms-PL}{Microsoft Public License}
\abbr[pgl]{PgL}{Postgres License}
\abbr[php]{PHP}{PHP (License)}
\abbr[epl]{EPL}{Eclipse Public License}
\abbr[eupl]{EUPL}{European Union Public License}
\abbr[lgpl]{LGPL}{GNU Lesser General Public License}
\abbr[mpl]{MPL}{Mozilla Public License}
\abbr[gpl]{GPL}{GNU General Public License}
\abbr[agpl]{AGPL}{GNU Affero General Public License}
\abbr[nabbr]{n.abbr.}{no abbreviation (known)}
% Telekom osCompendium English Nomenclation Tokens Include Module 
%
% (c) Karsten Reincke, Deutsche Telekom AG, Darmstadt 2011
%
% This LaTeX-File is licensed under the Creative Commons Attribution-ShareAlike
% 3.0 Germany License (http://creativecommons.org/licenses/by-sa/3.0/de/): Feel
% free 'to share (to copy, distribute and transmit)' or 'to remix (to adapt)'
% it, if you '... distribute the resulting work under the same or similar
% license to this one' and if you respect how 'you must attribute the work in
% the manner specified by the author ...':
%
% In an internet based reuse please link the reused parts to www.telekom.com and
% mention the original authors and Deutsche Telekom AG in a suitable manner. In
% a paper-like reuse please insert a short hint to www.telekom.com and to the
% original authors and Deutsche Telekom AG into your preface. For normal
% quotations please use the scientific standard to cite.
%
% [ Derived from 'mykeds Scholar Research Framework' 
%   mykeds-CSR-framework (c) K. Reincke CC BY 3.0  http://www.mykeds.net/ ]

%\abbr[]{[n.abbr.]}{ }
\abbr[zge]{ZGE / IPJ}{Zeitschrift für geistiges Eigentum [ISSN: 1867-237x]}
\abbr[itrb]{ITRB}{Der IT-Rechtsberater [ISSN: 1617-1527]}
\abbr[cri]{CRi}{Computer Law Review international [ISSN: 1610-7608]}
\abbr[btlj]{[n.abbr.]}{Berkeley Technology Law Journal}
\abbr[eclr]{E.C.L.R.}{European Competition Law Review}
\abbr[iesw]{[n.abbr.]}{IEEE Software [ISSN: 0740-7459]}
\abbr[cuitj]{[n.abbr.]}{Cutter IT Journal [ISSN: 1048-5600]}
\abbr[uoclr]{[n.abbr.]}{University of Chicago Law Review}
\abbr[uoilr]{[n.abbr.]}{University of Illinois Law Review}
\abbr[uoplr]{[n.abbr.]}{University of Pittsburgh Law Review}
\abbr[ddt]{DDT}{Drug Discovery Today [ISSN: 1359-6446]}
\abbr[rdm]{[n.abbr.]}{R\&D Management [ISSN: 1467-9310]}
\abbr[jleo]{JLEO}{Journal of Law, Economics, \& Organization [ISSN: 1465-7341]}
\abbr[ijomi]{[n.abbr.]}{International Journal of Medical Informatics [ISSN: 1386-5056]}
\abbr[slr]{[n.abbr.]}{Stanford Law Review [ISSN: 00389765]}
\abbr[bise]{BISE}{Business \& Information Systems Engineering [ISSN: 1867-0202]}
\abbr[joals]{[n.abbr.]}{Journal of Academic Librarianship [ISSN: 0099-1333]}
\abbr[eait]{[n.abbr.]}{Ethics and Information Technology [ISSN: 1388-1957]}
\abbr[jais]{JAIS}{Journal of the Association for Information Systems [ISSN:
1536-9323]}
\abbr[josas]{[n.abbr.]}{Journal of Systems and Software [ISSN: 0164-1212]}
\abbr[iialr]{[n.abbr.]}{International Information and Library Review [ISSN: 1057-2317]}
\abbr[sthv]{STHV}{Science, Technology \& Human Values [ISSN: 0162-2439]}
\abbr[cue]{[n.abbr.]}{Computers \& Education [ISSN: 0360-1315]}
\abbr[eer]{EER}{European Economic Review [ISSN: 0014-2921]}
\abbr[icc]{ICC}{Industrial and Corporate Change [ISSN: 0960-6491]}
\abbr[ca]{[n.abbr.]}{Cultural Anthropology [ISSN: 1548-1360]}
\abbr[sqj]{[n.abbr.]}{Software Qualilty Journal [ISSN: 0963-9314]}
\abbr[jmir]{JMIR}{Journal of Medical Information Research [ISSN: 1438-8871]}
\abbr[joce]{[n.abbr.]}{Journal of Comparative Economics [ISSN: 0147-5967]}
\abbr[orgsci]{[n.abbr.]}{Organization Science [ISSN: 1047-7039]}
\abbr[iam]{[n.abbr.]}{Information \& Management [ISSN: 0378-7206]}
\abbr[rp]{RP}{Research Policy [ISSN: 0048-7333]}
\abbr[jsis]{JSIS}{Journal of Strategic Information Systems [ISSN: 0963-8687]}
\abbr[isj]{ISJ}{Information Systems Journal [ISSN: 1365-2575]}
\abbr[jise]{JISE}{Journal of Information Science and Engineering [ISSN:
1016-2364]}
\abbr[dss]{DSS}{Decision Support Systems [ISSN: 0167-9236]}
\abbr[cihp]{CiHB}{Computers in Human Behavior [ISSN: 0747-5632]}
\abbr[iep]{IEaP}{Information Economics and Policy [ISSN: 0167-6245]}
\abbr[tosem]{ToSEM}{Transactions on Software Engineering Methodology [ISSN:
1049-331X]}
\abbr[commacm]{CotACM}{Communications of the ACM [ISSN: 0001-0782]}
\abbr[interactions]{[n.abbr.]}{interactions[ISSN: 1072-5520]}
\abbr[jcsc]{JCSC}{Journal of Computing Sciences in [Small] Colleges [ISSN:
1937-4771]}
\abbr[linuxjournal]{LJ}{Linux Journal [ISSN: 1075-3583]}
\abbr[networker]{[n.abbr.]}{netWorker [ISSN: 1091-3556]}
\abbr[queue]{[n.abbr.]}{Queue [ISSN: 1542-7730]}
\abbr[sigmisdb]{SIGMIS Database}{ACM SIGMIS - The Data Base for Advances in
Information Systems [ISSN: 0095-0033]}
\abbr[sigcas]{SIGCAS}{ACM SIGCAS Computers and Society [ISSN: 0095-2737]}
\abbr[sigsoft]{SIGSOFT SEN}{SIGSOFT Software Engineering Notes [ISSN:
0163-5948]}
\abbr[toit]{ToIT}{Transaction on Internet Technology [ISSN: 1533-5399]}
\abbr[sigbul]{SIGCSE Bulletin}{SIGCSE Bulletin [ISSN: 0097-8418]}
\abbr[ubiquity]{Ubiquity}{Ubiquity - The ACM IT Magazine and Forum [ISSN:
1530-2180]}
\abbr[bwv]{BWV}{Berliner Wissenschafts-Verlag GmbH}
\abbr[cr]{CR}{Computer und Recht. Zeitschrift für die Praxis des Rechts der
Informationstechnologien}


\printnomenclature

\end{document}

% Local Variables:
% mode: latex
% fill-column: 80
% End:
