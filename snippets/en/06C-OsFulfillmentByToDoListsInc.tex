% Telekom osCompendium 'for being included' snippet template
%
% (c) Karsten Reincke, Deutsche Telekom AG, Darmstadt 2011
%
% This LaTeX-File is licensed under the Creative Commons Attribution-ShareAlike
% 3.0 Germany License (http://creativecommons.org/licenses/by-sa/3.0/de/): Feel
% free 'to share (to copy, distribute and transmit)' or 'to remix (to adapt)'
% it, if you '... distribute the resulting work under the same or similar
% license to this one' and if you respect how 'you must attribute the work in
% the manner specified by the author ...':
%
% In an internet based reuse please link the reused parts to www.telekom.com and
% mention the original authors and Deutsche Telekom AG in a suitable manner. In
% a paper-like reuse please insert a short hint to www.telekom.com and to the
% original authors and Deutsche Telekom AG into your preface. For normal
% quotations please use the scientific standard to cite.
%
% [ File structure derived from 'mind your Scholar Research Framework' 
%   mycsrf (c) K. Reincke CC BY 3.0  http://mycsrf.fodina.de/ ]
%

% Chapter Abstract
% ----------------

\chapter{Open Source License Compliance: To-Do Lists}

\footnotesize
\begin{quote}\itshape
With respect to the defined open source use cases, this chapter lists what one
has to do for acting in accordance with the specific open source licenses
\end{quote}
\normalsize{}

\section{Some general remarks on 'giving' someone a file}

This chapter has to be started with some general hints being relevant for many
to-do lists. For not repeating these remarks to often, we are starting
with these general remarks and will later on refer to these remarks:

\label{DistributingFilesHint}
\begin{itemize}
  \item On the one hand, sometimes you will distribute binary packages
  containing open source software components or complete open source
  applications. Moreover, in some cases, you probably want to distribute them on
  a medium which doesn't allow the user, to see the package files directly --
  some mobile devices don't give their users the full access to all stored
  files. On the other hand, open source licenses often require 'to give' someone
  copies of text files, like the license itself, copyright notes, specific
  notice files or anything else. The safe interpretation of 'giving someone a
  text' means that the receiver must be able to read it\footnote{To give someone
  anything he can't touch, feel, see etc., is like not giving him the object
  ;-)}. Hence, on systems which offer a file browser and a suitable reader, it
  is sufficient, to put these file onto the files system. On the other systems,
  you \emph{must} present the content of the files by your application -- for
  example in a specific copyright dialog\footnote{Additionally, in the open
  source community, it is a good tradition, to present these reference data
  voluntarily.}. The OSLiC does not want to refine the taxonomies down to the
  level of operating systems. So, it is up to the reader to transfer these
  conditions into the to-do lists.
  
  \item Sometimes a product which uses and distributes open source software
  tries to fulfill the requirement 'to give the recipients the license etc.' by
  presenting links to general versions of these licensing files hosted somewhere
  on the internet. But be aware: Although it is a good tradition -- especially
  if you link to the homepages of the projects for being totally transparent --
  it is not sufficient to offer only the links. If you are required by the open
  source licenses to handover something to your users, \emph{you} must do it. It
  is not safe to delegate the task to anyone hoping that he will offer the files
  all the time your product is distributed\footnote{Moreover, the advantage
  of doing the job oneself is that one has not to struggle with uncommunicated
  implicite modifications of the link targets.}. But even it would be save, the
  point is: you have to fulfill the license, no one else.
\end{itemize}

\label{OSUCToDoLists}