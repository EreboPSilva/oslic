% Telekom osCompendium 'for being included' snippet template
%
% (c) Karsten Reincke, Deutsche Telekom AG, Darmstadt 2011
%
% This LaTeX-File is licensed under the Creative Commons Attribution-ShareAlike
% 3.0 Germany License (http://creativecommons.org/licenses/by-sa/3.0/de/): Feel
% free 'to share (to copy, distribute and transmit)' or 'to remix (to adapt)'
% it, if you '... distribute the resulting work under the same or similar
% license to this one' and if you respect how 'you must attribute the work in
% the manner specified by the author ...':
%
% In an internet based reuse please link the reused parts to www.telekom.com and
% mention the original authors and Deutsche Telekom AG in a suitable manner. In
% a paper-like reuse please insert a short hint to www.telekom.com and to the
% original authors and Deutsche Telekom AG into your preface. For normal
% quotations please use the scientific standard to cite.
%
% [ File structure derived from 'mind your Scholar Research Framework' 
%   mycsrf (c) K. Reincke CC BY 3.0  http://mycsrf.fodina.de/ ]
%

% Chapter Abstract
% ----------------

\chapter{Open Source License Compliance: To-Do Lists}

\footnotesize
\begin{quote}\itshape
With respect to the defined open source use cases, this chapter lists what one
has to do for acting in accordance with the specific open source licenses.
\end{quote}
\normalsize{}

\section{Some general remarks on 'giving' someone a file}

This chapter has to be started with some general points which are relevant for
many of the to-do lists. So that the same points are not repeated too often, we
will start with these general remarks and refer to them throughout the chapter.

\label{DistributingFilesHint}
\begin{itemize}
  \item
  Sometimes when delivering a binary package containing open source software,
  the medium doesn’t allow the recipient to view all files contained in that
  package. For example, a lot of mobile devices don’t give the user access to
  the file system. But open source licenses often require ‘to give’ someone
  copies of text files, such as the license text, copyright notes, or specific
  notice file. The safe interpretation of ‘giving someone a text’ is that the
  receiver must be able to read it\footnote{To give someone anything they can't
  touch, feel or see is like not giving him the object ;-)}. Thus, on
  systems which offer a file browser and a suitable reader, it is sufficient, to
  put these file onto the files system. On the other systems, you \emph{must}
  present the content of the files  through the UI of your application---for
  example in a specific copyright screen\footnote{Additionally, in the open
  source community, it is a good tradition, to present these reference data
  voluntarily.}. The \oslic{} does not want to refine the taxonomies down to the
  level of operating systems, so it is up to the user to keep this in mind when
  reading the to-do lists.
  
  \item Sometimes a product which uses and distributes open source software
  tries to fulfill the requirement 'to give the recipients the license etc.' by
  presenting links to general versions of these licensing files hosted somewhere
  on the internet. But be aware: Although it is a good tradition---especially
  if you link to the homepages of the projects for being totally transparent---
  it is not sufficient to offer only the links. If you are required by the open
  source licenses to handover something to your users, \emph{you} must do it. It
  is not safe to delegate the task to anyone hoping that they will offer the
  files all the time your product is being distributed\footnote{Moreover, the
  advantage of doing the job oneself is that one has not to struggle with
  uncommunicated implicit modifications of the link targets.}. Even if it would
  be safe to assume that the link will remain valid forever, the point is: you
  have to fulfill the license, no one else.
\end{itemize}

\label{OSUCToDoLists}

% ==============================================================================
% Some commands common to all to-do lists

% Common footnote, used in many task lists
\newcommand{\passingFilesCorrectly}{%
  \footnote{For implementing the handover of files correctly $\rightarrow$
    OSLiC, p. \pageref{DistributingFilesHint}}}

% A LSUC that covers many OSUCs
% #1 -> List of covered OSUCS, e. g., ``OSUC-01, OSUC-07S, and OSUC-10S''
% #2 -> number of first OSUC covered, e. g., '07B'
% #3 -> number of last OSUC covered
\newcommand{\coversOsucs}[3]{\lsuccovers{#1}%
  \footnote{For details $\rightarrow$ \oslic, pp.\ \osucpageref{#2} -- \osucpageref{#3}}}

% A LSUC that maps to exactly one OSUC
% # -> number of the OSUC, e. g., '07B'
\newcommand{\mapsToOsuc}[1]{\lsuccovers{OSUC-#1}%
  \footnote{For details $\rightarrow$ \oslic, pp.\ \osucpageref{#1}}}


% Local Variables:
% mode: latex
% fill-column: 80
% End:
