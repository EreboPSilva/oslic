% Telekom osCompendium 'for beeing included' snippet template
%
% (c) Karsten Reincke, Deutsche Telekom AG, Darmstadt 2011
%
% This LaTeX-File is licensed under the Creative Commons Attribution-ShareAlike
% 3.0 Germany License (http://creativecommons.org/licenses/by-sa/3.0/de/): Feel
% free 'to share (to copy, distribute and transmit)' or 'to remix (to adapt)'
% it, if you '... distribute the resulting work under the same or similar
% license to this one' and if you respect how 'you must attribute the work in
% the manner specified by the author ...':
%
% In an internet based reuse please link the reused parts to www.telekom.com and
% mention the original authors and Deutsche Telekom AG in a suitable manner. In
% a paper-like reuse please insert a short hint to www.telekom.com and to the
% original authors and Deutsche Telekom AG into your preface. For normal
% quotations please use the scientific standard to cite.
%
% [ Framework derived from 'mind your Scholar Research Framework' 
%   mycsrf (c) K. Reincke 2012 CC BY 3.0  http://mycsrf.fodina.de/ ]
%


%% use all entries of the bibliography
%\nocite{*}

\section{A standard form for gathering the relevant information}

\begin{table}
\scriptsize
\caption{information needed for finding the relevant Open Source Use Case}
\begin{center}
\begin{tabular}[h]{|l|l|l|l|}
\hline 
Class & Question & possible answers & your answer\\
\hline 
  Type & 
  Is the Open Source Software, you want to use, a software library in
  the broadest sense (an includable code snippet, a linkable module or library,
  or a loadable plugin), or is it an autonomous application or server
  which can be executed or processed? & 
  proapse or snimoli & ??\\
\hline 
  State & 
  Do you want to leave the evaluated Open Source Software as you have got it, or
  do you want to modify it before using and/or distributing it to 3rd parties? & 
  unmodified or modified & ??\\
\hline 
  Context & 
  ZS & 
  alone or combined & ?? \\
\hline 
Recipient & ZS & 4yourself or 4others & ?? \\
\hline 
Mode & ZS & statically linked, dynamically linked, or textually included & ?? \\
\hline 
\hline
\end{tabular}
\end{center}
\end{table}


\section{The taxonomic Open Source Use Case Finder}

\section{The Open Source Use Cases and their links to the to-do lists}

%\bibliography{../../../bibfiles/oscResourcesEn}
