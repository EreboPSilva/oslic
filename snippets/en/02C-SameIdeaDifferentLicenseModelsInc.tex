% Telekom osCompendium 'for being included' snippet template
%
% (c) Karsten Reincke, Deutsche Telekom AG, Darmstadt 2011
%
% This LaTeX-File is licensed under the Creative Commons Attribution-ShareAlike
% 3.0 Germany License (http://creativecommons.org/licenses/by-sa/3.0/de/): Feel
% free 'to share (to copy, distribute and transmit)' or 'to remix (to adapt)'
% it, if you '... distribute the resulting work under the same or similar
% license to this one' and if you respect how 'you must attribute the work in
% the manner specified by the author ...':
%
% In an internet based reuse please link the reused parts to www.telekom.com and
% mention the original authors and Deutsche Telekom AG in a suitable manner. In
% a paper-like reuse please insert a short hint to www.telekom.com and to the
% original authors and Deutsche Telekom AG into your preface. For normal
% quotations please use the scientific standard to cite.
%
% [ Framework derived from 'mind your Scholar Research Framework' 
%   mycsrf (c) K. Reincke 2012 CC BY 3.0  http://mycsrf.fodina.de/ ]
%

\chapter{Open Source: The Same Idea, Different Licenses}\label{sec:LicenseTaxonomies}

%% use all entries of the bibliography
%\nocite{*}
\footnotesize \begin{quote}\itshape In this chapter we describe different
license models which meet the common idea of being a piece of Free Open Source
Software. We want to discuss existing ways of grouping licenses to underline the
limits of building such clusters: These groups are often used as 'virtual
prototypic licenses' which shall deliver a simplified view onto the conditions
how to act according to the referred real license instances. But one has to
fulfill the requirements of a specific license, not one's own generalized idea
of a set of licenses. Nevertheless, also we wish to offer a new structuring view
into the world of the Open Source Licenses too. We will use a new set of
grouping criteria by referring to the common intended purpose of licenses:
each license wants to protect something or someone against something or someone.
Following this pattern, we can indeed summarize all Open Source Licenses in a
comparable way.
\end{quote}
\normalsize{}

Grouping Open Source licenses is often used. Even the set of \emph{Open Source
Li\-cen\-ses}\footcite[cf.][\nopage wp]{OSI2012b} itself is already a cluster
being established by a set of grouping criteria: The \enquote{distribution
terms} of each software license that wants to be an Open Source License,
\enquote{[\ldots] must comply with the [\ldots] criteria} of the \emph{Open
Source Definition}\footcite[cf.][\nopage wp]{OSI2012a}, maintained by the
\emph{Open Source Initiative}\footcite[cf.][\nopage wp]{OSI2012c} and often
abbreviated as \emph{OSD}. So, this \emph{OSD} demarcates 'the group of
[potential] Open Source Licenses' against 'the group of not Open Sources
Licenses'\footnote{For stating it more precisely: to meet the OSD, is only a
necessary condition for being an \emph{Open Source License}. The sufficient
condition for becoming an \emph{Open Source License}, is the approval by the
OSI which offers a process for becoming an officially approved \emph{Open
Source License} (\cite[cf.][\nopage wp.]{OSI2012d}).}.

Another way to cluster the \emph{Free Software Licenses} is specified by the
\enquote{Free Software Definition}. This \emph{FSD} contains four conditions
which must be met by any free software license: an FSD compliant license must
asign the \enquote{the freedom to run a program, for any purpose [\ldots]},
\enquote{the freedom to study how it works, and adapt it to (one's) needs
[\ldots]}, \enquote{the freedom to redistribute copies [\ldots]}, and finally
\enquote{the freedom to improve the program, and release your improvements
[\ldots]}\footcite[cf.][41]{Stallman1996a}. Surprisingly this definition
implies that the requirement \emph{the sourcecode must be openly accessible},
is 'only' a derived condition. If the \enquote{freedom to make changes and the
freedom to publish improved versions} shall be \enquote{meaningful}, then the
\enquote{access to the source code of the program} is a prerequisite.
\enquote{Therefore, accessibility of source code is a necessary condition for
free software.}\footcite[cf.][41]{Stallman1996a}

The difference between these the OSD and the FSD has often been described as a
difference of emphasizing\footnote{This is also the viewpoint of Richard M.
Stallman: On the one hand, he clearly states that the \enquote{Free Software
movement} and the \enquote{Open Source movement} overall \enquote{[\ldots]
disagree on the basic principles, but agree more or less on the practical
recommendations} and that he \enquote{[\ldots] (does) not think of the Open
Source movement as an enemy}.  On the other hand, he deliniates the two
movements by stating that \enquote{for the Open Source movement, the issue of
whether software should be open source is a practical question, not an ethical
one}, while \enquote{for the Free Software movement, non-free software is a
social problem and free software is the solution}
(\cite[cf.][55]{Stallman1998a}). \label{RmsFsPriority} As consequence, Richard
M. Stallman summarizes the positions in a simple way: \enquote{[\ldots] 'open
source' was designed not to raise [\ldots] the point that users deserve
freedom}. But he and his friends want \enquote{to spread the idea of freedom}
and therefore \enquote{[\ldots] stick to the term 'free software'}
(\cite[][59]{Stallman1998a}).}: Although both definitions \enquote{[\ldots]
(cover) almost exactly the same range of software}, the \emph{Free Software
Foundation} - as it is said - \enquote{prefers [\ldots] (to emphazise) the idea
of freedom [\ldots]} while the \emph{OSI} wants to underline the philosophically
indifferent \enquote{development methodolgy}\footcite[pars pro toto:
cf.][232]{Fogel2006a}.

A third method to collect a special group of free software and free software
licenses is specified by the \enquote{Debian Free Software Guideline} which is
embedded into the \enquote{Debian Social Contract}. This \enquote{DFSG} contains
nine defining criteria which - as Debian itself says - have been
\enquote{[\ldots] adopted by the free[sic!] software community as the basis of
the Open Source Definition}\footcite[cf.][wp]{DFSG2013a}.

A rough understanding of these methods might allow to conclude that these three
definitions are extensionally equal and only differ intensionally.
But that's not true. To unveil the differences, let us compare the clusters
\emph{OSI approved licenses}, \emph{OSD compliant licenses}, \emph{DFSG
compliant licenses}, and \emph{FSD compliant licenses} extensionally, by asking
whether they \emph{could} establish different sets of licenses\footnote{Indeed,
for analyzing the extensional power of the definition we have to regard all
potentially covered licenses, not only the already existing licenses, because
the subset of really existing licenses still could be expanded be developing new
licenses which fit the definition.}.

First, the most easily to determine difference is that of an unidirectional
inclusion: By definition, the \emph{OSI approved licenses} and the \emph{OSD
compliant licenses} meet the requirements of the OSD\footcite[cf.][\nopage
wp]{OSI2012a}. But only the \emph{OSI approved licenses} have successfully
passed the OSI process\footcite[cf.][\nopage wp]{OSI2012a} and therefore are
officially listed as \emph{Open Source Licenses}\footcite[cf.][\nopage
wp]{OSI2012b}. Hence, on the one hand, \emph{OSI approved licenses} are
\emph{Open Source Licenses} and vice versa. On the other hand, both - the
\emph{OSI approved licenses} and the \emph{Open Source Licenses} - are \emph{OSD
compliant licenses}, but not vice versa.

Second, a similar argumentation leeds to the differences between the
\emph{DFSG compliant licenses} and the \emph{OSI approved licenses}. As it is
stated, the OSD \enquote{[\ldots] is based on the Debian Free Software
Guildeline and any license that meets one definition almost meets the
other}\footcite[cf.][233]{Fogel2006a}. But again, meeting the definition is not
enough for being an official Open Source License: the license has to be approved
by the OSI\footcite[cf.][\nopage wp]{OSI2012b}. So, one can analogically say,
that all \emph{OSI approved licenses} are also \emph{DFSG compliant licenses},
but not vice versa.

Third, - by ignoring the \enquote{few exceptions} which have appeared
\enquote{over the years}\footcite[cf.][233]{Fogel2006a} - one can say that,
because of their 'kinsmanlike' relation, at least the \emph{OSD compliant
licenses} are also \emph{DFSG compliant licenses} and vice versa.

Last but not least one has to state that the (potential) set of free software
licenses must be greater than all the other three sets: On the one side, the FSD
only requires that the license of a software must allow to read and to use the
software, to modify and to distribute it\footcite[cf.][41]{Stallman1996a}. These
conditions are covered by at least the first three paragraphs of the OSD
concerning the topics \enquote{Free Redistribution}, \enquote{Source Code}, and
\enquote{Derived Works}\footcite[cf.][\nopage wp]{OSI2012a}. But the OSD
contains at least some requirements which are not mentioned by the FSD and
which nevertheless must be met by a license for being an OSD compliant
license\footnote{For example, see the condition that \enquote{the license must
be technology-neutral} (\cite[cf.][\nopage wp]{OSI2012a}).}. Hence, logically
regarded, there might exist licenses which fulfill all conditions of the FSD
and nevertheless do not fulfill at least some conditions of the OSD\footnote{
Let us repeat: we must consider the extensional potential of the definitions,
not the set of really existing licenses. In this context, it is irrelevant that
actually all existing Free Software Licenses like GPL, LGPL or AGPL indeed are
also classfied as Open Source Licenses. We are referring to the fact that there
might be generated licenses which fulfill the FSD, but not the OSD.}. So, the
set of all (potential) \emph{Free Software Licenses} must be greater than the
set of all (potential) \emph{Open Source Licenses} and greater than the set of
\emph{OSD compliant licenses}.

All in all, we can visualize the situation by a picture like this:

\begin{center}

\begin{tikzpicture}
\label{LICTAX}
\small

\node[ellipse,minimum height=5.8cm,minimum width=11.6cm,draw,fill=gray!10] (l0210) at (5,5)
{ };

\draw [-,dotted,line width=0pt,white,
    decoration={text along path,
              text align={center},
              text={|\itshape|All Software Licenses}},
              postaction={decorate}] (0,6.1) arc (120:60:10cm);

\node[ellipse,minimum height=4.4cm,minimum width=10cm,draw,fill=gray!20] (l0210) at (5,5)
{ };

\draw [-,dotted,line width=0pt,white,
    decoration={text along path,
              text align={center},
              text={|\itshape|FSD Compliant Licenses}},
              postaction={decorate}] (0,5.4) arc (120:60:10cm);
              

\node[ellipse,minimum height=3cm,minimum width=8.4cm,draw,fill=gray!30] (l0210) at (5,5)
{ };


             
\draw [-,dotted,line width=0pt,white,
    decoration={text along path,
              text align={center},
              text={|\itshape|OSD Compliant Licenses}},
              postaction={decorate}] (0,4.7) arc (120:60:10cm);
              
\draw [-,dotted,line width=0pt,white,
    decoration={text along path,
              text align={center},
              text={|\itshape|DFSG Compliant Licenses}},
              postaction={decorate}] (0,5) arc (240:300:10cm);
          

\node[ellipse,text width=4.4cm, text centered,minimum height=1.6cm,minimum width=6cm,draw,fill=gray!40] (l0210) at (5,5)
{ \textit{OSI approved licenses} = \\ \textit{\textbf{Open Source Licenses}}
};

\end{tikzpicture}
\end{center}

It should be clear without longer explanations that these clusters don't allow
to derive a correct compliant behaviour according to the \emph{Open Source
Licenses}: On the one hand, all larger clusters do not talk about the \emph{Open
Source Licenses}. On the other hand, the \emph{Open Source License cluster}
itself only collects his elements on the base of the OSD which does not
stipulates concrete license fulfilling actions of the licensee.

The next level of clustering \emph{Open Source Licenses} concerns the inner
structure of these \emph{OSI approved licenses}. Even the OSI itself has recently
discussed whether a better kind of grouping the listed licenses would better fit
the needs of the visitors of the OSI site\footcite[cf.][\nopage wp]{OSI2013a}.
And finally the OSI ends up in the categories \enquote{popular and widely used
(licenses) or with strong communities}, \enquote{special purpose licenses},
\enquote{other/miscellaneous licenses}, \enquote{licenses that are redundant
with more popular licenses}, \enquote{non-reusable licenses},\enquote{superseded
licenses}, \enquote{licenses that have been voluntarily retired}, and \enquote{
uncategorized licenses}\footcite[cf.][\nopage wp]{OSI2013b}.

Another way to structure the field of Open Source Licenses is to think in
\enquote{types of Open Source Licenses} by grouping the \enquote{\emph{academic
licenses}, so named because they were originally created by academic
institutions}\footcite[cf.][69]{Rosen2005a}, the \enquote{\emph{reciprocal
licenses}}, so named because they \enquote{[\ldots] require the distributors of
derivative works to dis\-tri\-bu\-te those works under same license including the
requirement that the source code of those derivative works be
published}\footcite[cf.][70]{Rosen2005a}, the \enquote{\emph{standard
licenses}}, so named because they refer to the reusability of \enquote{industry
standards}\footcite[cf.][70]{Rosen2005a}, and the \enquote{\emph{content
licenses}}, so named because they refer to
\enquote{[\ldots] other than software, such as music art, film, literary works}
and so on\footcite[cf.][71]{Rosen2005a}.

Both kind of taxonomies directly help to find the relevant licenses which should
be used for new (software) projects. But again: none of these categories does
allow to infer the license compliant behaviour, because the categories are
mostly defined on the base of license external criteria: answers to the
questions, whether a license is published by a specific kind of organization or
whether a license deals with industry standards or other kind of works than
software, inherently do not evoke a license fulfilling behaviour.

Only the act of grouping into the \enquote{\emph{academic licenses}} and the
\enquote{\emph{reciprocal licenses}} touches the idea of license fulfilling
doings, if one - as it has been done - expands the definition of the
\enquote{\emph{academic licenses}} by the specification that these licenses
\enquote{[\ldots] allow the software to be used for any purpose whatsoever with
no obligation on the part of the licensee to distribute the source code of
derivative works}\footcite[cf.][71]{Rosen2005a}. With respect to this additional
specification, the clusters \enquote{\emph{academic licenses}} and the
\enquote{\emph{reciprocal licenses}} indeed might be referred as the
\enquote{main categories} of (Open Source)
licenses\footcite[cf.][179]{Rosen2005a}: By definition, they are constituting
not only a contrary, but contradictory opposite. But, one has also to keep in
mind that they build an antinomy inside of the set of Open Source
Licenses\footnote{Hence, it is at least a little confusing to say that
\enquote{the Open Source License (OSL) is a reciprocal license} and \enquote{the
Academic Free License (AFL) is the exact same license without the reciprocity
provisions} (\cite[cf.][180]{Rosen2005a}): If the BSD license is an AFL and if
an AFL can't be an OSL and if the OSI approves only OSLs, then the BSD license
can't be an approved Open Source License. But in fact, it is (\cite[cf.][\nopage
wp]{OSI2012b}).}.

Connatural to the clustering into \emph{academic licenses} and \emph{reciprocal
licenses} is the grouping into \emph{permissive licenses}, \emph{weak copyleft
licenses}, and \emph{strong copyleft licenses}: Even Wikipedia already uses the
term \enquote{permissive free software licence} in the meaning of \enquote{a
class of free software licence[s] with minimal requirements about how the
software can be redistributed} and \enquote{contrasts} them with the
\enquote{copyleft licences} as those \enquote{with reciprocity / share-alike
requirements}\footcite[cf.][\Connatural to the clustering into \emph{academic licenses} and \emph{reciprocal
licenses} is the grouping into \emph{permissive licenses}, \emph{weak copyleft
licenses}, and \emph{strong copyleft licenses}: Even Wikipedia already uses the
term \enquote{permissive free software licence} in the meaning of \enquote{a
class of free software licence[s] with minimal requirements about how the
software can be redistributed} and \enquote{contrasts} them with the
\enquote{copyleft licences} as those \enquote{with reciprocity / share-alike
requirements}\footcite[cf.][\nopage wp]{wpPermLic2013a}. Some other authors name
the set of \emph{academic licenses} the \enquote{permissive licenses} and
specify the \emph{reciprocal licenses} as \enquote{restrictive licenses},
because in this case - as consequence of the embedded \enquote{copyleft} effect
- the source code must be published in case of modifications. They
additionally introduce the subset of \enquote{strong restrictive licenses} which
additionally require that an (overarching) derivative work must be published
under the same license\footcite[pars pro toto cf.][57]{Buchtala2007a}. The next
refinement of such clustering concepts directly uses the categories
\enquote{[Open Source] licenses with a strict copyleft
clause}\footcite[Originally stated as \enquote{Lizenzen mit einer strengen
Copyleft-Klausel}. Cf.][24]{JaeMet2011a}, \enquote{[Open Source] licenses with a
restricted copyleft clause}\footcite[Originally stated as \enquote{Lizenzen mit
einer beschränkten Copyleft-Klausel}. Cf.][71]{JaeMet2011a}, and \enquote{[Open
Source] licenses without any copyleft clause}\footcite[Originally stated as
\enquote{Lizenzen ohne Copyleft-Klausel}. Cf.][83]{JaeMet2011a}. Finally, this
viewpoint can directly be mapped to the categories \emph{strong copyleft} and
\emph{weak copyleft}: While on the one hand, \enquote{only changes to the
weak-copylefted software itself become subject to the copyleft provisions of
such a license, [and] not changes to the software that links to it}, on the
other hand, the \enquote{strong copyleft} states \enquote{[\ldots] that the
copyleft provisions can be efficiently imposed on all kinds of derived
works}\footcite[cf.][\nopage wp]{wpCopyleft2013a}.nopage wp]{wpPermLic2013a}. Some other authors name
the set of \emph{academic licenses} the \enquote{permissive licenses} and
specify the \emph{reciprocal licenses} as \enquote{restrictive licenses},
because in this case - as consequence of the embedded \enquote{copyleft} effect
- the source code must be published in case of modifications. They additionally
introduce the subset of \enquote{strong restrictive licenses} which additionally
require that an (overarching) derivative work must be published under the same
license\footcite[pars pro toto cf.][57]{Buchtala2007a}. The next refinement of
such clustering concepts directly uses the categories \enquote{[Open Source]
licenses with a strict copyleft clause}\footcite[Originally stated as
\enquote{Lizenzen mit einer strengen Copyleft-Klausel}. Cf.][24]{JaeMet2011a},
\enquote{[Open Source] licenses with a restricted copyleft
clause}\footcite[Originally stated as \enquote{Lizenzen mit einer beschränkten
Copyleft-Klausel}. Cf.][71]{JaeMet2011a}, and \enquote{[Open Source] licenses
without any copyleft clause}\footcite[Originally stated as \enquote{Lizenzen
ohne Copyleft-Klausel}. Cf.][83]{JaeMet2011a}. Finally, this viewpoint can
directly be mapped to the categories \emph{strong copyleft} and \emph{weak
copyleft}: While on the one hand, \enquote{only changes to the weak-copylefted
software itself become subject to the copyleft provisions of such a license,
[and] not changes to the software that links to it}, on the other hand, the
\enquote{strong copyleft} states \enquote{[\ldots] that the copyleft provisions
can be efficiently imposed on all kinds of derived works}\footcite[cf.][\nopage
wp]{wpCopyleft2013a}.

Based on this approach to an adequate clustering and labeling, we can develop
the following picture:

\begin{center}

\begin{tikzpicture}
\label{LICTAX}
\small



\node[ellipse,minimum height=8.5cm,minimum width=14cm,draw,fill=gray!10] (l0100) at (6.8,6.8)
{  };

\draw [-,dotted,line width=0pt,white,
    decoration={text along path,
              text align={center},
              text={|\itshape| OSI approved licenses}},
              postaction={decorate}] (-0.8,6.5) arc (142:38:9.5cm);

\draw [-,dotted,line width=0pt,white,
    decoration={text along path,
              text align={center},
              text={|\itshape|Open Source Licenses}},
              postaction={decorate}] (-0.8,6.5) arc (218:322:9.5cm);
              
\node[ellipse,minimum height=6.2cm,minimum width=4cm,draw,fill=gray!20] (l0100) at (2.75,6.8)
{  };

\draw [-,dotted,line width=0pt,white,
    decoration={text along path,
              text align={center},
              text={|\itshape| permissive licenses}},
              postaction={decorate}] (0.9,7.4) arc (180:0:1.8cm);

\node[circle,draw,text width=1cm, fill=gray!40, text centered] (l0101) at (2,8)
{  \footnotesize \bfseries \textit{ApL}};
\node[circle,draw,text width=1cm, fill=gray!40, text centered] (l0102) at (3.5,8)
{  \footnotesize \bfseries \textit{BSD}};
\node[circle,draw,text width=1cm, fill=gray!40, text centered] (l0103) at (2,6.5)
{  \footnotesize \bfseries \textit{MIT}};
\node[circle,draw,text width=1cm, fill=gray!40, text centered] (l0104) at (3.5,6.5)
{  \scriptsize \bfseries \textit{Ms-Pl}};
\node[circle,draw,text width=1cm, fill=gray!40, text centered] (l0105) at (2,5)
{  \footnotesize \bfseries \textit{PgL}};
\node[circle,draw,text width=1cm, fill=gray!40, text centered] (l0106) at (3.5,5)
{  \footnotesize \bfseries \textit{PHP}};

\node[ellipse,minimum height=6cm,minimum width=8.5cm,draw,fill=gray!20] (l0200) at (9.2,6.5)
{  };

\draw [-,dotted,line width=0pt,white,
    decoration={text along path,
              text align={center},
              text={|\itshape| copyleft licenses}},
              postaction={decorate}] (7.5,8.5) arc (120:60:4cm);


\node[ellipse,minimum height=4.5cm,minimum width=4.2cm,draw,fill=gray!30] (l0210) at (7.45,6.5)
{  };

\draw [-,dotted,line width=0pt,white,
    decoration={text along path,
              text align={center},
              text={|\itshape| weak copyleft licenses}},
              postaction={decorate}] (5.4,6.2) arc (180:0:2cm);

\node[circle,draw,text width=1cm, fill=gray!40, text centered] (l0211) at (6.7,7)
{  \footnotesize \bfseries \textit{EPL}};
\node[circle,draw,text width=1cm, fill=gray!40, text centered] (l0212) at (8.2,7)
{  \footnotesize \bfseries \textit{EUPL}};
\node[circle,draw,text width=1cm, fill=gray!40, text centered] (l0213) at (6.7,5.5)
{  \footnotesize \bfseries \textit{LGPL}};
\node[circle,draw,text width=1cm, fill=gray!40, text centered] (l0214) at (8.2,5.5)
{  \footnotesize \bfseries \textit{MPL}};

\node[ellipse,minimum height=4.5cm,minimum width=3cm,draw,fill=gray!30] (l0220) at (11.4,6.5)
{  };
 
% line width=0pt,white,
\draw [-,dotted,line width=0pt,white,
    decoration={text along path,
              text align={center},
              text={|\itshape| strong copyleft}},
              postaction={decorate}] (10.4,7) arc (180:0:1cm);

\draw [-,dotted,line width=0pt,white,
    decoration={text along path,
              text align={center},
              text={|\itshape| licenses}},
              postaction={decorate}] (10.4,5.4) arc (180:360:1cm);        

\node[circle,draw,text width=1cm, fill=gray!40, text centered] (l0221) at (11.4,7)
{  \footnotesize \bfseries \textit{GPL}};
\node[circle,draw,text width=1cm, fill=gray!40, text centered] (l0222) at (11.4,5.5)
{  \footnotesize \bfseries \textit{AGPL}};


\end{tikzpicture}
\end{center}

This extensionally based clarification of a possible Open Source License
taxonomy is probably well-known and often - more or less explicitly -
referred\footnote{Even the FSF itself uses the term 'permissive non-copyleft
free software license' (\cite[pars pro toto: cf.][\nopage wp/section 'Original BSD
license']{FsfLicenseList2013a}) and contrasts it with the terms 'weak copyleft'
and 'strong copyleft' (\cite[pars pro toto: cf.][\nopage wp/section 'European
Union Public License']{FsfLicenseList2013a})}. Unfortunately, this taxonomy
still contains some misleading underlying messages:

\emph{Permissive} is a very positively connoted word. So, the antinomy of
\emph{permissive licenses} versus \emph{copyleft licenses} implicitly signals,
that the \emph{permissive licenses} are in any meaning better, than the
\emph{copyleft licenses}. Naturally, this 'conclusion' is evoked by
confusing the extensionally definition and the intensional power of the labels.
But that's the way we - the human beings - like to think. 

Anyway, this underlying message is not necessarily 'wrong'. It might be
convenient for those people or companies who only want to use Open Source
software without being restricted by the \emph{giving back obligation} as it has
been introduced by the 'copyleft'. But there might be other people and companies
which emphasize the protecting effect of the copyleft licenses. And indeed, at
least the Open Source license\footnote{Although RMS naturally prefers to specify
it as a \emph{Free Software License} (s. p. \pageref{RmsFsPriority}) }
\emph{GPL}\footnote{As the original source \cite[cf.][\nopage
wp]{Gpl20FsfLicense1991a}. Inside of the OSLiC, we constantly refer to the
license versions which are published by the OSI, because we are dealing with
officially approved Open Source Licenses. For the 'OSI-GPL' \cite[cf.][\nopage
wp]{Gpl20OsiLicense1991a}} has initially been generated to protect the freedom,
to enable the developers to help their \enquote{neighbours} and to get the
modifications back\footnote{The history of the GNU project is multiply told. For
the GNU project and its' initiator \cite[cf. pars pro toto][\nopage
passim]{Williams2002a}. For a broader survey \cite[cf. pars pro toto][\nopage
passim]{Moody2001a}. A very short version is delivered by Richard M.
Stallman himself where he states that - in the years while the early free
community were destroyed - he saw the \enquote{nondisclosure agreement} which
must be signed , \enquote{[\ldots] even to get an executable copy} as a clear
\enquote{[\ldots] promise not to help your neighbour}: \enquote{A cooperating
community was forbidden.} (\cite[cf.][16]{Stallman1999a}).}: So,
\enquote{Copyleft} is defined as a \enquote{[\ldots] method for making a
programm free software and requiring all modified and extended versions of the
program to be free software as well}\footcite[cf.][89]{Stallman1996c}. It is a
method\footnote{Based on the American legal copyright system, this method uses
two steps: firstly one states, \enquote{[\ldots] that it is copyrighted
[\ldots]} and secondly one adds those \enquote{[\ldots] distribution terms,
which are a legal instrument that gives everyone the rights to use, modify, and
redistribute the program's code or any program derived from it but only if the
distribution terms are unchanged} (\cite[cf.][89]{Stallman1996c}).} by which
\enquote{[\ldots] the code and the freedoms become legally
inseparable}\footcite[cf.][89]{Stallman1996c}. Because of these disparate
interests of hoping not to be restricted and hoping to be protected, it could be
helpful to find a better label - an impartial name for the cluster of
\emph{permissive licenses}. But up to that time, we should at least know that
this taxonomy still contains an underlying declassing message.

The other misleading interpretation is - counter-intuitively - evoked by using
the concept 'copyleft licenses'. If one refers to a cluster of \emph{copyleft
licenses} as the opposite of the \emph{permissive licenses}, one implicitly also
sends two messages: First, that republishing one's own modifications
is sufficient to fulfill the \emph{copyleft licenses}. And secondly that the
\emph{permissive licenses} do not require anything which has to be done for
getting the right to use the software. Even if one does not wish to evoke such
an interpretation, we - the human beings - tend to take the things as simple as
possible\footnote{And indeed, it's the experience of the authors that -
sometimes - on the management level, such simplifications gain their independent
existence and determine decisions. But that's not the fault of the managers.
It's their task, to aggregate, generalize and simplify information. It's the
task of the experts, to offer better viewpoints without overwhelming the others
with details.}. But because of several aspects, this understanding of the
antinomy of \emph{copyleft licenses} and \emph{permissive licenses} is too
misleading for taking it as a serious generalization:

On the one hand, even the 'strongly copylefted' GPL requires also other license
fulfilling tasks than only republishing derivative works. For example, it
additionally demands to \enquote{[\ldots] give any other recipients of the [GPL
licensed] Program a copy of this License along with the
Program}\footcite[cf.][\nopage wp §1]{Gpl20OsiLicense1991a}. Furthermore, the
'weakly copylefted' licenses require also more and different criteria which has
to be fulfilled for acting according to these licenses. For example, the EUPL
requires that the licensor who does not directly deliver the binaries together
with the sourcecode, must offer a sourcecode version of his work free of
charge\footnote{The German version of the EUPL uses the phrase
\enquote{problemlos und unentgeltlich(sic!) auf den Quellcode (zugreifen
können)} (\cite[cf.][3, section 3]{EuplLicense2007de}) while the English version
contains the specification \enquote{the Source Code is easily and freely
accessible} (\cite[cf.][2, section 3]{EuplLicense2007en})}, while the MPL
requires that under the same circumstances a recipient \enquote{[\ldots] can
obtain a copy of such Source Code Form [\ldots] at a charge no more than the
cost of distribution to the recipient [\ldots]}\footcite[cf.][\nopage section
3.2.a]{Mpl20OsiLicense2013a}. And last but not least, also the \emph{permissive
licenses} require tasks which must be fulfilled for a license compliant usage -
moreover, they also require different things. For example, the BSD demands that
\enquote{the (r)edistributions [\ldots] must (retain [and/or]) reproduce the
above copyright notice [\ldots]}. Because of the structure of the
\enquote{copyright notice}, this required announcement implies that the authors
/ copyright holders of the software must be publicly named\footcite[cf.][\nopage
wp]{BsdLicense2Clause}. As opposed to this, the Apache License requires that
\enquote{if the (w)ork includes a "NOTICE" text file as part of its
distribution, then any (d)erivative (w)orks that (y)ou distribute must include a
readable copy of the attribution notices contained within such NOTICE file} what
often means that you have to present central parts of such file
publicly\footcite[cf.][\nopage wp. section 4.4]{Apl20OsiLicense2004a} - parts
which can contain many more information than only the names of the authors /
copyright holders.

So, no doubt - and against the intuitive interpretation of this taxonomy - each
\emph{Open Source License} must be fulfilled by some actions, even the most
permissive. And for ascertaining these tasks, one has to review these licenses
themselves, not the generalized concepts of licenses taxonomies. Hence again, we
have to state that even this well known kind of grouping of \emph{Open Source
Licenses} does not allow to derive a specific license compliant behavior: The
taxonomy might be appropiate, if one wants to live with the implicite messages
and generalizations of some of its concepts. But the taxonomy is not an adequate
tool to determine, what one has to do for fulfilling an \emph{Open Source
License}. A license compliant behaviour for getting the right to use a specific
piece of \emph{Open Source Software} must refer to the concrete \emph{Open
Source License} by which the licensor has licensed the software. There doesn't
exist any shortcut.

Nevertheless, human beings need generalizing and structuring viewpoints for
enabling themselves to talk about a domain - even if they finally have to regard
the single objects of the domain for specific purposes. We think that there is
a subtler method to regard and to structure the domain of \emph{Open Source
Licenses}. So, we want to offer this other possibility to cluster the \emph{Open
Source licenses}\footnote{even if finally also we have to concede that at the
end one has to look into the license itself.}:

We think that in general, licenses have a common purpose: they should protect
someone or something against something. The structure of this tasks is based on
the nature of the word 'protect' which is a 3 valent verb: it links someone or
something who protects, to someone or something who is protected and both
together to something against the protector protects and against the other one
is protected. Licenses in general do so. Therefore, it's also the purpose of
Open Source Licenses to protect: They can protect the user (receiver) of the
software, its' contributor resp. developer and/or distributor, and the software
itself. And they can protect them against different threats. With respect to
this viewpoint, we should specify the \emph{Open Source Licenses} in a specific,
purpose orientied way:

\section{The protecting power of the Apache License (ApL)}
\begin{itemize} 
  \item The Apache License protects \ldots
  \item But the Apache License does not protect \ldots
\end{itemize}

\section{The protecting power of the BSD licenses}
\begin{itemize}
  \item As approved \emph{Open Source Licenses}\footcite[cf.][\nopage
  wp]{OSI2012b}, the BSD Licenses\footnote{BSD has to be resolved as
  \emph{Berkely Software Distribution}. For details of the BSD license release
  and namings \cite[cf.][\nopage wp. editorial]{BsdLicense3Clause}} protect the
  user against the loss of the right to use, to modify and/or to distribute the
  received copy of the source code or the binary\footcite[cf.][\nopage wp
  §1ff]{OSI2012a}. Additionally, they protect the contributors and/or
  distributors against warranty claims of the software users, because these
  licenses contain a 'No Warranty Clause'\footcite[one for all version
  cf.][\nopage wp]{BsdLicense2Clause}. And finally they protect the distributed
  sources against a change of the license which closes the sources, because each
  modification and \enquote{redistributions of [the] source code must retain the
  [\ldots] copyright notice, this list of conditions and the [\ldots]
  disclaimer}\footcite[cf.][\nopage wp]{BsdLicense2Clause}: Therefore it is
  uncorrect to distribute a BSD licensed code under another license - regardless,
  whether it closes the sources or not\footnote{In common sense based discussions
  you may have heard that BSD licenses allow to republish the work under
  another, an own license. Taking the words of the BSD License seriously that's
  not valid under all circumstances: Yes, it is true, you are not required to 
  redistribute the sourcecode of a modified (derivative) work. You are allowed 
  to modify a received version and to distribute the results only as binary code 
  and to keep your improvements closed. But if you distribute the source code of 
  your modifications, you have retain the licensing, because 
  \enquote{Redistribution [\ldots] in source [\ldots], with or without 
  modification, are permitted provided that [\ldots] (the) redistributions of 
  source code [\ldots] retain the above copyright notice, this list of 
  conditions and the following disclaimer} 
  (\cite[cf.][\nopage wp]{BsdLicense2Clause}}.
  
  \item But the BSD Licenses protect neither the users nor the contributors
  and/or distributors against patent disputes (because they do not contain any
  patent clause). They do not protect the contributors against the loss of
  feedback (because they do not 'copyleft' the software). And they do not
  protect the undistributed software or the distributed binaries against
  re-closings, neither in unmodified nor in modified form (because they allow to
  redistribute only the binaries)\footnote{see both, the BSD-2CL License
  (\cite[cf.][\nopage wp]{BsdLicense2Clause}), and the BSD-3CL License
  (\cite[cf.][\nopage wp]{BsdLicense3Clause})}
  
\end{itemize}

\section{The protecting power of the MIT license}
\begin{itemize}
  \item As an approved \emph{Open Source Licenses}\footcite[cf.][\nopage
  wp]{OSI2012b}, the MIT License\footcite[MIT has to be resolved as
  \enquote{Massachusetts Institute of Technology} 
  (cf.][\nopage wp).]{wpMitLic2011a} protects the user against the loss of the
  right to use, to modify and/or to distribute the received copy of the source
  code or the binary\footcite[cf.][\nopage wp 1ff]{OSI2012a}. Additionally, it
  protects the contributors and/or distributors against warranty claims of the
  software users, because it contains a 'No Warranty
  Clause'\footcite[cf.][\nopage wp]{MitLicense2012a}. And finally it protects
  the distributed sources against a change of the license which would close the
  sources, because the \enquote{permission [\ldots] to use, copy, modify,
  [\ldots] distribute, [\ldots] (is granted) subject to the [\ldots] conditions,
  [that] the [\ldots] copyright notice and this permission notice shall be
  included in all copies or substantial portions of the
  Software}\footnote{\cite[cf.][\nopage wp]{MitLicense2012a}. The argumentation
  why the source code is protected, but not the binary form follows that of the
  BSD licenses: By these requirements, one is not obliged to redistribute the
  sourcecode of a modified (derivative) work. One is allowed to modify a
  received version and to distribute the results only in binary form and to keep
  one's improvements closed. But if one distribute the source code of the
  modifications, the licensing is retained, simply because the MIT
  \enquote{[\ldots] permission note shall be included in all copies or
  substantial portions of the software}.}
 
  \item But the MIT License protect neither the users nor the contributors
  and/or distributors against patent disputes (because it does not contain any
  patent clause). It does not protect the contributors against the loss of
  feedback (because it does not 'copyleft' the software). And it does do not
  protect the undistributed software or the distributed binaries against
  re-closings, neither in unmodified nor in modified form (because it allows to
  redistribute only the binaries)\footcite[cf.][\nopage wp]{MitLicense2012a}
  
\end{itemize}

\section{The protecting power of the Microsoft Public License (Ms-Pl)}
\begin{itemize} 
  \item The Microsoft Public License protects \ldots
  \item But the Microsoft Public License does not protect \ldots
\end{itemize}

\section{The protecting power of the Postgres License (PgL)}
\begin{itemize} 
  \item The Postgres License protects \ldots
  \item But the Postgres License does not protect \ldots
\end{itemize}

\section{The protecting power of the PHP License}
\begin{itemize} 
  \item The PHP License protects \ldots
  \item But the PHP License does not protect \ldots
\end{itemize}  
 
\ldots

All these specifications can also be covered by a table:

\begin{table}
\footnotesize
\caption{Open Source Licenses as Protectors}
\begin{center}

\begin{tabular}{|c|c|c|c|c|c|c|c|c|c|c|c|c|c|c|c|}
\hline
  \multicolumn{2}{|c|}{\textit{Open}} &
  \multicolumn{12}{c|}{\textit{are protecting}}\\
\cline{3-14}
  \multicolumn{2}{|c|}{\textit{Source}} &
  \multicolumn{4}{c|}{ \textbf{Users}} &
  \multicolumn{3}{c|}{\textbf{Contributors}} &
  \multicolumn{5}{c|}{\textbf{Software}} \\
\cline{10-14}
  \multicolumn{2}{|c|}{\textit{Licenses}} &
  \multicolumn{4}{c|}{} &
  \multicolumn{3}{c|}{\tiny{(Distributors)}} &  
  not &
  \multicolumn{4}{c|}{distributed as} \\
\cline{3-9}\cline{11-14}
  \multicolumn{2}{|c|}{} &
  \multicolumn{4}{c|}{\scriptsize{\textit{who have already got}}} &
  \multicolumn{3}{c|}{\scriptsize{\textit{who spread Open}}} & 
  distri- &
  \multicolumn{2}{c|}{modified} &
  \multicolumn{2}{c|}{unmodified} \\
  \cline{11-14}
  \multicolumn{2}{|c|}{} &
  \multicolumn{4}{c|}{\scriptsize{\textit{sources or binaries}}} &
  \multicolumn{3}{c|}{\scriptsize{\textit{Source software}}} & 
  buted & 
 \footnotesize{sources} &
 \footnotesize{binaries} &
 \footnotesize{sources} &
 \footnotesize{binaries} \\
\cline{3-14}
  \multicolumn{2}{|c|}{} &
  \multicolumn{12}{c|}{\textit{against}}\\
\cline{3-14}
  \multicolumn{2}{|c|}{} &
  \multicolumn{3}{c|}{the loss of} & 
  \multirow{3}{*}{\rotatebox{270}{Patent Disputes}} &
  \multirow{3}{*}{\rotatebox{270}{Loss of Feedback}} & 
  \multirow{3}{*}{\rotatebox{270}{Warranty Claims}} & 
  \multirow{3}{*}{\rotatebox{270}{Patent Disputes}} & 
  \multicolumn{5}{c|}{}\\
% no seperator line 
  \multicolumn{2}{|c|}{} &
  \multicolumn{3}{c|}{the right to} &
  & & & &
  \multicolumn{5}{c|}{\footnotesize{Re-Closings of}}\\
\cline{3-5}
  \multicolumn{2}{|c|}{} & 
  \rotatebox{270}{use it} & 
  \rotatebox{270}{modify it} & 
  \rotatebox{270}{redistribute it} &
  &  &  &  &
  \multicolumn{5}{c|}{already Opened Software}\\
\hline
\hline
  ApL & 2.0 & \checkmark  & \checkmark  & \checkmark & 
  - & - & - & - & - & - & - & - & - \\
\hline
  \multirow{2}{*}{BSD} & 3-Cl & \checkmark & \checkmark  & \checkmark  & 
    $\neg$ & $\neg$ & \checkmark & $\neg$  &
    $\neg$ & \checkmark  & $\neg$ & \checkmark & $\neg$ \\
\cline{2-14}
   & 2-Cl & \checkmark  & \checkmark  & \checkmark  & 
    $\neg$ & $\neg$ & \checkmark & $\neg$  &
    $\neg$ & \checkmark  & $\neg$ & \checkmark & $\neg$ \\
\hline
  MIT & ~ & \checkmark  & \checkmark  & \checkmark  &
  $\neg$ & $\neg$ & \checkmark & $\neg$ & $\neg$ &
   \checkmark  & $\neg$ & \checkmark & $\neg$ \\
\hline
  Ms-Pl & ~ & \checkmark  & \checkmark  & \checkmark  & 
  - & - & - & - & - & - & - & - & - \\
\hline
  PgL & ~ & \checkmark  & \checkmark  & \checkmark  &
  - & - & - & - & - & - & - & - & - \\
\hline
  PHP & 3.0 & \checkmark  & \checkmark  & \checkmark  &
  - & - & - & - & - & - & - & - & - \\
\hline
\hline
  \textit{CDDL} & 1.0 & \checkmark & \checkmark & \checkmark &
  - & - & - & - & - & - & - & - & - \\
\hline
  EPL & 1.0 & \checkmark & \checkmark & \checkmark &
  - & - & - & - & - & - & - & - & - \\
\hline
  EUPL & 1.1 & \checkmark & \checkmark & \checkmark &
  - & - & - & - & - & - & - & - & - \\
\hline
  \multirow{2}{*}{LGPL} & 2.1 & \checkmark & \checkmark & \checkmark &
  - & - & - & - & - & - & - & - & - \\
\cline{2-14}
   & 3.0 & \checkmark & \checkmark & \checkmark &
   - & - & - & - & - & - & - & - & - \\
\hline
  \multirow{2}{*}{MPL} & 1.1 & \checkmark & \checkmark & \checkmark &
  - & - & - & - & - & - & - & - & - \\
\cline{2-14}
  & 2.0 & \checkmark & \checkmark & \checkmark &
  - & - & - & - & - & - & - & - & - \\
\hline
\hline
  AGPL & 3.0 & \checkmark & \checkmark & \checkmark &
  - & - & - & - & - & - & - & - & - \\
\hline
  \multirow{2}{*}{GPL} & 2.1 & \checkmark & \checkmark & \checkmark &
   - & - & - & - & - & - & - & - & - \\
\cline{2-14}
  & 3.0 & \checkmark & \checkmark & \checkmark &
   - & - & - & - & - & - & - & - & - \\
\hline
\hline

\end{tabular}
\end{center}
\end{table}


And the content of this classifying table can also be transfered into a mindmap:

\begin{tikzpicture}
\label{LICTAX}
\footnotesize

% (1.A) list of all licenses and their release numbers Level 5/6
\node[rectangle,draw,text width=1.4cm] (l0100) at (10,0.5)
{ \textit{BSD License} };
\node[text width=1.4cm] (l0101) at (12,0)
{ \scriptsize{3-Clauses} };
\node[text width=1.4cm] (l0102) at (12,1)
{ \scriptsize{2-Clauses} };
  
\node[rectangle,draw,text width=1.4cm] (l0200) at (11,2)
{ \textit{MIT License} }; 
\node[text width=0.4cm] (l0201) at (12.5,2) {\scriptsize{1.1}};
  
\node[rectangle,draw,text width=1.4cm] (l0300) at (12.5,3)
{ \textit{\textbf{Ap}ache \textbf{L}icense}};
   \node[text width=0.4cm] (l0301) at (14,3) {\scriptsize{2.1}};

\node[rectangle,draw,text width=1.4cm] (l0400) at (13,4.5)
{ \textit{\textbf{M}icro\textbf{s}oft \textbf{P}ublic \textbf{L}icense} };
\node[text width=0.4cm] (l0401) at (14.5,4){\scriptsize{1.0}};
\node[text width=0.4cm,style=dotted] (l0402) at (14.5,5){\scriptsize{\textit{1.1}}};
  
\node[rectangle,draw,text width=1.4cm] (l0500) at (13,6)
{\textit{\textbf{P}ost\textbf{g}res \textbf{L}icense}};
\node[text width=0.4cm] (l0501) at (14.5,6){ \scriptsize{1.1}};
  
\node[rectangle,draw,text width=1.4cm] (l0600) at (13,7)
{\textit{PHP License}};
\node[text width=0.4cm] (l0601) at (14.5,7){\scriptsize{1.1}};
  
\node[rectangle,draw,text width=1.4cm] (l0700) at (13,8)
{ \textit{XXX License}};
\node[text width=0.4cm] (l0701) at (14.5,8){\scriptsize{1.1}};

\node[rectangle,draw,text width=1.4cm] (l0800) at (13,9.5)
{ \textit{\textbf{M}ozilla \textbf{P}ublic \textbf{L}icense}};
\node[text width=0.4cm] (l0801) at (14.5,9){\scriptsize{1.1}};
\node[text width=0.4cm] (l0802) at (14.5,10){\scriptsize{2.0}};

\node[rectangle,draw,text width=1.4cm] (l0900) at (13,11.5)
{\textit{\textbf{E}clipse \textbf{P}ublic \textbf{L}icense}};
\node[text width=0.4cm] (l0901) at (14.5,11) {\scriptsize{1.0}};
\node[text width=0.4cm,style=dotted] (l0902) at (14.5,12){\scriptsize{\textit{1.1}}};
 
\node[rectangle,draw,text width=1.5cm] (l1000) at (13,13.5)
{\textit{\textbf{E}uropean \textbf{P}ublic \textbf{L}icense}}; 
\node[text width=0.4cm] (l1001) at (14.5,13){\scriptsize{1.0}};
\node[text width=0.4cm,style=dotted] (l1002) at (14.5,14){\scriptsize{\textit{1.1}}};
  
\node[rectangle,draw,text width=1.4cm] (l1100) at (13,15.5)
{\textit{\textbf{L}esser \textbf{G}NU \textbf{P}ublic \textbf{L}icense}};

\node[text width=0.4cm] (l1101) at (14.5,15){\scriptsize{2.1}};
\node[text width=0.4cm] (l1102) at (14.5,16){\scriptsize{3.0} };

\node[rectangle,draw,text width=1.4cm] (l1200) at (13,17.5)
{\textit{\textbf{G}NU \textbf{P}ublic \textbf{L}icense}};

\node[text width=0.4cm] (l1201) at (14.5,17){\scriptsize{2.1}};
\node[text width=0.4cm] (l1202) at (14.5,18){\scriptsize{3.0} };

\node[rectangle,draw,text width=1.4cm] (l1300) at (13,19.5)
{ \textit{\textbf{A}ffero \textbf{G}NU \textbf{P}ublic \textbf{L}icense}};
\node[text width=0.4cm, style=dotted] (l1301) at (14.5,19){\scriptsize{2.1}};
\node[text width=0.4cm] (l1302) at (14.5,20){\scriptsize{3.0}};

% 2. the clustering concepts of licenses (level 4)
\node[rectangle,draw,text width=2.3cm] (n0100) at (10,8)
 { \textit{protecting the user, the con\-tri\-butor \& the initial code}\\
   \tiny{Permissive Licenses}      
 };

\node[rectangle,draw,text width=2.3cm] (n0200) at (10,11.5)
{ \textit{protecting the user, the con\-tri\-butor, the
  initial code, \& all di\-rect de\-ri\-va\-tions}\\
  \tiny{Weak Copyleft}        
};

\node[rectangle,draw,text width=2.3cm] (n0300) at (10,16.5)
{ \textit{protecting the user, the con\-tri\-bu\-tor, the 
  initial code, all di\-rect de\-ri\-va\-tions \& the 
  (in\-di\-rect\-ly de\-ri\-ved) on-top-deve\-lop\-ments}\\ 
  \tiny{Strong Copyleft}    
 };

% 3. the threats (level 3)
\node[ellipse,draw,text width=1.6cm] (c110000) at (4.5,0)
{ \textbf{\textit{Patent Disputes}}};

\node[ellipse,draw,text width=1.6cm] (c120000) at (4.5,2)
{ \textbf{\textit{Loss of Rights}} };

\node[ellipse,draw,text width=1.6cm] (c210000) at (4.5,4)
{ \textbf{\textit{Warranty Claims}} };
 
\node[ellipse,draw,text width=1.6cm] (c220000) at (4.5,6)
{ \textbf{\textit{Loss of Feeback}}};

\node[ellipse,draw,text width=0.4cm] (c311000) at (6.2,8)
{ \tiny{\textit{\textbf{clos\-ings}}}};

\node[ellipse,,draw,text width=0.4cm] (c321000) at (6.2,10)
{ \tiny{\textit{\textbf{clos\-ings}}} };

\node[ellipse,,draw,text width=0.4cm] (c331000) at (6.2,12)
{ \tiny{\textit{\textbf{clos\-ings}}} };

\node[ellipse,,draw,text width=0.4cm] (c341000) at (6.2,14)
{ \tiny{\textit{\textbf{clos\-ings}}} };

\node[ellipse,,draw,text width=0.4cm] (c351000) at (6.2,16)
{ \tiny{\textit{\textbf{clos\-ings}}} };

\node[ellipse,,draw,text width=1.6cm] (c411000) at (6.2,19)
{ \textit{\textbf{clos\-ings}} };


% 4. the subtypes of protected entities (level 2)
\node[ellipse,draw,text width=1.5cm] (c310000) at (3,8)
 { \scriptsize{un\-modified} \textbf{Sources}};

\node[ellipse,draw,text width=1.5cm] (c320000) at (3.25,10)
 { \scriptsize{un\-modified} \textbf{Binaries}};

\node[ellipse,draw,text width=1.2cm] (c330000) at (3.5,12)
 { \scriptsize{modified} \textbf{Sources}};

\node[ellipse,draw,text width=1.4cm] (c340000) at (3.25,14)
 { \scriptsize{modified} \textbf{Binaries}};

\node[ellipse,draw,text width=1.4cm] (c350000) at (3,16)
 { \scriptsize{part of \textbf{On-Top-Develop\-ments}}};


% 5. the protected entities (level 1)
\node[ellipse,draw,text width=1cm] (c100000) at (1,1)
 { \textbf{Users} };

\node[ellipse,draw,text width=0.8cm] (c200000) at (1,5)
 { \textbf{Con\-tribu\-tors}};

\node[ellipse,draw,text width=0.8cm] (c300000) at (1,12)
 { distri\-buted \textbf{Soft\-ware}};
 
\node[ellipse,draw,text width=2.2cm] (c400000) at (1,19)
 { un\-distri\-buted \textbf{Soft\-ware}}; 

% 6. main node (leve 0)
\node[ellipse,draw,text width=1.3cm] (c000000) at (0,8)
{ \textbf{Open Source License}};

% a linking Licenses to their release numbers (Linking level 5 to 6)
\foreach \father/\daughter in {
  l0100/l0101/,
  l0100/l0102/,
  l0200/l0201/,   
  l0300/l0301/,
  l0400/l0401/,
  l0400/l0402/,
  l0500/l0501/,
  l0600/l0601/,
  l0700/l0701/,
  l0800/l0801/,
  l0800/l0802/,
  l0900/l0901/,
  l0900/l0902/,
  l1000/l1001/,
  l1000/l1002/,
  l1100/l1101/,
  l1100/l1102/,
  l1200/l1201/,
  l1200/l1202/,
  l1300/l1301/,
  l1300/l1302/
  }
  \draw[dashed] (\father) to  (\daughter) ;

% b) linking Licenses to license concepts (Linking level 5 to 4)
\foreach \father/\daughter/\outangle/\inangle in {
  n0100/l0100/270/150,       
  n0100/l0200/280/155,
  n0100/l0300/290/160,
  n0100/l0400/300/165,
  n0100/l0500/310/150,
  n0100/l0600/340/160,
  n0100/l0700/360/180,
  n0200/l0800/300/160,
  n0200/l0900/340/170,
  n0200/l1000/20/190,
  n0200/l1100/60/200,
  n0300/l1200/40/180,
  n0300/l1300/80/180 
  }
  %\draw[dashed] (\father) to [out=\outangle,in=\inangle] (\daughter) ;
  \draw[dashed] (\father) to  (\daughter) ;

% c) linking license concepts to the threats against they protect
% c.1) strong copyleft licenses
\foreach \father/\daughter/\outangle/\inangle in {
  c351000/n0300/0/180,
  c341000/n0300/45/190,
  c331000/n0300/50/200,
  c321000/n0300/55/210,
  c311000/n0300/60/220,
  c220000/n0300/25/225,
  c210000/n0300/25/230,
  c120000/n0300/25/235
  }
  \draw[<-,color=blue] (\father) to [out=\outangle,in=\inangle] (\daughter) ;
% c.2) weak copyleft licenses
\foreach \father/\daughter/\outangle/\inangle in {
  c341000/n0200/330/170,
  c331000/n0200/0/180,
  c321000/n0200/0/180,
  c311000/n0200/20/190,
  c220000/n0200/15/220,
  c210000/n0200/15/230,
  c120000/n0200/15/235
  }
  \draw[<-,color=cyan] (\father) to [out=\outangle,in=\inangle] (\daughter) ;
% c.3) permissive licenses
\foreach \father/\daughter/\outangle/\inangle in {
  c331000/n0100/355/150,
  c311000/n0100/0/180,
  c210000/n0100/5/210,
  c120000/n0100/10/230
  }
  \draw[<-,color=red] (\father) to [out=\outangle,in=\inangle] (\daughter) ;
%c.4 agpl license
\foreach \father/\daughter/\outangle/\inangle in {
  c411000/l1300/0/180    
}
  \draw[<-,color=green] (\father) to [out=\outangle,in=\inangle] (\daughter) ;


%d linking protected entities, their subtypes and the the relations
\foreach \father/\daughter/\edgetext/\outangle/\inangle in {
  c000000/c100000/protecting/260/120,
  c100000/c110000/against/360/180,
  c100000/c120000/against/360/180,
  c000000/c200000/protecting/270/180,
  c200000/c210000/against/0/180,
  c200000/c220000/against/0/180,
  c000000/c300000/protecting/90/230,
  c300000/c310000/as/300/180,
  c300000/c320000/as/330/180,
  c300000/c330000/as/0/180,
  c300000/c340000/as/30/180,
  c300000/c350000/as/60/180,
  c000000/c400000/protecting/100/240,
  c400000/c411000/against/0/180        
}
  \draw[->,dotted,
    decoration={text along path,
              text align={center},
              text={|\itshape|\edgetext}},
              postaction={decorate},] (\father) to [out=\outangle,in=\inangle] (\daughter) ;

\foreach \father/\daughter/\edgetext/\outangle/\inangle in {
  c310000/c311000/against/0/180,
  c320000/c321000/against/0/180,
  c330000/c331000/against/0/180,
  c340000/c341000/against/00/180,
  c350000/c351000/against/0/180        
}
  \draw[->,dotted,
    decoration={text along path,
              text align={center},
              text={|\itshape \tiny|\edgetext}},
              postaction={decorate},] (\father) to [out=\outangle,in=\inangle] (\daughter) ;

%f linking the patent clauses
\foreach \father/\daughter/\outangle/\inangle in {
  c110000/l1302/0/295,
  c110000/l0401/0/360,
  c110000/l0301/0/360   
}
  \draw[<-,color=gray] (\father) to [out=\outangle,in=\inangle] (\daughter) ;

\end{tikzpicture}

Finally, one could generate new groups of Open Source license, new classes, like
'user protecting licenses'\footnote{all of them because all of them have to
fulfill the OSD}, 'patent disputes fending licenses', an so on.
% TODO nach Fertigstellung eigene Taxonomie entwickeln

But one has to know: all of these grouping viewpoints do not allow to conclude
that all members of a group can be respected by fulfilling the same
requirements. This would only be possible if the grouping criteria would
directly refer to the fulfilling tasks. And indeed, nearly all Open Source
licenses do differ with respect to these criteria, even if the differences are
very small, they can't be neglected\footnote{Pars pro toto: Both, the BSD
license and the Apache license require that you give hint to the developers of
the application. But in case of the BSD license you xýou have to \ldots. In case
of the Apache license you have exactly to present the content of the notice file
distributed together with the application.}. So: reflecting on possible classes
of Open Source licenses is a good method to become familiar with the area of
Open Source licenses. But it is not a method to determine, what one has to do
for getting the right to use the software. For getting these information, one
has to consider each single license.


%\bibliography{../../../bibfiles/oscResourcesEn}
