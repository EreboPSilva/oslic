% Telekom osCompendium 'for being included' snippet template
%
% (c) Karsten Reincke, Deutsche Telekom AG, Darmstadt 2011
%
% This LaTeX-File is licensed under the Creative Commons Attribution-ShareAlike
% 3.0 Germany License (http://creativecommons.org/licenses/by-sa/3.0/de/): Feel
% free 'to share (to copy, distribute and transmit)' or 'to remix (to adapt)'
% it, if you '... distribute the resulting work under the same or similar
% license to this one' and if you respect how 'you must attribute the work in
% the manner specified by the author ...':
%
% In an internet based reuse please link the reused parts to www.telekom.com and
% mention the original authors and Deutsche Telekom AG in a suitable manner. In
% a paper-like reuse please insert a short hint to www.telekom.com and to the
% original authors and Deutsche Telekom AG into your preface. For normal
% quotations please use the scientific standard to cite.
%
% [ Framework derived from 'mind your Scholar Research Framework' 
%   mycsrf (c) K. Reincke 2012 CC BY 3.0  http://mycsrf.fodina.de/ ]
%

\chapter{Open Source: The Same Idea, Different Licenses}\label{sec:LicenseTaxonomies}

%% use all entries of the bibliography
%\nocite{*}
\footnotesize \begin{quote}\itshape This chapter describes different license
models which follow the common idea of free open source software. We want to
discuss existing ways of grouping licenses to underline the limits of building
such clusters: These groups are often used as 'virtual prototypic licenses'
which shall deliver a simplified view at the conditions how to act according to
the respective real license instances. But one has to meet the requirements of a
specific license, not one's own generalized idea of a set of licenses.
Nonetheless, also we offer a new structuring view into the world of the open
source licenses. We will use a novel set of grouping criteria by referring
to the common intended purpose of licenses: each license is designed to protect
something or someone against something or someone. Following this pattern, we
can indeed summarize all Open Source Licenses in a comparable way.
\end{quote}
\normalsize{}

Grouping open source licenses is commonly done. Even the set of the \emph{open
source li\-cen\-ses}\footcite[cf.][\nopage wp]{OSI2012b} itself is already a
cluster being established by a set of grouping criteria: The
\enquote{distribution terms} of each software license that intends to become an
open source license, \enquote{[\ldots] must comply with the [\ldots] criteria}
of the \emph{Open Source De\-fi\-ni\-tion}\footcite[cf.][\nopage wp]{OSI2012a},
maintained by the \emph{Open Source Initiative}\footcite[cf.][\nopage
wp]{OSI2012c} and often abbreviated as \emph{OSD}. So, this \emph{OSD}
demarcates 'the group of [potential] open source licenses' against 'the group of
not open sources licenses'\footnote{More precisely: meeting the
OSD is only a necessary condition for becoming an \emph{open source license}. The
sufficient condition for becoming an \emph{open source license}, is the approval
by the OSI which offers a process for the officially approval of \emph{open
source license} (\cite[cf.][\nopage wp.]{OSI2012d}).}.

Another way to cluster the \emph{Free Software Licenses} is specified by the
\enquote{Free Software Definition}. This \emph{FSD} contains four conditions
which must be met by any free software license: any FSD compliant license must
grant \enquote{the freedom to run a program, for any purpose [\ldots]},
\enquote{the freedom to study how it works, and adapt it to (one's) needs
[\ldots]}, \enquote{the freedom to redistribute copies [\ldots]}, and finally
\enquote{the freedom to improve the program, and release your improvements
[\ldots]}\footcite[cf.][41]{Stallman1996a}. Surprisingly this definition
implies that the requirement \emph{the sourcecode must be openly accessible},
is 'only' a derived condition. If the \enquote{freedom to make changes and the
freedom to publish improved versions} shall be \enquote{meaningful}, then the
\enquote{access to the source code of the program} is a prerequisite.
\enquote{Therefore, accessibility of source code is a necessary condition for
free software.}\footcite[cf.][41]{Stallman1996a}

The difference between the OSD and the FSD has often been described as a
difference of emphasis\footnote{This is also the viewpoint of Richard M.
Stallman: On the one hand, he clearly states that the \enquote{Free Software
movement} and the \enquote{open source movement} generally \enquote{[\ldots]
disagree on the basic principles, but agree more or less on the practical
recommendations} and that he \enquote{[\ldots] (does) not think of the open
source movement as an enemy}.  On the other hand, he delineates the two
movements by stating that \enquote{for the open source movement, the issue of
whether software should be open source is a practical question, not an ethical
one}, while \enquote{for the Free Software movement, non-free software is a
social problem and free software is the solution}
(\cite[cf.][55]{Stallman1998a}). \label{RmsFsPriority} Consequently, Richard
M. Stallman summarizes the positions in a simple way: \enquote{[\ldots] 'open
source' was designed not to raise [\ldots] the point that users deserve
freedom}. But he and his friends want \enquote{to spread the idea of freedom}
and therefore \enquote{[\ldots] stick to the term 'free software'}
(\cite[][59]{Stallman1998a}). For a brush-up of this position, expressing
again that \enquote{(o)pen source is a development methodolgy [and that] free
software is a social movement} with an \enquote{ethical imparative}
\cite[cf.][31]{Stallman2009a} }: Although both definitions \enquote{[\ldots]
(cover) almost exactly the same range of software}, the \emph{Free Software
Foundation} -- as it is said -- \enquote{prefers [\ldots] (to emphazise) the
idea of freedom [\ldots]} while the \emph{OSI} wants to underline the
philosophically indifferent \enquote{development methodolgy}\footcite[pars pro
toto: cf.][232]{Fogel2006a}.

A third method to group of free software and free software licenses is specified
by the \enquote{Debian Free Software Guideline} which is embedded into the
\enquote{Debian Social Contract}. This \enquote{DFSG} contains nine defining
criteria which -- as Debian itself says -- have been \enquote{[\ldots] adopted
by the free[sic!] software community as the basis of the Open Source
Definition}\footcite[cf.][wp]{DFSG2013a}.

A rough understanding of these methods might result in the conclusion that these
three definitions are extensionally equal and only differ intensionally.
But that is not true. To unveil the differences, let us compare the clusters
\emph{OSI approved licenses}, \emph{OSD compliant licenses}, \emph{DFSG
compliant licenses}, and \emph{FSD compliant licenses} extensionally, by asking
whether they \emph{could} establish different sets of licenses\footnote{Indeed,
for analyzing the extensional power of the definition we have to regard all
potentially covered licenses, not only the already existing licenses, because
the subset of really existing licenses still could be expanded be developing new
licenses which fit the definition.}.

First, the difference most easy to determine is that of an unidirectional
inclusion: By definition, the \emph{OSI approved licenses} and the \emph{OSD
compliant licenses} meet the requirements of the OSD\footcite[cf.][\nopage
wp]{OSI2012a}. But only the \emph{OSI approved licenses} have successfully
passed the OSI process\footcite[cf.][\nopage wp]{OSI2012a} and therefore are
officially listed as \emph{open source licenses}\footcite[cf.][\nopage
wp]{OSI2012b}. Hence, on the one hand, \emph{OSI approved licenses} are
\emph{open source licenses} and vice versa. On the other hand, both -- the
\emph{OSI approved licenses} and the \emph{open source licenses} -- are
\emph{OSD compliant licenses}, but not vice versa.

Second, a similar argumentation allows to distinguish the \emph{DFSG compliant
licenses} from the \emph{OSI approved licenses}. As it is stated, the OSD
\enquote{[\ldots] is based on the Debian Free Software Guideline and any
license that meets one definition almost meets the
other}\footcite[cf.][233]{Fogel2006a}. But then again, meeting the definition is
not enough for being an official open source license: the license has to be
approved by the OSI\footcite[cf.][\nopage wp]{OSI2012b}. Thus, it follows that
all \emph{OSI approved licenses} are also \emph{DFSG compliant licenses}, but
not vice versa.

Third, -- by ignoring the \enquote{few exceptions} which have appeared
\enquote{over the years}\footcite[cf.][233]{Fogel2006a} -- it can be said that,
because of their 'kinsmanlike' relation, at least the \emph{OSD compliant
licenses} are also \emph{DFSG compliant licenses} and vice versa.

Last but not least, it must be stated that the (potential) set of free software
licenses must be greater than all the other three sets: On the one side, the FSD
requires that a license of free software must not only allow to read the
software, but must also permit to use, to modify, and to distribute
it\footcite[cf.][41]{Stallman1996a}. These conditions are covered by at least
the first three paragraphs of the OSD concerning the topics \enquote{Free
Redistribution}, \enquote{Source Code}, and \enquote{Derived
Works}\footcite[cf.][\nopage wp]{OSI2012a}. On the other side, the OSD contains
at least some requirements which are not mentioned by the FSD and which
nevertheless must be met by a license in order to be qualified as an OSD
compliant license\footnote{For example, see the condition that \enquote{the
license must be technology-neutral} (\cite[cf.][\nopage wp]{OSI2012a}).}. It
follows then that there may exist licenses which fulfill all conditions of the
FSD and nevertheless do not fulfill at least some conditions of the
OSD\footnote{Again: we must consider the extensional potential of the
definitions, not the set of really existing licenses. In this context, it is
irrelevant that actually all existing Free Software Licenses like GPL, LGPL or
AGPL indeed are also classfied as open source licenses. We are referring to the
fact that there might be generated licenses which fulfill the FSD, but not the
OSD.}. So, the set of all (potential) \emph{Free Software Licenses} must be
greater than the set of all (potential) \emph{open source licenses} and greater
than the set of \emph{OSD compliant licenses}.

All in all, we can visualize the situation as follows:

\begin{center}

\begin{tikzpicture}
\label{LICTAX}
\small

\node[ellipse,minimum height=5.8cm,minimum width=11.6cm,draw,fill=gray!10] (l0210) at (5,5)
{ };

\draw [-,dotted,line width=0pt,white,
    decoration={text along path,
              text align={center},
              text={|\itshape|All Software Licenses}},
              postaction={decorate}] (0,6.1) arc (120:60:10cm);

\node[ellipse,minimum height=4.4cm,minimum width=10cm,draw,fill=gray!20] (l0210) at (5,5)
{ };

\draw [-,dotted,line width=0pt,white,
    decoration={text along path,
              text align={center},
              text={|\itshape|FSD Compliant Licenses}},
              postaction={decorate}] (0,5.4) arc (120:60:10cm);
              

\node[ellipse,minimum height=3cm,minimum width=8.4cm,draw,fill=gray!30] (l0210) at (5,5)
{ };


             
\draw [-,dotted,line width=0pt,white,
    decoration={text along path,
              text align={center},
              text={|\itshape|OSD Compliant Licenses}},
              postaction={decorate}] (0,4.7) arc (120:60:10cm);
              
\draw [-,dotted,line width=0pt,white,
    decoration={text along path,
              text align={center},
              text={|\itshape|DFSG Compliant Licenses}},
              postaction={decorate}] (0,5) arc (240:300:10cm);
          

\node[ellipse,text width=4.4cm, text centered,minimum height=1.6cm,minimum width=6cm,draw,fill=gray!40] (l0210) at (5,5)
{ \textit{OSI approved licenses} = \\ \textit{\textbf{open source licenses}}
};

\end{tikzpicture}
\end{center}

It should be clear without longer explanations that these clusters don't allow
to extrapolare to the correct compliant behaviour according to the \emph{open source
licenses}: On the one hand, all larger clusters do not talk about the \emph{open
source licenses}. On the other hand, the \emph{open source license cluster}
itself only collects its elements on the base of the OSD which does not stipulates
concrete license fulfilling actions for the licensee.

The next level of clustering \emph{open source licenses} concerns the inner
structure of these \emph{OSI approved licenses}. Even the OSI itself has recently
discussed whether a better kind of grouping the listed licenses would better fit
the needs of the visitors of the OSI site\footcite[cf.][\nopage wp]{OSI2013a}.
And finally the OSI ends up in the categories \enquote{popular and widely used
(licenses) or with strong communities}, \enquote{special purpose licenses},
\enquote{other/miscellaneous licenses}, \enquote{licenses that are redundant
with more popular licenses}, \enquote{non-reusable licenses},\enquote{superseded
licenses}, \enquote{licenses that have been voluntarily retired}, and \enquote{
uncategorized licenses}\footcite[cf.][\nopage wp]{OSI2013b}.

Another way to structure the field of open source licenses is to think in
\enquote{types of open source licenses} by grouping the \enquote{\emph{academic
licenses}, named as such because they were originally created by academic
institutions}\footcite[cf.][69]{Rosen2005a}, the \enquote{\emph{reciprocal
licenses}}, named as such because they \enquote{[\ldots] require the
distributors of derivative works to dis\-tri\-bu\-te those works under same
license including the requirement that the source code of those derivative works
be published}\footcite[cf.][70]{Rosen2005a}, the \enquote{\emph{standard
licenses}}, named as such because they refer to the reusability of
\enquote{industry standards}\footcite[cf.][70]{Rosen2005a}, and the
\enquote{\emph{content licenses}}, named as such because they refer to
\enquote{[\ldots] other than software, such as music art, film, literary works}
and so on\footcite[cf.][71]{Rosen2005a}.

Both kinds of taxonomy directly help to find the relevant licenses which should
be used for new (software) projects. But again: none of these categories 
allow to infer the license compliant behaviour, because the categories are
mostly defined based on license external criteria: whether a license is
published by a specific kind of organization or whether a license deals with
industry standards or other kind of works than software, inherently do not
evoke a license fulfilling behaviour.

Only the act of grouping into the \enquote{\emph{academic licenses}} and the
\enquote{\emph{reciprocal licenses}} touches the idea of license fulfilling
doings, if one -- as it has been done -- expands the definition of the
\enquote{\emph{academic licenses}} by the specification that these licenses
\enquote{[\ldots] allow the software to be used for any purpose whatsoever with
no obligation on the part of the licensee to distribute the source code of
derivative works}\footcite[cf.][71]{Rosen2005a}. With respect to this additional
specification, the clusters \enquote{\emph{academic licenses}} and the
\enquote{\emph{reciprocal licenses}} indeed might be referred as the
\enquote{main categories} of (open source)
licenses\footcite[cf.][179]{Rosen2005a}: By definition, they are constituting
not only a contrary, but contradictory opposite. However, it must  be kept in
mind that they constitute an inherent antagonism, an antinomy inside of the set
of open source licenses\footnote{Hence, it is at least a little confusing to say
that \enquote{the open source license (OSL) is a reciprocal license} and
\enquote{the Academic Free License (AFL) is the exact same license without the
reciprocity provisions} (\cite[cf.][180]{Rosen2005a}): If the BSD license is an
AFL and if an AFL is not an OSL and if the OSI approves only OSLs, then the BSD
license can not be an approved open source license. But in fact, it still is
(\cite[cf.][\nopage wp]{OSI2012b}).}.

Connatural to the clustering into \emph{academic licenses} and \emph{reciprocal
licenses} is the grouping into \emph{permissive licenses}, \emph{weak copyleft
licenses}, and \emph{strong copyleft licenses}: Even Wikipedia already uses the
term \enquote{permissive free software licence} in the meaning of \enquote{a
class of free software licence[s] with minimal requirements about how the
software can be redistributed} and \enquote{contrasts} them with the
\enquote{copyleft licences} as those \enquote{with reciprocity / share-alike
requirements}\footcite[cf.][\nopage wp]{wpPermLic2013a}. 

Some other authors name the set of \emph{academic licenses} the
\enquote{permissive licenses} and specify the \emph{reciprocal licenses} as
\enquote{restrictive licenses}, because in this case -- as a consequence of the
embedded \enquote{copyleft} effect -- the source code must be published in case
of modifications. They additionally introduce the subset of \enquote{strong
restrictive licenses} which additionally require that an (overarching)
derivative work must be published under the same license\footcite[pars pro toto
cf.][57]{Buchtala2007a}. The next refinement of such clustering concepts
directly uses the categories \enquote{[open source] licenses with a strict
copyleft clause}\footcite[Originally stated as \enquote{Lizenzen mit einer
strengen Copyleft-Klausel}. Cf.][24]{JaeMet2011a}, \enquote{[open source]
licenses with a restricted copyleft clause}\footcite[Originally stated as
\enquote{Lizenzen mit einer beschränkten Copyleft-Klausel}.
Cf.][71]{JaeMet2011a}, and \enquote{[open source] licenses without any copyleft
clause}\footcite[Originally stated as \enquote{Lizenzen ohne Copyleft-Klausel}.
Cf.][83]{JaeMet2011a}. Finally, this viewpoint can directly be mapped to the
categories \emph{strong copyleft} and \emph{weak copyleft}: While on the one
hand, \enquote{only changes to the weak-copylefted software itself become
subject to the copyleft provisions of such a license, [and] not changes to the
software that links to it}, on the other hand, the \enquote{strong copyleft}
states \enquote{[\ldots] that the copyleft provisions can be efficiently imposed
on all kinds of derived works}\footcite[cf.][\nopage wp]{wpCopyleft2013a}.

Based on this approach to an adequate clustering and labeling\footnote{Finally,
we should also mention that there still exists other classifications which might
become important in other contexts. For example, the ifross license subsumes
under the main category \enquote{Open Source Licenses} the subcategories
\enquote{Licenses without Copyleft Effect}, \enquote{Licenses with Strong
Copyleft}, \enquote{Licenses with Restricted Copyleft}, \enquote{Licenses with
Restricted Choice}, or \enquote{Licenses with Privilegs} -- and let finally
denote these categories also licenses which are not listed by the OSI
(\cite[cf.][\nopage wp.]{ifross2011a}). This is well reasonable if one refers to
the meaning of the OSD (\cite[cf.][\nopage wp]{OSI2012a}). The OSLiC wants to
simplify its object of study by referring to the approved open source licenses
(\cite[cf.][\nopage wp]{OSI2012d}) listed by the OSI (\cite[cf.][\nopage
wp]{OSI2012b}).}, we can develop the following picture:

\begin{center}

\begin{tikzpicture}
\label{OSLICTAX}
\small



\node[ellipse,minimum height=8.5cm,minimum width=14cm,draw,fill=gray!10] (l0100) at (6.8,6.8)
{  };

\draw [-,dotted,line width=0pt,white,
    decoration={text along path,
              text align={center},
              text={|\itshape| OSI approved licenses}},
              postaction={decorate}] (-0.8,6.5) arc (142:38:9.5cm);

\draw [-,dotted,line width=0pt,white,
    decoration={text along path,
              text align={center},
              text={|\itshape|open source licenses}},
              postaction={decorate}] (-0.8,6.5) arc (218:322:9.5cm);
              
\node[ellipse,minimum height=6.2cm,minimum width=4cm,draw,fill=gray!20] (l0100) at (2.75,6.8)
{  };

\draw [-,dotted,line width=0pt,white,
    decoration={text along path,
              text align={center},
              text={|\itshape| permissive licenses}},
              postaction={decorate}] (0.9,7.4) arc (180:0:1.8cm);

\node[circle,draw,text width=1cm, fill=gray!40, text centered] (l0101) at (2,8)
{  \footnotesize \bfseries \textit{ApL}};
\node[circle,draw,text width=1cm, fill=gray!40, text centered] (l0102) at (3.5,8)
{  \footnotesize \bfseries \textit{BSD}};
\node[circle,draw,text width=1cm, fill=gray!40, text centered] (l0103) at (2,6.5)
{  \footnotesize \bfseries \textit{MIT}};
\node[circle,draw,text width=1cm, fill=gray!40, text centered] (l0104) at (3.5,6.5)
{  \scriptsize \bfseries \textit{MS-PL}};
\node[circle,draw,text width=1cm, fill=gray!40, text centered] (l0105) at (2,5)
{  \footnotesize \bfseries \textit{PgL}};
\node[circle,draw,text width=1cm, fill=gray!40, text centered] (l0106) at (3.5,5)
{  \footnotesize \bfseries \textit{PHP}};

\node[ellipse,minimum height=6cm,minimum width=8.5cm,draw,fill=gray!20] (l0200) at (9.2,6.5)
{  };

\draw [-,dotted,line width=0pt,white,
    decoration={text along path,
              text align={center},
              text={|\itshape| copyleft licenses}},
              postaction={decorate}] (7.5,8.5) arc (120:60:4cm);


\node[ellipse,minimum height=4.5cm,minimum width=4.2cm,draw,fill=gray!30] (l0210) at (7.45,6.5)
{  };

\draw [-,dotted,line width=0pt,white,
    decoration={text along path,
              text align={center},
              text={|\itshape| weak copyleft licenses}},
              postaction={decorate}] (5.4,6.2) arc (180:0:2cm);

\node[circle,draw,text width=1cm, fill=gray!40, text centered] (l0211) at (6.7,7)
{  \footnotesize \bfseries \textit{EPL}};
\node[circle,draw,text width=1cm, fill=gray!40, text centered] (l0212) at (8.2,7)
{  \footnotesize \bfseries \textit{EUPL}};
\node[circle,draw,text width=1cm, fill=gray!40, text centered] (l0213) at (6.7,5.5)
{  \footnotesize \bfseries \textit{LGPL}};
\node[circle,draw,text width=1cm, fill=gray!40, text centered] (l0214) at (8.2,5.5)
{  \footnotesize \bfseries \textit{MPL}};

\node[ellipse,minimum height=4.5cm,minimum width=3cm,draw,fill=gray!30] (l0220) at (11.4,6.5)
{  };
 
% line width=0pt,white,
\draw [-,dotted,line width=0pt,white,
    decoration={text along path,
              text align={center},
              text={|\itshape| strong copyleft}},
              postaction={decorate}] (10.4,7) arc (180:0:1cm);

\draw [-,dotted,line width=0pt,white,
    decoration={text along path,
              text align={center},
              text={|\itshape| licenses}},
              postaction={decorate}] (10.4,5.4) arc (180:360:1cm);        

\node[circle,draw,text width=1cm, fill=gray!40, text centered] (l0221) at (11.4,7)
{  \footnotesize \bfseries \textit{GPL}};
\node[circle,draw,text width=1cm, fill=gray!40, text centered] (l0222) at (11.4,5.5)
{  \footnotesize \bfseries \textit{AGPL}};


\end{tikzpicture}
\end{center}

This extensionally based clarification of a possible open source license
taxonomy is probably well-known and often -- more or less explicitly --
referred to\footnote{Even the FSF itself uses the term 'permissive non-copyleft
free software license' (\cite[pars pro toto: cf.][\nopage wp/section 'Original BSD
license']{FsfLicenseList2013a}) and contrasts it with the terms 'weak copyleft'
and 'strong copyleft' (\cite[pars pro toto: cf.][\nopage wp/section 'European
Union Public License']{FsfLicenseList2013a})}. Unfortunately, this taxonomy
still contains some misleading underlying messages:

\emph{Permissive} has a very positive connotation. So, the antinomy of
\emph{permissive licenses} versus \emph{copyleft licenses} implicitly signals,
that the \emph{permissive licenses} are in any meaning better, than the
\emph{copyleft licenses}. Naturally, this 'conclusion' is evoked by
confusing the extensionally definition and the intensional power of the labels.
But that is the way we -- the human beings -- like to think. 

Anyway, this underlying message is not necessarily 'wrong'. It might be
convenient for those people or companies who only want to use open source
software without being restricted by the \emph{obligation to give something
back} as it has been introduced by the 'copyleft'\footnote{De facto,
\emph{copyleft} is not \emph{copyleft}. Apart from the definition, its effect
depends on the par\-ti\-cu\-ar licenses which determine the conditions for
applying the copyleft 'method'. For example, in the GPL, the copyleft effect is
bound to the criteria 'being distributed'. Later on, we will collect these
conditions systematically (see chapter \emph{\nameref{sec:OSUCdeduction}}, pp.\
\pageref{sec:OSUCdeduction}). Therefore, here we still permit ourselves to use a
somewhat 'generalizing' mode of speaking.}. But there might be other people and
companies who emphasize the protecting effect of the copyleft licenses. And
indeed, at least the open source license\footnote{Although RMS naturally prefers
to specify it as a \emph{Free Software License} (s. p.\ \pageref{RmsFsPriority})
} \emph{GPL}\footnote{As the original source \cite[cf.][\nopage
wp]{Gpl20FsfLicense1991a}. Inside of the OSLiC, we constantly refer to the
license versions which are published by the OSI, because we are dealing with
officially approved open source licenses. For the 'OSI-GPL' \cite[cf.][\nopage
wp]{Gpl20OsiLicense1991a}} has initially been developed to protect the freedom,
to enable the developers to help their \enquote{neighbours} and to get the
modifications back\footnote{The history of the GNU project is multiply told. For
the GNU project and its initiator \cite[cf.\ pars pro toto][\nopage
passim]{Williams2002a}. For a broader survey \cite[cf.\ pars pro toto][\nopage
passim]{Moody2001a}. A very short version is delivered by Richard M. Stallman
himself where he states that -- in the years while the early free community were
destroyed -- he saw the \enquote{nondisclosure agreement} which must be signed ,
\enquote{[\ldots] even to get an executable copy} as a clear \enquote{[\ldots]
promise not to help your neighbour}: \enquote{A cooperating community was
forbidden.} (\cite[cf.][16]{Stallman1999a}).}: So, \enquote{Copyleft} is defined
as a \enquote{[\ldots] method for making a program free software and requiring
all modified and extended versions of the program to be free software as
well}\footcite[cf.][89]{Stallman1996c}. It is a method\footnote{Based on the
American legal copyright system, this method uses two steps: firstly one states,
\enquote{[\ldots] that it is copyrighted [\ldots]} and secondly one adds those
\enquote{[\ldots] distribution terms, which are a legal instrument that gives
everyone the rights to use, modify, and redistribute the program's code or any
program derived from it but only if the distribution terms are unchanged}
(\cite[cf.][89]{Stallman1996c}).} by which \enquote{[\ldots] the code and the
freedoms become legally inseparable}\footcite[cf.][89]{Stallman1996c}. Because
of these disparate interests of hoping not to be restricted and hoping to be
protected, it could be helpful to find a better label -- an impartial name for
the cluster of \emph{permissive licenses}. But until that time, we should at
least know that this taxonomy still contains an underlying declassing message.

The other misleading interpretation is -- counter-intuitively -- evoked by using
the concept of 'copyleft licenses'. By referring to a cluster of \emph{copyleft
licenses} as the opposite of the \emph{permissive licenses}, one implicitly also
sends two messages: First, that republishing one's own modifications is
sufficient to comply with the \emph{copyleft licenses}. And, secondly, that the
\emph{permissive licenses} do not require anything to be done for obtaining the
right to use the software. Even if one does not wish to evoke such an
interpretation, we -- the human beings -- tend to take the things as simple as
possible\footnote{And indeed, in the experience of the authors -- sometimes --
such simplifications gain their independent existence and determine decisions on
the management level. But that is not the fault of the managers. It is their
job, to aggregate, generalize and simplify information. It is the job of the
experts, to offer better viewpoints without overwhelming the others with
details.}. But because of several aspects, this understanding of the antinomy of
\emph{copyleft licenses} and \emph{permissive licenses} is too misleading for
taking it as a serious generalization:

On the one hand, even the 'strongly copylefted' GPL requires also other tasks
than just the republishing of derivative works. For example, it also calls for
to \enquote{[\ldots] give any other recipients of the [GPL licensed] Program a
copy of this License along with the Program}\footcite[cf.][\nopage wp.\
§1]{Gpl20OsiLicense1991a}. Furthermore, the 'weakly copylefted' licenses require
also more and different criteria which has to be fulfilled for acting according
to these licenses. For example, the EUPL requires that the licensor who does not
directly deliver the binaries together with the sourcecode, must offer a
sourcecode version of his work free of charge\footnote{The German version of the
EUPL uses the phrase \enquote{problemlos und unentgeltlich(sic!) auf den
Quellcode (zugreifen können)} (\cite[cf.][3, section 3]{EuplLicense2007de})
while the English version contains the specification \enquote{the Source Code is
easily and freely accessible} (\cite[cf.][2, section 3]{EuplLicense2007en})},
while the MPL requires that under the same circumstances a recipient
\enquote{[\ldots] can obtain a copy of such Source Code Form [\ldots] at a
charge no more than the cost of distribution to the recipient
[\ldots]}\footcite[cf.][\nopage section 3.2.a]{Mpl20OsiLicense2013a}.
And last but not least, also the \emph{permissive licenses} require tasks which
must be fulfilled for a license compliant usage -- moreover, they also require
different things. For example, the BSD demands that \enquote{the
(re)distributions [\ldots] must (retain [and/or]) reproduce the above copyright
notice [\ldots]}. Because of the structure of the \enquote{copyright notice},
this compulsory notice implies that the authors / copyright holders of the
software must be publicly named\footcite[cf.][\nopage wp]{BsdLicense2Clause}. As
opposed to this, the Apache License requires that \enquote{if the Work includes
a "NOTICE" text file as part of its distribution, then any Derivative Works that
You distribute must include a readable copy of the attribution notices contained
within such NOTICE file} which often means that you have to present central
parts of such file publicly\footcite[cf.][\nopage wp.\ section
4.4]{Apl20OsiLicense2004a} -- parts which can contain much more information than
only the names of the authors / copyright holders.

So, no doubt -- and contrary to the intuitive interpretation of this taxonomy --
each \emph{open source license} must be fulfilled by some actions, even the most
permissive one. And for ascertaining these tasks, one has to look into these
licenses themselves, not the generalized concepts of licenses taxonomies. Hence
again, we have to state that even this well known type of grouping of \emph{open
source licenses} does not allow to derive a specific license compliant behavior:
The taxonomy might be appropriate, if one wants to live with the implicite
messages and generalizations of some of its concepts. But the taxonomy is not an
adequate tool to determine, what one has to do for fulfilling an \emph{open
source license}. A license compliant behaviour for obtaining the right to use a
specific piece of \emph{open source software} must be based on the concrete
\emph{open source license} by which the licensor has licensed the software.
There is no shortcut.

Nevertheless, human beings need generalizing and structuring viewpoints for
enabling themselves to talk about a domain -- even if they finally have to regard
the single objects of the domain for specific purposes. We think that there is
a subtler method to regard and to structure the domain of \emph{open source
licenses}. So, we want to offer this other possibility to cluster the \emph{open
source licenses}\footnote{even if also we have to concede that, ultimately, 
one has to always look into the license itself}:

We think that, in general, licenses have a common purpose: they should protect
someone or something against something. The structure of this task is based on
the nature of the word 'protect' which is a trivalent verb: it links someone or
something who protects, to someone or something who is protected and both
combined to something against which the protector protects and against the other one
is protected. Licenses in general do that. Therefore, it is also the purpose of
open source licenses to protect: They can protect the user (recipient) of the
software, its contributor resp.\ developer and/or distributor, and the software
itself. And they can protect them against different threats:

\begin{itemize}
  \item First, we assume, that - in the context of open source software - the
  user can be protected against the loss of the right to use it, to modify it,
  and to redistribute it. Additionally, he can be protected against patent
  disputes.
  \item Second, we assume, that open source contributors and distributors can be
  protected against the loss of feedback in the form of code improvements and
  derivatives, against warranty claims, and against patent disputes.
  \item Third, we assume, that the open source programs and their specific forms
  -- may they be distributed or not, may they be modified or not, may they be
  distributed as binaries or as sources -- can be protected against the
  re-closing resp. against the re-privatizing of their further development.
  \item Fourth, we want to assume that new on-top developments being based on
  open source components can be protected against the privatizing for enlarging
  the world of freely usable software\footnote{In a more rigid version, this
  capability of a license could also be identified as the power to protect the
  community against a stagnation of the set of open source software -- but this
  description is at least a little to long to be used by the following pages}.
\end{itemize}

With respect to these viewpoints, one gets a subtler picture of the license
specific protecting power. Thus, we are going to describe and deduce the
protecting power of each of the open source licenses on the following pages.
Table \ref{tab:powerOfLicenses} summarizes the results as a quick
reference\footnote{$\rightarrow$ table \ref{tab:powerOfLicenses} on p.\
\pageref{tab:powerOfLicenses}}.

\begin{table}
\begin{minipage}{\textwidth}
\centering
\footnotesize
\caption{Open Source Licenses as Protectors}
\label{tab:powerOfLicenses}

\begin{tabular}{|c|c||c|c|c|c|c|c|c|c|c|c|c|c|c|c|c|}
\hline
  \multicolumn{2}{|c|}{\textit{Open}} &
  \multicolumn{13}{c|}{\textit{are protecting}}\\
\cline{3-15}
  \multicolumn{2}{|c|}{\textit{Source}} &
  \multicolumn{4}{c|}{ \textbf{Users}} &
  \multicolumn{3}{c|}{\textbf{Contributors}} &
  \multicolumn{5}{c|}{\textbf{Open Source Software}} &
  \multirow{4}{*}{\rotatebox{270}{\scriptsize{\textbf{On-Top Develop.\ }}}} 
  \\
\cline{10-14}
  \multicolumn{2}{|c|}{\textit{Licenses\footnote{'\checkmark' indicates that the
  license protects with respect to the meaning of the column, '$\neg$' indicates
  that the license does not protect with regard to the meaning of the column,
  and '--' indicates, that the corresponding statement must still be evaluated.
  \textit{Slanted names of licenses} indicate that these licenses are only
  listed in this table while the corresponding mindmap ($\rightarrow$ p.\
  \pageref{OSCLICMM}) does not cover them }}} &
  \multicolumn{4}{c|}{} &
  \multicolumn{3}{c|}{\tiny{(Distributors)}} &  
  not &
  \multicolumn{4}{c|}{distributed as} 
  & \\
\cline{3-9}\cline{11-14}
  \multicolumn{2}{|c|}{} &
  \multicolumn{4}{c|}{\scriptsize{\textit{who have already got}}} &
  \multicolumn{3}{c|}{\scriptsize{\textit{who spread open}}} & 
  distri- &
  \multicolumn{2}{c|}{unmodified} &
  \multicolumn{2}{c|}{modified} 
  & \\
  \cline{11-14}
  \multicolumn{2}{|c|}{} &
  \multicolumn{4}{c|}{\scriptsize{\textit{sources or binaries}}} &
  \multicolumn{3}{c|}{\scriptsize{\textit{source software}}} & 
  buted & 
 \rotatebox{270}{\footnotesize{sources\ }} &
 \rotatebox{270}{\footnotesize{binaries\ }} &
 \rotatebox{270}{\footnotesize{sources\ }} &
 \rotatebox{270}{\footnotesize{binaries\ }} 
 & \\
\cline{3-15}
  \multicolumn{2}{|c|}{} &
  \multicolumn{13}{c|}{\textit{against}}\\
\cline{3-15}
  \multicolumn{2}{|c|}{} &
  \multicolumn{3}{c|}{the loss of} & 
  \multirow{3}{*}{\rotatebox{270}{Patent Disputes}} &
  \multirow{3}{*}{\rotatebox{270}{Loss of Feedback}} & 
  \multirow{3}{*}{\rotatebox{270}{Warranty Claims}} & 
  \multirow{3}{*}{\rotatebox{270}{Patent Disputes}} & 
  \multicolumn{5}{c|}{}
  & \\
% no seperator line 
  \multicolumn{2}{|c|}{} &
  \multicolumn{3}{c|}{the right to} &
  & & & &
  \multicolumn{5}{c|}{\footnotesize{Re-Closings / Re-Privatizings}} &
  \multirow{3}{*}{\rotatebox{270}{Privatizings}}
   \\
\cline{3-5}
  \multicolumn{2}{|c|}{} & 
  \rotatebox{270}{use it} & 
  \rotatebox{270}{modify it} & 
  \rotatebox{270}{redistribute it\ } &
  &  &  &  &
  \multicolumn{5}{c|}{of already opened software}
  & \\
\hline
\hline
  ApL & 2.0 & \checkmark  & \checkmark  & \checkmark  &
  \checkmark & $\neg$ & \checkmark & \checkmark & $\neg$ &
   \checkmark  & $\neg$ & \checkmark & $\neg$ & $\neg$ \\
\hline
  \multirow{2}{*}{BSD} & 3-Cl & \checkmark & \checkmark  & \checkmark  & 
    $\neg$ & $\neg$ & \checkmark & $\neg$  &
    $\neg$ & \checkmark  & $\neg$ & \checkmark & $\neg$ & $\neg$ \\
\cline{2-15}
   & 2-Cl & \checkmark  & \checkmark  & \checkmark  & 
    $\neg$ & $\neg$ & \checkmark & $\neg$  &
    $\neg$ & \checkmark  & $\neg$ & \checkmark & $\neg$ & $\neg$ \\
\hline
  MIT & ~ & \checkmark  & \checkmark  & \checkmark  &
  $\neg$ & $\neg$ & \checkmark & $\neg$ & $\neg$ &
   \checkmark  & $\neg$ & \checkmark & $\neg$ & $\neg$ \\
\hline
  MS-PL & ~ & \checkmark  & \checkmark  & \checkmark  &
  \checkmark & $\neg$ & \checkmark & \checkmark & $\neg$ &
   \checkmark  & $\neg$ & \checkmark & $\neg$ & $\neg$ \\
\hline
  PgL & ~ & \checkmark  & \checkmark  & \checkmark  &
  $\neg$ & $\neg$ & \checkmark & $\neg$ & $\neg$ &
   \checkmark  & $\neg$ & \checkmark & $\neg$ & $\neg$ \\
\hline
  PHP & 3.0 & \checkmark  & \checkmark  & \checkmark  &
  $\neg$ & $\neg$ & \checkmark & $\neg$ & $\neg$ &
   \checkmark  & $\neg$ & \checkmark & $\neg$ & $\neg$ \\
\hline
\hline
  \textit{CDDL} & 1.0 & \checkmark & \checkmark & \checkmark &
  -- & -- & -- & -- & -- & -- & -- & -- & -- & -- \\
\hline
  EPL & 1.0 & \checkmark  & \checkmark  & \checkmark  &
  \checkmark  & \checkmark  & \checkmark & \checkmark & $\neg$ &
   \checkmark  & \checkmark & \checkmark & \checkmark & $\neg$ \\
\hline
  EUPL & 1.1 & \checkmark  & \checkmark  & \checkmark  &
  \checkmark  & \checkmark  & \checkmark & \checkmark & $\neg$ &
   \checkmark  & \checkmark & \checkmark & \checkmark & $\neg$ \\
\hline
  \multirow{2}{*}{LGPL} & 2.1 & \checkmark & \checkmark & \checkmark &
  -- & -- & -- & -- & -- & -- & -- & -- & -- & -- \\
\cline{2-15}
   & 3.0 & \checkmark & \checkmark & \checkmark &
   -- & -- & -- & -- & -- & -- & -- & -- & -- & -- \\
\hline
   MPL & 2.0 & \checkmark  & \checkmark  & \checkmark  &
  \checkmark  & \checkmark  & \checkmark & \checkmark & $\neg$ &
   \checkmark  & \checkmark & \checkmark & \checkmark & $\neg$ \\
\hline
  \textit{MS-RL} & ~ & \checkmark & \checkmark & \checkmark &
  -- & -- & -- & -- & -- & -- & -- & -- & -- & -- \\
\hline
\hline
  AGPL & 3.0 & \checkmark & \checkmark & \checkmark &
  -- & -- & -- & -- & -- & -- & -- & -- & -- & -- \\
\hline
  \multirow{2}{*}{GPL} & 2.1 & \checkmark & \checkmark & \checkmark &
   -- & -- & -- & -- & -- & -- & -- & -- & -- & -- \\
\cline{2-15}
  & 3.0 & \checkmark & \checkmark & \checkmark &
   -- & -- & -- & -- & -- & -- & -- & -- & -- & -- \\
\hline
\hline

\end{tabular}

\end{minipage}
\end{table}

\section{The protecting power of the Affero Gnu Public License (AGPL) [tbd]}
\begin{itemize} 
  \item The Affero Gnu Public License protects \ldots
  \item But the Affero Gnu Public License does not protect \ldots
\end{itemize}


\section{The protecting power of the Apache License (ApL)}
\label{sec:ProtectPowerOfApL}

As an approved \emph{open source license}\footcite[cf.][\nopage wp]{OSI2012b},
the Apache License\footnote{The Apache License, version 2.0 is maintained by the
Apache Software Foundation (\cite[cf.][\nopage wp]{AsfApacheLicense20a}).  Of
course, also the OSI is hosting a duplicate (\cite[cf.][\nopage
wp]{Apl20OsiLicense2004a}) in its list of the officially approved open source
licenses (\cite[cf.][\nopage wp]{OSI2012b}). The Apache license 1.1 is
classified by the OSI as \enquote{superseded license}(\cite[cf.][\nopage
wp]{OSI2013b}). In the same spirit, also the Apache Software Foundation itself
classifies the releases 1.0 and 1.1 as \enquote{historic} (\cite[cf.][\nopage
wp]{AsfLicenses2013a}). Thus, the OSLiC only focuses on the most recent APL-2.0
version. For those, who have to fulfill these earlier Apache licenses it could
be helpful to read them as siblings of the BSD-2CL and BSD-3CL licenses.}
protects the user against the loss of the right to use, to modify and/or to
distribute the received copy of the source code or the
binaries\footcite[cf.][\nopage wp §2]{Apl20OsiLicense2004a}. Furthermore, based
on its patent clause\footnote{$\rightarrow$ OSLiC pp.\
\pageref{subsec:ApLPatentClause}}, the ApL protects the users against patent
disputes\footcite[cf.][\nopage wp §3]{Apl20OsiLicense2004a}. Because of this
patent clause and of its \enquote{disclaimer of warranty} together with its
\enquote{limitation of liability}, the Apache license also protects the
contributors / distributors against patents disputes and warranty
claims\footcite[cf.][\nopage wp §3, §7, §8]{Apl20OsiLicense2004a}. Finally, the
ApL protects the distributed sources themselves \emph{against} a change of the
license which would \emph{reset} the work \emph{as closed software}, because
first, one \enquote{[\ldots] must give any other recipients of the work or
derivative works a copy of (the Apache) license}, second, \enquote{in the source
form of any derivative works that (one) distributes}, one has \enquote{[\ldots]
to retain [\ldots] all copyright, patent, trademark, and attribution notices
[\ldots]}, and third, one must \enquote{[\ldots] include a readable copy [\ldots
of the] NOTICE file} being supplied by the original package one has
received\footcite[cf.][\nopage wp §4]{Apl20OsiLicense2004a}.

But the Apache License does not protect the contributors against the loss of
feedback because it does not 'copyleft' the software: the Apache license does
not contain any sentence requiring that one has also to publish the source code.
In the same spirit, the ApL does not protect the undistributed software or the
distributed binaries against re-closings -- neither in unmodified nor in
modified form -- because the Apache License allows to (re)distribute the
binaries without also supplying the sources -- even if the binaries rest upon
sources modified by the distributor. Finally, the ApL does not protect the
on-top developments against a privatizing.


\section{The protecting power of the BSD licenses}

As approved \emph{open source licenses}\footcite[cf.][\nopage wp]{OSI2012b}, the
BSD Licenses\footnote{BSD has to be resolved as \emph{Berkely Software
Distribution}. For details of the BSD license release and namings
\cite[cf.][\nopage wp.\ editorial]{BsdLicense3Clause}} protect the user against
the loss of the right to use, to modify and/or to distribute the received copy
of the source code or the binaries\footcite[cf.][\nopage wp §1ff]{OSI2012a}.
Additionally, they protect the contributors and/or distributors against warranty
claims of the software users, because these licenses contain a 'No Warranty
Clause'\footcite[one for all version cf.][\nopage wp]{BsdLicense2Clause}. And
finally they protect the distributed sources against a change of the license
which closes the sources, because each modification and \enquote{redistributions
of [the] source code must retain the [\ldots] copyright notice, this list of
conditions and the [\ldots] disclaimer}\footcite[cf.][\nopage
wp]{BsdLicense2Clause}: Therefore it is incorrect to distribute a BSD licensed
code under another license -- regardless, whether it closes the sources or
not\footnote{In common sense based discussions you may have heard that BSD
licenses allow to republish the work under another, an own license. Taking the
words of the BSD License seriously that is not valid under all circumstances:
Yes, it is true, you are not required to redistribute the sourcecode of a
modified (derivative) work. You are allowed to modify a received version and to
distribute the results only as binary code and to keep your improvements closed.
But if you distribute the source code of your modifications, you have retain the
licensing, because \enquote{Redistribution [\ldots] in source [\ldots], with or
without modification, are permitted provided that [\ldots] (the) redistributions
of source code [\ldots] retain the above copyright notice, this list of
conditions and the following disclaimer} (\cite[cf.][\nopage
wp]{BsdLicense2Clause})}.

But the BSD Licenses protect neither the users nor the contributors
and/or distributors against patent disputes (because they do not contain any
patent clause). They do not protect the contributors against the loss of
feedback (because they do not 'copyleft' the software). Moreover, they do not
protect the undistributed software or the distributed binaries against
re-closings -- neither in unmodified nor in modified form -- because they
allow to redistribute only the binaries without also supplying the source
code\footnote{see both, the BSD-2CL License (\cite[cf.][\nopage
wp]{BsdLicense2Clause}), and the BSD-3CL License (\cite[cf.][\nopage
wp]{BsdLicense3Clause})}. Finally, the BSD licenses do not protect the on-top
developments against a privatizing.


\section{The protecting power of the Eclipse Public License (EPL)}
\label{sec:ProtectingPowerOfEpl}

As an approved \emph{open source license}\footcite[cf.][\nopage wp]{OSI2012b},
the Eclipse Public License\footnote{ The Eclipse Public License, version 1.0 is
maintained by the Eclipse Software Foundation (\cite[cf.][\nopage
wp]{Epl10EclipseFoundation2005a}).  Of course, also the OSI is hosting a
duplicate (\cite[cf.][\nopage wp]{Epl10OsiLicense2005a}).} protects the user
against the loss of the right to use, to modify and/or to distribute the
received copy of the source code or the binaries\footcite[cf.][\nopage wp
§2a]{Epl10OsiLicense2005a}. Furthermore, based on its patent
clause\footnote{$\rightarrow$ OSLiC pp.\ \pageref{subsec:EpLPatentClause}}, the
EPL protects the users also against patent disputes\footcite[cf.][\nopage wp §2b
\& §2c]{Epl10OsiLicense2005a}. Besides this patent clause, the EPL contains the
sections \enquote{no warranty} and \enquote{disclaimer of
liability}\footcite[cf.][\nopage wp §5 \& §6]{Epl10OsiLicense2005a}. These three
elements together protect the contributors / distributors against patents
disputes and warranty claims. Finally, the EPL protects the distributed sources
themselves \emph{against} a change of the license which would \emph{reset} the
work \emph{as closed software}: First, the Eclipse Public Licenses requires that
if a work -- released under the EPL -- \enquote{[\ldots] is made available in
source code form [\ldots] (then) it must be made available under this (EPL)
agreement, too} while this act of 'making avalaible' \enquote{must} incorporate
a \enquote{copy} of the EPL into \enquote{each copy of the [distributed]
program} or program package\footcite[cf.][\nopage wp §3]{Epl10OsiLicense2005a}.
But in opposite to the permissive licenses, the EPL does not only protect the
distributed source code -- regardless whether it is modified or not. The EPL
also protects the distributed modified or unmodified binaries: The EPL allows
each modifying \enquote{contributor} and distributor \enquote{[\ldots] to
distribute the Program in object code form under (one's) own license agreement
[\ldots]} provided this license clearly states that the \enquote{source code for
the Program is available} and where the \enquote{licensees} can
\enquote{[\ldots] obtain it in a reasonable manner on or through a medium
customarily used for software exchange}\footcite[cf.][\nopage wp. §3, esp.
§3.b.iv]{Epl10OsiLicense2005a}. Thus, one has to conclude that the EPL is a
copyleft license.

But the Eclipse Public License is not a license with strong copyleft; the EPL
uses 'only' a weak copyleft effect\footnote{Even if one can find contrary
specifications in the internet. \cite[Pars pro toto cf.][\nopage wp.
]{ifross2011a}: This page is listing the EPL in the section \enquote{Other
Licenses with strong Copyleft Effect}}: Indeed, the EPL says that for each EPL
licensed \enquote{program} -- distributed in object form -- a place must be made
known where one can get the corresponding source code\footcite[cf.][\nopage wp.
§3, esp. §3.b.iv]{Epl10OsiLicense2005a}. The term 'Program' is defined as any
\enquote{Contribution distributed in accordance with [\ldots] (the EPL)} while
the term 'Contribution' refers - besides other elements - to \enquote{changes to
the Program, and additions to the Program}\footcite[cf.][\nopage wp.
§1]{Epl10OsiLicense2005a}. Unfortunately, this is a circular definition:
'Program' is defined by 'Contribution'; and 'Contribution' is defined by
'Program'. Nevertheless, one has to read the license benevolently.
Uncontroversial should be this: If one distributes any modified EPL licensed
program, library, module, or plugin, then one has to publish the modified source
code, too. If one \enquote{adds} some own plugins or additional libraries which
are used by an EPL licensed program (which on behalf of this use must have been
modified by adding [sic!] procedure calls) then one has publish the code of both
parts: that of the program and that of the added elements. In this sense, the
EPL clearly protects the binaries against re-closings like other weak copyleft
using licenses. But if one distributes only an EPL licensed library which is
used as a component by another not EPL licensed on-top program, then this
library does not depend on the top development -- provided that the library
itself does not call any (program) functions or procedures delivered by the
overarching on-top development. Hence, nothing is added to the library; and
hence, no other code than that of the library must be published. Therefore, the
EPL does not use the strong copyleft effect in the meaning of the GPL.
 
\section{The protecting power of the European Union Public License (EUPL)}

As an approved \emph{open source license}\footcite[cf.][\nopage wp]{OSI2012b},
the European Union Public License\footnote{ The European Union Public License,
version 1.1 is maintained by the European Union and hosted under the label
\enquote{Joinup} (\cite[cf.][\nopage wp]{EuplLicense2007en}).
This EUPL has officially been translated into many languages, among others into
German (\cite[cf.][\nopage wp]{EuplLicense2007de}). Because of this multi
lingual instances, the OSI does not offer its own version, but just a landing
page linked to the lading page of the European host \enquote{Joinup}
(\cite[cf.][\nopage wp]{Eupl11OsiLicense2007a}).} protects the user against the
loss of the right to use, to modify and/or to distribute the received copy of
the source code or the binaries\footcite[cf.][\nopage wp.\
§2]{EuplLicense2007de}. Furthermore, based on its patent
clause\footnote{$\rightarrow$ OSLiC pp.\ \pageref{subsec:EupLPatentClause}}, the
EUPL protects the users against patent disputes\footcite[cf.][\nopage wp.\ §2,
at its tail]{EuplLicense2007en}. Besides this patent clause, the EUPL
additionally contains a \enquote{Disclaimer of Warranty} and a
\enquote{Disclaimer of Liability}\footcite[cf.][\nopage wp.\ §7 \&
§8]{EuplLicense2007en}. These three elements together protect the contributors /
distributors against patents disputes and warranty claims. Finally, the EUPL
also protects the distributed sources against a re-closing / re-privatizizing
and the contributors against the loss of feedback. This protection is based on
two steps: First, the Europrean Public License contains a particular paragraph
titled \enquote{Copyleft clause} which stipulates that \enquote{copies of the
Original Work or Derivative Works based upon the Original Work} must be
distributed \enquote{under the terms of (the European Union Public)
License}\footcite[cf.][\nopage wp.\ §5]{EuplLicense2007en}. Second, the EUPL
requires that each licensee -- as long as he \enquote{[\ldots] continues to
distribute and/or communicate the Work} -- has also to \enquote{[\ldots] provide
[\ldots] the Source Code}, either directly or by \enquote{[\ldots] (indicating)
a repository where this Source will be easily and freely available
[\ldots]}\footcite[cf.][\nopage wp.\ §5]{EuplLicense2007en}. This condition
ssems to be so important for the EUPL that the license repeats its message: in
another paragraph the EUPL requires again that \enquote{if the Work is provided
as Executable Code, the Licensor provides in addition a machine-readable copy of
the Source Code of the Work along with each copy of the Work [\ldots] or
indicates, in a notice [\ldots], a repository where the Source Code is easily
and freely accessible for as long as the Licensor continues to distribute
[\ldots] the Work}\footcite[cf.][\nopage wp.\ §3]{EuplLicense2007en}. Based on
the meaning of \enquote{Work} which is defined by the EUPL as \enquote{the
Original Work and/or its Derivative Works}\footcite[cf.][\nopage wp.\
§1]{EuplLicense2007en} it must be concluded that the EUPL is a copyleft license.

But nevertheless, the European Union Public License is not a license with strong
copyleft: On the one hand, if one takes the core of the EUPL then the license
seems to protect not only the modifications of the original work against
re-closings and (re-)privatizings, but also the on-top developments because
normally you have to publish the source code in both cases. Understood in this
way, the EUPL would be a 'strong copyleft license'. But on the other hand, the
EUPL additionally contains a \enquote{Compatibility clause} stating that
\enquote{if the Licensee Distributes [\ldots] Derivative Works or copies thereof
based upon both the Original Work and another work licensed under a Compatible
Licence, this Distribution [\ldots] can be done under the terms of this
Compatible Licence}\footcite[cf.][\nopage wp.\ §5]{EuplLicense2007en} -- while
the term \enquote{Compatible Licence} is explicitly defined by a list of
compatible licenses, for example the Eclipse Public
License\footcite[cf.][\nopage wp.\ Appendix]{EuplLicense2007en}. Based on this
compatibility clause the obligation to publish the code of an on-top development
can be subverted: As first step, you could release a little more or less futile
on-top application licensed under the Eclipse Public License\footnote{Taking the
license text very seriously, it is not even necessary that this other work is an
on-top application which depends on the EUPL library because it calls functions
of EUPL library  -- what would enforce the on-top application to be a derivative
work. The license text says that \enquote{another work licensed under a
Compatible Licence} can be distributed together with \enquote{derivative works}.
So, the wording of the licenses itself is establishing a contrast between the
derivative work and the other work - what indicates that the other work has not
necessarily to be a derivative work.} which uses a library licensed under the
EUPL. As second step, you add the library as originaly work which you now may
also distribute under the EPL instead of retaining the EUPL licensing. So,
finally you obtain the same work under the Eclipse Public License which is a
weak copyleft license\footnote{$\rightarrow$ OSLiC, p.\
\pageref{sec:ProtectingPowerOfEpl}}. Hence the protection of the EUPL-1.1 is not
as comprehensive as one might assume on the base of the license text
itself\footnote{This kind of specifiying the protective power of the EUPL is
initially be presented by the FSF (\cite[cf.][wp.\ section 'European Union
Public License']{FsfEuplStatement2013a}). The EU answers that publishing such a
trick will comprise its user in the eyes of the open source community
(\cite[cf.][wp.]{FsfEuplRecomment2013}. That is undoubtely true. But
unfortunately, this argument does not close the hole in the protecting shield
put up by the EUPL}, it can at most be a weak copyleft license -- even if the
reade might get the impression that the authors of the EUPL wished to write a
strong copyleft license: the EUPL license does not protect the on-top
developments against a privatizing.


\section{The protecting power of the Gnu Public License (GPL) [tbd]}

The Gnu Public License protects \ldots
 
But the Gnu Public License does not protect \ldots


\section{The protecting power of the Lesser Gnu Public License (LGPL) [tbd]}

The Lesser Gnu Public License  protects \ldots

But the Lesser Gnu Public License  does not protect \ldots


\section{The protecting power of the MIT license}

As an approved \emph{open source license}\footcite[cf.][\nopage wp]{OSI2012b},
the MIT License\footcite[MIT has to be resolved as \enquote{Massachusetts
Institute of Technology} (cf.][\nopage wp.)]{wpMitLic2011a} protects the user
against the loss of the right to use, to modify and/or to distribute the
received copy of the source code or the binaries\footcite[cf.][\nopage wp
1ff]{OSI2012a}. Additionally, it protects the contributors and/or distributors
against warranty claims of the software users, because it contains a 'No
Warranty Clause'\footcite[cf.][\nopage wp]{MitLicense2012a}. And finally it
protects the distributed sources against a change of the license which would
close the sources, because the \enquote{permission [\ldots] to use, copy,
modify, [\ldots] distribute, [\ldots] (is granted) subject to the [\ldots]
conditions, [that] the [\ldots] copyright notice and this permission notice
shall be included in all copies or substantial portions of the
Software}\footnote{\cite[cf.][\nopage wp]{MitLicense2012a}. The argumentation
why the source code is protected, but not the binary form follows that of the
BSD licenses: By these requirements, one is not obliged to redistribute the
sourcecode of a modified (derivative) work. One is allowed to modify a received
version and to distribute the results only in binary form and to keep one's
improvements closed. But if one distribute the source code of the modifications,
the licensing is retained, simply because the MIT \enquote{[\ldots] permission
note shall be included in all copies or substantial portions of the software}.}

But the MIT License does not protect the users or the contributors and/or
distributors against patent disputes (because it does not contain any patent
clause). Additionally, it does not protect the contributors against the loss of
feedback (because it does not 'copyleft' the software). Moreover, the MIT
license does not protect the undistributed software or the distributed binaries
against re-closings -- neither in unmodified nor in modified form -- because it
allows to redistribute only the binaries without also supplying the source
code\footcite[cf.][\nopage wp]{MitLicense2012a}. Finally, the MIT license does
not protect the on-top developments against a privatizing.



\section{The protecting power of the Mozilla Public License (MPl)}
 
As an approved \emph{open source license}\footcite[cf.][\nopage wp]{OSI2012b},
the Mozilla Public License\footnote{In 2012 the Mozilla Public License 2.0
(\cite[cf.][\nopage wp.]{Mpl20MozFoundation2012a}) has been released as a result
of a longer \enquote{Revision Process}(\cite[cf.][\nopage
wp.]{Mpl11To20MozFoundation2013a}) by which the  Mozilla Public License 1.1
(\cite[cf.][\nopage wp.]{Mpl11MozFoundation2013a}) has been ousted. The OSI is
also hosting its version of the MPL-2.0 (\cite[cf.][\nopage
wp.]{Mpl20OsiLicense2013a}) and is listing it as an OSI approved license
(\cite[cf.][\nopage wp.]{OSI2012b}) while it classifies the MPL-1.1 as a
\enquote{superseded license}(\cite[cf.][\nopage wp.]{OSI2013b}). The Mozilla
Foundation itself says with respect to the difference between the two licenses,
that \enquote{the most important part of the license - the file-level copyleft -
is essentially the same in MPL 2.0 and MPL 1.1} (\cite[cf.][\nopage
wp.]{Mpl11To20MozFoundation2013a}). While reading the MPL-1.1 one could get the
impression that if one fulfills all conditions of the MPL-2.0, then one also
acts in accordance to the MPL-1.1. Thus, for the moment the OSLiC focuses on the
MPL-2.0.} protects the user against the loss of the right to use, to modify
and/or to distribute the received copy of the source code or the
binaries\footcite[cf.][\nopage wp.\ §2.1.a]{Mpl20OsiLicense2013a}. Furthermore,
based on its split and distributed patent clause\footnote{$\rightarrow$ OSLiC
pp.\ \pageref{subsec:MplPatentClause}}, the MPL-2.0 protects the users against
patent disputes\footcite[cf.][\nopage wp.\ §2.1.b, §2.3,
§5.2]{Mpl20OsiLicense2013a}. Besides this patent sections, the MPL.2.0
additionally contains a \enquote{Disclaimer of Warranty} and a
\enquote{Limitation of Liability}\footcite[cf.][\nopage wp.\ §6 \&
§7]{Mpl20OsiLicense2013a}. These three elements together protect the
contributors / distributors against patents disputes and warranty claims.
Finally, the MPL-2.0 also protects the distributed sources against a re-closing
/ re-privatizizing and the contributors against the loss of feedback: The
MPL-2.0 clearly says that -- on the one hand -- \enquote{all distribution of
Covered Software in Source Code Form, including any Modifications[\ldots] must
be under the terms of this License}\footcite[cf.][\nopage wp.\
§3.1]{Mpl20OsiLicense2013a} and that -- on the other hand -- an MPL licensed
software \enquote{[\ldots] (distributed) in Executable Form [\ldots] must also
be made available in Source Code Form [\ldots]}\footcite[cf.][\nopage wp.\
§3.2]{Mpl20OsiLicense2013a}. So, it must be concluded that the MPL is a copyleft
license.

But nevertheless, the Mozilla Public License is not a license with strong
copyleft. It does not protect the on-top developments against a privatizing: At
first, the MPL does not use the term \emph{derivative work}\footnote{
\cite[cf.][\nopage wp]{Mpl20OsiLicense2013a}. The MPL-1.1 uses the term
\emph{derivative work} only in the context of writing new \enquote{versions of
the license}, not in the context of licensing software (\cite[cf.][\nopage wp.
§6.3]{Mpl11MozFoundation2013a}).}. Instead of this, the MPL denotes the
\enquote{[\ldots] (initial) Source Code Form [\ldots] and Modifications of such
Source Code Form} by the label \enquote{Covered Software}\footcite[cf.][\nopage
wp.\ §1.4]{Mpl20OsiLicense2013a} -- while the term \enquote{Modifications}
refers to \enquote{any file in Source Code Form that results from an addition
to, deletion from, or modification of the contents of Covered Software or any
file in Source Code Form that results from an addition to, deletion from, or
modification of the contents of Covered Software}\footnote{\cite[cf.][\nopage
wp.\ §1.10]{Mpl20OsiLicense2013a}. The Mozilla Foundation itself denotes this
definition by the term \enquote{file-level copyleft} (\cite[cf.][\nopage
wp]{Mpl11To20MozFoundation2013a}).}. Second, the MPL contrasts the source code
form and its modifications with the \enquote{Larger Work} by specifying that the
larger work is \enquote{[\ldots] material, in a seperate file or files, that is
not covered software}\footcite[cf.][\nopage wp.\ §1.7]{Mpl20OsiLicense2013a}.
Finally, the MPL-2.0 states, that \enquote{you may create and distribute a
Larger Work under terms of Your choice, provided that You also comply with the
requirements of this License for the Covered Software}\footcite[cf.][\nopage
wp.\ §3.3]{Mpl20OsiLicense2013a}. Based on these specifications, it must be
concluded that an on-top development which depends on MPL licensed libraries
because they calling some of its functions, can be taken as a larger work whose
code needs not to be puglished -- provided, that the library and the on-top
development are distributed as different files\footnote{It might be discussed
whether integrating a declaration of a function, class, or method into the
on-top development by including the corresponding header files indeed means that
one is \enquote{including portions (of the Source Code Form)} into a file which
therefore has to be taken as \enquote{Modification} (\cite[cf.][\nopage wp.
§1.4]{Mpl11MozFoundation2013a}). From the viewpoint of a benevolent developer it
should be difficult to argue that the including of declaring (header) files
alone can evoke a derivative work. It is the call of the function in one's code
which establishes the dependency. But that is not the point, the MPL focuses.
The MPL aims on the textual reuse of (defining) code snippets.
Hence, one could ignore the textual integration of parts of the declaring header
files: it should not trigger that one's own work becomes a modification in the
eyes of the Mozilla Findation. But of course, one would circumvent the idea of
the MPL if one hides defining code in header files and reuses that code by one's
own compilation. This would undoubtably be an incorporation of portions and
therefore would make the incorporating file becoming a modification of the MPL
licensed initial work. }. Hence, the MPL is license with a weak copyleft effect
and does not protect the on-top developments against a privatizing.

\section{The protecting power of the Microsoft Public License (MS-PL)}

As an approved \emph{open source license}\footcite[cf.][\nopage wp]{OSI2012b},
the Microsoft Public License protects the user against the loss of the right to
use, to modify and/or to distribute the received copy of the source code or the
binaries\footcite[cf.][\nopage wp.
§2]{MsplOsiLicense2013a}. Furthermore, based on its patent
clause\footnote{$\rightarrow$ OSLiC pp.\ \pageref{subsec:MsplPatentClause}}, the
MS-PL protects the users against patent disputes\footcite[cf.][\nopage wp.
§2.B and §3.B]{MsplOsiLicense2013a}. Because of this patent clause and of its
concise \emph{disclaimer of warranty}, the MS-PL also protects the contributors
/ distributors against patents disputes and warranty
claims\footcite[cf.][\nopage wp. §2B, §3B, §3D]{MsplOsiLicense2013a}.
Finally, the Microsoft Public License protects the distributed sources
themselves - and even \enquote{portions of these sources} -- \emph{against} a
change of the license which would \emph{reset} the work \emph{as closed
software}, because first, one \enquote{[\ldots] must retain all copyright,
patent, trademark, and attribution notices that are present in the
software}\footcite[cf.][\nopage wp. §3C]{MsplOsiLicense2013a}, and because
second, one must also incorporate \enquote{a complete copy of this license} into
one's own distribution premised one distributes the source
code\footcite[cf.][\nopage wp. §3D]{MsplOsiLicense2013a}.

But the Microsoft Public License does not protect the contributors against the
loss of feedback because it does not 'copyleft' the software: The license does
not contain any sentence which requires that one has to publish the sources,
too\footnote{There seems to be some misunderstandings on the internet: The
English wikipedia specifies the MS-PL as a permissive license and the MS-RL as a
license with copyleft effect (\cite[cf.][\nopage wp.]{wpMsSharedSources2013a}).
The German wikipedia says that the MS-PL is a license with a \enquote{schwachen
[weak] copyleft} (\cite[cf.][\nopage wp.]{wpMspl2013a}). And it says also that
the \enquote{Microsoft Reciprocal License} (MS-RL) is a license with weak
copyleft, too (\cite[cf.][\nopage wp.]{wpMsrl2013a}). But for the very
thoroughly working \enquote{ifross license center}, the MS-RL is a license with
restricted (weak) copyleft, while the MS-PL is a permissive license with some
selectable options (\cite[cf.][\nopage wp.]{ifross2011a}). Based on the license
text itself and these other readings, we decided to take the MS-PL as a
permissive license in accordance to the English wikipedia page and the ifross
page.}. In the same spirit, the MS-PL does not protect the undistributed
software or the distributed binaries against re-closings -- neither in
unmodified nor in modified form -- because the MS-PL License allows to
(re)distribute the binaries without also supplying the sources -- even if the
binaries rest upon sources modified by the distributor. Finally, also the MS-PL
does not protect the on-top developments against a privatizing.


\section{The protecting power of the Postgres License (PgL)}

As an approved \emph{open source license}\footcite[cf.][\nopage wp]{OSI2012b},
the PostgreSQL License protects the user against the loss of the right to use,
to modify and/or to distribute the received copy of the source code or the
binaries\footcite[cf.][\nopage wp.]{PglOsiLicense2013a}.
Because of its \emph{disclaimer of warranty}, the PgL also protects the
contributors / distributors against warranty claims\footcite[cf.][\nopage
wp.]{PglOsiLicense2013a}. Finally, the PgL protects the distributed sources
themselves \emph{against} a change of the license which would \emph{reset} the
work \emph{as closed software}, because the \enquote{copyright notice} and the
whole license must \enquote{[\ldots] appear in all copies}\footcite[cf.][\nopage
wp.]{PglOsiLicense2013a}.

But the PostgreSQL License does not protect the contributors against the loss of
feedback because it does not 'copyleft' the software: The license does not
contain any sentence which requires that one has to publish the sources, too. 
In the same spirit, the PgL does not protect the undistributed software or the
distributed binaries against re-closings -- neither in unmodified nor in
modified form -- because the PgL allows to (re)distribute the binaries without
also supplying the sources -- even if the binaries rest upon sources modified by
the distributor. Finally, the PgL does not protect the on-top developments
against a privatizing.


\section{The protecting power of the PHP License}

As an approved \emph{open source license}\footcite[cf.][\nopage wp]{OSI2012b},
the PHP-3.0 License protects the user against the loss of the right to use, to
modify and/or to distribute the received copy of the source code or the
binaries\footcite[cf.][\nopage wp.]{Php30OsiLicense2013a}. Because of its
\emph{disclaimer of warranty}, the PHP license also protects the contributors /
distributors against warranty claims\footcite[cf.][\nopage
wp.]{Php30OsiLicense2013a}. Finally, the PHP license protects the distributed
sources themselves \emph{against} a change of the license which would
\emph{reset} the work \emph{as closed software}, because
\enquote{redistributions of source code must retain the [\ldots] copyright
notice, this list of conditions and the [\ldots]
disclaimer}\footcite[cf.][\nopage wp.]{Php30OsiLicense2013a}.

But the PHP-3.0 License does not protect the contributors against the loss of
feedback because it does not 'copyleft' the software: The license does not
contain any sentence which requires that one has to publish the sources, too. 
In the same spirit, the PHP license does not protect the undistributed software
or the distributed binaries against re-closings -- neither in unmodified nor in
modified form -- because the PHP license allows to (re)distribute the binaries
without also supplying the sources -- even if the binaries rest upon sources
modified by the distributor.
  
\section{Summary}

All these specifications cannot only be summarized by a
table\footnote{$\rightarrow$ OSLiC, p. \pageref{tab:powerOfLicenses}}, but also
by a mindmap:

\begin{tikzpicture}
\label{OSCLICMM}
\footnotesize

% (1.A) list of all licenses and their release numbers Level 5/6
\node[rectangle,draw,text width=1.4cm] (l0100) at (9,4)
{ \textit{BSD License} };
\node[text width=1.4cm] (l0101) at (8.25,3)
{ \scriptsize{3-Clauses} };
\node[text width=1.4cm] (l0102) at (10,3)
{ \scriptsize{2-Clauses} };
  
\node[rectangle,draw,text width=1.4cm] (l0200) at (10.2,5)
{ \textit{MIT License} }; 
  
\node[rectangle,draw,text width=1.4cm] (l0300) at (12,5.5)
{ \textit{\textbf{Ap}ache \textbf{L}icense}};
\node[text width=0.4cm] (l0301) at (12,4.5) {\scriptsize{2.0}};

\node[rectangle,draw,text width=1.4cm] (l0400) at (13,6.8)
{ \scriptsize{\textit{\textbf{M}icro\textbf{s}oft} \textbf{P}ublic \textbf{L}icense} };
  
\node[rectangle,draw,text width=1.4cm] (l0500) at (13,8)
{\textit{\textbf{P}ost\textbf{g}res \textbf{L}icense}};
  
\node[rectangle,draw,text width=1.4cm] (l0600) at (13,9)
{\textit{\textbf{PHP} License}};
\node[text width=0.4cm] (l0601) at (14.5,9){\scriptsize{3.0}};
  

\node[rectangle,draw,text width=1.4cm] (l0800) at (13,10.7)
{ \textit{\textbf{M}ozilla \textbf{P}ublic \textbf{L}icense}};
\node[text width=0.4cm] (l0801) at (14.5,10.2){\scriptsize{1.1}};
\node[text width=0.4cm] (l0802) at (14.5,11.2){\scriptsize{2.0}};

\node[rectangle,draw,text width=1.4cm] (l0900) at (13,12.25)
{\textit{\textbf{E}clipse \textbf{P}ublic \textbf{L}icense}};
\node[text width=0.4cm] (l0901) at (14.5,12.25) {\scriptsize{1.0}};
 
\node[rectangle,draw,text width=1.5cm] (l1000) at (13,13.8)
{\textit{\textbf{E}uropean \textbf{P}ublic \textbf{L}icense}}; 
\node[text width=0.4cm] (l1001) at (14.5,13.3){\scriptsize{1.1}};
\node[text width=0.4cm,style=dotted] (l1002) at (14.5,14.3){\scriptsize{\textit{1.2}}};
  
\node[rectangle,draw,text width=1.4cm] (l1100) at (13,15.5)
{\textit{\textbf{L}esser \textbf{G}NU \textbf{P}ublic \textbf{L}icense}};

\node[text width=0.4cm] (l1101) at (14.5,15){\scriptsize{2.1}};
\node[text width=0.4cm] (l1102) at (14.5,16){\scriptsize{3.0} };

\node[rectangle,draw,text width=1.4cm] (l1200) at (13,17.5)
{\textit{\textbf{G}NU \textbf{P}ublic \textbf{L}icense}};

\node[text width=0.4cm] (l1201) at (14.5,17){\scriptsize{2.1}};
\node[text width=0.4cm] (l1202) at (14.5,18){\scriptsize{3.0} };

\node[rectangle,draw,text width=1.4cm] (l1300) at (13,19.5)
{ \textit{\textbf{A}ffero \textbf{G}NU \textbf{P}ublic \textbf{L}icense}};
\node[text width=0.4cm] (l1302) at (14.5,19.5){\scriptsize{3.0}};

% 2. the clustering concepts of licenses (level 4)
\node[rectangle,draw,text width=2.3cm] (n0100) at (10,8)
 { \textit{protecting the user, the con\-tri\-butor \& the initial code}\\
   \tiny{Permissive Licenses}      
 };

\node[rectangle,draw,text width=2.3cm] (n0200) at (10,12.5)
{ \textit{protecting the user, the con\-tri\-butor, the
  initial code, \& all di\-rect de\-ri\-va\-tions}\\
  \tiny{Weak Copyleft}        
};

\node[rectangle,draw,text width=2.3cm] (n0300) at (10,16.5)
{ \textit{protecting the user, the con\-tri\-bu\-tor, the 
  initial code, all di\-rect de\-ri\-va\-tions \& the 
  (in\-di\-rect\-ly de\-ri\-ved) on-top-deve\-lop\-ments}\\ 
  \tiny{Strong Copyleft}    
 };

% 3. the threats (level 3)
\node[ellipse,draw,text width=1.6cm] (c110000) at (4.5,0)
{ \textbf{\textit{Patent Disputes}}};

\node[ellipse,draw,text width=1.6cm] (c120000) at (4.5,2)
{ \textbf{\textit{Loss of Rights}} };

\node[ellipse,draw,text width=1.6cm] (c210000) at (4.5,4)
{ \textbf{\textit{Warranty Claims}} };
 
\node[ellipse,draw,text width=1.6cm] (c220000) at (4.5,6)
{ \textbf{\textit{Loss of Feeback}}};

\node[ellipse,draw,text width=0.6cm] (c311000) at (6.2,8)
{ \tiny{\textit{\textbf{reclos\-ings}}}};

\node[ellipse,,draw,text width=0.6cm] (c321000) at (6.2,10)
{ \tiny{\textit{\textbf{reclos\-ings}}} };

\node[ellipse,,draw,text width=0.6cm] (c331000) at (6.2,12)
{ \tiny{\textit{\textbf{reclos\-ings}}} };

\node[ellipse,,draw,text width=0.6cm] (c341000) at (6.2,14)
{ \tiny{\textit{\textbf{reclos\-ings}}} };

\node[ellipse,,draw,text width=0.6cm] (c351000) at (6.8,16.2)
{ \tiny{\textit{\textbf{reclos\-ings}}} };

\node[ellipse,,draw,text width=0.7cm] (c361000) at (7.5,17.5)
{ \tiny{\textit{\textbf{privati\-zings}}} };

\node[ellipse,,draw,text width=1.6cm] (c411000) at (6.5,19)
{ \textit{\textbf{clos\-ings}} };


% 4. the subtypes of protected entities (level 2)
\node[ellipse,draw,text width=1.5cm] (c310000) at (3,8)
 { \scriptsize{un\-modified} \textbf{Sources}};

\node[ellipse,draw,text width=1.5cm] (c320000) at (3.25,10)
 { \scriptsize{un\-modified} \textbf{Binaries}};

\node[ellipse,draw,text width=1.2cm] (c330000) at (3.5,12)
 { \scriptsize{modified} \textbf{Sources}};

\node[ellipse,draw,text width=1.4cm] (c340000) at (3.25,14)
 { \scriptsize{modified} \textbf{Binaries}};

\node[ellipse,draw,text width=2cm] (c350000) at (3.6,16)
 { \tiny{\textbf{part of} On-Top-Developments}};

\node[ellipse,draw,text width=2.9cm] (c360000) at (3.4,17.5)
 { \tiny{\textbf{On-Top-Developments}}};


% 5. the protected entities (level 1)
\node[ellipse,draw,text width=1cm] (c100000) at (1,1)
 { \textbf{Users} };

\node[ellipse,draw,text width=0.8cm] (c200000) at (1,5)
 { \textbf{Con\-tribu\-tors}};

\node[ellipse,draw,text width=0.8cm] (c300000) at (1,12)
 { distri\-buted \textbf{Soft\-ware}};
 
\node[ellipse,draw,text width=2.2cm] (c400000) at (1,19)
 { un\-distri\-buted \textbf{Soft\-ware}}; 

% 6. main node (leve 0)
\node[ellipse,draw,text width=1.3cm] (c000000) at (0,8)
{ \textbf{open source license}};

% a linking Licenses to their release numbers (Linking level 5 to 6)
\foreach \father/\daughter in {
  l0100/l0101/,
  l0100/l0102/,
  l0300/l0301/,
  l0600/l0601/,
  l0800/l0801/,
  l0800/l0802/,
  l0900/l0901/,
  l1000/l1001/,
  l1000/l1002/,
  l1100/l1101/,
  l1100/l1102/,
  l1200/l1201/,
  l1200/l1202/,
  l1300/l1302/
  }
  \draw[dashed] (\father) to  (\daughter) ;

% b) linking Licenses to license concepts (Linking level 5 to 4)
\foreach \father/\daughter/\outangle/\inangle in {
  n0100/l0100/270/150,       
  n0100/l0200/280/155,
  n0100/l0300/290/160,
  n0100/l0400/300/165,
  n0100/l0500/310/150,
  n0100/l0600/340/160,
  n0200/l0800/300/160,
  n0200/l0900/340/170,
  n0200/l1000/20/190,
  n0200/l1100/60/200,
  n0300/l1200/40/180,
  n0300/l1300/80/180 
  }
  %\draw[dashed] (\father) to [out=\outangle,in=\inangle] (\daughter) ;
  \draw[dashed] (\father) to  (\daughter) ;

% c) linking license concepts to the threats against they protect
% c.1) strong copyleft licenses
\foreach \father/\daughter/\outangle/\inangle in {
  c361000/n0300/0/180,
  c351000/n0300/0/180,
  c341000/n0300/45/190,
  c331000/n0300/50/200,
  c321000/n0300/55/210,
  c311000/n0300/60/220,
  c220000/n0300/25/225,
  c210000/n0300/25/230,
  c120000/n0300/25/235
  }
  \draw[<-,color=blue] (\father) to [out=\outangle,in=\inangle] (\daughter) ;
% c.2) weak copyleft licenses
\foreach \father/\daughter/\outangle/\inangle in {
  c341000/n0200/330/170,
  c331000/n0200/0/180,
  c321000/n0200/0/180,
  c311000/n0200/20/190,
  c220000/n0200/15/220,
  c210000/n0200/15/230,
  c120000/n0200/15/235
  }
  \draw[<-,color=cyan] (\father) to [out=\outangle,in=\inangle] (\daughter) ;
% c.3) permissive licenses
\foreach \father/\daughter/\outangle/\inangle in {
  c331000/n0100/355/150,
  c311000/n0100/0/180,
  c210000/n0100/5/210,
  c120000/n0100/10/230
  }
  \draw[<-,color=red] (\father) to [out=\outangle,in=\inangle] (\daughter) ;
%c.4 agpl license
\foreach \father/\daughter/\outangle/\inangle in {
  c411000/l1300/0/180    
}
  \draw[<-,color=green] (\father) to [out=\outangle,in=\inangle] (\daughter) ;


%d linking protected entities, their subtypes and the the relations
\foreach \father/\daughter/\edgetext/\outangle/\inangle in {
  c000000/c100000/protecting/260/120,
  c100000/c110000/against/360/180,
  c100000/c120000/against/360/180,
  c000000/c200000/protecting/270/180,
  c200000/c110000/against/340/150,
  c200000/c210000/against/0/180,
  c200000/c220000/against/0/180,
  c000000/c300000/protecting/90/230,
  c300000/c310000/as/300/180,
  c300000/c320000/as/330/180,
  c300000/c330000/as/0/180,
  c300000/c340000/as/30/180,
  c300000/c350000/as/60/180,
  c300000/c360000/as/70/180,
  c000000/c400000/protecting/100/240,
  c400000/c411000/against/0/180        
}
  \draw[->,dotted,
    decoration={text along path,
              text align={center},
              text={|\itshape|\edgetext}},
              postaction={decorate},] (\father) to [out=\outangle,in=\inangle] (\daughter) ;

\foreach \father/\daughter/\edgetext/\outangle/\inangle in {
  c310000/c311000/against/0/180,
  c320000/c321000/against/0/180,
  c330000/c331000/against/0/180,
  c340000/c341000/against/00/180,
  c350000/c351000/against/0/180,
  c360000/c361000/against/0/180      
}
  \draw[->,dotted,
    decoration={text along path,
              text align={center},
              text={|\itshape \tiny|\edgetext}},
              postaction={decorate},] (\father) to [out=\outangle,in=\inangle] (\daughter) ;

%f linking the patent clauses
\foreach \father/\daughter/\outangle/\inangle in {
  c110000/l1302/0/305,
  c110000/l1202/0/303,
  c110000/l1201/0/301,
  c110000/l1102/0/299,
  c110000/l1101/0/297,
  c110000/l1002/0/295,
  c110000/l0901/0/290,
  c110000/l0802/0/285,
  c110000/l0400/0/275,
  c110000/l0301/0/270   
}
  \draw[<-,color=gray] (\father) to [out=\outangle,in=\inangle] (\daughter) ;

\end{tikzpicture}

Finally, one could generate new groups of open source license, new classes, like
'user protecting licenses'\footnote{all of them because all of them have to
fulfill the OSD}, 'patent disputes fending licenses', an so on.
% TODO nach Fertigstellung eigene Taxonomie entwickeln

However, it must kept in mind that all of these grouping viewpoints do not
permit the conclusion that all members of a group can be respected by fulfilling
the same requirements. This would only be possible if the grouping criteria
would directly refer to the fulfilling tasks. Indeed, nearly all open source
licenses do differ with respect to these criteria, and even if the differences
are very small, they can't be neglected\footnote{Pars pro toto: Both, the BSD
license and the Apache license require that you provide an indication to the
developers of the application. But in case of the BSD license you have to
publish the copyright notice / line, while in case of the Apache license you
have exactly to present the content of the notice file distributed together with
the application.}. So: reflecting on possible classes of open source licenses is
a good method to become familiar with the area of open source licenses. But it
is not a method to determine, what one needs to be done to obtain the right to
use the software. For that purpose every license must be considered
individually.



%\bibliography{../../../bibfiles/oscResourcesEn}
