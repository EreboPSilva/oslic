% Telekom osCompendium 'for beeing included' snippet template
%
% (c) Karsten Reincke, Deutsche Telekom AG, Darmstadt 2011
%
% This LaTeX-File is licensed under the Creative Commons Attribution-ShareAlike
% 3.0 Germany License (http://creativecommons.org/licenses/by-sa/3.0/de/): Feel
% free 'to share (to copy, distribute and transmit)' or 'to remix (to adapt)'
% it, if you '... distribute the resulting work under the same or similar
% license to this one' and if you respect how 'you must attribute the work in
% the manner specified by the author ...':
%
% In an internet based reuse please link the reused parts to www.telekom.com and
% mention the original authors and Deutsche Telekom AG in a suitable manner. In
% a paper-like reuse please insert a short hint to www.telekom.com and to the
% original authors and Deutsche Telekom AG into your preface. For normal
% quotations please use the scientific standard to cite.
%
% [ Framework derived from 'mind your Scholar Research Framework' 
%   mycsrf (c) K. Reincke 2012 CC BY 3.0  http://mycsrf.fodina.de/ ]
%


%% use all entries of the bibliography
%\nocite{*}

\section{BSD Licensed Software \ldots}

\subsection{The BSD specific mini finder}

As an approved Open Source license, the BSD license exists in two
versions\footcite[Following the Open Source Initiative, initially a not approved
BSD license contained a fourth clause also known as advertising clause which
\enquote{(\ldots) officially was rescinded by the Director of the Office of
Technology Licensing of the University of California on July 22nd, 1999}.
 Cf.][\nopage wp. Because of the cancellation you can simply act according the
 \textit{BSD 3-Clause license} if you have to fulfill the eldest BSD
 license]{BsdLicense3Clause}. The latest, most modern release is the \textit{BSD
 2-Clause license}\footcite[cf.][\nopage wp]{BsdLicense2Clause}, the elder
 release is the \textit{BSD 3-Clause license}\footcite[cf.][\nopage
 wp]{BsdLicense3Clause}. Nevertheless, the little differences between the
 two versions have to be respected.

All BSD Open Source Licenses explicitly focuses 'only' on the (re-)distribution
\textit{Open Source Use Cases}, which we have specified by our token
\textit{4others}. Conditions for the use cases specified by the token
\textit{4yourself} can be derived\footnote{For details of the \textit{Open
Source Use Cases tokens} see p. \pageref{OsucTokens}. For Details of the
\textit{Open Source Use Cases} based on these token see p.
\pageref{OsucDefinitionTree} }. Additionally the BSD license considers the form
of the distribution, e.g. whether the work is distributed as a (set of) source
code file(s) or as a (set of) the binary file(s). Use the following tree to find
the BSD license fulfilling to-do lists.

\begin{center}
\begin{footnotesize}
\pstree[levelsep=*1,treesep=0.2]{\Toval{BSD}}{
  \pstree[levelsep=*0.2,treesep=1]{
    \Toval{2 Clause License}
  }{
  \pstree{
    \Tr{\Ovalbox{\shortstack{recipient: \textit{4yourself}\\
    \textbf{\textit{used by yourself}}}}} 
  }{
    \Tr{\doublebox{\shortstack{\tiny{\textbf{BSD-1}:}\\
    \tiny{\textit{using the}}\\\tiny{\textit{software}}\\
    \tiny{\textit{only for}}\\\tiny{\textit{yourself}} }}} 
  }
  \pstree[levelsep=*0.2,treesep=0.2]{
    \Tr{\Ovalbox{\shortstack{recipient: \textit{4others}\\
      \textbf{\textit{distributed to 3rd. parties}}}}} 
  }{ 
    \pstree[levelsep=*0.2,treesep=0.2]{
      \Tr{\Ovalbox{\shortstack{state:\\\textbf{\textit{unmodified}}}}}
    }{        
      \pstree{
        \Tr{\Ovalbox{\shortstack{form:\\\textbf{\textit{source}}}}}
      }{        
        \Tr{\doublebox{\shortstack{\tiny{\textbf{BSD-2}:}\\
        \tiny{\textit{distributing}}\\\tiny{\textit{unmodified}}\\
        \tiny{\textit{software as}}\\\tiny{\textit{source code}} }}} 
      }
      \pstree{
        \Tr{\Ovalbox{\shortstack{form:\\\textbf{\textit{binary}}}}}
      }{        
        \Tr{\doublebox{\shortstack{\tiny{\textbf{BSD-3}:}\\
        \tiny{\textit{distributing}}\\\tiny{\textit{unmodified}}\\
        \tiny{\textit{software as}}\\\tiny{\textit{binary pkg}} }}} 
      }

    }

    \pstree[levelsep=*0.2,treesep=0.2]{
      \Tr{\Ovalbox{\shortstack{state:\\\textbf{\textit{modified}}}}}
    }{ 
      \pstree{
        \Tr{\Ovalbox{\shortstack{type:\\\textbf{\textit{proapse}}}}}
      }{
           
        \pstree{
          \Tr{\Ovalbox{\shortstack{form:\\\textbf{\textit{source}}}}}
        }{        
          \Tr{\doublebox{\shortstack{\tiny{\textbf{BSD-4}:}\\
          \tiny{\textit{distributing}}\\\tiny{\textit{a modified}}\\
          \tiny{\textit{program as}}\\\tiny{\textit{source code}} }}} 
        }
        \pstree{
          \Tr{\Ovalbox{\shortstack{form:\\\textbf{\textit{binary}}}}}
        }{        
          \Tr{\doublebox{\shortstack{\tiny{\textbf{BSD-5}:}\\
          \tiny{\textit{distributing}}\\\tiny{\textit{a modified}}\\
          \tiny{\textit{program as}}\\\tiny{\textit{binary pkg}} }}} 
        }
      }
      \pstree{
        \Tr{\Ovalbox{\shortstack{type:\\\textbf{\textit{snimoli}}}}}
      }{
        \pstree{
          \Tr{\Ovalbox{\shortstack{context:\\\textbf{\textit{independent}}}}}
        }{        
           
          \pstree{
            \Tr{\Ovalbox{\shortstack{form:\\\textbf{\textit{source}}}}}
          }{        
            \Tr{\doublebox{\shortstack{\tiny{\textbf{BSD-6}:}\\
            \tiny{\textit{distributing}}\\\tiny{\textit{a modified}}\\
            \tiny{\textit{library as}}\\\tiny{\textit{independent}}\\
            \tiny{\textit{source pkg}} }}} 
          }
          \pstree{
            \Tr{\Ovalbox{\shortstack{form:\\\textbf{\textit{binary}}}}}
          }{        
            \Tr{\doublebox{\shortstack{\tiny{\textbf{BSD-7}:}\\
            \tiny{\textit{distributing}}\\\tiny{\textit{a modified}}\\
            \tiny{\textit{library as}}\\\tiny{\textit{independent}}\\
            \tiny{\textit{binary pkg}} }}} 
          }
        }
        \pstree{
          \Tr{\Ovalbox{\shortstack{context:\\\textbf{\textit{embedded}}}}}
        }{        
           
           \pstree{
            \Tr{\Ovalbox{\shortstack{form:\\\textbf{\textit{source}}}}}
          }{        
            \Tr{\doublebox{\shortstack{\tiny{\textbf{BSD-8}:}\\
            \tiny{\textit{distributing}}\\\tiny{\textit{a modified}}\\
            \tiny{\textit{library as}}\\\tiny{\textit{embedded}}\\
            \tiny{\textit{source pkg}} }}} 
          }
          \pstree{
            \Tr{\Ovalbox{\shortstack{form:\\\textbf{\textit{binary}}}}}
          }{        
            \Tr{\doublebox{\shortstack{\tiny{\textbf{BSD-9}:}\\
            \tiny{\textit{distributing}}\\\tiny{\textit{a modified}}\\
            \tiny{\textit{library as}}\\\tiny{\textit{embedded}}\\
            \tiny{\textit{binary pkg}} }}} 
          }
        }


      }
 
    }
   }   

  }
  \pstree[levelsep=*0.2,treesep=0.2]{
    \Toval{3 Clause License}
  }{
    \Ttri{}
  }
}
\end{footnotesize}
\end{center}

\subsection{Software licensed by the \emph{BSD 2-Clause License}}

\subsubsection{BSD-1: using the software only for yourself}
\label{OSUC-01-BSD} 
\label{OSUC-03-BSD} 
\label{OSUC-06-BSD}
\label{OSUC-09-BSD}
  
\begin{description}
\item[means] that you are going to use a received BSD software only for yourself
and that you do not handover it to any 3rd. party in any sense.
\item[covers] OSUC-01, OSUC-03, OSUC-06, and OSUC-09\footnote{For details see pp.
  \pageref{OSUC-01-DEF} - \pageref{OSUC-09-DEF}}
\item[requires] the tasks in order to fulfill the conditions
    of the BSD license:
  \begin{itemize}
    \item You are allowed to use any kind of BSD software in any sense and in
    any context without any obligations if you do not handover the software to
    3rd parties.
  \end{itemize}
\end{description}


\subsubsection{BSD-2: Passing the unmodified software as a source code package}
\label{OSUC-02-BSD} \label{OSUC-05-BSD} \label{OSUC-07-BSD} 

\begin{description}
\item[means] that you are going distribute an unmodified version of the received
BSD software to 3rd. parties in form of a set of source code files or an
integrated source code package\footnote{In this case it doesn't matter whether
you  distribute a program, an application, a server, a snippet, a module, a
library, or a plugin as an independent or an embedded unit} 

\item[covers] OSUC-02, OSUC-05, OSUC-07\footnote{For details see pp.
\pageref{OSUC-02-DEF} - \pageref{OSUC-07-DEF}}

\item[requires] the tasks in order to fulfill the license conditions
\begin{itemize}
  \item \textbf{[mandatory:]} Ensure, that the licensing elements - e.g.
  the BSD license text, the specific copyright notice of the original author(s),
  and the BSD disclaimer - are retained in your package in the form you have got
  them.
  \item \textbf{[voluntary:]} Let the documentation of your distribution
  and/or your additional material also contain the original copyright notice, the
  BSD conditions, and the BSD disclaimer.
\end{itemize}
\end{description}

\subsubsection{BSD-3: Passing the unmodified software as a binary package}

\begin{description}
\item[means] that you are going distribute an unmodified version of the BSD
received software to 3rd. parties in form of a set of binary files or an
integrated bi\-na\-ry package\footnote{In this case it doesn't matter whether
you distribute a program, an application, a server, a snippet, a module, a library,
or a plugin as an independent or an embedded unit}
\item[covers] OSUC-02, OSUC-05, OSUC-07\footnote{For details see pp.
\pageref{OSUC-02-DEF} - \pageref{OSUC-07-DEF}}
\item[requires] the tasks in order to fulfill the license conditions
\begin{itemize}
  \item  \textbf{[mandatory:]} Ensure, that your distribution contains the
  original copyright notice, the BSD license, and the BSD disclaimer in the form
  you have got them. If you compile the binary file on the base of the source
  code package and if this compilation does not also generate and integrate the
  licensing files, then create the copyright notice, the BSD conditions, and the
  BSD disclaimer according to the form of the source code package and insert
  these files into your distribution manually.
  \item  \textbf{[mandatory:]} Ensure, that the documentation of your
  distribution and/or your additional material also contain the author specific
  copyright notice, the BSD conditions, and the BSD disclaimer.
\end{itemize}
\end{description}

\begin{itshape}
\emph{\textbf{General remark for all binary distribitions}:} 
\label{MobileDeviceHint} Even if you are distributing a BSD bi\-na\-ry package
on a medium, which doesn't allow the user, to see package files directly - some
mobile devices don't give their users the full access to all stored elements -
you have to fulfill the mandatory requirements. So, sometimes, it is necessary,
to let simply the BSD license and the copyright notice become an unvisible part
of the binary pakage and to deploy the files together with the package. The
point is: you have fulfilled the license.
\end{itshape}

\subsubsection{BSD-4: Passing a modified program as a source code package}
\label{OSUC-04-BSD}

\begin{description}
\item[means] that you are going distribute a modified version of the received
BSD program, application, or server (proapse) to 3rd. parties in form of a set
of source code files or an integrated source code package.
\item[covers] OSUC-04\footnote{For details see pp. \pageref{OSUC-04-DEF}}
\item[requires] the tasks in order to fulfill the license conditions
\begin{itemize}
  \item \textbf{[mandatory:]} Ensure, that the licensing elements - e.g.
  the BSD license text, the specific copyright notice of the original author(s),
  and the BSD disclaimer - are retained in your package in the form you have got
  them. 
  \item \textbf{[voluntary:]} Let the documentation of your distribution
  and/or your additional material also contain the original copyright notice, the
  BSD conditions, and the BSD disclaimer.
  
  \item \textbf{[voluntary:]} It is a good practice of the Open Source
  community, to let the copyright notice which is shown by the running program
  also state that the program is licensed under the BSD license. Because you are
  already modifying the program, you can also add such a hint, if the presented
  original copyright notice lacks such a statement.
\end{itemize}
\end{description}

\subsubsection{BSD-5: Passing a modified program as a binary package}

\begin{description}
\item[means] that you are going distribute a modified version of the received
BSD pro\-gram, application, or server (proapse) to 3rd. parties in form of a set
of binary files or an integrated binary package.
\item[covers] OSUC-04\footnote{For details see pp. \pageref{OSUC-04-DEF}}
\item[requires] the tasks in order to fulfill the license conditions
\begin{itemize}

  \item  \textbf{[mandatory:]} Ensure, that your distribution contains the
  original copyright notice, the BSD license, and the BSD disclaimer in the form
  you have got them. If you compile the binary file on the base of the source
  code package and if this compilation does not also generate and integrate the
  licensing files, then create the copyright notice, the BSD conditions, and the
  BSD disclaimer according to the form of the source code package and insert
  these files into your distribution manually\footnote{see also our 'Mobile
  Device Hint' on p. \pageref{MobileDeviceHint}}.

  \item  \textbf{[mandatory:]} Ensure, that the documentation of your
  distribution and/or your additional material also contain the author specific
  copyright notice, the BSD conditions, and the BSD disclaimer.
  
  \item \textbf{[voluntary:]} It is a good practice of the Open Source
  community, to let the copyright notice which is shown by the running program
  also state that the program is licensed under the BSD license. Because you are
  already modifying the program, you can also add such a hint, if the presented
  original copyright notice lacks such a statement.
\end{itemize}
\end{description}

\subsubsection{BSD-6: Passing a modified library as independent source code
package}
\label{OSUC-08-BSD}
\begin{description}
\item[means] that you are going distribute a modified version of the received
BSD code snippet, module, library, or plugin (snimoli) to 3rd. parties in form
of a set of source code files or an integrated source code package, but without
embedding it into another larger software unit.
\item[covers] OSUC-08\footnote{For details see pp. \pageref{OSUC-08-DEF}}
\item[requires] the tasks in order to fulfill the license conditions
\begin{itemize}
  \item \textbf{[mandatory:]} Ensure, that the licensing elements - e.g.
  the BSD license text, the specific copyright notice of the original author(s),
  and the BSD disclaimer - are retained in your package in the form you have got
  them.
  \item \textbf{[voluntary:]} Let the documentation of your distribution
  and/or your additional material also contain the original copyright notice, the
  BSD conditions, and the BSD disclaimer.
\end{itemize}
\end{description}


\subsubsection{BSD-7: Passing a modified library as independent binary
package}

\begin{description}
\item[means] that you are going distribute a modified version of the received
BSD code snippet, module, library, or plugin (snimoli) to 3rd. parties in form
of a set of binary files or an integrated binary package, but without embedding
it into another larger software unit.
\item[covers] OSUC-08\footnote{For details see pp. \pageref{OSUC-08-DEF}}
\item[requires] the tasks in order to fulfill the license conditions
\begin{itemize}
   \item  \textbf{[mandatory:]} Ensure, that your distribution contains the
  original copyright notice, the BSD license, and the BSD disclaimer in the form
  you have got them. If you compile the binary file on the base of the source
  code package and if this compilation does not also generate and integrate the
  licensing files, then create the copyright notice, the BSD conditions, and the
  BSD disclaimer according to the form of the source code package and insert
  these files into your distribution manually\footnote{see also our 'Mobile
  Device Hint' on p. \pageref{MobileDeviceHint}}.
  \item  \textbf{[mandatory:]} Ensure, that the documentation of your
  distribution and/or your additional material also contain the author specific
  copyright notice, the BSD conditions, and the BSD disclaimer.
\end{itemize}
\end{description}

\subsubsection{BSD-8: Passing a modified library as an embedded source code
package}
\label{OSUC-10-BSD}
\begin{description}
\item[means] that you are going distribute a modified version of the received
BSD code snippet, module, library, or plugin (snimoli) to 3rd. parties in form
of a set of source code files or an integrated source code package together with
another larger software unit, which contains this code snippet, module, library,
or plugin as an embedded component.
\item[covers] OSUC-10\footnote{For details see pp. \pageref{OSUC-10-DEF}}
\item[requires] the tasks in order to fulfill the license conditions
\begin{itemize}
  \item \textbf{[mandatory:]} Ensure, that the licensing elements - e.g.
  the BSD license text, the specific copyright notice of the original author(s),
  and the BSD disclaimer - are retained in your package in the form you have got
  them.
  \item \textbf{[voluntary:]} Let the documentation of your distribution
  and/or your additional material also contain the original copyright notice, the
  BSD conditions, and the BSD disclaimer.
 \item \textbf{[voluntary:]} It is a good practice of the Open Source
  community, to let the copyright notice which is shown by the running program
  also state that it contains components licensed under the BSD license. Because
  you are embedding this snimoli into a larger software unit, you are
  developing this larger unit. Hence, you can also expand the copyright notice
  of this larger unit by such a hint to its BSD components.
\end{itemize}
\end{description}


\subsubsection{BSD-9: Passing a modified library as an embedded binary
package}

\begin{description}
\item[means] that you are going distribute a modified version of the received
BSD code snippet, module, library, or plugin to 3rd. parties in form of a set of
binary files or an integrated binary package together with another larger
software unit, which contains this code snippet, module, library, or plugin as
an embedded component.
\item[covers] OSUC-10\footnote{For details see pp. \pageref{OSUC-10-DEF}}
\item[requires] the tasks in order to fulfill the license conditions
\begin{itemize}
  \item  \textbf{[mandatory:]} Ensure, that your distribution contains the
  original copyright notice, the BSD license, and the BSD disclaimer in the form
  you have got them. If you compile the binary file on the base of the source
  code package and if this compilation does not also generate and integrate the
  licensing files, then create the copyright notice, the BSD conditions, and the
  BSD disclaimer according to the form of the source code package and insert
  these files into your distribution manually\footnote{see also our 'Mobile
  Device Hint' on p. \pageref{MobileDeviceHint}}.
  \item  \textbf{[mandatory:]} Ensure, that the documentation of your
  distribution and/or your additional material also contain the author specific
  copyright notice, the BSD conditions, and the BSD disclaimer.
 \item \textbf{[voluntary:]} It is a good practice of the Open Source
  community, to let the copyright notice which is shown by the running program
  also state that it contains components licensed under the BSD license. Because
  you are embedding this snimoli into a larger software unit, you are
  developing this larger unit. Hence, you can also expand the copyright notice
  of this larger unit by such a hint to its BSD components.
\end{itemize}
\end{description}

\subsection{Software licensed by the \emph{BSD 3-Clause License}}

Compared with the \textit{BSD 2-Clause license}, the \textit{BSD 3-Clause
license} only contains one additional 3rd. clause, the rest is the same. So, for
acting according the \textit{BSD 3-Clause license}, do that, what you would have
to do for fulfilling the 2 clause license. And additionally do not use the name
of licensing organization or the names of the licensing distributors to promote
your own work.

\subsection{Discussions and Explanations}

The \textit{BSD 2-Clause license} has a simple textual structure: In the
beginning, it generally \enquote{(permits) [the] redistribution and [the] use in
source and binary forms, with or without modification, [\ldots]}, if one
fulfills the two rules of the license\footcite[cf.][\nopage
wp]{BsdLicense2Clause}. The first rule concerns the (re)distribution in form of
source code, the second the (re)distribution of binary packages. Here are some
explanations why we translated the rules into which sets of executable tasks:

\begin{itemize}
\item For the \enquote{redistribution of source code} the license requires,
that the package must \enquote{ [\ldots] retain the above copyright notice, this
list of conditions and the following disclaimer}\footcite[cf.][\nopage
wp]{BsdLicense2Clause}. Hence, you are not allowed, to modify any of the
copyright notes, which are already embedded in the received (source) files. And
from a logical point of view, there must exist an explicit or implicit
assertion, that the software is lincensed under the \textit{BSD 2-Clause
license}\footcite[The BSD license requires, that a re-distributed software
package must contain the (package specific) copyright notice, the (license
specific) conditions and the BSD discaimer. (cf.][\nopage wp.) You might ask
what you should do, if these elements are missed in the package you got. If so,
the package you got had not been licensed adequately. Hence, you do not know
reliably whether you have received it under a BSD license. In other words: If
you have received a BSD licensed software package, it must contain sufficient
license fulfilling elements, or it is not a BSD licensed
software]{BsdLicense2Clause}. This is often implemented by simply adding a copy
of the license into the package. Hence, you are furthermore not allowed to
modify these files or corresponding text snippets. For our purposes, we
translated the bans into the following executable task:

\begin{quote}\textit{Ensure, that the licensing elements - e.g. the BSD license
text, the specific copyright notice of the original author(s), and the BSD disclaimer
- are retained in your package in the form you have got them.}\end{quote}

\item For the redistribution in form of binary files, the license requires,
that the licensing elements must be \enquote{[\ldots] (reproduced) in the documentation
and/or other materials provided with the
distribution}\footcite[cf.][\nopage wp]{BsdLicense2Clause}. Hence, this is
not required as necessary condition for the (re)distribution as source code
package . But nevertheless, even for a distribution in form of source code, it
is often possible, to fulfill this rule too - e.g., if you offer an own download
site for source code packages. In such cases, it is a sign of respect, to
mention the licensing not only inside of the packages, but also in the text of
your site. Because of that, we added thow following voluntary task for all BSD
Open Source Use Cases, which deal with the redistribution in form of source
code'

\begin{quote}\textit{Let the documentation of your distribution and/or your
additional material also contain the original copyright notice, the BSD
conditions, and the BSD disclaimer.}\end{quote}

\item Naturally, because the reproduction of the licensing elements \enquote{in
the documentation and/or other materials provided with the distribution}
is explicitly required for the \enquote{redistribution in binary
form}\footcite[cf.][\nopage wp]{BsdLicense2Clause}, we had to rewrite the
facultative task for a distribution in form of source code as a mandatory task
for all BSD Open Source Use Cases, which deal with the redistribution in binary
form':

\begin{quote}\textit{Ensure, that the documentation of your distribution and/or
your additional material also contain the author specific copyright notice, the
BSD conditions, and the BSD disclaimer.}\end{quote}

\item In case of (re)distributing the program in form of binary files, it is
sometimes not enough, to pass the licensing elements as one has got them. If you
compile the binary package from the source code, it is not necessarily true,
that the licensing elements are also automatically generated and embedded into
the 'binary package'. But nevertheless, you have to add the copyright notice,
the conditions and the disclaimer to this package for acting according to the
BSD license. Therefore we choosed the following form of an executable, license
fulfilling task for all binary oriented distributions:

\begin{quote}\textit{Ensure, that your distribution contains the original
copyright notice, the BSD license, and the BSD disclaimer in the form you have
got them. If you compile the binary file on the base of the source code package
and if this compilation does not also generate and integrate the licensing
files, then create the copyright notice, the BSD conditions, and the BSD
disclaimer according to the form of the source code package and insert these
files into your distribution manually.}\end{quote}

\item Finally, we wished to insert a hint to the general (Open Source)
tradition, to mention the used Open Source Software and their licenses as a
remark of the 'copyright widget' of an application. This is not required by the
BSD license. But it is a general, good tradition. Naturally, because of the
freedom, to use and modify Open Source Software, and to redistribute a modified
version of it, you are also allowed to insert such references, even if they are
missed. Therefore we added a third voluntary, license tradition fulfilling task
for all relevant Open Source Use Cases.

\end{itemize}




%\bibliography{../../../bibfiles/oscResourcesEn}
