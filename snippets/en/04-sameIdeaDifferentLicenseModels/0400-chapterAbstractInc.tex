% Telekom osCompendium 'for being included' snippet template
%
% (c) Karsten Reincke, Deutsche Telekom AG, Darmstadt 2011
%
% This LaTeX-File is licensed under the Creative Commons Attribution-ShareAlike
% 3.0 Germany License (http://creativecommons.org/licenses/by-sa/3.0/de/): Feel
% free 'to share (to copy, distribute and transmit)' or 'to remix (to adapt)'
% it, if you '... distribute the resulting work under the same or similar
% license to this one' and if you respect how 'you must attribute the work in
% the manner specified by the author ...':
%
% In an internet based reuse please link the reused parts to www.telekom.com and
% mention the original authors and Deutsche Telekom AG in a suitable manner. In
% a paper-like reuse please insert a short hint to www.telekom.com and to the
% original authors and Deutsche Telekom AG into your preface. For normal
% quotations please use the scientific standard to cite.
%
% [ File structure derived from 'mind your Scholar Research Framework' 
%   mycsrf (c) K. Reincke CC BY 3.0  http://mycsrf.fodina.de/ ]
%

% Chapter Abstract
% ----------------

\footnotesize \begin{quote}\itshape In this chapter we describe different
license models which meet the common idea of being a piece of Free Open Source
Software. We want to discuss existing types of grouping licenses to underline
the limits of building such clusters: These groups are often used as 'virtual
prototypic licenses' which shall deliver a simplified view onto the conditions
how to act according to the referred real license instances. But one has to
fulfill the requirements of a specific license, not one's own generalized idea
of a set of licenses. Nevertheless, also we wish to offer a new structuring view
into the world of the Open Source Licenses too. We will use a new set of
grouping criteria by referring to the common intended purpose of each license:
each license wants to protect something or someone against something or someone.
Following this pattern, we can indeed summarize the essence of each license in a
comparable way.
\end{quote}
\normalsize{}

