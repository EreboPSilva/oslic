% Telekom osCompendium 'for beeing included' snippet template
%
% (c) Karsten Reincke, Deutsche Telekom AG, Darmstadt 2011
%
% This LaTeX-File is licensed under the Creative Commons Attribution-ShareAlike
% 3.0 Germany License (http://creativecommons.org/licenses/by-sa/3.0/de/): Feel
% free 'to share (to copy, distribute and transmit)' or 'to remix (to adapt)'
% it, if you '... distribute the resulting work under the same or similar
% license to this one' and if you respect how 'you must attribute the work in
% the manner specified by the author ...':
%
% In an internet based reuse please link the reused parts to www.telekom.com and
% mention the original authors and Deutsche Telekom AG in a suitable manner. In
% a paper-like reuse please insert a short hint to www.telekom.com and to the
% original authors and Deutsche Telekom AG into your preface. For normal
% quotations please use the scientific standard to cite.
%
% [ File structure derived from 'mind your Scholar Research Framework' 
%   mycsrf (c) K. Reincke CC BY 3.0  http://mycsrf.fodina.de/ ]
%

% Chapter Abstract
% ----------------

\footnotesize
\begin{quote}\itshape
In this chapter we describe different license models which fulfill the common
idea. We discusss existing types of grouping single Open Source Licenses. We
highlight the limits of building such clusters for being able to analyse
prototypic licenses. But finally for learning the field we ourselves cluster
these license models in three groups, the \textit{Minimal Prescribing
Licensemodels} for \textit{protecting the developer} like MIT, BSD or Apache,
the \textit{Reflexive Prescribing Licensemodels } for \textit{protecting the
licensed code} like LGPL, EPL? or EUPL and the 'viral' \textit{Overlapping
Prescribing Licensemodels} for \textit{protecting the on-top-developments}. At
the end you will know the main obligations in generally.
\end{quote}
\normalsize{}

