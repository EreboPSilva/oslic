% Telekom osCompendium 'for being included' snippet template
%
% (c) Karsten Reincke, Deutsche Telekom AG, Darmstadt 2011
%
% This LaTeX-File is licensed under the Creative Commons Attribution-ShareAlike
% 3.0 Germany License (http://creativecommons.org/licenses/by-sa/3.0/de/): Feel
% free 'to share (to copy, distribute and transmit)' or 'to remix (to adapt)'
% it, if you '... distribute the resulting work under the same or similar
% license to this one' and if you respect how 'you must attribute the work in
% the manner specified by the author ...':
%
% In an internet based reuse please link the reused parts to www.telekom.com and
% mention the original authors and Deutsche Telekom AG in a suitable manner. In
% a paper-like reuse please insert a short hint to www.telekom.com and to the
% original authors and Deutsche Telekom AG into your preface. For normal
% quotations please use the scientific standard to cite.
%
% [ Framework derived from 'mind your Scholar Research Framework' 
%   mycsrf (c) K. Reincke 2012 CC BY 3.0  http://mycsrf.fodina.de/ ]
%


%% use all entries of the bibliography
%\nocite{*}

\chapter{Open Source Use Cases: Find the License Fulfilling To-do Lists}\label{sec:OSUCfinder}

\footnotesize
\begin{quote}\itshape
This chapter offers the \emph{Open Source Use Case Finder}: Based on the
information gathered by a form, it allows to traverse a tree whose leaves are
linked to the \emph{open source use cases} which finally refer to the respective
to-do lists.
\end{quote}
\normalsize{}

\section{A standard form for gathering the relevant information}
\label{OSLiCStandardFormForGatheringInformation}
 
{
% The 8.5cm below were found by trial and error - 8.6cm will generate an
% overfull hbox. 
\newcommand{\question[1]}{\parbox[c][#1][c]{8.5cm}}
\newcommand{\checkboxes}[2]{%
    \parbox{7.5em}{
        $\square$\hspace{1em}#1\\
        $\square$\hspace{1em}#2}}

\begin{small}
\begin{tabular}[h]{|l|l|l|l|}
\hline 
  \ & \textit{Which open source software do you want to use?} & \ \\
\hline 
  \ & \textit{Under which open source license is it released?} & \ \\
\hline
\hline 
\textbf{Focus} & \textbf{Questions} & \textbf{Answers}\\
\hline 
\hline 
  Type
  & \question[2.9cm]{
    \textit{Is the open source software you want to use a library in the
    broadest sense (an includable code \textbf{\underline{sni}}ppet, a linkable
    \textbf{\underline{mo}}dule or \textbf{\underline{li}}brary, or a loadable
    plugin), or is it an autonomous \textbf{\underline{pro}}gram,
    \textbf{\underline{ap}}plication, or \textbf{\underline{se}}rver which can be
    executed?}} 
  & \checkboxes{snimoli}{proapse}
  \\
\hline 
  State 
  & \question[1.7cm]{
    \textit{Do you want to leave the open source software
    \textbf{\underline{unmodified}} as you have received it, or are you going to
    create a \textbf{\underline{modified}} version of it?}} 
  & \checkboxes{unmodified}{modified}
  \\
\hline 
  Context 
  & \question[2.15cm]{ 
    \textit{Are you going to use / distribute the open source software as an
    \textbf{\underline{independent}} unit, or do you plan to integrate it as an
    \textbf{\underline{embedded}} component into a complexer piece of software?}}
  & \checkboxes{independent}{embedded}
  \\
\hline 
  Recipient 
  & \question[1.7cm]{ 
    \textit{Are you going to use the open source
    software only \textbf{\underline{for}} \textbf{\underline{yourself}}, or do
    you plan to (re)distribute it (also) \textbf{\underline{to}}
    \textbf{\underline{other}} third parties?}}
  & \checkboxes{4yourself}{2others}
  \\
\hline 
  Form 
  & \question[1.7cm]{
    \textit{Given you want to (re)distribute an open source based work [2others],
    do you focus on distributing the \textbf{\underline{binaries}} or the
    \textbf{\underline{sources}}?}}
  & \checkboxes{binaries}{sources}
  \\
\hline 
\hline
\end{tabular}
\end{small}
}

As discussed earlier, there are of course some invalid
combinations.\footnote{type::proapse excludes state::embedded;
recipient::4yourself excludes the combination with state::independent and
type::snimoli; any value of class 'mode' implies state::embedded [for details
see page \pageref{InvalidFinderTokenCombinations}]. If you have encountered one of
these invalid combinations, please check the corresponding explanations.} Here
are some extra explanations concerning the classes resp. the focuses:

\begin{description}
\item[Type:] A piece of (open source) software is a program, an application, or
a server, only if you can start its binary form with your normal program
launcher, or (in case of a text file which still must be interpreted by an
interpreter like php, perl, bash etc.) if you can start an interpreter which
takes the file as one of its arguments and executes the commands.
\item[State:] You are modifying a piece of (open source) software if you expand,
reduce or modify at least one of the received software files, and---in case of
dealing with binary object code---if you (re)compile and (re)link the modified
software to a new binary file. But if you only modify some of the configuration
files, you are not modifying the open source software itself.
\item[Context:] You are using a piece of open source software as an embedded
component of a larger unit \ldots
  \begin{itemize}
  \item  if one of your files of the larger unit contains a verbatim or a
  modified copy (i.e.\ a snippet) of the received open source software, or
  \item if your larger unit contains an include statement referring to a
  functionally defining file of the received open source software, or
  \item if your larger unit calls a function defined in the received open source
  software, or
  \item if your development environment contains a compiler or linker directive
  referring to the received open source software (binaries) and if your larger
  unit can't be executed without resolving this linker directive.
  \end{itemize}
\item[Recipient:] You are using the received open source software only for
yourself, if you as a person do not pass it to other entities like persons,
organizations, companies etc., or if you---as a member of a specific
development group---pass it only to the other members of your development
group. But if you store open source software on any device such as a mobile
phone, an USB stick, etc.\ or if you attach it to any transport medium like
email etc.\ and if you then sell, give away, or simply send this device or
transport medium to anyone (other than a direct member of your development
group) then you indeed hand the open source software over to third
parties.\footnote{Please remember that---at least in Germany---there are
opinions that even handing over software to another legal entity or department
of the same company is also a kind of distribution. It is always safest to take
the broadest possible meaning of distributing or handing over.}
\item[Form:] Open source software knows two ways to distribute the software: in
the form of binaries and in the form of sources. Mostly it is up to you to
decide whether you want to distribute only the binaries or whether you are
intentionally going to distribute the sources (too). At a first glance, the
concepts 'sources' and 'binaries' seems to be clearly distinguished.
On the one hand, compiled sources should be taken as binaries. On the other
hand, editable pieces of software are denoted by the concept 'sources'. But
sometimes the difference is not as clear as wished: For example, you can modify
even already compiled object files by using an hex-editor. Or it is very
difficult to modify the minimized versions of javascript files even if they are
indeed text files. Therefore, the OSLiC 'reuses' a famous \textbf{rule of
thumb}: \enquote{The source code for a work means the preferred form of the work
for making modifications to it}.\citeGPLtwo{§3} All other forms are denoted by
the concept of 'binaries'. Based on this specification, you can respect some
special conditions if you want to distribute the sources and/or the binaries.
\end{description}

\section{The taxonomic Open Source Use Case Finder}

Now, after having gathered the necessary information, determine your 
open source use case by traversing the following tree and its corresponding
branches:

{
\newcommand{\choicetext}[2]{\tiny #1:\\ \textbf{\textit{#2}}}

\newcommand{\cunmodified}{\choicetext{state}{unmodified}}
\newcommand{\cmodified}{\choicetext{state}{modified}}
\newcommand{\cindependent}{\choicetext{context}{independent}}
\newcommand{\cembedded}{\choicetext{context}{embedded}}
\newcommand{\cyourself}{\choicetext{recipient}{4yourself}}
\newcommand{\cothers}{\choicetext{recipient}{2others}}
\newcommand{\csources}{\choicetext{form}{sources}}
\newcommand{\cbinaries}{\choicetext{form}{binaries}}

\newcommand{\osuctxtshort}[1]{$\Rightarrow$ OSUC-#1: \textit{p.\ \pageref{OSUC-#1-DEF}}}
\newcommand{\osuctxtbreak}[1]{$\Rightarrow$ OSUC-#1\\ \textit{(see p.\ \pageref{OSUC-#1-DEF})}}
\newcommand{\osucchild}[1]{child { node[anchor=west] {#1} edge from parent[draw=none] }}

\tikzset{choice/.style={rectangle, draw, rounded corners}}

\begin{tikzpicture}[
    font=\scriptsize,
    align=left,
    grow'=right,
    level 1/.style={sibling distance=24em, level distance=10mm},
    level 2/.style={sibling distance=14em, level distance=10mm},
    level 3/.style={sibling distance=7em,  level distance=18mm, anchor=west, minimum width=2cm},
    level 4/.style={sibling distance=3em,  level distance=18mm, minimum width=1.65cm},
    level 5/.style={sibling distance=3em,  level distance=18mm, minimum width=1.45cm},
    level 6/.style={sibling distance=3em,  level distance=6mm},
]
\node [ellipse,draw] {OSS}
    child { node [choice] { \choicetext{type}{proapse} }
      child { node [choice] {\cunmodified}
        child { node [choice] {\cindependent}
          child { node[choice] {\cyourself} 
            \osucchild{\osuctxtshort{01}}
          }
          child { node[choice] {\cothers} 
            child { node[choice] {\csources} 
              \osucchild{\osuctxtbreak{02S}}
            }
            child { node[choice] {\cbinaries} 
             \osucchild{\osuctxtbreak{02B}}
            }
          }
        }
      }
      child { node [choice] {\cmodified}
        child { node [choice] {\cindependent}
          child { node[choice] {\cyourself} 
            \osucchild{\osuctxtshort{03}}
          }
          child { node[choice] {\cothers} 
            child { node[choice] {\csources} 
              \osucchild{\osuctxtbreak{04S}}
            }
            child { node[choice] {\cbinaries} 
              \osucchild{\osuctxtbreak{04B}}
            }
          }
        }
      }
    }
    child { node [choice] { \choicetext{type}{snimoli} }
      child { node [choice] {\cunmodified}
        child { node [choice] {\cindependent} 
          child { node[choice] {\cothers} 
            child { node[choice] {\csources} 
              \osucchild{\osuctxtbreak{05S}}
            }
            child { node[choice] {\cbinaries} 
              \osucchild{\osuctxtbreak{05B}}
            }
          }
        }
        child { node [choice] {\cembedded} 
          child { node[choice] {\cyourself} 
            \osucchild{\osuctxtshort{06}}
          }
          child { node[choice] {\cothers} 
            child { node[choice] {\csources} 
              \osucchild{\osuctxtbreak{07S}}
            }
            child { node[choice] {\cbinaries} 
              \osucchild{\osuctxtbreak{07B}}
            }
          }
        }
      }
      child { node [choice] {\cmodified}
        child { node [choice] {\cindependent} 
          child { node[choice] {\cothers} 
            child { node[choice] {\csources} 
              \osucchild{\osuctxtbreak{08S}}
            }
            child { node[choice] {\cbinaries} 
              \osucchild{\osuctxtbreak{08B}}
            }
          }
        }
        child { node [choice] {\cembedded} 
          child { node[choice] {\cyourself} 
            \osucchild{\osuctxtshort{09}}
          }
          child { node[choice] {\cothers} 
            child { node[choice] {\csources} 
              \osucchild{\osuctxtbreak{10S}}
            }
            child { node[choice] {\cbinaries} 
              \osucchild{\osuctxtbreak{10B}}
            }
          }
        }
      }
    };
\end{tikzpicture}
\label{OSLiCUseCaseFinder}
}


\section{The open source use cases and its to-do list references}

On the following pages, each \textbf{O}pen \textbf{S}ource \textbf{U}se
\textbf{C}ase is textually specified one more time and complemented by a list of
page numbers. Each of these pages covers the license-specific to-do list whose
items together offer a processable way for acting according to the license under
the circumstances of the described \textbf{O}pen \textbf{S}ource \textbf{U}se
\textbf{C}ase.


\begin{osucdefinitions}
\bgroup
\newcommand{\osuclinktable}[1]{%
  To see the \textit{specific, license fulfilling to-do lists}
  jump to the following pages:
  \begin{itemize}
    \item p.\ \pageref{OSUC-#1-AGPL} for the \textbf{AGPL-3.0}
      \textit{(= GNU Affero General Public License)} 
    \item p.\ \pageref{OSUC-#1-APL} for the \textbf{Apache-2.0}
      \textit{(= Apache License)}
    \item p.\ \pageref{OSUC-#1-BSD2} for the \textbf{BSD-2-Clause} License
      \textit{(= Berkeley Software Distribution)}
    \item p.\ \pageref{OSUC-#1-BSD3} for the \textbf{BSD-3-Clause} License
      \textit{(= Berkeley Software Distribution)}
    \item p.\ \pageref{OSUC-#1-CDDL} for the \textbf{CDDL-1.0}
      \textit{(= Common Develop and Distribution License)}  
    \item p.\ \pageref{OSUC-#1-EPL} for the \textbf{EPL-1.0}
      \textit{(= Eclipse Public License)}     
    \item p.\ \pageref{OSUC-01-EUPL} for the \textbf{EUPL-1.1}
      \textit{(= European Union Public License)} 
    \item p.\ \pageref{OSUC-#1-GPL2} for the \textbf{GPL-2.0}
       \textit{(= GNU General Public License Version 2)} 
    \item p.\ \pageref{OSUC-#1-GPL3} for the \textbf{GPL-3.0}
       \textit{(= GNU General Public License Version 3)} 
    \item p.\ \pageref{OSUC-#1-LGPL2} for the \textbf{LGPL-2.1}
      \textit{(= GNU Lesser General Public License Version 2.1)}           
    \item p.\ \pageref{OSUC-#1-LGPL3} for the \textbf{LGPL-3.0}
      \textit{(= GNU Lesser General Public License Version 3)}           
    \item p.\ \pageref{OSUC-#1-MIT} for the \textbf{MIT} License
       \textit{(= Massachusetts Institute of Technology)} 
    \item p.\ \pageref{OSUC-#1-MPL} for the \textbf{MPL}
      \textit{(= Mozilla Public License)}     
    \item p.\ \pageref{OSUC-#1-MSPL} for the \textbf{MS-PL}
      \textit{(= Microsoft Public License)} 
    \item p.\ \pageref{OSUC-#1-PGL} for the \textbf{PostgreSQL}
      \textit{(= Postgres License)} 
    \item p.\ \pageref{OSUC-#1-PHP} for the \textbf{PHP-3.0} License 
  \end{itemize}}

\newcommand{\osucitem}[3]{%
  \osucdef{#1}{#2}{#3}
  \osuclinktable{#1}}

\begin{description}
\label{OSUCList}
\osucitem{01}{proapse, unmodified, independent, 4yourself}{%
Only for yourself, you are going to use an unmodified open source program,
application, or server just as you received it. But you do not combine it with
other components in the sense of software development} 

\osucitem{02S}{proapse, unmodified, independent, 2others, sources}{%
Just as you received it, you are going to distribute an unmodified open source
program, application, or server to third parties in the form of sources. In this
act of distribution, you do not combine this program, application, or server
with other software components in the sense of software development} 

\osucitem{02B}{proapse, unmodified, independent, 2others, binaries}{%
Just as you received it, you are going to distribute an unmodified open source
program, application, or server to third parties in the form of binaries. In
this act of distribution, you do not combine this program, application, or
server with other software components in the sense of software development} 
  
\osucitem{03}{proapse, modified, independent, 4yourself}{%
Only for yourself, you are going to modify an open source program, application,
or server after you received it and before you will use it. But you do not
combine it with other components in the sense of software development} 

\osucitem{04S}{proapse, modified, independent, 2others, sources}{%
You are going to modify an open source program, application, or server after you
received it and  before you will distribute it to third parties in the form of
sources. But you do not combine this modified program, application, or server
with other software components in the sense of software development}
  
\osucitem{04B}{proapse, modified, independent, 2others, binaries}{%
You are going to modify an open source program, application, or server after you
received it and before you will distribute it to third parties in the form of
binaries. But you do not combine this modified program, application, or server
with other software components in the sense of software development}

\osucitem{05S}{snimoli, unmodified, independent, 2others, sources}{%
Just as you received it, you are going to distribute an unmodified open source
library, code snippet, module, or plugin to third parties in the form of
sources. In this act of distribution, you do not combine this library, code
snippet, module, or plugin with other software components in the sense of
software development} 

\osucitem{05B}{snimoli, unmodified,independent, 2others, binaries}{%
Just as you received it, you are going to distribute an unmodified open source
library, code snippet, module, or plugin to third parties in the form of
binaries. In this act of distribution, you do not combine this library, code
snippet, module, or plugin with other software components in the sense of
software development} 

\osucitem{06}{snimoli, unmodified, embedded, 4yourself}{%
Only for yourself and just as you received it, you are going to combine an
unmodified open source library, code snippet, module, or plugin into a larger
software unit as one of its parts}  

\osucitem{07S}{snimoli, unmodified, embedded, 2others, sources}{%
Just as you received it and before you will distribute it to third parties in
the form of sources and together with a larger software unit, you are going to
combine and embed an unmodified open source library, code snippet, module, or
plugin into that larger software unit in the sense of software development}

\osucitem{07B}{snimoli, unmodified, embedded, 2others, binaries}{%
Just as you received it and before you will distribute it to third parties in
the form of binaries and together with a larger software unit, you are going to
combine and embed an unmodified open source library, code snippet, module, or
plugin into that larger software unit in the sense of software development}

\osucitem{08S}{snimoli, modified, independent, 2others, sources}{%
Before you will distribute it to third parties in the form of sources, you are
going to modify an open source library, code snippet, module, or plugin. But you
do not combine it with other software components in the sense of software
development}  

\osucitem{08B}{snimoli, modified, independent, 2others, binaries}{%
Before you will distribute it to third parties in the form of binaries, you are
going to modify an open source library, code snippet, module, or plugin. But you
do not combine it with other software components in the sense of software
development} 

\osucitem{09}{snimoli, modified, embedded, 4yourself}{%
Only for yourself, you are going to modify an open source library, code snippet,
module, or plugin, and you will combine it in the sense of software development
into a larger software unit as one of its parts} 

\osucitem{10S}{snimoli, modified, independent, 2others, sources}{%
Before you will distribute it to third parties in the form of sources, you are
going to modify an open source library, code snippet, module, or plugin, which
you combine with other software components in the sense of software development}

\osucitem{10B}{snimoli, modified, independent, 2others, binaries}{%
Before you will distribute it to third parties in the form of binaries, you are
going to modify an open source library, code snippet, module, or plugin, which
you combine with other software components in the sense of software development}
\end{description}
\egroup
\end{osucdefinitions}

%\bibliography{../../../bibfiles/oscResourcesEn}

% Local Variables:
% mode: latex
% fill-column: 80
% End:
