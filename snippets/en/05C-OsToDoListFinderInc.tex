% Telekom osCompendium 'for being included' snippet template
%
% (c) Karsten Reincke, Deutsche Telekom AG, Darmstadt 2011
%
% This LaTeX-File is licensed under the Creative Commons Attribution-ShareAlike
% 3.0 Germany License (http://creativecommons.org/licenses/by-sa/3.0/de/): Feel
% free 'to share (to copy, distribute and transmit)' or 'to remix (to adapt)'
% it, if you '... distribute the resulting work under the same or similar
% license to this one' and if you respect how 'you must attribute the work in
% the manner specified by the author ...':
%
% In an internet based reuse please link the reused parts to www.telekom.com and
% mention the original authors and Deutsche Telekom AG in a suitable manner. In
% a paper-like reuse please insert a short hint to www.telekom.com and to the
% original authors and Deutsche Telekom AG into your preface. For normal
% quotations please use the scientific standard to cite.
%
% [ Framework derived from 'mind your Scholar Research Framework' 
%   mycsrf (c) K. Reincke 2012 CC BY 3.0  http://mycsrf.fodina.de/ ]
%


%% use all entries of the bibliography
%\nocite{*}

\chapter{Open Source Use Cases: Find the License Fulfilling To-do Lists}\label{sec:OSUCfinder}

\footnotesize
\begin{quote}\itshape
This chapter offers the \emph{Open Source Use Case Finder}: Based on the
information gathered by a form, it allows to traverse a tree whose leafs are
linked to the \emph{open source use cases} which finally refer to the respective
to-do lists.
\end{quote}
\normalsize{}

\section{A standard form for gathering the relevant information}
\label{OSLiCStandardFormForGatheringInformation}
 
\begin{small}
\begin{tabular}[h]{|l|l|l|l|}
\hline 
  \ & \textit{Which open source software do you want to use?} & \ \\
\hline 
  \ & \textit{Under which open source license is it released?} & \ \\
\hline
\hline 
\textbf{Class} & \textbf{Questions} & \textbf{Answers}\\
\hline 
\hline 
  Type
  & \parbox[c][2.6cm][c]{10cm}{
    \textit{Is the open source software you want to use a library in the
    broadest sense (an includable code \textbf{\underline{sni}}ppet, a linkable
    \textbf{\underline{mo}}dule or \textbf{\underline{li}}brary, or a loadable
    plugin), or is it an autonomous \textbf{\underline{pro}}gram,
    \textbf{\underline{ap}}plication or \textbf{\underline{se}}rver which can be
    executed?}} & \parbox{7.5em}{
      $\square$\hspace{1em}snimoli\\
      $\square$\hspace{1em}proapse}
    \\
\hline 
  State & 
  \parbox[c][1.6cm][c]{10cm}{
  \textit{Do you want to leave the open source software
  \textbf{\underline{unmodified}} as you have received it, or are you going to
  create a \textbf{\underline{modified}} version of it?}} &
  \parbox{7.5em}{
    $\square$\hspace{1em}unmodified\\
    $\square$\hspace{1em}modified} \\
\hline 
  Context & 
  \parbox[c][1.6cm][c]{10cm}{
  \textit{Are you going to use / distribute the open source software as an
  \textbf{\underline{independent}} unit, or do you plan to integrate it as an
  \textbf{\underline{embedded}} component into a complexer piece of
  software?}} & \parbox{7.5em}{
  $\square$\hspace{1em}independent\\
    $\square$\hspace{1em}embedded}\\
\hline 
  Recipient & 
  \parbox[c][1.6cm][c]{10cm}{ \textit{Are you going to use the open source
  software only \textbf{\underline{for}} \textbf{\underline{yourself}}, or do
  you plan to (re)distribute it (also) \textbf{\underline{to}}
  \textbf{\underline{other}} third parties?}}
  & \parbox{7.5em}{
    $\square$\hspace{1em}4yourself\\
    $\square$\hspace{1em}2others}\\
\hline 
  Form & 
  \parbox[c][1.6cm][c]{10cm}{
  \textit{Given you want to (re)distribute an open source based work [2others],
  do you focus on distributing the \textbf{\underline{binaries}} or the
  \textbf{\underline{sources}}?}}
  & \parbox{7.5em}{
    $\square$\hspace{1em}binaries\\
    $\square$\hspace{1em}sources}\\
\hline 
\hline
\end{tabular}
\end{small}

As discussed earlier, there are of course some invalid
combinations\footnote{type::proapse excludes state::embedded;
recipient::4yourself excludes the combination with state::independent and
type::snimoli; any value of class 'mode' implies state::embedded [for details
see page \pageref{InvalidFinderTokenCombinations}]. If you have gathered one of
these invalid combinations, please check the corresponding explanations}. Here
are some extra explanations about each class:

\begin{description}
\item[Type:] A piece of (open source) software is a program, an application, or
a server, if you can start its binary form with your normal program launcher, or
(in case of a text file which still must be interpreted by an interpreter like
php, perl, bash etc.) if you can start an interpreter which takes the file as
one of its arguments and executes the commands.
\item[State:] You are modifying a piece of (open source) software if you expand,
reduce or modify at least one of the received software files, and -- in case of
dealing with binary object code -- if you (re)compile and (re)link the modified
software to a new binary file. But if you only modify some of the configuration
files, you are not modifying the open source software itself.
\item[Context:] You are using a piece of open source software as an embedded
component of a larger unit if
  \begin{itemize}
  \item  if one of your files of the larger unit contains a verbatim or a
  modified copy (i.e.\ a snippet) of the received open source software, or
  \item if your larger unit contains an include statement referring to a
  functionally defining file of the received open source software, or
  \item if your larger unit calls a function defined in the received open source
  software, or
  \item if your development environment contains a compiler or linker directive
  referring to the received open source software (binaries) and if your larger
  unit can't be executed without resolving this linker directive.
  \end{itemize}
\item[Recipient:] You are using the received open source software only for
yourself, if you as a person do not pass it to other entities like persons,
organizations, companies etc., or if you -- as a member of a specific
development group -- pass it only to the other members of your development
group. But if you store open source software on any device such as a mobile
phone, an USB stick, etc.\ or if you attach it to any transport medium like
email etc.\ and if you then sell, give away, or simply send this device or
transport medium to anyone (other than a direct member of your development
group) then you indeed handover the open source software to third
parties\footnote{Please remember that -- at least in Germany -- there are
opinions that even handing over software to another legal entity or department
of the same company is also a kind of distribution. It is always safest to take
the broadest possible meaning of distributing or handing over.}.
\item[Form:] Mostly it is up to you to decide whether you want to distribute
only the binaries or whether you are intentionally going to distribute the
sources (too). But in some cases, you have to respect some special
conditions.\footnote{For details concerning a necessary refinement of the open
source use case taxonomy, please see $\rightarrow$ OSLiC, p.\
\pageref{sec:SourceBinaryDifference}}.
\end{description}

\section{The taxonomic Open Source Use Case Finder}

Now, after having gathered the necessary information, determine your specific
open source use case by traversing the following tree and its corresponding
branches:

\begin{footnotesize}

\pstree[treemode=R,levelsep=*0.2,treesep=0.2]{\Toval{OSS}}{
  \pstree{
    \Tr{\Ovalbox{\shortstack{type:\\\textbf{\textit{proapse}}\\
      \tiny (= Program,\\\tiny Application,\\\tiny Server)      
      }   
    }}
    
  }{
    \pstree{
      \Tr{\Ovalbox{\shortstack{state:\\\textbf{\textit{unmodified}}}}}
    }{
      \pstree{
        \Tr{\Ovalbox{\shortstack{context:\\\textbf{\textit{independent}}}}}
      }{
        
        \pstree{
          \Tr{\Ovalbox{\shortstack{recipient:\\\textbf{\textit{4yourself}}}}}
        }{
          \Tr[edge=none]{\begin{minipage}[b][2em][c]{9em} 
              $\Rightarrow$ OSUC-01: \textit{p.\ \pageref{OSUC-01-DEF}}\end{minipage} } 
          }
        
        \pstree{
          \Tr{\Ovalbox{\shortstack{recipient:\\\textbf{\textit{4others}}}}}
        }{
          
         \pstree{
            \Tr{\Ovalbox{\shortstack{form:\\\textbf{\textit{sources}}}}}
          }{
            \Tr[edge=none]{\begin{minipage}[b][2em][c]{6.5em} 
              $\Rightarrow$ OSUC-02S\\
              \textit{(see p.\ \pageref{OSUC-02S-DEF})}\end{minipage} } 
           } 
           
          \pstree{
            \Tr{\Ovalbox{\shortstack{form:\\\textbf{\textit{binaries}}}}}
          }{
            \Tr[edge=none]{\begin{minipage}[b][2em][c]{6.5em} 
              $\Rightarrow$ OSUC-02B\\
              \textit{(see p.\ \pageref{OSUC-02B-DEF})}\end{minipage} } 
           }
           
         }       
      }
    }
    \pstree{
      \Tr{\Ovalbox{\shortstack{state:\\\textbf{\textit{modified}}}}}
    }{
      \pstree{
        \Tr{\Ovalbox{\shortstack{context:\\\textbf{\textit{independent}}}}}
      }{
        \pstree{
          \Tr{\Ovalbox{\shortstack{recipient:\\\textbf{\textit{4yourself}}}}}
        }{
          \Tr[edge=none]{\begin{minipage}[b][2em][c]{9em} 
              $\Rightarrow$ OSUC-03: 
              \textit{p.\ \pageref{OSUC-03-DEF}}\end{minipage} } 
          }
        
        \pstree{
          \Tr{\Ovalbox{\shortstack{recipient:\\\textbf{\textit{4others}}}}}
        }{
        
         \pstree{
            \Tr{\Ovalbox{\shortstack{form:\\\textbf{\textit{sources}}}}}
          }{
            \Tr[edge=none]{\begin{minipage}[b][2em][c]{6.5em} 
              $\Rightarrow$ OSUC-04S\\
              \textit{(see p.\ \pageref{OSUC-04S-DEF})}\end{minipage} } 
           } 
           
          \pstree{
            \Tr{\Ovalbox{\shortstack{form:\\\textbf{\textit{binaries}}}}}
          }{
            \Tr[edge=none]{\begin{minipage}[b][2em][c]{6.5em} 
              $\Rightarrow$ OSUC-04B\\
              \textit{(see p.\ \pageref{OSUC-04B-DEF})}\end{minipage} } 
           }
         }        
      }
    }
  }
  \pstree{
    \Tr{\Ovalbox{
      \shortstack{type:\\\textbf{\textit{snimoli}}\\
      \tiny (= Snippet,\\\tiny Module,\\\tiny Plugin,\\\tiny Library)            
      }
    }}
  }{
    \pstree{
      \Tr{\Ovalbox{\shortstack{state:\\\textbf{\textit{unmodified}}}}}
    }{
      \pstree{
        \Tr{\Ovalbox{\shortstack{context:\\\textbf{\textit{independent}}}}}
      }{
      
        \pstree{
          \Tr{\Ovalbox{\shortstack{recipient:\\\textbf{\textit{4others}}}}}
        }{
              
         \pstree{
            \Tr{\Ovalbox{\shortstack{form:\\\textbf{\textit{sources}}}}}
          }{
            \Tr[edge=none]{\begin{minipage}[b][2em][c]{6.5em} 
              $\Rightarrow$ OSUC-05S\\
              \textit{(see p.\ \pageref{OSUC-05S-DEF})}\end{minipage} } 
           } 
           
          \pstree{
            \Tr{\Ovalbox{\shortstack{form:\\\textbf{\textit{binaries}}}}}
          }{
            \Tr[edge=none]{\begin{minipage}[b][2em][c]{6.5em} 
              $\Rightarrow$ OSUC-05B\\
              \textit{(see p.\ \pageref{OSUC-05B-DEF})}\end{minipage} } 
           }   
   
         }
      }
      \pstree{
        \Tr{\Ovalbox{\shortstack{context:\\\textbf{\textit{embedded}}}}}
      }{
      
        \pstree{
          \Tr{
            \Ovalbox{\shortstack{recipient:\\\textbf{\textit{4yourself}}}}
           }
        }{
        
          \Tr[edge=none]{\begin{minipage}[b][2em][c]{9em}
              $\Rightarrow$ OSUC-06:
              \textit{p.\ \pageref{OSUC-06-DEF}}\end{minipage} }
          }
        
        
        \pstree{
          \Tr{\Ovalbox{\shortstack{recipient:\\\textbf{\textit{4others}}}}}
        }{

          \pstree{
            \Tr{\Ovalbox{\shortstack{form:\\\textbf{\textit{sources}}}}}
          }{
            \Tr[edge=none]{\begin{minipage}[b][2em][c]{6.5em} 
              $\Rightarrow$ OSUC-07S\\
              \textit{(see p.\ \pageref{OSUC-07S-DEF})}\end{minipage} } 
           } 
           
          \pstree{
            \Tr{\Ovalbox{\shortstack{form:\\\textbf{\textit{binaries}}}}}
          }{
            \Tr[edge=none]{\begin{minipage}[b][2em][c]{6.5em} 
              $\Rightarrow$ OSUC-07B\\
              \textit{(see p.\ \pageref{OSUC-07B-DEF})}\end{minipage} } 
           } 

         }
      }
    }
    \pstree{
      \Tr{\Ovalbox{\shortstack{state:\\\textbf{\textit{modified}}}}}
    }{
      \pstree{
        \Tr{\Ovalbox{\shortstack{context:\\\textbf{\textit{independent}}}}}
      }{
        \pstree{
          \Tr{\Ovalbox{\shortstack{recipient:\\\textbf{\textit{4others}}}}}
        }{
                      
          \pstree{
            \Tr{\Ovalbox{\shortstack{form:\\\textbf{\textit{sources}}}}}
          }{
            \Tr[edge=none]{\begin{minipage}[b][2em][c]{6.5em} 
              $\Rightarrow$ OSUC-08S\\
              \textit{(see p.\ \pageref{OSUC-08S-DEF})}\end{minipage} } 
           } 
           
          \pstree{
            \Tr{\Ovalbox{\shortstack{form:\\\textbf{\textit{binaries}}}}}
          }{
            \Tr[edge=none]{\begin{minipage}[b][2em][c]{6.5em} 
              $\Rightarrow$ OSUC-08B\\
              \textit{(see p.\ \pageref{OSUC-08B-DEF})}\end{minipage} } 
           }              

          }        
      }
      \pstree{
        \Tr{\Ovalbox{\shortstack{context:\\\textbf{\textit{embedded}}}}}
      }{
        \pstree{
          \Tr{\Ovalbox{\shortstack{recipient:\\\textbf{\textit{4yourself}}}}}
        }{
           \Tr[edge=none]{\begin{minipage}[b][2em][c]{9em} 
              $\Rightarrow$ OSUC-09:
              \textit{p.\ \pageref{OSUC-09-DEF}}\end{minipage} } 
          } 
        
        \pstree{
          \Tr{\Ovalbox{\shortstack{recipient:\\\textbf{\textit{4others}}}}}
        }{

          \pstree{
            \Tr{\Ovalbox{\shortstack{form:\\\textbf{\textit{sources}}}}}
          }{
            \Tr[edge=none]{\begin{minipage}[b][2em][c]{6.5em} 
              $\Rightarrow$ OSUC-10S\\
              \textit{(see p.\ \pageref{OSUC-10S-DEF})}\end{minipage} } 
           } 
           
          \pstree{
            \Tr{\Ovalbox{\shortstack{form:\\\textbf{\textit{binaries}}}}}
          }{
            \Tr[edge=none]{\begin{minipage}[b][2em][c]{6.5em} 
              $\Rightarrow$ OSUC-10B\\
              \textit{(see p.\ \pageref{OSUC-10B-DEF})}\end{minipage} } 
           } 
           
         }

      }
    }
  }
}
\end{footnotesize}
\label{OSLiCUseCaseFinder}

\section{The open source use cases and its to-do list references}

On the following pages, each \textbf{O}pen \textbf{S}ource \textbf{U}se
\textbf{C}ase is textually specified one more time and complemented by a list of
page numbers. Each of these pages covers the license-specific to-do list whose
items together offer a processable way for acting according to the license under
the circumstances of the described \textbf{O}pen \textbf{S}ource \textbf{U}se
\textbf{C}ase.

\begin{description}
\label{OSUCList}
\item[OSUC-01:]\label{OSUC-01-DEF}
Only for yourself, you are going to use an unmodified open source program,
application, or server -- just as you received it. But you do not combine it
with other components in the sense of software development (= \textit{proapse,
unmodified, independent, 4yourself}).
To see the \textit{specific, license fulfilling to-do lists} jump to the
following pages:
  \begin{itemize}
    \item p.\ \pageref{OSUC-01-AGPL} for the \textbf{AGPL}
      \textit{(= Affero GNU Public License)} 
    \item p.\ \pageref{OSUC-01-Apache20} for the \textbf{ApL}
      \textit{(= Apache License)}
    \item p.\ \pageref{OSUC-01-BSD} for the \textbf{BSD} License
      \textit{(= Berkeley Software Distribution)}
    \item p.\ \pageref{OSUC-01-EPL} for the \textbf{EPL}
      \textit{(= Eclipse Public License)}     
    \item p.\ \pageref{OSUC-01-EUPL} for the \textbf{EUPL}
      \textit{(= European Union Public License)} 
    \item p.\ \pageref{OSUC-01-GPL} for the \textbf{GPL}
       \textit{(= GNU Public License)} 
    \item p.\ \pageref{OSUC-01-LGPL} for the \textbf{LGPL}
      \textit{(= Lesser GNU Public License)}           
    \item p.\ \pageref{OSUC-01-MIT} for the \textbf{MIT} License
       \textit{(= Massachusetts Institute of Technology)} 
    \item p.\ \pageref{OSUC-01-MPL} for \textbf{MPL}
      \textit{(= Mozilla Public License)}     
    \item p.\ \pageref{OSUC-01-MS-PL} for the \textbf{MS-PL}
      \textit{(= Microsoft Public License)} 
    \item p.\ \pageref{OSUC-01-PGL} for the \textbf{PGL}
      \textit{(= Postgres License)} 
    \item p.\ \pageref{OSUC-01-PHP} for the \textbf{PHP} License 
  \end{itemize}

\item[OSUC-02S:]\label{OSUC-02S-DEF} Just as you received it, you are going to
distribute an unmodified open source program, application, or server to 3rd
parties -- in the form of sources. In this act of distribution, you do not
combine this program, application, or server with other software components in
the sense of software development (= \textit{proapse, unmodified, independent,
4others, sources}) To see the \textit{specific, license fulfilling to-do lists}
jump to the following pages:
   \begin{itemize}
    \item p.\ \pageref{OSUC-02S-AGPL} for the \textbf{AGPL}
      \textit{(= Affero GNU Public License)} 
    \item p.\ \pageref{OSUC-02S-Apache20} for the \textbf{ApL}
      \textit{(= Apache License)}
    \item p.\ \pageref{OSUC-02S-BSD} for the \textbf{BSD} License
      \textit{(= Berkeley Software Distribution)}
    \item p.\ \pageref{OSUC-02S-EPL} for the \textbf{EPL}
      \textit{(= Eclipse Public License)}     
    \item p.\ \pageref{OSUC-02S-EUPL} for the \textbf{EUPL}
      \textit{(= European Union Public License)} 
    \item p.\ \pageref{OSUC-02S-GPL} for the \textbf{GPL}
       \textit{(= GNU Public License)} 
    \item p.\ \pageref{OSUC-02S-LGPL} for the \textbf{LGPL}
      \textit{(= Lesser GNU Public License)}           
    \item p.\ \pageref{OSUC-02S-MIT} for the \textbf{MIT} License
       \textit{(= Massachusetts Institute of Technology)} 
    \item p.\ \pageref{OSUC-02S-MPL} for \textbf{MPL}
      \textit{(= Mozilla Public License)}     
    \item p.\ \pageref{OSUC-02S-MS-PL} for the \textbf{MS-PL}
      \textit{(= Microsoft Public License)} 
    \item p.\ \pageref{OSUC-02S-PGL} for the \textbf{PGL}
      \textit{(= Postgres License)} 
    \item p.\ \pageref{OSUC-02S-PHP} for the \textbf{PHP} License 
  \end{itemize}

\item[OSUC-02B:]\label{OSUC-02B-DEF} Just as you received it, you are going to
distribute an unmodified open source program, application, or server to 3rd
parties -- in the form of binaries. In this act of distribution, you do not
combine this program, application, or server with other software components in
the sense of software development (= \textit{proapse, unmodified, independent,
4others, binaries}). To see the \textit{specific, license fulfilling to-do
lists} jump to the following pages:
   \begin{itemize}
    \item p.\ \pageref{OSUC-02B-AGPL} for the \textbf{AGPL}
      \textit{(= Affero GNU Public License)} 
    \item p.\ \pageref{OSUC-02B-Apache20} for the \textbf{ApL}
      \textit{(= Apache License)}
    \item p.\ \pageref{OSUC-02B-BSD} for the \textbf{BSD} License
      \textit{(= Berkeley Software Distribution)}
    \item p.\ \pageref{OSUC-02B-EPL} for the \textbf{EPL}
      \textit{(= Eclipse Public License)}     
    \item p.\ \pageref{OSUC-02B-EUPL} for the \textbf{EUPL}
      \textit{(= European Union Public License)} 
    \item p.\ \pageref{OSUC-02B-GPL} for the \textbf{GPL}
       \textit{(= GNU Public License)} 
    \item p.\ \pageref{OSUC-02B-LGPL} for the \textbf{LGPL}
      \textit{(= Lesser GNU Public License)}           
    \item p.\ \pageref{OSUC-02B-MIT} for the \textbf{MIT} License
       \textit{(= Massachusetts Institute of Technology)} 
    \item p.\ \pageref{OSUC-02B-MPL} for \textbf{MPL}
      \textit{(= Mozilla Public License)}     
    \item p.\ \pageref{OSUC-02B-MS-PL} for the \textbf{MS-PL}
      \textit{(= Microsoft Public License)} 
    \item p.\ \pageref{OSUC-02B-PGL} for the \textbf{PGL}
      \textit{(= Postgres License)} 
    \item p.\ \pageref{OSUC-02B-PHP} for the \textbf{PHP} License 
  \end{itemize}
  
\item[OSUC-03:]\label{OSUC-03-DEF} Only for yourself, you are going to modify a
received open source program, application, or server, before you will use it.
But you do not combine it with other components in the sense of software
development (= \textit{proapse, modified, independent, 4yourself}).
To see the \textit{specific, license fulfilling to-do lists} jump to the
following pages:
   \begin{itemize}
    \item p.\ \pageref{OSUC-03-AGPL} for the \textbf{AGPL}
      \textit{(= Affero GNU Public License)} 
    \item p.\ \pageref{OSUC-03-Apache20} for the \textbf{ApL}
      \textit{(= Apache License)}
    \item p.\ \pageref{OSUC-03-BSD} for the \textbf{BSD} License
      \textit{(= Berkeley Software Distribution)}
    \item p.\ \pageref{OSUC-03-EPL} for the \textbf{EPL}
      \textit{(= Eclipse Public License)}     
    \item p.\ \pageref{OSUC-03-EUPL} for the \textbf{EUPL}
      \textit{(= European Union Public License)} 
    \item p.\ \pageref{OSUC-03-GPL} for the \textbf{GPL}
       \textit{(= GNU Public License)} 
    \item p.\ \pageref{OSUC-03-LGPL} for the \textbf{LGPL}
      \textit{(= Lesser GNU Public License)}           
    \item p.\ \pageref{OSUC-03-MIT} for the \textbf{MIT} License
       \textit{(= Massachusetts Institute of Technology)} 
    \item p.\ \pageref{OSUC-03-MPL} for \textbf{MPL}
      \textit{(= Mozilla Public License)}     
    \item p.\ \pageref{OSUC-03-MS-PL} for the \textbf{MS-PL}
      \textit{(= Microsoft Public License)} 
    \item p.\ \pageref{OSUC-03-PGL} for the \textbf{PGL}
      \textit{(= Postgres License)} 
    \item p.\ \pageref{OSUC-03-PHP} for the \textbf{PHP} License 
  \end{itemize}

\item[OSUC-04S:]\label{OSUC-04S-DEF} You are going to modify a received open
source program, application, or server, before you will distribute it to 3rd
parties -- in the form of sources. But you do not combine this modified program,
application, or server with other software components in the sense of software
development (= \textit{proapse, modified, independent, 4others, sources}).
To see the \textit{specific, license fulfilling to-do lists} jump to the
following pages:
  \begin{itemize}
    \item p.\ \pageref{OSUC-04S-AGPL} for the \textbf{AGPL}
      \textit{(= Affero GNU Public License)} 
    \item p.\ \pageref{OSUC-04S-Apache20} for the \textbf{ApL}
      \textit{(= Apache License)}
    \item p.\ \pageref{OSUC-04S-BSD} for the \textbf{BSD} License
      \textit{(= Berkeley Software Distribution)}
    \item p.\ \pageref{OSUC-04S-EPL} for the \textbf{EPL}
      \textit{(= Eclipse Public License)}     
    \item p.\ \pageref{OSUC-04S-EUPL} for the \textbf{EUPL}
      \textit{(= European Union Public License)} 
    \item p.\ \pageref{OSUC-04S-GPL} for the \textbf{GPL}
       \textit{(= GNU Public License)} 
    \item p.\ \pageref{OSUC-04S-LGPL} for the \textbf{LGPL}
      \textit{(= Lesser GNU Public License)}           
    \item p.\ \pageref{OSUC-04S-MIT} for the \textbf{MIT} License
       \textit{(= Massachusetts Institute of Technology)} 
    \item p.\ \pageref{OSUC-04S-MPL} for \textbf{MPL}
      \textit{(= Mozilla Public License)}     
    \item p.\ \pageref{OSUC-04S-MS-PL} for the \textbf{MS-PL}
      \textit{(= Microsoft Public License)} 
    \item p.\ \pageref{OSUC-04S-PGL} for the \textbf{PGL}
      \textit{(= Postgres License)} 
    \item p.\ \pageref{OSUC-04S-PHP} for the \textbf{PHP} License 
  \end{itemize}
  
\item[OSUC-04B:]\label{OSUC-04B-DEF} You are going to modify a received open
source program, application, or server, before you will distribute it to 3rd
parties -- in the form of binaries. But you do not combine this modified
program, application, or server with other software components in the sense of
software development (= \textit{proapse, modified, independent, 4others,
binaries}). To see the \textit{specific, license fulfilling to-do lists} jump to
the following pages:
  \begin{itemize}
    \item p.\ \pageref{OSUC-04B-AGPL} for the \textbf{AGPL}
      \textit{(= Affero GNU Public License)} 
    \item p.\ \pageref{OSUC-04B-Apache20} for the \textbf{ApL}
      \textit{(= Apache License)}
    \item p.\ \pageref{OSUC-04B-BSD} for the \textbf{BSD} License
      \textit{(= Berkeley Software Distribution)}
    \item p.\ \pageref{OSUC-04B-EPL} for the \textbf{EPL}
      \textit{(= Eclipse Public License)}     
    \item p.\ \pageref{OSUC-04B-EUPL} for the \textbf{EUPL}
      \textit{(= European Union Public License)} 
    \item p.\ \pageref{OSUC-04B-GPL} for the \textbf{GPL}
       \textit{(= GNU Public License)} 
    \item p.\ \pageref{OSUC-04B-LGPL} for the \textbf{LGPL}
      \textit{(= Lesser GNU Public License)}           
    \item p.\ \pageref{OSUC-04B-MIT} for the \textbf{MIT} License
       \textit{(= Massachusetts Institute of Technology)} 
    \item p.\ \pageref{OSUC-04B-MPL} for \textbf{MPL}
      \textit{(= Mozilla Public License)}     
    \item p.\ \pageref{OSUC-04B-MS-PL} for the \textbf{MS-PL}
      \textit{(= Microsoft Public License)} 
    \item p.\ \pageref{OSUC-04B-PGL} for the \textbf{PGL}
      \textit{(= Postgres License)} 
    \item p.\ \pageref{OSUC-04B-PHP} for the \textbf{PHP} License 
  \end{itemize}


\item[OSUC-05S:]\label{OSUC-05S-DEF} Just as you received it, you are going to
distribute an unmodified open source library, code snippet, module, or plugin to
3rd parties  -- in the form of sources. In this act of distribution, you do not
combine this library, code snippet, module, or plugin with other software
components in the sense of software development (= \textit{snimoli, unmodified,
independent, 4others, sources}). To see the \textit{specific, license fulfilling
to-do lists} jump to the following pages:
  \begin{itemize}
    \item p.\ \pageref{OSUC-05S-AGPL} for the \textbf{AGPL}
      \textit{(= Affero GNU Public License)} 
    \item p.\ \pageref{OSUC-05S-Apache20} for the \textbf{ApL}
      \textit{(= Apache License)}
    \item p.\ \pageref{OSUC-05S-BSD} for the \textbf{BSD} License
      \textit{(= Berkeley Software Distribution)}
    \item p.\ \pageref{OSUC-05S-EPL} for the \textbf{EPL}
      \textit{(= Eclipse Public License)}     
    \item p.\ \pageref{OSUC-05S-EUPL} for the \textbf{EUPL}
      \textit{(= European Union Public License)} 
    \item p.\ \pageref{OSUC-05S-GPL} for the \textbf{GPL}
       \textit{(= GNU Public License)} 
    \item p.\ \pageref{OSUC-05S-LGPL} for the \textbf{LGPL}
      \textit{(= Lesser GNU Public License)}           
    \item p.\ \pageref{OSUC-05S-MIT} for the \textbf{MIT} License
       \textit{(= Massachusetts Institute of Technology)} 
    \item p.\ \pageref{OSUC-05S-MPL} for \textbf{MPL}
      \textit{(= Mozilla Public License)}     
    \item p.\ \pageref{OSUC-05S-MS-PL} for the \textbf{MS-PL}
      \textit{(= Microsoft Public License)} 
    \item p.\ \pageref{OSUC-05S-PGL} for the \textbf{PGL}
      \textit{(= Postgres License)} 
    \item p.\ \pageref{OSUC-05S-PHP} for the \textbf{PHP} License 
  \end{itemize}

\item[OSUC-05B:]\label{OSUC-05B-DEF} Just as you received it, you are going to
distribute an unmodified open source library, code snippet, module, or plugin to
3rd parties -- in the form of binaries. In this act of distribution, you do not
combine this library, code snippet, module, or plugin with other software
components in the sense of software development (= \textit{snimoli, unmodified,
independent, 4others, binaries}). To see the \textit{specific, license
fulfilling to-do lists} jump to the following pages:
  \begin{itemize}
    \item p.\ \pageref{OSUC-05B-AGPL} for the \textbf{AGPL}
      \textit{(= Affero GNU Public License)} 
    \item p.\ \pageref{OSUC-05B-Apache20} for the \textbf{ApL}
      \textit{(= Apache License)}
    \item p.\ \pageref{OSUC-05B-BSD} for the \textbf{BSD} License
      \textit{(= Berkeley Software Distribution)}
    \item p.\ \pageref{OSUC-05B-EPL} for the \textbf{EPL}
      \textit{(= Eclipse Public License)}     
    \item p.\ \pageref{OSUC-05B-EUPL} for the \textbf{EUPL}
      \textit{(= European Union Public License)} 
    \item p.\ \pageref{OSUC-05B-GPL} for the \textbf{GPL}
       \textit{(= GNU Public License)} 
    \item p.\ \pageref{OSUC-05B-LGPL} for the \textbf{LGPL}
      \textit{(= Lesser GNU Public License)}           
    \item p.\ \pageref{OSUC-05B-MIT} for the \textbf{MIT} License
       \textit{(= Massachusetts Institute of Technology)} 
    \item p.\ \pageref{OSUC-05B-MPL} for \textbf{MPL}
      \textit{(= Mozilla Public License)}     
    \item p.\ \pageref{OSUC-05B-MS-PL} for the \textbf{MS-PL}
      \textit{(= Microsoft Public License)} 
    \item p.\ \pageref{OSUC-05B-PGL} for the \textbf{PGL}
      \textit{(= Postgres License)} 
    \item p.\ \pageref{OSUC-05B-PHP} for the \textbf{PHP} License 
  \end{itemize}


\item[OSUC-06:]\label{OSUC-06-DEF} Only for yourself and just as you received
it, you are going to combine an unmodified open source library, code snippet,
module, or plugin into a larger software unit as one of its parts. (=
\textit{snimoli, unmodified, embedded, 4yourself}).
To see the \textit{specific, license fulfilling to-do lists} jump to the
following pages:
   \begin{itemize}
    \item p.\ \pageref{OSUC-06-AGPL} for the \textbf{AGPL}
      \textit{(= Affero GNU Public License)} 
    \item p.\ \pageref{OSUC-06-Apache20} for the \textbf{ApL}
      \textit{(= Apache License)}
    \item p.\ \pageref{OSUC-06-BSD} for the \textbf{BSD} License
      \textit{(= Berkeley Software Distribution)}
    \item p.\ \pageref{OSUC-06-EPL} for the \textbf{EPL}
      \textit{(= Eclipse Public License)}     
    \item p.\ \pageref{OSUC-06-EUPL} for the \textbf{EUPL}
      \textit{(= European Union Public License)} 
    \item p.\ \pageref{OSUC-06-GPL} for the \textbf{GPL}
       \textit{(= GNU Public License)} 
    \item p.\ \pageref{OSUC-06-LGPL} for the \textbf{LGPL}
      \textit{(= Lesser GNU Public License)}           
    \item p.\ \pageref{OSUC-06-MIT} for the \textbf{MIT} License
       \textit{(= Massachusetts Institute of Technology)} 
    \item p.\ \pageref{OSUC-06-MPL} for \textbf{MPL}
      \textit{(= Mozilla Public License)}     
    \item p.\ \pageref{OSUC-06-MS-PL} for the \textbf{MS-PL}
      \textit{(= Microsoft Public License)} 
    \item p.\ \pageref{OSUC-06-PGL} for the \textbf{PGL}
      \textit{(= Postgres License)} 
    \item p.\ \pageref{OSUC-06-PHP} for the \textbf{PHP} License 
  \end{itemize}

\item[OSUC-07S:]\label{OSUC-07S-DEF} Just as you received it and before you will
distribute it to 3rd parties -- in the form of binaries and together with the
larger software unit, you are going to combine and embed an unmodified open
source library, code snippet, module, or plugin into a larger software unit in
the sense of software development (= \textit{snimoli, unmodified, embedded,
4others, sources}). To see the \textit{specific, license fulfilling to-do lists}
jump to the following pages:
   \begin{itemize}
    \item p.\ \pageref{OSUC-07S-AGPL} for the \textbf{AGPL}
      \textit{(= Affero GNU Public License)} 
    \item p.\ \pageref{OSUC-07S-Apache20} for the \textbf{ApL}
      \textit{(= Apache License)}
    \item p.\ \pageref{OSUC-07S-BSD} for the \textbf{BSD} License
      \textit{(= Berkeley Software Distribution)}
    \item p.\ \pageref{OSUC-07S-EPL} for the \textbf{EPL}
      \textit{(= Eclipse Public License)}     
    \item p.\ \pageref{OSUC-07S-EUPL} for the \textbf{EUPL}
      \textit{(= European Union Public License)} 
    \item p.\ \pageref{OSUC-07S-GPL} for the \textbf{GPL}
       \textit{(= GNU Public License)} 
    \item p.\ \pageref{OSUC-07S-LGPL} for the \textbf{LGPL}
      \textit{(= Lesser GNU Public License)}           
    \item p.\ \pageref{OSUC-07S-MIT} for the \textbf{MIT} License
       \textit{(= Massachusetts Institute of Technology)} 
    \item p.\ \pageref{OSUC-07S-MPL} for \textbf{MPL}
      \textit{(= Mozilla Public License)}     
    \item p.\ \pageref{OSUC-07S-MS-PL} for the \textbf{MS-PL}
      \textit{(= Microsoft Public License)} 
    \item p.\ \pageref{OSUC-07S-PGL} for the \textbf{PGL}
      \textit{(= Postgres License)} 
    \item p.\ \pageref{OSUC-07S-PHP} for the \textbf{PHP} License 
  \end{itemize}

\item[OSUC-07B:]\label{OSUC-07B-DEF} Just as you received it and before you will
distribute it to 3rd parties -- in the form of binaries and together with the
larger software unit, you are going to combine and embed an unmodified open
source library, code snippet, module, or plugin into a larger software unit in
the sense of software development (= \textit{snimoli, unmodified, embedded,
4others, binaries}). To see the \textit{specific, license fulfilling to-do
lists} jump to the following pages:
   \begin{itemize}
    \item p.\ \pageref{OSUC-07B-AGPL} for the \textbf{AGPL}
      \textit{(= Affero GNU Public License)} 
    \item p.\ \pageref{OSUC-07B-Apache20} for the \textbf{ApL}
      \textit{(= Apache License)}
    \item p.\ \pageref{OSUC-07B-BSD} for the \textbf{BSD} License
      \textit{(= Berkeley Software Distribution)}
    \item p.\ \pageref{OSUC-07B-EPL} for the \textbf{EPL}
      \textit{(= Eclipse Public License)}     
    \item p.\ \pageref{OSUC-07B-EUPL} for the \textbf{EUPL}
      \textit{(= European Union Public License)} 
    \item p.\ \pageref{OSUC-07B-GPL} for the \textbf{GPL}
       \textit{(= GNU Public License)} 
    \item p.\ \pageref{OSUC-07B-LGPL} for the \textbf{LGPL}
      \textit{(= Lesser GNU Public License)}           
    \item p.\ \pageref{OSUC-07B-MIT} for the \textbf{MIT} License
       \textit{(= Massachusetts Institute of Technology)} 
    \item p.\ \pageref{OSUC-07B-MPL} for \textbf{MPL}
      \textit{(= Mozilla Public License)}     
    \item p.\ \pageref{OSUC-07B-MS-PL} for the \textbf{MS-PL}
      \textit{(= Microsoft Public License)} 
    \item p.\ \pageref{OSUC-07B-PGL} for the \textbf{PGL}
      \textit{(= Postgres License)} 
    \item p.\ \pageref{OSUC-07B-PHP} for the \textbf{PHP} License 
  \end{itemize}

\item[OSUC-08S:]\label{OSUC-08S-DEF} Before you will distribute it, you are going
to modify an open source library, code snippet, module, or plugin to 3rd parties
 -- in the form of sources But you do not combine it with other software
components in the sense of software development (= \textit{snimoli, modified,
independent, 4others, sources}). To see the \textit{specific, license fulfilling
to-do lists} jump to the following pages:
  \begin{itemize}
    \item p.\ \pageref{OSUC-08S-AGPL} for the \textbf{AGPL}
      \textit{(= Affero GNU Public License)} 
    \item p.\ \pageref{OSUC-08S-Apache20} for the \textbf{ApL}
      \textit{(= Apache License)}
    \item p.\ \pageref{OSUC-08S-BSD} for the \textbf{BSD} License
      \textit{(= Berkeley Software Distribution)}
    \item p.\ \pageref{OSUC-08S-EPL} for the \textbf{EPL}
      \textit{(= Eclipse Public License)}     
    \item p.\ \pageref{OSUC-08S-EUPL} for the \textbf{EUPL}
      \textit{(= European Union Public License)} 
    \item p.\ \pageref{OSUC-08S-GPL} for the \textbf{GPL}
       \textit{(= GNU Public License)} 
    \item p.\ \pageref{OSUC-08S-LGPL} for the \textbf{LGPL}
      \textit{(= Lesser GNU Public License)}           
    \item p.\ \pageref{OSUC-08S-MIT} for the \textbf{MIT} License
       \textit{(= Massachusetts Institute of Technology)} 
    \item p.\ \pageref{OSUC-08S-MPL} for \textbf{MPL}
      \textit{(= Mozilla Public License)}     
    \item p.\ \pageref{OSUC-08S-MS-PL} for the \textbf{MS-PL}
      \textit{(= Microsoft Public License)} 
    \item p.\ \pageref{OSUC-08S-PGL} for the \textbf{PGL}
      \textit{(= Postgres License)} 
    \item p.\ \pageref{OSUC-08S-PHP} for the \textbf{PHP} License 
  \end{itemize}

\item[OSUC-08B:]\label{OSUC-08B-DEF} Before you will distribute it, you are going
to modify an open source library, code snippet, module, or plugin to 3rd parties
-- in the form of binaries. But you do not combine it with other software
components in the sense of software development (= \textit{snimoli, modified,
independent, 4others}). To see the \textit{specific, license fulfilling to-do
lists} jump to the following pages:
  \begin{itemize}
    \item p.\ \pageref{OSUC-08B-AGPL} for the \textbf{AGPL}
      \textit{(= Affero GNU Public License)} 
    \item p.\ \pageref{OSUC-08B-Apache20} for the \textbf{ApL}
      \textit{(= Apache License)}
    \item p.\ \pageref{OSUC-08B-BSD} for the \textbf{BSD} License
      \textit{(= Berkeley Software Distribution)}
    \item p.\ \pageref{OSUC-08B-EPL} for the \textbf{EPL}
      \textit{(= Eclipse Public License)}     
    \item p.\ \pageref{OSUC-08B-EUPL} for the \textbf{EUPL}
      \textit{(= European Union Public License)} 
    \item p.\ \pageref{OSUC-08B-GPL} for the \textbf{GPL}
       \textit{(= GNU Public License)} 
    \item p.\ \pageref{OSUC-08B-LGPL} for the \textbf{LGPL}
      \textit{(= Lesser GNU Public License)}           
    \item p.\ \pageref{OSUC-08B-MIT} for the \textbf{MIT} License
       \textit{(= Massachusetts Institute of Technology)} 
    \item p.\ \pageref{OSUC-08B-MPL} for \textbf{MPL}
      \textit{(= Mozilla Public License)}     
    \item p.\ \pageref{OSUC-08B-MS-PL} for the \textbf{MS-PL}
      \textit{(= Microsoft Public License)} 
    \item p.\ \pageref{OSUC-08B-PGL} for the \textbf{PGL}
      \textit{(= Postgres License)} 
    \item p.\ \pageref{OSUC-08B-PHP} for the \textbf{PHP} License 
  \end{itemize}

\item[OSUC-09:]\label{OSUC-09-DEF} Only for yourself, you are going to modify an
open source library, code snippet, module, or plugin, and you will combine it
-- in the sense of software development -- into a larger software unit as one of
its parts. (= \textit{snimoli, modified, embedded, 4yourself}). 
To see the \textit{specific, license fulfilling to-do lists} jump to the
following pages:
  \begin{itemize}
    \item p.\ \pageref{OSUC-09-AGPL} for the \textbf{AGPL}
      \textit{(= Affero GNU Public License)} 
    \item p.\ \pageref{OSUC-09-Apache20} for the \textbf{ApL}
      \textit{(= Apache License)}
    \item p.\ \pageref{OSUC-09-BSD} for the \textbf{BSD} License
      \textit{(= Berkeley Software Distribution)}
    \item p.\ \pageref{OSUC-09-EPL} for the \textbf{EPL}
      \textit{(= Eclipse Public License)}     
    \item p.\ \pageref{OSUC-09-EUPL} for the \textbf{EUPL}
      \textit{(= European Union Public License)} 
    \item p.\ \pageref{OSUC-09-GPL} for the \textbf{GPL}
       \textit{(= GNU Public License)} 
    \item p.\ \pageref{OSUC-09-LGPL} for the \textbf{LGPL}
      \textit{(= Lesser GNU Public License)}           
    \item p.\ \pageref{OSUC-09-MIT} for the \textbf{MIT} License
       \textit{(= Massachusetts Institute of Technology)} 
    \item p.\ \pageref{OSUC-09-MPL} for \textbf{MPL}
      \textit{(= Mozilla Public License)}     
    \item p.\ \pageref{OSUC-09-MS-PL} for the \textbf{MS-PL}
      \textit{(= Microsoft Public License)} 
    \item p.\ \pageref{OSUC-09-PGL} for the \textbf{PGL}
      \textit{(= Postgres License)} 
    \item p.\ \pageref{OSUC-09-PHP} for the \textbf{PHP} License 
  \end{itemize}


\item[OSUC-10S:]\label{OSUC-10S-DEF} Before you will distribute it to 3rd parties
in the form of sources, you are going to modify an open source library, code
snippet, module, or plugin, which you combine with other software components in
the sense of software development (= \textit{snimoli, modified, independent,
4others, sources}). To see the \textit{specific, license fulfilling to-do lists}
jump to the following pages:
  \begin{itemize}
    \item p.\ \pageref{OSUC-10S-AGPL} for the \textbf{AGPL}
      \textit{(= Affero GNU Public License)} 
    \item p.\ \pageref{OSUC-10S-Apache20} for the \textbf{ApL}
      \textit{(= Apache License)}
    \item p.\ \pageref{OSUC-10S-BSD} for the \textbf{BSD} License
      \textit{(= Berkeley Software Distribution)}
    \item p.\ \pageref{OSUC-10S-EPL} for the \textbf{EPL}
      \textit{(= Eclipse Public License)}     
    \item p.\ \pageref{OSUC-10S-EUPL} for the \textbf{EUPL}
      \textit{(= European Union Public License)} 
    \item p.\ \pageref{OSUC-10S-GPL} for the \textbf{GPL}
       \textit{(= GNU Public License)} 
    \item p.\ \pageref{OSUC-10S-LGPL} for the \textbf{LGPL}
      \textit{(= Lesser GNU Public License)}           
    \item p.\ \pageref{OSUC-10S-MIT} for the \textbf{MIT} License
       \textit{(= Massachusetts Institute of Technology)} 
    \item p.\ \pageref{OSUC-10S-MPL} for \textbf{MPL}
      \textit{(= Mozilla Public License)}     
    \item p.\ \pageref{OSUC-10S-MS-PL} for the \textbf{MS-PL}
      \textit{(= Microsoft Public License)} 
    \item p.\ \pageref{OSUC-10S-PGL} for the \textbf{PGL}
      \textit{(= Postgres License)} 
    \item p.\ \pageref{OSUC-10S-PHP} for the \textbf{PHP} License 
  \end{itemize}

\item[OSUC-10B:]\label{OSUC-10B-DEF} Before you will distribute it to 3rd parties
in the form of binaries, you are going to modify an open source library, code
snippet, module, or plugin, which you combine with other software components in
the sense of software development (= \textit{snimoli, modified, independent,
4others, binaries}). To see the \textit{specific, license fulfilling to-do
lists} jump to the following pages:
  \begin{itemize}
    \item p.\ \pageref{OSUC-10B-AGPL} for the \textbf{AGPL}
      \textit{(= Affero GNU Public License)} 
    \item p.\ \pageref{OSUC-10B-Apache20} for the \textbf{ApL}
      \textit{(= Apache License)}
    \item p.\ \pageref{OSUC-10B-BSD} for the \textbf{BSD} License
      \textit{(= Berkeley Software Distribution)}
    \item p.\ \pageref{OSUC-10B-EPL} for the \textbf{EPL}
      \textit{(= Eclipse Public License)}     
    \item p.\ \pageref{OSUC-10B-EUPL} for the \textbf{EUPL}
      \textit{(= European Union Public License)} 
    \item p.\ \pageref{OSUC-10B-GPL} for the \textbf{GPL}
       \textit{(= GNU Public License)} 
    \item p.\ \pageref{OSUC-10B-LGPL} for the \textbf{LGPL}
      \textit{(= Lesser GNU Public License)}           
    \item p.\ \pageref{OSUC-10B-MIT} for the \textbf{MIT} License
       \textit{(= Massachusetts Institute of Technology)} 
    \item p.\ \pageref{OSUC-10B-MPL} for \textbf{MPL}
      \textit{(= Mozilla Public License)}     
    \item p.\ \pageref{OSUC-10B-MS-PL} for the \textbf{MS-PL}
      \textit{(= Microsoft Public License)} 
    \item p.\ \pageref{OSUC-10B-PGL} for the \textbf{PGL}
      \textit{(= Postgres License)} 
    \item p.\ \pageref{OSUC-10B-PHP} for the \textbf{PHP} License 
  \end{itemize}

\end{description}

%\bibliography{../../../bibfiles/oscResourcesEn}
