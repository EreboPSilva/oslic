% Telekom osCompendium 'for being included' snippet template
%
% (c) Karsten Reincke, Deutsche Telekom AG, Darmstadt 2011
%
% This LaTeX-File is licensed under the Creative Commons Attribution-ShareAlike
% 3.0 Germany License (http://creativecommons.org/licenses/by-sa/3.0/de/): Feel
% free 'to share (to copy, distribute and transmit)' or 'to remix (to adapt)'
% it, if you '... distribute the resulting work under the same or similar
% license to this one' and if you respect how 'you must attribute the work in
% the manner specified by the author ...':
%
% In an internet based reuse please link the reused parts to www.telekom.com and
% mention the original authors and Deutsche Telekom AG in a suitable manner. In
% a paper-like reuse please insert a short hint to www.telekom.com and to the
% original authors and Deutsche Telekom AG into your preface. For normal
% quotations please use the scientific standard to cite.
%
% [ Framework derived from 'mind your Scholar Research Framework' 
%   mycsrf (c) K. Reincke 2012 CC BY 3.0  http://mycsrf.fodina.de/ ]
%


%% use all entries of the bibliography
%\nocite{*}

\chapter{Open Source Use Cases: Find the License Fulfilling To-do Lists}\label{sec:OSUCfinder}

\footnotesize
\begin{quote}\itshape
This chapter offers the \emph{Open Source Use Case Finder}: First, it presents
a form to gather the specifying information. Second, it offers a tree which
can easily be transversed by the gathered information. And finally it contains
the list of \emph{Open Source Use Cases}. Each leaf of the tree leads to one
\emph{Open Source Use Case} which itself then refers to the specific license
fulfilling to-do lists.
\end{quote}
\normalsize{}

\section{A standard form for gathering the relevant information}
\label{OSLiCStandardFormForGatheringInformation}
 
\begin{small}
\begin{tabular}[h]{|l|l|l|l|}
\hline 
Class & Questions & Answers\\
\hline 
  Type
  & \parbox[c][2.6cm][c]{9.4cm}{
    \textit{Is the Open Source Software you want to use a software library
    in the broadest sense (an includable code snippet, a linkable module or
    library, or a loadable plugin) [=snimoli], or is it an autonomous
    program, application or server which can be executed or processed
    [=proapse]?}} & \parbox{10em}{ 
      $\square$\hspace{1em}proapse\\ 
      $\square$\hspace{1em}snimoli}
    \\
\hline 
  State & 
  \parbox[c][1.6cm][c]{9.4cm}{
  \textit{Do you want to leave your Open Source Software as you have
  got it, or do you want to modify it before using and/or distributing it to 3rd
  parties?}} &
  \parbox{10em}{
    $\square$\hspace{1em}unmodified\\
    $\square$\hspace{1em}modified} \\
\hline 
  Context & 
  \parbox[c][2cm][c]{9.4cm}{
  \textit{Are you using your Open Source Software as an au\-to\-no\-mous piece
  of software [=independent], or are you using it as an embedded part or component
  of a larger, more complex piece of software [=embedded]?}} &
  \parbox{10em}{ $\square$\hspace{1em}independent\\
    $\square$\hspace{1em}embedded}\\
\hline 
  Recipient & 
  \parbox[c][1.6cm][c]{9.4cm}{
  \textit{Are you are going to use the received Open Source Software only for
  yourself [=4yourself], or do you plan to (re)distribute it (also) to third
  parties [=4others]?}}
  & \parbox{10em}{
    $\square$\hspace{1em}4yourself\\
    $\square$\hspace{1em}4others}\\
\hline 
  Mode & 
  \parbox[c][2.6cm][c]{9.4cm}{
  \textit{Are you going to combine the received Open Source Software with other
  software components by linking all together statically, by linking them
  dynamically, or by textually including (parts of) the Open Source Software
  into your larger unit?}} &
  \parbox{10em}{
    $\square$\hspace{1em}statically linked\\   
    $\square$\hspace{1em}dynamically linked\\
    $\square$\hspace{1em}textually included}\\
\hline 
\hline
\end{tabular}
\end{small}

As discussed earlier, there are of course some invalid
combinations\footnote{type::proapse excludes state::embedded;
recipient::4yourself excludes the combination with state::independent and
type::snimoli; any value of class 'mode' implies state::embedded [for details
see page \pageref{InvalidFinderTokenCombinations}]. If you have gathered one of
these invalid combinations, please check the corresponding explanations}. Here
are some extra explanations about each class:

\begin{description}
\item[Type:] A piece of (Open Source) Software shall be viewed as a program, an
application, or a server, if you can start its binary form with your normal
program launcher, or (in case of a text file which still must be interpreted by
an interpreter like php, perl, bash etc.) if you can start an interpreter taking
the file as one of its' arguments. \item[State:] You modify Open Source Software
if you expand, reduce or modify at least one of the received software files, and
- in case of dealing with binary object code - if you (re)compile and (re)link
the modified software to a new binary file. If you only modify configuration
files, you do not modify the Open Source Software.
\item[Context:] You use Open Source Software embedded into a larger unit, if one
of your files of the larger unit contains a verbatim or modified copy (i.e. a
snippet) of the received Open Source Software, or if the larger unit contains an
include statement referring to a file of the received Open Source Software, or
if your development environment contains a compiler or linker directive
referring to the received Open Source Software.
\item[Recipient] You use the received Open Source Software only for yourself, if
you as person do not pass it to other persons, or if you - as a member of a
specific development group - pass it only to the other members of your
development group. But if you store Open Source Software on any device such as a
mobile phone, an USB stick, etc. or if you attach it to any transport
medium like email etc. and if you then sell, give away, or simply send this
device or transport medium to anyone (other than a direct member
of your development group) then you indeed handover the Open Source Software to
third parties\footnote{Please remember that - at least in Germany - there are
opinions that even handing over software to another legal entity or department
of the same company is also a kind of distribution. It is always safest
to take the broadest possible meaning of distributing or handing over.}.
\item[Mode] is defined by standard terms of the software development.
\end{description}

\section{The taxonomic Open Source Use Case Finder}

Now, after having gathered the necessary information, determine your specific
Open Source Use Case by traversing the following tree and its corresponding
branches:

\begin{footnotesize}

\pstree[treemode=R,levelsep=*0.2,treesep=0.2]{\Toval{OSS}}{
  \pstree{
    \Tr{\Ovalbox{\shortstack{type:\\\textbf{\textit{proapse}}\\
      \tiny (= Program,\\\tiny Application,\\\tiny Server)      
      }   
    }}
    
  }{
    \pstree{
      \Tr{\Ovalbox{\shortstack{state:\\\textbf{\textit{unmodified}}}}}
    }{
      \pstree{
        \Tr{\Ovalbox{\shortstack{context:\\\textbf{\textit{independent}}}}}
      }{
        
        \pstree{
          \Tr{\Ovalbox{\shortstack{recipient:\\\textbf{\textit{4yourself}}}}}
        }{
          \Tr[edge=none]{\begin{minipage}[b][2em][c]{6em} 
              $\Rightarrow$ OSUC-01\\
              \textit{(see p. \pageref{OSUC-01-DEF})}\end{minipage} } 
          }
        
        \pstree{
          \Tr{\Ovalbox{\shortstack{recipient:\\\textbf{\textit{4others}}}}}
        }{
          \Tr[edge=none]{\begin{minipage}[b][2em][c]{6em} 
              $\Rightarrow$ OSUC-02\\
              \textit{(see p. \pageref{OSUC-02-DEF})}\end{minipage} } 
          }        
      }
    }
    \pstree{
      \Tr{\Ovalbox{\shortstack{state:\\\textbf{\textit{modified}}}}}
    }{
      \pstree{
        \Tr{\Ovalbox{\shortstack{context:\\\textbf{\textit{independent}}}}}
      }{
        \pstree{
          \Tr{\Ovalbox{\shortstack{recipient:\\\textbf{\textit{4yourself}}}}}
        }{
          \Tr[edge=none]{\begin{minipage}[b][2em][c]{6em} 
              $\Rightarrow$ OSUC-03\\
              \textit{(see p. \pageref{OSUC-03-DEF})}\end{minipage} } 
          }
        
        \pstree{
          \Tr{\Ovalbox{\shortstack{recipient:\\\textbf{\textit{4others}}}}}
        }{
          \Tr[edge=none]{\begin{minipage}[b][2em][c]{6em} 
              $\Rightarrow$ OSUC-04\\
              \textit{(see p. \pageref{OSUC-04-DEF})}\end{minipage} } 
          }        
      }
    }
  }
  \pstree{
    \Tr{\Ovalbox{
      \shortstack{type:\\\textbf{\textit{snimoli}}\\
      \tiny (= Snippet,\\\tiny Module,\\\tiny Plugin,\\\tiny Library)            
      }
    }}
  }{
    \pstree{
      \Tr{\Ovalbox{\shortstack{state:\\\textbf{\textit{unmodified}}}}}
    }{
      \pstree{
        \Tr{\Ovalbox{\shortstack{context:\\\textbf{\textit{independent}}}}}
      }{
      
        \pstree{
          \Tr{\Ovalbox{\shortstack{recipient:\\\textbf{\textit{4others}}}}}
        }{
          \Tr[edge=none]{\begin{minipage}[b][2em][c]{6em} 
              $\Rightarrow$ OSUC-05\\
              \textit{(see p. \pageref{OSUC-05-DEF})}\end{minipage} } 
          }        
      
      }
      \pstree{
        \Tr{\Ovalbox{\shortstack{context:\\\textbf{\textit{embedded}}}}}
      }{
        \pstree{
          \Tr{
            \Ovalbox{\shortstack{recipient:\\\textbf{\textit{4yourself}}}}
           }
        }{
          \pstree{
            \Tr{
              \begin{tiny}
              \Ovalbox{\shortstack{mode:\\\textit{statically linked}}}
              \end{tiny}
            }
          }{
            \Tr[edge=none]{\begin{minipage}[b][2em][c]{7em} 
              $\Rightarrow$ OSUC-06a\\
              \textit{(see p. \pageref{OSUC-06-DEF})}\end{minipage} } 
          }        
        
          \pstree{
            \Tr{
              \begin{tiny}
              \Ovalbox{\shortstack{mode:\\\textit{dynamically linked}}}
              \end{tiny}
            }
          }{
            \Tr[edge=none]{\begin{minipage}[b][2em][c]{7em} 
              $\Rightarrow$ OSUC-06b\\
              \textit{(see p. \pageref{OSUC-06-DEF})}\end{minipage} } 
          }        
 
          \pstree{
            \Tr{
              \begin{tiny}
              \Ovalbox{\shortstack{mode:\\\textit{textually included}}}
              \end{tiny}
            }
          }{
            \Tr[edge=none]{\begin{minipage}[b][2em][c]{7em} 
              $\Rightarrow$ OSUC-06c\\
              \textit{(see p. \pageref{OSUC-06-DEF})}\end{minipage} } 
          }        
        
        }
        \pstree{
          \Tr{\Ovalbox{\shortstack{recipient:\\\textbf{\textit{4others}}}}}
        }{
          \pstree{
            \Tr{
              \begin{tiny}
              \Ovalbox{\shortstack{mode:\\\textit{statically linked}}}
              \end{tiny}
            }
          }{
            \Tr[edge=none]{\begin{minipage}[b][2em][c]{7em} 
              $\Rightarrow$ OSUC-07a\\
              \textit{(see p. \pageref{OSUC-07-DEF})}\end{minipage} } 
          }        
        
          \pstree{
            \Tr{
              \begin{tiny}
              \Ovalbox{\shortstack{mode:\\\textit{dynamically linked}}}
              \end{tiny}
            }
          }{
            \Tr[edge=none]{\begin{minipage}[b][2em][c]{7em} 
              $\Rightarrow$ OSUC-07b\\
              \textit{(see p. \pageref{OSUC-07-DEF})}\end{minipage} } 
          }        
 
          \pstree{
            \Tr{
              \begin{tiny}
              \Ovalbox{\shortstack{mode:\\\textit{textually included}}}
              \end{tiny}
             }
          }{
            \Tr[edge=none]{\begin{minipage}[b][2em][c]{7em} 
              $\Rightarrow$ OSUC-07c\\
              \textit{(see p. \pageref{OSUC-07-DEF})}\end{minipage} } 
          }        
        }
      }
    }
    \pstree{
      \Tr{\Ovalbox{\shortstack{state:\\\textbf{\textit{modified}}}}}
    }{
      \pstree{
        \Tr{\Ovalbox{\shortstack{context:\\\textbf{\textit{independent}}}}}
      }{
        \pstree{
          \Tr{\Ovalbox{\shortstack{recipient:\\\textbf{\textit{4others}}}}}
        }{
          \Tr[edge=none]{\begin{minipage}[b][2em][c]{6em} 
              $\Rightarrow$ OSUC-08\\
              \textit{(see p. \pageref{OSUC-08-DEF})}\end{minipage} } 
          }        
      }
      \pstree{
        \Tr{\Ovalbox{\shortstack{context:\\\textbf{\textit{embedded}}}}}
      }{
        \pstree{
          \Tr{\Ovalbox{\shortstack{recipient:\\\textbf{\textit{4yourself}}}}}
        }{
          \pstree{
            \Tr{
              \begin{tiny}
              \Ovalbox{\shortstack{mode:\\\textit{statically linked}}}
              \end{tiny}
            }
          }{
            \Tr[edge=none]{\begin{minipage}[b][2em][c]{7em} 
              $\Rightarrow$ OSUC-09a\\
              \textit{(see p. \pageref{OSUC-09-DEF})}\end{minipage} } 
          }        
        
          \pstree{
            \Tr{
              \begin{tiny}
              \Ovalbox{\shortstack{mode:\\\textit{dynamically linked}}}
              \end{tiny}
            }
          }{
            \Tr[edge=none]{\begin{minipage}[b][2em][c]{7em} 
              $\Rightarrow$ OSUC-09b\\
              \textit{(see p. \pageref{OSUC-09-DEF})}\end{minipage} } 
          }        
 
          \pstree{
            \Tr{
              \begin{tiny}
              \Ovalbox{\shortstack{mode:\\\textit{textually included}}}
              \end{tiny}
             }
          }{
            \Tr[edge=none]{\begin{minipage}[b][2em][c]{7em} 
              $\Rightarrow$ OSUC-09c\\
              \textit{(see p. \pageref{OSUC-09-DEF})}\end{minipage} } 
          }        
        }
        \pstree{
          \Tr{\Ovalbox{\shortstack{recipient:\\\textbf{\textit{4others}}}}}
        }{
          \pstree{
            \Tr{
              \begin{tiny}
              \Ovalbox{\shortstack{mode:\\\textit{\tiny statically linked}}}
              \end{tiny}
            }
          }{
            \Tr[edge=none]{\begin{minipage}[b][2em][c]{7em} 
              $\Rightarrow$ OSUC-10a\\
              \textit{(see p. \pageref{OSUC-10-DEF})}\end{minipage} } 
          }        
        
          \pstree{
            \Tr{
              \begin{tiny}
              \Ovalbox{\shortstack{mode:\\\textit{dynamically linked}}}
              \end{tiny}              
            }
          }{
            \Tr[edge=none]{\begin{minipage}[b][2em][c]{7em} 
              $\Rightarrow$ OSUC-10b\\
              \textit{(see p. \pageref{OSUC-10-DEF})}\end{minipage} } 
          }        
 
          \pstree{
            \Tr{
              \begin{tiny}
              \Ovalbox{\shortstack{mode:\\\textit{textually included}}} 
              \end{tiny}
             }
          }{
            \Tr[edge=none]{\begin{minipage}[b][2em][c]{7em} 
              $\Rightarrow$ OSUC-10c\\
              \textit{(see p. \pageref{OSUC-10-DEF})}\end{minipage} } 
          }        
        }
      }
    }
  }
}
\end{footnotesize}
\label{OSLiCUseCaseFinder}



\section{The Open Source Use Cases and their links to the to-do lists}

On the following pages, each \textbf{O}pen \textbf{S}ource \textbf{U}se
\textbf{C}ase is textually specified one more time and added by a list of page
numbers. Each of these pages hints to that license-specific to-do list whose
items together offer a processable way for acting according to the license under
the circumstances of the described \textbf{O}pen \textbf{S}ource \textbf{U}se
\textbf{C}ase.

\begin{description}
\label{OSUCList}
\item[OSUC-01:]\label{OSUC-01-DEF}
Only for yourself, you are using an unmodified Open Source program, application,
or server - just as you received it. You are not going to combine it with other
components in the sense of software development (= \textit{proapse, unmodified,
independent, 4yourself}). 
To see the \textit{specific, license fulfilling to-do lists} jump to the
following pages:
  \begin{itemize}
    \item p. \pageref{OSUC-01-AGPL} for the \textbf{AGPL}
      \textit{(= Affero GNU Pulic License)} 
    \item p. \pageref{OSUC-01-Apache20} for the \textbf{APL}
      \textit{(= Apache License)}
    \item p. \pageref{OSUC-01-BSD} for the \textbf{BSD} License
      \textit{(= Berkeley Software Distribution)}
    \item p. \pageref{OSUC-01-EPL} for the \textbf{EPL}
      \textit{(= Eclipse Pulic License)}     
    \item p. \pageref{OSUC-01-EUPL} for the \textbf{EUPL}
      \textit{(= European Pulic License)} 
    \item p. \pageref{OSUC-01-GPL} for the \textbf{GPL}
       \textit{(= GNU Pulic License)} 
    \item p. \pageref{OSUC-01-LGPL} for the \textbf{LGPL}
      \textit{(= Lesser GNU Pulic License)}           
    \item p. \pageref{OSUC-01-MIT} for the \textbf{MIT} License
       \textit{(= Massachusetts Institute of Technology)} 
    \item p. \pageref{OSUC-01-MPL} for \textbf{MPL}
      \textit{(= Mozilla Pulic License)}     
    \item p. \pageref{OSUC-01-MsPL} for the \textbf{MSPL}
      \textit{(= Microsoft Public License)} 
    \item p. \pageref{OSUC-01-PGL} for the \textbf{PGL}
      \textit{(= Postgres License)} 
    \item p. \pageref{OSUC-01-PHP} for the \textbf{PHP} license 
  \end{itemize}

\item[OSUC-02:]\label{OSUC-02-DEF} Just as you received it, you are going to
distribute an unmodified Open Source program, application, or server to 3rd
parties. In this act of distribution, you do not combine this program,
application, or server with other software components in the sense of software
development (= \textit{proapse, unmodified, independent, 4others}). 
To see the \textit{specific, license fulfilling to-do lists} jump to the
following pages:
   \begin{itemize}
    \item p. \pageref{OSUC-02-AGPL} for the \textbf{AGPL}
      \textit{(= Affero GNU Pulic License)} 
    \item p. \pageref{OSUC-02-Apache20} for the \textbf{APL}
      \textit{(= Apache License)}
    \item p. \pageref{OSUC-02-BSD} for the \textbf{BSD} License
      \textit{(= Berkeley Software Distribution)}
    \item p. \pageref{OSUC-02-EPL} for the \textbf{EPL}
      \textit{(= Eclipse Pulic License)}     
    \item p. \pageref{OSUC-02-EUPL} for the \textbf{EUPL}
      \textit{(= European Pulic License)} 
    \item p. \pageref{OSUC-02-GPL} for the \textbf{GPL}
       \textit{(= GNU Pulic License)} 
    \item p. \pageref{OSUC-02-LGPL} for the \textbf{LGPL}
      \textit{(= Lesser GNU Pulic License)}           
    \item p. \pageref{OSUC-02-MIT} for the \textbf{MIT} License
       \textit{(= Massachusetts Institute of Technology)} 
    \item p. \pageref{OSUC-02-MPL} for \textbf{MPL}
      \textit{(= Mozilla Pulic License)}     
    \item p. \pageref{OSUC-02-MsPL} for the \textbf{MSPL}
      \textit{(= Microsoft Public License)} 
    \item p. \pageref{OSUC-02-PGL} for the \textbf{PGL}
      \textit{(= Postgres License)} 
    \item p. \pageref{OSUC-02-PHP} for the \textbf{PHP} license 
  \end{itemize}

\item[OSUC-03:]\label{OSUC-03-DEF} Only for yourself, you are modifying a
received Open Source program, application, or server, before you are using it.
But you do not combine it with other components in the sense of software
development (= \textit{proapse, modified, independent, 4yourself}).
To see the \textit{specific, license fulfilling to-do lists} jump to the
following pages:
   \begin{itemize}
    \item p. \pageref{OSUC-03-AGPL} for the \textbf{AGPL}
      \textit{(= Affero GNU Pulic License)} 
    \item p. \pageref{OSUC-03-Apache20} for the \textbf{APL}
      \textit{(= Apache License)}
    \item p. \pageref{OSUC-03-BSD} for the \textbf{BSD} License
      \textit{(= Berkeley Software Distribution)}
    \item p. \pageref{OSUC-03-EPL} for the \textbf{EPL}
      \textit{(= Eclipse Pulic License)}     
    \item p. \pageref{OSUC-03-EUPL} for the \textbf{EUPL}
      \textit{(= European Pulic License)} 
    \item p. \pageref{OSUC-03-GPL} for the \textbf{GPL}
       \textit{(= GNU Pulic License)} 
    \item p. \pageref{OSUC-03-LGPL} for the \textbf{LGPL}
      \textit{(= Lesser GNU Pulic License)}           
    \item p. \pageref{OSUC-03-MIT} for the \textbf{MIT} License
       \textit{(= Massachusetts Institute of Technology)} 
    \item p. \pageref{OSUC-03-MPL} for \textbf{MPL}
      \textit{(= Mozilla Pulic License)}     
    \item p. \pageref{OSUC-03-MsPL} for the \textbf{MSPL}
      \textit{(= Microsoft Public License)} 
    \item p. \pageref{OSUC-03-PGL} for the \textbf{PGL}
      \textit{(= Postgres License)} 
    \item p. \pageref{OSUC-03-PHP} for the \textbf{PHP} license 
  \end{itemize}

\item[OSUC-04:]\label{OSUC-04-DEF} You are going to modify a received Open
Source program, application, or server, before you distribute it to 3rd parties.
But you do not combine this modified program, application, or server with other
software components in the sense of software development (= \textit{proapse,
modified, independent, 4others}).
To see the \textit{specific, license fulfilling to-do lists} jump to the
following pages:
  \begin{itemize}
    \item p. \pageref{OSUC-04-AGPL} for the \textbf{AGPL}
      \textit{(= Affero GNU Pulic License)} 
    \item p. \pageref{OSUC-04-Apache20} for the \textbf{APL}
      \textit{(= Apache License)}
    \item p. \pageref{OSUC-04-BSD} for the \textbf{BSD} License
      \textit{(= Berkeley Software Distribution)}
    \item p. \pageref{OSUC-04-EPL} for the \textbf{EPL}
      \textit{(= Eclipse Pulic License)}     
    \item p. \pageref{OSUC-04-EUPL} for the \textbf{EUPL}
      \textit{(= European Pulic License)} 
    \item p. \pageref{OSUC-04-GPL} for the \textbf{GPL}
       \textit{(= GNU Pulic License)} 
    \item p. \pageref{OSUC-04-LGPL} for the \textbf{LGPL}
      \textit{(= Lesser GNU Pulic License)}           
    \item p. \pageref{OSUC-04-MIT} for the \textbf{MIT} License
       \textit{(= Massachusetts Institute of Technology)} 
    \item p. \pageref{OSUC-04-MPL} for \textbf{MPL}
      \textit{(= Mozilla Pulic License)}     
    \item p. \pageref{OSUC-04-MsPL} for the \textbf{MSPL}
      \textit{(= Microsoft Public License)} 
    \item p. \pageref{OSUC-04-PGL} for the \textbf{PGL}
      \textit{(= Postgres License)} 
    \item p. \pageref{OSUC-04-PHP} for the \textbf{PHP} license 
  \end{itemize}

\item[OSUC-05:]\label{OSUC-05-DEF} Just as you received it, you are going to
distribute an unmodified Open Source library, code snippet, module, or plugin to
3rd parties. In this act of distribution, you do not combine this library, code
snippet, module, or plugin with other software components in the sense of
software development (= \textit{snimoli, unmodified, independent, 4others}).
To see the \textit{specific, license fulfilling to-do lists} jump to the
following pages:
  \begin{itemize}
    \item p. \pageref{OSUC-05-AGPL} for the \textbf{AGPL}
      \textit{(= Affero GNU Pulic License)} 
    \item p. \pageref{OSUC-05-Apache20} for the \textbf{APL}
      \textit{(= Apache License)}
    \item p. \pageref{OSUC-05-BSD} for the \textbf{BSD} License
      \textit{(= Berkeley Software Distribution)}
    \item p. \pageref{OSUC-05-EPL} for the \textbf{EPL}
      \textit{(= Eclipse Pulic License)}     
    \item p. \pageref{OSUC-05-EUPL} for the \textbf{EUPL}
      \textit{(= European Pulic License)} 
    \item p. \pageref{OSUC-05-GPL} for the \textbf{GPL}
       \textit{(= GNU Pulic License)} 
    \item p. \pageref{OSUC-05-LGPL} for the \textbf{LGPL}
      \textit{(= Lesser GNU Pulic License)}           
    \item p. \pageref{OSUC-05-MIT} for the \textbf{MIT} License
       \textit{(= Massachusetts Institute of Technology)} 
    \item p. \pageref{OSUC-05-MPL} for \textbf{MPL}
      \textit{(= Mozilla Pulic License)}     
    \item p. \pageref{OSUC-05-MsPL} for the \textbf{MSPL}
      \textit{(= Microsoft Public License)} 
    \item p. \pageref{OSUC-05-PGL} for the \textbf{PGL}
      \textit{(= Postgres License)} 
    \item p. \pageref{OSUC-05-PHP} for the \textbf{PHP} license 
  \end{itemize}

\item[OSUC-06:]\label{OSUC-06-DEF} Only for yourself and just as you received
it, you are going to combine an unmodified Open Source library, code snippet,
module, or plugin into a larger software unit as one of its parts. (=
\textit{snimoli, unmodified, embedded, 4yourself}).
To see the \textit{specific, license fulfilling to-do lists} jump to the
following pages:
   \begin{itemize}
    \item p. \pageref{OSUC-06-AGPL} for the \textbf{AGPL}
      \textit{(= Affero GNU Pulic License)} 
    \item p. \pageref{OSUC-06-Apache20} for the \textbf{APL}
      \textit{(= Apache License)}
    \item p. \pageref{OSUC-06-BSD} for the \textbf{BSD} License
      \textit{(= Berkeley Software Distribution)}
    \item p. \pageref{OSUC-06-EPL} for the \textbf{EPL}
      \textit{(= Eclipse Pulic License)}     
    \item p. \pageref{OSUC-06-EUPL} for the \textbf{EUPL}
      \textit{(= European Pulic License)} 
    \item p. \pageref{OSUC-06-GPL} for the \textbf{GPL}
       \textit{(= GNU Pulic License)} 
    \item p. \pageref{OSUC-06-LGPL} for the \textbf{LGPL}
      \textit{(= Lesser GNU Pulic License)}           
    \item p. \pageref{OSUC-06-MIT} for the \textbf{MIT} License
       \textit{(= Massachusetts Institute of Technology)} 
    \item p. \pageref{OSUC-06-MPL} for \textbf{MPL}
      \textit{(= Mozilla Pulic License)}     
    \item p. \pageref{OSUC-06-MsPL} for the \textbf{MSPL}
      \textit{(= Microsoft Public License)} 
    \item p. \pageref{OSUC-06-PGL} for the \textbf{PGL}
      \textit{(= Postgres License)} 
    \item p. \pageref{OSUC-06-PHP} for the \textbf{PHP} license 
  \end{itemize}

\item[OSUC-07:]\label{OSUC-07-DEF} Just as you received it and before you will
distribute it to 3rd parties together with the larger software unit, you
combine an unmodified Open Source library, code snippet, module, or plugin into
a larger software unit in the sense of software development (= \textit{snimoli,
unmodified, embedded, 4others}). 
To see the \textit{specific, license fulfilling to-do lists} jump to the
following pages:
   \begin{itemize}
    \item p. \pageref{OSUC-07-AGPL} for the \textbf{AGPL}
      \textit{(= Affero GNU Pulic License)} 
    \item p. \pageref{OSUC-07-Apache20} for the \textbf{APL}
      \textit{(= Apache License)}
    \item p. \pageref{OSUC-07-BSD} for the \textbf{BSD} License
      \textit{(= Berkeley Software Distribution)}
    \item p. \pageref{OSUC-07-EPL} for the \textbf{EPL}
      \textit{(= Eclipse Pulic License)}     
    \item p. \pageref{OSUC-07-EUPL} for the \textbf{EUPL}
      \textit{(= European Pulic License)} 
    \item p. \pageref{OSUC-07-GPL} for the \textbf{GPL}
       \textit{(= GNU Pulic License)} 
    \item p. \pageref{OSUC-07-LGPL} for the \textbf{LGPL}
      \textit{(= Lesser GNU Pulic License)}           
    \item p. \pageref{OSUC-07-MIT} for the \textbf{MIT} License
       \textit{(= Massachusetts Institute of Technology)} 
    \item p. \pageref{OSUC-07-MPL} for \textbf{MPL}
      \textit{(= Mozilla Pulic License)}     
    \item p. \pageref{OSUC-07-MsPL} for the \textbf{MSPL}
      \textit{(= Microsoft Public License)} 
    \item p. \pageref{OSUC-07-PGL} for the \textbf{PGL}
      \textit{(= Postgres License)} 
    \item p. \pageref{OSUC-07-PHP} for the \textbf{PHP} license 
  \end{itemize}

\item[OSUC-08:]\label{OSUC-08-DEF} Before you will distribute it, you are going
to modify an Open Source library, code snippet, module, or plugin to 3rd
parties, but you do not combine it with other software components in the sense of
software development (= \textit{snimoli, modified, independent, 4others}). 
To see the \textit{specific, license fulfilling to-do lists} jump to the
following pages:
  \begin{itemize}
    \item p. \pageref{OSUC-08-AGPL} for the \textbf{AGPL}
      \textit{(= Affero GNU Pulic License)} 
    \item p. \pageref{OSUC-08-Apache20} for the \textbf{APL}
      \textit{(= Apache License)}
    \item p. \pageref{OSUC-08-BSD} for the \textbf{BSD} License
      \textit{(= Berkeley Software Distribution)}
    \item p. \pageref{OSUC-08-EPL} for the \textbf{EPL}
      \textit{(= Eclipse Pulic License)}     
    \item p. \pageref{OSUC-08-EUPL} for the \textbf{EUPL}
      \textit{(= European Pulic License)} 
    \item p. \pageref{OSUC-08-GPL} for the \textbf{GPL}
       \textit{(= GNU Pulic License)} 
    \item p. \pageref{OSUC-08-LGPL} for the \textbf{LGPL}
      \textit{(= Lesser GNU Pulic License)}           
    \item p. \pageref{OSUC-08-MIT} for the \textbf{MIT} License
       \textit{(= Massachusetts Institute of Technology)} 
    \item p. \pageref{OSUC-08-MPL} for \textbf{MPL}
      \textit{(= Mozilla Pulic License)}     
    \item p. \pageref{OSUC-08-MsPL} for the \textbf{MSPL}
      \textit{(= Microsoft Public License)} 
    \item p. \pageref{OSUC-08-PGL} for the \textbf{PGL}
      \textit{(= Postgres License)} 
    \item p. \pageref{OSUC-08-PHP} for the \textbf{PHP} license 
  \end{itemize}


\item[OSUC-09:]\label{OSUC-09-DEF} Only for yourself, you are going to modify an
Open Source library, code snippet, module, or plugin, before you will combine it
- in the sense of software development - into a larger software unit as one of
its parts . (= \textit{snimoli, modified, embedded, 4yourself}). 
To see the \textit{specific, license fulfilling to-do lists} jump to the
following pages:
  \begin{itemize}
    \item p. \pageref{OSUC-09-AGPL} for the \textbf{AGPL}
      \textit{(= Affero GNU Pulic License)} 
    \item p. \pageref{OSUC-09-Apache20} for the \textbf{APL}
      \textit{(= Apache License)}
    \item p. \pageref{OSUC-09-BSD} for the \textbf{BSD} License
      \textit{(= Berkeley Software Distribution)}
    \item p. \pageref{OSUC-09-EPL} for the \textbf{EPL}
      \textit{(= Eclipse Pulic License)}     
    \item p. \pageref{OSUC-09-EUPL} for the \textbf{EUPL}
      \textit{(= European Pulic License)} 
    \item p. \pageref{OSUC-09-GPL} for the \textbf{GPL}
       \textit{(= GNU Pulic License)} 
    \item p. \pageref{OSUC-09-LGPL} for the \textbf{LGPL}
      \textit{(= Lesser GNU Pulic License)}           
    \item p. \pageref{OSUC-09-MIT} for the \textbf{MIT} License
       \textit{(= Massachusetts Institute of Technology)} 
    \item p. \pageref{OSUC-09-MPL} for \textbf{MPL}
      \textit{(= Mozilla Pulic License)}     
    \item p. \pageref{OSUC-09-MsPL} for the \textbf{MSPL}
      \textit{(= Microsoft Public License)} 
    \item p. \pageref{OSUC-09-PGL} for the \textbf{PGL}
      \textit{(= Postgres License)} 
    \item p. \pageref{OSUC-09-PHP} for the \textbf{PHP} license 
  \end{itemize}


\item[OSUC-10:]\label{OSUC-10-DEF} Before you will distribute it to 3rd parties,
you are going to modify an Open Source library, code snippet, module, or plugin,
and to combine it with other software components in the sense of
software development (= \textit{snimoli, modified, independent, 4others}). 
To see the \textit{specific, license fulfilling to-do lists} jump to the
following pages:
  \begin{itemize}
    \item p. \pageref{OSUC-10-AGPL} for the \textbf{AGPL}
      \textit{(= Affero GNU Pulic License)} 
    \item p. \pageref{OSUC-10-Apache20} for the \textbf{APL}
      \textit{(= Apache License)}
    \item p. \pageref{OSUC-10-BSD} for the \textbf{BSD} License
      \textit{(= Berkeley Software Distribution)}
    \item p. \pageref{OSUC-10-EPL} for the \textbf{EPL}
      \textit{(= Eclipse Pulic License)}     
    \item p. \pageref{OSUC-10-EUPL} for the \textbf{EUPL}
      \textit{(= European Pulic License)} 
    \item p. \pageref{OSUC-10-GPL} for the \textbf{GPL}
       \textit{(= GNU Pulic License)} 
    \item p. \pageref{OSUC-10-LGPL} for the \textbf{LGPL}
      \textit{(= Lesser GNU Pulic License)}           
    \item p. \pageref{OSUC-10-MIT} for the \textbf{MIT} License
       \textit{(= Massachusetts Institute of Technology)} 
    \item p. \pageref{OSUC-10-MPL} for \textbf{MPL}
      \textit{(= Mozilla Pulic License)}     
    \item p. \pageref{OSUC-10-MsPL} for the \textbf{MSPL}
      \textit{(= Microsoft Public License)} 
    \item p. \pageref{OSUC-10-PGL} for the \textbf{PGL}
      \textit{(= Postgres License)} 
    \item p. \pageref{OSUC-10-PHP} for the \textbf{PHP} license 
  \end{itemize}

\end{description}

%\bibliography{../../../bibfiles/oscResourcesEn}
