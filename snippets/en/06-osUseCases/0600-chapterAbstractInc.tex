% Telekom osCompendium 'for beeing included' snippet template
%
% (c) Karsten Reincke, Deutsche Telekom AG, Darmstadt 2011
%
% This LaTeX-File is licensed under the Creative Commons Attribution-ShareAlike
% 3.0 Germany License (http://creativecommons.org/licenses/by-sa/3.0/de/): Feel
% free 'to share (to copy, distribute and transmit)' or 'to remix (to adapt)'
% it, if you '... distribute the resulting work under the same or similar
% license to this one' and if you respect how 'you must attribute the work in
% the manner specified by the author ...':
%
% In an internet based reuse please link the reused parts to www.telekom.com and
% mention the original authors and Deutsche Telekom AG in a suitable manner. In
% a paper-like reuse please insert a short hint to www.telekom.com and to the
% original authors and Deutsche Telekom AG into your preface. For normal
% quotations please use the scientific standard to cite.
%
% [ File structure derived from 'mind your Scholar Research Framework' 
%   mycsrf (c) K. Reincke CC BY 3.0  http://mycsrf.fodina.de/ ]
%

% Chapter Abstract
% ----------------

\footnotesize
\begin{quote}\itshape
This chapter establishes our concept of \emph{Open Source Use Cases} as a
classification system for to-do lists. The conditions of a specific license, in
the context of a par\-ti\-cu\-lar \emph{Open Source Use Case}, will be fulfilled
by following the corresponding to-do list. Additionally this chapter introduces a
taxonomy for these \emph{Open Source Use Cases}. Later on, this taxonomy will
organize the \emph{Open Sourse Use Case Finder}.
\end{quote}
\normalsize{}

