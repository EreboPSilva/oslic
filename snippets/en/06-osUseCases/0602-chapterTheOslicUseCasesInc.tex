% Telekom osCompendium 'for beeing included' snippet template
%
% (c) Karsten Reincke, Deutsche Telekom AG, Darmstadt 2011
%
% This LaTeX-File is licensed under the Creative Commons Attribution-ShareAlike
% 3.0 Germany License (http://creativecommons.org/licenses/by-sa/3.0/de/): Feel
% free 'to share (to copy, distribute and transmit)' or 'to remix (to adapt)'
% it, if you '... distribute the resulting work under the same or similar
% license to this one' and if you respect how 'you must attribute the work in
% the manner specified by the author ...':
%
% In an internet based reuse please link the reused parts to www.telekom.com and
% mention the original authors and Deutsche Telekom AG in a suitable manner. In
% a paper-like reuse please insert a short hint to www.telekom.com and to the
% original authors and Deutsche Telekom AG into your preface. For normal
% quotations please use the scientific standard to cite.
%
% [ Framework derived from 'mind your Scholar Research Framework' 
%   mycsrf (c) K. Reincke 2012 CC BY 3.0  http://mycsrf.fodina.de/ ]
%


%% use all entries of the bibliography
%\nocite{*}

After all these introducing remarks let us summarize our idea: We know that the
right to use Open Source Software depends on doings required by the Open Source
Licenses. In opposite to the commercial licenses you can not buy the right to
use a piece of Open Source software - never. It's part of the Open Source
Definition that the right to use the software may not be sold. The Open Source
Definition states firstly that an Open Source License may \glqq{}\ldots not
restrict any party from selling or giving away the software as a component of (any)
aggregate software distribution\grqq{} and adds secondly that an Open Source
License \glqq{}\ldots shall not require a royalty or other fee for such
sale\grqq{}\footcite[cf.][\nopage wp. §1]{OSI2012a}.

On the other hand it is wrong to simply conclude that you are allowed to use
Open Source Software without any service in return: generally you have to do
something for getting the right to use the software. In other words: Open Source
Software is specified by the idea of 'paying by doing'.

Therefore these Open Source Licenses describe specific circumstances under
which the user must execute some tasks. These set of conditions may be viewed as
triggers for a compliant behaviour. So, if we want to offer license fulfilling
to-do lists, we have to consider these triggers. 

The real challenge is, that such circumstances are not linear and simple. They
contain combinations of (sometimes) context sensitive conditions. These
conditions refer to different class of tokens. Such a class of token might refer
to a feature of the software itself - like being an application or a library. Or
such class of tokens might refer to the circumstances of using the software -
like 'using the software only for yourself' or 'distributing the software also
to third parties'.

In the end we want to determine a set of specific use cases and deliver for each
of these use cases and for each of the considered Open Source Licenses one list
of actions, which fulfill the license in the context of this use case. A use
case is a set of tokens describing the circumstances of the usage. So, in the
beginning we have to specify the relevant classes of tokens. Then we build the
valid combinations of tokens, our use cases. And finally - based on these
specifying tokens - we generate a taxonomy of our use cases for being able to
offer an easily to use reference to the relevant use cases and the
corresponding, the license fulfilling to-do list. 

So, let's now start with the classes of tokens by which the circumstances of
using a piece of Open Source Software can be specified:

\begin{itemize}
  \item Firstly we will consider the type of the licensed software. We will
  discriminate between code snippets, modules, libraries and plugins on the one
  hand and autonomous, processable applications, programs, servers on the other
  hand. Let's name the first set the 'snipmolibs' and the second the
  'progappsers' for indicating, that we are not only talking about libraries and
  applications in the strict, sense. More detailedly spoken, we will ask you,
  whether the Open Source Software, you want to use, is a software library in
  the broadest sense (an includable code snippet, a linkable module or library,
  or a loadable plugin) or whether it is an autonomous application or server
  which can be executed or processed. In the first case, the answer should be
  'it's a snipmolib', in the second 'it's a progappser'.
  \item Secondly we will consider the state of the software: it might be used,
  as one has got it, or one can modify it, before using it. More detailedly
  spoken, we will ask you whether you want to leave the evaluated Open Source
  Software as you have got it or whether you want to modify it before using
  and/or distributing it to 3rd parties? In the firstcase, the answer should be
  'unmodified', in the second 'modified'.
  \item Thirdly we will consider the 
\end{itemize}




 Based on this information we can no derive and define the use cases by
which the OSLiC classifies its license fulfilling to-do lists:

Firstly we have to discriminate the usage of Open Source software by the nature
of the software itself:

On the one side you can use an application intended to support the work of an
end-user. It takes input data and generates output data, mostly by using a more
or less elaborated end-user interface. On the other side you can use a computer
librabry intended to support the work of a software developer. It mostly offers
functions and/or objects and is embedded into an overarching work like an
application.

The use of an application is different from the use of a software library
because the use of a software library implies the act of developing a new
overarching piece of software.

Secondly we have to discriminate the usage of Open Source software by the
addressee:

On the one side you can intend to use the software directly for your own
purpose. On the other side you can intend to distribute the software to a third
party.

And thirdly we have to discriminate the usage of Open Source software by nature of
the usage itself:

One the one side you can intend to use the software as it is, respectively as
you got it. On the other side you can intend to modify the software before using
it.

Let's form the corresponding dichotomies

use-it-as-you-got-it <> modify-it
use-it-for-yourself <> distribute-it
application <> library

Now we have 8 possibilities to combine these attributes:

Use an application as you got it only for yourself
Use a library as you got it only for your self
Distribute an application as you got it to a third party
Distribute a library as you got it to a third party




%\bibliography{../../../bibfiles/oscResourcesEn}
