% Telekom osCompendium 'for being included' snippet template
%
% (c) Karsten Reincke, Deutsche Telekom AG, Darmstadt 2011
%
% This LaTeX-File is licensed under the Creative Commons Attribution-ShareAlike
% 3.0 Germany License (http://creativecommons.org/licenses/by-sa/3.0/de/): Feel
% free 'to share (to copy, distribute and transmit)' or 'to remix (to adapt)'
% it, if you '... distribute the resulting work under the same or similar
% license to this one' and if you respect how 'you must attribute the work in
% the manner specified by the author ...':
%
% In an internet based reuse please link the reused parts to www.telekom.com and
% mention the original authors and Deutsche Telekom AG in a suitable manner. In
% a paper-like reuse please insert a short hint to www.telekom.com and to the
% original authors and Deutsche Telekom AG into your preface. For normal
% quotations please use the scientific standard to cite.
%
% [ Framework derived from 'mind your Scholar Research Framework' 
%   mycsrf (c) K. Reincke 2012 CC BY 3.0  http://mycsrf.fodina.de/ ]
%


%% use all entries of the bibliography
%\nocite{*}
\section{Open Source, OSI, and OSD}
\footnotesize
\begin{quote}\itshape
Here we describe the meaning of Open Source. At the end of this chapter you
should know that Open Source Software is defined by a set of necessary criteria
which together determine the common basic features of Open Source Licenses.
Additionally you will have understood that the opposite of Open Source Software
can not only by defined ex negativo. But you should also know that these features
can differently be implemented. Therefore the OSD can not be read as a set of
rules describing what we have to do if we want to fulfill the Open Source
Licenses. You should know that you have to go back to the license itself.
\end{quote}
\normalsize
\ldots

%\bibliography{../../../bibfiles/oscResourcesEn}
