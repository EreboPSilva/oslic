% Telekom osCompendium 'for being included' snippet template
%
% (c) Karsten Reincke, Deutsche Telekom AG, Darmstadt 2011
%
% This LaTeX-File is licensed under the Creative Commons Attribution-ShareAlike
% 3.0 Germany License (http://creativecommons.org/licenses/by-sa/3.0/de/): Feel
% free 'to share (to copy, distribute and transmit)' or 'to remix (to adapt)'
% it, if you '... distribute the resulting work under the same or similar
% license to this one' and if you respect how 'you must attribute the work in
% the manner specified by the author ...':
%
% In an internet based reuse please link the reused parts to www.telekom.com and
% mention the original authors and Deutsche Telekom AG in a suitable manner. In
% a paper-like reuse please insert a short hint to www.telekom.com and to the
% original authors and Deutsche Telekom AG into your preface. For normal
% quotations please use the scientific standard to cite.
%
% [ File structure derived from 'mind your Scholar Research Framework' 
%   mycsrf (c) K. Reincke CC BY 3.0  http://mycsrf.fodina.de/ ]

%


%% use all entries of the bibliography


\begin{table}
\footnotesize
\caption{History of the Open Source License Compendium}
\begin{center}
\begin{tabular}{|r|c|p{10cm}|}
\hline
\hline
    \texttt{2013-02-10}
  & \texttt{0.8.6} 
  & Towards the CeBIT Release\newline
    $\vartriangleright$ many refinements concerning the arguing structure\newline
    $\vartriangleright$ collected all meanwhile elabrorrated modifications\newline
    $\vartriangleright$ new licenses classifying chapter\newline   
    $\vartriangleright$ new top down introduction\newline
    \\
\hline
    \texttt{2012-08-25}
  & \texttt{0.5.2} 
  & Expanded Break Through Release\newline
    $\vartriangleright$ MIT license fulfilling to-do lists\newline
    $\vartriangleright$ using integrated eclipse spell checking methods\newline
    \\
\hline
    \texttt{2012-07-06}
  & \texttt{0.4.0} 
  & Break Through Release\newline
    $\vartriangleright$ open source use case definition and taxonomy
    (Chapter 6)\newline 
    $\vartriangleright$ open source use case based general finder(Chapter
    7)\newline 
    $\vartriangleright$ corresponding BSD specific mini finder(Chapter
    8.2)\newline 
    $\vartriangleright$ BSD license fulfilling to-do lists(Chapter 8.2)\\
\hline
    \texttt{2012-03-22}
  & \texttt{0.2.1} 
  & $\vartriangleright$ frameworke published as first community edition\\
\hline
    \texttt{2012-01-31}
  & \texttt{0.1.8} 
  & $\vartriangleright$ renamed existing introduction as prolegomena\newline
    $\vartriangleright$ inserted a shorter top-down written introduction\newline
    $\vartriangleright$ inserted an OSLiC disclaimer\\
\hline
    \texttt{2012-01-21}
  & \texttt{0.1.7} 
  & $\vartriangleright$ oscCopiedButNotRead.bib expanded by many verified
  bibliographic data \newline 
  $\vartriangleright$ list of periodicals and shortcuts added\\
\hline
    \texttt{2011-12-29}
  & \texttt{0.1.6} 
  & $\vartriangleright$ many bibliographic data added\\
\hline
    \texttt{2011-10-17}
  & \texttt{0.1.5} 
  & $\vartriangleright$ bibliographic data updated\\
\hline
    \texttt{2011-09-29}
  & \texttt{0.1.4} 
  & $\vartriangleright$ Document history integrated\newline
    $\vartriangleright$ Some Typos erased\\
\hline
    \texttt{2011-09-28}
  & \texttt{0.1.3} 
  & $\vartriangleright$ Review of english teacher integrated \\
\hline
    \texttt{2011-09-19}
  & \texttt{0.1.2} 
  & $\vartriangleright$ First comments of english teacher integrated \\
\hline
    \texttt{2011-09-15}
  & \texttt{0.1.1} 
  & $\vartriangleright$ Improvements of John integrated\\
\hline
    \texttt{2011-09-12}
  & \texttt{0.1.0} 
  & $\vartriangleright$ Introduction completed: purpose and methods \\
\hline
\hline 
\end{tabular}
\end{center}
\end{table}


%\bibliography{../bibfiles/oscResourcesEn}
