% Telekom osCompendium 'for being included' snippet template
%
% (c) Karsten Reincke, Deutsche Telekom AG, Darmstadt 2011
%
% This LaTeX-File is licensed under the Creative Commons Attribution-ShareAlike
% 3.0 Germany License (http://creativecommons.org/licenses/by-sa/3.0/de/): Feel
% free 'to share (to copy, distribute and transmit)' or 'to remix (to adapt)'
% it, if you '... distribute the resulting work under the same or similar
% license to this one' and if you respect how 'you must attribute the work in
% the manner specified by the author ...':
%
% In an internet based reuse please link the reused parts to www.telekom.com and
% mention the original authors and Deutsche Telekom AG in a suitable manner. In
% a paper-like reuse please insert a short hint to www.telekom.com and to the
% original authors and Deutsche Telekom AG into your preface. For normal
% quotations please use the scientific standard to cite.
%
% [ File structure derived from 'mind your Scholar Research Framework' 
%   mycsrf (c) K. Reincke CC BY 3.0  http://mycsrf.fodina.de/ ]

%


%% use all entries of the bibliography

%\chapter*{History}

\begin{table}
\footnotesize
\caption{History of the Open Source License Compendium}
\begin{center}
\begin{tabular}{|r|c|p{9.4cm}|}
\hline
    \texttt{2015-01-21}
  & \texttt{1.0.0}
  & $\vartriangleright$ added solution for the reverse engineering challenge \\
\hline
    \texttt{2014-03-09}
  & \texttt{0.99.1}
  & $\vartriangleright$ Generate data file for use in OSCAd from the \LaTeX source\newline
  $\vartriangleright$ Fixed Bug in LGPL C9 Case \newline
  $\vartriangleright$ general copy-editing of chapter 6\\
\hline
    \texttt{2014-01-08}
  & \texttt{0.98.2}
  & $\vartriangleright$ New section about the patent clauses in the CDDL\newline
  $\vartriangleright$ hyperlinked PDF file (using hyperref and pdftex)\newline
  $\vartriangleright$ general copy-editing of chapter 1 to 5\\
\hline
    \texttt{2013-11-27}
  & \texttt{0.98.1}
  & Korean FLOSS conference release\\
\hline
    \texttt{2013-08-19}
  & \texttt{0.97.2}
  & $\vartriangleright$ incorpation of the typo fixes offered by M.
    Schierl\newline
  $\vartriangleright$ some improvements concerning the derivative work\newline
  $\vartriangleright$ enhancing that the OSLiC deals with prototypic cases\\
\hline
    \texttt{2013-07-28}
  & \texttt{0.97.1} 
  & W7 pre release\newline
    $\vartriangleright$ indirectly used secondary literature added\newline
    $\vartriangleright$ LGPL specific finder improved\newline
    $\vartriangleright$ OSCAd aligned, interface improved\\
\hline
    \texttt{2013-05-20}
  & \texttt{0.96.1} 
  & Linux Days release\newline    
    $\vartriangleright$ open source use cases and licenses specific usecase renamed\newline
    $\vartriangleright$ version matches the content of OSCAd\\
\hline
    \texttt{2013-04-15}
  & \texttt{0.95.2} 
  & FSFE LLW post release\newline
    $\vartriangleright$ to-do lists for nearly all popular OSI licenses\newline
    $\vartriangleright$ improved finder for GPL and EUPL\newline
    $\vartriangleright$ simplified form and improved structure of the OSLiC finder\\
\hline
    \texttt{2013-04-05}
  & \texttt{0.95.1} 
  & FSFE LLW pre release\newline
    $\vartriangleright$ to-do lists for all permissive and all weak copyleft licenses\newline
    $\vartriangleright$ branches merged and new master published\\
\hline
    \texttt{2013-03-15}
  & \texttt{0.94.1} 
  & Chemnitzer Linux Day release\newline
    $\vartriangleright$ to-do lists for all permissive and some weak copyleft licenses\newline
    $\vartriangleright$ branches merged and new master published\\
\hline
    \texttt{2013-03-08}
  & \texttt{0.90.1} 
  & CeBIT release\newline
    $\vartriangleright$ to-do lists for the some important licenses added\newline
    $\vartriangleright$ branches merged and new master published\\
\hline
    \texttt{2013-02-16}
  & \texttt{0.8.90} 
  & CeBIT pre release\newline
    $\vartriangleright$ new arguing structure focused on the topic license fulfillment\newline
    $\vartriangleright$ new classifying license review\newline   
    $\vartriangleright$ new top down introduction\\
\hline
    \texttt{2012-12-28}
  & \texttt{0.8.0} 
  & internal EOY release\newline
    $\vartriangleright$ many distributed improvements unified in branch kreinck\\
\hline
    \texttt{2012-08-25}
  & \texttt{0.5.2} 
  & expanded break through release\newline
    $\vartriangleright$ MIT license fulfilling to-do lists\newline
    $\vartriangleright$ using integrated Eclipse spell checking methods\\
\hline
    \texttt{2012-07-06}
  & \texttt{0.4.0} 
  & break through release\newline
    $\vartriangleright$ open source use case definition and taxonomy\newline 
    $\vartriangleright$ open source use case based general finder\newline 
    $\vartriangleright$ corresponding BSD specific mini finder\newline 
    $\vartriangleright$ BSD license fulfilling to-do lists\\
\hline
    \texttt{2012-03-22}
  & \texttt{0.2.1} 
  & $\vartriangleright$ framework published as first community edition\\
\hline
    \texttt{2012-01-31}
  & \texttt{0.1.8} 
  & $\vartriangleright$ renamed existing introduction as prolegomena\newline
    $\vartriangleright$ inserted a shorter top-down written introduction\newline
    $\vartriangleright$ added an OSLiC disclaimer \& many bibliographic data\\
\hline
    \texttt{2011-09-29}
  & \texttt{0.1.4} 
  & $\vartriangleright$ document history integrated\\
\hline
    \texttt{2011-09-12}
  & \texttt{0.1.0} 
  & $\vartriangleright$ introduction completed: purpose and methods \\
\hline
\hline 
\end{tabular}
\end{center}
\end{table}

%\bibliography{../bibfiles/oscResourcesEn}

% Local Variables:
% mode: latex
% fill-column: 80
% End:
