% Telekom osCompendium 'for being included' snippet template
%
% (c) Karsten Reincke, Deutsche Telekom AG, Darmstadt 2011
%
% This LaTeX-File is licensed under the Creative Commons Attribution-ShareAlike
% 3.0 Germany License (http://creativecommons.org/licenses/by-sa/3.0/de/): Feel
% free 'to share (to copy, distribute and transmit)' or 'to remix (to adapt)'
% it, if you '... distribute the resulting work under the same or similar
% license to this one' and if you respect how 'you must attribute the work in
% the manner specified by the author ...':
%
% In an internet based reuse please link the reused parts to www.telekom.com and
% mention the original authors and Deutsche Telekom AG in a suitable manner. In
% a paper-like reuse please insert a short hint to www.telekom.com and to the
% original authors and Deutsche Telekom AG into your preface. For normal
% quotations please use the scientific standard to cite.
%
% [ Framework derived from 'mind your Scholar Research Framework' 
%   mycsrf (c) K. Reincke 2012 CC BY 3.0  http://mycsrf.fodina.de/ ]
%


%% use all entries of the bibliography
%\nocite{*}

\section{MPL licensed software}

Also, the Mozilla Public License clearly distinguishes the distribution in the
form of source code from that in the form of binaries: First, it allows the
\enquote{Distribution of Source Form}\footcite[cf.][\nopage wp.\
§3.1]{Mpl20OsiLicense2013a}. Then, it specifies the conditions for a
\enquote{Distribution of Executable Form}\footcite[cf.][\nopage wp.\
§3.2]{Mpl20OsiLicense2013a}. Additionally, the MPL confronts the
\enquote{distribution of Covered Software} with the \enquote{distribution of a
Larger Work}\footcite[cf.][\nopage wp.\ §3.3]{Mpl20OsiLicense2013a}. So, taken
as whole, the MPL mainly focusses on the distribution of software. Thus, for
finding the relevant, simply processable task lists, also the following MPL
specific open source use case structure\footnote{For details of the general OSUC
finder $\rightarrow$ OSLiC, pp.\ \pageref{OsucTokens} and
\pageref{OsucDefinitionTree}} can be used:
 
\tikzstyle{nodv} = [font=\small, ellipse, draw, fill=gray!10, 
    text width=2cm, text centered, minimum height=2em]

\tikzstyle{nods} = [font=\footnotesize, rectangle, draw, fill=gray!20, 
    text width=1.2cm, text centered, rounded corners, minimum height=3em]

\tikzstyle{nodb} = [font=\footnotesize, rectangle, draw, fill=gray!20, 
    text width=2.2cm, text centered, rounded corners, minimum height=3em]
    
\tikzstyle{leaf} = [font=\tiny, rectangle, draw, fill=gray!30, 
    text width=1.2cm, text centered, minimum height=6em]

\tikzstyle{edge} = [draw, -latex']

\begin{tikzpicture}[]

\node[nodv] (l71) at (4,10) {MPL};

\node[nodb] (l61) at (0,8.6) {\textit{recipient:} \\ \textbf{4yourself}};
\node[nodb] (l62) at (6.5,8.6) {\textit{recipient:} \\ \textbf{2others}};

\node[nodb] (l51) at (2.5,7) {\textit{state:} \\ \textbf{unmodified}};
\node[nodb] (l52) at (9.3,7) {\textit{state:} \\ \textbf{modified}};

\node[nods] (l41) at (1.8,5.4) {\textit{form:} \textbf{source}};
\node[nods] (l42) at (3.6,5.4) {\textit{form:} \textbf{binary}};
\node[nodb] (l43) at (6.5,5.4) {\textit{type:} \\ \textbf{proapse}};
\node[nodb] (l44) at (12,5.4) {\textit{type:} \\ \textbf{snimoli}};


\node[nods] (l31) at (5.4,3.8) {\textit{form:} \textbf{source}};
\node[nods] (l32) at (7.2,3.8) {\textit{form:} \textbf{binary}};
\node[nodb] (l33) at (10,3.8) {\textit{context:} \\ \textbf{independent}};
\node[nodb] (l34) at (13.5,3.8) {\textit{context:} \\ \textbf{embedded}};

\node[nods] (l21) at (9,2.2) {\textit{form:} \textbf{source}};
\node[nods] (l22) at (10.8,2.2) {\textit{form:} \textbf{binary}};
\node[nods] (l23) at (12.6,2.2) {\textit{form:} \textbf{source}};
\node[nods] (l24) at (14.4,2.2) {\textit{form:} \textbf{binary}};

\node[leaf] (l11) at (0,0) {\textbf{MPL-C1} \textit{using software only
for yourself}};

\node[leaf] (l12) at (1.8,0) { \textbf{MPL-C2} \textit{ distributing unmodified
software as sources}};

\node[leaf] (l13) at (3.6,0) { \textbf{MPL-C3}  \textit{ distributing unmodified
software as binaries}};

\node[leaf] (l14) at (5.4,0) { \textbf{MPL-C4}  \textit{ distributing modified
program as sources}};

\node[leaf] (l15) at (7.2,0) { \textbf{MPL-C5}  \textit{ distributing modified
program as binaries}};

\node[leaf] (l16) at (9,0) { \textbf{MPL-C6}  \textit{ distributing modified
library as independent sources}};

\node[leaf] (l17) at (10.8,0) { \textbf{MPL-C7} \textit{distributing modified
library as independent binaries}};

\node[leaf] (l18) at (12.6,0) { \textbf{MPL-C8}  \textit{distributing
modified library as embedded sources}};

\node[leaf] (l19) at (14.4,0) { \textbf{MPL-C9}  \textit{ distributing modified
library as embedded binaries}};


\path [edge] (l71) -- (l61);
\path [edge] (l71) -- (l62);
\path [edge] (l61) -- (l11);
\path [edge] (l62) -- (l51);
\path [edge] (l62) -- (l52);
\path [edge] (l51) -- (l41);
\path [edge] (l51) -- (l42);
\path [edge] (l52) -- (l43);
\path [edge] (l52) -- (l44);
\path [edge] (l41) -- (l12);
\path [edge] (l42) -- (l13);
\path [edge] (l43) -- (l31);
\path [edge] (l43) -- (l32);
\path [edge] (l44) -- (l33);
\path [edge] (l44) -- (l34);
\path [edge] (l31) -- (l14);
\path [edge] (l32) -- (l15);
\path [edge] (l33) -- (l21);
\path [edge] (l33) -- (l22);
\path [edge] (l34) -- (l23);
\path [edge] (l34) -- (l24);
\path [edge] (l21) -- (l16);
\path [edge] (l22) -- (l17);
\path [edge] (l23) -- (l18);
\path [edge] (l24) -- (l19);

\end{tikzpicture}

\subsection{MPL-C1: Using the software only for yourself}
\label{OSUC-01-MPL} \label{OSUC-03-MPL} 
\label{OSUC-06-MPL} \label{OSUC-09-MPL}

\begin{description}

\item[means] that you are going to use a received MPL licensed software only
for yourself and that you do not hand it over to any 3rd party in any sense.

\item[covers] OSUC-01, OSUC-03, OSUC-06, and OSUC-09\footnote{For details 
$\rightarrow$ OSLiC, pp.\ \pageref{OSUC-01-DEF} - \pageref{OSUC-09-DEF}}

\item[requires] no tasks in order to fulfill the conditions of the MPL 2.0
license with respect to this use case:
  \begin{itemize}
    \item You are allowed to use any kind of MPL software in any sense and in
    any context without being obliged to do anything as long as you do not
    give the software to 3rd parties.
  \end{itemize}

\item[prohibits] \ldots
\begin{itemize}
  \item to remove or to alter any license notices -- including copyright
  notices, patent notices, disclaimers of warranty, or limitations of liablility
  -- contained within the software package you have received.
  \item to promote any of your services -- based on the this software -- by
  trademarks, service marks, or logos linked to this MPL software, except as 
  required for reasonable and customary use in describing the origin
  of the software and reproducing the  copyright notice.
\end{itemize}

\end{description}

\subsection{MPL-C2: Passing the unmodified software as source code}
\label{OSUC-02S-MPL} \label{OSUC-05S-MPL} \label{OSUC-07S-MPL} 

\begin{description}

\item[means] that you are going to distribute an unmodified version of the
received MPL software to 3rd parties - in the form of source code files or as a
source code package. In this case it is not discriminating to distribute a
program, an application, a server, a snippet, a module, a library, or a plugin
as an independent or an embedded unit

\item[covers] OSUC-02S, OSUC-05S, OSUC-07S\footnote{For details $\rightarrow$
OSLiC, pp.\ \pageref{OSUC-02S-DEF} - \pageref{OSUC-07S-DEF}}

\item[requires] the following tasks in order to fulfill the license conditions:
\begin{itemize}
  
  \item \textbf{[mandatory:]} Ensure that the licensing elements -- esp.\ all
  copyright notices, patent notices, disclaimers of warranty, or limitations of
  liability -- are retained in your package in exact the form you have received
  them.

  \item \textbf{[mandatory:]} Give the recipient a copy of the MPL 2.0 license.
  If it is not already part of the software package, add it\footnote{For
  implementing the handover of files correctly $\rightarrow$ OSLiC, p.
  \pageref{DistributingFilesHint}}. If the licensing statement in the licensing
  file of the package does still not clearly state that the package is licensed
  under the MPL, additionally insert your own correct MPL licensing file
  containing the sentence: \emph{This Source Code Form is subject to the terms
  of the Mozilla Public License, v. 2.0. If a copy of the MPL was not
  distributed with this file, You can obtain one at
  http://mozilla.org/MPL/2.0/}.

  \item \textbf{[voluntary:]} Let the documentation of your distribution and/or
  your additional material also reproduce the content of the existing
  \emph{copyright notice text files}, a hint to the software name, a link to its
  homepage, and a link to the MPL 2.0 license.
\end{itemize}

\item[prohibits] \ldots
\begin{itemize}
  \item to remove or to alter any license notices -- including copyright
  notices, patent notices, disclaimers of warranty, or limitations of liablility
  -- contained within the software package you have received.
  \item to promote any of your products -- based on the this software -- by
  trademarks, service marks, or logos linked to this MPL software,  except as 
  required for reasonable and customary use in describing the origin
  of the software and reproducing the  copyright notice.
\end{itemize}
\end{description}


\subsection{MPL-C3: Passing the unmodified software as binaries} 
\label{OSUC-02B-MPL} \label{OSUC-05B-MPL} \label{OSUC-07B-MPL}

\begin{description}
\item[means] that you are going to distribute an unmodified version of the
received MPL software to 3rd parties -- in the form of binary files or as a
bi\-na\-ry package. In this case it is not discriminating to distribute a
program, an application, a server, a snippet, a module, a library, or a plugin
as an independent or an embedded unit.

\item[covers] OSUC-02B, OSUC-05B, OSUC-07B\footnote{For details $\rightarrow$
OSLiC, pp.\ \pageref{OSUC-02B-DEF} - \pageref{OSUC-07B-DEF}}

\item[requires] the following tasks in order to fulfill the license conditions:
\begin{itemize}
  
  \item \textbf{[mandatory:]} Ensure that the licensing elements -- esp.\ all
  copyright notices, patent notices, disclaimers of warranty, or limitations of
  liability -- are retained in your package in exact the form you have received
  them. If you compile the binary from the sources, ensure that all these
  licensing elements are also incorporated into the package.
  
  \item \textbf{[mandatory:]} Make the source code of the software accessible
  via a repository under your own control: Push the source code package into an
  internet repository and enable its download function without requiring any fee
  from the downloading user. Integrate an easily to find description into your
  distribution package which explains how the code can be received from where.
  Ensure, that this repository is usable reasonably long enough.
  
  \item \textbf{[mandatory:]} Insert a prominent hint to the download repository
  into your distribution and/or your additional material.
  
  \item \textbf{[mandatory:]} Execute the to-do list of use case MPL-C2\footnote{
  Making the code accessible via a repository means distributing the software in
  the form of source code. Hence, you must also fulfill all tasks of the
  corresponding use case.}.
  
  \item \textbf{[voluntary:]} Give the recipient a copy of the MPL 2.0 license.
  If it is not already part of the software package, add it\footnote{For
  implementing the handover of files correctly $\rightarrow$ OSLiC, p.
  \pageref{DistributingFilesHint}}. If the licensing statement in the licensing
  file of the package does still not clearly state that the package is licensed
  under the MPL, additionally insert your own correct MPL licensing file
  containing the sentence: \emph{This Source Code Form is subject to the terms
  of the Mozilla Public License, v. 2.0. If a copy of the MPL was not
  distributed with this file, You can obtain one at
  http://mozilla.org/MPL/2.0/}.
  
  \item \textbf{[voluntary:]} Let the documentation of your distribution and/or
  your additional material also reproduce the content of the existing
  \emph{copyright notice text files}, a hint to the software name, a link to its
  homepage, and a link to the MPL 2.0 license.
    
\end{itemize}

\item[prohibits] \ldots
\begin{itemize}
  \item to remove or to alter any license notices -- including copyright
  notices, patent notices, disclaimers of warranty, or limitations of liablility
  -- contained within the software package you have received.
  \item to promote any of your products -- based on the this software -- by
  trademarks, service marks, or logos linked to this MPL software, except as 
  required for reasonable and customary use in describing the origin
  of the software and reproducing the  copyright notice.
\end{itemize}

\end{description}

\subsection{MPL-C4: Passing a modified program as source code}
\label{OSUC-04S-MPL} 

\begin{description}
\item[means] that you are going to distribute a modified version of the received
MPL licensed program, application, or server (proapse) to 3rd parties -- in the
form of source code files or a source code package.
\item[covers] OSUC-04S\footnote{For details $\rightarrow$ OSLiC, pp.\
\pageref{OSUC-04S-DEF}}
\item[requires] the following tasks in order to fulfill the license conditions:
\begin{itemize}
  
  \item \textbf{[mandatory:]} Ensure that the licensing elements -- esp.\ all
  copyright notices, patent notices, disclaimers of warranty, or limitations of
  liability -- are retained in your package in exact the form you have received
  them.

  \item \textbf{[mandatory:]} Give the recipient a copy of the MPL 2.0 license.
  If it is not already part of the software package, add it\footnote{For
  implementing the handover of files correctly $\rightarrow$ OSLiC, p.
  \pageref{DistributingFilesHint}}. If the licensing statement in the licensing
  file of the package does still not clearly state that the package is licensed
  under the MPL, additionally insert your own correct MPL licensing file
  containing the sentence: \emph{This Source Code Form is subject to the terms
  of the Mozilla Public License, v. 2.0. If a copy of the MPL was not
  distributed with this file, You can obtain one at
  http://mozilla.org/MPL/2.0/}.  
  
  \item \textbf{[mandatory:]} Organize your modifications in a way that they are
  covered by the existing MPL licensing statements. If you add new source code
  files, insert a header containing your copyright line and an MPL adequate
  licensing the statement.
  
  \item \textbf{[voluntary:]} Create a \emph{modification text file}, if such a
  notice file still does not exist. \emph{Expand} the \emph{modification text
  file} by a more general description of your modifications. Incorporate it into
  your distribution package.
  
  \item \textbf{[voluntary:]} Mark all modifications of the source code of the
  program (proapse) thoroughly -- namely within the modified source code.

  \item \textbf{[voluntary:]} Let the documentation of your distribution and/or
  your additional material also reproduce the content of the existing
  \emph{copyright notice text files}, a hint to the software name, a link to its
  homepage, and a link to the MPL 2.0 license.
  
 \end{itemize}
 
\item[prohibits] \ldots
\begin{itemize}
  \item to remove or to alter any license notices -- including copyright
  notices, patent notices, disclaimers of warranty, or limitations of liablility
  -- contained within the software package you have received.
  \item to promote any of your products -- based on the this software -- by
  trademarks, service marks, or logos linked to this MPL software, except as 
  required for reasonable and customary use in describing the origin
  of the software and reproducing the  copyright notice.
\end{itemize}

\end{description}

\subsection{MPL-C5: Passing a modified program as binary}
\label{OSUC-04B-MPL} 

\begin{description}
\item[means] that you are going to distribute a modified version of the received
MPL licensed pro\-gram, application, or server (proapse) to 3rd parties -- in
the form of binary files or as a binary package.
\item[covers] OSUC-04B\footnote{For details $\rightarrow$ OSLiC, pp.\
\pageref{OSUC-04B-DEF}}
\item[requires] the following tasks in order to fulfill the license conditions:
\begin{itemize}

  \item \textbf{[mandatory:]} Ensure that the licensing elements -- esp.\ all
  copyright notices, patent notices, disclaimers of warranty, or limitations of
  liability -- are retained in your package in exact the form you have received
  them. If you compile the binary from the sources, ensure that all these
  licensing elements are also incorporated into the package.

  \item \textbf{[mandatory:]} Make the source code of the software accessible
  via a repository under your own control: Push the source code package into an
  internet repository and enable its download function without requiring any fee
  from the downloading user. Integrate an easily to find description into your
  distribution package which explains how the code can be received from where.
  Ensure, that this repository is usable reasonably long enough.
  
  \item \textbf{[mandatory:]} Insert a prominent hint to the download repository
  into your distribution and/or your additional material. 

  \item \textbf{[mandatory:]} Execute the to-do list of use case MPL-C4\footnote{
  Making the code accessible via a repository means distributing the software in
  the form of source code. Hence, you must also fulfill all tasks of the
  corresponding use case.}.

  \item \textbf{[mandatory:]} Organize your modifications in a way that they are
  covered by the existing MPL licensing statements.
  
  \item \textbf{[voluntary:]} Create a \emph{modification text file}, if such a
  notice file still does not exist. \emph{Expand} the \emph{modification text
  file} by a more general description of your modifications. Incorporate it into
  your distribution package.
  
  \item \textbf{[voluntary:]} Give the recipient a copy of the MPL 2.0 license.
  If it is not already part of the software package, add it\footnote{For
  implementing the handover of files correctly $\rightarrow$ OSLiC, p.
  \pageref{DistributingFilesHint}}. If the licensing statement in the licensing
  file of the package does still not clearly state that the package is licensed
  under the MPL, additionally insert your own correct MPL licensing file
  containing the sentence: \emph{This Source Code Form is subject to the terms
  of the Mozilla Public License, v. 2.0. If a copy of the MPL was not
  distributed with this file, You can obtain one at
  http://mozilla.org/MPL/2.0/}.
 
  \item \textbf{[voluntary:]} Let the documentation of your distribution and/or
  your additional material  also reproduce the content of the existing
  \emph{copyright notice text files}, a hint to the software name, a link to its
  homepage, and a link to the MPL 2.0 license -- especially as a subsection of
  your own copyright notice.

\end{itemize}  

\item[prohibits] \ldots
\begin{itemize}
  \item to remove or to alter any license notices -- including copyright
  notices, patent notices, disclaimers of warranty, or limitations of liablility
  -- contained within the software package you have received.
  \item to promote any of your products -- based on the this software -- by
  trademarks, service marks, or logos linked to this MPL software, except as 
  required for reasonable and customary use in describing the origin
  of the software and reproducing the  copyright notice.
\end{itemize}

\end{description}

\subsection{MPL-C6: Passing a modified library as independent source code}
\label{OSUC-08S-MPL}

\begin{description}
\item[means] that you are going to distribute a modified version of the received
MPL licensed code snippet, module, library, or plugin (snimoli) to 3rd parties
-- in the form of source code files or as a source code package, but without
embedding it into another larger software unit.
\item[covers] OSUC-08S\footnote{For details $\rightarrow$ OSLiC, pp.\
\pageref{OSUC-08S-DEF}}
\item[requires] the following tasks in order to fulfill the license conditions:
\begin{itemize}

  \item \textbf{[mandatory:]} Ensure that the licensing elements -- esp.\ all
  copyright notices, patent notices, disclaimers of warranty, or limitations of
  liability -- are retained in your package in exact the form you have received
  them.
  
  \item \textbf{[mandatory:]} Give the recipient a copy of the MPL 2.0 license.
  If it is not already part of the software package, add it\footnote{For
  implementing the handover of files correctly $\rightarrow$ OSLiC, p.
  \pageref{DistributingFilesHint}}. If the licensing statement in the licensing
  file of the package does still not clearly state that the package is licensed
  under the MPL, additionally insert your own correct MPL licensing file
  containing the sentence: \emph{This Source Code Form is subject to the terms
  of the Mozilla Public License, v. 2.0. If a copy of the MPL was not
  distributed with this file, You can obtain one at
  http://mozilla.org/MPL/2.0/}.
  
  \item \textbf{[mandatory:]} Organize your modifications in a way that they are
  covered by the existing MPL licensing statements. If you add new source code
  files, insert a header containing your copyright line and an MPL adequate
  licensing the statement.
  
  \item \textbf{[voluntary:]} Create a \emph{modification text file}, if such a
  notice file still does not exist. \emph{Expand} the \emph{modification text
  file} by a more general description of your modifications. Incorporate it into
  your distribution package.

  \item \textbf{[voluntary:]} Mark all modifications of the source code of the
  library (snimoli) thoroughly -- namely within
  the modified source code.
  
  \item \textbf{[voluntary:]} Let the documentation of your distribution and/or
  your additional material  also reproduce the content of the existing
  \emph{copyright notice text files}, a hint to the software name, a link to its
  homepage, and a link to the MPL 2.0 license.

\end{itemize}

\item[prohibits] \ldots
\begin{itemize}
  \item to remove or to alter any license notices -- including copyright
  notices, patent notices, disclaimers of warranty, or limitations of liablility
  -- contained within the software package you have received.
  \item to promote any of your products -- based on the this software -- by
  trademarks, service marks, or logos linked to this MPL software, except as 
  required for reasonable and customary use in describing the origin
  of the software and reproducing the  copyright notice.
\end{itemize}

\end{description}


\subsection{MPL-C7: Passing a modified library as independent binary}
\label{OSUC-08B-MPL}

\begin{description}
\item[means] that you are going to distribute a modified version of the received
MPL licensed code snippet, module, library, or plugin (snimoli) to 3rd parties
-- in the form of binary files or as a binary package but without embedding it
into another larger software unit.
\item[covers] OSUC-08B\footnote{For details $\rightarrow$ OSLiC, pp.\
\pageref{OSUC-08B-DEF}}
\item[requires] the following tasks in order to fulfill the license conditions:
\begin{itemize}

  \item \textbf{[mandatory:]} Ensure that the licensing elements -- esp.\ all
  copyright notices, patent notices, disclaimers of warranty, or limitations of
  liability -- are retained in your package in exact the form you have received
  them. If you compile the binary from the sources, ensure that all these
  licensing elements are also incorporated into the package.

  \item \textbf{[mandatory:]} Make the source code of the software accessible
  via a repository under your own control: Push the source code package into an
  internet repository and enable its download function without requiring any fee
  from the downloading user. Integrate an easily to find description into your
  distribution package which explains how the code can be received from where.
  Ensure, that this repository is usable reasonably long enough.
  
  \item \textbf{[mandatory:]} Insert a prominent hint to the download repository
  into your distribution and/or your additional material.

  \item \textbf{[mandatory:]} Execute the to-do list of use case MPL-6\footnote{
  Making the code accessible via a repository means distributing the software in
  the form of source code. Hence, you must also fulfill all tasks of the
  corresponding use case.}.
  
  \item \textbf{[mandatory:]} Organize your modifications in a way that they are
  covered by the existing MPL licensing statements.
  
  \item \textbf{[voluntary:]} Create a \emph{modification text file}, if such a
  notice file still does not exist. \emph{Expand} the \emph{modification text
  file} by a more general description of your modifications. Incorporate it into
  your distribution.
  
  \item \textbf{[voluntary:]} Give the recipient a copy of the MPL 2.0 license.
  If it is not already part of the software package, add it\footnote{For
  implementing the handover of files correctly $\rightarrow$ OSLiC, p.
  \pageref{DistributingFilesHint}}. If the licensing statement in the licensing
  file of the package does still not clearly state that the package is licensed
  under the MPL, additionally insert your own correct MPL licensing file
  containing the sentence: \emph{This Source Code Form is subject to the terms
  of the Mozilla Public License, v. 2.0. If a copy of the MPL was not
  distributed with this file, You can obtain one at
  http://mozilla.org/MPL/2.0/}.
  
  \item \textbf{[voluntary:]} Let the documentation of your distribution and/or
  your additional material  also reproduce the content of the existing
  \emph{copyright notice text files}, a hint to the software name, a link to its
  homepage, and a link to the MPL 2.0 license -- especially as a subsection of
  your own copyright notice.
  
\end{itemize}

\item[prohibits] \ldots
\begin{itemize}
  \item to remove or to alter any license notices -- including copyright
  notices, patent notices, disclaimers of warranty, or limitations of liablility
  -- contained within the software package you have received.
  \item to promote any of your products -- based on the this software -- by
  trademarks, service marks, or logos linked to this MPL software, except as 
  required for reasonable and customary use in describing the origin
  of the software and reproducing the  copyright notice.
\end{itemize}

\end{description}

\subsection{MPL-C8: Passing a modified library as embedded source code}
\label{OSUC-10S-MPL}

\begin{description}
\item[means] that you are going to distribute a modified version of the received
MPL licensed code snippet, module, library, or plugin (snimoli) to 3rd parties
-- in the form of source code files or as a source code package together with
another larger software unit which contains this code snippet, module, library,
or plugin as an embedded component.
\item[covers] OSUC-10S\footnote{For details $\rightarrow$ OSLiC, pp.\
\pageref{OSUC-10S-DEF}}
\item[requires] the following tasks in order to fulfill the license conditions:
\begin{itemize}

  \item \textbf{[mandatory:]} Ensure that the licensing elements -- esp.\ all
  copyright notices, patent notices, disclaimers of warranty, or limitations of
  liability -- are retained in your package in exact the form you have received
  them.
  
  \item \textbf{[mandatory:]} Give the recipient a copy of the MPL 2.0 license.
  If it is not already part of the software package, add it\footnote{For
  implementing the handover of files correctly $\rightarrow$ OSLiC, p.
  \pageref{DistributingFilesHint}}. If the licensing statement in the licensing
  file of the package does still not clearly state that the package is licensed
  under the MPL, additionally insert your own correct MPL licensing file
  containing the sentence: \emph{This Source Code Form is subject to the terms
  of the Mozilla Public License, v. 2.0. If a copy of the MPL was not
  distributed with this file, You can obtain one at
  http://mozilla.org/MPL/2.0/}.

  \item \textbf{[mandatory:]} Organize your modifications of the embedded
  library in a way that they are covered by the existing MPL licensing
  statements. If you add new source code files to the library itself, insert a
  header containing your copyright line and an MPL adequate licensing the
  statement.
  
  \item \textbf{[voluntary:]} Arrange your source code distribution so that the
  integrated MPL and the \emph{licensing files} clearly refer only to the
  embedded library and do not disturb the licensing of your own overarching
  work. It's a good tradition to keep the embedded components like libraries,
  modules, snippets, or plugins in specific directory which contains also all
  additional licensing elements.


  \item \textbf{[voluntary:]} Create a \emph{modification text file}, if such a
  notice file still does not exist. \emph{Expand} the \emph{modification text
  file} by a more general description of your modifications. Incorporate it into
  your distribution package.
  
  \item \textbf{[voluntary:]} Mark all modifications of the source code of the
  embedded library (snimoli) thoroughly -- namely within the source code.
      
  \item \textbf{[voluntary:]} Let the documentation of your distribution and/or
  your additional material also reproduce the content of the existing
  \emph{copyright notice text files}, a hint to the name of the used MPL
  licensed component, a link to its homepage, and a link to the MPL 2.0 license
  -- especially as subsection of your own copyright notice.
 
\end{itemize}

\item[prohibits] \ldots
\begin{itemize}
  \item to remove or to alter any license notices -- including copyright
  notices, patent notices, disclaimers of warranty, or limitations of liablility
  -- contained within the software package you have received.
  \item to promote any of your products -- based on the this software -- by
  trademarks, service marks, or logos linked to this MPL software, except as 
  required for reasonable and customary use in describing the origin
  of the software and reproducing the  copyright notice.
\end{itemize}

\end{description}


\subsection{MPL-C9: Passing a modified library as embedded binary}
\label{OSUC-10B-MPL}

\begin{description}
\item[means] that you are going to distribute a modified version of the received
MPL licensed code snippet, module, library, or plugin to 3rd parties -- in the
form of binary files or as a binary package together with another larger
software unit which contains this code snippet, module, library, or plugin as an
embedded component.
\item[covers] OSUC-10B\footnote{For details $\rightarrow$ OSLiC, pp.\
\pageref{OSUC-10B-DEF}}
\item[requires] the following tasks in order to fulfill the license conditions:
\begin{itemize}

  \item \textbf{[mandatory:]} Ensure that the licensing elements -- esp.\ all
  copyright notices, patent notices, disclaimers of warranty, or limitations of
  liability -- are retained in your package in exact the form you have received
  them. If you compile the binary from the sources, ensure that all these
  licensing elements are also incorporated into the package.

  \item \textbf{[mandatory:]} Make the source code of the embedded library
  accessible via a repository under your own control: Push the source code
  package into an internet repository and enable its download function without
  requiring any fee from the downloading user. Integrate an easily to find
  description into your distribution package which explains how the code can be
  received from where. Ensure, that this repository is usable reasonably long
  enough.
  
  \item \textbf{[mandatory:]} Insert a prominent hint to the download repository
  into your distribution and/or your additional material.

  \item \textbf{[mandatory:]} Execute the to-do list of use case MPL-C8\footnote{
  Making the code accessible via a repository means distributing the software in
  the form of source code. Hence, you must also fulfill all tasks of the
  corresponding use case.}.

  \item \textbf{[mandatory:]} Organize your modifications of the embedded
  library in a way that they are covered by the existing MPL licensing
  statements. 
    
  \item \textbf{[voluntary:]} Create a \emph{modification text file}, if such a
  notice file still does not exist. \emph{Expand} the \emph{modification text
  file} by a more general description of your modifications. Incorporate it into
  your distribution package.
  
  \item \textbf{[voluntary:]} Give the recipient a copy of the MPL 2.0 license.
  If it is not already part of the software package, add it\footnote{For
  implementing the handover of files correctly $\rightarrow$ OSLiC, p.
  \pageref{DistributingFilesHint}}. If the licensing statement in the licensing
  file of the package does still not clearly state that the package is licensed
  under the MPL, additionally insert your own correct MPL licensing file
  containing the sentence: \emph{This Source Code Form is subject to the terms
  of the Mozilla Public License, v. 2.0. If a copy of the MPL was not
  distributed with this file, You can obtain one at
  http://mozilla.org/MPL/2.0/}.

  \item \textbf{[voluntary:]} Arrange your binary distribution so that the
  integrated MPL and the \emph{licensing files} clearly refer only to the
  embedded library and do not disturb the licensing of your own overarching
  work. It's a good tradition to keep the embedded components like libraries,
  modules, snippets, or plugins in specific directory which contains also all
  additional licensing elements.
  
  
  \item \textbf{[voluntary:]} Let the documentation of your distribution and/or
  your additional material  also reproduce the content of the existing
  \emph{copyright notice text files}, a hint to the name of the used MPL
  licensed component, a link to its homepage, and a link to the MPL 2.0 license
  -- especially as subsection of your own copyright notice.
  
\end{itemize}

\item[prohibits] \ldots
\begin{itemize}
  \item to remove or to alter any license notices -- including copyright
  notices, patent notices, disclaimers of warranty, or limitations of liablility
  -- contained within the software package you have received.
  \item to promote any of your products -- based on the this software -- by
  trademarks, service marks, or logos linked to this MPL software, except as 
  required for reasonable and customary use in describing the origin
  of the software and reproducing the  copyright notice.
\end{itemize}

\end{description}

\subsection{Discussions and Explanations}

The MPL offers a section \enquote{Responsibilities} which contains nearly all
requirements\footcite[cf.][\nopage wp.\ §3]{Mpl20OsiLicense2013a}. Only for some
subordinated aspects, one has also to reflect other paragraphs\footcite[pars
pro to cf.][\nopage wp.\ §3 - concerning the trademarks]{Mpl20OsiLicense2013a}.
With respect to this structure, we can detect the following tasks:

\begin{itemize}

  \item In a more general attitude, the MPL states that it \enquote{[\ldots]
  does not grant any rights in the trademarks, service marks, or logos of any
  Contributor} -- except as it may be necessary \enquote{to comply with} other
  requirements of the license\footcite[cf.][\nopage wp.\
  §2.3]{Mpl20OsiLicense2013a}. The OSLiC rewrites the message as the
  interdiction to promote own services and products by and with such elements.
  
  \item The MPL also generally prescribes that \enquote{you may not remove or
  alter the substance of any license notice (including copyright notices, patent
  notices, disclaimer of warranties, or limitations of liabiliy) contained
  within the Source Code Form [\ldots]}\footcite[cf.][\nopage wp.\
  §3.4]{Mpl20OsiLicense2013a}. This focussing to the \enquote{substance of any
  license notice} refers to the allowance to \enquote{[\ldots] alter any license
  notices to the extent required to remedy known factual
  innacuracies}\footcite[cf.][\nopage wp.\ §3.4]{Mpl20OsiLicense2013a}.
  Following its principle to offer one reliable way and to ignore variants of
  secondary importance, the OSLiC simplifies this condition to the general
  proscription to modify any licensing material -- namely for all use cases
  [MPL-C1 - MPL-C9]. But for emphasizing that this is a job which must be activily
  done, the OSLiC additionally rewrites this interdiction into all
  \emph{2others} use cases [MPL-C2 - MPL-C9] as the task to retain the licensing
  notices in the form one has obtained them.
  
  \item Moreover, the MPL requires for all \enquote{distributions of [the]
  source [code] form} that all modifications of the software \enquote{[\ldots]
  must be under the terms of (the MPL)} and that the distributor
  \enquote{[\ldots] must inform} all \enquote{recipients} that the software
  \enquote{[\ldots] is governed by the terms of (the MPL), and how (the
  recipients) can obtain a copy of this license}\footcite[cf.][\nopage wp.\
  §3.1]{Mpl20OsiLicense2013a}. For the respective use case (MPL-C2, MPL-C4, MPL-C6,
  MPL-C8), the OSLiC rewrites these conditions so that each MPL source code
  package must mandatorily contain the MPL itself as textfile and an additional
  licensing file or statement strictly following the text given by the addendum
  of the MPL\footcite[cf.][\nopage wp.\ Exhibit A]{Mpl20OsiLicense2013a}. Because
  the MPL is 'only' a license with weak copyleft, the OSLiC proposes to separate
  the MPL licensed, embedded component from the overarching program (MPL-C8).
  
  \item But the MPL does not explicitly require to mark all modifications.
  Nevertheless, this is state of the art in computer emgineering. Therefore,
  with respect to the cases of distributing modified source code (MPL-C4, MPL-C6
  and MPL-C8), the OSLiC proposes to mark all modifications inside of the source
  code and to update the description of the functional changes. In case of
  distributing the modified software in the form of binaries, it should be
  sufficient, to describe the modifications only on the functional level.
  
  \item Furthermore, the MPL requires that the \enquote{Covered Software} -- in
  all cases of distributing it in an \enquote{Executable Form} (MPL-C3, MPL-C5,
  MPL-C7, MPL-C9) -- \enquote{[\ldots] must also be made available in Source Code
  Form [\ldots]} and that the distributor \enquote{[\ldots] must inform
  recipients of the Executable Form how they can obtain a copy of such Source
  Code Form by reasonable means in a timely manner, at a charge no more than the
  cost of distribution to the recipient}\footcite[cf.][\nopage wp.\
  §3.2.a]{Mpl20OsiLicense2013a}. The OSLiC rewrites these conditions as the
  obligation to offer a download service at no charge and to point towards this
  services inside of the distributed package.
  
  \item In this context, the MPL allows to distribute the binaries under terms
  of another license \enquote{[\ldots] provided that that the license for the
  Executable Form does not attempt to limit or alter the recipients’ rights in
  the Source Code Form under this License}\footcite[cf.][\nopage
  wp.\ §3.2.b]{Mpl20OsiLicense2013a}. This possibility might become important for
  those cases where the license compatibility must explicitly be managed.
  Normally, it should be sufficient also to distribute the binaries under the
  MPL. Thus, in case of distributing binaries (MPL-C3, MPL-C5, MPL-C7, MPL-C9), the
  OSLiC proposes to insert into the distribution packages the MPL itself and an
  additional licensing file or statement strictly following the text given by
  the addendum of the MPL\footcite[cf.][\nopage wp.\ Exhibit
  A]{Mpl20OsiLicense2013a}. But again, because the MPL is 'only' a license with
  weak copyleft, the OSLiC proposes to separate the MPL licensed embedded
  component from the overarching program (MPL-C9).
  
  
  \item Finally, one clearly has to state that this rule above evokes a real
  source code distribution which therefore must follow the rules of distributing
  the software. Thus, the OSLiC requires in all cases of a binary distribution
  to execute also the task-lists of the respective source code use cases.

\end{itemize}

%\bibliography{../../../bibfiles/oscResourcesEn}

% Local Variables:
% mode: latex
% fill-column: 80
% End:
