% Telekom osCompendium 'for being included' snippet template
%
% (c) Karsten Reincke, Deutsche Telekom AG, Darmstadt 2011
%
% This LaTeX-File is licensed under the Creative Commons Attribution-ShareAlike
% 3.0 Germany License (http://creativecommons.org/licenses/by-sa/3.0/de/): Feel
% free 'to share (to copy, distribute and transmit)' or 'to remix (to adapt)'
% it, if you '... distribute the resulting work under the same or similar
% license to this one' and if you respect how 'you must attribute the work in
% the manner specified by the author ...':
%
% In an internet based reuse please link the reused parts to www.telekom.com and
% mention the original authors and Deutsche Telekom AG in a suitable manner. In
% a paper-like reuse please insert a short hint to www.telekom.com and to the
% original authors and Deutsche Telekom AG into your preface. For normal
% quotations please use the scientific standard to cite.
%
% [ Framework derived from 'mind your Scholar Research Framework' 
%   mycsrf (c) K. Reincke 2012 CC BY 3.0  http://mycsrf.fodina.de/ ]
%


%% use all entries of the bibliography
%\nocite{*}

\section{AGPL licensed software}

\agplUseCaseFinder{AGPL}{3.0}

%% ============================================================================= 
%% Common Building Blocks

\newcommand{\useCaseOne}{%
  \agtbUseCaseOne{AGPL-\ver}
  \agtbCoversOne{AGPL-\ver}}

\newcommand{\useCaseTwo}{%
  \gtbUseCaseTwo{AGPL-\ver}
  \gtbCoversTwo{AGPL-\ver}}

\newcommand{\useCaseThree}{%
  \gtbUseCaseThree{AGPL-\ver}
  \gtbCoversThree{AGPL-\ver}}

\newcommand{\useCaseFour}{%
  \gtbUseCaseFour{AGPL-\ver}{an}
  \gtbCoversFour{AGPL-\ver}}

\newcommand{\useCaseFive}{%
  \gtbUseCaseFive{AGPL-\ver}{an}
  \gtbCoversFive{AGPL-\ver}}

\newcommand{\useCaseSix}{%
  \gtbUseCaseSix{AGPL-\ver}{an}
  \gtbCoversSix{AGPL-\ver}}

\newcommand{\useCaseSeven}{%
  \gtbUseCaseSeven{AGPL-\ver}{an}
  \gtbCoversSeven{GPL-\ver}}

\newcommand{\useCaseEight}{%
  \gtbUseCaseEight{AGPL-\ver}{an}
  \gtbCoversEight{AGPL-\ver}}

\newcommand{\useCaseNine}{%
  \gtbUseCaseNine{AGPL-\ver}{an}
  \gtbCoversNine{AGPL-\ver}}

\newcommand{\useCaseA}{%
  \gtbUseCaseA{AGPL-\ver}{an}
  \gtbCoversA{AGPL-\ver}}

\newcommand{\useCaseB}{%
  \gtbUseCaseB{AGPL-\ver}{an}
  \gtbCoversB{AGPL-\ver}}

\newcommand{\useCaseC}{%
  \agtbUseCaseC{AGPL-\ver}
  \agtbCoversC{AGPL-\ver}}

\newcommand{\useCaseD}{%
  \agtbUseCaseD{AGPL-\ver}
  \agtbCoversD{AGPL-\ver}}
  
% ------------------------------------------------------------------------------
% Common Text Blocks from 0600-commomn-text-blocks.tex

\newcommand{\keepLicenseElements}{\gtbKeepLicenseElements{AGPL-\ver}}
\newcommand{\addToDocumentation}{\gtbAddToDocumentation{AGPL-\ver}}
\newcommand{\giveLicense}{\gtbGiveLicense{AGPL-\ver}}
\newcommand{\retainCopyrightNotices}{\gtbKeepCopyrightNotices{AGPL-\ver}}
\newcommand{\describeHowToGetSource}{\gtbDescribeHowToGetSource{AGPL-\ver}}
\newcommand{\createChangelog}{\gtbCreateChangelog{AGPL-\ver}}
\newcommand{\markEmbeddedModifications}{\gtbMarkEmbeddedModifications{AGPL-\ver}}
\newcommand{\markLibraryModifications}{\gtbMarkLibraryModifications{AGPL-\ver}}
\newcommand{\markProgramModifications}{\gtbMarkProgramModifications{AGPL-\ver}}
\newcommand{\gpltwoEnsureCopyrightNoticeSource}{\gtbVTwoCopyrightNotice{AGPL-2.0}{source code}}
\newcommand{\gpltwoEnsureCopyrightNoticeBinary}{\gtbVTwoCopyrightNotice{AGPL-2.0}{binary}}
\newcommand{\gplthreeEnsureCopyrightNoticeSource}{\gtbVThreeCopyrightNotice{AGPL-3.0}{source code}}
\newcommand{\gplthreeEnsureCopyrightNoticeBinary}{\gtbVThreeCopyrightNotice{AGPL-3.0}{binary}}
\newcommand{\makeUnmodifiedSourceAvailable}{\gtbMakeUnmodifiedSourceAvailable{AGPL-\ver}} 
\newcommand{\makeModifiedSourceAvailable}{\gtbMakeModifiedSourceAvailable{AGPL-\ver}}
\newcommand{\makeExecModifiedSourceAvailable}{\agtbMakeModifiedSourceAvailable{AGPL-\ver}} 
\newcommand{\makeAllSourcesAvailable}{\gtbMakeAllSourcesAvailable{AGPL-\ver}}
\newcommand{\makeExecAllSourcesAvailable}{\agtbMakeAllSourcesAvailable{AGPL-\ver}}
\newcommand{\arrangeProgramChanges}{\gtbArrangeProgramChanges{AGPL-\ver}}
\newcommand{\arrangeLibraryChanges}{\gtbArrangeLibraryChanges{AGPL-\ver}}
\newcommand{\arrangeEmbeddedChanges}{\gtbArrangeEmbeddedChanges{AGPL-\ver}}
\newcommand{\howToApplyTheseTerms}{\gtbHowToApplyTheseTerms{AGPL-\ver}}
\newcommand{\noPatentLitigation}{\gtbNoPatentLitigation{AGPL-\ver}}
\newcommand{\addToCopyrightDialogLib}{\gtbAddToCopyrightDialogStrongCopyleft{AGPL-\ver}}
\newcommand{\addToCopyrightDialogApp}{\gtbAddToCopyrightDialogApp{AGPL-\ver}}

% ------------------------------------------------------------------------------
% Make sure, licensing statements apply to enclosing program

\newcommand{\auxArrange}[1]{Arrange the #1 of the on-top development in a way
  that they are covered by the AGPL-\ver{} licensing statements.} 

\newcommand{\arrangeEnclosingBinaries}{%
  \auxArrange{the binaries of the on-top development}}

\newcommand{\arrangeEnclosingSources}{%
  \auxArrange{the sources of the on-top development}}


%% =============================================================================
%% AGPL-3.0 Use Cases

\newcommand{\ver}{3.0}

\begin{license}{AGPL3} 
\licensename{AGPL-3.0}
\licensespec{GNU Affero General Public License Version 3}
\licenseabbrev{AGPL}
%\licenseversion{3.0}

% ------------------------------------------------------------------------------
\subsection{AGPL-\ver-C1: Using the software only for yourself under additional restrictions}
\begin{lsuc}{AGPL-\ver-C1}
  \linkosuc{01}
  \linkosuc{03L} 
  \linkosuc{06L}
  \linkosuc{09L}
  \label{OSUC-01-AGPL}
  \label{OSUC-03L-AGPL}
  \label{OSUC-06L-AGPL}
  \label{OSUC-09L-AGPL}
      
  \useCaseOne

  \begin{lsucrequiresnothing}
    \lsucitem{You are allowed to execute an unmodified AGPL program without
    being obliged to do anything, as long as you do not give the program to
    third parties. And you are allowed to embed any AGPL licensed library, 
    snippet or module into your own program and to execute that program without
    being obliged to do anything, as long as no other than you can interact with
    it remotely through a computer network and as long as you do not give the
    library or your program to third parties.}
  \end{lsucrequiresnothing}

  \begin{lsucprohibits}
    \lsucitem{\noPatentLitigation}
  \end{lsucprohibits}
\end{lsuc}

% ------------------------------------------------------------------------------
\subsection{AGPL-\ver-C2: Passing the unmodified software as independent sources}
\begin{lsuc}{AGPL-\ver-C2}
  \linkosuc{02S}
  \linkosuc{05S}
  \label{OSUC-02S-AGPL}
  \label{OSUC-05S-AGPL}

  \useCaseTwo

  \begin{lsucrequires}
    \lsucmandatory{\keepLicenseElements}
    \lsucmandatory{\gplthreeEnsureCopyrightNoticeSource}
    \lsucmandatory{\giveLicense}\passingFilesCorrectly
    \lsucmandatory{\retainCopyrightNotices}
    \lsucoptional{\addToDocumentation}
  \end{lsucrequires}

  \begin{lsucprohibits}
    \lsucitem{\noPatentLitigation}
  \end{lsucprohibits}
\end{lsuc}

% ------------------------------------------------------------------------------
\subsection{AGPL-\ver-C3: Passing the unmodified software as independent binaries} 
\begin{lsuc}{AGPL-\ver-C3}
  \linkosuc{02B} 
  \linkosuc{05B}
  \label{OSUC-02B-AGPL}
  \label{OSUC-05B-AGPL}


  \useCaseThree

  \begin{lsucrequires}
    \lsucmandatory{\keepLicenseElements}
    \lsucmandatory{\gplthreeEnsureCopyrightNoticeBinary}
    \lsucmandatory{\giveLicense}\passingFilesCorrectly  
    \lsucmandatory{\makeUnmodifiedSourceAvailable}
    \lsucmandatory{\describeHowToGetSource}
    \lsucmandatory{\retainCopyrightNotices}
    \lsucsourcedist{AGPL-\ver-C2}
    \lsucoptional{\addToDocumentation}
  \end{lsucrequires}

  \begin{lsucprohibits}
    \lsucitem{\noPatentLitigation}
  \end{lsucprohibits}
\end{lsuc}

% ------------------------------------------------------------------------------
\subsection{AGPL-\ver-C4: Passing the unmodified library as embedded sources}
\begin{lsuc}{AGPL-\ver-C4}
  \linkosuc{07S} 
  \label{OSUC-07S-AGPL}

  \useCaseFour

  \begin{lsucrequires}
    \lsucmandatory{\keepLicenseElements}
    \lsucmandatory{\gplthreeEnsureCopyrightNoticeSource}
    \lsucmandatory{\giveLicense}\passingFilesCorrectly
    \lsucmandatory{\retainCopyrightNotices}
    \lsucmandatory{\addToCopyrightDialogLib}
    \lsucmandatory{\arrangeEnclosingSources}
    \lsucoptional{\addToDocumentation}
  \end{lsucrequires}

  \begin{lsucprohibits}
    \lsucitem{\noPatentLitigation}
  \end{lsucprohibits}
\end{lsuc}

% ------------------------------------------------------------------------------
\subsection{AGPL-\ver-C5: Passing the unmodified library as embedded binaries} 
\begin{lsuc}{AGPL-\ver-C5}
  \linkosuc{07B} 
  \label{OSUC-07B-AGPL}

  \useCaseFive

  \begin{lsucrequires}
    \lsucmandatory{\keepLicenseElements}
    \lsucmandatory{\gplthreeEnsureCopyrightNoticeBinary}
    \lsucmandatory{\giveLicense}\passingFilesCorrectly
    \lsucmandatory{\makeAllSourcesAvailable}
    \lsucmandatory{\describeHowToGetSource}
    \lsucmandatory{\addToCopyrightDialogLib}
    \lsucmandatory{\arrangeEnclosingBinaries}
    \lsucmandatory{\retainCopyrightNotices}
    \lsucsourcedist{AGPL-\ver-C4}
    \lsucoptional{\addToDocumentation}
  \end{lsucrequires}

  \begin{lsucprohibits}
    \lsucitem{\noPatentLitigation}
  \end{lsucprohibits}
\end{lsuc}

% ------------------------------------------------------------------------------
\subsection{AGPL-\ver-C6: Passing a modified program as source code}
\begin{lsuc}{AGPL-\ver-C6}
  \linkosuc{04S} 
  \label{OSUC-04S-AGPL}

  \useCaseSix

  \begin{lsucrequires}
    \lsucmandatory{\keepLicenseElements}
    \lsucmandatory{\gplthreeEnsureCopyrightNoticeSource}
    \lsucmandatory{\giveLicense}\passingFilesCorrectly
    \lsucmandatory{\retainCopyrightNotices}
    \lsucmandatory{\addToCopyrightDialogApp}
    \lsucmandatory{\markProgramModifications}
    \lsucmandatory{\arrangeProgramChanges}\howToApplyTheseTerms
    \lsucoptional{\createChangelog}
    \lsucoptional{\addToDocumentation}
  \end{lsucrequires}

  \begin{lsucprohibits}
    \lsucitem{\noPatentLitigation}
  \end{lsucprohibits}
\end{lsuc}

% ------------------------------------------------------------------------------
\subsection{AGPL-\ver-C7: Passing a modified program as binary}
\begin{lsuc}{AGPL-\ver-C7}
  \linkosuc{04B}
  \label{OSUC-04B-AGPL}

  \useCaseSeven

  \begin{lsucrequires}
    \lsucmandatory{\keepLicenseElements}
    \lsucmandatory{\gplthreeEnsureCopyrightNoticeBinary}
    \lsucmandatory{\giveLicense}\passingFilesCorrectly
    \lsucmandatory{\retainCopyrightNotices}
    \lsucmandatory{\markProgramModifications}
    \lsucmandatory{\addToCopyrightDialogApp}
    \lsucmandatory{\arrangeProgramChanges}\howToApplyTheseTerms
    \lsucmandatory{\makeModifiedSourceAvailable}
    \lsucmandatory{\describeHowToGetSource}
    \lsucsourcedist{AGPL-\ver-C6}
    \lsucoptional{\createChangelog}
    \lsucoptional{\addToDocumentation}
  \end{lsucrequires}

  \begin{lsucprohibits}
    \lsucitem{\noPatentLitigation}
  \end{lsucprohibits}
\end{lsuc}

% ------------------------------------------------------------------------------
\subsection{AGPL-\ver-C8: Passing a modified library as independent source code}
\begin{lsuc}{AGPL-\ver-C8}
  \linkosuc{08S}
  \label{OSUC-08S-AGPL}

  \useCaseEight

  \begin{lsucrequires}
     \lsucmandatory{\keepLicenseElements}
    \lsucmandatory{\gplthreeEnsureCopyrightNoticeSource}
    \lsucmandatory{\giveLicense}\passingFilesCorrectly
    \lsucmandatory{\retainCopyrightNotices}
    \lsucmandatory{\markLibraryModifications}
    \lsucmandatory{\arrangeLibraryChanges}\howToApplyTheseTerms
    \lsucoptional{\createChangelog}
    \lsucoptional{\addToDocumentation}
  \end{lsucrequires}

  \begin{lsucprohibits}
    \lsucitem{\noPatentLitigation}
  \end{lsucprohibits}
\end{lsuc}

% ------------------------------------------------------------------------------
\subsection{AGPL-\ver-C9: Passing a modified library as independent binary}
\begin{lsuc}{AGPL-\ver-C9}
  \linkosuc{08B}
  \label{OSUC-08B-AGPL}

  \useCaseNine

  \begin{lsucrequires}
    \lsucmandatory{\keepLicenseElements}
    \lsucmandatory{\gplthreeEnsureCopyrightNoticeSource}  
    \lsucmandatory{\giveLicense}\passingFilesCorrectly
    \lsucmandatory{\retainCopyrightNotices}
    \lsucmandatory{\makeModifiedSourceAvailable}
    \lsucmandatory{\describeHowToGetSource}
    \lsucsourcedist{AGPL-\ver-C8}
    \lsucmandatory{\markLibraryModifications}
    \lsucmandatory{\arrangeLibraryChanges}\howToApplyTheseTerms
    \lsucoptional{\createChangelog}
    \lsucoptional{\addToDocumentation}
  \end{lsucrequires}

  \begin{lsucprohibits}
    \lsucitem{\noPatentLitigation}
  \end{lsucprohibits}
\end{lsuc}

% ------------------------------------------------------------------------------
\subsection{AGPL-\ver-CA: Passing a modified library as embedded source code}
\begin{lsuc}{AGPL-\ver-CA}
  \linkosuc{10S}
  \label{OSUC-10S-AGPL}

  \useCaseA

  \begin{lsucrequires}
    \lsucmandatory{\keepLicenseElements}
    \lsucmandatory{\gplthreeEnsureCopyrightNoticeSource}
    \lsucmandatory{\giveLicense}\passingFilesCorrectly
    \lsucmandatory{\retainCopyrightNotices}
    \lsucmandatory{\addToCopyrightDialogLib}
    \lsucmandatory{\markEmbeddedModifications}
    \lsucmandatory{\arrangeEmbeddedChanges}\howToApplyTheseTerms
    \lsucmandatory{\arrangeEnclosingSources}
    \lsucoptional{\createChangelog}
    \lsucoptional{\addToDocumentation}
  \end{lsucrequires}

  \begin{lsucprohibits}
    \lsucitem{\noPatentLitigation}
  \end{lsucprohibits}
\end{lsuc}

% ------------------------------------------------------------------------------
\subsection{AGPL-\ver-CB: Passing a modified library as embedded binary}
\begin{lsuc}{AGPL-\ver-CB}
  \linkosuc{10B}
  \label{OSUC-10B-AGPL}
  
  \useCaseB

  \begin{lsucrequires}
    \lsucmandatory{\keepLicenseElements}
    \lsucmandatory{\gplthreeEnsureCopyrightNoticeBinary}
    \lsucmandatory{\giveLicense}\passingFilesCorrectly
    \lsucmandatory{\retainCopyrightNotices}
    \lsucmandatory{\makeAllSourcesAvailable}
    \lsucmandatory{\describeHowToGetSource}
    \lsucsourcedist{AGPL-\ver-CA}
    \lsucmandatory{\addToCopyrightDialogLib}
    \lsucmandatory{\markEmbeddedModifications}
    \lsucmandatory{\arrangeEmbeddedChanges}\howToApplyTheseTerms
    \lsucmandatory{\arrangeEnclosingBinaries}
    \lsucoptional{\createChangelog}
    \lsucoptional{\addToDocumentation}
  \end{lsucrequires}

  \begin{lsucprohibits}
    \lsucitem{\noPatentLitigation}
  \end{lsucprohibits}
\end{lsuc}


% ------------------------------------------------------------------------------
\subsection{AGPL-\ver-CC: Executing a modified program with network interaction}
\begin{lsuc}{AGPL-\ver-CC}
  \linkosuc{03N}
  \label{OSUC-03N-AGPL}
  
  \useCaseC

  \begin{lsucrequires}
    \lsucmandatory{\keepLicenseElements}
    \lsucmandatory{\gplthreeEnsureCopyrightNoticeBinary}
    \lsucmandatory{\giveLicense}\passingFilesCorrectly
    \lsucmandatory{\retainCopyrightNotices}
    \lsucmandatory{\markProgramModifications}
    \lsucmandatory{\addToCopyrightDialogApp}
    \lsucmandatory{\arrangeProgramChanges}\howToApplyTheseTerms
    \lsucmandatory{\makeExecModifiedSourceAvailable}
    \lsucmandatory{\describeHowToGetSource}
    \lsucsourcedist{AGPL-\ver-C6}
    \lsucoptional{\createChangelog}
    \lsucoptional{\addToDocumentation}
  \end{lsucrequires}

  \begin{lsucprohibits}
    \lsucitem{\noPatentLitigation}
  \end{lsucprohibits}
\end{lsuc}

\subsection{AGPL-\ver-CD: Executing a (modified) library as embedded component
with network interaction}
\begin{lsuc}{AGPL-\ver-CD}
  \linkosuc{09N}
  \linkosuc{06N}

  \label{OSUC-06N-AGPL}
  \label{OSUC-09N-AGPL}
  
  
  \useCaseD

  \begin{lsucrequires}
    \lsucmandatory{\keepLicenseElements}
    \lsucmandatory{\gplthreeEnsureCopyrightNoticeBinary}
    \lsucmandatory{\giveLicense}\passingFilesCorrectly
    \lsucmandatory{\retainCopyrightNotices}
    \lsucmandatory{\makeExecAllSourcesAvailable}
    \lsucmandatory{\describeHowToGetSource}
    \lsucsourcedist{AGPL-\ver-CA}
    \lsucmandatory{\addToCopyrightDialogLib}
    \lsucmandatory{\markEmbeddedModifications}
    \lsucmandatory{\arrangeEmbeddedChanges}\howToApplyTheseTerms
    \lsucmandatory{\arrangeEnclosingBinaries}
    \lsucoptional{\createChangelog}
    \lsucoptional{\addToDocumentation}
  \end{lsucrequires}

  \begin{lsucprohibits}
    \lsucitem{\noPatentLitigation}
  \end{lsucprohibits}
\end{lsuc}
% ------------------------------------------------------------------------------
\end{license}

%% =============================================================================
%% Discussion

\subsection{Discussions and Explanations}
\label{AGPL3Discussion}
  
For simplifying the justifications of our AGPL interpretation, we can state,
that the AGPL-3.0 and the GPL-3.0 are very similar: apart from some differences
caused by the varying names and passings remarks\footnote{Very similar are the
preamble and §0.  Compare \cite[][\nopage wp.]{Gpl30OsiLicense2007a} versus
\cite[][\nopage wp.]{Agpl30OsiLicense2007a}}, the most paragraphs of the two
licenses exactly offer the same text\footnote{Equal are §1 - 12 and §14 - §17.
Compare \cite[][\nopage wp.]{Gpl30OsiLicense2007a} versus \cite[][\nopage
wp.]{Agpl30OsiLicense2007a}}. Only the §13 of the AGPL-3.0 does not match to the
§13 of the GPL-3.0: §13 of the GPL-3.0 permits \enquote{[\ldots] to link or
combine any covered work with a work lincesed under version 3 of the GNU Affero
Generasl Public License}\footcite[cf.][\nopage wp. §13]{Gpl30OsiLicense2007a};
while §13 of the AGPL-3.0 deals with the \enquote{remote network
interaction}\footcite[cf.][\nopage wp. §13]{Agpl30OsiLicense2007a}. Therefore,
the analysis of the GPL-3.0 lincense\footnote{$\rightarrow$ p.
\pageref{GPL3Discussion}} is also valid for the AGPL-3.0; it is not necessary to
repeat that discussion here.

So, we can focus on the difference. The AGPL-3.0 tries to close a gap of the
GPL-3.0:

Purpose of all GNU licenses is to preserve the freedom to use, to study, to
share, and to modify the GNU programs and libraries\footcite[cf.][\nopage
wp.]{FsfFreeSoftware2015a}. These licenses want to prevent that users circumvent
the tasks which establish and maintain this freedom: Only if someone uses the
program / library only for himself, he shall not be obliged to do anything. But
if any third party was involved into the use of the GNU software, this third
party should receive all those rights and possibilities to use the software
which all the other users already have got.

In a time, where using the benefits of a program meant \emph{executing the
software on ons's own machine} (and hence \emph{having received the program at
least as a binary}), it was enough to let the obligations of -- for example --
handing over the license or the source code be triggered by the act of
'distributing the software'. Nowadays, in the times of cloud software systems,
users can let profit other users from the free software without conveying the
software. In these cases, they  execute the free program on their own machines,
but they nevertheless do not use the free program any longer only for
themselves. So, in time of cloud service technologies, the trigger of executing
the license fulfilling tasks must be complemented by a criterion which indicates
that a third party is involved into the context of using the software. And this
criteroin must no longer presuppose that this third party has received the
software itself.

For that purpose, the AGPL-3.0 states, that such an executed AGPL program must
\enquote{[\ldots] prominently offer all useres \emph{interacting with it
remotely through a computer network} [\ldots] an opportunity to receive the
Corresponding Source of your version by providing access to the Corresponding
Source from a network server at no charge, through some standard or customary
means of facilitating copying of software}\footcite[cf.][\nopage wp.
§13]{Agpl30OsiLicense2007a}. Obviously, the trigger of \emph{distributing the
AGPL software} now has been expanded by the feature \emph{being able to interact
with the AGPL software remotely through a computer network}.

The first consequence of this analysis is, that we can take over all the GPL
uses cases which deal with \emph{distributing the software} (2others) and all
the corresponding license fulfilling tasklists of
GPL-3-C2\footnote{$\rightarrow$ OSLiC, p. \pageref{OSUC-02S-GPL3}} until
GPL-3-CB\footnote{$\rightarrow$ OSLiC, p. \pageref{OSUC-10B-GPL3}} -- as we have
defined them in the GPL chapter.

The second consequence is, that we now have to subclassify the open source use
case \emph{recipient:4yourself}: we have to distinguish the use with internet
input-output access from that with only local input-output acesss.

Additionally, the AGPL limits the requirement to the condition, that the used
program is modified. The license exactly says that \enquote{[\ldots] if you
\emph{modify} the Program, your modfied version must prominently offer all
useres \emph{interacting with it remotely through a computer network} [\ldots]
an opportunity to receive the Corresponding Source of your version
[\ldots]}\footcite[cf.][\nopage wp. §13]{Agpl30OsiLicense2007a}. Thus, the third
consequence is, that we have to subclassify the open source use case 
\emph{recipient:4yourself} not only by the features \emph{ioAccess:viaInternet}
and \emph{ioAccess:onlyLocal}, but also by the features
\emph{state:modified} and \emph{state:unmodified}.

Finally, there is another little complication: One can only execute a program. A
library can not be directly executed. So, the question arises, what the user has
to be do if executes an own program which uses an unmodified AGPL licensed
library or module?

On the first glance, the license in §13 says only that he has to publish the
sources too, if executes a modified program. But on further reflection, one has
also to consider the other paragraphs of the AGPL: If one embeds an AGPL
licensed library, snippet or module into an own program, then -- due to the
Copyleft effect of the AGPL -- this program which uses the library, snippet or
module, has to be licensed under the AGPL too. And finally, every new program
has to be regarded as a modification of the first empty file. In other words:
one can only execute an own program using an unmodified AGPL library compliantly,
if one respects the §13 for the complete software complex being comprised of the
library itself and the pure code of the overarching program.

Based on this analysis, we had only to introduce two new AGPL specific open
source use cases and could recycle the complete set of GPL specific open source
use cases:

\begin{itemize}
  \item All GPL-3.0 use cases triggered by the distribution of the software
  \emph{recipient:2others} are transfered into the AGPL-3.0 finder and the
  AGPL-3.0 tasklist chapter as they have been defined in the GPL finder and the
  GPL-3.0 tasklist chapter.
  \item All combinations of \emph{recipient:4yourself} and
  \emph{ioAccess:onlyLocally} are covered by the old GPL 'yourself' use case
  which says, that one has not do anything as long as one uses the software only
  for oneself. But in the context of AGPL, this use case has additional
  conditions: one has not do anything if one does not distribute the software to
  other parties in any thing and if one executes this software on one's own
  machines in an environment which does not allow anyone else than
  oneself to interact with it remotely through a computer network.
  \item If one executes an unmodified AGPL program, which one has received and
  which one has not modified, then one also has not do anything.
  \item If one 'executes' an unmodified library as an embedded component of the
  really executed overarching program, then one has also to license this
  overarching program under the AGPL and hence has to fulfill the conditions of §13.
  \item If one executes a modified AGPL program, which one has received, has to
  fulfill the conditions of §13.
  \item If one executes an modified library as an embedded component of the
  really executed overarching program, then one has also to license this
  overarching program under the AGPL and hence has to fulfill the conditions of
  §13 with respect to bot parts, to the overarching program and the library.
\end{itemize}

There is a last point, which should also be discussed here. It concerns the
question of granularity:

The AGPL-3.0 requires that the \enquote{[\ldots] modified version (of an
[executed] program) must prominently offer all users interacting with it remotely through a
computer network [\ldots] an opportunity to receive the Corresponding Source of
your version by providing access to the Corresponding Source from a network
server at no charge [\ldots]}\footcite[cf.][\nopage wp.
§13]{Agpl30OsiLicense2007a}. For respecting this rule, one has to know what the
term \emph{Corresponding Source} means: how many of the embedded components of the
program must be conveyed together with the overarching program.

Fortunately, the AGPL-3.0 (and the GPL-3.0) defines the used terms: \enquote{The
\enquote{Corresponding Source} for a work in object code form means all the
source code needed to generate, install, and (for an executable work) run the
object code and to modify the work, including scripts to control those
activities.\footcite[cf.][\nopage wp. §1]{Agpl30OsiLicense2007a}} If one took
this statements seriously, one would have to \enquote{provide access to} the
complete software stack of the executed AGPL program -- just down to the glibc.

But the AGPL does not want to be to greedy. Therefore it limits the scope by
determining, that the \emph{Corresponding Source} \enquote{[\ldots] does not
include the work's System Libraries, or general-purpose tools or generally
available free programs which are used unmodified in performing those activities
but which are not part of the work}\footcite[cf.][\nopage wp.
§1]{Agpl30OsiLicense2007a}. For understanding this rule, one has to know, what
the term \emph{System Libraries} means. The AGPl says, that \enquote{the
\enquote{System Libraries} of an executable work include anything, other than
the work as a whole, that (a) is included in the normal form of packaging a
Major Component, but which is not part of that Major Component, and (b) serves
only to enable use of the work with that Major Component, or to implement a
Standard Interface for which an implementation is available to the public in
source code form.\footcite[cf.][\nopage wp. §1]{Agpl30OsiLicense2007a}}
Unfortunately, one has now to analyse, what the AGPL defines as a \emph{Major
Component}: \enquote{A enquote{Major Component}, in this context, means a major
essential component (kernel, window system, and so on) of the specific operating
system (if any) on which the executable work runs, or a compiler used to produce
the work, or an object code interpreter used to run it\footcite[cf.][\nopage wp.
§1]{Agpl30OsiLicense2007a}}.

Based on these specifications, one can give some rule of thumbs concerning the
question down to which level one has to give access to the corresponding source
code of an an executed AGPL program:
\begin{itemize}
  \item If one lets execute a modified AGPL licensed binary program, then one has
  to give access to the code of
  \begin{itemize}
  \item the executed program itself
  \item every modified embedded component of that program
  \item every not freely accessible embedded component of that program
  \item all not freely accessible tools, scripts, data which are necessary to
  compile the sources of the program in a freely accessible compilation /
  developement environment
  \end{itemize}
  But it is not necessary to give access to unmodified standard libraries,
  compilers, or tools which can freely be downloaded from their standard
  repositories.
  \item If one lets execute a modified AGPL licensed script, then one has
  to give access to the code of
  \begin{itemize}
  \item the executed script itself
  \item every modified embedded script component included by the main script
  \item every not freely accessible embedded script component included by the main script
  \item all not freely accessible tools, scripts, data which are necessary to
  to let that main script be executed by a freely accessible interpreter
  \item the interpreter itself if it is not freely accessible.
  \end{itemize}
  But it is not necessary to give access to unmodified standard script
  libraries, interpreters, or tools which can freely be downloaded from their
  standard repositories
  
\end{itemize}



