% Telekom osCompendium 'for being included' snippet template
%
% (c) Karsten Reincke, Deutsche Telekom AG, Darmstadt 2011
%
% This LaTeX-File is licensed under the Creative Commons Attribution-ShareAlike
% 3.0 Germany License (http://creativecommons.org/licenses/by-sa/3.0/de/): Feel
% free 'to share (to copy, distribute and transPGL)' or 'to remix (to adapt)'
% it, if you '... distribute the resulting work under the same or similar
% license to this one' and if you respect how 'you must attribute the work in
% the manner specified by the author ...':
%
% In an internet based reuse please link the reused parts to www.telekom.com and
% mention the original authors and Deutsche Telekom AG in a suitable manner. In
% a paper-like reuse please insert a short hint to www.telekom.com and to the
% original authors and Deutsche Telekom AG into your preface. For normal
% quotations please use the scientific standard to cite.
%
% [ Framework derived from 'mind your Scholar Research Framework' 
%   mycsrf (c) K. Reincke 2012 CC BY 3.0  http://mycsrf.fodina.de/ ]
%


%% use all entries of the bibliography
%\nocite{*}

\section{Postgres Licensed Software}

Like the MIT License Postgres License is a very permissive licenses. Thus, the
PGL specific finder can be simplified:

\begin{center}
\begin{footnotesize}
\pstree[levelsep=*1,treesep=0.2]{\Toval{PgL}}{
  \pstree{
    \Tr{\Ovalbox{\shortstack{recipient: \textit{4yourself}\\
    \textbf{\textit{used by yourself}}}}} 
  }{
    \Tr{\doublebox{\shortstack{\tiny{\textbf{PGL-C1}:}\\
    \tiny{\textit{using the}}\\\tiny{\textit{software}}\\
    \tiny{\textit{only for}}\\\tiny{\textit{yourself}} }}} 
  }
  \pstree[levelsep=*0.2,treesep=0.2]{
    \Tr{\Ovalbox{\shortstack{recipient: \textit{2others}\\
      \textbf{\textit{distributed to 3rd parties}}}}} 
  }{ 
    \pstree[levelsep=*0.2,treesep=0.2]{
      \Tr{\Ovalbox{\shortstack{state:\\\textbf{\textit{unmodified}}}}}
    }{
        \Tr{\doublebox{\shortstack{\tiny{\textbf{PGL-C2}:}\\
        \tiny{\textit{distributing an}}\\
        \tiny{\textit{unmodified pkg}} }}}

    }

    \pstree[levelsep=*0.2,treesep=0.2]{
      \Tr{\Ovalbox{\shortstack{state:\\\textbf{\textit{modified}}}}}
    }{ 
      \pstree{
        \Tr{\Ovalbox{\shortstack{type:\\\textbf{\textit{proapse}}}}}
      }{
          \Tr{\doublebox{\shortstack{\tiny{\textbf{PGL-C3}:}\\
          \tiny{\textit{distributing}}\\\tiny{\textit{a modified}}\\
          \tiny{\textit{program}} }}} 
           
      }
      \pstree{
        \Tr{\Ovalbox{\shortstack{type:\\\textbf{\textit{snimoli}}}}}
      }{
        \pstree{
          \Tr{\Ovalbox{\shortstack{context:\\\textbf{\textit{independent}}}}}
        }{        
            \Tr{\doublebox{\shortstack{\tiny{\textbf{PGL-C4}:}\\
            \tiny{\textit{distributing}}\\\tiny{\textit{a modified}}\\
            \tiny{\textit{library as}}\\\tiny{\textit{independent}}
            \tiny{\textit{pkg}} }}} 
           
        }
        \pstree{
          \Tr{\Ovalbox{\shortstack{context:\\\textbf{\textit{embedded}}}}}
        }{        
            \Tr{\doublebox{\shortstack{\tiny{\textbf{PGL-C5}:}\\
            \tiny{\textit{distributing}}\\\tiny{\textit{a modified}}\\
            \tiny{\textit{library as}}\\\tiny{\textit{embedded}}
            \tiny{\textit{pkg}} }}}     
        }
      }
    }
   }   
}
\end{footnotesize}
\end{center}

\subsection{PGL-C1: Using the software only for yourself}
\label{OSUC-01-PGL} 
\label{OSUC-03-PGL} 
\label{OSUC-06-PGL}
\label{OSUC-09-PGL}
  
\begin{description}

\item[means] that you are going to use a received PGL software only for yourself
and that you do not hand it over to any 3rd party in any sense.

\item[covers] OSUC-01, OSUC-03, OSUC-06, and OSUC-09\footnote{For details
$\rightarrow$ OSLiC, pp.\ \pageref{OSUC-01-DEF} - \pageref{OSUC-09-DEF}}

\item[requires] no tasks in order to fulfill the conditions of the PGL with
respect to this use case:
  \begin{itemize}
    \item You are allowed to use any kind of PGL licensed software in any sense
    and in any context without any other obligations if you do not handover the
    software to 3rd parties and if you do not modify the existing copyright
    notes and the existing permission notice.
  \end{itemize}

\item[prohibits] nothing explicitly.

\end{description}

\subsection{PGL-C2: Passing the unmodified software}
\label{OSUC-02S-PGL} \label{OSUC-05S-PGL} \label{OSUC-07S-PGL} 
\label{OSUC-02B-PGL} \label{OSUC-05B-PGL} \label{OSUC-07B-PGL} 

\begin{description}

\item[means] that you are going to distribute an unmodified version of the
received PGL software to 3rd parties -- regardless whether you distribute it in
the form of binaries or as source code files. In this case it is not
discriminating to distribute a program, an application, a server, a snippet, a
module, a library, or a plugin as an independent package.

\item[covers] OSUC-02S,  OSUC-02B, OSUC-05S, OSUC-05B, OSUC-07S,
OSUC-07B\footnote{For details $\rightarrow$ OSLiC, pp.\ \pageref{OSUC-02S-DEF} -
\pageref{OSUC-07B-DEF}}

\item[requires] the following tasks in order to fulfill the license conditions:
\begin{itemize}
  \item \textbf{[mandatory:]} Ensure that the complete Postgres License
  including the copyright notice, the permission notices, and the PGL disclaimer
  -- are retained in your package in the form you have received them.
  \item \textbf{[voluntary:]} It's a good tradition to let the documentation of
  your distribution and/or your additional material also contain a link to the
  original software (project) and its homepage.
\end{itemize}

\item[prohibits] nothing explicitly.

\end{description}

\subsection{PGL-C3: Passing a modified program}
\label{OSUC-04S-PGL} \label{OSUC-04B-PGL}

\begin{description}

\item[means] that you are going to distribute a modified version of the received
PGL program, application, or server (proapse) to 3rd parties -- regardless
whether you distribute it in the form of binaries or as source code files.
\item[covers] OSUC-04S, OSUC-04B,\footnote{For details $\rightarrow$ OSLiC, pp.\
\pageref{OSUC-04S-DEF}} 

\item[requires] the following tasks in order to fulfill the license conditions:
\begin{itemize}
  \item \textbf{[mandatory:]} Ensure that the complete Postgres License
  including the copyright notice, the permission notices, and the PGL disclaimer
  -- are retained in your package in the form you have received them.
  \item \textbf{[voluntary:]} Mark your modifications in the source code,
  regardless whether you want to distribute the code or not.
  \item \textbf{[voluntary:]} It's a good tradition to let the documentation of
  your distribution and/or your additional material also contain a link to the
  original software (project) and its homepage.
  \item \textbf{[voluntary:]} You can expand an existing copyright notice
  presented by the program with information about your own work or
  modifications.
  \item \textbf{[voluntary:]} It is a good practice of the open source
  community, to let the copyright notice which is shown by the program also
  state that it is based on a version originally licensed under the PGL license.
  Because you are already modifying the program, you can also add such a hint,
  if the presented original copyright notice lacks such a statement.
\end{itemize}

\item[prohibits] nothing explicitly.

\end{description}

\subsection{PGL-C4: Passing a modified library independently}
\label{OSUC-08S-PGL}\label{OSUC-08B-PGL}
\begin{description}

\item[means] that you are going to distribute a modified version of the received
PGL code snippet, module, library, or plugin (snimoli) to 3rd parties without
embedding it into another larger software unit -- regardless whether you
distribute it in the form of binaries or as source code files.

\item[covers] OSUC-08S, OSUC-08B\footnote{For details $\rightarrow$ OSLiC, pp.\
\pageref{OSUC-08B-DEF}}

\item[requires] the following tasks in order to fulfill the license conditions:
\begin{itemize}
  \item \textbf{[mandatory:]} Ensure that the complete Postgres License
  including the copyright notice, the permission notices, and the PGL disclaimer
  -- are retained in your package in the form you have received them.
  \item \textbf{[voluntary:]} Mark your modifications in the source code,
  regardless whether you want to distribute the code or not.
  \item \textbf{[voluntary:]} It's a good tradition to let the documentation of
  your distribution and/or your additional material also contain a link to the
  original software (project) and its homepage.
\end{itemize}

\item[prohibits] nothing explicitly.

\end{description}


\subsection{PGL-C5: Passing a modified library as embedded component}
\label{OSUC-10S-PGL} \label{OSUC-10B-PGL}

\begin{description}

\item[means] that you are going to distribute a modified version of the received
PGL code snippet, module, library, or plugin (snimoli) to 3rd parties together
with another larger software unit which contains this code snippet, module,
library, or plugin as an embedded component -- regardless whether you distribute
it in the form of binaries or as source code files.

\item[covers] OSUC-10S, OSUC-10B\footnote{For details $\rightarrow$ OSLiC, pp.\
\pageref{OSUC-10S-DEF}}

\item[requires] the following tasks in order to fulfill the license conditions:
\begin{itemize}
  \item \textbf{[mandatory:]} Ensure that the complete Postgres License
  including the copyright notice, the permission notices, and the PGL disclaimer
  -- are retained in your package in the form you have received them.

  \item \textbf{[voluntary:]} Mark your modifications in the source code,
  regardless whether you want to distribute the code or not.
  
  \item \textbf{[voluntary:]} It is a good practice of the open source
  community, to let the copyright notice which is shown by the running program
  also state that the program uses a component being licensed under the PGL
  license. And it is a good tradition to insert links to the homepage / download
  page of this used component.

  \item \textbf{[voluntary:]} It's also a good tradition to let the
  documentation of your program and/or your additional material also mention
  that you have used this component added by a link to the original software
  component and its homepage.
  
  \item \textbf{[voluntary:]} Arrange your distribution so that the original
  licensing elements -- esp.\ the PGL license text containing the specific
  copyright notices of the original author(s), the permission notices and the
  PGL disclaimer --  clearly refer only to the embedded library and do not
  disturb the licensing of your own overarching work. It's a good tradition to
  keep the libraries, modules, snippet, or plugins in specific directories which
  contain also all licensing elements.
  
\end{itemize}

\item[prohibits] nothing explicitly.

\end{description}

\subsection{Discussions and Explanations}

The PGL-License follows the structure of the MIT license: it also contains (1) a
copyright notice, (2) a paragraph saying that you are allowed to do almost
anything you want, followed (3) by the condition that the copyright notice, the
permission notes, and the disclaimer \enquote{[\ldots] apperar in all copies},
and (4) the well known disclaimer\footcite[cf.][\nopage wp]{PglOsiLicense2013a}.
Moreover, as the MIT license, the PGL doesn't talk about the difference of
source code and object code. So, you can transfer the MIT
analysis\footnote{$\rightarrow$ OSLiC, p. \pageref{sec:MitDiscussions}} to the
PGL analogically.



%\bibliography{../../../bibfiles/oscResourcesEn}
