% Telekom osCompendium 'for being included' snippet template
%
% (c) Karsten Reincke, Deutsche Telekom AG, Darmstadt 2011
%
% This LaTeX-File is licensed under the Creative Commons Attribution-ShareAlike
% 3.0 Germany License (http://creativecommons.org/licenses/by-sa/3.0/de/): Feel
% free 'to share (to copy, distribute and transPGL)' or 'to remix (to adapt)'
% it, if you '... distribute the resulting work under the same or similar
% license to this one' and if you respect how 'you must attribute the work in
% the manner specified by the author ...':
%
% In an internet based reuse please link the reused parts to www.telekom.com and
% mention the original authors and Deutsche Telekom AG in a suitable manner. In
% a paper-like reuse please insert a short hint to www.telekom.com and to the
% original authors and Deutsche Telekom AG into your preface. For normal
% quotations please use the scientific standard to cite.
%
% [ Framework derived from 'mind your Scholar Research Framework' 
%   mycsrf (c) K. Reincke 2012 CC BY 3.0  http://mycsrf.fodina.de/ ]
%


%% use all entries of the bibliography
%\nocite{*}

\section{PostgreSQL License}
\begin{license}{PGL} % ends at end of file
\licensename{PostgreSQL}
\licensespec{PostgreSQL license}
\licenseabbrev{PGL}

Like the MIT License Postgres License is a very permissive licenses. Thus, the
PostgreSQL specific finder can be simplified:

\tikzstyle{nodv} = [font=\small, ellipse, draw, fill=gray!10, 
    text width=2cm, text centered, minimum height=2em]

\tikzstyle{nods} = [font=\footnotesize, rectangle, draw, fill=gray!20, 
    text width=1.2cm, text centered, rounded corners, minimum height=3em]

\tikzstyle{nodb} = [font=\footnotesize, rectangle, draw, fill=gray!20, 
    text width=2.2cm, text centered, rounded corners, minimum height=3em]

\tikzstyle{nodx} = [font=\footnotesize, rectangle, draw, fill=gray!20, 
    text width=2.4cm, text centered, rounded corners, minimum height=3em]
    
\tikzstyle{leaf} = [font=\tiny, rectangle, draw, fill=gray!30, 
    text width=2cm, text centered, minimum height=4em]

\tikzstyle{edge} = [draw, -latex']

\begin{tikzpicture}[]

\node[nodv] (l61) at ( 2.4, 9.2) {PostgreSQL};

\node[nodb] (l51) at ( 0.0, 7.8) {\textit{recipient:} \\ \textbf{4yourself}};
\node[nodb] (l52) at ( 4.8, 7.8) {\textit{recipient:} \\ \textbf{2others}};

\node[nodb] (l41) at ( 2.5, 6.2) {\textit{state:} \\ \textbf{unmodified}};
\node[nodb] (l42) at ( 7.0, 6.2) {\textit{state:} \\ \textbf{modified}};

\node[nodb] (l31) at ( 5.0, 4.6) {\textit{type:} \\ \textbf{proapse}};
\node[nodb] (l32) at ( 9.0, 4.6) {\textit{type:} \\ \textbf{snimoli}};

\node[nodx] (l21) at ( 7.5, 2.8) {\textit{context:} \\ \textbf{independent}};
\node[nodx] (l22) at (10.5, 2.8) {\textit{context:} \\ \textbf{embedded}};

\node[leaf] (l11) at ( 0.0, 0.0) {\textbf{PostgreSQL-C1} \textit{using software only for yourself}};
\node[leaf] (l12) at ( 2.5, 0.0) {\textbf{PostgreSQL-C2} \textit{distributing unmodified package}};
\node[leaf] (l13) at ( 5.0, 0.0) {\textbf{PostgreSQL-C3} \textit{distributing modified program}};
\node[leaf] (l14) at ( 7.5, 0.0) {\textbf{PostgreSQL-C4} \textit{distributing modified library as independent package}};
\node[leaf] (l15) at (10.5, 0.0) {\textbf{PostgreSQL-C5} \textit{distributing modified library as embedded package}};


\path [edge] (l61) -- (l51);
\path [edge] (l61) -- (l52);
\path [edge] (l51) -- (l11);
\path [edge] (l52) -- (l41);
\path [edge] (l52) -- (l42);
\path [edge] (l41) -- (l12);
\path [edge] (l42) -- (l31);
\path [edge] (l42) -- (l32);
\path [edge] (l31) -- (l13);
\path [edge] (l32) -- (l21);
\path [edge] (l32) -- (l22);
\path [edge] (l21) -- (l14);
\path [edge] (l22) -- (l15);

\end{tikzpicture}

%% =============================================================================
%% Common Building Blocks
%%

% ------------------------------------------------------------------------------
% Include license in package

\newcommand{\giveLicense}{Ensure that the complete PostgreSQL license including
  the copyright notice, the permission notices, and the PostgreSQL disclaimer
  are retained in your package in the form you have received them.}

% ------------------------------------------------------------------------------
% Add a link to the project's home page to the documentation

\newcommand{\linkToHomepage}{It's a good tradition to let the documentation of
  your distribution or your additional material also contain a link to the
  original software (project) and its homepage.}

% ------------------------------------------------------------------------------
% mark your modifications

\newcommand{\markModifications}{Mark your modifications in the source code,
  regardless whether you want to distribute the code or not.}

% ------------------------------------------------------------------------------
% add name and link to copyright dialog

\newcommand{\addToCopyrightDialog}{It is a good practice of the open source
  community to let the copyright notice, which is shown by the running program,
  also state that the program uses a component being licensed under the
  PostgreSQL license.  And it is a good tradition to insert links to the
  homepage or download page of this embedded component.}

% ------------------------------------------------------------------------------
% separate the components but keep component and license together

\newcommand{\separateComponents}{Arrange your distribution so that the original
  licensing elements (in particular the PostgreSQL license text containing the
  copyright notices of the original author(s), the permission notices and the
  PostgrSGL disclaimer) clearly refer only to the embedded library and do not
  affect the licensing of your own overarching work. Consider keeping embedded
  libraries, modules, snippets, or plugins in separate directories which also
  contain all their licensing elements.}

% ------------------------------------------------------------------------------
% add your own copyright notice to the copyright dialog

\newcommand{\addYourOwnCopyright}{You can add information about your own work or
  modifications to an existing copyright notice presented by the program.}

% ------------------------------------------------------------------------------
% acknowledge the original work in the copyright notice

\newcommand{\acknowledgeOriginalWork}{It is a good practice of the open source
  community to let the copyright notice, which is shown by the program, also
  state that it is based on a version originally licensed under the PostgreSQL
  license. Because you are already modifying the program, you may want to add
  such a hint, if the original copyright notice lacks such a statement.}

%% =============================================================================
%% Use Cases
%%

\subsection{PostgreSQL-C1: Using the software only for yourself}
\begin{lsuc}{PGL-C1}
  \linkosuc{01} 
  \linkosuc{03} 
  \linkosuc{06}
  \linkosuc{09}
  
  \lsucmeans{that you received PostgreSQL licensed software, that you will use it
  only for yourself, and that you do not hand it over to any 3rd party in any
  sense.} 

  \lsuccovers{OSUC-01, OSUC-03, OSUC-06, and OSUC-09\footnote{For details
      $\rightarrow$ \oslic, pp.\ \pageref{OSUC-01-DEF} - \pageref{OSUC-09-DEF}}} 

  \begin{lsucrequiresnothing}
    \lsucitem{You are allowed to use any kind of PostgreSQL licensed software in any
      sense and in any context without any other obligations if you do not give
      the software to third parties and if you do not modify the existing
      copyright notices or the existing permission notice.}
  \end{lsucrequiresnothing}

  \lsucprohibitsnothing 

\end{lsuc}

% ------------------------------------------------------------------------------
\subsection{PostgreSQL-C2: Passing the unmodified software}
\begin{lsuc}{PGL-C2}
  \linkosuc{02S} 
  \linkosuc{05S} 
  \linkosuc{07S} 
  \linkosuc{02B} 
  \linkosuc{05B} 
  \linkosuc{07B} 

  \lsucmeans{that you received PostgreSQL licensed software which you are now going to
  distribute to third parties in the form of unmodified binaries or as unmodifed
  source code files. In this case it makes no difference if you distribute a
  program, an application, a server, a snippet, a module, a library, or a plugin
  as an independent package.} 

  \lsuccovers{OSUC-02S,  OSUC-02B, OSUC-05S, OSUC-05B, OSUC-07S, OSUC-07B%
    \footnote{For details $\rightarrow$ \oslic, 
      pp.\ \pageref{OSUC-02S-DEF} -- \pageref{OSUC-07B-DEF}}}

  \begin{lsucrequires}
    \lsucmandatory{\giveLicense}
    \lsucoptional{\linkToHomepage}
  \end{lsucrequires}

  \lsucprohibitsnothing
\end{lsuc}

% ------------------------------------------------------------------------------
\subsection{PostgreSQL-C3: Passing a modified program}
\begin{lsuc}{PGL-C3}
  \linkosuc{04S} 
  \linkosuc{04B}

  \lsucmeans{that you received a PostgreSQL licensed program, application, or
  server (proapse), that you modified it, and that you are now going to distribute this
  modified version to third parties in the form binaries or as source code
  files.}
 
  \lsuccovers{OSUC-04S, OSUC-04B%
    \footnote{For details $\rightarrow$ \oslic, pp.\ \pageref{OSUC-04S-DEF}}}

  \begin{lsucrequires}
    \lsucmandatory{\giveLicense}
    \lsucoptional{\markModifications}
    \lsucoptional{\linkToHomepage}
    \lsucoptional{\addYourOwnCopyright}
    \lsucoptional{\acknowledgeOriginalWork}
  \end{lsucrequires}

  \lsucprohibitsnothing
\end{lsuc}

% ------------------------------------------------------------------------------
\subsection{PostgreSQL-C4: Passing a modified library independently}
\begin{lsuc}{PGL-C4}
  \linkosuc{08S}
  \linkosuc{08B}

  \lsucmeans{that you received a PostgreSQL licensed code snippet, module, library, or
  plugin (snimoli), that you modified it, and that you are now going to
  distribute this modified version to third parties in the form of binaries or
  as source code files together with another larger software unit which contains
  this code snippet, module, library, or plugin as an embedded component,
  regardless whether you distribute it in the form of binaries or as source code
  files.}

  \lsuccovers{OSUC-08S, OSUC-08B%
    \footnote{For details $\rightarrow$ \oslic, pp.\ \pageref{OSUC-08B-DEF}}}

  \begin{lsucrequires}
    \lsucmandatory{\giveLicense}
    \lsucoptional{\markModifications}
    \lsucoptional{\linkToHomepage}
  \end{lsucrequires}

  \lsucprohibitsnothing
\end{lsuc}

% ------------------------------------------------------------------------------
\subsection{PostgreSQL-C5: Passing a modified library as embedded component}
\begin{lsuc}{PGL-C5}
  \linkosuc{10S} 
  \linkosuc{10B}

  \lsucmeans{that you received a PostgreSQL licensed code snippet, module, library, or
  plugin (snimoli), that you modified it, and that you are now going to
  distribute this modified version to third parties in the form of binaries or
  as source code files together with another larger software unit which contains
  this code snippet, module, library, or plugin as an embedded component,
  regardless whether you distribute it in the form of binaries or as source code
  files.}

  \lsuccovers{OSUC-10S, OSUC-10B%
    \footnote{For details $\rightarrow$ \oslic, pp.\ \pageref{OSUC-10S-DEF}}}

  \begin{lsucrequires}
    \lsucmandatory{\giveLicense}
    \lsucoptional{\markModifications}
    \lsucoptional{\addToCopyrightDialog} 
    \lsucoptional{\linkToHomepage}
    \lsucoptional{\separateComponents}
  \end{lsucrequires}

  \lsucprohibitsnothing
\end{lsuc}

%% =============================================================================
%% Discussion
%%

\subsection{Discussions and Explanations}
\label{PGLDiscussion}

The PostgreSQL-License follows the structure of the MIT license: it, too, contains 
(1) a copyright notice, 
(2) a paragraph saying that you are allowed to do almost anything you want,
    followed 
(3) by the condition that the copyright notice, the permission notes, and the
    disclaimer \enquote{[\ldots] apperar in all copies}, and 
(4) the well known disclaimer.\citePGL{}
Moreover, like the MIT license, the PostgreSQL does not talk about the
difference between source code and object code. So, you can apply the analysis
of the MIT license\footnote{$\rightarrow$ \oslic, p. \pageref{MITDiscussion}} 
also to the PostgreSQL.

\end{license}
%\bibliography{../../../bibfiles/oscResourcesEn}

% Local Variables:
% mode: latex
% fill-column: 80
% End:
