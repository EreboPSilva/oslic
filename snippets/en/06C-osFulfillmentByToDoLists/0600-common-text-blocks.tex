% Telekom osCompendium 'for being included' snippet template
%
% (c) Karsten Reincke, Deutsche Telekom AG, and Ronald Dauster, GIDO GmbH
%     Darmstadt 2014
%
% This LaTeX-File is licensed under the Creative Commons Attribution-ShareAlike
% 3.0 Germany License (http://creativecommons.org/licenses/by-sa/3.0/de/): Feel
% free 'to share (to copy, distribute and transmit)' or 'to remix (to adapt)'
% it, if you '... distribute the resulting work under the same or similar
% license to this one' and if you respect how 'you must attribute the work in
% the manner specified by the author ...':
%
% In an Internet based reuse please link the reused parts to www.telekom.com and
% mention the original authors and Deutsche Telekom AG in a suitable manner. In
% a paper-like reuse please insert a short hint to www.telekom.com and to the
% original authors and Deutsche Telekom AG into your preface. For normal
% quotations please use the scientific standard to cite.
%
% [ Framework derived from 'mind your Scholar Research Framework' 
%   mycsrf (c) K. Reincke 2012 CC BY 3.0  http://mycsrf.fodina.de/ ]
%

%% =============================================================================
%% All commands take (at least) one parameter: the name of the license, for
%% example, 'GPL-2.0'.  This parameter is always present, even if it isn't used
%% and it is always the first parameter.
%%
%% Commands with the prefix gtb (GPL Text Block) are common parametrized text
%% blocks for GPL, LGPL, and AGPL 

% ------------------------------------------------------------------------------
% Common description of license specific use cases

\newcommand{\gtbUseCaseOne}[1]{\lsucmeans{that you received #1 licensed
    software, that you will use it only for yourself, and that you do not hand
    over to any third party in any sense.}}

\newcommand{\gtbUseCaseTwo}[1]{\lsucmeans{that you received #1 licensed software 
    that you are now going to distribute to third parties as an independent unit
    and in the form of unmodified source code files or as an unmodified source
    code package. In this case it makes no difference if you distribute a
    program, an application, a server, a snippet, a module, a library, or a
    plugin.}}

\newcommand{\gtbUseCaseThree}[1]{\lsucmeans{that you received #1 licensed
    software, which you are now going to distribute to third parties as an
    independent unit and in the form of unmodified binary files or as an
    unmodified binary package. In this case it does not matter if you distribute
    a program, an application, a server, a snippet, a module, a library, or a
    plugin.}}

\newcommand{\gtbUseCaseFour}[1]{\lsucmeans{that you received a #1 licensed
    snippet, module or library that you are now going to distribute to third
    parties as an embedded component of a larger unit and in the form of
    unmodified source code files or as an unmodified source code package.}}

\newcommand{\gtbUseCaseFive}[1]{\lsucmeans{that you received a #1 licensed
    snippet, module or library that you are now going to distribute to third
    parties as an embedded component of a larger unit and in the form of
    unmodified binary files or as unmodified binary package.}}

\newcommand{\gtbUseCaseSix}[1]{\lsucmeans{that you received a #1 licensed
    program, application, or server (proapse), that you modified it, and that
    you are now going to distribute this modified version to third parties in
    the form of source code files or as a source code package.}}

\newcommand{\gtbUseCaseSeven}[1]{\lsucmeans{that you received a #1 licensed
    program, application, or server (proapse), that you modified it, and that
    you are now going todistribute this modified version to third parties in the
    form of binary files or as a binary package.}}

\newcommand{\gtbUseCaseEight}[1]{\lsucmeans{that you received a #1 licensed code
    snippet, module, library, or plugin (snimoli), that you modified it, and
    that you are now going to distribute this modified version to third parties
    in the form of source code files or as a source code package, but without
    embedding it into another larger software unit.}}

\newcommand{\gtbUseCaseNine}[1]{\lsucmeans{that you received a #1 licensed code
    snippet, module, library, or plugin (snimoli), that you modified it, and
    that you are now going to distribute this modified version to third parties
    in the form of binary files or as a binary package but without embedding it
    into another larger software unit.} }

\newcommand{\gtbUseCaseA}[1]{\lsucmeans{that you received a #1 licensed code
    snippet, module, library, or plugin (snimoli), that you modified it, and
    that you are now going to distribute this modified version to third parties
    in the form of source code files or as a source code package together with
    another larger software unit which contains this code snippet, module,
    library, or plugin as an embedded component.}}

\newcommand{\gtbUseCaseB}[1]{\lsucmeans{that you received a #1 licensed code
    snippet, module, library, or plugin (snimoli), that you modified it, and
    that you are now going to distribute this modified version to third
    partiesin the form of binary files or as a binary package together with
    another larger software unit which contains this code snippet, module,
    library, or plugin as an embedded component.}}

% ------------------------------------------------------------------------------

\newcommand{\gtbCoversOne}[1]{\coversOsucs{OSUC-01, OSUC-03, OSUC-06, and OSUC-09}{01}{09}}
\newcommand{\gtbCoversTwo}[1]{\coversOsucs{OSUC-02S, OSUC-05S}{02S}{05S}}
\newcommand{\gtbCoversThree}[1]{\coversOsucs{OSUC-02B, OSUC-05B}{02B}{05B}}
\newcommand{\gtbCoversFour}[1]{\mapsToOsuc{07S}}
\newcommand{\gtbCoversFive}[1]{\mapsToOsuc{07B}}
\newcommand{\gtbCoversSix}[1]{\mapsToOsuc{04S}}
\newcommand{\gtbCoversSeven}[1]{\mapsToOsuc{04B}}
\newcommand{\gtbCoversEight}[1]{\mapsToOsuc{08S}}
\newcommand{\gtbCoversNine}[1]{\mapsToOsuc{08B}}
\newcommand{\gtbCoversA}[1]{\mapsToOsuc{10S}}
\newcommand{\gtbCoversB}[1]{\mapsToOsuc{10B}}

% ------------------------------------------------------------------------------
% Keep license elements 
% #1 -> the license name

\newcommand{\gtbKeepLicenseElements}[1]{Ensure that the licensing elements
  (especially all notices that refer to the #1 and to the absence of any
  warranty) are retained in your package in the form in which you have received
  them.} 

% ------------------------------------------------------------------------------
% Give a copy of the license to the recipient of the software
% #1 -> the license name

\newcommand{\gtbGiveLicense}[1]{Give the recipient a copy of the #1 license.
  If it is not already part of the software package, add it.}

% ------------------------------------------------------------------------------
% Add license elements and acknowledgement to the documemtation
% #1 -> the license name

\newcommand{\gtbAddToDocumentation}[1]{Let the documentation of your
  distribution and/or your additional material also reproduce the content of the
  existing copyright notices, a hint to the software name, a link to its
  homepage, the respective disclaimer of warranty, and a link to the #1.}

% ------------------------------------------------------------------------------
% Keep all copyright notices intact
% #1 -> the license name

\newcommand{\gtbKeepCopyrightNotices}[1]{Retain all existing copyright notices.}

% ------------------------------------------------------------------------------
% Publish the source code
% #1 -> the license name

\newcommand{\gtbSourceRepository}[1]{Push the source code package into a
  repository under your control and make it downloadable via the Internet.
  Ensure, that this repository is online for at least 3 years after you ceased
  distributing the software package.}

% program or independent library, unmodified
\newcommand{\gtbMakeUnmodifiedSourceAvailable}[1]{Make the source code of the
  distributed software publicly available (even though you did not modify it):
  \gtbSourceRepository{#1}}

% program or independent library, modified
\newcommand{\gtbMakeModifiedSourceAvailable}[1]{Make the source code of the
  distributed software publicly available: \gtbSourceRepository{#1}} 

% embedded library, modified or unmodified, GPL and AGPL
\newcommand{\gtbMakeAllSourcesAvailable}[1]{Make the \emph{complete} source code
  of the program embedding the library publicly available (and, therefore, also
  the source code of the library itself): \gtbSourceRepository{#1}}

% embedded library, modified or unmodified, LGPL 
\newcommand{\gtbMakeEmbeddedSourcesAvailable}[1]{Make the source code of the 
  embedded library publicly available: \gtbSourceRepository{#1}}

% ------------------------------------------------------------------------------
% Explain where to find the sources
% #1 -> the license name

\newcommand{\gtbDescribeHowToGetSource}[1]{Insert an easy to find description
  into the distribution package that explains how and where the code can be
  retrieved.}

% ------------------------------------------------------------------------------
% Create and update the modification text file
% #1 -> the license name

\newcommand{\gtbCreateChangelog}[1]{Create a \emph{modification text file,} if
  such a file does not yet exist. \emph{Add} a description of your modifications
  on a functional level to the \emph{modification text file.}}

% ------------------------------------------------------------------------------
% Mark all modifications in the source files themselves
% #1 -> the license name

\newcommand{\gtbauxMarkChanges}[1]{Mark all modifications of the source code #1
  thoroughly within the source code and include the date of the modification.}

\newcommand{\gtbMarkEmbeddedModifications}[1]{%
  \gtbauxMarkChanges{of the embedded library (snimoli)}}

\newcommand{\gtbMarkLibraryModifications}[1]{%
  \gtbauxMarkChanges{of the library (snimoli)}}

\newcommand{\gtbMarkProgramModifications}[1]{%
  \gtbauxMarkChanges{the program (proapse)}}

% ------------------------------------------------------------------------------
% Ensure the copyright notice and the disclaimer (V2.x only) are present
% #1 -> the license name
% #2 -> type of distribution (binary or source code)

% GPL-3.0/LGPL-3.0/AGPL-3.0
\newcommand{\gtbVThreeCopyrightNotice}[2]{Ensure that the
  distributed #2 package contains a conspicuous, easy to find copyright notice.
  If this element is missing, add a new file containing the main copyright
  notice.} 

% GPL-2.0/LGPL-2.1
\newcommand{\gtbVTwoCopyrightNotice}[2]{Ensure that the distributed
  #2 package contains a conspicuous, easy to find copyright notice and
  disclaimer of warranty. If these elements are missing, add a new file
  containing the main copyright notice and the disclaimer of warranty in the
  form which is textually defined by the #1 license itself. (Yes, repeat
  the disclaimer although it is also part of the license itself and although you
  are required to hand the license itself over to the receiver.)}

% ------------------------------------------------------------------------------
% Make sure licensing statements apply to your modifications
% #1 -> the license name

\newcommand{\gtbauxArrangeChanges}[2]{Arrange your modifications of #2 in a way
  that they are covered by existing #1 licensing statements. If you add new
  source code files to the #2, insert a header containing your copyright line
  and a licensing statement in the form recommended by the #1.}

\newcommand{\gtbArrangeProgramChanges}[1]{%
  \gtbauxArrangeChanges{#1}{program}}

\newcommand{\gtbArrangeLibraryChanges}[1]{%
  \gtbauxArrangeChanges{#1}{library}}

\newcommand{\gtbArrangeEmbeddedChanges}[1]{%
  \gtbauxArrangeChanges{#1}{embedded library}}

\newcommand{\gtbHowToApplyTheseTerms}[1]{%
  \footnote{For details see section `How to Apply These Terms to Your New
    Programs' in the #1 license.}} 

% ------------------------------------------------------------------------------
% Forbid patent litigation
% #1 -> license name

\newcommand{\gtbNoPatentLitigation}[1]{%
  to institute a patent litigation against anyone alleging that the software
  constitutes patent infringement.} 

% ------------------------------------------------------------------------------
% Copyright dialog
% #1 -> license name

\newcommand{\gtbauxCopyrightDialogContent}[1]{Let it reproduce the content of
  the existing copyright notices, the software name, a link to its homepage, the 
  respective disclaimer of warranty, and a link to the #1.}

\newcommand{\gtbAddToCopyrightDialogWeakCopyleft}[1]{Let the copyright dialog of 
  the on-top development clearly say that it uses the #1 licensed library. 
  \gtbauxCopyrightDialogContent{#1}}

\newcommand{\gtbAddToCopyrightDialogStrongCopyleft}[1]{Let the copyright dialog
  of the on-top development clearly say that it uses the #1 licensed library and 
  that it is itself licensed under the #1, too. 
  \gtbauxCopyrightDialogContent{#1}}

\newcommand{\gtbAddToCopyrightDialogApp}[1]{Let the copyright dialog of the
  program clearly say that it is a #1 licensed program. 
  \gtbauxCopyrightDialogContent{#1}\ 
  If these conditions are not already met, add the missing elements.}


%% =============================================================================
%% Use Case Finder

% ------------------------------------------------------------------------------
% Common license specific use case finder for GPL, LGPL
% (does not apply to AGPL because there is no version 2.x of the AGPL)

\newcommand{\gplUseCaseFinder}[3]{
\tikzstyle{nodv} = [font=\scriptsize, ellipse, draw, fill=gray!10, 
    text width=2cm, text centered, minimum height=2em]

\tikzstyle{nods} = [font=\tiny, rectangle, draw, fill=gray!20, 
    text width=1cm, text centered, rounded corners, minimum height=3em]

\tikzstyle{nodb} = [font=\tiny, rectangle, draw, fill=gray!20, 
    text width=1.5cm, text centered, rounded corners, minimum height=3em]
    
\tikzstyle{leaf} = [font=\tiny, rectangle, draw, fill=gray!30, 
    text width=1.2cm, text centered, minimum height=6em]

\tikzstyle{slimleaf} = [font=\tiny, rectangle, draw, fill=gray!30, 
    text width=1cm, text centered, minimum height=6em]


\tikzstyle{edge} = [draw, -latex']

\begin{tikzpicture}[]

  \node[nodv] (l801) at (4,11.8)   {#1};

  \node[nodv] (l701) at (0,10.2)   {#2};
  \node[nodv] (l702) at (7.5,10.2) {#3};
  

  \node[nodb] (l601) at (0,8.6) {\textit{recipient:} \\ \textbf{4yourself}};
  \node[nodb] (l602) at (7.5,8.6) {\textit{recipient:} \\ \textbf{2others}};
  
  \node[nodb] (l501) at (4,7) {\textit{state:} \\ \textbf{unmodified}};
  \node[nodb] (l502) at (11,7) {\textit{state:} \\ \textbf{modified}};
  
  \node[nodb] (l401) at (2.25,5.4) {\textit{type:} \\ \textbf{proapse or snimoli}};
  \node[nodb] (l402) at (5.4,5.4) {\textit{type:} \\ \textbf{snimoli}};
  \node[nodb] (l403) at (8.4,5.4) {\textit{type:} \\ \textbf{proapse}};
  \node[nodb] (l404) at (12.8,5.4) {\textit{type:} \\ \textbf{snimoli}};
  
  
  \node[nodb] (l301) at (2.25,3.8) {\textit{context:} \\ \textbf{independent}};
  \node[nodb] (l302) at (5.4,3.8) {\textit{context:} \\ \textbf{embedded}};
  \node[nodb] (l303) at (8.4,3.8) {\textit{context:} \\ \textbf{independent}};
  \node[nodb] (l304) at (11.3,3.8) {\textit{context:} \\ \textbf{independent}};
  \node[nodb] (l305) at (14.3,3.8) {\textit{context:} \\ \textbf{embedded}};
  
  \node[nods] (l201) at (1.45,2.2) {\textit{form:} \textbf{source}};
  \node[nods] (l202) at (3.0,2.2) {\textit{form:} \textbf{binary}};
  \node[nods] (l203) at (4.6,2.2) {\textit{form:} \textbf{source}};
  \node[nods] (l204) at (6.2,2.2) {\textit{form:} \textbf{binary}};
  \node[nods] (l205) at (7.7,2.2) {\textit{form:} \textbf{source}};
  \node[nods] (l206) at (9.1,2.2) {\textit{form:} \textbf{binary}};
  \node[nods] (l207) at (10.5,2.2) {\textit{form:} \textbf{source}};
  \node[nods] (l208) at (11.9,2.2) {\textit{form:} \textbf{binary}};
  \node[nods] (l209) at (13.4,2.2) {\textit{form:} \textbf{source}};
  \node[nods] (l210) at (15.0,2.2) {\textit{form:} \textbf{binary}};
  
  \node[slimleaf] (l101) at (0,0) {
    \textbf{#1-*-C1} 
    \textit{using software only for yourself}};
  
  \node[leaf] (l102) at (1.45,0) { 
    \textbf{#1-*-C2} 
    \textit{distributing unmodified software as independent sources}};
  
  \node[leaf] (l103) at (3.0,0) { 
    \textbf{#1-*-C3}  
    \textit{distributing unmodified software as independent binaries}};
  
  \node[leaf] (l104) at (4.6,0) { 
    \textbf{#1-*-C4} 
    \textit{distributing unmodified library as embedded sources}};
  
  \node[leaf] (l105) at (6.2,0) { 
    \textbf{#1-*-C5}  
    \textit{distributing unmodified library as embedded binaries}};
  
  \node[slimleaf] (l106) at (7.7,0) { 
    \textbf{#1-*-C6}  
    \textit{distributing modified program as sources}};
  
  \node[slimleaf] (l107) at (9.1,0) { 
    \textbf{#1-*-C7}  
    \textit{distributing modified program as binaries}};
  
  \node[slimleaf] (l108) at (10.5,0) { 
    \textbf{#1-*-C8}  
    \textit{distributing modified library as independent sources}};
  
  \node[slimleaf] (l109) at (11.9,0) { 
    \textbf{#1-*-C9}
    \textit{distributing modified library as independent binaries}};
  
  \node[leaf] (l110) at (13.4,0) { 
    \textbf{#1-*-CA}  
    \textit{distributing modified library as embedded sources}};
  
  \node[leaf] (l111) at (15,0) { 
    \textbf{#1-*-CB}  
    \textit{ distributing modified library as embedded binaries}};
  
  \path [edge] (l801) -- (l701);
  \path [edge] (l801) -- (l702);
  \path [edge] (l701) -- (l601);
  \path [edge] (l701) -- (l602);
  \path [edge] (l702) -- (l601);
  \path [edge] (l702) -- (l602);
  
  \path [edge] (l602) -- (l501);
  \path [edge] (l602) -- (l502);
  
  \path [edge] (l501) -- (l401);
  \path [edge] (l501) -- (l402);
  \path [edge] (l502) -- (l403);
  \path [edge] (l502) -- (l404);
  
  \path [edge] (l401) -- (l301);
  \path [edge] (l402) -- (l302);
  \path [edge] (l403) -- (l303);
  \path [edge] (l404) -- (l304);
  \path [edge] (l404) -- (l305);
  
  \path [edge] (l301) -- (l201);
  \path [edge] (l301) -- (l202);
  \path [edge] (l302) -- (l203);
  \path [edge] (l302) -- (l204);
  \path [edge] (l303) -- (l205);
  \path [edge] (l303) -- (l206);
  \path [edge] (l304) -- (l207);
  \path [edge] (l304) -- (l208);
  \path [edge] (l305) -- (l209);
  \path [edge] (l305) -- (l210);
  
  \path [edge] (l601) -- (l101);
  \path [edge] (l201) -- (l102);
  \path [edge] (l202) -- (l103);
  \path [edge] (l203) -- (l104);
  \path [edge] (l204) -- (l105);
  \path [edge] (l205) -- (l106);
  \path [edge] (l206) -- (l107);
  \path [edge] (l207) -- (l108);
  \path [edge] (l208) -- (l109);
  \path [edge] (l209) -- (l110);
  \path [edge] (l210) -- (l111);
  
\end{tikzpicture}
} % end of gplUseCaseFinder

\newcommand{\agplUseCaseFinder}[2]{
\tikzstyle{nodv} = [font=\scriptsize, ellipse, draw, fill=gray!10, 
    text width=2cm, text centered, minimum height=2em]

\tikzstyle{nods} = [font=\tiny, rectangle, draw, fill=gray!20, 
    text width=0.8cm, text centered, rounded corners, minimum height=3em]

\tikzstyle{nodb} = [font=\tiny, rectangle, draw, fill=gray!20, 
    text width=1cm, text centered, rounded corners, minimum height=3em]
    
\tikzstyle{leaf} = [font=\tiny, rectangle, draw, fill=gray!30, 
    text width=1cm, text centered, minimum height=7em]

\tikzstyle{slimleaf} = [font=\tiny, rectangle, draw, fill=gray!30, 
    text width=0.84cm, text centered, minimum height=7em]


\tikzstyle{edge} = [draw, -latex']

\begin{tikzpicture}[]

  \node[nodv] (l801) at (4,10)   {#1 #2};

  \node[nodb] (l601) at (0,9.6) {\textit{recipient:} \\ \textbf{4your-} \\\textbf{self}};
  \node[nodb] (l602) at (8.8,9.6) {\textit{recipient:} \\ \textbf{2others}};
  
  \node[nods] (l503) at (0.8,7) {\textit{state:} \\ \textbf{unmo-} \\ \textbf{dified}}; 
  \node[nods] (l504) at (2.1,7) {\textit{state:} \\ \textbf{mo-} \\ \textbf{dified}}; 
  \node[nodb] (l501) at (5.8,7) {\textit{state:} \\ \textbf{unmo-} \\ \textbf{dified}}; 
  \node[nodb] (l502) at (11.5,7) {\textit{state:} \\ \textbf{mo-} \\ \textbf{dified}}; 
  
  \node[nods] (l405) at (1.3,5.4) {\textit{type:} \\ \textbf{pro-apse}};
  \node[nods] (l406) at (2.7,5.4) {\textit{type:} \\ \textbf{sni-moli}};
  
  \node[nodb] (l401) at (4.6,5.4) {\textit{type:} \\ \textbf{proapse or snimoli}};
  \node[nodb] (l402) at (7.2,5.4) {\textit{type:} \\ \textbf{snimoli}};
  \node[nodb] (l403) at (9.6,5.4) {\textit{type:} \\ \textbf{proapse}};
  \node[nodb] (l404) at (13.3,5.4) {\textit{type:} \\ \textbf{snimoli}};
  
  \node[nodb] (l306) at (1.3,3.8) {\textit{context:} \\ \textbf{inde-} \\ \textbf{pendent}};
  \node[nodb] (l307) at (2.7,3.8) {\textit{context:} \\ \textbf{em-} \\ \textbf{bedded}};
    
  \node[nodb] (l301) at (4.6,3.8) {\textit{context:} \\ \textbf{inde-} \\ \textbf{pendent}};
  \node[nodb] (l302) at (7.2,3.8) {\textit{context:} \\ \textbf{em-} \\ \textbf{bedded}};
  \node[nodb] (l303) at (9.6,3.8) {\textit{context:} \\ \textbf{inde-} \\ \textbf{pendent}};
  \node[nodb] (l304) at (11.9,3.8) {\textit{context:} \\ \textbf{inde-} \\ \textbf{pendent}};
  \node[nodb] (l305) at (14.3,3.8) {\textit{context:} \\ \textbf{em-} \\ \textbf{bedded}};
  
  \node[nods] (l201) at (4.0,2.2) {\textit{form:} \textbf{source}};
  \node[nods] (l202) at (5.3,2.2) {\textit{form:} \textbf{binary}};
  \node[nods] (l203) at (6.6,2.2) {\textit{form:} \textbf{source}};
  \node[nods] (l204) at (7.8,2.2) {\textit{form:} \textbf{binary}};
  \node[nods] (l205) at (9.0,2.2) {\textit{form:} \textbf{source}};
  \node[nods] (l206) at (10.2,2.2) {\textit{form:} \textbf{binary}};
  \node[nods] (l207) at (11.4,2.2) {\textit{form:} \textbf{source}};
  \node[nods] (l208) at (12.6,2.2) {\textit{form:} \textbf{binary}};
  \node[nods] (l209) at (13.8,2.2) {\textit{form:} \textbf{source}};
  \node[nods] (l210) at (15.0,2.2) {\textit{form:} \textbf{binary}};
  
  \node[slimleaf] (l101) at (0,0) {
    \textbf{#1-C1} 
    \textit{using software only for yourself}};
    
   \node[leaf] (l112) at (1.3,0) { 
    \textbf{#1-CC}
    \textit{execu-ting a modi-fied #1 program with net-io-Access}};   
    
  \node[leaf] (l113) at (2.7,0) { 
    \textbf{#1-CD}
    \textit{execu-ting a program with net-io-Access using a modified #1 library}}; 
 
  \node[slimleaf] (l102) at (4.0,0) { 
    \textbf{#1-C2} 
    \textit{distri-buting unmo-dified software as independent sources}};
  
  \node[leaf] (l103) at (5.3,0) { 
    \textbf{#1-C3}  
    \textit{distribu-ting unmo-dified software as independent binaries}};
  
  \node[slimleaf] (l104) at (6.6,0) { 
    \textbf{#1-C4} 
    \textit{distribu-ting an un-modified library as embedded sources}};
  
  \node[slimleaf] (l105) at (7.8,0) { 
    \textbf{#1-C5}  
    \textit{distribu-ting an un-modified library as embedded binaries}};
  
  \node[slimleaf] (l106) at (9,0) { 
    \textbf{#1-C6}  
    \textit{distri-buting a modi-fied program as sources}};
  
  \node[slimleaf] (l107) at (10.2,0) { 
    \textbf{#1-C7}  
    \textit{distri-buting a modi-fied program as binaries}};
  
  \node[slimleaf] (l108) at (11.4,0) { 
    \textbf{#1-C8}  
    \textit{distri-buting a modi-fied library as independent sources}};
  
  \node[slimleaf] (l109) at (12.6,0) { 
    \textbf{#1-C9}
    \textit{distri-buting a modi-fied library as independent binaries}};
  
  \node[slimleaf] (l110) at (13.8,0) { 
    \textbf{#1-CA}  
    \textit{distri-buting a modi-fied library as embedded sources}};
  
  \node[slimleaf] (l111) at (15,0) { 
    \textbf{#1-CB}  
    \textit{distri-buting a modi-fied library as embedded binaries}};
  
  \path [edge] (l801) -- (l601);
  \path [edge] (l801) -- (l602);
  
  \path [edge] (l601) -- (l503);
  \path [edge] (l601) -- (l504);
  \path [edge] (l602) -- (l501);
  \path [edge] (l602) -- (l502);
  
  \path [edge] (l501) -- (l401);
  \path [edge] (l501) -- (l402);
  \path [edge] (l502) -- (l403);
  \path [edge] (l502) -- (l404);
  
  \path [edge] (l504) -- (l405);  
  \path [edge] (l504) -- (l406); 
    
  \path [edge] (l401) -- (l301);
  \path [edge] (l402) -- (l302);
  \path [edge] (l403) -- (l303);
  \path [edge] (l404) -- (l304);
  \path [edge] (l404) -- (l305);

  \path [edge] (l405) -- (l306);
  \path [edge] (l406) -- (l307);  
  
  
  \path [edge] (l301) -- (l201);
  \path [edge] (l301) -- (l202);
  \path [edge] (l302) -- (l203);
  \path [edge] (l302) -- (l204);
  \path [edge] (l303) -- (l205);
  \path [edge] (l303) -- (l206);
  \path [edge] (l304) -- (l207);
  \path [edge] (l304) -- (l208);
  \path [edge] (l305) -- (l209);
  \path [edge] (l305) -- (l210);
  
  \path [edge] (l601) -- (l101);
  \path [edge] (l503) -- (l101);
  \path [edge] (l306) -- (l112);
  \path [edge] (l307) -- (l113);
        
  \path [edge] (l201) -- (l102);
  \path [edge] (l202) -- (l103);
  \path [edge] (l203) -- (l104);
  \path [edge] (l204) -- (l105);
  \path [edge] (l205) -- (l106);
  \path [edge] (l206) -- (l107);
  \path [edge] (l207) -- (l108);
  \path [edge] (l208) -- (l109);
  \path [edge] (l209) -- (l110);
  \path [edge] (l210) -- (l111);
  
\end{tikzpicture}
} % end of agplUseCaseFinder

