% Telekom osCompendium 'for being included' snippet template
%
% (c) Karsten Reincke, Deutsche Telekom AG, Darmstadt 2011
%
% This LaTeX-File is licensed under the Creative Commons Attribution-ShareAlike
% 3.0 Germany License (http://creativecommons.org/licenses/by-sa/3.0/de/): Feel
% free 'to share (to copy, distribute and transmit)' or 'to remix (to adapt)'
% it, if you '... distribute the resulting work under the same or similar
% license to this one' and if you respect how 'you must attribute the work in
% the manner specified by the author ...':
%
% In an internet based reuse please link the reused parts to www.telekom.com and
% mention the original authors and Deutsche Telekom AG in a suitable manner. In
% a paper-like reuse please insert a short hint to www.telekom.com and to the
% original authors and Deutsche Telekom AG into your preface. For normal
% quotations please use the scientific standard to cite.
%
% [ Framework derived from 'mind your Scholar Research Framework' 
%   mycsrf (c) K. Reincke 2012 CC BY 3.0  http://mycsrf.fodina.de/ ]
%


%% use all entries of the bibliography
%\nocite{*}

\section{Apache licensed software}

Officially, only the Apache License, Version 2.0 is an approved open source
license\footnote{For details $\rightarrow$ OSLiC, pp.\
\pageref{sec:ProtectPowerOfApL}}.
Explicitly, it 'only' focuses on the
\enquote{redistribution}\footcite[cf.][\nopage wp. §4]{OSI2012b}. So, you can
also use the folloing simplified Apache specific open source use case
finder\footnote{For details of the general OSUC finder $\rightarrow$ OSLiC, pp.\
\pageref{OsucTokens} and \pageref{OsucDefinitionTree}}:
 
\tikzstyle{nodv} = [font=\small, ellipse, draw, fill=gray!10, 
    text width=2cm, text centered, minimum height=2em]

\tikzstyle{nods} = [font=\footnotesize, rectangle, draw, fill=gray!20, 
    text width=1.2cm, text centered, rounded corners, minimum height=3em]

\tikzstyle{nodb} = [font=\footnotesize, rectangle, draw, fill=gray!20, 
    text width=2.2cm, text centered, rounded corners, minimum height=3em]
    
\tikzstyle{leaf} = [font=\tiny, rectangle, draw, fill=gray!30, 
    text width=1.2cm, text centered, minimum height=6em]

\tikzstyle{edge} = [draw, -latex']

\begin{tikzpicture}[]

\node[nodv] (l71) at (4,10) {ApL};

\node[nodb] (l61) at (0,8.6) {\textit{recipient:} \\ \textbf{4yourself}};
\node[nodb] (l62) at (6.5,8.6) {\textit{recipient:} \\ \textbf{4others}};

\node[nodb] (l51) at (2.5,7) {\textit{state:} \\ \textbf{unmodified}};
\node[nodb] (l52) at (9.3,7) {\textit{state:} \\ \textbf{modified}};

\node[nods] (l41) at (1.8,5.4) {\textit{form:} \textbf{source}};
\node[nods] (l42) at (3.6,5.4) {\textit{form:} \textbf{binary}};
\node[nodb] (l43) at (6.5,5.4) {\textit{type:} \\ \textbf{proapse}};
\node[nodb] (l44) at (12,5.4) {\textit{type:} \\ \textbf{snimoli}};


\node[nods] (l31) at (5.4,3.8) {\textit{form:} \textbf{source}};
\node[nods] (l32) at (7.2,3.8) {\textit{form:} \textbf{binary}};
\node[nodb] (l33) at (10,3.8) {\textit{context:} \\ \textbf{independent}};
\node[nodb] (l34) at (13.5,3.8) {\textit{context:} \\ \textbf{embedded}};

\node[nods] (l21) at (9,2.2) {\textit{form:} \textbf{source}};
\node[nods] (l22) at (10.8,2.2) {\textit{form:} \textbf{binary}};
\node[nods] (l23) at (12.6,2.2) {\textit{form:} \textbf{source}};
\node[nods] (l24) at (14.4,2.2) {\textit{form:} \textbf{binary}};

\node[leaf] (l11) at (0,0) {\textbf{ApL-1} \textit{using software only
for yourself}};

\node[leaf] (l12) at (1.8,0) { \textbf{ApL-2} \textit{ distributing unmodified
software as sources}};

\node[leaf] (l13) at (3.6,0) { \textbf{ApL-3}  \textit{ distributing unmodified
software as binaries}};

\node[leaf] (l14) at (5.4,0) { \textbf{ApL-4}  \textit{ distributing modified
program as sources}};

\node[leaf] (l15) at (7.2,0) { \textbf{ApL-5}  \textit{ distributing modified
program as binaries}};

\node[leaf] (l16) at (9,0) { \textbf{ApL-6}  \textit{ distributing modified
library as independent sources}};

\node[leaf] (l17) at (10.8,0) { \textbf{ApL-7} \textit{distributing modified
library as independent binaries}};

\node[leaf] (l18) at (12.6,0) { \textbf{ApL-8}  \textit{distributing
modified library as embedded sources}};

\node[leaf] (l19) at (14.4,0) { \textbf{ApL-9}  \textit{ distributing modified
library as embedded binaries}};


\path [edge] (l71) -- (l61);
\path [edge] (l71) -- (l62);
\path [edge] (l61) -- (l11);
\path [edge] (l62) -- (l51);
\path [edge] (l62) -- (l52);
\path [edge] (l51) -- (l41);
\path [edge] (l51) -- (l42);
\path [edge] (l52) -- (l43);
\path [edge] (l52) -- (l44);
\path [edge] (l41) -- (l12);
\path [edge] (l42) -- (l13);
\path [edge] (l43) -- (l31);
\path [edge] (l43) -- (l32);
\path [edge] (l44) -- (l33);
\path [edge] (l44) -- (l34);
\path [edge] (l31) -- (l14);
\path [edge] (l32) -- (l15);
\path [edge] (l33) -- (l21);
\path [edge] (l33) -- (l22);
\path [edge] (l34) -- (l23);
\path [edge] (l34) -- (l24);
\path [edge] (l21) -- (l16);
\path [edge] (l22) -- (l17);
\path [edge] (l23) -- (l18);
\path [edge] (l24) -- (l19);

\end{tikzpicture}


\subsection{ApL-1: Using the software only for yourself}
\label{OSUC-01-Apache20} \label{OSUC-03-Apache20} 
\label{OSUC-06-Apache20} \label{OSUC-09-Apache20}

\begin{description}
\item[means] that you are going to use a received Apache licensed software only
for yourself and that you do not handover it to any 3rd party in any sense.
\item[covers] OSUC-01, OSUC-03, OSUC-06, and OSUC-09\footnote{For details see
pp. \pageref{OSUC-01-DEF} - \pageref{OSUC-09-DEF}}
\item[requires] no tasks in order to fulfill the conditions of the Apache 2.0
license with respect to this use case:
  \begin{itemize}
    \item You are allowed to use any kind of Apache software in any sense and in
    any context without any obligations premised you do not handover the
    software to 3rd parties.
  \end{itemize}
\item[prohibits] nothing explicitly.
\end{description}

\subsection{ApL-2: Passing the unmodified software as source code}
\label{OSUC-02-Apache20} \label{OSUC-05-Apache20} \label{OSUC-07-Apache20} 

\begin{description}
\item[means] that you are going to distribute an unmodified version of the
received Apache software to 3rd parties in form of a set of source code files or
an integrated source code package\footnote{In this case it doesn't matter
whether you  distribute a program, an application, a server, a snippet, a
module, a library, or a plugin as an independent or an embedded unit}

\item[covers] OSUC-02, OSUC-05, OSUC-07\footnote{For details see pp.\ 
\pageref{OSUC-02-DEF} - \pageref{OSUC-07-DEF}}

\item[requires] the following tasks in order to fulfill the license conditions
\begin{itemize}
  \item \textbf{[mandatory:]} Give the recipient a copy of the Apache 2.0
  license. If it is still not incorporated into the software package, add
  it\footnote{For implementing the handover of files correctly $\rightarrow$
  OSLiC \pageref{DistributingFilesHint}}.
  \item \textbf{[mandatory:]} Ensure that the licensing elements -- esp.\ the
  specific copyright notice of the original author(s) -- are retained in your
  package in the form you have received them.
  \item \textbf{[mandatory:]} Ensure that a NOTICE text file is retained in
  your package in the form you have received it.
  \item \textbf{[voluntary:]} Let the documentation of your distribution
  and/or your additional material also reproduce the content of the NOTICE text
  file, a hint to the software name, a link to its homepage, and a link to the
  Apache 2.0 license.
\end{itemize}

\item[prohibits] to institute any patent litigation against anyone alleging that
the software constitutes patent infringement.

\end{description}


\subsection{ApL-3: Passing the unmodified software as binaries} 

\begin{description}
\item[means] that you are going to distribute an unmodified version of the
received Apache software to 3rd parties in form of a set of binary files or an
integrated bi\-na\-ry package\footnote{In this case it doesn't matter
whether you  distribute a program, an application, a server, a snippet, a
module, a library, or a plugin as an independent or an embedded unit}

\item[covers] OSUC-02, OSUC-05, OSUC-07\footnote{For details see pp.
\pageref{OSUC-02-DEF} - \pageref{OSUC-07-DEF}}

\item[requires] the following tasks in order to fulfill the license conditions
\begin{itemize}
  \item \textbf{[mandatory:]} Give the recipient a copy of the Apache 2.0
  license. If it is still not incorporated into your binary package, add
  it\footnote{For implementing the handover of files correctly $\rightarrow$
  OSLiC \pageref{DistributingFilesHint}}.
  
  \item \textbf{[mandatory:]} Ensure that the licensing elements -- esp.\ the
  specific copyright notice of the original author(s) -- are retained in your
  package in the form you have received them. If you compile the binary from the
  sources, ensure that all the licensing elements are also incorprated into the
  package.
  \item \textbf{[mandatory:]} Ensure that the NOTICE text file is retained or
  integrated into your binary package in the form you have initially received
  it.
  \item \textbf{[mandatory:]} Ensure that the NOTICE text file is also
  reproduced if and whereever such third-party notices normally appear --
  especially, if you are distributing an unmodified, Apache licensed library as
  embedded component of your own work which displays its own copyright notice.
  
  \item \textbf{[voluntary:]} Let the documentation of your distribution
  and/or your additional material also reproduce the content of the NOTICE text
  file, a hint to the software name, a link to its homepage, and a link to the
  Apache 2.0 license -- especially as subsection of your own copyright notice.
\end{itemize}

\item[prohibits] to institute any patent litigation against anyone alleging that
the software constitutes patent infringement.

\end{description}

\subsection{ApL-4: Passing a modified program as source code}
\label{OSUC-04-Apache20} 

\begin{description}
\item[means] that you are going to distribute a modified version of the received
Apache licensed program, application, or server (proapse) to 3rd parties in form
of a set of source code files or an integrated source code package.
\item[covers] OSUC-04\footnote{For details see pp.\ \pageref{OSUC-04-DEF}}
\item[requires] the tasks in order to fulfill the license conditions
\begin{itemize}
  
  \item \textbf{[mandatory:]} Give the recipient a copy of the Apache 2.0
  license. If it is still not incorporated into the source code package, add
  it\footnote{For implementing the handover of files correctly $\rightarrow$
  OSLiC \pageref{DistributingFilesHint}}.

  \item \textbf{[mandatory:]} Ensure that the licensing elements -- esp.\ the
  specific copyright notice of the original author(s) -- are retained in your
  package in the form you have received them.
  
  \item \textbf{[mandatory:]} Ensure that the NOTICE text file contains at least
  all the information of that NOTICE text file you have received.

  \item \textbf{[mandatory:]} Ensure that the NOTICE text file is also
  reproduced if and whereever such third-party notices normally appear. If the
  program already displays a copyright dialog, update it in an appropriate
  manner.
  
  \item \textbf{[voluntary:]} Inside of the source code, mark all your
  modifications thoroughly. Generate a NOTICE text file, if it still does not
  exist. Expand (sic!) the NOTICE text file by a description of your
  modifications.
   
  \item \textbf{[voluntary:]} Let the documentation of your distribution
  and/or your additional material also reproduce the content of the NOTICE text
  file, a hint to the software name, a link to its homepage, and a link to the
  Apache 2.0 license.
  
 \end{itemize}
 
\item[prohibits] to institute any patent litigation against anyone alleging that
the software constitutes patent infringement.

\end{description}

\subsection{ApL-5: Passing a modified program as binary}

\begin{description}
\item[means] that you are going to distribute a modified version of the received
Apache licensed pro\-gram, application, or server (proapse) to 3rd parties in
form of a set of binary files or an integrated binary package.
\item[covers] OSUC-04\footnote{For details see pp.\ \pageref{OSUC-04-DEF}}
\item[requires] the tasks in order to fulfill the license conditions
\begin{itemize}

 \item \textbf{[mandatory:]} Give the recipient a copy of the Apache 2.0
  license. If it is still not incorporated into your binary package, add
  it\footnote{For implementing the handover of files correctly $\rightarrow$
  OSLiC \pageref{DistributingFilesHint}}.
  
  \item \textbf{[mandatory:]} Ensure that the licensing elements -- esp.\ the
  specific copyright notice of the original author(s) -- are retained in your
  package in the form you have received them. If you compile the binary from the
  sources, ensure that all the licensing elements are also incorprated into the
  package.
  
  \item \textbf{[mandatory:]} Ensure that the NOTICE text file contains at least
  all the information of that NOTICE text file you have received.
  
  \item \textbf{[mandatory:]} Ensure that the NOTICE text file is also
  reproduced if and whereever such third-party notices normally appear. If the
  program already displays a copyright dialog, update it in an appropriate
  manner.
 
  \item \textbf{[voluntary:]} Inside of the source code, mark all your
  modifications thoroughly. Generate a NOTICE text file, if it still does not
  exist. Expand (sic!) the NOTICE text file by a description of your
  modifications.
 
  \item \textbf{[voluntary:]} Let the documentation of your distribution
  and/or your additional material also reproduce the content of the NOTICE text
  file, a hint to the software name, a link to its homepage, and a link to the
  Apache 2.0 license -- especially as a subsection of your own copyright notice.

\end{itemize}

\item[prohibits] to institute any patent litigation against anyone alleging that
the software constitutes patent infringement.

\end{description}

\subsection{ApL-6: Passing a modified library as independent source code}
\label{OSUC-08-Apache20}

\begin{description}
\item[means] that you are going to distribute a modified version of the received
Apache licensed code snippet, module, library, or plugin (snimoli) to 3rd
parties in form of a set of source code files or an integrated source code
package, but without embedding it into another larger software unit.
\item[covers] OSUC-08\footnote{For details see pp.\ \pageref{OSUC-08-DEF}}
\item[requires] the tasks in order to fulfill the license conditions
\begin{itemize}
  
  \item \textbf{[mandatory:]} Give the recipient a copy of the Apache 2.0
  license. If it is still not incorporated into the software package, add
  it\footnote{For implementing the handover of files correctly $\rightarrow$
  OSLiC \pageref{DistributingFilesHint}}.

  \item \textbf{[mandatory:]} Ensure that the licensing elements -- esp.\ the
  specific copyright notice of the original author(s) -- are retained in your
  package in the form you have received them.
  
  \item \textbf{[mandatory:]} Ensure that the NOTICE text file contains at least
  all the information of that NOTICE text file you have received.
 
  \item \textbf{[voluntary:]} Inside of the source code, mark all your
  modifications thoroughly. Generate a NOTICE text file, if it still does not
  exist. Expand (sic!) the NOTICE text file by a description of your
  modifications.
   
  \item \textbf{[voluntary:]} Let the documentation of your distribution
  and/or your additional material also reproduce the content of the NOTICE text
  file, a hint to the software name, a link to its homepage, and a link to the
  Apache 2.0 license.

\end{itemize}

\item[prohibits] to institute any patent litigation against anyone alleging that
the software constitutes patent infringement.

\end{description}


\subsection{ApL-7: Passing a modified library as independent binary}

\begin{description}
\item[means] that you are going to distribute a modified version of the received
Appache licensed code snippet, module, library, or plugin (snimoli) to 3rd
parties in form of a set of binary files or an integrated binary package but
without embedding it into another larger software unit.
\item[covers] OSUC-08\footnote{For details see pp.\ \pageref{OSUC-08-DEF}}
\item[requires] the tasks in order to fulfill the license conditions
\begin{itemize}
  
 \item \textbf{[mandatory:]} Give the recipient a copy of the Apache 2.0
  license. If it is still not incorporated into your binary package, add
  it\footnote{For implementing the handover of files correctly $\rightarrow$
  OSLiC \pageref{DistributingFilesHint}}.
  
  \item \textbf{[mandatory:]} Ensure that the licensing elements -- esp.\ the
  specific copyright notice of the original author(s) -- are retained in your
  package in the form you have received them. If you compile the binary from the
  sources, ensure that all the licensing elements are also incorprated into the
  package.
  
  \item \textbf{[mandatory:]} Ensure that the NOTICE text file contains at least
  all the information of that NOTICE text file you have received.
   
  \item \textbf{[voluntary:]} Inside of the source code, mark all your
  modifications thoroughly. Generate a NOTICE text file, if it still does not
  exist. Expand (sic!) the NOTICE text file by a description of your
  modifications.
 
  \item \textbf{[voluntary:]} Let the documentation of your distribution
  and/or your additional material also reproduce the content of the NOTICE text
  file, a hint to the software name, a link to its homepage, and a link to the
  Apache 2.0 license -- especially as a subsection of your own copyright notice.
  
\end{itemize}

\item[prohibits] to institute any patent litigation against anyone alleging that
the software constitutes patent infringement.

\end{description}

\subsection{ApL-8: Passing a modified library as embedded source code}
\label{OSUC-10-Apache20}

\begin{description}
\item[means] that you are going to distribute a modified version of the received
Apache licensed code snippet, module, library, or plugin (snimoli) to 3rd
parties in form of a set of source code files or an integrated source code
package together with another larger software unit which contains this code
snippet, module, library, or plugin as an embedded component.
\item[covers] OSUC-10\footnote{For details see pp.\ \pageref{OSUC-10-DEF}}
\item[requires] the tasks in order to fulfill the license conditions
\begin{itemize}
  
  \item \textbf{[mandatory:]} Give the recipient a copy of the Apache 2.0
  license. If it is still not incorporated into the software package, add
  it\footnote{For implementing the handover of files correctly $\rightarrow$
  OSLiC \pageref{DistributingFilesHint}}.

  \item \textbf{[mandatory:]} Ensure that the licensing elements -- esp.\ the
  specific copyright notice of the original author(s) -- are retained in your
  package in the form you have received them.
  
  \item \textbf{[mandatory:]} Ensure that the NOTICE text file contains at least
  all the information of that NOTICE text file you have received.
 
  \item \textbf{[mandatory:]} Ensure that the NOTICE text file is also
  reproduced if and whereever such third-party notices normally appear. If your
  overarching program displays an own copyright dialog, insert these
  information.
 
  \item \textbf{[voluntary:]} Inside of the source code, mark all your
  modifications thoroughly. Generate a NOTICE text file, if it still does not
  exist. Expand (sic!) the NOTICE text file by a description of your
  modifications.
  
  \item \textbf{[voluntary:]} Let the documentation of your distribution
  and/or your additional material also reproduce the content of the NOTICE text
  file, a hint to the software name, a link to its homepage, and a link to the
  Apache 2.0 license.

  \item \textbf{[voluntary:]} Arrange your source code distribution so that the
  integrated Apache license and the NOTICE text file clearly refers only to the
  embedded library and does not disturb the licensing of your own overarching
  work. It's a good tradition to keep the embedded components like libraries,
  modules, snippets, or plugins in specific directory which contains also all
  additional licensing elements.
 
\end{itemize}

\item[prohibits] to institute any patent litigation against anyone alleging that
the software constitutes patent infringement.

\end{description}


\subsection{ApL-9: Passing a modified library as embedded binary}

\begin{description}
\item[means] that you are going to distribute a modified version of the received
Apache licensed code snippet, module, library, or plugin to 3rd parties in form
of a set of binary files or an integrated binary package together with another
larger software unit which contains this code snippet, module, library, or
plugin as an embedded component.
\item[covers] OSUC-10\footnote{For details see pp.\ \pageref{OSUC-10-DEF}}
\item[requires] the tasks in order to fulfill the license conditions
\begin{itemize}
  
  \item \textbf{[mandatory:]} Give the recipient a copy of the Apache 2.0
  license. If it is still not incorporated into your binary package, add
  it\footnote{For implementing the handover of files correctly $\rightarrow$
  OSLiC \pageref{DistributingFilesHint}}.
  
  \item \textbf{[mandatory:]} Ensure that the licensing elements -- esp.\ the
  specific copyright notice of the original author(s) -- are retained in your
  package in the form you have received them. If you compile the binary from the
  sources, ensure that all the licensing elements are also incorprated into the
  package.
  
  \item \textbf{[mandatory:]} Ensure that the NOTICE text file contains at least
  all the information of that NOTICE text file you have received.
 
  \item \textbf{[mandatory:]} Ensure that the NOTICE text file is also
  reproduced if and whereever such third-party notices normally appear. If your
  overarching program displays an own copyright dialog, insert these
  information.
     
  \item \textbf{[voluntary:]} Inside of the source code, mark all your
  modifications thoroughly. Generate a NOTICE text file, if it still does not
  exist. Expand (sic!) the NOTICE text file by a description of your
  modifications.
 
  \item \textbf{[voluntary:]} Let the documentation of your distribution
  and/or your additional material also reproduce the content of the NOTICE text
  file, a hint to the software name, a link to its homepage, and a link to the
  Apache 2.0 license -- especially as subsection of your own copyright notice.
  

 \item \textbf{[voluntary:]} Arrange your binary distribution so that the
  integrated Apache license and the NOTICE text file clearly refers only to the
  embedded library and does not disturb the licensing of your own overarching
  work. It's a good tradition to keep the librabries, modules, snippet, or
  plugins in specific directiers which contain also all licensing elements.
  
\end{itemize}

\item[prohibits] to institute any patent litigation against anyone alleging that
the software constitutes patent infringement.

\end{description}








%\bibliography{../../../bibfiles/oscResourcesEn}
