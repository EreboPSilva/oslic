% Telekom osCompendium 'for being included' snippet template
%
% (c) Karsten Reincke, Deutsche Telekom AG, Darmstadt 2011
%
% This LaTeX-File is licensed under the Creative Commons Attribution-ShareAlike
% 3.0 Germany License (http://creativecommons.org/licenses/by-sa/3.0/de/): Feel
% free 'to share (to copy, distribute and transmit)' or 'to remix (to adapt)'
% it, if you '... distribute the resulting work under the same or similar
% license to this one' and if you respect how 'you must attribute the work in
% the manner specified by the author ...':
%
% In an internet based reuse please link the reused parts to www.telekom.com and
% mention the original authors and Deutsche Telekom AG in a suitable manner. In
% a paper-like reuse please insert a short hint to www.telekom.com and to the
% original authors and Deutsche Telekom AG into your preface. For normal
% quotations please use the scientific standard to cite.
%
% [ Framework derived from 'mind your Scholar Research Framework' 
%   mycsrf (c) K. Reincke 2012 CC BY 3.0  http://mycsrf.fodina.de/ ]
%


%% use all entries of the bibliography
%\nocite{*}

\section{MIT Licensed Software in the usage context of \ldots}

The MIT license is known as one of the most permissive licenses. Therefore the
MIT specific finder can be simplified too:

\begin{center}
\begin{footnotesize}
\pstree[levelsep=*1,treesep=0.2]{\Toval{MIT}}{
  \pstree{
    \Tr{\Ovalbox{\shortstack{recipient: \textit{4yourself}\\
    \textbf{\textit{used by yourself}}}}} 
  }{
    \Tr{\doublebox{\shortstack{\tiny{\textbf{MIT-1}:}\\
    \tiny{\textit{using the}}\\\tiny{\textit{software}}\\
    \tiny{\textit{only for}}\\\tiny{\textit{yourself}} }}} 
  }
  \pstree[levelsep=*0.2,treesep=0.2]{
    \Tr{\Ovalbox{\shortstack{recipient: \textit{4others}\\
      \textbf{\textit{distributed to 3rd parties}}}}} 
  }{ 
    \pstree[levelsep=*0.2,treesep=0.2]{
      \Tr{\Ovalbox{\shortstack{state:\\\textbf{\textit{unmodified}}}}}
    }{
        \Tr{\doublebox{\shortstack{\tiny{\textbf{MIT-2}:}\\
        \tiny{\textit{distributing an}}\\
        \tiny{\textit{unmodified pkg}} }}}

    }

    \pstree[levelsep=*0.2,treesep=0.2]{
      \Tr{\Ovalbox{\shortstack{state:\\\textbf{\textit{modified}}}}}
    }{ 
      \pstree{
        \Tr{\Ovalbox{\shortstack{type:\\\textbf{\textit{proapse}}}}}
      }{
          \Tr{\doublebox{\shortstack{\tiny{\textbf{MIT-3}:}\\
          \tiny{\textit{distributing}}\\\tiny{\textit{a modified}}\\
          \tiny{\textit{program}} }}} 
           
      }
      \pstree{
        \Tr{\Ovalbox{\shortstack{type:\\\textbf{\textit{snimoli}}}}}
      }{
        \pstree{
          \Tr{\Ovalbox{\shortstack{context:\\\textbf{\textit{independent}}}}}
        }{        
            \Tr{\doublebox{\shortstack{\tiny{\textbf{MIT-4}:}\\
            \tiny{\textit{distributing}}\\\tiny{\textit{a modified}}\\
            \tiny{\textit{library as}}\\\tiny{\textit{independent}}\\
            \tiny{\textit{pkg}} }}} 
           
        }
        \pstree{
          \Tr{\Ovalbox{\shortstack{context:\\\textbf{\textit{embedded}}}}}
        }{        
            \Tr{\doublebox{\shortstack{\tiny{\textbf{MIT-5}:}\\
            \tiny{\textit{distributing}}\\\tiny{\textit{a modified}}\\
            \tiny{\textit{library as}}\\\tiny{\textit{embedded}}\\
            \tiny{\textit{pkg}} }}} 
           
        }


      }
 
    }
   }   
}
\end{footnotesize}
\end{center}

\subsection{MIT-1: Using the software only for yourself}
\label{OSUC-01-MIT} 
\label{OSUC-03-MIT} 
\label{OSUC-06-MIT}
\label{OSUC-09-MIT}
  
\begin{description}
\item[means] that you are going to use a received MIT software only for yourself
and that you do not handover it to any 3rd party in any sense.
\item[covers] OSUC-01, OSUC-03, OSUC-06, and OSUC-09\footnote{For details see pp.
  \pageref{OSUC-01-DEF} - \pageref{OSUC-09-DEF}}
\item[requires] the tasks in order to fulfill the conditions
    of the MIT license:
  \begin{itemize}
    \item You are allowed to use any kind of MIT licensed software in any sense
    and in any context without any other obligations if you do not handover the
    software to 3rd parties and if you do not modify the existing copyright
    notes and the existing permission notice.
  \end{itemize}
\item[prohibits] nothing explicitly.
\end{description}

\subsection{MIT-2: Passing the unmodified software}
\label{OSUC-02-MIT} \label{OSUC-05-MIT} \label{OSUC-07-MIT} 

\begin{description}
\item[means] that you are going to distribute an unmodified version of the
received MIT software to 3rd parties - regardless whether you distribute it in
form of binaries or of source code files\footnote{In this case it also doesn't
matter whether you distribute a program, an application, a server, a snippet, a
module, a library, or a plugin as an independent package}

\item[covers] OSUC-02, OSUC-05, OSUC-07\footnote{For details see pp.
\pageref{OSUC-02-DEF} - \pageref{OSUC-07-DEF}}

\item[requires] the tasks in order to fulfill the license conditions
\begin{itemize}
  \item \textbf{[mandatory:]} Ensure that the licensing elements - eg.
  the MIT license text containing the specific copyright notices of the original
  author(s), the permission notices and the MIT disclaimer - are retained in
  your package in the form you have got them.
  \item \textbf{[voluntary:]} It's a good tradition to let the documentation of
  your distribution and/or your additional material also contain a link to the
  original software (project) and its' homepage.
\end{itemize}
\item[prohibits] nothing explicitly.
\end{description}

\subsection{MIT-3: Passing a modified program}
\label{OSUC-04-MIT}

\begin{description}
\item[means] that you are going distribute a modified version of the received
MIT program, application, or server (proapse) to 3rd parties\footnote{In this
case it doesn't matter whether you are going to distribute it in form of a set
of source code files or as an integrated source code package.}.
\item[covers] OSUC-04\footnote{For details see pp. \pageref{OSUC-04-DEF}}
\item[requires] the tasks in order to fulfill the license conditions
\begin{itemize}
  \item \textbf{[mandatory:]} Ensure that the original licensing elements - eg.
  the MIT license text containing the specific copyright notices of the original
  author(s), the permission notices and the MIT disclaimer - are retained in
  your package in the form you have got them.
  \item \textbf{[voluntary:]} Mark your modifications in the sourcecode,
  regardless whether you want to distribute the code or not.
  \item \textbf{[voluntary:]} It's a good tradition to let the documentation of
  your distribution and/or your additional material also contain a link to the
  original software (project) and its' homepage.
  \item \textbf{[voluntary:]} You are allowed to expand an existing copyright
  notice presented by the program during its user interaction by a hint to your
  own work or part.
  \item \textbf{[voluntary:]} It is a good practice of the open source
  community, to let the copyright notice which is shown by the running program
  also state that the program is licensed under the MIT license. Because you are
  already modifying the program, you can also add such a hint, if the presented
  original copyright notice lacks such a statement.
\end{itemize}
\item[prohibits] nothing explicitly.
\end{description}

\subsection{MIT-4: Passing a modified library independently}
\label{OSUC-08-MIT}
\begin{description}
\item[means] that you are going distribute a modified version of the received
MIT code snippet, module, library, or plugin (snimoli) to 3rd parties without
embedding it into another larger software unit.
\item[covers] OSUC-08\footnote{For details see pp. \pageref{OSUC-08-DEF}}
\item[requires] the tasks in order to fulfill the license conditions
\begin{itemize}
  \item \textbf{[mandatory:]} Ensure that the original licensing elements - eg.
  the MIT license text containing the specific copyright notices of the original
  author(s), the permission notices and the MIT disclaimer - are retained in
  your package in the form you have got them.
  \item \textbf{[voluntary:]} Mark your modifications in the sourcecode,
  regardless whether you want to distribute this source code or not.
  \item \textbf{[voluntary:]} It's a good tradition to let the documentation of
  your distribution and/or your additional material also contain a link to the
  original software (project) and its' homepage.
\end{itemize}
\item[prohibits] nothing explicitly.
\end{description}


\subsection{MIT-5: Passing a modified library as embedded component}
\label{OSUC-10-MIT}
\begin{description}
\item[means] that you are going to distribute a modified version of the received
MIT code snippet, module, library, or plugin (snimoli) to 3rd parties together
with another larger software unit which contains this code snippet, module,
library, or plugin as an embedded component.
\item[covers] OSUC-10\footnote{For details see pp. \pageref{OSUC-10-DEF}}
\item[requires] the tasks in order to fulfill the license conditions
\begin{itemize}
  \item \textbf{[mandatory:]} Ensure that the original licensing elements - eg.
  the MIT license text containing the specific copyright notices of the original
  author(s), the permission notices and the MIT disclaimer - are retained in
  your package in the form you have got them.
  \item \textbf{[voluntary:]} Mark your modifications in the sourcecode,
  regardless whether you want to distribute this source code or not.
  
  \item \textbf{[voluntary:]} It is a good practice of the open source
  community, to let the copyright notice which is shown by the running program
  also state that the program uses a component being licensed under the MIT
  license. And it's a good tradition to insert links to the homepage / download
  page of this used component.

  \item \textbf{[voluntary:]} It's also a good tradition to let the
  documentation of your programm and/or your additional material also mention
  that you have used this component added by a link to the original software
  component and its' homepage.
\end{itemize}
\item[prohibits] nothing explicitly.
\end{description}

\subsection{Discussions and Explanations}

The MIT-License is known as one of the most permissive licenses. It is a very
short license containing (1) a paragraph saying that you are allowed to do
almost anything you want, followed (2) by the condition that you have to
\enquote{include} the existing copyright notes and the permission notes
\enquote{[\ldots] in all copies or substantial portions of the software}, and
(3) closed by the well known disclaimer\footcite[cf.][\nopage
wp]{MitLicense2012a}. But the license doesn't talk about the difference of
source code and object code. So, you have to find the right way by yourself.
Here are our interpretations:

\begin{itemize}
  \item If you do not modify the received MIT licensed application, neither for
  your own purposes, nor for handing over the program to 3rd parties, you can
  conclude that all copyright notices and permission notices are already correct.
  \item If you modify the received MIT licensed application, regardless for
  which purposes, you are simply not allowed to erase or modify existing
  copyright notes and permission notices. You may add your own modifications
  under new conditions, but the old base must survive.
\end{itemize}

%\bibliography{../../../bibfiles/oscResourcesEn}
