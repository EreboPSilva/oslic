% Telekom osCompendium 'for being included' snippet template
%
% (c) Karsten Reincke, Deutsche Telekom AG, Darmstadt 2011
%
% This LaTeX-File is licensed under the Creative Commons Attribution-ShareAlike
% 3.0 Germany License (http://creativecommons.org/licenses/by-sa/3.0/de/): Feel
% free 'to share (to copy, distribute and transmit)' or 'to remix (to adapt)'
% it, if you '... distribute the resulting work under the same or similar
% license to this one' and if you respect how 'you must attribute the work in
% the manner specified by the author ...':
%
% In an internet based reuse please link the reused parts to www.telekom.com and
% mention the original authors and Deutsche Telekom AG in a suitable manner. In
% a paper-like reuse please insert a short hint to www.telekom.com and to the
% original authors and Deutsche Telekom AG into your preface. For normal
% quotations please use the scientific standard to cite.
%
% [ Framework derived from 'mind your Scholar Research Framework' 
%   mycsrf (c) K. Reincke 2012 CC BY 3.0  http://mycsrf.fodina.de/ ]
%


%% use all entries of the bibliography
%\nocite{*}

\section{Apache licensed software}

\begin{license}{APL} % ends at end of file
\licensename{Apache-2.0}
\licensespec{Apache License 2.0}
\licenseversion{2.0}
\licenseabbrev{Apache}

Today, the current release of the Apache open source license is version 2.0,
older versions are deprecated.\footnote{For details $\rightarrow$ OSLiC, pp.\
\protectionpageref{APL}} Because it focusses primarily on the
\enquote{redistribution,}\footcite[cf.][\nopage wp.\ §4]{Apl20OsiLicense2004a}
the following simplified Apache specific open source use case
finder\footnote{For details of the general OSUC finder $\rightarrow$ OSLiC, pp.\
\pageref{OsucTokens} and \pageref{OsucDefinitionTree}} can be used:
 
\tikzstyle{nodv} = [font=\small, ellipse, draw, fill=gray!10, 
    text width=2cm, text centered, minimum height=2em]

\tikzstyle{nods} = [font=\footnotesize, rectangle, draw, fill=gray!20, 
    text width=1.2cm, text centered, rounded corners, minimum height=3em]

\tikzstyle{nodb} = [font=\footnotesize, rectangle, draw, fill=gray!20, 
    text width=2.2cm, text centered, rounded corners, minimum height=3em]
    
\tikzstyle{leaf} = [font=\tiny, rectangle, draw, fill=gray!30, 
    text width=1.2cm, text centered, minimum height=6em]

\tikzstyle{edge} = [draw, -latex']

\begin{tikzpicture}[]

\node[nodv] (l71) at (4,10) {ApL};

\node[nodb] (l61) at (0,8.6) {\textit{recipient:} \\ \textbf{4yourself}};
\node[nodb] (l62) at (6.5,8.6) {\textit{recipient:} \\ \textbf{2others}};

\node[nodb] (l51) at (2.5,7) {\textit{state:} \\ \textbf{unmodified}};
\node[nodb] (l52) at (9.3,7) {\textit{state:} \\ \textbf{modified}};

\node[nods] (l41) at (1.8,5.4) {\textit{form:} \textbf{source}};
\node[nods] (l42) at (3.6,5.4) {\textit{form:} \textbf{binary}};
\node[nodb] (l43) at (6.5,5.4) {\textit{type:} \\ \textbf{proapse}};
\node[nodb] (l44) at (12,5.4) {\textit{type:} \\ \textbf{snimoli}};


\node[nods] (l31) at (5.4,3.8) {\textit{form:} \textbf{source}};
\node[nods] (l32) at (7.2,3.8) {\textit{form:} \textbf{binary}};
\node[nodb] (l33) at (10,3.8) {\textit{context:} \\ \textbf{independent}};
\node[nodb] (l34) at (13.5,3.8) {\textit{context:} \\ \textbf{embedded}};

\node[nods] (l21) at (9,2.2) {\textit{form:} \textbf{source}};
\node[nods] (l22) at (10.8,2.2) {\textit{form:} \textbf{binary}};
\node[nods] (l23) at (12.6,2.2) {\textit{form:} \textbf{source}};
\node[nods] (l24) at (14.4,2.2) {\textit{form:} \textbf{binary}};

\node[leaf] (l11) at (0,0) {\textbf{ApL-C1} \textit{using software only
for yourself}};

\node[leaf] (l12) at (1.8,0) { \textbf{ApL-C2} \textit{ distributing unmodified
software as sources}};

\node[leaf] (l13) at (3.6,0) { \textbf{ApL-C3}  \textit{ distributing unmodified
software as binaries}};

\node[leaf] (l14) at (5.4,0) { \textbf{ApL-C4}  \textit{ distributing modified
program as sources}};

\node[leaf] (l15) at (7.2,0) { \textbf{ApL-C5}  \textit{ distributing modified
program as binaries}};

\node[leaf] (l16) at (9,0) { \textbf{ApL-C6}  \textit{ distributing modified
library as independent sources}};

\node[leaf] (l17) at (10.8,0) { \textbf{ApL-C7} \textit{distributing modified
library as independent binaries}};

\node[leaf] (l18) at (12.6,0) { \textbf{ApL-C8}  \textit{distributing
modified library as embedded sources}};

\node[leaf] (l19) at (14.4,0) { \textbf{ApL-C9}  \textit{ distributing modified
library as embedded binaries}};


\path [edge] (l71) -- (l61);
\path [edge] (l71) -- (l62);
\path [edge] (l61) -- (l11);
\path [edge] (l62) -- (l51);
\path [edge] (l62) -- (l52);
\path [edge] (l51) -- (l41);
\path [edge] (l51) -- (l42);
\path [edge] (l52) -- (l43);
\path [edge] (l52) -- (l44);
\path [edge] (l41) -- (l12);
\path [edge] (l42) -- (l13);
\path [edge] (l43) -- (l31);
\path [edge] (l43) -- (l32);
\path [edge] (l44) -- (l33);
\path [edge] (l44) -- (l34);
\path [edge] (l31) -- (l14);
\path [edge] (l32) -- (l15);
\path [edge] (l33) -- (l21);
\path [edge] (l33) -- (l22);
\path [edge] (l34) -- (l23);
\path [edge] (l34) -- (l24);
\path [edge] (l21) -- (l16);
\path [edge] (l22) -- (l17);
\path [edge] (l23) -- (l18);
\path [edge] (l24) -- (l19);

\end{tikzpicture}


%%
%% Common building blocks
%%

% Common footnote, used with every 'give license' task
\newcommand{\passingFiles}{%
  \footnote{For implementing the handover of files correctly $\rightarrow$
    OSLiC, p. \pageref{DistributingFilesHint}}}

% ------------------------------------------------------------------------------
% Give a copy of the license
\newcommand{\auxGiveLicenseFor}[1]{%
  Give the recipient a copy of the Apache 2.0 license. If it is not already part
  of the #1 package, add it.} 
\newcommand{\copyLicenseWithBinary}{\auxGiveLicenseFor{binary}}
\newcommand{\copyLicenseWithSource}{\auxGiveLicenseFor{software}}

% ------------------------------------------------------------------------------
% Keep licensing elements intact
\newcommand{\keepLicenseElements}{
  Ensure that the licensing elements (especially the specific copyright notice
  of the original author(s)) are retained in your package in the form you have
  received them.} 
\newcommand{\keepLicenseElementsBinary}{%
  \keepLicenseElements{}
  If you compile the binary from the sources, ensure that all the licensing
  elements are also incorporated into the package.} 

% ------------------------------------------------------------------------------
% Keep notice file intact
\newcommand{\auxCreateAndAddToNoticeFile}{%
  Create a \emph{notice text file}, if it still does not exist. \emph{Add} a
  description of your modifications into the \emph{notice text file.}}

\newcommand{\auxKeepNoticeFile}[1]{%
  Ensure that the \emph{notice text file} is #1 your package in the form you
  have initially received it.}  
\newcommand{\keepNoticeFile}{\auxKeepNoticeFile{retained in}}
\newcommand{\includeNoticeFile}{\auxKeepNoticeFile{retained in or integrated into}}

\newcommand{\keepAllNotices}{%
  Ensure that the \emph{notice text file} contains at least all the information
  in the \emph{notice text file} that you have received.} 
\newcommand{\expandNotices}{%
  \keepAllNotices{} 
  \auxCreateAndAddToNoticeFile{}}

% ------------------------------------------------------------------------------
% Display the notice file
\newcommand{\auxDisplayNotice}{%
  Ensure that the \emph{notice text file} is also reproduced if and whereever
  such third-party notices normally appear.}

\newcommand{\displayNotice}{%
  \auxDisplayNotice{}
  If the program already displays a copyright dialog, update it in an
  appropriate manner.} 

\newcommand{\displayNoticeOfEmbeddedLibrary}{%
  \auxDisplayNotice{}
  If the software that embeds this library displays its own copyright dialog,
  insert this information there.}

% ------------------------------------------------------------------------------
% Mark your modifications
\newcommand{\auxMarkModifiedSource}[1]{%
  Inside of the #1 source code, mark all your modifications thoroughly. 
  \auxCreateAndAddToNoticeFile{}}
\newcommand{\markModifiedSource}{\auxMarkModifiedSource{}}
\newcommand{\markModifiedLibrarySource}{\auxMarkModifiedSource{library}}

\newcommand{\auxMarkUndistributedChanges}[1]{%
  Even if you do not want to distribute your modified source code, mark all your
  modifications #1 thoroughly.} 
\newcommand{\markUndistributedChanges}{%
  \auxMarkUndistributedChanges{}}
\newcommand{\markUndistributedLibraryChanges}{%
  \auxMarkUndistributedChanges{of the embedded libary}}

% ------------------------------------------------------------------------------
% Add info to documentation
\newcommand{\auxDocumentation}{%
  Let the documentation of your distribution and/or your additional material
  also reproduce the content of the \emph{notice text file}, a hint to the
  software name, a link to its homepage, and a link to the Apache 2.0 license}

\newcommand{\documentation}{%
  \auxDocumentation.} 
\newcommand{\documentationBinary}{%
  \auxDocumentation, especially as subsection of your own copyright notice.} 

% ------------------------------------------------------------------------------
% Forbid promoting services using trademarks, etc.
\newcommand{\noTrademarks}{%
  to promote any of your services based on the this software by trademarks,
  service marks, or product names linked to the software except as required for
  unambiguously denoting the software.}
\newcommand{\noTrademarksExceptNotice}{%
  to use any trademark, service mark, or product name linked to the Apache
  software to promote your products and services based on the this software,
  except as required for unambiguously denoting the software and for reproducing
  the notice text file.}  

% ------------------------------------------------------------------------------
% Forbid patent litigation
\newcommand{\noPatentLitigation}{%
  to institute a patent litigation against anyone alleging that the software
  constitutes patent infringement.} 

% ------------------------------------------------------------------------------

\subsection{ApL-C1: Using the software only for yourself}
\begin{lsuc}{APL-C1}
  \linkosuc{01}
  \linkosuc{03} 
  \linkosuc{06}
  \linkosuc{09}

  \lsucmeans{that you will use a Apache licensed software that you received only
    for yourself and that you do not hand it over to any 3rd party in any sense.}

  \lsuccovers{OSUC-01, OSUC-03, OSUC-06, and OSUC-09\footnote{For details 
      $\rightarrow$ OSLiC, pp.\ \pageref{OSUC-01-DEF} - \pageref{OSUC-09-DEF}}}

  \begin{lsucrequiresnothing}
    \item You are allowed to use any kind of Apache software in any sense and in
      any context without being obliged to do anything as long as you do not
      give the software to third parties. 
  \end{lsucrequiresnothing}
  
  \begin{lsucprohibits}
    \lsucitem{\noTrademarks}
    \lsucitem{\noPatentLitigation}
  \end{lsucprohibits}
\end{lsuc}

\subsection{ApL-C2: Passing the unmodified software as source code}
\begin{lsuc}{APL-C2}
  \linkosuc{02S}
  \linkosuc{05S}
  \linkosuc{07S}

  \lsucmeans{that you are going to distribute an unmodified version of the
    Apache software that you received to third parties in the form of source
    code files or as a source code package. In this case it makes no difference 
    if you distribute a program, an application, a server, a snippet, a module, a
    library, or a plugin as an independent or as an embedded unit.}

  \lsuccovers{OSUC-02S, OSUC-05S, OSUC-07S\footnote{For details $\rightarrow$
      OSLiC, pp.\ \pageref{OSUC-02S-DEF} - \pageref{OSUC-07S-DEF}}}

  \begin{lsucrequires}
    \lsucmandatory{\copyLicenseWithSource}\passingFiles
    \lsucmandatory{\keepLicenseElements}
    \lsucmandatory{\keepNoticeFile}%
    \footnote{The Apache license seems purposely to be a bit ambiguous: it
      uses the term \enquote{\enquote{Notice} text file}. In its strict sense,
      the term refers to a file named
      `NOTICE.[txt\textbar{}pdf\textbar{}\ldots]'. In a weaker sense, it may 
      denote any (text) file containing (licensing) notices. To be sure to act
      according to this requirement you should also read this term in the
      broader sense if there is no text file named `NOTICE'}
      
    \lsucoptional{\documentation}
  \end{lsucrequires}

  \begin{lsucprohibits}
    \lsucitem{\noTrademarks}
    \lsucitem{\noPatentLitigation}
  \end{lsucprohibits}
\end{lsuc}


\subsection{ApL-C3: Passing the unmodified software as binaries}
\begin{lsuc}{APL-C3}
  \linkosuc{02B} 
  \linkosuc{05B} 
  \linkosuc{07B} 

  \lsucmeans{that you are going to distribute an unmodified version of the
    Apache software that you received to third parties in the form of binary
    files or as a binary package. In this case it does not matter if you
    distribute a program, an application, a server, a snippet, a module, a
    library, or a plugin as an independent or an embedded unit.}

  \lsuccovers{OSUC-02B, OSUC-05B, OSUC-07B\footnote{For details $\rightarrow$
      OSLiC, pp.\ \pageref{OSUC-02B-DEF} - \pageref{OSUC-07B-DEF}}}

  \begin{lsucrequires}
    \lsucmandatory{\copyLicenseWithBinary}\passingFiles
    \lsucmandatory{\keepLicenseElementsBinary}
    \lsucmandatory{\includeNoticeFile}

    \lsucmandatory{Ensure that the \emph{notice text file} is also reproduced if
      and whereever such third-party notices normally appear (especially, if you
      are distributing an unmodified Apache licensed library as embedded
      component of your own work which displays its own copyright notice.)} 
    
    \lsucoptional{\documentationBinary}
  \end{lsucrequires}

  \begin{lsucprohibits}
    \lsucitem{\noTrademarks}
    \lsucitem{\noPatentLitigation}
  \end{lsucprohibits}

\end{lsuc}

\subsection{ApL-C4: Passing a modified program as source code}
\begin{lsuc}{APL-C4}
  \linkosuc{04S} 
  
  \lsucmeans{that you are going to distribute a modified version of the Apache
    licensed program, application, or server (proapse) that you received to
    third parties in the form of source code files or as a source code package.}
 
  \lsuccovers{OSUC-04S\footnote{For details $\rightarrow$ OSLiC, pp.\
      \pageref{OSUC-04S-DEF}}}

  \begin{lsucrequires}
    \lsucmandatory{\copyLicenseWithSource}\passingFiles
    \lsucmandatory{\keepLicenseElements}
    \lsucmandatory{\keepAllNotices}
    \lsucmandatory{\displayNotice}
  
    \lsucmandatory{Inside of the source code, mark all your modifications
      thoroughly. Generate a \emph{notice text file}, if it still does not
      exist. \emph{Add} a description of your modifications into the
      \emph{notice text file.}}

    \lsucoptional{\documentationBinary}
  \end{lsucrequires}
 
  \begin{lsucprohibits}
    \lsucitem{\noTrademarks}
    \lsucitem{\noPatentLitigation}
  \end{lsucprohibits}
\end{lsuc}

\subsection{ApL-C5: Passing a modified program as binary}
\begin{lsuc}{APL-C5}
  \linkosuc{04B}

  \lsucmeans{that you are going to distribute a modified version of the Apache
    licensed program, application, or server (proapse) that you received to
    third parties in the form of binary files or as a binary package.}

  \lsuccovers{OSUC-04B\footnote{For details $\rightarrow$ OSLiC,
      pp.\ \pageref{OSUC-04B-DEF}}}

  \begin{lsucrequires}
    \lsucmandatory{\copyLicenseWithBinary}\passingFiles
    \lsucmandatory{\keepLicenseElementsBinary}
    \lsucmandatory{\expandNotices}
    \lsucmandatory{\displayNotice}
 
    \lsucoptional{Even if you do not want to distribute your modified source
      code, mark all your modifications thoroughly.} 

    \lsucoptional{\documentationBinary}
  \end{lsucrequires}

  \begin{lsucprohibits}
    \lsucitem{\noTrademarks}
    \lsucitem{\noPatentLitigation}
  \end{lsucprohibits}
\end{lsuc}

\subsection{ApL-C6: Passing a modified library as independent source code}
\begin{lsuc}{APL-C6}
  \linkosuc{08S}

  \lsucmeans{that you are going to distribute a modified version of the Apache
    licensed code snippet, module, library, or plugin (snimoli) that you
    received to third parties in the form of source code files or as a source
    code package, but without embedding it into another larger software unit.}

  \lsuccovers{OSUC-08S\footnote{For details $\rightarrow$ OSLiC,
      pp.\ \pageref{OSUC-08S-DEF}}}

  \begin{lsucrequires}
    \lsucmandatory{\copyLicenseWithSource}\passingFiles
    \lsucmandatory{\keepLicenseElements}
    \lsucmandatory{\keepAllNotices}
 
    \lsucmandatory{Inside of the source code, mark all your modifications
      thoroughly. Generate a \emph{notice text file}, if it still does not
      exist. \emph{Expand} the \emph{notice text file} by a description of your
      modifications.}

    \lsucoptional{\documentation}
  \end{lsucrequires}

  \begin{lsucprohibits}
    \lsucitem{\noTrademarks}
    \lsucitem{\noPatentLitigation}
  \end{lsucprohibits}

\end{lsuc}


\subsection{ApL-C7: Passing a modified library as independent binary}
\begin{lsuc}{APL-C7}
  \linkosuc{08B}

  \lsucmeans{that you are going to distribute a modified version of the Apache
    licensed code snippet, module, library, or plugin (snimoli) that you
    received to third parties in the form of binary files or as a binary package
    but without embedding it into another larger software unit.}

  \lsuccovers{OSUC-08B\footnote{For details $\rightarrow$ OSLiC,
      pp.\ \pageref{OSUC-08B-DEF}} }

  \begin{lsucrequires}
    \lsucmandatory{\copyLicenseWithBinary}\passingFiles
    \lsucmandatory{\keepLicenseElementsBinary}
    \lsucmandatory{\expandNotices}
   
    \lsucoptional{Even if you do not want to distribute your modified source
      code, mark all your modifications thoroughly.} 

    \lsucoptional{\documentationBinary}
  \end{lsucrequires}

  \begin{lsucprohibits}
    \lsucitem{\noTrademarks}
    \lsucitem{\noPatentLitigation}
  \end{lsucprohibits}
\end{lsuc}

\subsection{ApL-C8: Passing a modified library as embedded source code}
\begin{lsuc}{APL-C8}
  \linkosuc{10S}

  \lsucmeans{that you are going to distribute a modified version of the Apache
    licensed code snippet, module, library, or plugin (snimoli) thath you
    received to third parties in the form of source code files or as a source
    code package together with another larger software unit which contains this
    code snippet, module, library, or plugin as an embedded component.}

  \lsuccovers{OSUC-10S\footnote{For details $\rightarrow$ OSLiC,
      pp.\ \pageref{OSUC-10S-DEF}}} 

  \begin{lsucrequires}
    \lsucmandatory{\copyLicenseWithSource}\passingFiles
    \lsucmandatory{\keepLicenseElements}
    \lsucmandatory{\keepAllNotices}
 
    \lsucmandatory{\displayNoticeOfEmbeddedLibrary}
 
    \lsucmandatory{Inside of the library source code, mark all your
      modifications thoroughly. Generate a \emph{notice text file}, if it still
      does not exist. \emph{Expand} the \emph{notice text file} by a description
      of your modifications.}%
    \footnote{The term library also includes snippet, module, and plugin.} 

    \lsucoptional{\documentation}

    \lsucoptional{Arrange your source code distribution so that the integrated
      Apache license and the \emph{notice text file} clearly refer only to the
      embedded library and do not disturb the licensing of your own overarching
      work. It's a good tradition to keep the embedded components like
      libraries, modules, snippets, or plugins in specific directory which
      contains also all additional licensing elements.}
  \end{lsucrequires}

  \begin{lsucprohibits}
    \lsucitem{\noTrademarks}
    \lsucitem{\noPatentLitigation}
  \end{lsucprohibits}
\end{lsuc}


\subsection{ApL-C9: Passing a modified library as embedded binary}
\begin{lsuc}{APL-C9}
  \linkosuc{10B}

  \lsucmeans{that you are going to distribute a modified version of the Apache
    licensed code snippet, module, library, or plugin that you received to third
    parties in the form of binary files or as a binary package together with
    another larger software unit which contains this code snippet, module,
    library, or plugin as an embedded component.}

  \lsuccovers{OSUC-10B\footnote{For details $\rightarrow$ OSLiC,
      pp.\ \pageref{OSUC-10B-DEF}}} 

  \begin{lsucrequires}
    \lsucmandatory{\copyLicenseWithBinary}\passingFiles
    \lsucmandatory{\keepLicenseElementsBinary}
    \lsucmandatory{\expandNotices}
 
    \lsucmandatory{\displayNoticeOfEmbeddedLibrary}
     
    \lsucoptional{Even if you do not want to distribute your modified source
      code, mark all your modifications of the embedded libary thoroughly.}
    \footnote{library or snippet, or module, or plugin} 

    \lsucoptional{\documentationBinary}
  
    \lsucoptional{Arrange your binary distribution so that the integrated Apache
      license and the \emph{notice text file} clearly refer only to the embedded
      library and do not disturb the licensing of your own overarching
      work. It's a good tradition to keep the libraries, modules, snippet, or
      plugins in specific directories which contain also all licensing elements.}
  \end{lsucrequires}

  \begin{lsucprohibits}
    \lsucitem{\noTrademarks}
    \lsucitem{\noPatentLitigation}
  \end{lsucprohibits}
\end{lsuc}

\subsection{Discussions and Explanations}
\label{APLDiscussion}
\begin{itemize}
  \item On the one hand, the Apache 2.0 license does not permit
  \enquote{[\ldots] to use the trade names, trademarks, service marks, or
  product names of the Licensor, except as required for reasonable and customary
  use in describing the origin of the Work and reproducing the content of the
  NOTICE file}\footcite[cf.][\nopage wp.\ §6]{Apl20OsiLicense2004a}. On the other
  hand, this license alerts that all the patent licenses granted to those who
  \enquote{[\ldots] institute a patent litigation} will terminate
  automatically\footcite[cf.][\nopage wp.\ §3]{Apl20OsiLicense2004a}. Hence, the
  OSLiC generally (ApL-C1 - ApL-C9) interdicts to promote products or services by
  these elements and to legally fight against patents linked to the software.
  
  \item The ApL also requires to \enquote{[\ldots] give any other recipients of
  the Work or Derivative Works a copy of this License}\footcite[cf.][\nopage wp.\
  §4.1]{Apl20OsiLicense2004a}. Therefore, all \emph{2others} use cases contain
  the respective mandatory condition (ApL-C2 - ApL-C9).
   
  \item Additionally, the ApL requires, that modifications must be
  marked\footcite[cf.][\nopage wp.\ §4.2]{Apl20OsiLicense2004a}. Thus, in all
  cases of passing the modified software in the form of source code the OSLiC
  requires to mark the modifications and to integrate a hint into the notice
  file---while in all the cases of passing the modified software in the form of
  binaries it inserts only a voluntary condition (ApL-C4 - ApL-C9).
  
  \item Furthermore, the ApL requires that one must \enquote{[\ldots] retain, in
  the Source form of any Derivative Works that You distribute, all copyright,
  patent, trademark, and attribution notices from the Source form of the Work}
  So, the OSLIC requires in all contexts (ApL-C1 - ApL-C9) that the licensing
  elements are retained in the form you have received them\footnote{This might
  confuse some readers: Yes, even if you distribute a modified version in the
  form of binaries you must fulfill this condition. Moreover, you must also hand
  the license over to your receipient. But, nevertheless, you are not obliged to
  publish the modified source code, too. ($\rightarrow$ OSLiC, p.
  \protectionpageref{APL})}.
  
  \item Finally, the ApL requires that the received ``NOTICE text file'' must be
  integrated as readable copy to each package distributed in the form of source
  code, or---in case of binary distibutions---must be displayed
  \enquote{[\ldots] if and wherever such third-party notices normally
  appear}\footcite[cf.][\nopage wp.\ §4.4]{Apl20OsiLicense2004a}. Thus, the OSLiC
  requires mandatorily that all source code distributions must include the
  notice text file (ApL-C2, ApL-C4, ApL-C6, ApL-C8) and that all distributions of
  binary applications which normally show such a copyrigth screen must integrate
  the content of the notice file into this screen (ApL-C5, ApL9). For libraries
  distributed in the form of binaries it is assumed that they normally do not
  contain such copyright dialogs (ApL-C7)
\end{itemize}

\end{license}

%\bibliography{../../../bibfiles/oscResourcesEn}

% Local Variables:
% mode: latex
% fill-column: 80
% End:
