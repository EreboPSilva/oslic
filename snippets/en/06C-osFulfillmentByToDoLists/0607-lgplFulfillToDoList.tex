% Telekom osCompendium 'for being included' snippet template
%
% (c) Karsten Reincke, Deutsche Telekom AG, Darmstadt 2011
%
% This LaTeX-File is licensed under the Creative Commons Attribution-ShareAlike
% 3.0 Germany License (http://creativecommons.org/licenses/by-sa/3.0/de/): Feel
% free 'to share (to copy, distribute and transmit)' or 'to remix (to adapt)'
% it, if you '... distribute the resulting work under the same or similar
% license to this one' and if you respect how 'you must attribute the work in
% the manner specified by the author ...':
%
% In an internet based reuse please link the reused parts to www.telekom.com and
% mention the original authors and Deutsche Telekom AG in a suitable manner. In
% a paper-like reuse please insert a short hint to www.telekom.com and to the
% original authors and Deutsche Telekom AG into your preface. For normal
% quotations please use the scientific standard to cite.
%
% [ Framework derived from 'mind your Scholar Research Framework' 
%   mycsrf (c) K. Reincke 2012 CC BY 3.0  http://mycsrf.fodina.de/ ]
%


%% use all entries of the bibliography
%\nocite{*}

\section{LGPL licensed software}

Both versions of the GNU Lesser General Public License explicitly distinguish
the distribution of the source code from that of the binaries: On the one hand,
the LGPL-2.1 mainly talks about copying and distributing the source
code\footcite[cf.][\nopage wp.\ §1, §2, §5, §6]{Lgpl21OsiLicense1999a}. But it
also directly mentions the specific conditions for \enquote{[\ldots] (copying)
and (distributing) the Library [\ldots] in object code or executable form
[\ldots]}\footcite[cf.][\nopage wp.\ §4]{Lgpl21OsiLicense1999a}. On the other
hand, also the LGPL-3.0 and the GPL-3.0 -- which have to be considered together
because the GPL-3.0 is included into the LGPL-3.0\footcite[cf.][\nopage wp, just
before §0]{Lgpl30OsiLicense2007a} -- are treating the distribution of source
code and the distribution of the object code as different aspects of the same
phenomenon\footnote{The GPL-3.0 contains a specific section named
\enquote{Conveying Non-Source Forms} which describes the conditions to
\enquote{[\ldots] convey a covered work in object code form [\ldots]}
(\cite[cf.][\nopage wp.\ §6]{Gpl30OsiLicense2007a})), while the LGPL-3.0
explicitly deals with the \enquote{object code incorporating material from (the)
library header files}(\cite[cf.][\nopage wp.\ §3]{Lgpl30OsiLicense2007a})).}.
Additionally, LGPL-2.1 and LGPL-3.0 mainly talk about copying and distribution
the software; the private use is nearly complete unspecified\footnote{The
LGPL-2.1 lists its 'restrictions' only with respect to the act of copying and
distributing \enquote{copies of the library} (\cite[cf.][\nopage wp.\ §1, §2, §4
et passim]{Lgpl21OsiLicense1999a}) while the GPL-3.0 explicitly specifies that
one \enquote{[\ldots] may make, run and propagate covered works that (one does)
not convey, without conditions so long as (the) license otherwise remains in
force} (\cite[cf.][\nopage wp.\ §2]{Gpl30OsiLicense2007a}).}. Finally, the
LGPL-2.1 and the LGPL-3.0 are aiming for the same results and the same spirit by
requiring nearly the same license fulfilling tasks. Therefore it is mostly
appropriate to cover both versions in one chapter\footnote{The exception
concerns the distribution of a modified program, application, or server under
the terms of the LGPL} and to offer the same LGPL specific open source use
case structure\footnote{For details of the general OSUC finder $\rightarrow$
OSLiC, pp.\ \pageref{OsucTokens} and \pageref{OsucDefinitionTree}} for finding
the easily processable corresponding task lists:
 
\tikzstyle{nodv} = [font=\small, ellipse, draw, fill=gray!10, 
    text width=2cm, text centered, minimum height=2em]


\tikzstyle{nods} = [font=\footnotesize, rectangle, draw, fill=gray!20, 
    text width=1.2cm, text centered, rounded corners, minimum height=3em]

\tikzstyle{nodb} = [font=\footnotesize, rectangle, draw, fill=gray!20, 
    text width=2.2cm, text centered, rounded corners, minimum height=3em]
    
\tikzstyle{leaf} = [font=\tiny, rectangle, draw, fill=gray!30, 
    text width=1.2cm, text centered, minimum height=6em]

\tikzstyle{edge} = [draw, -latex']

\begin{tikzpicture}[]

\node[nodv] (l81) at (4,11.8) {LGPL};

\node[nodv] (l71) at (0,10.2) {version 2.1};
\node[nodv] (l72) at (6.5,10.2) {version 3.0};


\node[nodb] (l61) at (0,8.6) {\textit{recipient:} \\ \textbf{4yourself}};
\node[nodb] (l62) at (6.5,8.6) {\textit{recipient:} \\ \textbf{2others}};

\node[nodb] (l51) at (2.5,7) {\textit{state:} \\ \textbf{unmodified}};
\node[nodb] (l52) at (9.3,7) {\textit{state:} \\ \textbf{modified}};

\node[nods] (l41) at (1.8,5.4) {\textit{form:} \textbf{source}};
\node[nods] (l42) at (3.6,5.4) {\textit{form:} \textbf{binary}};
\node[nodb] (l43) at (6.5,5.4) {\textit{type:} \\ \textbf{proapse}};
\node[nodb] (l44) at (12,5.4) {\textit{type:} \\ \textbf{snimoli}};


\node[nods] (l31) at (5.4,3.8) {\textit{form:} \textbf{source}};
\node[nods] (l32) at (7.2,3.8) {\textit{form:} \textbf{binary}};
\node[nodb] (l33) at (10,3.8) {\textit{context:} \\ \textbf{independent}};
\node[nodb] (l34) at (13.5,3.8) {\textit{context:} \\ \textbf{embedded}};

\node[nods] (l21) at (9,2.2) {\textit{form:} \textbf{source}};
\node[nods] (l22) at (10.8,2.2) {\textit{form:} \textbf{binary}};
\node[nods] (l23) at (12.6,2.2) {\textit{form:} \textbf{source}};
\node[nods] (l24) at (14.4,2.2) {\textit{form:} \textbf{binary}};

\node[leaf] (l11) at (0,0) {\textbf{LGPL-1} \textit{using software only
for yourself}};

\node[leaf] (l12) at (1.8,0) { \textbf{LGPL-2} \textit{ distributing unmodified
software as sources}};

\node[leaf] (l13) at (3.6,0) { \textbf{LGPL-3}  \textit{ distributing unmodified
software as binaries}};

\node[leaf] (l14) at (5.4,0) { \textbf{LGPL-4}  \textit{ distributing modified
program as sources}};

\node[leaf] (l15) at (7.2,0) { \textbf{LGPL-5}  \textit{ distributing modified
program as binaries}};

\node[leaf] (l16) at (9,0) { \textbf{LGPL-6}  \textit{ distributing modified
library as independent sources}};

\node[leaf] (l17) at (10.8,0) { \textbf{LGPL-7} \textit{distributing modified
library as independent binaries}};

\node[leaf] (l18) at (12.6,0) { \textbf{LGPL-8}  \textit{distributing
modified library as embedded sources}};

\node[leaf] (l19) at (14.4,0) { \textbf{LGPL-9}  \textit{ distributing modified
library as embedded binaries}};

\path [edge] (l81) -- (l71);
\path [edge] (l81) -- (l72);
\path [edge] (l71) -- (l61);
\path [edge] (l71) -- (l62);
\path [edge] (l72) -- (l61);
\path [edge] (l72) -- (l62);
\path [edge] (l61) -- (l11);
\path [edge] (l62) -- (l51);
\path [edge] (l62) -- (l52);
\path [edge] (l51) -- (l41);
\path [edge] (l51) -- (l42);
\path [edge] (l52) -- (l43);
\path [edge] (l52) -- (l44);
\path [edge] (l41) -- (l12);
\path [edge] (l42) -- (l13);
\path [edge] (l43) -- (l31);
\path [edge] (l43) -- (l32);
\path [edge] (l44) -- (l33);
\path [edge] (l44) -- (l34);
\path [edge] (l31) -- (l14);
\path [edge] (l32) -- (l15);
\path [edge] (l33) -- (l21);
\path [edge] (l33) -- (l22);
\path [edge] (l34) -- (l23);
\path [edge] (l34) -- (l24);
\path [edge] (l21) -- (l16);
\path [edge] (l22) -- (l17);
\path [edge] (l23) -- (l18);
\path [edge] (l24) -- (l19);

\end{tikzpicture}


\subsection{LGPL-1: Using the software only for yourself}
\label{OSUC-01-LGPL} \label{OSUC-03-LGPL}
\label{OSUC-06-LGPL} \label{OSUC-09-LGPL}

\begin{description}

\item[means] that you are going to use a received LGPL licensed software only
for yourself and that you do not hand it over to any 3rd party in any sense.

\item[covers] OSUC-01, OSUC-03, OSUC-06, and OSUC-09\footnote{For details
$\rightarrow$ OSLiC, pp.\ \pageref{OSUC-01-DEF} - \pageref{OSUC-09-DEF}}

\item[requires] no tasks in order to fulfill the conditions of the LGPL-2.1 or
the LGPL-3.0 with respect to this use case:
  \begin{itemize}
    \item You are allowed to use any kind of LGPL software in any sense and in
    any context without being obliged to do anything as long as you do not
    give the software to 3rd parties.
  \end{itemize}

\item[prohibits] nothing explictly with respect to this use case.
\end{description}


\subsection{LGPL-2: Passing the unmodified software as source code}
\label{OSUC-02S-LGPL} \label{OSUC-05S-LGPL} \label{OSUC-07S-LGPL} 

\begin{description}

\item[means] that you are going to distribute an unmodified version of the
received LGPL software to 3rd parties -- in the form of source code files or as
a source code package. In this case it doesn't matter whether you distribute a
program, an application, a server, a snippet, a module, a library, or a plugin
as an independent or an embedded unit.

\item[covers] OSUC-02S, OSUC-05S, OSUC-07S\footnote{For details $\rightarrow$
OSLiC, pp.\ \pageref{OSUC-02S-DEF} - \pageref{OSUC-07S-DEF}}

\item[requires] the following tasks in order to fulfill the license conditions:
\begin{itemize}
 
  \item \textbf{[mandatory:]} Ensure that the licensing elements -- esp.\ all
  notices that refer to the LGPL-2.1 or LGPL-3.0 and to the absence of any
  warranty -- are retained in your package in the form you have received them.

  \item \textbf{[mandatory:]} Ensure that the distributed software package
  contains a conspicuously and appropriately designed, easily to find copyright
  notice and a disclaimer of warranty. If these elements are missed, add a new
  file containing the main copyright notice and the disclaimer of warranty in the
  form which is textually defined by the license LGPL-2.1 itself resp. by the
  LGPL-3.0 itself. (Yes, repeat the disclaimer although it is also part of the
  license itself and although you are required to hand the license itself over
  to the receiver.)
  
  \item \textbf{[mandatory:]} Give the recipient a copy of the LGPL-2.1 resp.
  LGPL-3.0 license. If it is not already part of the software package, add
  it\footnote{For implementing the handover of files correctly $\rightarrow$
  OSLiC, p. \pageref{DistributingFilesHint}}.
  
  \item \textbf{[voluntary:]} Let the documentation of your distribution and/or
  your additional material also reproduce the content of the existing
  copyright notices, a hint to the software name, a link to its homepage,
  the respective disclaimer of warranty, and a link to the LGPL-2.1 resp.
  LGPL-3.0.
  
  \item \textbf{[voluntary:]} Retain all existing copyright notices and
  licensing elements.
  
\end{itemize}

\item[prohibits] nothing explictly with respect to this use case.

\end{description}


\subsection{LGPL-3: Passing the unmodified software as binaries} 
\label{OSUC-02B-LGPL} \label{OSUC-05B-LGPL} \label{OSUC-07B-LGPL} 

\begin{description}
\item[means] that you are going to distribute an unmodified version of the
received LGPL software to 3rd parties -- in the form of binary files or as a
bi\-na\-ry package. In this case it doesn't matter whether you distribute a
program, an application, a server, a snippet, a module, a library, or a plugin
as an independent or an embedded unit.

\item[covers] OSUC-02B, OSUC-05B, OSUC-07B\footnote{For details $\rightarrow$ OSLiC, pp.\
\pageref{OSUC-02B-DEF} - \pageref{OSUC-07B-DEF}}

\item[requires] the following tasks in order to fulfill the license conditions:
\begin{itemize}
  
  \item \textbf{[mandatory:]} Ensure that the licensing elements -- esp.\ all
  notices that refer to the LGPL-2.1 or LGPL-3.0 and to the absence of any
  warranty -- are retained in your package in the form you have received them.

  \item \textbf{[mandatory:]} Ensure that the distributed software binary
  package contains a conspicuously and appropriately designed, easily to find
  copyright notice and a disclaimer of warranty. If these elements are missed,
  add a new file containing the main copyright notice and the disclaimer of
  warranty in the form which is textually defined by the license LGPL-2.1 itself
  resp. by the LGPL-3.0 itself. (Yes, repeat the disclaimer although it is also
  part of the license itself and although you are required to hand the license
  itself over to the receiver.)
  
  \item \textbf{[mandatory:]} Give the recipient a copy of the LGPL-2.1 resp.
  LGPL-3.0 license. If it is not already part of the software package, add
  it\footnote{For implementing the handover of files correctly $\rightarrow$
  OSLiC, p. \pageref{DistributingFilesHint}}.
  
  \item \textbf{[mandatory:]} Make the source code of the distributed software
  accessible via a repository under your own control (even if you do not
  modified it): Push the source code package into a repository, make it
  downloadable via the internet, and integrate an easily to find description
  into the distribution package which explains how the code can be received from
  where. Ensure, that this repository is online for for at least 3 years after
  having distributed the last instance of your software package.
  
  \item \textbf{[mandatory:]} Insert a prominent hint to the download repository
  into your distribution and/or your additional material.
  
  \item \textbf{[mandatory:]} Execute the to-do list of use case LGPL-2\footnote{
  Making the code accessible via a repository means distributing the software in
  the form of source code. Hence, you must also fulfill all tasks of the
  corresponding use case.}.
   
  \item \textbf{[voluntary:]} Let the documentation of your distribution and/or
  your additional material also reproduce the content of the existing
  copyright notices, a hint to the software name, a link to its homepage,
  the respective disclaimer of warranty, and a link to the LGPL-2.1 resp.
  LGPL-3.0.
  
  \item \textbf{[voluntary:]} Retain all existing copyright notices and
  licensing elements.

\end{itemize}

\item[prohibits] nothing explictly with respect to this use case.

\end{description}

\subsection{LGPL-4: Passing a modified program as source code}
\label{OSUC-04S-LGPL} 

\begin{description}
\item[means] that you are going to distribute a modified version of the received
LGPL licensed program, application, or server (proapse) to 3rd parties -- in the
form of source code files or as a source code package.
\item[covers] OSUC-04S\footnote{For details $\rightarrow$ OSLiC, pp.\
\pageref{OSUC-04S-DEF}}
\end{description}

\subsubsection{under terms of LGPL-2.1}

\begin{description}
  \item[requires] \ldots [irrelevant] 
  \item[forbids] to modify the received work in a way that the resulting
  \enquote{modified work} is no longer a software library (but a
  program)\footnote{The LGPL-2.1 explictly requires that \enquote{the modified
  work must itself be a software library} (\cite[cf.][\nopage wp.\
  §2a]{Lgpl21OsiLicense1999a}). For details $\rightarrow$ OSLiC, p.\
  \pageref{para:libislib}}. \textbf{Hence: \emph{you are not allowed to
  distribute a modified program under th terms of LGPL-2.1.}}
\end{description}

\subsubsection{under terms of LGPL-3.0}

\begin{description}
\item[requires] to respect
\begin{itemize}
  
  \item \textbf{[mandatory:]} Ensure that the licensing elements -- esp.\ all
  notices that refer to the LGPL-3.0 and to the absence of any
  warranty -- are retained in your package in the form you have received them.

  \item \textbf{[mandatory:]} Ensure that the distributed source code package
  contains a conspicuously and appropriately designed, easily to find copyright
  notice and a disclaimer of warranty. If these elements are missed, add a new
  file containing the main copyright notice and the disclaimer of warranty in the
  form which is textually defined by the license LGPL-3.0 itself. (Yes, repeat
  the disclaimer although it is also part of the license itself and although you
  are required to hand the license itself over to the receiver.)
  
  \item \textbf{[mandatory:]} Give the recipient a copy of the LGPL-3.0 license.
  If it is not already part of the software package, add it\footnote{For
  implementing the handover of files correctly $\rightarrow$ OSLiC, p.
  \pageref{DistributingFilesHint}}.

  \item \textbf{[mandatory:]} Mark all modifications of source code of the
  program (proapse) thoroughly -- namely within the source code and including
  the date of the modification.
  
  \item \textbf{[mandatory:]} Organize your modifications in a way that they are
  covered by the existing LGPL licensing statements.
  
  \item \textbf{[voluntary:]} Create a \emph{modification text file}, if such a
  notice file still does not exist. \emph{Expand} the \emph{modification text
  file} by a description of your modifications on a more functional level.
    
  \item \textbf{[voluntary:]} Let the documentation of your distribution and/or
  your additional material also reproduce the content of the existing
  copyright notices, a hint to the software name, a link to its homepage,
  the respective disclaimer of warranty, and a link to the LGPL-3.0.
  
  \item \textbf{[voluntary:]} Retain all existing copyright notices and
  licensing elements. 
  
 \end{itemize}
 
\item[prohibits] nothing explictly with respect to this use case.

\end{description}

\subsection{LGPL-5: Passing a modified program as binary}
\label{OSUC-04B-LGPL} 

\begin{description}
\item[means] that you are going to distribute a modified version of the received
LGPL licensed pro\-gram, application, or server (proapse) to 3rd parties -- in
the form of binary files or as a binary package.
\item[covers] OSUC-04B\footnote{For details $\rightarrow$ OSLiC, pp.\
\pageref{OSUC-04B-DEF}}
\end{description}

\subsubsection{under terms of LGPL-2.1}

\begin{description}
  \item[requires] irrelevant
  \item[forbids] to modify the received work in a way that the resulting
  \enquote{modified work} is no longer a software library (but a
  program)\footnote{The LGPL-2.1 explictly requires that \enquote{the modified
  work must itself be a software library} (\cite[cf.][\nopage wp.\
  §2a]{Lgpl21OsiLicense1999a}. For details $\rightarrow$ OSLiC, pp.\
  \pageref{para:libislib}}. \textbf{Hence: \emph{you are not allowed to
  distribute a modified program under th terms of LGPL-2.1.}}
\end{description}

\subsubsection{under terms of LGPL-3.0}
\begin{description}
\item[requires] to respect:
\begin{itemize}

  \item \textbf{[mandatory:]} Ensure that the licensing elements -- esp.\ all
  notices that refer to the LGPL-3.0 and to the absence of any
  warranty -- are retained in your package in the form you have received them.

  \item \textbf{[mandatory:]} Ensure that the distributed software binary
  package contains a conspicuously and appropriately designed, easily to find
  copyright notice and a disclaimer of warranty. If these elements are missed,
  add a new file containing the main copyright notice and the disclaimer of
  warranty in the form which is textually defined by the license LGPL-3.0
  itself. (Yes, repeat the disclaimer although it is also part of the license
  itself and although you are required to hand the license itself over to the
  receiver.)
  
  \item \textbf{[mandatory:]} Give the recipient a copy of the LGPL-3.0 license.
  If it is not already part of the software package, add it\footnote{For
  implementing the handover of files correctly $\rightarrow$ OSLiC, p.
  \pageref{DistributingFilesHint}}.

  \item \textbf{[mandatory:]} Mark all modifications of source code of the
  program (proapse) thoroughly namely within the source code and including
  the date of the modification.
  
  \item \textbf{[mandatory:]} Organize your modifications in a way that they are
  covered by the existing LGPL licensing statements.
  
  \item \textbf{[mandatory:]} Make the source code of the distributed software
  accessible via a repository under your own control: Push the source code
  package into a repository, make it downloadable via the internet, and
  integrate an easily to find description into the distribution package which
  explains how the code can be received from where. Ensure, that this repository
  is online for for at least 3 years after having distributed the last instance
  of your software package.
  
  \item \textbf{[mandatory:]} Insert a prominent hint to the download repository
  into your distribution and/or your additional material.
  
  \item \textbf{[mandatory:]} Execute the to-do list of use case LGPL-4\footnote{
  Making the code accessible via a repository means distributing the software in
  the form of source code. Hence, you must also fulfill all tasks of the
  corresponding use case.}.
    
  \item \textbf{[voluntary:]} Create a \emph{modification text file}, if such a
  notice file still does not exist. \emph{Expand} the \emph{modification text
  file} by a description of your modifications on a more functional level.
  
  \item \textbf{[voluntary:]} Let the documentation of your distribution and/or
  your additional material also reproduce the content of the existing
  copyright notices, a hint to the software name, a link to its homepage,
  the respective disclaimer of warranty, and a link to the LGPL-3.0.
  
  \item \textbf{[voluntary:]} Retain all existing copyright notices and
  licensing elements. 


\end{itemize}

\item[prohibits] nothing explictly with respect to this use case.

\end{description}

\subsection{LGPL-6: Passing a modified library as independent source code}
\label{OSUC-08S-LGPL}

\begin{description}
\item[means] that you are going to distribute a modified version of the received
LGPL licensed code snippet, module, library, or plugin (snimoli) to 3rd parties
-- in the form of source code files or as a source code package, but without
embedding it into another larger software unit.
\item[covers] OSUC-08S\footnote{For details $\rightarrow$ OSLiC, pp.\
\pageref{OSUC-08S-DEF}}
\item[requires] the tasks in order to fulfill the license conditions:
\begin{itemize}
 
   \item \textbf{[mandatory:]} Ensure that the licensing elements -- esp.\ all
  notices that refer to the LGPL-2.1 or LGPL-3.0 and to the absence of any
  warranty -- are retained in your package in the form you have received them.

  \item \textbf{[mandatory:]} Ensure that the distributed source code package
  contains a conspicuously and appropriately designed, easily to find copyright
  notice and a disclaimer of warranty. If these elements are missed, add a new
  file containing the main copyright notice and the disclaimer of warranty in the
  form which is textually defined by the license LGPL-2.1 itself resp. by the
  LGPL-3.0 itself. (Yes, repeat the disclaimer although it is also part of the
  license itself and although you are required to hand the license itself over
  to the receiver.)
  
  \item \textbf{[mandatory:]} Give the recipient a copy of the LGPL-2.1 resp.
  LGPL-3.0 license. If it is not already part of the software package, add
  it\footnote{For implementing the handover of files correctly $\rightarrow$
  OSLiC, p. \pageref{DistributingFilesHint}}.
  
  \item \textbf{[mandatory:]} Mark all modifications of source code of the
  library (snimoli) thoroughly -- namely within the source code and including
  the date of the modification.
  
  \item \textbf{[mandatory:]} Organize your modifications in a way that they are
  covered by the existing LGPL licensing statements.
    
  \item \textbf{[voluntary:]} Create a \emph{modification text file}, if such a
  notice file still does not exist. \emph{Expand} the \emph{modification text
  file} by a description of your modifications.
  
  \item \textbf{[voluntary:]} Let the documentation of your distribution and/or
  your additional material also reproduce the content of the existing
  copyright notices, a hint to the software name, a link to its homepage,
  the respective disclaimer of warranty, and a link to the LGPL-2.1 resp.
  LGPL-3.0.
  
  \item \textbf{[voluntary:]} Retain all existing copyright notices and
  licensing elements. 

\end{itemize}

\item[prohibits] \ldots
\begin{itemize}
  \item to modify the library in a way that it is no longer a library
  (LGPL-2.1).
\end{itemize}

\end{description}


\subsection{LGPL-7: Passing a modified library as independent binary}
\label{OSUC-08B-LGPL}

\begin{description}
\item[means] that you are going to distribute a modified version of the received
LGPL licensed code snippet, module, library, or plugin (snimoli) to 3rd parties
-- in the form of binary files or as a binary package but without embedding it
into another larger software unit.
\item[covers] OSUC-08B\footnote{For details $\rightarrow$ OSLiC, pp.\ \pageref{OSUC-08-DEF}}
\item[requires] the tasks in order to fulfill the license conditions:
\begin{itemize}

  \item \textbf{[mandatory:]} Ensure that the licensing elements -- esp.\ all
  notices that refer to the LGPL-2.1 or LGPL-3.0 and to the absence of any
  warranty -- are retained in your package in the form you have received them.

  \item \textbf{[mandatory:]} Ensure that the distributed software binary
  package contains a conspicuously and appropriately designed, easily to find
  copyright notice and a disclaimer of warranty. If these elements are missed,
  add a new file containing the main copyright notice and the disclaimer of
  warranty in the form which is textually defined by the license LGPL-2.1 itself
  resp. by the LGPL-3.0 itself. (Yes, repeat the disclaimer although it is also
  part of the license itself and although you are required to hand the license
  itself over to the receiver.)
  
  \item \textbf{[mandatory:]} Give the recipient a copy of the LGPL-2.1 resp.
  LGPL-3.0 license. If it is not already part of the software package, add
  it\footnote{For implementing the handover of files correctly $\rightarrow$
  OSLiC, p. \pageref{DistributingFilesHint}}.

  \item \textbf{[mandatory:]} Make the source code of the distributed software
  accessible via a repository under your own control: Push the source code
  package into a repository, make it downloadable via the internet, and
  integrate an easily to find description into the distribution package which
  explains how the code can be received from where. Ensure, that this repository
  is online for for at least 3 years after having distributed the last instance
  of your software package.
  
  \item \textbf{[mandatory:]} Insert a prominent hint to the download repository
  into your distribution and/or your additional material.
  
  \item \textbf{[mandatory:]} Execute the to-do list of use case LGPL-6\footnote{
  Making the code accessible via a repository means distributing the software in
  the form of source code. Hence, you must also fulfill all tasks of the
  corresponding use case.}.

  \item \textbf{[mandatory:]} Mark all modifications of source code of the
  library (snimoli) thoroughly -- namely within the source code and including
  the date of the modification.
  
  \item \textbf{[mandatory:]} Organize your modifications in a way that they are
  covered by the existing LGPL licensing statements.

  \item \textbf{[voluntary:]} Create a \emph{modification text file}, if such a
  notice file still does not exist. \emph{Expand} the \emph{modification text
  file} by a description of your modifications.

  \item \textbf{[voluntary:]} Let the documentation of your distribution and/or
  your additional material also reproduce the content of the existing
  copyright notices, a hint to the software name, a link to its homepage,
  the respective disclaimer of warranty, and a link to the LGPL-2.1 resp.
  LGPL-3.0.
  
  \item \textbf{[voluntary:]} Retain all existing copyright notices and
  licensing elements. 
  
\end{itemize}

\item[prohibits] \ldots
\begin{itemize}
  \item to modify the library in a way that it is no longer a library
  (LGPL-2.1).
\end{itemize}

\end{description}

\subsection{LGPL-8: Passing a modified library as embedded source code}
\label{OSUC-10S-LGPL}

\begin{description}
\item[means] that you are going to distribute a modified version of the received
LGPL licensed code snippet, module, library, or plugin (snimoli) to 3rd parties
-- in the form of source code files or as a source code package together with
another larger software unit which contains this code snippet, module, library,
or plugin as an embedded component.
\item[covers] OSUC-10S\footnote{For details $\rightarrow$ OSLiC, pp.\
\pageref{OSUC-10S-DEF}}
\item[requires] the tasks in order to fulfill the license conditions:
\begin{itemize}


  \item \textbf{[mandatory:]} Ensure that the licensing elements -- esp.\ all
  notices that refer to the LGPL-2.1 or LGPL-3.0 and to the absence of any
  warranty -- are retained in your package in the form you have received them.

  \item \textbf{[mandatory:]} Ensure that the distributed software package
  contains a conspicuously and appropriately designed, easily to find copyright
  notice and a disclaimer of warranty. If these elements are missed, add a new
  file containing the main copyright notice and the disclaimer of warranty in the
  form which is textually defined by the license LGPL-2.1 itself resp. by the
  LGPL-3.0 itself. (Yes, repeat the disclaimer although it is also part of the
  license itself and although you are required to hand the license itself over
  to the receiver.)
  
  \item \textbf{[mandatory:]} Give the recipient a copy of the LGPL-2.1 resp.
  LGPL-3.0 license. If it is not already part of the software package, add
  it\footnote{For implementing the handover of files correctly $\rightarrow$
  OSLiC, p. \pageref{DistributingFilesHint}}.
    
  \item \textbf{[mandatory:]} Mark all modifications of source code of the
  embedded library (snimoli) thoroughly -- namely within the source code and
  including the date of the modification.
  
  \item \textbf{[mandatory:]} Organize your modifications of the embedded
  library in a way that they are covered by the existing LGPL licensing
  statements. 

  \item \textbf{[mandatory:]}  Maintain the structural independence of the
  library.
   
  \item \textbf{[mandatory:]} Let the copyright dialog of the on-top development
  clearly say, that it uses the LGPL licensed library. Let it reproduce the
  content of the existing copyright notices, a hint to the software name, a link
  to its homepage, the respective disclaimer of warranty, and a link to the
  LGPL-2.1 resp. LGPL-3.0.
     
  \item \textbf{[voluntary:]} Create a \emph{modification text file}, if such a
  notice file still does not exist. \emph{Expand} the \emph{modification text
  file} by a description of your modifications.
  
 \item \textbf{[voluntary:]} Let the documentation of your distribution and/or
  your additional material also clearly say, that it uses the LGPL licensed
  library. Let it reproduce the content of the existing copyright notices, a
  hint to the software name, a link to its homepage, the respective disclaimer
  of warranty, and a link to the LGPL-2.1 resp. LGPL-3.0.
  
  \item \textbf{[voluntary:]} Retain all existing copyright notices and
  licensing elements.
  
\end{itemize}

\item[prohibits] \ldots
\begin{itemize}
  \item to modify the library in a way that it is no longer a library
  (LGPL-2.1).
\end{itemize}

\end{description}


\subsection{LGPL-9: Passing a modified library as embedded binary}
\label{OSUC-10B-LGPL}

\begin{description}
\item[means] that you are going to distribute a modified version of the received
LGPL licensed code snippet, module, library, or plugin to 3rd parties -- in the
form of binary files or as a binary package together with another larger
software unit which contains this code snippet, module, library, or plugin as an
embedded
component.
\item[covers] OSUC-10B\footnote{For details $\rightarrow$ OSLiC, pp.\
\pageref{OSUC-10B-DEF}}
\item[requires] the tasks in order to fulfill the license conditions:
\begin{itemize}

  \item \textbf{[mandatory:]} Ensure that the licensing elements -- esp.\ all
  notices that refer to the LGPL-2.1 or LGPL-3.0 and to the absence of any
  warranty -- are retained in your package in the form you have received them.

  \item \textbf{[mandatory:]} Ensure that the distributed software binary
  package contains a conspicuously and appropriately designed, easily to find
  copyright notice and a disclaimer of warranty. If these elements are missed,
  add a new file containing the main copyright notice and the disclaimer of
  warranty in the form which is textually defined by the license LGPL-2.1 itself
  resp. by the LGPL-3.0 itself. (Yes, repeat the disclaimer although it is also
  part of the license itself and although you are required to hand the license
  itself over to the receiver.)
  
  \item \textbf{[mandatory:]} Give the recipient a copy of the LGPL-2.1 resp.
  LGPL-3.0 license. If it is not already part of the software package, add
  it\footnote{For implementing the handover of files correctly $\rightarrow$
  OSLiC, p. \pageref{DistributingFilesHint}}.

  \item \textbf{[mandatory:]} Make the source code of the embedded library and
  the source code of your overarching program accessible via a repository under
  your own control: Push the source code package into a repository and make it
  downloadable via the internet. Integrate an easily to find description into
  the distribution package which explains how the code can be received from
  where. Ensure, that this repository is online for as long as you continue to
  distribute the software.

  \item \textbf{[mandatory:]} Insert a prominent hint to the download repository
  into your distribution and/or your additional material.
    
  \item \textbf{[mandatory:]} Execute the to-do list of use case LGPL-8\footnote{
  Making the code accessible via a repository means distributing the software in
  the form of source code. Hence, you must also fulfill all tasks of the
  corresponding use case.}.

  \item \textbf{[mandatory:]} Mark all modifications of source code of the
  embedded library (snimoli) thoroughly -- namely within the source code and
  including the date of the modification.
  
  \item \textbf{[mandatory:]} Organize your modifications of the embedded
  library in a way that they are covered by the existing LGPL licensing
  statements. 

  \item \textbf{[mandatory:]}  Maintain the structural independence of the
  library.
    
  \item \textbf{[mandatory:]} Let the copyright dialog of the on-top development
  clearly say, that it uses the LGPL licensed library. Let it reproduce the
  content of the existing copyright notices, a hint to the software name, a link
  to its homepage, the respective disclaimer of warranty, and a link to the
  LGPL-2.1 resp. LGPL-3.0.
  
  \item \textbf{[mandatory:]} Either distribute the on-top development and the
  library in the form of dynamically linkable parts or distribute the statically
  linked application together with a written offer, valid for at least three
  years, to give the user all object-files of the on-top development and the
  library, so that he can relink the application on its own behalf.
     
  \item \textbf{[voluntary:]} Create a \emph{modification text file}, if such a
  notice file still does not exist. \emph{Expand} the \emph{modification text
  file} by a description of your modifications. 
  
  \item \textbf{[voluntary:]} Let the documentation of your distribution and/or
  your additional material also clearly say, that it uses the LGPL licensed
  library. Let it reproduce the content of the existing copyright notices, a
  hint to the software name, a link to its homepage, the respective disclaimer
  of warranty, and a link to the LGPL-2.1 resp. LGPL-3.0.

  \item \textbf{[voluntary:]} Retain all existing copyright notices and
  licensing elements. 
  
\end{itemize}

\item[prohibits] \ldots
\begin{itemize}
  \item to modify the library in a way that it is no longer a library
  (LGPL-2.1).
\end{itemize}

\end{description}

\subsection{Discussions and Explanations}

\begin{itemize}
  
  \item The LGPL-2.1 allows to \enquote{[\ldots] to copy \emph{and} (to)
  distribute verbatim copies of the Library's complete source code as you
  receive it [...] provided that you [a] conspicuously and appropriately publish
  on each copy an appropriate copyright notice and disclaimer of warranty; [b]
  keep intact all the notices that refer to this License and to the absence of
  any warranty; and [c] distribute a copy of this License along with the
  Library}\footcite[cf.][\nopage wp.\ §1, emphasizes by
  KR]{Lgpl21OsiLicense1999a}. Additionally, the LGPL-2.1 allows the distribution
  of the modfied source code \enquote{under the terms of Section
  1}\footcite[cf.][\nopage wp.\ §2]{Lgpl21OsiLicense1999a} and the distribution
  of binaries \enquote{under the terms of Sections 1 and
  2}\footcite[cf.][\nopage wp.\ §4]{Lgpl21OsiLicense1999a}. But the LGPL does
  not require any tasks if you are using the work only for yourself. Thus, the
  quoted conditions of \enquote{Section 1} are mandatory for all use cases
  concerning the distribution of an LGPL licensed work (LGPL-2 -
  LGPL-9)\footnote{The GPL-3.0, which is included into the LGPL-3.0, uses a
  similar structure to establish the same requirements ($\rightarrow$ OSLiC, p.\
  \pageref{Gpl3ConditionsDistri}). Based on this fact one may conclude that the
  tasks which fulfill the corresponding LGPL-2.1 requirements together also fit
  the GPL-3.0 conditions and hence those of the LGPL-3.0.}.

  \item Although both versions of the LGPL does not explicitly require to retain
  the copyright notices in the form you have received them, it is nevertheless a
  very good idea not to modify these elements (LGPL-2 - LGPL-9).
  
  \item The LGPL-2.1 allows to \enquote{[\ldots] copy and (to) distribute the
  Library (or a portion or derivative of it [\ldots]) in object code or
  executable form u[\ldots] provided that you accompany it with the complete
  corresponding machine-readable source code [\ldots] on a medium customarily
  used for software interchange}. And the license subspecifies this condition in
  the meaning that if one makes the object code accessible without distributing
  it directly, then the same 'download' method for the source code fulfills this
  condition\footcite[cf.][\nopage wp.\ §4]{Lgpl21OsiLicense1999a}. So, no doubt:
  in a very strict reading, the LGPL requires to distribute the source code and
  the object code together and by the same method: either both on (for example)
  DVD or both for being downloaded; but not the one on DVD and the other by a
  download repository. But the first specification also says, that the
  \enquote{complete corresponding machine readable source code} has to be
  distributed \enquote{on a medium customarily used for software
  interchange}\footcite[cf.][\nopage wp.\ §4]{Lgpl21OsiLicense1999a}. The OSLiC
  understands the possibility to download files from the internet as a
  distribution \emph{on a medium [being today] customarily used for software
  interchange}. Therefore, the OSLiC requires for all open source use cases
  which refer to the distribution of binaries (LGPL-3, LGPL-5, LGPL-7, LGPL-9)
  to make the source code of the corresponding library accessible via an
  internet repository\footnote{In opposite to the LGPL-2.1, the GPL-3.0, which
  is included into the LGPL-3.0, explictily offers the option to distribute the
  sources via an internet server ($\rightarrow$ OSLiC, p.\
  \pageref{Gpl3CondCopyleft}). So, one may again conclude that the tasks
  which fulfill the corresponding LGPL-2.1 requirements together also fit the
  GPL-3.0 and the LGPL-3.0 conditions.}.
  
  \item The LGPL allows to \enquote{[\ldots] modify your copy or copies of
  the Library or any portion of it [\ldots] and (to) copy and distribute such
  modifications [\ldots]} only under some restrictions and
  condtions\footcite[cf.][\nopage wp.\ §2]{Lgpl21OsiLicense1999a}:
  \begin{itemize}
    \item First, modified files must be marked as modifications and marked by
    the date of the modification\footnote{For LGPL-2.1 see \cite[cf.][\nopage
    wp.\ §2]{Lgpl21OsiLicense1999a}. For GPL-3.0 (being included into the
    LGPL-3.0) see \cite[cf.][\nopage wp.\ §5]{Gpl30OsiLicense2007a}}. This
    condition must be respected by all open source use cases concerning the
    distribution of the modified work [LGPL-4 - LGPL -9], because even if one
    primarily intends to distribute binaries, one has also to deliver the source
    code. The OSLiC 'replaces' this requirement by the mandatory condition to
    mark each modified file and by the voluntary condition to update / generate
    a general changing file.
    
    \item Second, it requires not to let the modified version depend on external
    data structures without \enquote{[\ldots] (making) a good faith effort to
    ensure that, in the event an application does not supply such function or
    table, the facility still operates, and performs whatever part of its
    purpose remains meaningful}\footnote{For LGPL-2.1 see \cite[cf.][\nopage
    wp.\ §2d]{Lgpl21OsiLicense1999a}. For LGPL-3.0 see \cite[cf.][\nopage wp.\
    §2a]{Lgpl30OsiLicense2007a}}. The OSLiC rewrites this condition as the
    obligation to maintain the structural independence of the library in case of
    using the modified library as embedded component [LGPL-8 - LGPL-9].
    
    \item \label{para:libislib}Third, the LGPL-2.1 definitely requires, that
    \enquote{the modified work must itself be a software
    library}\footcite[cf.][\nopage wp.\ §2]{Lgpl21OsiLicense1999a}. This
    conditions can directly be incorprated as an interdiction into all use cases
    which refer to the modification of a library [LGPL-6 - LGPL-9]. But is
    difficult to respect this condition if one wants to modify a program which
    one has received under the terms of the LGPL-2.1. Logically, one can write
    an application and license it under the LGPL. But - as a consequence - that
    impedes the modification of this work because the result must be a library.
    The LGPL-3.0 does not contain such a requirement. Hence, the OSLiC allows
    the distribution of modified programs (LGPL-4, LGPL5) only if they are
    licensed under the terms of LGPL-3.0{as long as nobody has shown us an exit
    out of this trap.}.
  \end{itemize}
  
  \item Additionally, the LGPL-2.1 allows the licensee to distribute an
  overarching on-top development -- in the wording of the LGPL-2.1: a
  \enquote{work that uses the libary} \footcite[cf.][\nopage wp.\ §5,
  §6]{Lgpl21OsiLicense1999a} -- \enquote{as an exception to the Sections above}
  in \enquote{combination} with the library \enquote{under terms of your
  choice}\footcite[cf.][\nopage wp.\ §6]{Lgpl21OsiLicense1999a}, provided that
  the licensee fulfills additional conditions:  First, it must clearly be stated
  that the on-top development depends on the (modified) library. Second, the
  LGPL must be added into the distributed package. Third, in its own copyright
  dialog, the on-top development must mention the library, its copyright holder,
  and that it is licensed under the LGPL\footcite[cf.][\nopage wp.\
  §6]{Lgpl21OsiLicense1999a}. In the LGPL-3.0, this condition is similarily
  integrated: On the one hand, the \enquote{combined work} is defined as
  \enquote{a work produced by combining or linking an Application with the
  Library}\footcite[cf.][\nopage wp.\ §0]{Lgpl30OsiLicense2007a}. On the other
  hand, the LGPL-3.0 states that one \enquote{[\ldots] may convey a Combined
  Work under terms of (his own) choice} provided that one [a] clearly says that
  the overarching on-top development uses the LGPL licensed library, [b]
  distributes the LGPL-3.0 and the GPL-3.0 license as part of the package, [c]
  displays all these (licensing) information by the existing display
  technologies, [d] requires an appropriate shared library mechanism, and [e]
  offers the respectiv installion information\footcite[cf.][\nopage wp.\
  §4]{Lgpl30OsiLicense2007a}. These requirements can directly be inserted as
  conditions into the respective use cases -- namely for both LGPL versions
  (LGPL-8, LGPL-9).
  
  \item The most difficult requirements of the LGPL-2.1 concern the distribution
  in the form of binaries. In a very strict reading, the LGPL does not require
  to link the on-top development and the libary only dynamically. At first, the
  LGPL mentions, that the \enquote{[..] work (that uses the Library), in
  isolation, is not a derivative work of the Library [\ldots]}. But if it is
  linked to the library the resulting executable program -- of course -- becomes
  \enquote{a derivative of the Library} and that it is therefore
  \enquote{[\ldots] covered by this License (LGPL-2.1)}. But the LGPL-2.1
  directly continues this statement with the hint, that \enquote{Section 6
  states terms for distribution of such executables}\footcite[cf.][\nopage wp.\
  §5]{Lgpl21OsiLicense1999a}. Finally, this section 6 directly starts with
  statement: \enquote{As an exception to the Sections above, you may also
  combine or link a "work that uses the Library" with the Library to produce a
  work containing portions of the Library, and distribute that work under terms
  of your choice}\footcite[cf.][\nopage wp.\ §6]{Lgpl21OsiLicense1999a}.
  
  This is important to know, because until this section 6 one can not directly
  read or indirectly infer that the LGPL-2.1 distinguished the act of
  dynamically linking a program and a lirbrary from that of statically linking
  these parts. The LGPL only wants to ensure that the binaries of the library
  itself can be replaced by a newer version. And that is required by section
  6\footcite[cf.][\nopage wp.\ §6]{Lgpl21OsiLicense1999a}.
  From a technically viewpoint, this can only be guaranteed, if the binaries of
  the on-top development and the library together are \enquote{used in a
  suitable shared library mechanism}\footcite[cf.][\nopage wp.\
  §6]{Lgpl21OsiLicense1999a} or if one also gets all compiled, but not linked
  object-files of the on-top development and the library, either directly, or
  via using a \enquote{a written offer, valid for at least three years, to give
  the same user the (respective) materials}\footcite[cf.][\nopage wp.\
  §6]{Lgpl21OsiLicense1999a}. In the first case, the user can replace the
  received version of the library and can let the application be relinked
  automatically. In the second case, he has to do it manually. It is important
  to know that these ways exist if one wants or must distribute statically
  linked works. The LGPL-2.1 does not forbid to distribute statically linked
  applications. But it requires to enable the receiver to relink the work.
  
  The LGPL-3.0 has reduced these complex conditions in a special way: First, it
  does not use the words 'statically linked' or 'dynamically' linked. Second it
  defines the combined work 'only' as the result of \enquote{combining or
  linking an Application with the Library}\footcite[cf.][\nopage wp.\
  §0]{Lgpl30OsiLicense2007a}. But then it requires for the distribution of the
  combined works that one has either to \enquote{convey the Minimal
  Corresponding Source under the terms of this License, and the Corresponding
  Application Code in a form suitable for, and under terms that permit, the user
  to recombine or relink the Application with a modified version of the Linked
  Version to produce a modified Combined Work [\ldots]} or that one must
  presuppose that the receiver uses \enquote{[\ldots] suitable shared library
  mechanism for linking with the Library [\ldots] that [\ldots] operate properly
  with a modified version of the Library [\ldots]}\footcite[cf.][\nopage wp.\
  §4]{Lgpl30OsiLicense2007a}. Finally, the LGPL-3.0 adds that in the first case
  the these materials which enables the relinking must be distributed
  \enquote{[\ldots] in the manner specified by section 6 of the GNU GPL[-3.0]
  for conveying Corresponding Source}\footcite[cf.][\nopage wp.\
  §4]{Lgpl30OsiLicense2007a}. And this section 6 of the GPL-3.0 allows the well
  known method to \enquote{convey the object code [\ldots] accompanied by a
  written offer [\ldots] to give anyone [\ldots] access to copy the
  Corresponding Source from a network server at no charge}\footcite[cf.][\nopage
  wp.\ §6]{Gpl30OsiLicense2007a}

  Therefore, the OSLiC can condense these conditions into the requirement,
  either to distribute dynamically linkable parts, or to distribute statically
  linked applications \enquote{(accompanied) [\ldots] with a written offer,
  valid for at least three years, to give the same user the [complete]
  materials}\footcite[cf.][\nopage wp.\ §6]{Lgpl21OsiLicense1999a}, so that he
  can relink the application on its own behalf. It is clear, that this condition
  is only valid for the use case LGPL-9.
  
\end{itemize}





%\bibliography{../../../bibfiles/oscResourcesEn}
