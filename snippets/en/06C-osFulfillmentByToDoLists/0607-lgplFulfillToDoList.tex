% Telekom osCompendium 'for being included' snippet template
%
% (c) Karsten Reincke, Deutsche Telekom AG, Darmstadt 2011
%
% This LaTeX-File is licensed under the Creative Commons Attribution-ShareAlike
% 3.0 Germany License (http://creativecommons.org/licenses/by-sa/3.0/de/): Feel
% free 'to share (to copy, distribute and transmit)' or 'to remix (to adapt)'
% it, if you '... distribute the resulting work under the same or similar
% license to this one' and if you respect how 'you must attribute the work in
% the manner specified by the author ...':
%
% In an Internet based reuse please link the reused parts to www.telekom.com and
% mention the original authors and Deutsche Telekom AG in a suitable manner. In
% a paper-like reuse please insert a short hint to www.telekom.com and to the
% original authors and Deutsche Telekom AG into your preface. For normal
% quotations please use the scientific standard to cite.
%
% [ Framework derived from 'mind your Scholar Research Framework' 
%   mycsrf (c) K. Reincke 2012 CC BY 3.0  http://mycsrf.fodina.de/ ]
%


%% use all entries of the bibliography
%\nocite{*}

\section{LGPL licensed software}

Both versions of the GNU Lesser General Public License explicitly distinguish
the distribution of the source code from that of the binaries: On the one hand,
the LGPL-2.1 mainly talks about copying and distributing the source
code.\citeLGPLtwo{§1, §2, §5, §6} But it also directly mentions the specific
conditions for \enquote{[\ldots] (copying) and (distributing) the Library
[\ldots] in object code or executable form [\ldots]}\citeLGPLtwo{§4} On the other
hand, the LGPL-3.0 and the GPL-3.0---which have to be considered together
because the GPL-3.0 is included into the LGPL-3.0\citeLGPLthree{just before
  §0}--- treat the distribution of source code and the distribution of object
code as different aspects of the same phenomenon%
  \footnote{The GPL-3.0 contains a specific section named 
    \enquote{Conveying Non-Source Forms} which describes the conditions to
    \enquote{[\ldots] convey a covered work in object code form [\ldots]}
    (\cite[cf.][\nopage wp.\ §6]{Gpl30OsiLicense2007a}), while the LGPL-3.0
    explicitly deals with the 
    \enquote{object code incorporating material from (the) library header files}
    (\cite[cf.][\nopage wp.\ §3]{Lgpl30OsiLicense2007a}).}
Additionally, LGPL-2.1 and LGPL-3.0 mainly talk about copying and distributing
the software; the private use is almost complete unspecified.%
  \footnote{The LGPL-2.1 lists its `restrictions' only with respect to the act
    of copying and distributing \enquote{copies of the library} 
    (\cite[cf.][\nopage wp.\ §1, §2, §4 et passim]{Lgpl21OsiLicense1999a}) 
    while the GPL-3.0 explicitly specifies that one \enquote{[\ldots] may make,
    run and propagate covered works that (one does) not convey, without
    conditions so long as (the) license otherwise remains in force} 
    (\cite[cf.][\nopage wp.\ §2]{Gpl30OsiLicense2007a}).}
Finally, the LGPL-2.1 and the LGPL-3.0 aim for the same results and share the
same spirit by requiring nearly the same license fulfilling tasks. Therefore it
seems appropriate to cover both versions in one chapter%
  \footnote{The exception concerns the distribution of a modified program,
    application, or server under the terms of the LGPL} 
and to offer the same LGPL specific open source use case structure%
  \footnote{For details of the general OSUC finder $\rightarrow$ \oslic,
    pp.\ \pageref{OsucTokens} and \pageref{OsucDefinitionTree}} 
for finding the corresponding task lists: 
 
\tikzstyle{nodv} = [font=\scriptsize, ellipse, draw, fill=gray!10, 
    text width=2cm, text centered, minimum height=2em]


\tikzstyle{nods} = [font=\tiny, rectangle, draw, fill=gray!20, 
    text width=1cm, text centered, rounded corners, minimum height=3em]

\tikzstyle{nodb} = [font=\tiny, rectangle, draw, fill=gray!20, 
    text width=1.5cm, text centered, rounded corners, minimum height=3em]
    
\tikzstyle{leaf} = [font=\tiny, rectangle, draw, fill=gray!30, 
    text width=1.4cm, text centered, minimum height=6em]

\tikzstyle{slimleaf} = [font=\tiny, rectangle, draw, fill=gray!30, 
    text width=1.4cm, text centered, minimum height=6em]


\tikzstyle{edge} = [draw, -latex']

\begin{tikzpicture}[]

\node[nodv] (l801) at (3,11.8) {LGPL};

\node[nodv] (l701) at (0,10.2) {2.1};
\node[nodv] (l702) at (4.8,10.2) {3.0};


\node[nodb] (l601) at (0,8.6) {\textit{recipient:} \\ \textbf{4yourself}};
\node[nodb] (l602) at (5.4,8.6) {\textit{recipient:} \\ \textbf{2others}};

\node[nodb] (l501) at (3,7) {\textit{state:} \\ \textbf{unmodified}};
\node[nodb] (l502) at (9,7) {\textit{state:} \\ \textbf{modified}};

\node[nodb] (l401) at (1.8,5.4) {\textit{type:} \\ \textbf{proapse or snimoli}};
\node[nodb] (l402) at (3.7,5.4) {\textit{type:} \\ \textbf{snimoli}};
\node[nodb] (l403) at (6.4,5.4) {\textit{type:} \\ \textbf{proapse}};
\node[nodb] (l404) at (12,5.4) {\textit{type:} \\ \textbf{snimoli}};


\node[nodb] (l301) at (1.8,3.8) {\textit{context:} \\ \textbf{independent}};
\node[nodb] (l302) at (3.7,3.8) {\textit{context:} \\ \textbf{embedded}};
\node[nodb] (l303) at (6.4,3.8) {\textit{context:} \\ \textbf{independent}};
\node[nodb] (l304) at (10,3.8) {\textit{context:} \\ \textbf{independent}};
\node[nodb] (l305) at (13.6,3.8) {\textit{context:} \\ \textbf{embedded}};

\node[nods] (l201) at (1.9,2.2) {\textit{form:} \textbf{source}};
\node[nods] (l204) at (3.7,2.2) {\textit{form:} \textbf{binary}};
\node[nods] (l205) at (5.5,2.2) {\textit{form:} \textbf{source}};
\node[nods] (l206) at (7.3,2.2) {\textit{form:} \textbf{binary}};
\node[nods] (l207) at (9.1,2.2) {\textit{form:} \textbf{source}};
\node[nods] (l208) at (10.9,2.2) {\textit{form:} \textbf{binary}};
\node[nods] (l209) at (12.7,2.2) {\textit{form:} \textbf{source}};
\node[nods] (l210) at (14.5,2.2) {\textit{form:} \textbf{binary}};

\node[slimleaf] (l101) at (0,0) {\textbf{LGPL-*-C1} \newline \textit{using
software only for yourself}};

\node[leaf] (l102) at (1.8,0) { \textbf{LGPL-*-C2} \textit{ distributing
unmodified software as independent or embedded sources}};

\node[leaf] (l105) at (3.6,0) { \textbf{LGPL-*-C3}  \textit{ distributing
unmodified library as independent or embedded binaries}};

\node[slimleaf] (l106) at (5.4,0) { \textbf{LGPL-*-C4}  \textit{ distributing
modified program as sources}};

\node[slimleaf] (l107) at (7.2,0) { \textbf{LGPL-*-C5}  \textit{ distributing
modified program as binaries}};

\node[slimleaf] (l108) at (9.0,0) { \textbf{LGPL-*-C6}  \textit{ distributing
modified library as independent sources}};

\node[slimleaf] (l109) at (10.8,0) { \textbf{LGPL-*-C7} \textit{distributing
modified library as independent binaries}};

\node[leaf] (l110) at (12.6,0) { \textbf{LGPL-*-C8}  \textit{distributing
modified library as embedded sources}};

\node[leaf] (l111) at (14.4,0) { \textbf{LGPL-*-C9}  \textit{ distributing
modified library as embedded binaries}};

\path [edge] (l801) -- (l701);
\path [edge] (l801) -- (l702);
\path [edge] (l701) -- (l601);
\path [edge] (l701) -- (l602);
\path [edge] (l702) -- (l601);
\path [edge] (l702) -- (l602);

\path [edge] (l602) -- (l501);
\path [edge] (l602) -- (l502);

\path [edge] (l501) -- (l401);
\path [edge] (l501) -- (l402);
\path [edge] (l502) -- (l403);
\path [edge] (l502) -- (l404);

\path [edge] (l401) -- (l301);
\path [edge] (l402) -- (l302);
\path [edge] (l403) -- (l303);
\path [edge] (l404) -- (l304);
\path [edge] (l404) -- (l305);

\path [edge] (l301) -- (l201);
\path [edge] (l301) -- (l204);
\path [edge] (l302) -- (l201);
\path [edge] (l302) -- (l204);
\path [edge] (l303) -- (l205);
\path [edge] (l303) -- (l206);
\path [edge] (l304) -- (l207);
\path [edge] (l304) -- (l208);
\path [edge] (l305) -- (l209);
\path [edge] (l305) -- (l210);

\path [edge] (l601) -- (l101);
\path [edge] (l201) -- (l102);

\path [edge] (l204) -- (l105);
\path [edge] (l205) -- (l106);
\path [edge] (l206) -- (l107);
\path [edge] (l207) -- (l108);
\path [edge] (l208) -- (l109);
\path [edge] (l209) -- (l110);
\path [edge] (l210) -- (l111);

\end{tikzpicture}

%% ============================================================================= 
%% Common Building Blocks

% ------------------------------------------------------------------------------
% Common description of license specific use cases

\newcommand{\useCaseOne}{\lsucmeans{that you received LGPL-\ver{} licensed
    software, that you will use it only for yourself, and that you do not hand
    it over to any third party in any sense.}} 

\newcommand{\useCaseTwo}{\lsucmeans{that you received LGPL-\ver{} licensed
    software which you are now going to distribute to third parties in the form
    of unmodified source code files or as unmodified source code package. In
    this case it makes no difference if you distribute a program, an
    application, a server, a snippet, a module, a library, or a plugin as an
    independent or as an embedded unit.}} 
 
\newcommand{\useCaseThree}{\lsucmeans{that you received LGPL-\ver{} licensed
    software which you are now going to distribute to third parties in the form
    of unmodified binary files or as unmodified binary package. In this case it
    does not matter if you distribute a program, an application, a server, a
    snippet, a module, a library, or a plugin as an independent or an embedded
    unit.}} 

\newcommand{\useCaseFour}{\lsucmeans{that you received an LGPL-\ver{} licensed
    program, application, or server (proapse), that you modified it, and that
    you are now going to distribute this modified version to third parties in
    the form of source code files or as a source code package.}}

\newcommand{\useCaseFive}{\lsucmeans{that you received an LGPL-\ver{} licensed
    program, application, or server (proapse), that you modified it, and that
    you are now going to distribute this modified version to third parties in
    the form of binary files or as a binary package.}}

\newcommand{\useCaseSix}{\lsucmeans{that you received an LGPL-\ver{} licensed
    code snippet, module, library, or plugin (snimoli), that you modified it,
    and that you are now going to distribute this modified version to third
    parties in the form of source code files or as a source code package, but
    without embedding it into another larger software unit.}}

\newcommand{\useCaseSeven}{\lsucmeans{that you received an LGPL-\ver{} licensed
    code snippet, module, library, or plugin (snimoli), that you modified it,
    and that you are now going to distribute this modified version to third
    parties in the form of binary files or as a binary package but without
    embedding it into another larger software unit.}} 

\newcommand{\useCaseEight}{\lsucmeans{that you received an LGPL-\ver{} licensed
    code snippet, module, library, or plugin (snimoli), that you modified it,
    and that you are now going to distribute this modified version to third
    parties in the form of source code files or as a source code package
    together with another larger software unit which contains this code snippet,
    module, library, or plugin as an embedded component.}}

\newcommand{\useCaseNine}{\lsucmeans{that you received an LGPL-\ver{} licensed
    code snippet, module, library, or plugin (snimoli), that you modified it,
    and that you are now going to distribute this modified version to third
    parties in the form of binary files or as a binary package together with
    another larger software unit which contains this code snippet, module,
    library, or plugin as an embedded component.}}

% ------------------------------------------------------------------------------
% OSUCs covered

\newcommand{\coversOne}{\coversOsucs{OSUC-01, OSUC-03, OSUC-06, and OSUC-09}{01}{09}}
\newcommand{\coversTwo}{\coversOsucs{OSUC-02S, OSUC-05S, OSUC-07S}{02S}{07S}}
\newcommand{\coversThree}{\coversOsucs{OSUC-02B, OSUC-05B, OSUC-07B}{02B}{02B}}
\newcommand{\coversFour}{\mapsToOsuc{04S}}
\newcommand{\coversFive}{\mapsToOsuc{04B}}
\newcommand{\coversSix}{\mapsToOsuc{08S}}
\newcommand{\coversSeven}{\mapsToOsuc{08B}}
\newcommand{\coversEight}{\mapsToOsuc{10S}}
\newcommand{\coversNine}{\mapsToOsuc{10B}}

% ------------------------------------------------------------------------------
% Common Text Blocks from 0600-common-text-blocks.tex

\newcommand{\keepLicensingElements}{\gtbKeepLicenseElements{LGPL-\ver}}
\newcommand{\giveLicense}{\gtbGiveLicense{LGPL-\ver}}
\newcommand{\addToDocumentation}{\gtbAddToDocumentation{LGPL-\ver}} 
\newcommand{\publishUnmodifiedSource}{\gtbMakeUnmodifiedSourceAvailable{LGPL-\ver}}
\newcommand{\publishModifiedSource}{\gtbMakeModifiedSourceAvailable{LGPL-\ver}}
\newcommand{\publishEmbeddedSource}{\gtbMakeEmbeddedSourcesAvailable{LGPL-\ver}}
\newcommand{\describeHowToGetSource}{gtbDescribeHowToGetSource{LGPL-\ver}}
\newcommand{\createChangelog}{\gtbCreateChangelog{LGPL-\ver}}
\newcommand{\keepAllCopyrightNotices}{\gtbKeepCopyrightNotices{LGPL-\ver}}
\newcommand{\markEmbeddedModifications}{\gtbMarkEmbeddedModifications{LGPL-\ver}}
\newcommand{\markLibraryModifications}{\gtbMarkLibraryModifications{LGPL-\ver}}
\newcommand{\markProgramModifications}{\gtbMarkProgramModifications{LGPL-\ver}}
\newcommand{\lgpltwoEnsureCopyrightNoticeSource}{\gtbVTwoCopyrightNotice{LGPL-2.1}{source code}}
\newcommand{\lgpltwoEnsureCopyrightNoticeBinary}{\gtbVTwoCopyrightNotice{LGPL-2.1}{binary}}
\newcommand{\lgplthreeEnsureCopyrightNoticeSource}{\gtbVThreeCopyrightNotice{LGPL-3.0}{source code}}
\newcommand{\lgplthreeEnsureCopyrightNoticeBinary}{\gtbVThreeCopyrightNotice{LGPL-3.0}{binary}}
\newcommand{\arrangeProgramChanges}{\gtbArrangeProgramChanges{LGPL-\ver}}
\newcommand{\arrangeLibraryChanges}{\gtbArrangeLibraryChanges{LGPL-\ver}}
\newcommand{\arrangeEmbeddedChanges}{\gtbArrangeEmbeddedChanges{LGPL-\ver}}
\newcommand{\howToApplyTheseTerms}{\gtbHowToApplyTheseTerms{LGPL-\ver}}

% ------------------------------------------------------------------------------
% 

% ------------------------------------------------------------------------------
% 


% ------------------------------------------------------------------------------
% 


% ------------------------------------------------------------------------------
% 

%% ============================================================================= 
%% LGPL-2.1 Use Cases

\newcommand{\ver}{2.1}

\begin{license}{LGPL2} % ends at end of file
\licensename{LGPL-\ver}
\licensespec{GNU Lesser General Public License \ver}
%\licenseversion{2.1}
\licenseabbrev{LGPL}

% ------------------------------------------------------------------------------
\subsection{LGPL-\ver-C1: Using the software only for yourself}
\begin{lsuc}{LGPL-\ver-C1}
  \linkosuc{01} 
  \linkosuc{03}
  \linkosuc{06} 
  \linkosuc{09}
  
  \useCaseOne
  \coversOne

  \begin{lsucrequiresnothing}
    \lsucitem{You are allowed to use any kind of LGPL-\ver{} licensed software
      in any sense and in any context without being obliged to do anything as
      long as you do not give the software to third parties.}
  \end{lsucrequiresnothing}

  \lsucprohibitsnothing
\end{lsuc}

% ------------------------------------------------------------------------------
\subsection{LGPL-\ver-C2: Passing the unmodified software as source code}
\begin{lsuc}{LGPL-\ver-C2}
  \linkosuc{02S} 
  \linkosuc{05S} 
  \linkosuc{07S} 

  \useCaseTwo
  \coversTwo

  \begin{lsucrequires}
    \lsucmandatory{\keepLicensingElements}
    \lsucmandatory{\lgpltwoEnsureCopyrightNoticeSource}
    \lsucmandatory{\giveLicense}\passingFilesCorrectly
    \lsucoptional{\addToDocumentation}
    \lsucoptional{\keepAllCopyrightNotices}
  \end{lsucrequires}

  \lsucprohibitsnothing
\end{lsuc}

% ------------------------------------------------------------------------------
\subsection{LGPL-\ver-C3: Passing the unmodified software as binaries}
\begin{lsuc}{LGPL-\ver-C3} 
  \linkosuc{02B} 
  \linkosuc{05B} 
  \linkosuc{07B} 

  \useCaseThree
  \coversThree

  \begin{lsucrequires}
    \lsucmandatory{\keepLicensingElements}
    \lsucmandatory{\lgpltwoEnsureCopyrightNoticeBinary}
    \lsucmandatory{\giveLicense}\passingFilesCorrectly
    \lsucmandatory{\publishUnmodifiedSource}
    \lsucmandatory{\describeHowToGetSource}
    \lsucsourcedist{LGPL-\ver-C2}
    \lsucoptional{\addToDocumentation}
    \lsucoptional{\keepAllCopyrightNotices}
  \end{lsucrequires}

  \lsucprohibitsnothing
\end{lsuc}

% ------------------------------------------------------------------------------
\subsection{LGPL-\ver-C4: Passing a modified program as source code}
\begin{lsuc}{LGPL-\ver-C4}
  \linkosuc{04S} 
  
  \useCaseFour
  \coversFour

% ------------------------------------------------------------------------------
%\subsubsection{under terms of LGPL-2.1}
%
%\begin{description}
%  \item[requires] \ldots [irrelevant] 
%  \item[forbids] to modify the received work in a way that the resulting
%  \enquote{modified work} is no longer a software library (but a
%  program)\footnote{The LGPL-2.1 explictly requires that \enquote{the modified
%  work must itself be a software library} (\cite[cf.][\nopage wp.\
%  §2a]{Lgpl21OsiLicense1999a}). For details $\rightarrow$ \oslic, p.\
%  \pageref{para:libislib}}. \textbf{Hence: \emph{you are not allowed to
%  distribute a modified program under th terms of LGPL-2.1.}}
%\end{description}

  \begin{lsucrequires}
    \lsucmandatory{\keepLicensingElements}
    \lsucmandatory{\lgpltwoEnsureCopyrightNoticeSource}
    \lsucmandatory{\giveLicense}\passingFilesCorrectly
    \lsucmandatory{\markProgramModifications}
    \lsucmandatory{\arrangeProgramChanges}\gtbHowToApplyTheseTerms{GPL-3.0}
    \lsucoptional{\createChangelog}
    \lsucoptional{\addToDocumentation}
    \lsucoptional{\keepLicensingElements}
  \end{lsucrequires}
 
  \lsucprohibitsnothing
\end{lsuc}

% ------------------------------------------------------------------------------
\subsection{LGPL-\ver-C5: Passing a modified program as binary}
\begin{lsuc}{LGPL-\ver-C5}
  \linkosuc{04B} 

  \useCaseFive
  \coversFive

%\subsubsection{under terms of LGPL-2.1}
%
%\begin{description}
%  \item[requires] irrelevant
%  \item[forbids] to modify the received work in a way that the resulting
%  \enquote{modified work} is no longer a software library (but a
%  program)\footnote{The LGPL-2.1 explictly requires that \enquote{the modified
%  work must itself be a software library} (\cite[cf.][\nopage wp.\
%  §2a]{Lgpl21OsiLicense1999a}. For details $\rightarrow$ \oslic, pp.\
%  \pageref{para:libislib}}. \textbf{Hence: \emph{you are not allowed to
%  distribute a modified program under th terms of LGPL-2.1.}}
%\end{description}

  \begin{lsucrequires}
    \lsucmandatory{\keepLicensingElements}
    \lsucmandatory{\lgpltwoEnsureCopyrightNoticeBinary}
    \lsucmandatory{\giveLicense}\passingFilesCorrectly
    \lsucmandatory{\markProgramModifications}
    \lsucmandatory{\arrangeProgramChanges}\gtbHowToApplyTheseTerms{GPL-3.0}
    \lsucmandatory{\publishModifiedSource}
    \lsucmandatory{\describeHowToGetSource}  
    \lsucsourcedist{LGPL-\ver-C4}
    \lsucoptional{\createChangelog}  
    \lsucoptional{\addToDocumentation}
    \lsucoptional{\keepAllCopyrightNotices}
  \end{lsucrequires}

  \lsucprohibitsnothing
\end{lsuc}

% ------------------------------------------------------------------------------
\subsection{LGPL-\ver-C6: Passing a modified library as independent source code}
\begin{lsuc}{LGPL-\ver-C6}
  \linkosuc{08S}

  \useCaseSix
  \coversSix

  \begin{lsucrequires}
    \lsucmandatory{\keepLicensingElements}
    \lsucmandatory{\lgpltwoEnsureCopyrightNoticeSource}
    \lsucmandatory{\giveLicense}\passingFilesCorrectly
    \lsucmandatory{\markLibraryModifications}
    \lsucmandatory{Organize your modifications in a way that they are
      covered by the existing LGPL licensing statements. If you add new source code
      files, insert a header containing your copyright line and a licensing the
      statement in the form required by the GNU project.}%
    \footnote{For details see section `How to Apply These Terms to Your New
      Programs' in the LGPL-v2 resp. (L)GPL-v3 license.}

    \lsucoptional{\createChangelog}  
    \lsucoptional{\addToDocumentation}
    \lsucoptional{\keepAllCopyrightNotices}
  \end{lsucrequires}

  \begin{lsucprohibits}
    \lsucitem{to modify the library in a way that it is no longer a library}
  \end{lsucprohibits}
\end{lsuc}

% ------------------------------------------------------------------------------
\subsection{LGPL-\ver-C7: Passing a modified library as independent binary}
\begin{lsuc}{LGPL-\ver-C7}
  \linkosuc{08B}

  \useCaseSeven
  \coversSeven

  \begin{lsucrequires}
    \lsucmandatory{\keepLicensingElements}
    \lsucmandatory{\lgpltwoEnsureCopyrightNoticeBinary}
    \lsucmandatory{\giveLicense}\passingFilesCorrectly
    \lsucmandatory{\publishModifiedSource}
    \lsucmandatory{\describeHowToGetSource}
    \lsucsourcedist{LGPL-\ver-C6}
    \lsucmandatory{\markLibraryModifications}
  
    \lsucmandatory{Organize your modifications in a way that they are
      covered by the existing LGPL licensing statements.}

    \lsucoptional{\createChangelog}
    \lsucoptional{\addToDocumentation}  
    \lsucoptional{\keepAllCopyrightNotices}  
  \end{lsucrequires}

  \begin{lsucprohibits}
    \lsucitem{to modify the library in a way that it is no longer a library.}
  \end{lsucprohibits}
\end{lsuc}

% ------------------------------------------------------------------------------
\subsection{LGPL-\ver-C8: Passing a modified library as embedded source code}
\begin{lsuc}{LGPL-\ver-C8}
  \linkosuc{10S}

  \useCaseEight
  \coversEight

  \begin{lsucrequires}
    \lsucmandatory{\keepLicensingElements}
    \lsucmandatory{\lgpltwoEnsureCopyrightNoticeSource}
    \lsucmandatory{\giveLicense}\passingFilesCorrectly
    \lsucmandatory{\markEmbeddedModifications}
  
    \lsucmandatory{Organize your modifications of the embedded
      library in a way that they are covered by the existing LGPL licensing
      statements. If you add new source code files to the library, insert a header
      containing your copyright line and a licensing the statement in the form
      required by the GNU project.}%
    \footnote{For details see section `How to Apply These Terms to Your New
      Programs' in the LGPL-2.1 resp. (L)GPL-v3 license.} 

    \lsucmandatory{Maintain the structural independence of the library.}
   
    \lsucmandatory{Let the copyright dialog of the on-top development
      clearly say, that it uses the LGPL licensed library. Let it reproduce the
      content of the existing copyright notices, a hint to the software name, a link
      to its homepage, the respective disclaimer of warranty, and a link to the
      LGPL-2.1 resp. LGPL-3.0.}
     
    \lsucoptional{\createChangelog}  
   \lsucoptional{\addToDocumentation}
    \lsucoptional{\keepAllCopyrightNotices}
  \end{lsucrequires}

  \begin{lsucprohibits}
    \lsucitem{to modify the library in a way that it is no longer a library.}
  \end{lsucprohibits}
\end{lsuc}

% ------------------------------------------------------------------------------
\subsection{LGPL-\ver-C9: Passing a modified library as embedded binary}
\begin{lsuc}{LGPL-\ver-C9}
  \linkosuc{10B}

  \useCaseNine
  \coversNine

  \begin{lsucrequires}
    \lsucmandatory{\keepLicensingElements}
    \lsucmandatory{\lgpltwoEnsureCopyrightNoticeBinary}
    \lsucmandatory{\giveLicense}\passingFilesCorrectly
    \lsucmandatory{\publishEmbeddedSource}
    \lsucmandatory{\describeHowToGetSource}
    \lsucsourcedist{LGPL-\ver-C8}
    \lsucmandatory{\markEmbeddedModifications}
  
    \lsucmandatory{Organize your modifications of the embedded
      library in a way that they are covered by the existing LGPL licensing
      statements.}

    \lsucmandatory{Maintain the structural independence of the library.}
    
    \lsucmandatory{Let the copyright dialog of the on-top development
      clearly say, that it uses the LGPL licensed library. Let it reproduce the
      content of the existing copyright notices, a hint to the software name, a link
      to its homepage, the respective disclaimer of warranty, and a link to the
      LGPL-2.1 resp. LGPL-3.0.}
  
    \lsucmandatory{Either distribute the on-top development and the
      library in the form of dynamically linkable parts or distribute the statically
      linked application together with a written offer, valid for at least three
      years, to give the user all object-files of the on-top development and the
      library, so that he can relink the application himself.}
     
    \lsucoptional{\createChangelog}  
    \lsucoptional{\addToDocumentation}
    \lsucoptional{\keepAllCopyrightNotices}
  \end{lsucrequires}

  \begin{lsucprohibits}
    \lsucitem{to modify the library in a way that it is no longer a library.}
  \end{lsucprohibits}
\end{lsuc}

% ------------------------------------------------------------------------------
\end{license}

%% ============================================================================= 
%% LGPL-3.0 Use Cases

%% =============================================================================
%% Discussion

\subsection{Discussions and Explanations}

\newcommand{\lgplTwoAndGplThree}[2]{\footnote{%
    For LGPL-2.1 see \cite[cf.][\nopage wp.\ #1]{Lgpl21OsiLicense1999a}.
    \par\noindent
    For GPL-3.0, which is included in the LGPL-3.0, see \cite[cf.][\nopage
      wp.\ #2]{Gpl30OsiLicense2007a}.}} 

\newcommand{\lgplTwoAndThree}[2]{\footnote{%
    For LGPL-2.1 see \cite[cf.][\nopage wp.\ #1]{Lgpl21OsiLicense1999a}.
    \par\noindent
    For LGPL-3.0 see \cite[cf.][\nopage wp.\ #2]{Lgpl30OsiLicense2007a}.}}

\begin{itemize}
  
\item The LGPL-2.1 allows to \enquote{[\ldots] copy and (to) distribute
  verbatim copies of the Library's complete source code as you receive it [...]
  provided that you 
  [a] conspicuously and appropriately publish on each copy an appropriate
  copyright notice and disclaimer of warranty; 
  [b] keep intact all the notices that refer to this License and to the absence
  of any warranty; and 
  [c] distribute a copy of this License along with the Library.}\citeLGPLtwo{§1}
  Additionally, the LGPL-2.1 allows the distribution of the modified source code
  \enquote{under the terms of Section~1}\citeLGPLtwo{§2} and the distribution
  of binaries \enquote{under the terms of Sections~1 and~2}.\citeLGPLtwo{§4}
  But the LGPL does not require any tasks if you are using the work only for
  yourself. Thus, the quoted conditions of \enquote{Section 1} are mandatory for
  all use cases concerning the distribution of an LGPL licensed work
  (LGPL-2.1-C2 -- LGPL-2.1-C9).
  \footnote{The GPL-3.0, which is included into the LGPL-3.0, uses a similar
    structure to establish the same requirements ($\rightarrow$ \oslic, p.\ 
    \pageref{Gpl3ConditionsDistri}). Based on this fact, one may conclude that
    the tasks which fulfill the corresponding LGPL-2.1 requirements together
    also fit the GPL-3.0 conditions and hence those of the LGPL-3.0.}

\item Although both versions of the LGPL do not explicitly require to retain the
  copyright notices in the form you have received them, it is nevertheless a
  very good idea not to modify these elements (LGPL-*-C2 - LGPL-*-C9).
  
\item The LGPL-2.1 allows to \enquote{[\ldots] copy and (to) distribute the
  Library (or a portion or derivative of it [\ldots]) in object code or
  executable form [\ldots] provided that you accompany it with the complete
  corresponding machine-readable source code [\ldots] on a medium customarily
  used for software interchange.} And the license further states that, if one
  makes the object code accessible without distributing it directly, then the
  same `download' method for the source code fulfills this
  condition.\citeLGPLtwo{§4}  So, no doubt: Taken literally, the LGPL requires
  you to distribute the source code and the object code together and by the same
  method: either both on (for example) DVD or both offered for download; but not
  the one on DVD and the other by a download from a repository. But the first
  specification also says, that the \enquote{complete corresponding machine
  readable source code} has to be distributed \enquote{on a medium customarily
  used for software interchange.}\citeLGPLtwo{§4}  The \oslic{} considers the
  possibility to download files from the Internet as a distribution \emph{on a
  medium [today] customarily used for software interchange.} Therefore, the
  \oslic{} requires for all open source use cases that refer to the distribution
  of binaries (LGPL-2.1-C3, LGPL-2.1-C5, LGPL-2.1-C7, LGPL-2.1-C9) to make the
  source code of the corresponding library accessible via an Internet
  repository%
  \footnote{In contrast to the LGPL-2.1, the GPL-3.0, which is included in the
    LGPL-3.0, explictily offers the option to distribute the sources via an
    Internet server ($\rightarrow$ \oslic, p.\ \pageref{Gpl3CondCopyleft}). So,
    one may again conclude that the tasks that fulfill the corresponding
    LGPL-2.1 requirements together also fit the GPL-3.0 and the LGPL-3.0
    conditions.} 
  
\item The LGPL allows to \enquote{[\ldots] modify your copy or copies of the
  Library or any portion of it [\ldots] and (to) copy and distribute such
  modifications [\ldots]} only under some restrictions and
  condtions''\citeLGPLtwo{§2} 
  \begin{itemize}
  \item First, modified files must be marked as modifications and this must
    include the date of the modification.\lgplTwoAndGplThree{§2}{§5} This
    condition must be respected by all open source use cases concerning the
    distribution of the modified work [LGPL-*-C4 - LGPL-*-C9], because even if
    one primarily intends to distribute binaries, one has also to deliver the
    source code. The \oslic{} `replaces' this requirement by the mandatory
    condition to mark each modified file and by the voluntary condition to
    update or create a general changelog file.
    
  \item Second, the license requires that the modified version does not depend
    on external data structures without \enquote{[\ldots] (making) a good faith
    effort to ensure that, in the event an application does not supply such (a)
    function or table, the facility still operates, and performs whatever part
    of its purpose remains meaningful.}\lgplTwoAndThree{§2d}{§2a} The \oslic{}
    rewrites this condition as the obligation to maintain the structural
    independence of the library in case of using the modified library as
    embedded component [LGPL-*-C8 - LGPL-*-C9]. 
    
  \item \label{para:libislib}Third, the LGPL-2.1 definitely requires, that
    \enquote{the modified work must itself be a software
    library.}\citeLGPLtwo{§2}  This conditions can directly be incorprated as an
    interdiction into all use cases which refer to the modification of a library
    [LGPL-2.1-C6 - LGPL-2.1-C9]. But it is difficult to respect this condition
    if one wants to modify a program which one has received under the terms of
    the LGPL-2.1. In principal, one can write an application and license it
    under the LGPL-2.1. But, as a consequence, that impedes the modification of
    this work because the result must be a library. 
    The LGPL-3.0 does not contain any such requirement. Hence, the \oslic{} allows
    the distribution of modified programs (LGPL-*-C4, LGPL-*-C5) only if they
    are licensed under the terms of LGPL-3.0. For programs licensed under
    LGPL-2.1, the only option is to relicense the software under the terms of
    the regular GPL-2.0 (or, at your discretion, GPL-3.0).  This is explicitely
    allowed by the LGPL-2.1: \enquote{You may opt to apply the terms of the
    ordinary GNU General Public License instead of this License to a given
    copy of the Library.}\citeLGPLtwo{§3} 
  \end{itemize}
  
\item Additionally, the LGPL-2.1 allows the licensee to distribute a
  program\footnote{or another library} developed on-top of the library---what
  the LGPL-2.1 calls a \enquote{work that uses the libary}\citeLGPLtwo{§5,
  §6}---\enquote{as an exception to the Sections above} in \enquote{combination}
  with the library \enquote{under terms of your choice,}\citeLGPLtwo{§6},
  provided that the licensee fulfills additional conditions:  
  First, it must clearly be stated that the on-top development depends on the
  (modified) library. Second, the LGPL must be added into the distributed
  package. Third, in its own copyright dialog, the on-top development must
  mention the library, its copyright holder, and that it is licensed under the
  LGPL.\citeLGPLtwo{§6} In the LGPL-3.0, this condition is similarily
  integrated: On the one hand, the \enquote{combined work} is defined as
  \enquote{a work produced by combining or linking an Application with the
  Library}.\citeLGPLthree{§0} On the other hand, the LGPL-3.0 states that one
  \enquote{[\ldots] may convey a Combined Work under terms of (his own) choice}
  provided that one [a] clearly says that the on-top development uses the LGPL
  licensed library, [b] distributes the LGPL-3.0 and the GPL-3.0 license as part
  of the package, [c] displays all these (licensing) information by the existing
  display technologies, [d] requires an appropriate shared library mechanism,
  and [e] offers the respective installion information.\citeLGPLthree{§4} These
  requirements can directly be inserted as conditions into the respective use
  cases for both LGPL versions (LGPL-*-C8, LGPL-*-C9).
  
\item The most difficult requirements of the LGPL-2.1 concern the distribution
  in the form of binaries. In a very strict reading, the LGPL does not require
  to link the on-top development and the libary only dynamically. At first, the
  LGPL mentions, that the \enquote{[..] work (that uses the Library), in
  isolation, is not a derivative work of the Library [\ldots]}. But if it is
  linked to the library the resulting executable program (of course) becomes
  \enquote{a derivative of the Library} and that it is therefore
  \enquote{[\ldots] covered by this License (LGPL-2.1)}. But the LGPL-2.1 
  directly continues this statement with the hint, that \enquote{Section 6
  states terms for distribution of such executables.}\citeLGPLtwo{§5} Finally,
  section 6 directly starts with the statement: \enquote{As an exception to the
  Sections above, you may also combine or link a ``work that uses the
  Library'' with the Library to produce a work containing portions of the
  Library, and distribute that work under terms of your choice}.\citeLGPLtwo{§6}
  
  This is important to know, because until this section 6 one can not directly
  read or indirectly infer that the LGPL-2.1 distinguished the act of
  dynamically linking a program and a library from that of statically linking
  these parts. The LGPL only wants to ensure that the binaries of the library
  itself can be replaced by a newer version. And that is required by
  section~6.\citeLGPLtwo{§6} 
  From a practical point of view, this can only be guaranteed, if the binaries of
  the on-top development and the library are linked using a \enquote{suitable
  shared library mechanism}\citeLGPLtwo{§6} or if one also gets all compiled,
  but not linked object-files of the on-top development and the library, either
  directly, or via using a \enquote{a written offer, valid for at least three
  years, to give the same user the (respective) materials}.\citeLGPLtwo{§6}  In
  the first case, the user can replace the received version of the library and
  can let the application be relinked  automatically. In the second case, he has
  to do it manually. It is important to know that both these ways exist if one
  wants or must distribute statically linked works. The LGPL-2.1 does not forbid
  to distribute statically linked applications. But it requires to enable the
  receiver to relink the work. 
  
  The LGPL-3.0 has reduced these complex conditions in a special way: First, it
  does not use the words `statically linked' or `dynamically' linked at
  all. Second it defines the combined work `only' as the result of
  \enquote{combining or linking an Application with the
  Library}.\citeLGPLthree{§0}  But then it requires for the distribution of the
  combined works that one has either to \enquote{convey the Minimal
  Corresponding Source under the terms of this License, and the Corresponding
  Application Code in a form suitable for, and under terms that permit, the user
  to recombine or relink the Application with a modified version of the Linked
  Version to produce a modified Combined Work [\ldots]} or that one must
  presuppose that the receiver uses \enquote{[\ldots] suitable shared library
  mechanism for linking with the Library [\ldots] that [\ldots] operate properly
  with a modified version of the Library [\ldots]}\citeLGPLthree{§4} Finally,
  the LGPL-3.0 adds that in the first case the these materials which enables the
  relinking must be distributed \enquote{[\ldots] in the manner specified by
  section~6 of the GNU GPL[-3.0] for conveying Corresponding
  Source.}\citeLGPLthree{§4}  And this section~6 of the GPL-3.0 allows the well
  known method to \enquote{convey the object code [\ldots] accompanied by a
  written offer [\ldots] to give anyone [\ldots] access to copy the
  Corresponding Source from a network server at no charge}.\citeGPLthree{§6}

  Therefore, the \oslic{} can condense these conditions into the requirement,
  either to distribute dynamically linkable parts, or to distribute statically
  linked applications \enquote{(accompanied) [\ldots] with a written offer,
  valid for at least three years, to give the same user the [complete]
  materials,}\citeLGPLtwo{§6} so that he can relink the application. It is
  clear, that this condition is only valid for the use case LGPL-*-C9.
  
\end{itemize}

%\bibliography{../../../bibfiles/oscResourcesEn}

% Local Variables:
% mode: latex
% fill-column: 80
% End:
