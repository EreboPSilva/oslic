% Telekom osCompendium 'for being included' snippet template
%
% (c) Karsten Reincke, Deutsche Telekom AG, Darmstadt 2011
%
% This LaTeX-File is licensed under the Creative Commons Attribution-ShareAlike
% 3.0 Germany License (http://creativecommons.org/licenses/by-sa/3.0/de/): Feel
% free 'to share (to copy, distribute and transmit)' or 'to remix (to adapt)'
% it, if you '... distribute the resulting work under the same or similar
% license to this one' and if you respect how 'you must attribute the work in
% the manner specified by the author ...':
%
% In an internet based reuse please link the reused parts to www.telekom.com and
% mention the original authors and Deutsche Telekom AG in a suitable manner. In
% a paper-like reuse please insert a short hint to www.telekom.com and to the
% original authors and Deutsche Telekom AG into your preface. For normal
% quotations please use the scientific standard to cite.
%
% [ Framework derived from 'mind your Scholar Research Framework' 
%   mycsrf (c) K. Reincke 2012 CC BY 3.0  http://mycsrf.fodina.de/ ]
%


%% use all entries of the bibliography
%\nocite{*}

\section{BSD licensed software}

As an approved open source license, the BSD license exists in two
versions\footcite[Following the OSI, there is another 'ancient' 
BSD license -- containing a fourth clause known as advertising clause -- which
\enquote{(\ldots) officially was rescinded by the Director of the Office of
Technology Licensing of the University of California on July 22nd, 1999}.
 Cf.][\nopage wp. Because of that cancellation you can simply act according the
 \textit{BSD 3-Clause license} if you have to fulfill the eldest BSD
 license]{BsdLicense3Clause}. The latest release is the \textit{BSD 2-Clause
 license}\footcite[cf.][\nopage wp]{BsdLicense2Clause}, the elder release is the
 \textit{BSD 3-Clause license}\footcite[cf.][\nopage wp]{BsdLicense3Clause}.
 The very little differences between the two versions have to be respected
 exactly. Nevertheless, we could integrate the requirements into one to-do list
 per use case.

Explicitly, all BSD open source licenses 'only' focus on the (re-)distribution
\textit{open source use cases} which we have specified by our token
\textit{4others}. Conditions for the other use cases specified by the token
\textit{4yourself} can be derived\footnote{For details of the \textit{open
source use case tokens} see p.\ \pageref{OsucTokens}. For details of the
\textit{open source use cases} based on these token see p.
\pageref{OsucDefinitionTree} }. Additionally the BSD licenses consider the form
of the distribution, esp.\ whether the work is distributed as a (set of) source
code file(s) or as a (set of) the binary file(s). Use the following tree to find
the BSD license fulfilling to-do lists.

\tikzstyle{nodv} = [font=\small, ellipse, draw, fill=gray!10, 
    text width=2cm, text centered, minimum height=2em]


\tikzstyle{nods} = [font=\footnotesize, rectangle, draw, fill=gray!20, 
    text width=1.2cm, text centered, rounded corners, minimum height=3em]

\tikzstyle{nodb} = [font=\footnotesize, rectangle, draw, fill=gray!20, 
    text width=2.2cm, text centered, rounded corners, minimum height=3em]
    
\tikzstyle{leaf} = [font=\tiny, rectangle, draw, fill=gray!30, 
    text width=1.2cm, text centered, minimum height=6em]

\tikzstyle{edge} = [draw, -latex']

\begin{tikzpicture}[]

\node[nodv] (l81) at (4,11.8) {BSD};

\node[nodv] (l71) at (0,10.2) {3-Clause License};
\node[nodv] (l72) at (6.5,10.2) {2-Clause License};


\node[nodb] (l61) at (0,8.6) {\textit{recipient:} \\ \textbf{4yourself}};
\node[nodb] (l62) at (6.5,8.6) {\textit{recipient:} \\ \textbf{4others}};

\node[nodb] (l51) at (2.5,7) {\textit{state:} \\ \textbf{unmodified}};
\node[nodb] (l52) at (9.3,7) {\textit{state:} \\ \textbf{modified}};

\node[nods] (l41) at (1.8,5.4) {\textit{form:} \textbf{source}};
\node[nods] (l42) at (3.6,5.4) {\textit{form:} \textbf{binary}};
\node[nodb] (l43) at (6.5,5.4) {\textit{type:} \\ \textbf{proapse}};
\node[nodb] (l44) at (12,5.4) {\textit{type:} \\ \textbf{snimoli}};


\node[nods] (l31) at (5.4,3.8) {\textit{form:} \textbf{source}};
\node[nods] (l32) at (7.2,3.8) {\textit{form:} \textbf{binary}};
\node[nodb] (l33) at (10,3.8) {\textit{context:} \\ \textbf{independent}};
\node[nodb] (l34) at (13.5,3.8) {\textit{context:} \\ \textbf{embedded}};

\node[nods] (l21) at (9,2.2) {\textit{form:} \textbf{source}};
\node[nods] (l22) at (10.8,2.2) {\textit{form:} \textbf{binary}};
\node[nods] (l23) at (12.6,2.2) {\textit{form:} \textbf{source}};
\node[nods] (l24) at (14.4,2.2) {\textit{form:} \textbf{binary}};

\node[leaf] (l11) at (0,0) {\textbf{BSD-1} \textit{using software only
for yourself}};

\node[leaf] (l12) at (1.8,0) { \textbf{BSD-2} \textit{ distributing unmodified
software as sources}};

\node[leaf] (l13) at (3.6,0) { \textbf{BSD-3}  \textit{ distributing unmodified
software as binaries}};

\node[leaf] (l14) at (5.4,0) { \textbf{BSD-4}  \textit{ distributing modified
program as sources}};

\node[leaf] (l15) at (7.2,0) { \textbf{BSD-5}  \textit{ distributing modified
program as binaries}};

\node[leaf] (l16) at (9,0) { \textbf{BSD-6}  \textit{ distributing modified
library as independent sources}};

\node[leaf] (l17) at (10.8,0) { \textbf{BSD-7} \textit{distributing modified
library as independent binaries}};

\node[leaf] (l18) at (12.6,0) { \textbf{BSD-8}  \textit{distributing
modified library as embedded sources}};

\node[leaf] (l19) at (14.4,0) { \textbf{BSD-9}  \textit{ distributing modified
library as embedded binaries}};

\path [edge] (l81) -- (l71);
\path [edge] (l81) -- (l72);
\path [edge] (l71) -- (l61);
\path [edge] (l71) -- (l62);
\path [edge] (l72) -- (l61);
\path [edge] (l72) -- (l62);
\path [edge] (l61) -- (l11);
\path [edge] (l62) -- (l51);
\path [edge] (l62) -- (l52);
\path [edge] (l51) -- (l41);
\path [edge] (l51) -- (l42);
\path [edge] (l52) -- (l43);
\path [edge] (l52) -- (l44);
\path [edge] (l41) -- (l12);
\path [edge] (l42) -- (l13);
\path [edge] (l43) -- (l31);
\path [edge] (l43) -- (l32);
\path [edge] (l44) -- (l33);
\path [edge] (l44) -- (l34);
\path [edge] (l31) -- (l14);
\path [edge] (l32) -- (l15);
\path [edge] (l33) -- (l21);
\path [edge] (l33) -- (l22);
\path [edge] (l34) -- (l23);
\path [edge] (l34) -- (l24);
\path [edge] (l21) -- (l16);
\path [edge] (l22) -- (l17);
\path [edge] (l23) -- (l18);
\path [edge] (l24) -- (l19);

\end{tikzpicture}

\subsection{BSD-1: Using the software only for yourself}
\label{OSUC-01-BSD} 
\label{OSUC-03-BSD} 
\label{OSUC-06-BSD}
\label{OSUC-09-BSD}
  
\begin{description}
\item[means] that you are going to use a received BSD software only for yourself
and that you do not hand it over to any 3rd party in any sense.
\item[covers] OSUC-01, OSUC-03, OSUC-06, and OSUC-09\footnote{For details see pp.
  \pageref{OSUC-01-DEF} - \pageref{OSUC-09-DEF}}
\item[requires] no tasks in order to fulfill the conditions of the BSD license
with respect to this use case:
  \begin{itemize}
    \item You are allowed to use any kind of BSD software in any sense and in
    any context without any obligations as long as you do not give the software
    to 3rd parties.
  \end{itemize}
\item[prohibits] nothing explicitly.
\end{description}


\subsection{BSD-2: Passing the unmodified software as source code}
\label{OSUC-02-BSD} \label{OSUC-05-BSD} \label{OSUC-07-BSD} 

\begin{description}
\item[means] that you are going to distribute an unmodified version of the
received BSD software to 3rd parties in the form of a set of source code files or an
integrated source code package\footnote{In this case it doesn't matter whether
you  distribute a program, an application, a server, a snippet, a module, a
library, or a plugin as an independent or an embedded unit}

\item[covers] OSUC-02, OSUC-05, OSUC-07\footnote{For details see pp.\ 
\pageref{OSUC-02-DEF} - \pageref{OSUC-07-DEF}}

\item[requires] the following tasks in order to fulfill the license conditions:
\begin{itemize}
  \item \textbf{[mandatorily:]} Ensure that the licensing elements -- esp.\ the
  BSD license text, the specific copyright notice of the original author(s), and
  the BSD disclaimer -- are retained in your package in the form you have
  received them.
  \item \textbf{[voluntarily:]} Let the documentation of your distribution
  and/or your additional material also contain the original copyright notice, the
  BSD conditions, and the BSD disclaimer.
\end{itemize}

\item[prohibits] nothing explicitly if you are using the \emph{BSD 2 Clause
License}. But the \emph{BSD 3 Clause License} explicitly prohibits to use the
name of the licensing organization or the names of the licensing distributors to
promote your own work.

\end{description}

\subsection{BSD-3: Passing the unmodified software as binary}

\begin{description}
\item[means] that you are going to distribute an unmodified version of the BSD
received software to 3rd parties in the form of a set of binary files or an
integrated bi\-na\-ry package\footnote{In this case it doesn't matter whether
you distribute a program, an application, a server, a snippet, a module, a
library, or a plugin as an independent or an embedded unit}
\item[covers] OSUC-02, OSUC-05, OSUC-07\footnote{For details see pp.
\pageref{OSUC-02-DEF} - \pageref{OSUC-07-DEF}}
\item[requires] the following tasks in order to fulfill the license conditions:
\begin{itemize}
  
  \item  \textbf{[mandatory:]} Ensure that your distribution contains the
  original copyright notice, the BSD license, and the BSD disclaimer in the form
  you have received them. If you compile the binary file on the base of the
  source code package and if this compilation does not also generate and
  integrate the licensing files then create the copyright notice, the BSD
  conditions, and the BSD disclaimer according to the form of the source code
  package and insert these files into your distribution manually\footnote{For
  implementing the handover of files correctly $\rightarrow$ OSLiC
  \pageref{DistributingFilesHint}}.
  
  \item  \textbf{[mandatory:]} Ensure that the documentation of your
  distribution and/or your additional material also contain the author specific
  copyright notice, the BSD conditions, and the BSD disclaimer.
\end{itemize}

\item[prohibits] nothing explicitly if you are using the \emph{BSD 2 Clause
License}. But the \emph{BSD 3 Clause License} explicitly prohibits to use the
name of the licensing organization or the names of the licensing distributors to
promote your own work.

\end{description}

\subsection{BSD-4: Passing a modified program as source code}
\label{OSUC-04-BSD}

\begin{description}
\item[means] that you are going to distribute a modified version of the received
BSD program, application, or server (proapse) to 3rd parties in the form of a set
of source code files or an integrated source code package.
\item[covers] OSUC-04\footnote{For details see pp.\ \pageref{OSUC-04-DEF}}
\item[requires] the following tasks in order to fulfill the license conditions:
\begin{itemize}
  \item \textbf{[mandatory:]} Ensure that the licensing elements -- esp.\
  the BSD license text, the specific copyright notice of the original author(s),
  and the BSD disclaimer -- are retained in your package in the form you have
  received them.
  
  \item \textbf{[voluntary:]} Let the documentation of your distribution and/or
  your additional material also contain the original copyright notice, the BSD
  conditions, and the BSD disclaimer.
  
  \item \textbf{[voluntary:]} It is a good practice of the open source
  community, to let the copyright notice which is shown by the running program
  also state that the program is licensed under the BSD license. Because you are
  already modifying the program you can also add such a hint if the presented
  original copyright notice lacks such a statement.
\end{itemize}

\item[prohibits] nothing explicitly if you are using the \emph{BSD 2 Clause
License}. But the \emph{BSD 3 Clause License} explicitly prohibits to use the
name of the licensing organization or the names of the licensing distributors to
promote your own work.

\end{description}

\subsection{BSD-5: Passing a modified program as binary}

\begin{description}
\item[means] that you are going to distribute a modified version of the received
BSD pro\-gram, application, or server (proapse) to 3rd parties in the form of a set
of binary files or an integrated binary package.
\item[covers] OSUC-04\footnote{For details see pp.\ \pageref{OSUC-04-DEF}}
\item[requires] the following tasks in order to fulfill the license conditions:
\begin{itemize}

  \item  \textbf{[mandatory:]} Ensure that your distribution contains the
  original copyright notice, the BSD license, and the BSD disclaimer in the form
  you have received them. If you compile the binary file on the base of the
  source code package and if this compilation does not also generate and
  integrate the licensing files then create the copyright notice, the BSD
  conditions, and the BSD disclaimer according to the form of the source code
  package and insert these files into your distribution manually\footnote{For
  implementing the
  handover of files correctly $\rightarrow$ OSLiC
  \pageref{DistributingFilesHint}}.

  \item  \textbf{[mandatory:]} Ensure that the documentation of your
  distribution and/or your additional material also contain the author specific
  copyright notice, the BSD conditions, and the BSD disclaimer.
  
  \item \textbf{[voluntary:]} It is a good practice of the open source
  community, to let the copyright notice which is shown by the running program
  also state that the program is licensed under the BSD license. Because you are
  already modifying the program you can also add such a hint if the presented
  original copyright notice lacks such a statement.
\end{itemize}

\item[prohibits] nothing explicitly if you are using the \emph{BSD 2 Clause
License}. But the \emph{BSD 3 Clause License} explicitly prohibits to use the
name of the licensing organization or the names of the licensing distributors to
promote your own work.

\end{description}

\subsection{BSD-6: Passing a modified library as independent source code}
\label{OSUC-08-BSD}
\begin{description}
\item[means] that you are going to distribute a modified version of the received
BSD code snippet, module, library, or plugin (snimoli) to 3rd parties in form
of a set of source code files or an integrated source code package, but without
embedding it into another larger software unit.
\item[covers] OSUC-08\footnote{For details see pp.\ \pageref{OSUC-08-DEF}}
\item[requires] the following tasks in order to fulfill the license conditions:
\begin{itemize}
  \item \textbf{[mandatory:]} Ensure that the licensing elements -- esp.\ the
  BSD license text, the specific copyright notice of the original author(s), and
  the BSD disclaimer -- are retained in your package in the form you have
  received them.
  \item \textbf{[voluntary:]} Let the documentation of your distribution
  and/or your additional material also contain the original copyright notice, the
  BSD conditions, and the BSD disclaimer.
\end{itemize}

\item[prohibits] nothing explicitly if you are using the \emph{BSD 2 Clause
License}. But the \emph{BSD 3 Clause License} explicitly prohibits to use the
name of the licensing organization or the names of the licensing distributors to
promote your own work.

\end{description}


\subsection{BSD-7: Passing a modified library as independent binary}

\begin{description}
\item[means] that you are going to distribute a modified version of the received
BSD code snippet, module, library, or plugin (snimoli) to 3rd parties in form
of a set of binary files or an integrated binary package but without embedding
it into another larger software unit.
\item[covers] OSUC-08\footnote{For details see pp.\ \pageref{OSUC-08-DEF}}
\item[requires] the following tasks in order to fulfill the license conditions:
\begin{itemize}
   \item  \textbf{[mandatory:]} Ensure that your distribution contains the
  original copyright notice, the BSD license, and the BSD disclaimer in the form
  you have received them. If you compile the binary file on the base of the
  source code package and if this compilation does not also generate and
  integrate the licensing files, then create the copyright notice, the BSD
  conditions, and the BSD disclaimer according to the form of the source code
  package and insert these files into your distribution manually\footnote{For
  implementing the handover of files correctly $\rightarrow$ OSLiC
  \pageref{DistributingFilesHint}}.
  \item  \textbf{[mandatory:]} Ensure that the documentation of your
  distribution and/or your additional material also contain the author specific
  copyright notice, the BSD conditions, and the BSD disclaimer.
\end{itemize}

\item[prohibits] nothing explicitly if you are using the \emph{BSD 2 Clause
License}. But the \emph{BSD 3 Clause License} explicitly prohibits to use the
name of the licensing organization or the names of the licensing distributors to
promote your own work.

\end{description}

\subsection{BSD-8: Passing a modified library as embedded source code}
\label{OSUC-10-BSD}
\begin{description}
\item[means] that you are going to distribute a modified version of the received
BSD code snippet, module, library, or plugin (snimoli) to 3rd parties in form
of a set of source code files or an integrated source code package together with
another larger software unit which contains this code snippet, module, library,
or plugin as an embedded component.
\item[covers] OSUC-10\footnote{For details see pp.\ \pageref{OSUC-10-DEF}}
\item[requires] the following tasks in order to fulfill the license conditions:
\begin{itemize}
  \item \textbf{[mandatory:]} Ensure that the licensing elements -- esp.\ the
  BSD license text, the specific copyright notice of the original author(s), and
  the BSD disclaimer -- are retained in your package in the form you have
  received them.
  \item \textbf{[voluntary:]} Let the documentation of your distribution
  and/or your additional material also contain the original copyright notice, the
  BSD conditions, and the BSD disclaimer.
 \item \textbf{[voluntary:]} It is a good practice of the open source
  community, to let the copyright notice which is shown by the running program
  also state that it contains components licensed under the BSD license. Because
  you are embedding this snimoli into a larger software unit, you are
  developing this larger unit. Hence, you can also expand the copyright notice
  of this larger unit by such a hint to its BSD components.
  
  \item \textbf{[voluntary:]} Arrange your source code distribution so that the
  the licensing elements -- esp.\ the BSD license text, the specific copyright
  notice of the original author(s), and the BSD disclaimer -- clearly refer
  only to the embedded library and do not disturb the licensing of your own
  overarching work. It's a good tradition to keep the embedded components like
  libraries, modules, snippets, or plugins in specific directory which contains
  also all additional licensing elements.
  
\end{itemize}

\item[prohibits] nothing explicitly if you are using the \emph{BSD 2 Clause
License}. But the \emph{BSD 3 Clause License} explicitly prohibits to use the
name of the licensing organization or the names of the licensing distributors to
promote your own work.

\end{description}


\subsection{BSD-9: Passing a modified library as embedded binary}

\begin{description}
\item[means] that you are going to distribute a modified version of the received
BSD code snippet, module, library, or plugin to 3rd parties in the form of a set of
binary files or an integrated binary package together with another larger
software unit which contains this code snippet, module, library, or plugin as
an embedded component.
\item[covers] OSUC-10\footnote{For details see pp.\ \pageref{OSUC-10-DEF}}
\item[requires] the following tasks in order to fulfill the license conditions:
\begin{itemize}
  \item  \textbf{[mandatory:]} Ensure that your distribution contains the
  original copyright notice, the BSD license, and the BSD disclaimer in the form
  you have received them. If you compile the binary file on the base of the
  source code package and if this compilation does not also generate and
  integrate the licensing files, then create the copyright notice, the BSD
  conditions, and the BSD disclaimer according to the form of the source code
  package and insert these files into your distribution manually\footnote{For
  implementing the handover of files correctly $\rightarrow$ OSLiC
  \pageref{DistributingFilesHint}}.
  \item  \textbf{[mandatory:]} Ensure that the documentation of your
  distribution and/or your additional material also contain the author specific
  copyright notice, the BSD conditions, and the BSD disclaimer.
 \item \textbf{[voluntary:]} It is a good practice of the open source
  community, to let the copyright notice which is shown by the running program
  also state that it contains components licensed under the BSD license. Because
  you are embedding this snimoli into a larger software unit, you are
  developing this larger unit. Hence, you can also expand the copyright notice
  of this larger unit by such a hint to its BSD components.
  
  \item \textbf{[voluntary:]} Arrange your binary distribution so that the
  licensing elements -- esp.\ the BSD license text, the specific copyright
  notice of the original author(s), and the BSD disclaimer -- clearly refer
  only to the embedded library and do not disturb the licensing of your own
  overarching work. It's a good tradition to keep the librabries, modules,
  snippet, or plugins in specific directiers which contain also all licensing
  elements.
\end{itemize}

\item[prohibits] nothing explicitly if you are using the \emph{BSD 2 Clause
License}. But the \emph{BSD 3 Clause License} explicitly prohibits to use the
name of the licensing organization or the names of the licensing distributors to
promote your own work.

\end{description}

\subsection{Discussions and Explanations}

The \textit{BSD 2-Clause license} has a simple structure: In the
beginning, it generally \enquote{(permits) [the] redistribution and [the] use in
source and binary forms, with or without modification, [\ldots]}, if one
fulfills the two rules of the license\footcite[cf.][\nopage
wp]{BsdLicense2Clause}. The first rule concerns the (re)distribution in the form of
source code, the second the (re)distribution of binary packages. Here are some
explanations why we translated the rules into which sets of executable tasks:

\begin{itemize}
\item For the \enquote{redistribution of source code} the license requires,
that the package must \enquote{ [\ldots] retain the above copyright notice, this
list of conditions and the following disclaimer}\footcite[cf.][\nopage
wp]{BsdLicense2Clause}. Hence, you are not allowed, to modify any of the
copyright notes which are already embedded in the received (source) files. And
from a logical point of view, there must exist an explicit or implicit
assertion that the software is licensed under the \textit{BSD 2-Clause
license}\footcite[The BSD license requires that a re-distributed software
package must contain the (package specific) copyright notice, the (license
specific) conditions and the BSD disclaimer. (cf.][\nopage wp.) You might ask
what you should do, if these elements are missed in the package you received. If so,
the package you received had not been licensed adequately. Hence, you do not know
reliably whether you have received it under a BSD license. In other words: If
you have received a BSD licensed software package, it must contain sufficient
license fulfilling elements, or it is not a BSD licensed
software]{BsdLicense2Clause}. This is often implemented by simply adding a copy
of the license into the package. Hence, you are furthermore not allowed to
modify these files or corresponding text snippets. For our purposes, we
translated the bans into the following executable task:

\begin{quote}\textit{Ensure that the licensing elements -- esp.\ the BSD license
text, the specific copyright notice of the original author(s), and the BSD
disclaimer -- are retained in your package in the form you have received
them.}\end{quote}

\item For the redistribution in the form of binary files, the license requires, that
the licensing elements must be \enquote{[\ldots] (reproduced) in the
documentation and/or other materials provided with the
distribution}\footcite[cf.][\nopage wp]{BsdLicense2Clause}. Hence, this is not
required as a necessary condition for the (re)distribution as source code
package. But nevertheless, even for a distribution in the form of source code, it is
often possible to fulfill this rule too -- e.g.\ if you offer your own download
site for source code packages. In such cases, it is a sign of respect to
mention the licensing not only inside of the packages, but also in the text of
your site. Because of that, we added the following voluntary task for all BSD
open source use cases which deal with the redistribution in the form of source code:

\begin{quote}\textit{Let the documentation of your distribution and/or your
additional material also contain the original copyright notice, the BSD
conditions, and the BSD disclaimer.}\end{quote}

\item Naturally, because the reproduction of the licensing elements \enquote{in
the documentation and/or other materials provided with the distribution}
is explicitly required for the \enquote{redistribution in binary
form}\footcite[cf.][\nopage wp]{BsdLicense2Clause}, we had to rewrite the
facultative task for a distribution in the form of source code as a mandatory task
for all BSD open source use cases which deals with the redistribution in binary
form:

\begin{quote}\textit{Ensure that the documentation of your distribution and/or
your additional material also contains the author specific copyright notice, the
BSD conditions, and the BSD disclaimer.}\end{quote}

\item In case of (re)distributing the program in the form of binary files, it is
sometimes not enough, to pass the licensing elements as one has received them.
If you compile the binary package from the source code, it is not necessarily
true, that the licensing elements are also automatically generated and embedded
into the 'binary package'. But nevertheless, you have to add the copyright
notice, the conditions and the disclaimer to this package for acting according
to the BSD license. Therefore we chose the following form of an executable,
license fulfilling task for all binary oriented distributions:

\begin{quote}\textit{Ensure that your distribution contains the original
copyright notice, the BSD license, and the BSD disclaimer in the form you have
received them. If you compile the binary file on the base of the source code
package and if this compilation does not also generate and integrate the
licensing files, then create the copyright notice the BSD conditions, and the
BSD disclaimer according to the form of the source code package and insert these
files into your distribution manually.}\end{quote}

\item Finally, we wished to insert a hint to the general (open source)
tradition, to mention the used open source software and their licenses as a
remark of the 'copyright widget' of an application. This is not required by the
BSD license. But it is a general, good tradition. Naturally, because of the
freedom to use and modify open source software and to redistribute a modified
version of it, you are also allowed to insert such references, even if they are
missing. Therefore we added a third voluntary license tradition fulfilling
task for all relevant open source use cases.

\end{itemize}




%\bibliography{../../../bibfiles/oscResourcesEn}
