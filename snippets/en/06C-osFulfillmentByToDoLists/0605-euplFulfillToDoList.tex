% Telekom osCompendium 'for being included' snippet template
%
% (c) Karsten Reincke, Deutsche Telekom AG, Darmstadt 2011
%
% This LaTeX-File is licensed under the Creative Commons Attribution-ShareAlike
% 3.0 Germany License (http://creativecommons.org/licenses/by-sa/3.0/de/): Feel
% free 'to share (to copy, distribute and transmit)' or 'to remix (to adapt)'
% it, if you '... distribute the resulting work under the same or similar
% license to this one' and if you respect how 'you must attribute the work in
% the manner specified by the author ...':
%
% In an internet based reuse please link the reused parts to www.telekom.com and
% mention the original authors and Deutsche Telekom AG in a suitable manner. In
% a paper-like reuse please insert a short hint to www.telekom.com and to the
% original authors and Deutsche Telekom AG into your preface. For normal
% quotations please use the scientific standard to cite.
%
% [ Framework derived from 'mind your Scholar Research Framework' 
%   mycsrf (c) K. Reincke 2012 CC BY 3.0  http://mycsrf.fodina.de/ ]
%


%% use all entries of the bibliography
%\nocite{*}

\section{EUPL-1.1 licensed software}
\begin{license}{EUPL} % ends at end of file
\licensename{EUPL-1.1}
\licensespec{European Union Public License 1.1}
\licenseversion{1.1}
\licenseabbrev{EUPL}

The European Union Public License explicitly distinguishes the distribution of
the source code from that of the binaries: In the chapter \enquote{Communication
of the Source Code,} it allows to \enquote{provide the Work either in its Source
Code form, or as Executable Code.}\citeEUPL{§3} But if a piece of EUPL-1.1 licensed
software is distributed as binary package, then the license additionally
requires that the distributor either \enquote{[\ldots] provides a
machine-readable copy of the Source Code [\ldots]} directly together with the
binaries\citeEUPL{§5} or that he \enquote{[\ldots] indicates [\ldots] a
repository where the Source Code is easily and freely accessible for as long 
as the Licensor continues to distribute [\ldots] the Work.}\citeEUPL{§3} For
respecting this conditions it is irrelevant whether the software has been
modified or not and all the other \enquote{obligations of the licensee} refer to
both forms.\citeEUPL{§5}

There is a particular aspect which has to be considered for acting in
accordance to the EUPL-1.1: Taken literally, the EUPL is a license with a weak
copyleft, no doubt. But this happens only a result of the fact that the EUPL-1.1
allows the licensee to relicense the software by following the conditions of the
\enquote{Compatibility clause}\citeEUPL{§5} and an license listed in an
appendix, which also includes some licenses with a weak copyleft.%
  \footnote{($\rightarrow$ \oslic, p.\ \protectionpageref{EUPL})} 
But, with respect to question how to fulfill the license best, it is safer to
treat the EUPL-1.1 as a license with a strong copyleft. Concerning the use of an
unmodified or a modified library as an embedded component, a license with 
a strong copyleft implies that the application which is using the
(un)modified library has also to be licensed under the same conditions as the
library itself. 
Thus, to find a simple to process task lists, use the following EUPL-1.1
specific open source use case structure:%
  \footnote{For details of the general OSUC finder $\rightarrow$ \oslic,
    pp.\ \pageref{OsucTokens} and \pageref{OsucDefinitionTree}} 

 
\tikzstyle{nodv} = [font=\scriptsize, ellipse, draw, fill=gray!10, 
    text width=2cm, text centered, minimum height=2em]

\tikzstyle{nods} = [font=\tiny, rectangle, draw, fill=gray!20, 
    text width=1cm, text centered, rounded corners, minimum height=3em]

\tikzstyle{nodb} = [font=\tiny, rectangle, draw, fill=gray!20, 
    text width=1.5cm, text centered, rounded corners, minimum height=3em]
    
\tikzstyle{leaf} = [font=\tiny, rectangle, draw, fill=gray!30, 
    text width=1.2cm, text centered, minimum height=6em]

\tikzstyle{slimleaf} = [font=\tiny, rectangle, draw, fill=gray!30, 
    text width=1cm, text centered, minimum height=6em]


\tikzstyle{edge} = [draw, -latex']

\begin{tikzpicture}[]

\node[nodv] (l801) at (4,10.2) {EUPL-1.1};

\node[nodb] (l601) at (0,8.6) {\textit{recipient:} \\ \textbf{4yourself}};
\node[nodb] (l602) at (7.5,8.6) {\textit{recipient:} \\ \textbf{2others}};

\node[nodb] (l501) at (4,7) {\textit{state:} \\ \textbf{unmodified}};
\node[nodb] (l502) at (11,7) {\textit{state:} \\ \textbf{modified}};

\node[nodb] (l401) at (2.25,5.4) {\textit{type:} \\ \textbf{proapse or snimoli}};
\node[nodb] (l402) at (5.4,5.4) {\textit{type:} \\ \textbf{snimoli}};
\node[nodb] (l403) at (8.4,5.4) {\textit{type:} \\ \textbf{proapse}};
\node[nodb] (l404) at (12.8,5.4) {\textit{type:} \\ \textbf{snimoli}};


\node[nodb] (l301) at (2.25,3.8) {\textit{context:} \\ \textbf{independent}};
\node[nodb] (l302) at (5.4,3.8) {\textit{context:} \\ \textbf{embedded}};
\node[nodb] (l303) at (8.4,3.8) {\textit{context:} \\ \textbf{independent}};
\node[nodb] (l304) at (11.3,3.8) {\textit{context:} \\ \textbf{independent}};
\node[nodb] (l305) at (14.3,3.8) {\textit{context:} \\ \textbf{embedded}};

\node[nods] (l201) at (1.45,2.2) {\textit{form:} \textbf{source}};
\node[nods] (l202) at (3.0,2.2) {\textit{form:} \textbf{binary}};
\node[nods] (l203) at (4.6,2.2) {\textit{form:} \textbf{source}};
\node[nods] (l204) at (6.2,2.2) {\textit{form:} \textbf{binary}};
\node[nods] (l205) at (7.7,2.2) {\textit{form:} \textbf{source}};
\node[nods] (l206) at (9.1,2.2) {\textit{form:} \textbf{binary}};
\node[nods] (l207) at (10.5,2.2) {\textit{form:} \textbf{source}};
\node[nods] (l208) at (11.9,2.2) {\textit{form:} \textbf{binary}};
\node[nods] (l209) at (13.4,2.2) {\textit{form:} \textbf{source}};
\node[nods] (l210) at (15.0,2.2) {\textit{form:} \textbf{binary}};

\node[slimleaf] (l101) at (0,0) {\textbf{EUPL-1.1-C1} \textit{using software only
for yourself}};

\node[leaf] (l102) at (1.45,0) { \textbf{EUPL-1.1-C2} \textit{ distributing unmodified
software as independent sources}};

\node[leaf] (l103) at (3.0,0) { \textbf{EUPL-1.1-C3}  \textit{ distributing unmodified
software as independent binaries}};

\node[leaf] (l104) at (4.6,0) { \textbf{EUPL-1.1-C4} \textit{ distributing unmodified
library as embedded sources}};

\node[leaf] (l105) at (6.2,0) { \textbf{EUPL-1.1-C5}  \textit{ distributing unmodified
library as embedded binaries}};

\node[slimleaf] (l106) at (7.7,0) { \textbf{EUPL-1.1-C6}  \textit{ distributing modified
program as sources}};

\node[slimleaf] (l107) at (9.1,0) { \textbf{EUPL-1.1-C7}  \textit{ distributing modified
program as binaries}};

\node[slimleaf] (l108) at (10.5,0) { \textbf{EUPL-1.1-C8}  \textit{ distributing modified
library as independent sources}};

\node[slimleaf] (l109) at (11.9,0) { \textbf{EUPL-1.1-C9} \textit{distributing modified
library as independent binaries}};

\node[leaf] (l110) at (13.4,0) { \textbf{EUPL-1.1-CA}  \textit{distributing
modified library as embedded sources}};

\node[leaf] (l111) at (15,0) { \textbf{EUPL-1.1-CB}  \textit{ distributing modified
library as embedded binaries}};

\path [edge] (l801) -- (l601);
\path [edge] (l801) -- (l602);


\path [edge] (l602) -- (l501);
\path [edge] (l602) -- (l502);

\path [edge] (l501) -- (l401);
\path [edge] (l501) -- (l402);
\path [edge] (l502) -- (l403);
\path [edge] (l502) -- (l404);

\path [edge] (l401) -- (l301);
\path [edge] (l402) -- (l302);
\path [edge] (l403) -- (l303);
\path [edge] (l404) -- (l304);
\path [edge] (l404) -- (l305);

\path [edge] (l301) -- (l201);
\path [edge] (l301) -- (l202);
\path [edge] (l302) -- (l203);
\path [edge] (l302) -- (l204);
\path [edge] (l303) -- (l205);
\path [edge] (l303) -- (l206);
\path [edge] (l304) -- (l207);
\path [edge] (l304) -- (l208);
\path [edge] (l305) -- (l209);
\path [edge] (l305) -- (l210);

\path [edge] (l601) -- (l101);
\path [edge] (l201) -- (l102);
\path [edge] (l202) -- (l103);
\path [edge] (l203) -- (l104);
\path [edge] (l204) -- (l105);
\path [edge] (l205) -- (l106);
\path [edge] (l206) -- (l107);
\path [edge] (l207) -- (l108);
\path [edge] (l208) -- (l109);
\path [edge] (l209) -- (l110);
\path [edge] (l210) -- (l111);

\end{tikzpicture}

%%
%% Common Building Blocks
%%

% ------------------------------------------------------------------------------
% Text of repeated footnotes

\newcommand{\reasonForOtherUseCase}{Making the code accessible via a repository
  means distributing the software in the form of source code. Hence, you must
  also fulfill all tasks of the corresponding use case.}

% ------------------------------------------------------------------------------
% Don't use trademarks, etc for advertising
\newcommand{\noTrademarks}{to promote any of your services or products based on
  the this software by trade names, trademarks, service marks, or names linked
  to this EUPL-1.1 software, except as required for reasonable and customary use in
  describing the origin of the software and reproducing the copyright notice.}

% ------------------------------------------------------------------------------
% Do not remove copyright notices and license files
\newcommand{\keepLicensingElements}{Ensure that the licensing elements
  (particularly the copyright, patent, and trademark notices and all notices
  that refer to the license or to the disclaimer of warranties) are retained in
  your package in the form you have received them.}

\newcommand{\addWhenCompiling}{If you compile the binary from the sources,
  ensure that all the licensing elements are also incorporated into the
  package.} 

% ------------------------------------------------------------------------------
% Give receiver a copy of the license text
\newcommand{\giveLicense}{Give the recipient a copy of the EUPL-1.1 license. If
  it is not already part of the software package, add it.} 

% ------------------------------------------------------------------------------
% Make the source code available
\newcommand{\auxMakeSourceAvailable}[1]{Make the source code of #1 accessible
  via a repository under your own control (even if you did not modify it): 
  Push the source code package into a repository, make it downloadable via the 
  internet, and include an easy to find description in the distribution package,
  which explains how and where the code can be received. Ensure, that this
  repository is online for as long as you continue to distribute the software.} 

% TODO: replace 'overarching'
\newcommand{\makeSourceAvailable}{\auxMakeSourceAvailable{%
    the distributed software}}
\newcommand{\makeAllSourcesAvailable}{\auxMakeSourceAvailable{%
    the embedded library \emph{and} your overarching program}} 

% ------------------------------------------------------------------------------
% Add location of source repository to documentation
\newcommand{\mentionRepositoryInDocumentation}{Insert a prominent hint to the
  download repository into your distribution or your additional material.}

% ------------------------------------------------------------------------------
% Add links and license to documentation
\newcommand{\auxAddToDoc}{Let the documentation of your distribution or
  your additional material also reproduce the content of the existing
  \emph{copyright notice text files}, a hint to the software name, a link to its
  homepage, and a link to the EUPL-1.1 license}

\newcommand{\addToDocumentation}{\auxAddToDoc.}
\newcommand{\addToYourCopyrightNotice}{\auxAddToDoc, preferably as a subsection
  of your own copyright notice.}

% ------------------------------------------------------------------------------
% mark all modifications
\newcommand{\auxMarkAllModifications}[1]{Mark all modifications of source code
  of the #1 thoroughly within the source code and include the date of the
  modification.}  

\newcommand{\markAllProgramModifications}{
  \auxMarkAllModifications{program}}
\newcommand{\markAllLibraryModifications}{%
  \auxMarkAllModifications{library}}
\newcommand{\markAllEmbeddedModifications}{%
  \auxMarkAllModifications{embedded library}}

% ------------------------------------------------------------------------------
% Copyleft
\newcommand{\auxApplyCopyleft}[1]{License your program, which includes the
  library, also under the EUPL-1.1.  Arrange the #1 of the on-top development in 
  a way that they are also covered by the EUPL-1.1 licensing statements.}

\newcommand{\applyCopyleftToSources}{\auxApplyCopyleft{sources}}
\newcommand{\applyCopyleftToBinaries}{\auxApplyCopyleft{binaries}}

% ------------------------------------------------------------------------------
% Marking modification

\newcommand{\auxNewSources}{If you add new source
  code files, insert a header containing your copyright line and an EUPL-1.1
  adequate licensing the statement.}

\newcommand{\auxArrangeModifications}{Arrange your modifications in a way that
  they are covered by the existing EUPL-1.1 licensing statements.}

\newcommand{\arrangeBinaryModifications}{\auxArrangeModifications}
\newcommand{\arrangeSourceModifications}{\auxArrangeModifications\ \auxNewSources}

% ------------------------------------------------------------------------------
% Modification text file

\newcommand{\addModificationTextFile}{Create a \emph{modification text file}, 
  if such a file still does not exist. \emph{Add} a description of your
  modifications to the \emph{modification text file.}}

% ------------------------------------------------------------------------------
% Copyright dialog

\newcommand{\copyrightDialog}{Let the copyright dialog of the on-top development
  clearly say, that it uses the EUPL-1.1 licensed library and that it is itself
  licensed under the EUPL-1.1, too.}

% ------------------------------------------------------------------------------
\subsection{EUPL-1.1-C1: Using the software only for yourself}
\begin{lsuc}{EUPL-1.1-C1}
  \linkosuc{01}
  \linkosuc{03L} 
  \linkosuc{03N} 
  \linkosuc{06L}
  \linkosuc{06N}
  \linkosuc{09L}
  \linkosuc{09N}

  \lsucmeans{that you received EUPL-1.1 licensed software, that you will use it
  only for yourself and that you do not hand it over to any 3rd party in any
  sense.}

  \lsuccovers{OSUC-01, OSUC-03L, OSUC-03N, OSUC-06L, OSUC-06N, OSUC-09L, and
  OSUC-09N\footnote{For details $\rightarrow$ \oslic, pp.\ \pageref{OSUC-01-DEF}
  - \pageref{OSUC-09N-DEF}}}

  \begin{lsucrequiresnothing}
    \lsucitem{You are allowed to use any kind of EUPL-1.1 software in any sense
      and in any context without being obliged to do anything as long as you do
      not give the software to third parties.}
  \end{lsucrequiresnothing}
  
  \begin{lsucprohibits}
    \lsucitem{\noTrademarks}
  \end{lsucprohibits}
\end{lsuc}

% ------------------------------------------------------------------------------
\subsection{EUPL-1.1-C2: Passing the unmodified software as independent sources}
\begin{lsuc}{EUPL-1.1-C2}
  \linkosuc{02S}
  \linkosuc{05S}

  \lsucmeans{that you received EUPL-1.1 licensed software which you are now going
  to distribute to third parties as an independent unit and in the form of
  unmodified source code files or as unmodified source code package. In this
  case it makes no difference if you distribute a program, an application, a
  server, a snippet, a module, a library, or a plugin as an independent or as an
  embedded unit.}

  \lsuccovers{OSUC-02S, OSUC-05S\footnote{For details $\rightarrow$ \oslic,
      pp.\ \pageref{OSUC-02S-DEF} - \pageref{OSUC-05S-DEF}}} 

  \begin{lsucrequires}
    \lsucmandatory{\keepLicensingElements}
    \lsucmandatory{\giveLicense}\passingFilesCorrectly
    \lsucoptional{\addToDocumentation}
  \end{lsucrequires}

  \begin{lsucprohibits}
    \lsucitem{\noTrademarks}
  \end{lsucprohibits}

\end{lsuc}

% ------------------------------------------------------------------------------
\subsection{EUPL-1.1-C3: Passing the unmodified software as independent binaries} 
\begin{lsuc}{EUPL-1.1-C3}
  \linkosuc{02B}
  \linkosuc{05B}

  \lsucmeans{that you received EUPL-1.1 licensed software which you are now going to
  distribute to third parties as an independent unit and in the form of
  unmodified binary files or as unmodified binary package. In this case it does
  not matter if you distribute a program, an application, a server, a snippet, a
  module, a library, or a plugin as an independent or an embedded unit.}

  \lsuccovers{OSUC-02B, OSUC-05B\footnote{For details $\rightarrow$ \oslic,
      pp.\ \pageref{OSUC-02B-DEF} - \pageref{OSUC-05B-DEF}}} 

  \begin{lsucrequires}
    \lsucmandatory{\keepLicensingElements\ \addWhenCompiling}
    \lsucmandatory{\giveLicense}\passingFilesCorrectly
    \lsucmandatory{\makeSourceAvailable}
    \lsucmandatory{\mentionRepositoryInDocumentation}
    \lsucsourcedist{EUPL-1.1-C2}
    \lsucoptional{\addToDocumentation}
  \end{lsucrequires}

  \begin{lsucprohibits}
    \lsucitem{\noTrademarks}
  \end{lsucprohibits}

\end{lsuc}

% ------------------------------------------------------------------------------
\subsection{EUPL-1.1-C4: Passing the unmodified library as embedded sources}
\begin{lsuc}{EUPL-1.1-C4}
  \linkosuc{07S}

  \lsucmeans{that you received a EUPL-1.1 licensed snippet, module or
  library which you are now going to distribute to third parties as an embedded
  component of a larger unit and in the form of unmodified source code files or as
  unmodified source code package.}

  \lsuccovers{OSUC-07S\footnote{For details $\rightarrow$ \oslic,
      pp.\ \pageref{OSUC-07S-DEF}}} 

  \begin{lsucrequires}
    \lsucmandatory{\keepLicensingElements}
    \lsucmandatory{\giveLicense}\passingFilesCorrectly
    \lsucmandatory{\applyCopyleftToSources}
    \lsucoptional{\copyrightDialog}
    \lsucoptional{\addToYourCopyrightNotice}
  \end{lsucrequires}

  \begin{lsucprohibits}
    \lsucitem{\noTrademarks}
  \end{lsucprohibits}

\end{lsuc}

% ------------------------------------------------------------------------------
\subsection{EUPL-1.1-C5: Passing the unmodified library as embedded binaries} 
\begin{lsuc}{EUPL-1.1-C5}
  \linkosuc{07B}

  \lsucmeans{that you received a EUPL-1.1 licensed snippet, module or
  library which you are now going to distribute to third parties as an embedded
  component of a larger unit and in the form of unmodified binary files or as
  unmodified binary package.}

  \lsuccovers{OSUC-07B\footnote{For details $\rightarrow$ \oslic,
      pp.\ \pageref{OSUC-07B-DEF}}} 

  \begin{lsucrequires}
    \lsucmandatory{\keepLicensingElements\ \addWhenCompiling}
    \lsucmandatory{\giveLicense}\passingFilesCorrectly
    \lsucmandatory{\makeAllSourcesAvailable}
    \lsucmandatory{\mentionRepositoryInDocumentation}
    \lsucmandatory{\applyCopyleftToBinaries}
    \lsucsourcedist{EUPL-1.1-C4}
    \lsucoptional{\copyrightDialog}
    \lsucoptional{\addToYourCopyrightNotice}
  \end{lsucrequires}

  \begin{lsucprohibits}
    \lsucitem{\noTrademarks}
  \end{lsucprohibits}

\end{lsuc}

% ------------------------------------------------------------------------------
\subsection{EUPL-1.1-C6: Passing a modified program as source code}
\begin{lsuc}{EUPL-1.1-C6}
  \linkosuc{04S} 

  \lsucmeans{that you received a EUPL-1.1 licensed program, application, or
  server (proapse), that you modified it, and that you are now going to
  distribute this modified version to third parties in the form of source code files or as
  a source code package.} 

  \lsuccovers{OSUC-04S\footnote{For details $\rightarrow$ \oslic,
      pp.\ \pageref{OSUC-04S-DEF}}} 

  \begin{lsucrequires}
    \lsucmandatory{\keepLicensingElements}
    \lsucmandatory{\giveLicense}\passingFilesCorrectly
    \lsucmandatory{\addModificationTextFile}
    \lsucmandatory{\markAllProgramModifications}
    \lsucmandatory{\arrangeSourceModifications}
    \lsucoptional{\addToDocumentation}
  \end{lsucrequires}
 
  \begin{lsucprohibits}
    \lsucitem{\noTrademarks}
  \end{lsucprohibits}

\end{lsuc}

% ------------------------------------------------------------------------------
\subsection{EUPL-1.1-C7: Passing a modified program as binary}
\begin{lsuc}{EUPL-1.1-C7}
  \linkosuc{04B}

  \lsucmeans{that you received a EUPL-1.1 licensed program, application, or
  server (proapse), that you modified it, and that you are now going to
  distribute this modified version to third parties in the form of binary files or as a
  binary package.}

  \lsuccovers{OSUC-04B\footnote{For details $\rightarrow$ \oslic,
      pp.\ \pageref{OSUC-04B-DEF}}} 

  \begin{lsucrequires}
    \lsucmandatory{\keepLicensingElements\ \addWhenCompiling}
    \lsucmandatory{\giveLicense}\passingFilesCorrectly
    \lsucmandatory{\addModificationTextFile}
    \lsucmandatory{\arrangeBinaryModifications}
    \lsucmandatory{\makeSourceAvailable}
    \lsucmandatory{\mentionRepositoryInDocumentation}
    \lsucsourcedist{EUPL-1.1-C6}
    \lsucoptional{\markAllProgramModifications}
    \lsucoptional{\addToDocumentation}
  \end{lsucrequires}

  \begin{lsucprohibits}
    \lsucitem{\noTrademarks}
  \end{lsucprohibits}

\end{lsuc}

% ------------------------------------------------------------------------------
\subsection{EUPL-1.1-C8: Passing a modified library as independent source code}
\begin{lsuc}{EUPL-1.1-C8}
  \linkosuc{08S}

  \lsucmeans{that you received a EUPL-1.1 licensed code snippet, module, library,
  or plugin (snimoli), that you modified it, and that you are now going to
  distribute this modified version to third parties in the form of source code
  files or as a source code package, but without embedding it into another
  larger software unit.}

  \lsuccovers{OSUC-08S\footnote{For details $\rightarrow$ \oslic,
      pp.\ \pageref{OSUC-08S-DEF}}} 

  \begin{lsucrequires}
    \lsucmandatory{\keepLicensingElements}
    \lsucmandatory{\giveLicense}\passingFilesCorrectly
    \lsucmandatory{\addModificationTextFile}
    \lsucmandatory{\markAllLibraryModifications}
    \lsucmandatory{\arrangeSourceModifications}
    \lsucoptional{\addToDocumentation}
  \end{lsucrequires}

  \begin{lsucprohibits}
    \lsucitem{\noTrademarks}
  \end{lsucprohibits}

\end{lsuc}

% ------------------------------------------------------------------------------
\subsection{EUPL-1.1-C9: Passing a modified library as independent binary}
\begin{lsuc}{EUPL-1.1-C9}
  \linkosuc{08B}

  \lsucmeans{that you received a EUPL-1.1 licensed code snippet, module, library,
  or plugin (snimoli), that you modified it, and that you are now going to
  distribute this modified version to third parties in the form of binary files
  or as a binary package but without embedding it into another larger software
  unit.}

  \lsuccovers{OSUC-08B\footnote{For details $\rightarrow$ \oslic,
      pp.\ \pageref{OSUC-08B-DEF}}} 

  \begin{lsucrequires}
    \lsucmandatory{\keepLicensingElements\ \addWhenCompiling}
    \lsucmandatory{\giveLicense}\passingFilesCorrectly
    \lsucmandatory{\addModificationTextFile}
    \lsucmandatory{\arrangeBinaryModifications}
    \lsucmandatory{\makeSourceAvailable}
    \lsucmandatory{\mentionRepositoryInDocumentation}
    \lsucsourcedist{EUPL-1.1-C8}
    \lsucoptional{\markAllLibraryModifications}
    \lsucoptional{\addToDocumentation}
  \end{lsucrequires}

  \begin{lsucprohibits}
    \lsucitem{\noTrademarks}
  \end{lsucprohibits}

\end{lsuc}

% ------------------------------------------------------------------------------
\subsection{EUPL-1.1-CA: Passing a modified library as embedded source code}
\begin{lsuc}{EUPL-1.1-CA}
  \linkosuc{10S}

  \lsucmeans{that you received a EUPL-1.1 licensed code snippet, module, library,
  or plugin (snimoli), that you modified it, and that you are now going to
  distribute this modified version to third parties in the form of source code
  files or as a source code package together with another larger software unit
  which contains this code snippet, module, library, or plugin as an embedded
  component.}

  \lsuccovers{OSUC-10S\footnote{For details $\rightarrow$ \oslic,
      pp.\ \pageref{OSUC-10S-DEF}}} 

  \begin{lsucrequires}
    \lsucmandatory{\keepLicensingElements}
    \lsucmandatory{\giveLicense}\passingFilesCorrectly
    \lsucmandatory{\addModificationTextFile}
    \lsucmandatory{\arrangeSourceModifications}
    \lsucmandatory{\applyCopyleftToSources}
    \lsucmandatory{\markAllEmbeddedModifications}
    \lsucoptional{\copyrightDialog}
    \lsucoptional{\addToYourCopyrightNotice}
  \end{lsucrequires}

  \begin{lsucprohibits}
    \lsucitem{\noTrademarks}
  \end{lsucprohibits}

\end{lsuc}

% ------------------------------------------------------------------------------
\subsection{EUPL-1.1-CB: Passing a modified library as embedded binary}
\begin{lsuc}{EUPL-1.1-CB}
  \linkosuc{10B}

  \lsucmeans{that you received a EUPL-1.1 licensed code snippet, module, library,
  or plugin (snimoli), that you modified it, and that you are now going to
  distribute this modified version to third parties in the form of binary files
  or as a binary package together with another larger software unit which
  contains this code snippet, module, library, or plugin as an embedded component.}

  \lsuccovers{OSUC-10B\footnote{For details $\rightarrow$ \oslic,
      pp.\ \pageref{OSUC-10B-DEF}}} 

  \begin{lsucrequires}
    \lsucmandatory{\keepLicensingElements \addWhenCompiling}
    \lsucmandatory{\giveLicense}\passingFilesCorrectly
    \lsucmandatory{\addModificationTextFile}
    \lsucmandatory{\makeAllSourcesAvailable}
    \lsucmandatory{\mentionRepositoryInDocumentation}
    \lsucsourcedist{EUPL-1.1-CA}
    \lsucmandatory{\arrangeBinaryModifications}
    \lsucmandatory{\applyCopyleftToBinaries}
    \lsucoptional{\markAllEmbeddedModifications}
    \lsucoptional{\addToYourCopyrightNotice}
  \end{lsucrequires}

  \begin{lsucprohibits}
    \lsucitem{\noTrademarks}
  \end{lsucprohibits}

\end{lsuc}

% ------------------------------------------------------------------------------
\subsection{Discussions and Explanations}
\label{EUPLDiscussion}
\begin{itemize}
  
\item The EUPL-1.1 generally \enquote{[\ldots] does not grant permission to use
  the trade names, trademarks, service marks, or names of the Licensor, except
  as required for reasonable and customary use in describing the origin of the
  Work and reproducing the content of the copyright notice.}\citeEUPL{§5} 
  Therefore, the \oslic{} genreally interdicts (EUPL-1.1-C1 -- EUPL-1.1-CB) to promote any
  service or product based on this software by such elements. 

\item The EUPL-1.1 generally requires that \enquote{[\ldots] the Licensee shall
  keep intact all copyright, patent or trademarks notices and all notices that
  refer to the Licence and to the disclaimer of warranties.}\citeEUPL{§5} 
  In a very strict reading, the EUPL-1.1 does not limit this requirement to the
  distribution of the software. But in practise, it will be impossible to
  control the compliant use of the software in those cases (\emph{4yourself})
  unless you also start to distribute the software. Therefore the \oslic{} only
  inserts this requirement as a mandatory clause only for the \emph{2others} use 
  cases (EUPL-1.1-C2 -- EUPL-1.1-CB). 
  
\item The EUPL-1.1 also requires to \enquote{[\ldots] include [\ldots] a copy of
  the (EUPL-1.1) Licence with every (distributed) copy of the Work}.\citeEUPL{§5}
  Therefore, all \emph{2others} use cases contain the respective mandatory
  condition (EUPL-1.1-C2 -- EUPL-1.1-CB).
  
\item Additionally, the EUPL-1.1 requires that the \enquote{licensee} who
  distributes a modified work \enquote{[\ldots] must cause any Derivative Work 
  to carry prominent notices stating that the Work has been modified and the
  date of modification.}\citeEUPL{§5} 
  Thus, the \oslic{} integrates the mandatory requirement to generate (update) a
  respective notice file into all `modification' use cases and recommends to mark
  all modifications in the source code (EUPL-1.1-C6 -- EUPL-1.1-CB).
  
\item Furthermore, the EUPL-1.1 requires that any distributor of the software
  \enquote{[\ldots] provide a machine-readable copy of the Source Code [\ldots]}
  by \enquote{[\ldots] (indicating) a repository where this Source will be
  easily and freely available for as long as the Licensee continues to
  distribute [\ldots] the Work.}%
  \footnote{\cite[cf.][\nopage wp.\ §5]{EuplLicense2007en}. To be precise, the
    EUPL-1.1 also allows to directly distribute the source code together with the
    binary packages (\cite[cf.][\nopage wp.\ §3]{EuplLicense2007en}). With
    respect to the \oslic{} principle to offer only one reliable way, the \oslic{}
    simplifies this option: It `only' asks for the repository solution.} 
  Therefore the \oslic{} inserts a respective requirement into the task list of all
  cases concerning a binary distribution (EUPL-1.1-C3, EUPL-1.1-C7, EUPL-1.1-C9, and EUPL-1.1-CB)
  
\item Finally, the EUPL-1.1 contains a \enquote{copyleft clause} stating that if a
  \enquote{[\ldots] Licensee distributes [\ldots] copies of the Original Works
  or Derivative Works based upon the Original Work, this Distribution [...] will
  be done under the terms of this (EUPL-1.1) Licence [\ldots]}. In all the use cases
  which do not concern the use of an embedded component (EUPL-1.1-C2 -- EUPL-1.1-C9) this
  copyleft clause is already fulfilled by either distributing the modified
  sources themselves or by making them accessible via a repository. In those
  cases where the licensee distributes an program that uses an embedded EUPL-1.1
  licensed component (EUPL-1.1-CA -- EUPL-1.1-CB), in general, the code of the embedding
  program must also be distributed. Thus, with respect to the use case (EUPL-1.1-CA)
  this is already fulfilled by definition. Therefore, the \oslic{} only mentions
  this default view in the case EUPL-1.1-CB implying a strong copyleft effect.%
  \footnote{Formally, the EUPL-1.1 is only a license with weak copyleft. 
    But this is only a result of allowing to relicense the software
    ($\rightarrow$ \oslic, p.\ \protectionpageref{EUPL}). So, as long as
    you do not relicense the embedded library with respect to the list of
    \enquote{compatible licenses according to article 5 EUPL-1.1} 
    (\cite[cf.][\nopage wp §5 and Appendix]{EuplLicense2007en}), 
    you also have to publish the code of your overarching work.}

\end{itemize}

\end{license}

%\bibliography{../../../bibfiles/oscResourcesEn}

% Local Variables:
% mode: latex
% fill-column: 80
% End:
