% Telekom osCompendium 'for being included' snippet template
%
% (c) Karsten Reincke, Deutsche Telekom AG, Darmstadt 2011
%
% This LaTeX-File is licensed under the Creative Commons Attribution-ShareAlike
% 3.0 Germany License (http://creativecommons.org/licenses/by-sa/3.0/de/): Feel
% free 'to share (to copy, distribute and transmit)' or 'to remix (to adapt)'
% it, if you '... distribute the resulting work under the same or similar
% license to this one' and if you respect how 'you must attribute the work in
% the manner specified by the author ...':
%
% In an internet based reuse please link the reused parts to www.telekom.com and
% mention the original authors and Deutsche Telekom AG in a suitable manner. In
% a paper-like reuse please insert a short hint to www.telekom.com and to the
% original authors and Deutsche Telekom AG into your preface. For normal
% quotations please use the scientific standard to cite.
%
% [ Framework derived from 'mind your Scholar Research Framework' 
%   mycsrf (c) K. Reincke 2012 CC BY 3.0  http://mycsrf.fodina.de/ ]
%


%% use all entries of the bibliography
%\nocite{*}

\section{EUPL licensed software}

The European Union Public License explicitly distinguishes the distribution
of the source code from that of the binaries: In the chapter
\enquote{Communication of the Source Code}, it allows to \enquote{provide the
Work either in its Source Code form, or as Executable
Code}\footcite[cf.][\nopage wp.\ §3]{EuplLicense2007en}. But if a piece of EUPL
licensed software is distributed as binary package, then the license
additionally requires that the distributor either \enquote{[\ldots] provides a
machine-readable copy of the Source Code [\ldots]} directly together with the
binaries\footcite[cf.][\nopage wp.\ §5]{EuplLicense2007en} or that he
\enquote{[\ldots] indicates [\ldots] a repository where the Source Code is
easily and freely accessible for as long as the Licensor continues to distribute
[\ldots] the Work}\footnote{\cite[cf.][\nopage wp.\ §3.]{EuplLicense2007en}}. For
respecting this conditions it is irrelevant whether the software has been
modified or not; and all the other \enquote{obligations of the licensee} refer
to both forms\footcite[cf.][\nopage wp.\ §5]{EuplLicense2007en}. 

But there is a specific aspect which has to be considered for acting in
accordance to the EUPL: In the proper meaning of the words, the EUPL is a
license with a weak copy left, no doubt. But this character is only evoked by
the fact that the EUPL allows the licensee to relicense the software by
following the conditions of a specific clause and an licenses listed in an
appendix which also lists some licenses with a weak copy
left.\footnote{($\rightarrow$ OSLiC, p.\ \pageref{sec:ProtectingPowerOfEupl})}.
Thus, with respect to question how to fulfill the license best, it is safer to
treat the EUPL as a license with a default strong copy left. Concerning the use
of an unmodified or a modified library as an embedded component, a license with
a strong copy left evokes that also the application which is using the
(un)modified library has to be licensed under the same conditions as the library
itself. Thus, for finding the simply processable task lists, the following EUPL
specific and a little more sophisticated open source use case
structure\footnote{For details of the general OSUC finder $\rightarrow$ OSLiC,
pp.\ \pageref{OsucTokens} and \pageref{OsucDefinitionTree}} can be used:
 
 
 
\tikzstyle{nodv} = [font=\scriptsize, ellipse, draw, fill=gray!10, 
    text width=2cm, text centered, minimum height=2em]

\tikzstyle{nods} = [font=\tiny, rectangle, draw, fill=gray!20, 
    text width=1cm, text centered, rounded corners, minimum height=3em]

\tikzstyle{nodb} = [font=\tiny, rectangle, draw, fill=gray!20, 
    text width=1.5cm, text centered, rounded corners, minimum height=3em]
    
\tikzstyle{leaf} = [font=\tiny, rectangle, draw, fill=gray!30, 
    text width=1.2cm, text centered, minimum height=6em]

\tikzstyle{slimleaf} = [font=\tiny, rectangle, draw, fill=gray!30, 
    text width=1cm, text centered, minimum height=6em]


\tikzstyle{edge} = [draw, -latex']

\begin{tikzpicture}[]

\node[nodv] (l801) at (4,10.2) {EUPL};

\node[nodb] (l601) at (0,8.6) {\textit{recipient:} \\ \textbf{4yourself}};
\node[nodb] (l602) at (7.5,8.6) {\textit{recipient:} \\ \textbf{2others}};

\node[nodb] (l501) at (4,7) {\textit{state:} \\ \textbf{unmodified}};
\node[nodb] (l502) at (11,7) {\textit{state:} \\ \textbf{modified}};

\node[nodb] (l401) at (2.25,5.4) {\textit{type:} \\ \textbf{proapse or snimoli}};
\node[nodb] (l402) at (5.4,5.4) {\textit{type:} \\ \textbf{snimoli}};
\node[nodb] (l403) at (8.4,5.4) {\textit{type:} \\ \textbf{proapse}};
\node[nodb] (l404) at (12.8,5.4) {\textit{type:} \\ \textbf{snimoli}};


\node[nodb] (l301) at (2.25,3.8) {\textit{context:} \\ \textbf{independent}};
\node[nodb] (l302) at (5.4,3.8) {\textit{context:} \\ \textbf{embedded}};
\node[nodb] (l303) at (8.4,3.8) {\textit{context:} \\ \textbf{independent}};
\node[nodb] (l304) at (11.3,3.8) {\textit{context:} \\ \textbf{independent}};
\node[nodb] (l305) at (14.3,3.8) {\textit{context:} \\ \textbf{embedded}};

\node[nods] (l201) at (1.45,2.2) {\textit{form:} \textbf{source}};
\node[nods] (l202) at (3.0,2.2) {\textit{form:} \textbf{binary}};
\node[nods] (l203) at (4.6,2.2) {\textit{form:} \textbf{source}};
\node[nods] (l204) at (6.2,2.2) {\textit{form:} \textbf{binary}};
\node[nods] (l205) at (7.7,2.2) {\textit{form:} \textbf{source}};
\node[nods] (l206) at (9.1,2.2) {\textit{form:} \textbf{binary}};
\node[nods] (l207) at (10.5,2.2) {\textit{form:} \textbf{source}};
\node[nods] (l208) at (11.9,2.2) {\textit{form:} \textbf{binary}};
\node[nods] (l209) at (13.4,2.2) {\textit{form:} \textbf{source}};
\node[nods] (l210) at (15.0,2.2) {\textit{form:} \textbf{binary}};

\node[slimleaf] (l101) at (0,0) {\textbf{EUPL-C1} \textit{using software only
for yourself}};

\node[leaf] (l102) at (1.45,0) { \textbf{EUPL-C2} \textit{ distributing unmodified
software as independent sources}};

\node[leaf] (l103) at (3.0,0) { \textbf{EUPL-C3}  \textit{ distributing unmodified
software as independent binaries}};

\node[leaf] (l104) at (4.6,0) { \textbf{EUPL-C4} \textit{ distributing unmodified
library as embedded sources}};

\node[leaf] (l105) at (6.2,0) { \textbf{EUPL-C5}  \textit{ distributing unmodified
library as embedded binaries}};

\node[slimleaf] (l106) at (7.7,0) { \textbf{EUPL-C6}  \textit{ distributing modified
program as sources}};

\node[slimleaf] (l107) at (9.1,0) { \textbf{EUPL-C7}  \textit{ distributing modified
program as binaries}};

\node[slimleaf] (l108) at (10.5,0) { \textbf{EUPL-C8}  \textit{ distributing modified
library as independent sources}};

\node[slimleaf] (l109) at (11.9,0) { \textbf{EUPL-C9} \textit{distributing modified
library as independent binaries}};

\node[leaf] (l110) at (13.4,0) { \textbf{EUPL-CA}  \textit{distributing
modified library as embedded sources}};

\node[leaf] (l111) at (15,0) { \textbf{EUPL-CB}  \textit{ distributing modified
library as embedded binaries}};

\path [edge] (l801) -- (l601);
\path [edge] (l801) -- (l602);


\path [edge] (l602) -- (l501);
\path [edge] (l602) -- (l502);

\path [edge] (l501) -- (l401);
\path [edge] (l501) -- (l402);
\path [edge] (l502) -- (l403);
\path [edge] (l502) -- (l404);

\path [edge] (l401) -- (l301);
\path [edge] (l402) -- (l302);
\path [edge] (l403) -- (l303);
\path [edge] (l404) -- (l304);
\path [edge] (l404) -- (l305);

\path [edge] (l301) -- (l201);
\path [edge] (l301) -- (l202);
\path [edge] (l302) -- (l203);
\path [edge] (l302) -- (l204);
\path [edge] (l303) -- (l205);
\path [edge] (l303) -- (l206);
\path [edge] (l304) -- (l207);
\path [edge] (l304) -- (l208);
\path [edge] (l305) -- (l209);
\path [edge] (l305) -- (l210);

\path [edge] (l601) -- (l101);
\path [edge] (l201) -- (l102);
\path [edge] (l202) -- (l103);
\path [edge] (l203) -- (l104);
\path [edge] (l204) -- (l105);
\path [edge] (l205) -- (l106);
\path [edge] (l206) -- (l107);
\path [edge] (l207) -- (l108);
\path [edge] (l208) -- (l109);
\path [edge] (l209) -- (l110);
\path [edge] (l210) -- (l111);

\end{tikzpicture}


\subsection{EUPL-C1: Using the software only for yourself}
\label{OSUC-01-EUPL} \label{OSUC-03-EUPL} 
\label{OSUC-06-EUPL} \label{OSUC-09-EUPL}

\begin{description}

\item[means] that you are going to use a received EUPL licensed software only
for yourself and that you do not hand it over to any 3rd party in any sense.

\item[covers] OSUC-01, OSUC-03, OSUC-06, and OSUC-09\footnote{For details 
$\rightarrow$ OSLiC, pp.\ \pageref{OSUC-01-DEF} - \pageref{OSUC-09-DEF}}

\item[requires] no tasks in order to fulfill the conditions of the EUPL 1.1
license with respect to this use case:
  \begin{itemize}
    \item You are allowed to use any kind of EUPL software in any sense and in
    any context without being obliged to do anything as long as you do not
    give the software to 3rd parties.
  \end{itemize}
  
\item[prohibits] \ldots
\begin{itemize}
  \item to promote any of your services or products -- based on the this software
  -- by trade names, trademarks, service marks, or names linked to this EUPL
  software, except as required for unpartially describing the used software and
  reproducing the copyright notice.
\end{itemize}

\end{description}

\subsection{EUPL-C2: Passing the unmodified software as independent sources}
\label{OSUC-02S-EUPL} \label{OSUC-05S-EUPL}

\begin{description}

\item[means] that you are going to distribute an unmodified version of the
received EUPL software to 3rd parties -- as an independent unit and in the form
of source code files or as a source code package. In this case, it doesn't
matter whether you distribute a program, an application, a server, a snippet, a
module, a library, or a plugin.

\item[covers] OSUC-02S, OSUC-05S\footnote{For details $\rightarrow$
OSLiC, pp.\ \pageref{OSUC-02S-DEF} - \pageref{OSUC-05S-DEF}}

\item[requires] the following tasks in order to fulfill the license conditions:
\begin{itemize}
  
  \item \textbf{[mandatory:]} Ensure that the licensing elements -- esp.\ the
  copyright, patent or trademarks notices and all notices that refer to the
  license and to the disclaimer of warranties -- are retained in your package in
  the form you have received them.
  
  \item \textbf{[mandatory:]} Give the recipient a copy of the EUPL 1.1
  license. If it is not already part of the software package, add
  it\footnote{For implementing the handover of files correctly $\rightarrow$
  OSLiC, p. \pageref{DistributingFilesHint}}.
  
  \item \textbf{[voluntary:]} Let the documentation of your distribution and/or
  your additional material also reproduce the content of the existing
  \emph{copyright notice text files}, a hint to the software name, a link to its
  homepage, and a link to the EUPL 1.1 license.
\end{itemize}

\item[prohibits] \ldots
\begin{itemize}
  \item to promote any of your services or products -- based on the this software
  -- by trade names, trademarks, service marks, or names linked to this EUPL
  software, except as required for unpartially describing the used software and
  reproducing the copyright notice.
\end{itemize}

\end{description}


\subsection{EUPL-C3: Passing the unmodified software as independent binaries} 
\label{OSUC-02B-EUPL} \label{OSUC-05B-EUPL}

\begin{description}

\item[means] that you are going to distribute an unmodified version of the
received EUPL software to 3rd parties -- as an independent unit and in the form
of binary files or as a binary package. In this case, it doesn't matter whether
you distribute a program, an application, a server, a snippet, a module, a
library, or a plugin.

\item[covers] OSUC-02B, OSUC-05B\footnote{For details $\rightarrow$
OSLiC, pp.\ \pageref{OSUC-02B-DEF} - \pageref{OSUC-05B-DEF}}

\item[requires] the following tasks in order to fulfill the license conditions:
\begin{itemize}
  
  \item \textbf{[mandatory:]} Ensure that the licensing elements -- esp.\ the
  copyright, patent or trademarks notices and all notices that refer to the
  license and to the disclaimer of warranties -- are retained in your package in
  the form you have received them. If you compile the binary from the sources,
  ensure that all the licensing elements are also incorporated into the package.
  
  \item \textbf{[mandatory:]} Give the recipient a copy of the EUPL 1.1
  license. If it is not already part of the binary package, add
  it\footnote{For implementing the handover of files correctly $\rightarrow$
  OSLiC, p. \pageref{DistributingFilesHint}}.

  \item \textbf{[mandatory:]} Make the source code of the distributed software
  accessible via a repository under your own control (even if you do not
  modified it): Push the source code package into a repository, make it
  downloadable via the internet, and integrate an easily to find description
  into the distribution package which explains how the code can be received from
  where. Ensure, that this repository is online for as long as you continue to
  distribute the software.
  
 \item \textbf{[mandatory:]} Insert a prominent hint to the download repository
  into your distribution and/or your additional material.
  
  \item \textbf{[mandatory:]} Execute the to-do list of use case EUPL-C2\footnote{
  Making the code accessible via a repository means distributing the software in
  the form of source code. Hence, you must also fulfill all tasks of the
  corresponding use case.}.
 
  \item \textbf{[voluntary:]} Let the documentation of your distribution and/or
  your additional material also reproduce the content of the existing
  \emph{copyright notice text files}, a hint to the software name, a link to its
  homepage, and a link to the EUPL 1.1 license.
\end{itemize}

\item[prohibits] \ldots
\begin{itemize}
  \item to promote any of your services or products -- based on the this software
  -- by trade names, trademarks, service marks, or names linked to this EUPL
  software, except as required for unpartially describing the used software and
  reproducing the copyright notice.
\end{itemize}

\end{description}

\subsection{EUPL-C4: Passing the unmodified library as embedded sources}
\label{OSUC-07S-EUPL}

\begin{description}

\item[means] that you are going to distribute an unmodified version of the
received EUPL licensed snippet, module or library to 3rd parties -- as embedded
component of a larger unit and in the form of source code files or as a source
code package.

\item[covers] OSUC-07S\footnote{For details $\rightarrow$
OSLiC, pp.\ \pageref{OSUC-057-DEF}}

\item[requires] the following tasks in order to fulfill the license conditions:
\begin{itemize}
  
  \item \textbf{[mandatory:]} Ensure that the licensing elements -- esp.\ the
  copyright, patent or trademarks notices and all notices that refer to the
  license and to the disclaimer of warranties -- are retained in your package in
  the form you have received them.
  
  \item \textbf{[mandatory:]} Give the recipient a copy of the EUPL 1.1
  license. If it is not already part of the software package, add
  it\footnote{For implementing the handover of files correctly $\rightarrow$
  OSLiC, p. \pageref{DistributingFilesHint}}.
  
  \item \textbf{[mandatory:]} License your overarching program also under the
  EUPL 1.1; Organize the sources of the on-top development in a way that they
  are also covered by the EUPL-1.1 licensing statements.

  \item \textbf{[voluntary:]} Let the copyright dialog of the on-top development
  clearly say, that it uses the EUPL-1.1 licensed library and that it is itself
  licensed under the EUPL-1.1 too.
  
  \item \textbf{[voluntary:]} Let the documentation of your distribution and/or
  your additional material also reproduce the content of the existing
  \emph{copyright notice text files}, a hint to the software name, a link to its
  homepage, and a link to the EUPL 1.1 license.
\end{itemize}

\item[prohibits] \ldots
\begin{itemize}
  \item to promote any of your services or products -- based on the this software
  -- by trade names, trademarks, service marks, or names linked to this EUPL
  software, except as required for unpartially describing the used software and
  reproducing the copyright notice.
\end{itemize}

\end{description}


\subsection{EUPL-C5: Passing the unmodified library as embedded binaries} 
\label{OSUC-07B-EUPL}

\begin{description}
\item[means] that you are going to distribute an unmodified version of the
received EUPL licensed snippet, module or library to 3rd parties -- as embedded
component of a larger unit and in the form of binary files or as a bi\-na\-ry
package.

\item[covers] OSUC-07B\footnote{For details $\rightarrow$
OSLiC, pp.\ \pageref{OSUC-07B-DEF}}

\item[requires] the following tasks in order to fulfill the license conditions:
\begin{itemize}
  
  \item \textbf{[mandatory:]} Ensure that the licensing elements -- esp.\ the
  copyright, patent or trademarks notices and all notices that refer to the
  license and to the disclaimer of warranties -- are retained in your package in
  the form you have received them. If you compile the binary from the sources,
  ensure that all the licensing elements are also incorporated into the package.
  
  \item \textbf{[mandatory:]} Give the recipient a copy of the EUPL 1.1
  license. If it is not already part of the binary package, add
  it\footnote{For implementing the handover of files correctly $\rightarrow$
  OSLiC, p. \pageref{DistributingFilesHint}}.

  \item \textbf{[mandatory:]} Make the source code of the embedded library
  \textbf{\emph{and}} the source code of your overarching program accessible via
  a repository under your own control (even if you do not modified it): Push the
  source code package into a repository, make it downloadable via the internet,
  and integrate an easily to find description into the distribution package
  which explains how the code can be received from where. Ensure, that this
  repository is online for as long as you continue to distribute the software.
  
  \item \textbf{[mandatory:]} Insert a prominent hint to the download repository
  into your distribution and/or your additional material.
  
  \item \textbf{[mandatory:]} License your overarching program also under the
  EUPL 1.1: Organize the binaries of the on-top development in a way that they
  are also covered by the EUPL-1.1 licensing statements.
  
  \item \textbf{[mandatory:]} Execute the to-do list of use case EUPL-C4\footnote{
  Making the code accessible via a repository means distributing the software in
  the form of source code. Hence, you must also fulfill all tasks of the
  corresponding use case.}.
  
 \item \textbf{[voluntary:]} Let the copyright dialog of the on-top development
  clearly say, that it uses the EUPL-1.1 licensed library and that it is itself
  licensed under the EUPL-1.1 too.
 
  \item \textbf{[voluntary:]} Let the documentation of your distribution and/or
  your additional material also reproduce the content of the existing
  \emph{copyright notice text files}, a hint to the software name, a link to its
  homepage, and a link to the EUPL 1.1 license.
\end{itemize}

\item[prohibits] \ldots
\begin{itemize}
  \item to promote any of your services or products -- based on the this software
  -- by trade names, trademarks, service marks, or names linked to this EUPL
  software, except as required for unpartially describing the used software and
  reproducing the copyright notice.
\end{itemize}

\end{description}

\subsection{EUPL-C6: Passing a modified program as source code}
\label{OSUC-04S-EUPL} 

\begin{description}
\item[means] that you are going to distribute a modified version of the received
EUPL licensed program, application, or server (proapse) to 3rd parties -- in the
form of source code files or as a source code package.
\item[covers] OSUC-04S\footnote{For details $\rightarrow$ OSLiC, pp.\
\pageref{OSUC-04S-DEF}}
\item[requires] the tasks in order to fulfill the license conditions:
\begin{itemize}
  
  \item \textbf{[mandatory:]} Ensure that the licensing elements -- esp.\ the
  copyright, patent or trademarks notices and all notices that refer to the
  license and to the disclaimer of warranties -- are retained in your package in
  the form you have received them.
  
  \item \textbf{[mandatory:]} Give the recipient a copy of the EUPL 1.1
  license. If it is not already part of the software package, add
  it\footnote{For implementing the handover of files correctly $\rightarrow$
  OSLiC, p. \pageref{DistributingFilesHint}}.

  \item \textbf{[mandatory:]} Create a \emph{modification text file}, if such a
  notice file still does not exist. \emph{Expand} the \emph{modification text
  file} by a description of your modifications.
    
  \item \textbf{[mandatory:]} Mark all modifications of source code of the
  program (proapse) thoroughly -- namely within the source code and including
  the date of the modification.
   
  \item \textbf{[mandatory:]} Organize your modifications in a way that they are
  covered by the existing EUPL licensing statements.
   
  \item \textbf{[voluntary:]} Let the documentation of your distribution and/or
  your additional material also reproduce the content of the existing
  \emph{copyright notice text files}, a hint to the software name, a link to its
  homepage, and a link to the EUPL 1.1 license.
  
 \end{itemize}
 
\item[prohibits] \ldots
\begin{itemize}
  \item to promote any of your services or products -- based on the this software
  -- by trade names, trademarks, service marks, or names linked to this EUPL
  software, except as required for unpartially describing the used software and
  reproducing the copyright notice.
\end{itemize}

\end{description}

\subsection{EUPL-C7: Passing a modified program as binary}
\label{OSUC-04B-EUPL}

\begin{description}
\item[means] that you are going to distribute a modified version of the received
EUPL licensed pro\-gram, application, or server (proapse) to 3rd parties -- in
the form of binary files or as a binary package.
\item[covers] OSUC-04B\footnote{For details $\rightarrow$ OSLiC, pp.\
\pageref{OSUC-04B-DEF}}
\item[requires] the tasks in order to fulfill the license conditions:
\begin{itemize}

  \item \textbf{[mandatory:]} Ensure that the licensing elements -- esp.\ the
  copyright, patent or trademarks notices and all notices that refer to the
  license and to the disclaimer of warranties -- are retained in your package in
  the form you have received them. If you compile the binary from the sources,
  ensure that all the licensing elements are also incorporated into the package.

 \item \textbf{[mandatory:]} Give the recipient a copy of the EUPL 1.1
  license. If it is not already part of the binary package, add
  it\footnote{For implementing the handover of files correctly $\rightarrow$
  OSLiC, p. \pageref{DistributingFilesHint}}.
  
  \item \textbf{[mandatory:]} Create a \emph{modification text file}, if such a
  notice file still does not exist. \emph{Expand} the \emph{modification text
  file} by a description of your modifications.

  \item \textbf{[mandatory:]} Organize your modifications in a way that they are
  covered by the existing EUPL licensing statements.
  
  \item \textbf{[mandatory:]} Make the source code of the distributed software
  accessible via a repository under your own control: Push the source code
  package into a repository, make it downloadable via the internet, and
  integrate an easily to find description into the distribution package which
  explains how the code can be received from where. Ensure, that this repository
  is online for as long as you continue to distribute the software.
  
  \item \textbf{[mandatory:]} Insert a prominent hint to the download repository
  into your distribution and/or your additional material.
    
  \item \textbf{[mandatory:]} Execute the to-do list of use case EUPL-C6\footnote{
  Making the code accessible via a repository means distributing the software in
  the form of source code. Hence, you must also fulfill all tasks of the
  corresponding use case.}.
  
  \item \textbf{[voluntary:]} Mark all modifications of source code of the
  program (proapse) thoroughly -- namely within the
  source code and including the date of the modification.
 
  \item \textbf{[voluntary:]} Let the documentation of your distribution and/or
  your additional material  also reproduce the content of the existing
  \emph{copyright notice text files}, a hint to the software name, a link to its
  homepage, and a link to the EUPL 1.1 license -- especially as a subsection of
  your own copyright notice.


\end{itemize}

\item[prohibits] \ldots
\begin{itemize}
  \item to promote any of your services or products -- based on the this software
  -- by trade names, trademarks, service marks, or names linked to this EUPL
  software, except as required for unpartially describing the used software and
  reproducing the copyright notice.
\end{itemize}

\end{description}

\subsection{EUPL-C8: Passing a modified library as independent source code}
\label{OSUC-08S-EUPL}

\begin{description}
\item[means] that you are going to distribute a modified version of the received
EUPL licensed code snippet, module, library, or plugin (snimoli) to 3rd parties
-- in the form of source code files or as a source code package, but without
embedding it into another larger software unit.
\item[covers] OSUC-08S\footnote{For details $\rightarrow$ OSLiC, pp.\
\pageref{OSUC-08S-DEF}}
\item[requires] the tasks in order to fulfill the license conditions:
\begin{itemize}
  
  \item \textbf{[mandatory:]} Ensure that the licensing elements -- esp.\ the
  copyright, patent or trademarks notices and all notices that refer to the
  license and to the disclaimer of warranties -- are retained in your package in
  the form you have received them.
  
  \item \textbf{[mandatory:]} Give the recipient a copy of the EUPL 1.1
  license. If it is not already part of the software package, add
  it\footnote{For implementing the handover of files correctly $\rightarrow$
  OSLiC, p. \pageref{DistributingFilesHint}}.

  \item \textbf{[mandatory:]} Create a \emph{modification text file}, if such a
  notice file still does not exist. \emph{Expand} the \emph{modification text
  file} by a description of your modifications.
  
  \item \textbf{[mandatory:]} Mark all modifications of source code of the
  library (snimoli) thoroughly -- namely within the source code and including
  the date of the modification.
   
  \item \textbf{[mandatory:]} Organize your modifications in a way that they are
  covered by the existing EUPL licensing statements.
  
  \item \textbf{[voluntary:]} Let the documentation of your distribution and/or
  your additional material  also reproduce the content of the existing
  \emph{copyright notice text files}, a hint to the software name, a link to its
  homepage, and a link to the EUPL 1.1 license.

\end{itemize}

\item[prohibits] \ldots
\begin{itemize}
  \item to promote any of your services or products -- based on the this software
  -- by trade names, trademarks, service marks, or names linked to this EUPL
  software, except as required for unpartially describing the used software and
  reproducing the copyright notice.
\end{itemize}

\end{description}


\subsection{EUPL-C9: Passing a modified library as independent binary}
\label{OSUC-08B-EUPL}

\begin{description}
\item[means] that you are going to distribute a modified version of the received
EUPL licensed code snippet, module, library, or plugin (snimoli) to 3rd parties
-- in the form of binary files or as a binary package but without embedding it
into another larger software unit.
\item[covers] OSUC-08B\footnote{For details $\rightarrow$ OSLiC, pp.\ \pageref{OSUC-08B-DEF}}
\item[requires] the tasks in order to fulfill the license conditions:
\begin{itemize}

  \item \textbf{[mandatory:]} Ensure that the licensing elements -- esp.\ the
  copyright, patent or trademarks notices and all notices that refer to the
  license and to the disclaimer of warranties -- are retained in your package in
  the form you have received them. If you compile the binary from the sources,
  ensure that all the licensing elements are also incorporated into the package.
  
 \item \textbf{[mandatory:]} Give the recipient a copy of the EUPL 1.1
  license. If it is not already part of the binary package, add
  it\footnote{For implementing the handover of files correctly $\rightarrow$
  OSLiC, p. \pageref{DistributingFilesHint}}.

  \item \textbf{[mandatory:]} Create a \emph{modification text file}, if such a
  notice file still does not exist. \emph{Expand} the \emph{modification text
  file} by a description of your modifications.

  \item \textbf{[mandatory:]} Organize your modifications in a way that they are
  covered by the existing EUPL licensing statements.
  
  \item \textbf{[mandatory:]} Make the source code of the distributed software
  accessible via a repository under your own control: Push the source code
  package into a repository, make it downloadable via the internet, and
  integrate an easily to find description into the distribution package which
  explains how the code can be received from where. Ensure, that this repository
  is online for as long as you continue to distribute the software.
  
  \item \textbf{[mandatory:]} Insert a prominent hint to the download repository
  into your distribution and/or your additional material.

  \item \textbf{[mandatory:]} Execute the to-do list of use case EUPL-C8\footnote{
  Making the code accessible via a repository means distributing the software in
  the form of source code. Hence, you must also fulfill all tasks of the
  corresponding use case.}.
    
  \item \textbf{[voluntary:]} Mark all modifications of source code of the
  library (snimoli) thoroughly -- namely within the source code and including
  the date of the modification.
  
  \item \textbf{[voluntary:]} Let the copyright dialog of the on-top development
  clearly say, that it uses the EUPL-1.1 licensed library and that it is itself
  licensed under the EUPL-1.1 too.
  
  \item \textbf{[voluntary:]} Let the documentation of your distribution and/or
  your additional material  also reproduce the content of the existing
  \emph{copyright notice text files}, a hint to the software name, a link to its
  homepage, and a link to the EUPL 1.1 license -- especially as a subsection of
  your own copyright notice.
  
\end{itemize}

\item[prohibits] \ldots
\begin{itemize}
  \item to promote any of your services or products -- based on the this software
  -- by trade names, trademarks, service marks, or names linked to this EUPL
  software, except as required for unpartially describing the used software and
  reproducing the copyright notice.
\end{itemize}

\end{description}

\subsection{EUPL-CA: Passing a modified library as embedded source code}
\label{OSUC-10S-EUPL}

\begin{description}
\item[means] that you are going to distribute a modified version of the received
EUPL licensed code snippet, module, library, or plugin (snimoli) to 3rd parties
-- in the form of source code files or as a source code package together with
another larger software unit which contains this code snippet, module, library,
or plugin as an embedded component.
\item[covers] OSUC-10S\footnote{For details $\rightarrow$ OSLiC, pp.\
\pageref{OSUC-10S-DEF}}
\item[requires] the tasks in order to fulfill the license conditions:
\begin{itemize}

  \item \textbf{[mandatory:]} Ensure that the licensing elements -- esp.\ the
  copyright, patent or trademarks notices and all notices that refer to the
  license and to the disclaimer of warranties -- are retained in your package in
  the form you have received them.

  \item \textbf{[mandatory:]} Give the recipient a copy of the EUPL 1.1
  license. If it is not already part of the software package, add
  it\footnote{For implementing the handover of files correctly $\rightarrow$
  OSLiC, p. \pageref{DistributingFilesHint}}.

  \item \textbf{[mandatory:]} Create a \emph{modification text file}, if such a
  notice file still does not exist. \emph{Expand} the \emph{modification text
  file} by a description of your modifications.
  
  \item \textbf{[mandatory:]} Organize your modifications of the embedded
  library in a way that they are covered by the existing EUPL licensing
  statements. 
  
  \item \textbf{[mandatory:]} License your overarching program also under the
  EUPL 1.1.
  
  \item \textbf{[mandatory:]} Mark all modifications of source code of the
  embedded library (snimoli) thoroughly --
  namely within the source code and including the date of the modification.
  
  \item \textbf{[voluntary:]} Let the copyright dialog of the on-top development
  clearly say, that it uses the EUPL-1.1 licensed library and that it is itself
  licensed under the EUPL-1.1 too.
  
  \item \textbf{[voluntary:]} Let the documentation of your distribution and/or
  your additional material  also reproduce the content of the existing
  \emph{copyright notice text files}, a hint to the name of the used EUPL
  licensed component, a link to its homepage, and a link to the EUPL 1.1 license
  -- especially as subsection of your own copyright notice.
 
\end{itemize}

\item[prohibits] \ldots
\begin{itemize}
  \item to promote any of your services or products -- based on the this software
  -- by trade names, trademarks, service marks, or names linked to this EUPL
  software, except as required for unpartially describing the used software and
  reproducing the copyright notice.
\end{itemize}

\end{description}


\subsection{EUPL-CB: Passing a modified library as embedded binary}
\label{OSUC-10B-EUPL}

\begin{description}
\item[means] that you are going to distribute a modified version of the received
EUPL licensed code snippet, module, library, or plugin to 3rd parties -- in the
form of binary files or as a binary package together with another larger
software unit which contains this code snippet, module, library, or plugin as an
embedded component.
\item[covers] OSUC-10B\footnote{For details $\rightarrow$ OSLiC, pp.\
\pageref{OSUC-10B-DEF}}
\item[requires] the tasks in order to fulfill the license conditions:
\begin{itemize}
  
  
  \item \textbf{[mandatory:]} Ensure that the licensing elements -- esp.\ the
  copyright, patent or trademarks notices and all notices that refer to the
  license and to the disclaimer of warranties -- are retained in your package in
  the form you have received them. If you compile the binary from the sources,
  ensure that all the licensing elements are also incorporated into the package.
  
  \item \textbf{[mandatory:]} Give the recipient a copy of the EUPL 1.1
  license. If it is not already part of the binary package, add
  it\footnote{For implementing the handover of files correctly $\rightarrow$
  OSLiC, p. \pageref{DistributingFilesHint}}.
 
  \item \textbf{[mandatory:]} Create a \emph{modification text file}, if such a
  notice file still does not exist. \emph{Expand} the \emph{modification text
  file} by a description of your modifications.
  
  \item \textbf{[mandatory:]} Make the source code of the embedded library and
  the source code of your overarching program accessible via a repository under
  your own control \footnote{Formally, the EUPL-1.1 is only a license with weak
  copyleft. But this is only a result of the allowance to relicense the software
  ($\rightarrow$ OSLiC, p.\ \pageref{sec:ProtectingPowerOfEupl}). So, as long as
  you do not relicense the embedded library with respect to the list of
  \enquote{compatible licenses accodring to article 5 EUPL} (\cite[cf.][\nopage
  wp §5 and Appendix]{EuplLicense2007en}), you also have to publish the code of
  your overarching work.}: Push the source code package into a repository and
  make it downloadable via the internet. Integrate an easily to find description
  into the distribution package which explains how the code can be received from
  where. Ensure, that this repository is online for as long as you continue to
  distribute the software.
  
  \item \textbf{[mandatory:]} Insert a prominent hint to the download repository
  into your distribution and/or your additional material.
  
  \item \textbf{[mandatory:]} Execute the to-do list of use case EUPL-CA\footnote{
  Making the code accessible via a repository means distributing the software in
  the form of source code. Hence, you must also fulfill all tasks of the
  corresponding use case.}.
  
  \item \textbf{[mandatory:]} Organize your modifications of the embedded
  library in a way that they are covered by the existing EUGPL licensing
  statements. 
  
  \item \textbf{[mandatory:]} License your overarching program also under the
  EUPL 1.1.
  
  \item \textbf{[voluntary:]} Mark all modifications of source code of the
  embedded library (snimoli) thoroughly -- namely within the source code and
  including the date of the modification.

  \item \textbf{[voluntary:]} Let the documentation of your distribution and/or
  your additional material  also reproduce the content of the existing
  \emph{copyright notice text files}, a hint to the name of the used EUPL
  licensed component, a link to its homepage, and a link to the EUPL 1.1 license
  -- especially as subsection of your own copyright notice.
  
\end{itemize}

\item[prohibits] \ldots
\begin{itemize}
  \item to promote any of your services or products -- based on the this software
  -- by trade names, trademarks, service marks, or names linked to this EUPL
  software, except as required for unpartially describing the used software and
  reproducing the copyright notice.
\end{itemize}

\end{description}

\subsection{Discussions and Explanations}
\begin{itemize}
  
  \item The EUPL generally \enquote{[\ldots] does not grant permission to use
  the trade names, trademarks, service marks, or names of the Licensor , except
  as required for reasonable and customary use in describing the origin of the
  Work and reproducing the content of the copyright
  notice}\footcite[cf.][\nopage wp.\ §5]{EuplLicense2007en}. Therefore, the OSLiC
  genreally interdicts (EUPL-C1 - EUPL-CB) to promote any service or product --
  based on this software -- by such elements.

  \item The EUPL generally requires that \enquote{[\ldots] the Licensee shall
  keep intact all copyright, patent or trademarks notices and all notices that
  refer to the Licence and to the disclaimer of
  warranties}\footcite[cf.][\nopage wp.\ §5]{EuplLicense2007en}. In a very strict
  reading, the EUPL does not limit this requirement to the distribution of the
  software. But practically, it will be impossible to control the compliant use
  of the software in those cases (\emph{4yourself}) without 'switching' into the
  use case 'distribution'. Therefore the OSLiC only inserts this requirement as
  a mandatory clause for the \emph{2others} use cases (EUPL-C2 - EUPL-CB).
  
  \item The EUPL also requires to \enquote{[\ldots] include [\ldots] a copy of
  the (EUPL) Licence with every (distributed) copy of the Work
  [\ldots]}\footcite[cf.][\nopage wp.\ §5]{EuplLicense2007en}. Therefore, all
  \emph{2others} use cases contain the respective mandatory condition (EUPL-C2 -
  EUPL-CB).
  
  \item Additionally, the EUPL requires, that the \enquote{licensee} who
  distribiutes a modified work \enquote{[\ldots] must cause any Derivative Work
  to carry prominent notices stating that the Work has been modified and the
  date of modification}\footcite[cf.][\nopage wp.\ §5]{EuplLicense2007en}. Thus,
  the OSLiC integrates the mandatory requirement to generate (update) a
  respective notice file into all 'modification use cases and recommends to mark
  all modifications in the source code (EUPL-C6 - EUPL-CB)
  
  \item Furthermore, the EUPL requires that any distributor of the software
  \enquote{[\ldots] provide a machine-readable copy of the Source Code [\ldots]}
  by \enquote{[\ldots] (indicating) a repository where this Source will be
  easily and freely available for as long as the Licensee continues to
  distribute [\ldots] the Work}\footnote{\cite[cf.][\nopage wp.\
  §5]{EuplLicense2007en}. To be precise, the EUPL also allows to directly
  distribute the source code together with the binary packages
  (\cite[cf.][\nopage wp.\ §3]{EuplLicense2007en}). With respect to the OSLiC
  principle to offer only one reliable way, the OSLiC simplifies this option:
  It 'only' asks for the repository solution.}. Therefore the OSLiC inserts a
  respective requirement into the task list of all cases concerning a binary
  distribution  (EUPL-C3, EUPL-C7, EUPL-C9, EUPL-CB)
  
  \item Finally, the EUPL contains a \enquote{copyleft clause} stating that if a
  \enquote{[\ldots] Licensee distributes [\ldots] copies of the Original Works
  or Derivative Works based upon the Original Work, this Distribution [...] will
  be done under the terms of this (EUPL) Licence [\ldots]}. In all the use cases
  which do not concern the use of an embedded component (EUPL-C2 - EUPL-C9) this
  copyleft clause is already fulfilled by either distributing the modified
  sources themselves of by making them accessible via a repository. In those
  cases where the licensee distributes an overarching program which uses an EUPL
  licensed component (EUPL-CA - EUPL-CB) normally also the code of the overarching
  work must be distributed. Thus, with respect to the use case (EUPL-CA) this is
  already fulfilled by definition. So, the OSLiC only mentions this default view
  in the case EUPL-CB and therefore implicitly evokes a strong copyleft effect of
  the EUPL\footnote{Formally, the EUPL-1.1 is a license with weak copyleft. But
  this is only a result of the allowance to relicense the software
  ($\rightarrow$ OSLiC, p.\ \pageref{sec:ProtectingPowerOfEupl}). So, as long as
  you do not relicense the embedded library with respect to the list of
  \enquote{compatible licenses according to article 5 EUPL}, you also have to
  publish the code of your overarching work. Therefore, you can only avoid this
  consequence by relicensing the embedded component by one of the compatible
  licenses with a weak copyleft listed in the EUPL appendix (\cite[cf.][\nopage
  wp §5 and Appendix]{EuplLicense2007en})}.
  

\end{itemize}








%\bibliography{../../../bibfiles/oscResourcesEn}
