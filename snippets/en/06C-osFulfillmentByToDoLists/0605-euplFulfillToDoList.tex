% Telekom osCompendium 'for being included' snippet template
%
% (c) Karsten Reincke, Deutsche Telekom AG, Darmstadt 2011
%
% This LaTeX-File is licensed under the Creative Commons Attribution-ShareAlike
% 3.0 Germany License (http://creativecommons.org/licenses/by-sa/3.0/de/): Feel
% free 'to share (to copy, distribute and transmit)' or 'to remix (to adapt)'
% it, if you '... distribute the resulting work under the same or similar
% license to this one' and if you respect how 'you must attribute the work in
% the manner specified by the author ...':
%
% In an internet based reuse please link the reused parts to www.telekom.com and
% mention the original authors and Deutsche Telekom AG in a suitable manner. In
% a paper-like reuse please insert a short hint to www.telekom.com and to the
% original authors and Deutsche Telekom AG into your preface. For normal
% quotations please use the scientific standard to cite.
%
% [ Framework derived from 'mind your Scholar Research Framework' 
%   mycsrf (c) K. Reincke 2012 CC BY 3.0  http://mycsrf.fodina.de/ ]
%


%% use all entries of the bibliography
%\nocite{*}

\section{EUPL licensed software}

The European Union Public License does explicitly distinguish the distribution
of the source code from that of the binaries: In the chapter
\enquote{Communication of the Source Code}, it allows to \enquote{provide the
Work either in its Source Code form, or as Executable
Code}\footcite[cf.][\nopage wp\ §3]{EuplLicense2007en}. But if any pieces of
EUPL licensed software is distributed as binary package, then the license
additionally requires that the distributor either \enquote{[\ldots] provides a
machine-readable copy of the Source Code [\ldots]} directly together with the
binaries\footcite[cf.][\nopage wp\ §5]{EuplLicense2007en} or that he
\enquote{[\ldots] indicates [\ldots] a repository where the Source Code is
easily and freely accessible for as long as the Licensor continues to distribute
[\ldots] the Work}\footnote{\cite[cf.][\nopage wp\ §3.]{EuplLicense2007en}} --
namely regardless whether the software has been modified or not. The other
\enquote{obligations of the licensee} refer to both forms\footcite[cf.][\nopage
wp\ §5]{EuplLicense2007en}. Thus, for getting simply processable task list, the
following EUPL specific open source use case structure\footnote{For details of
the general OSUC finder $\rightarrow$ OSLiC, pp.\ \pageref{OsucTokens} and
\pageref{OsucDefinitionTree}} can be used:
 
\tikzstyle{nodv} = [font=\small, ellipse, draw, fill=gray!10, 
    text width=2cm, text centered, minimum height=2em]

\tikzstyle{nods} = [font=\footnotesize, rectangle, draw, fill=gray!20, 
    text width=1.2cm, text centered, rounded corners, minimum height=3em]

\tikzstyle{nodb} = [font=\footnotesize, rectangle, draw, fill=gray!20, 
    text width=2.2cm, text centered, rounded corners, minimum height=3em]
    
\tikzstyle{leaf} = [font=\tiny, rectangle, draw, fill=gray!30, 
    text width=1.2cm, text centered, minimum height=6em]

\tikzstyle{edge} = [draw, -latex']

\begin{tikzpicture}[]

\node[nodv] (l71) at (4,10) {EUPL};

\node[nodb] (l61) at (0,8.6) {\textit{recipient:} \\ \textbf{4yourself}};
\node[nodb] (l62) at (6.5,8.6) {\textit{recipient:} \\ \textbf{4others}};

\node[nodb] (l51) at (2.5,7) {\textit{state:} \\ \textbf{unmodified}};
\node[nodb] (l52) at (9.3,7) {\textit{state:} \\ \textbf{modified}};

\node[nods] (l41) at (1.8,5.4) {\textit{form:} \textbf{source}};
\node[nods] (l42) at (3.6,5.4) {\textit{form:} \textbf{binary}};
\node[nodb] (l43) at (6.5,5.4) {\textit{type:} \\ \textbf{proapse}};
\node[nodb] (l44) at (12,5.4) {\textit{type:} \\ \textbf{snimoli}};


\node[nods] (l31) at (5.4,3.8) {\textit{form:} \textbf{source}};
\node[nods] (l32) at (7.2,3.8) {\textit{form:} \textbf{binary}};
\node[nodb] (l33) at (10,3.8) {\textit{context:} \\ \textbf{independent}};
\node[nodb] (l34) at (13.5,3.8) {\textit{context:} \\ \textbf{embedded}};

\node[nods] (l21) at (9,2.2) {\textit{form:} \textbf{source}};
\node[nods] (l22) at (10.8,2.2) {\textit{form:} \textbf{binary}};
\node[nods] (l23) at (12.6,2.2) {\textit{form:} \textbf{source}};
\node[nods] (l24) at (14.4,2.2) {\textit{form:} \textbf{binary}};

\node[leaf] (l11) at (0,0) {\textbf{EUPL-1} \textit{using software only
for yourself}};

\node[leaf] (l12) at (1.8,0) { \textbf{EUPL-2} \textit{ distributing unmodified
software as sources}};

\node[leaf] (l13) at (3.6,0) { \textbf{EUPL-3}  \textit{ distributing unmodified
software as binaries}};

\node[leaf] (l14) at (5.4,0) { \textbf{EUPL-4}  \textit{ distributing modified
program as sources}};

\node[leaf] (l15) at (7.2,0) { \textbf{EUPL-5}  \textit{ distributing modified
program as binaries}};

\node[leaf] (l16) at (9,0) { \textbf{EUPL-6}  \textit{ distributing modified
library as independent sources}};

\node[leaf] (l17) at (10.8,0) { \textbf{EUPL-7} \textit{distributing modified
library as independent binaries}};

\node[leaf] (l18) at (12.6,0) { \textbf{EUPL-8}  \textit{distributing
modified library as embedded sources}};

\node[leaf] (l19) at (14.4,0) { \textbf{EUPL-9}  \textit{ distributing modified
library as embedded binaries}};


\path [edge] (l71) -- (l61);
\path [edge] (l71) -- (l62);
\path [edge] (l61) -- (l11);
\path [edge] (l62) -- (l51);
\path [edge] (l62) -- (l52);
\path [edge] (l51) -- (l41);
\path [edge] (l51) -- (l42);
\path [edge] (l52) -- (l43);
\path [edge] (l52) -- (l44);
\path [edge] (l41) -- (l12);
\path [edge] (l42) -- (l13);
\path [edge] (l43) -- (l31);
\path [edge] (l43) -- (l32);
\path [edge] (l44) -- (l33);
\path [edge] (l44) -- (l34);
\path [edge] (l31) -- (l14);
\path [edge] (l32) -- (l15);
\path [edge] (l33) -- (l21);
\path [edge] (l33) -- (l22);
\path [edge] (l34) -- (l23);
\path [edge] (l34) -- (l24);
\path [edge] (l21) -- (l16);
\path [edge] (l22) -- (l17);
\path [edge] (l23) -- (l18);
\path [edge] (l24) -- (l19);

\end{tikzpicture}


\subsection{EUPL-1: Using the software only for yourself}
\label{OSUC-01-EUPL} \label{OSUC-03-EUPL} 
\label{OSUC-06-EUPL} \label{OSUC-09-EUPL}

\begin{description}

\item[means] that you are going to use a received EUPL licensed software only
for yourself and that you do not hand it over to any 3rd party in any sense.

\item[covers] OSUC-01, OSUC-03, OSUC-06, and OSUC-09\footnote{For details 
$\rightarrow$ OSLiC, pp.\ \pageref{OSUC-01-DEF} - \pageref{OSUC-09-DEF}}

\item[requires] no tasks in order to fulfill the conditions of the EUPL 1.1
license with respect to this use case:
  \begin{itemize}
    \item You are allowed to use any kind of EUPL software in any sense and in
    any context without any obligations as long as you do not give the software
    to 3rd parties.
  \end{itemize}
  
\item[prohibits] \ldots
\begin{itemize}
  \item to promote any of your services or products – based on the this software
  – by trade names, trademarks, service marks, or names linked to this EUPL
  software, except as required for unpartially describing the used software and
  reproducing the copyright notice.
\end{itemize}.

\end{description}

\subsection{EUPL-2: Passing the unmodified software as source code}
\label{OSUC-02-EUPL} \label{OSUC-05-EUPL} \label{OSUC-07-EUPL} 

\begin{description}

\item[means] that you are going to distribute an unmodified version of the
received EUPL software to 3rd parties in the form of a set of source code files or
an integrated source code package\footnote{In this case it doesn't matter
whether you  distribute a program, an application, a server, a snippet, a
module, a library, or a plugin as an independent or an embedded unit}

\item[covers] OSUC-02, OSUC-05, OSUC-07\footnote{For details $\rightarrow$ OSLiC, pp.\ 
\pageref{OSUC-02-DEF} - \pageref{OSUC-07-DEF}}

\item[requires] the following tasks in order to fulfill the license conditions:
\begin{itemize}
  
  \item \textbf{[mandatory:]} Ensure that the licensing elements -- esp.\ the
  copyright, patent or trademarks notices and all notices that refer to the
  Licence and to the disclaimer of warranties -- are retained in your package in
  the form you have received them. If you compile the binary from the sources,
  ensure that all the licensing elements are also incorporated into the package.
  
  \item \textbf{[mandatory:]} Give the recipient a copy of the EUPL 1.1
  license. If it is not already present in the software package, add
  it\footnote{For implementing the handover of files correctly $\rightarrow$
  OSLiC, p. \pageref{DistributingFilesHint}}.
  
  \item \textbf{[voluntary:]} Let the documentation of your distribution and/or
  your additional material also reproduce the content of the \emph{notice text
  file}, a hint to the software name, a link to its homepage, and a link to the
  EUPL 1.1 license.
\end{itemize}

\item[prohibits] \ldots
\begin{itemize}
  \item to promote any of your services or products – based on the this software
  – by trade names, trademarks, service marks, or names linked to this EUPL
  software, except as required for unpartially describing the used software and
  reproducing the copyright notice.
\end{itemize}.

\end{description}


\subsection{EUPL-3: Passing the unmodified software as binaries} 

\begin{description}
\item[means] that you are going to distribute an unmodified version of the
received EUPL software to 3rd parties in the form of a set of binary files or an
integrated bi\-na\-ry package\footnote{In this case it doesn't matter
whether you  distribute a program, an application, a server, a snippet, a
module, a library, or a plugin as an independent or an embedded unit}

\item[covers] OSUC-02, OSUC-05, OSUC-07\footnote{For details $\rightarrow$ OSLiC, pp.\
\pageref{OSUC-02-DEF} - \pageref{OSUC-07-DEF}}

\item[requires] the following tasks in order to fulfill the license conditions:
\begin{itemize}
  
  \item \textbf{[mandatory:]} Ensure that the licensing elements -- esp.\ the
  copyright, patent or trademarks notices and all notices that refer to the
  Licence and to the disclaimer of warranties -- are retained in your package in
  the form you have received them. If you compile the binary from the sources,
  ensure that all the licensing elements are also incorporated into the package.
  
  \item \textbf{[mandatory:]} Give the recipient a copy of the EUPL 1.1
  license. If it is not already present in the binary package, add
  it\footnote{For implementing the handover of files correctly $\rightarrow$
  OSLiC, p. \pageref{DistributingFilesHint}}.

  \item \textbf{[mandatory:]} Make the source code of the distributed software
  accessible via a repository (even if you do not modified it): Push the source
  code package into the repository, make it downloadable via the internet, and
  integrate easily to find descriptions into the distribution package how to the
  code can be received from where. Ensure, that this repository is online for as
  long as you continue to distribute the software.
  
  \item \textbf{[voluntary:]} Let the documentation of your distribution and/or
  your additional material also reproduce the content of the \emph{notice text
  file}, a hint to the software name, a link to its homepage, and a link to the
  EUPL 1.1 license -- especially as subsection of your own copyright notice.
\end{itemize}

\item[prohibits] \ldots
\begin{itemize}
  \item to promote any of your services or products – based on the this software
  – by trade names, trademarks, service marks, or names linked to this EUPL
  software, except as required for unpartially describing the used software and
  reproducing the copyright notice.
\end{itemize}.

\end{description}

\subsection{EUPL-4: Passing a modified program as source code}
\label{OSUC-04-EUPL} 

\begin{description}
\item[means] that you are going to distribute a modified version of the received
EUPL licensed program, application, or server (proapse) to 3rd parties in form
of a set of source code files or an integrated source code package.
\item[covers] OSUC-04\footnote{For details $\rightarrow$ OSLiC, pp.\ \pageref{OSUC-04-DEF}}
\item[requires] the tasks in order to fulfill the license conditions:
\begin{itemize}
  
  \item \textbf{[mandatory:]} Ensure that the licensing elements -- esp.\ the
  copyright, patent or trademarks notices and all notices that refer to the
  Licence and to the disclaimer of warranties -- are retained in your package in
  the form you have received them.
  
  \item \textbf{[mandatory:]} Give the recipient a copy of the EUPL 1.1
  license. If it is not already present in the software package, add
  it\footnote{For implementing the handover of files correctly $\rightarrow$
  OSLiC, p. \pageref{DistributingFilesHint}}.

  \item \textbf{[mandatory:]} Create a \emph{modification text file}, if such a
  notice file still does not exist. \emph{Expand} the \emph{modification text
  file} by a description of your modifications.
    
  \item \textbf{[voluntary:]} Mark all modifications of source code of the
  embedded libary\footnote{or snippet, or module, or plugin} thoroughly.
   
  \item \textbf{[voluntary:]} Let the documentation of your distribution and/or
  your additional material also reproduce the content of the \emph{notice text
  file}, a hint to the software name, a link to its homepage, and a link to the
  EUPL 1.1 license.
  
 \end{itemize}
 
\item[prohibits] \ldots
\begin{itemize}
  \item to promote any of your services or products – based on the this software
  – by trade names, trademarks, service marks, or names linked to this EUPL
  software, except as required for unpartially describing the used software and
  reproducing the copyright notice.
\end{itemize}.

\end{description}

\subsection{EUPL-5: Passing a modified program as binary}

\begin{description}
\item[means] that you are going to distribute a modified version of the received
EUPL licensed pro\-gram, application, or server (proapse) to 3rd parties in
the form of a set of binary files or an integrated binary package.
\item[covers] OSUC-04\footnote{For details $\rightarrow$ OSLiC, pp.\ \pageref{OSUC-04-DEF}}
\item[requires] the tasks in order to fulfill the license conditions:
\begin{itemize}

  \item \textbf{[mandatory:]} Ensure that the licensing elements -- esp.\ the
  copyright, patent or trademarks notices and all notices that refer to the
  Licence and to the disclaimer of warranties -- are retained in your package in
  the form you have received them. If you compile the binary from the sources,
  ensure that all the licensing elements are also incorporated into the package.

 \item \textbf{[mandatory:]} Give the recipient a copy of the EUPL 1.1
  license. If it is not already present in the binary package, add
  it\footnote{For implementing the handover of files correctly $\rightarrow$
  OSLiC, p. \pageref{DistributingFilesHint}}.
  
  \item \textbf{[mandatory:]} Create a \emph{modification text file}, if such a
  notice file still does not exist. \emph{Expand} the \emph{modification text
  file} by a description of your modifications.

  \item \textbf{[mandatory:]} Make the source code of the distributed software
  accessible via a repository (although you modified it): Push the source code
  package into the repository, make it downloadable via the internet, and
  integrate easily to find descriptions into the distribution package how to the
  code can be received from where. Ensure, that this repository is online for as
  long as you continue to distribute the software.
    
  \item \textbf{[voluntary:]} Mark all modifications of source code of the
  embedded libary\footnote{or snippet, or module, or plugin} thoroughly.
 
  \item \textbf{[voluntary:]} Let the documentation of your distribution and/or
  your additional material also reproduce the content of the \emph{notice text
  file}, a hint to the software name, a link to its homepage, and a link to the
  EUPL 1.1 license -- especially as a subsection of your own copyright notice.

\end{itemize}

\item[prohibits] \ldots
\begin{itemize}
  \item to promote any of your services or products – based on the this software
  – by trade names, trademarks, service marks, or names linked to this EUPL
  software, except as required for unpartially describing the used software and
  reproducing the copyright notice.
\end{itemize}.

\end{description}

\subsection{EUPL-6: Passing a modified library as independent source code}
\label{OSUC-08-EUPL}

\begin{description}
\item[means] that you are going to distribute a modified version of the received
EUPL licensed code snippet, module, library, or plugin (snimoli) to 3rd
parties in the form of a set of source code files or an integrated source code
package, but without embedding it into another larger software unit.
\item[covers] OSUC-08\footnote{For details $\rightarrow$ OSLiC, pp.\ \pageref{OSUC-08-DEF}}
\item[requires] the tasks in order to fulfill the license conditions:
\begin{itemize}
  
  \item \textbf{[mandatory:]} Ensure that the licensing elements -- esp.\ the
  copyright, patent or trademarks notices and all notices that refer to the
  Licence and to the disclaimer of warranties -- are retained in your package in
  the form you have received them.
  
  \item \textbf{[mandatory:]} Give the recipient a copy of the EUPL 1.1
  license. If it is not already present in the software package, add
  it\footnote{For implementing the handover of files correctly $\rightarrow$
  OSLiC, p. \pageref{DistributingFilesHint}}.

  \item \textbf{[mandatory:]} Create a \emph{modification text file}, if such a
  notice file still does not exist. \emph{Expand} the \emph{modification text
  file} by a description of your modifications.
  
  \item \textbf{[voluntary:]} Mark all modifications of source code of the
  embedded libary\footnote{or snippet, or module, or plugin} thoroughly.
   
  \item \textbf{[voluntary:]} Let the documentation of your distribution and/or
  your additional material also reproduce the content of the \emph{notice text
  file}, a hint to the software name, a link to its homepage, and a link to the
  EUPL 1.1 license.

\end{itemize}

\item[prohibits] \ldots
\begin{itemize}
  \item to promote any of your services or products – based on the this software
  – by trade names, trademarks, service marks, or names linked to this EUPL
  software, except as required for unpartially describing the used software and
  reproducing the copyright notice.
\end{itemize}.

\end{description}


\subsection{EUPL-7: Passing a modified library as independent binary}

\begin{description}
\item[means] that you are going to distribute a modified version of the received
Appache licensed code snippet, module, library, or plugin (snimoli) to 3rd
parties in the form of a set of binary files or an integrated binary package but
without embedding it into another larger software unit.
\item[covers] OSUC-08\footnote{For details $\rightarrow$ OSLiC, pp.\ \pageref{OSUC-08-DEF}}
\item[requires] the tasks in order to fulfill the license conditions:
\begin{itemize}

  \item \textbf{[mandatory:]} Ensure that the licensing elements -- esp.\ the
  copyright, patent or trademarks notices and all notices that refer to the
  Licence and to the disclaimer of warranties -- are retained in your package in
  the form you have received them. If you compile the binary from the sources,
  ensure that all the licensing elements are also incorporated into the package.
  
 \item \textbf{[mandatory:]} Give the recipient a copy of the EUPL 1.1
  license. If it is not already present in the binary package, add
  it\footnote{For implementing the handover of files correctly $\rightarrow$
  OSLiC, p. \pageref{DistributingFilesHint}}.

  \item \textbf{[mandatory:]} Create a \emph{modification text file}, if such a
  notice file still does not exist. \emph{Expand} the \emph{modification text
  file} by a description of your modifications.

  \item \textbf{[mandatory:]} Make the source code of the distributed software
  accessible via a repository (although you modified it): Push the source code
  package into the repository, make it downloadable via the internet, and
  integrate easily to find descriptions into the distribution package how to the
  code can be received from where. Ensure, that this repository is online for as
  long as you continue to distribute the software.
    
  \item \textbf{[voluntary:]} Mark all modifications of source code of the
  embedded libary\footnote{or snippet, or module, or plugin} thoroughly.
 
  \item \textbf{[voluntary:]} Let the documentation of your distribution and/or
  your additional material also reproduce the content of the \emph{notice text
  file}, a hint to the software name, a link to its homepage, and a link to the
  EUPL 1.1 license -- especially as a subsection of your own copyright notice.
  
\end{itemize}

\item[prohibits] \ldots
\begin{itemize}
  \item to promote any of your services or products – based on the this software
  – by trade names, trademarks, service marks, or names linked to this EUPL
  software, except as required for unpartially describing the used software and
  reproducing the copyright notice.
\end{itemize}.

\end{description}

\subsection{EUPL-8: Passing a modified library as embedded source code}
\label{OSUC-10-EUPL}

\begin{description}
\item[means] that you are going to distribute a modified version of the received
EUPL licensed code snippet, module, library, or plugin (snimoli) to 3rd
parties in the form of a set of source code files or an integrated source code
package together with another larger software unit which contains this code
snippet, module, library, or plugin as an embedded component.
\item[covers] OSUC-10\footnote{For details $\rightarrow$ OSLiC, pp.\ \pageref{OSUC-10-DEF}}
\item[requires] the tasks in order to fulfill the license conditions:
\begin{itemize}

  \item \textbf{[mandatory:]} Ensure that the licensing elements -- esp.\ the
  copyright, patent or trademarks notices and all notices that refer to the
  Licence and to the disclaimer of warranties -- are retained in your package in
  the form you have received them.

  \item \textbf{[mandatory:]} Give the recipient a copy of the EUPL 1.1
  license. If it is not already present in the software package, add
  it\footnote{For implementing the handover of files correctly $\rightarrow$
  OSLiC, p. \pageref{DistributingFilesHint}}.

  \item \textbf{[mandatory:]} Create a \emph{modification text file}, if such a
  notice file still does not exist. \emph{Expand} the \emph{modification text
  file} by a description of your modifications.
  
  \item \textbf{[voluntary:]} Mark all modifications of source code of the
  embedded libary\footnote{or snippet, or module, or plugin} thoroughly.
  
  \item \textbf{[voluntary:]} Let the documentation of your distribution and/or
  your additional material also reproduce the content of the \emph{notice text
  file}, a hint to the name of the used EUPL licensed component, a link to its
  homepage, and a link to the EUPL 1.1 license -- especially as subsection of
  your own copyright notice.

  \item \textbf{[voluntary:]} Arrange your source code distribution so that the
  integrated EUPL license and the \emph{notice text file} clearly refer only
  to the embedded library and do not disturb the licensing of your own
  overarching work. It's a good tradition to keep the embedded components like
  libraries, modules, snippets, or plugins in specific directory which contains
  also all additional licensing elements.
 
\end{itemize}

\item[prohibits] \ldots
\begin{itemize}
  \item to promote any of your services or products – based on the this software
  – by trade names, trademarks, service marks, or names linked to this EUPL
  software, except as required for unpartially describing the used software and
  reproducing the copyright notice.
\end{itemize}.

\end{description}


\subsection{EUPL-9: Passing a modified library as embedded binary}

\begin{description}
\item[means] that you are going to distribute a modified version of the received
EUPL licensed code snippet, module, library, or plugin to 3rd parties in form
of a set of binary files or an integrated binary package together with another
larger software unit which contains this code snippet, module, library, or
plugin as an embedded component.
\item[covers] OSUC-10\footnote{For details $\rightarrow$ OSLiC, pp.\ \pageref{OSUC-10-DEF}}
\item[requires] the tasks in order to fulfill the license conditions:
\begin{itemize}
  
  
  \item \textbf{[mandatory:]} Ensure that the licensing elements -- esp.\ the
  copyright, patent or trademarks notices and all notices that refer to the
  Licence and to the disclaimer of warranties -- are retained in your package in
  the form you have received them. If you compile the binary from the sources,
  ensure that all the licensing elements are also incorporated into the package.
  
  \item \textbf{[mandatory:]} Give the recipient a copy of the EUPL 1.1
  license. If it is not already present in the binary package, add
  it\footnote{For implementing the handover of files correctly $\rightarrow$
  OSLiC, p. \pageref{DistributingFilesHint}}.
 
  \item \textbf{[mandatory:]} Create a \emph{modification text file}, if such a
  notice file still does not exist. \emph{Expand} the \emph{modification text
  file} by a description of your modifications.
  
  \item \textbf{[mandatory:]} Make the source code of the distributed software
  accessible via a repository: Push the source code package into the repository,
  make it downloadable via the internet, and integrate easily to find
  descriptions into the distribution package how to the code can be received
  from where.
  
  \item \textbf{[mandatory:]} Make the source code of the embedded library and
  th source code of your overarching program accessible via a
  repository\footnote{Formally, the EUPL-1.1 is only a license with weak
  copyleft. But this is only a result of the allowance to relicense the software
  ($\rightarrow$ OSLiC, p.\ \pageref{sec:ProtectingPowerOfEupl}). So, as long as
  you do not relicense the embedded library with respect to the list of
  \enquote{compatible licenses accodring to article 5 EUPL} (\cite[cf.][\nopage
  wp §5 and Appendix]{EuplLicense2007en}), you also have to publish the code of
  your overarching work.}: Push the source code package into the repository,
  make it downloadable via the internet, and integrate easily to find
  descriptions into the distribution package how to the code can be received
  from where. Ensure, that this repository is online for as long as you continue
  to distribute the software.
 
  \item \textbf{[voluntary:]} Mark all modifications of source code of the
  embedded libary\footnote{or snippet, or module, or plugin} thoroughly.

  \item \textbf{[voluntary:]} Let the documentation of your distribution and/or
  your additional material also reproduce the content of the \emph{notice text
  file}, a hint to the name of the used EUPL licensed component, a link to its
  homepage, and a link to the EUPL 1.1 license -- especially as subsection of
  your own copyright notice.
  
 \item \textbf{[voluntary:]} Arrange your binary distribution so that the
  integrated EUPL license and the \emph{notice text file} clearly refer only
  to the embedded library and do not disturb the licensing of your own
  overarching work. It's a good tradition to keep the librabries, modules,
  snippet, or plugins in specific directiers which contain also all licensing
  elements.
  
\end{itemize}

\item[prohibits] \ldots
\begin{itemize}
  \item to promote any of your services or products – based on the this software
  – by trade names, trademarks, service marks, or names linked to this EUPL
  software, except as required for unpartially describing the used software and
  reproducing the copyright notice.
\end{itemize}.

\end{description}

\subsection{Discussions and Explanations}
\begin{itemize}
  
  \item The EUPL generally \enquote{[\ldots] does not grant permission to use
  the trade names, trademarks, service marks, or names of the Licensor , except
  as required for reasonable and customary use in describing the orig in of the
  Work and reproducing the content of the copyright
  notice}\footcite[cf.][\nopage wp\ §5]{EuplLicense2007en}. Therefore, the OSLiC
  genreally interdicts (EUPL-1 - EUPL-9) to promote any service or product --
  based on this software -- by such elements.

  \item The EUPL generally requires that \enquote{[\ldots] the Licensee shall
  keep intact all copyright, patent or trademarks notices and all notices that
  refer to the Licence and to the disclaimer of
  warranties}\footcite[cf.][\nopage wp\ §5]{EuplLicense2007en}. In a very strict
  reading, the EUPL does not limit this requirement to the cases of
  'distributing' the software (\emph{4others}). But -- in the cases of not
  distributing the software -- the control of a license compliant fulfillment
  could be difficult without automatically switching into the use case class
  'distributing'. Therefore the OSLiC inserts this requirement as a mandatory
  clause for the 4others use cases (EUPL-2 - EUPL-9).
  
  \item The EUPL also requires to \enquote{[\ldots] include [\ldots] a copy of
  the (EUPL) Licence with every (distributed) copy of the Work
  [\ldots]}\footcite[cf.][\nopage wp\ §5]{EuplLicense2007en}. Therefore, all
  \emph{4others} use cases contain the respective mandatory condition (EUPL-2 -
  EUPL-9).
  
  \item Additionally, the EUPL requires, that the \enquote{licensee} who
  distribiutes a modified work \enquote{[\ldots] must cause any Derivative Work
  to carry prominent notices stating that the Work has been modified and the
  date of modification}\footcite[cf.][\nopage wp\ §5]{EuplLicense2007en}. Thus,
  the OSLiC integrates the mandatory requirement to generate (update) a
  respective notice file into all 'modification use cases and recommends to mark
  all modifications in the source code (EUPL-4 - EUPL-9)
  
  \item Furthermore, the EUPL requires that any distributor of the software
  \enquote{[\ldots] provide a machine-readable copy of the Source Code [\ldots]}
  by \enquote{[\ldots] (indicating) a repository where this Source will be
  easily and freely available for as long as the Licensee continues to
  distribute [\ldots] the Work}\footnote{\cite[cf.][\nopage wp\
  §5]{EuplLicense2007en}. To be precise, the EUPL also allows to directly
  distribute the source code together with the binary packages
  (\cite[cf.][\nopage wp\ §3]{EuplLicense2007en}). With respect to the OSLiC
  principle to offer only one reliable way, the OSLiC simplifies this option:
  It 'only' asks for the repository solution.}. Therefore the OSLiC inserts a
  respective requirement into the task list of all cases concerning a binary
  distribution  (EUPL-3, EUPL-5, EUPL-7, EUPL-9)
  
  \item Finally, the EUPL contains a \enquote{copyleft clause} stating that if a
  \enquote{[\ldots] Licensee distributes [\ldots] copies of the Original Works
  or Derivative Works based upon the Original Work, this Distribution [...] will
  be done under the terms of this (EUPL) Licence [\ldots]}. In all use cases
  which do not concern the use of an embedded component (EUPL-2 - EUPL-7) this
  copyleft clause is already fulfilled by either distributing the modified
  sources themselves of by making them accissble via a repository. In those
  cases where the licensee distributes an overarching program which uses an EUPL
  licensed component (EUPL-8 - EUPL-9) normally also the code of the overarching
  work must be distributed. In case of distributing the source code of the
  embedded component and the overarching work (EUPL-8) this is already
  implicitly fulfilled. So, the OSLiC only mentions this default view in the
  case EUPL-9 and therefore evokes a strong copyleft effect of the
  EUPL\footnote{Formally, the EUPL-1.1 is a license with weak copyleft. But this
  is only a result of the allowance to relicense the software ($\rightarrow$
  OSLiC, p.\ \pageref{sec:ProtectingPowerOfEupl}). So, as long as you do not
  relicense the embedded library with respect to the list of \enquote{compatible
  licenses accoring to article 5 EUPL}, you also have to publish the code of
  your overarching work. Therefore, you can only avoid this consequence by
  relicensing the embedded component by one of the compatible licenses with weak
  copyleft protection listed in the EUPL appendix and its list of
  \enquote{compatible licenses according to article 5 EUPL} (\cite[cf.][\nopage
  wp §5 and Appendix]{EuplLicense2007en})}.
  

\end{itemize}








%\bibliography{../../../bibfiles/oscResourcesEn}
