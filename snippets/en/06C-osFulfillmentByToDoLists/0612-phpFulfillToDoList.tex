% Telekom osCompendium 'for being included' snippet template
%
% (c) Karsten Reincke, Deutsche Telekom AG, Darmstadt 2011
%
% This LaTeX-File is licensed under the Creative Commons Attribution-ShareAlike
% 3.0 Germany License (http://creativecommons.org/licenses/by-sa/3.0/de/): Feel
% free 'to share (to copy, distribute and transmit)' or 'to remix (to adapt)'
% it, if you '... distribute the resulting work under the same or similar
% license to this one' and if you respect how 'you must attribute the work in
% the manner specified by the author ...':
%
% In an internet based reuse please link the reused parts to www.telekom.com and
% mention the original authors and Deutsche Telekom AG in a suitable manner. In
% a paper-like reuse please insert a short hint to www.telekom.com and to the
% original authors and Deutsche Telekom AG into your preface. For normal
% quotations please use the scientific standard to cite.
%
% [ Framework derived from 'mind your Scholar Research Framework' 
%   mycsrf (c) K. Reincke 2012 CC BY 3.0  http://mycsrf.fodina.de/ ]
%


%% use all entries of the bibliography
%\nocite{*}

\section{PHP licensed software}

The PHP-3.0 license contains a shade more conditions than the MIT license and
additionally distinguishes the \enquote{redistribution of source
code}\footcite[cf.][wp. §1]{Php30OsiLicense2013a} from the
\enquote{redistribution in binary form}\footcite[cf.][wp.
§2]{Php30OsiLicense2013a} . Nevertheless, the PHP license focusses only on the
redistribution or - as we are going to say in the OSLiC - \emph{the 2others use
cases}. Thus, the PHP specific finder can be simplified:

\tikzstyle{nodv} = [font=\small, ellipse, draw, fill=gray!10, 
    text width=2cm, text centered, minimum height=2em]


\tikzstyle{nods} = [font=\footnotesize, rectangle, draw, fill=gray!20, 
    text width=1.2cm, text centered, rounded corners, minimum height=3em]

\tikzstyle{nodb} = [font=\footnotesize, rectangle, draw, fill=gray!20, 
    text width=2.2cm, text centered, rounded corners, minimum height=3em]
    
\tikzstyle{leaf} = [font=\tiny, rectangle, draw, fill=gray!30, 
    text width=1.2cm, text centered, minimum height=6em]

\tikzstyle{edge} = [draw, -latex']

\begin{tikzpicture}[]

\node[nodv] (l71) at (3.5,10) {PHP};

\node[nodb] (l61) at (0,8.6) {\textit{recipient:} \\ \textbf{4yourself}};
\node[nodb] (l62) at (6.5,8.6) {\textit{recipient:} \\ \textbf{2others}};

\node[nodb] (l51) at (2.5,7) {\textit{state:} \\ \textbf{unmodified}};
\node[nodb] (l52) at (9.3,7) {\textit{state:} \\ \textbf{modified}};

\node[nods] (l41) at (1.8,5.4) {\textit{form:} \textbf{source}};
\node[nods] (l42) at (3.6,5.4) {\textit{form:} \textbf{binary}};
\node[nodb] (l43) at (6.5,5.4) {\textit{type:} \\ \textbf{proapse}};
\node[nodb] (l44) at (12,5.4) {\textit{type:} \\ \textbf{snimoli}};


\node[nods] (l31) at (5.4,3.8) {\textit{form:} \textbf{source}};
\node[nods] (l32) at (7.2,3.8) {\textit{form:} \textbf{binary}};
\node[nodb] (l33) at (10,3.8) {\textit{context:} \\ \textbf{independent}};
\node[nodb] (l34) at (13.5,3.8) {\textit{context:} \\ \textbf{embedded}};

\node[nods] (l21) at (9,2.2) {\textit{form:} \textbf{source}};
\node[nods] (l22) at (10.8,2.2) {\textit{form:} \textbf{binary}};
\node[nods] (l23) at (12.6,2.2) {\textit{form:} \textbf{source}};
\node[nods] (l24) at (14.4,2.2) {\textit{form:} \textbf{binary}};

\node[leaf] (l11) at (0,0) {\textbf{PHP-C1} \textit{using software only
for yourself}};

\node[leaf] (l12) at (1.8,0) { \textbf{PHP-C2} \textit{ distributing unmodified
software as sources}};

\node[leaf] (l13) at (3.6,0) { \textbf{PHP-C3}  \textit{ distributing unmodified
software as binaries}};

\node[leaf] (l14) at (5.4,0) { \textbf{PHP-C4}  \textit{ distributing modified
program as sources}};

\node[leaf] (l15) at (7.2,0) { \textbf{PHP-C5}  \textit{ distributing modified
program as binaries}};

\node[leaf] (l16) at (9,0) { \textbf{PHP-C6}  \textit{ distributing modified
library as independent sources}};

\node[leaf] (l17) at (10.8,0) { \textbf{PHP-C7} \textit{distributing modified
library as independent binaries}};

\node[leaf] (l18) at (12.6,0) { \textbf{PHP-C8}  \textit{distributing
modified library as embedded sources}};

\node[leaf] (l19) at (14.4,0) { \textbf{PHP-C9}  \textit{ distributing modified
library as embedded binaries}};

\path [edge] (l71) -- (l61);
\path [edge] (l71) -- (l62);
\path [edge] (l61) -- (l11);
\path [edge] (l62) -- (l51);
\path [edge] (l62) -- (l52);
\path [edge] (l51) -- (l41);
\path [edge] (l51) -- (l42);
\path [edge] (l52) -- (l43);
\path [edge] (l52) -- (l44);
\path [edge] (l41) -- (l12);
\path [edge] (l42) -- (l13);
\path [edge] (l43) -- (l31);
\path [edge] (l43) -- (l32);
\path [edge] (l44) -- (l33);
\path [edge] (l44) -- (l34);
\path [edge] (l31) -- (l14);
\path [edge] (l32) -- (l15);
\path [edge] (l33) -- (l21);
\path [edge] (l33) -- (l22);
\path [edge] (l34) -- (l23);
\path [edge] (l34) -- (l24);
\path [edge] (l21) -- (l16);
\path [edge] (l22) -- (l17);
\path [edge] (l23) -- (l18);
\path [edge] (l24) -- (l19);

\end{tikzpicture}

\subsection{PHP-C1: Using the software only for yourself}
\label{OSUC-01-PHP} 
\label{OSUC-03-PHP} 
\label{OSUC-06-PHP}
\label{OSUC-09-PHP}
  
\begin{description}

\item[means] that you are going to use a received PHP software only for yourself
and that you do not hand it over to any 3rd party in any sense.

\item[covers] OSUC-01, OSUC-03, OSUC-06, and OSUC-09\footnote{For details $\rightarrow$ OSLiC, pp.\
  \pageref{OSUC-01-DEF} - \pageref{OSUC-09-DEF}}
  
\item[requires] no tasks in order to fulfill the conditions of the PHP license
with respect to this use case:
  \begin{itemize}
    \item You are allowed to use any kind of PHP software in any sense and in
    any context without any obligations as long as you do not give the software
    to 3rd parties.
  \end{itemize}

\item[prohibits] to endorse or promote any service you establish on the base of
this privately used software by the name 'PHP'.

\end{description}


\subsection{PHP-C2: Passing the unmodified software as source code}
\label{OSUC-02S-PHP} \label{OSUC-05S-PHP} \label{OSUC-07S-PHP} 

\begin{description}
\item[means] that you are going to distribute an unmodified version of the
received PHP software to 3rd parties -- in the form of source code files or as a
source code package. In this case it is not discriminating to distribute a
program, an application, a server, a snippet, a module, a library, or a plugin
as an independent or an embedded unit.

\item[covers] OSUC-02S, OSUC-05S, OSUC-07S\footnote{For details $\rightarrow$
OSLiC, pp.\ \pageref{OSUC-02S-DEF} - \pageref{OSUC-07S-DEF}}

\item[requires] the following tasks in order to fulfill the license conditions:
\begin{itemize}
  
  \item \textbf{[mandatory:]} Ensure that the complete PHP license -- esp.\
  the copyright notice, the PHP conditions, and the PHP disclaimer -- are
  retained in your package in the form you have received them.
  
  \item \textbf{[mandatory:]} Let the documentation of your distribution and/or
  your additional material also contain a line of acknowledgment in the form
  \begin{footnotesize}"This product includes PHP, freely available from
  $\langle$http://www.php.net/$\rangle$"\end{footnotesize}.
  
  \item \textbf{[voluntary:]} Let the documentation of your distribution and/or
  your additional material also contain the original copyright notice, the PHP
  conditions, and the PHP disclaimer.
  
\end{itemize}

\item[prohibits] to endorse or promote your product by mentioning PHP, esp. not
by make the string 'PHP' part of its name.

\end{description}

\subsection{PHP-C3: Passing the unmodified software as binary}
\label{OSUC-02B-PHP} \label{OSUC-05B-PHP} \label{OSUC-07B-PHP} 

\begin{description}

\item[means] that you are going to distribute an unmodified version of the PHP
received software to 3rd parties -- in the form of binary files or as a
bi\-na\-ry package. In this case it is not discriminating to distribute a
program, an application, a server, a snippet, a module, a library, or a plugin
as an independent or an embedded unit.

\item[covers] OSUC-02B, OSUC-05B, OSUC-07B\footnote{For details $\rightarrow$
OSLiC, pp.\ \pageref{OSUC-02B-DEF} - \pageref{OSUC-07B-DEF}}

\item[requires] the following tasks in order to fulfill the license conditions:
\begin{itemize}

  \item \textbf{[mandatory:]} Ensure that the complete PHP license -- esp.\ the
  copyright notice, the PHP conditions, and the PHP disclaimer -- are
  \textbf{reproduced} by your package in the form you have received
  them\footnote{Because you are distributing an unmodified binary, you could
  assume that the copright screens of the application do already what they have
  to do}. If you compile the binary file on the base of the source code package
  and if this compilation does not also generate and integrate the licensing
  files then create the copyright notice, the PHP conditions, and the PHP
  disclaimer according to the form of the source code package and insert these
  files into your distribution manually\footnote{For implementing the handover
  of files correctly $\rightarrow$ OSLiC, p.\ \pageref{DistributingFilesHint}}.
  
  \item \textbf{[mandatory:]} Let the documentation of your distribution and/or
  your additional material also contain a line of acknowledgment in the form
   \begin{footnotesize}"This product includes PHP, freely available from
  $\langle$http://www.php.net/$\rangle$"\end{footnotesize}.
    
  \item \textbf{[voluntary:]} Let the documentation of your distribution and/or
  your additional material also contain the original copyright notice, the PHP
  conditions, and the PHP disclaimer.

\end{itemize}

\item[prohibits] to endorse or promote your product by mentioning PHP, esp. not
by make the string 'PHP' part of its name.

\end{description}

\subsection{PHP-C4: Passing a modified program as source code}
\label{OSUC-04S-PHP}

\begin{description}
\item[means] that you are going to distribute a modified version of the received
PHP program, application, or server (proapse) to 3rd parties -- in the form of
source code files or as a source code package.
\item[covers] OSUC-04S\footnote{For details $\rightarrow$ OSLiC, pp.\
\pageref{OSUC-04S-DEF}}
\item[requires] the following tasks in order to fulfill the license conditions:
\begin{itemize}

  \item \textbf{[mandatory:]} Ensure that the complete PHP license -- esp.\
  the copyright notice, the PHP conditions, and the PHP disclaimer -- are
  retained in your package in the form you have received them.
  
  \item \textbf{[mandatory:]} Let the documentation of your distribution and/or
  your additional material also contain a line of acknowledgment in the form
  \begin{footnotesize}"This product includes PHP, freely available from
  $\langle$http://www.php.net/$\rangle$"\end{footnotesize}.
    
  \item \textbf{[voluntary:]} Let the documentation of your distribution and/or
  your additional material also contain the original copyright notice, the PHP
  conditions, and the PHP disclaimer.
     
  \item \textbf{[voluntary:]} It is a good practice of the open source
  community, to let the copyright notice which is shown by the running program
  also state that the program is licensed under the PHP license. Because you are
  already modifying the program you can also add such a hint if the presented
  original copyright notice lacks such a statement. If such a notice is missed
  in the copyright screen, consider, whether it is possible, to let it
  \emph{reproduce} the complete PHP license including the copyright notice, the
  PHP conditions, and the PHP disclaimer -- as it is required for binary
  distributions\footnote{Following distributors of compiled versions will
  appreciate your prepatory work.}
  
  \item \textbf{[voluntary:]} Mark your modifications in the source code.
  
  
\end{itemize}

\item[prohibits] to endorse or promote your product by mentioning PHP, esp. not
by make the string 'PHP' part of its name.

\end{description}

\subsection{PHP-C5: Passing a modified program as binary}
\label{OSUC-04B-PHP}

\begin{description}
\item[means] that you are going to distribute a modified version of the received
PHP pro\-gram, application, or server (proapse) to 3rd parties -- in the form of
binary files or as a binary package.
\item[covers] OSUC-04B\footnote{For details $\rightarrow$ OSLiC, pp.\
\pageref{OSUC-04B-DEF}}
\item[requires] the following tasks in order to fulfill the license conditions:
\begin{itemize}
  
  \item \textbf{[mandatory:]} Let the documentation of your distribution and/or
  your additional material also contain a line of acknowledgment in the form
  \begin{footnotesize}"This product includes PHP, freely available from
  $\langle$http://www.php.net/$\rangle$"\end{footnotesize}.
    
  \item \textbf{[mandatory:]} Let the documentation of your distribution and/or
  your additional material also contain the original copyright notice, the PHP
  conditions, and the PHP disclaimer.

  \item \textbf{[voluntary:]} Ensure that the complete PHP license -- esp.\ the
  copyright notice, the PHP conditions, and the PHP disclaimer -- are
  \textbf{reproduced} by your package. If such a notice is missed in the
  copyright screen, modify the screen so  that it \emph{reproduces} the complete
  PHP license including the copyright notice, the PHP conditions, and the PHP
  disclaimer.
  
  \item \textbf{[voluntary:]} Mark your modifications in the source code,
  even if you do not want to distribute the code.

\end{itemize}

\item[prohibits] to endorse or promote your product by mentioning PHP, esp. not
by make the string 'PHP' part of its name.

\end{description}

\subsection{PHP-C6: Passing a modified library as independent source code}
\label{OSUC-08S-PHP}
\begin{description}
\item[means] that you are going to distribute a modified version of the received
PHP code snippet, module, library, or plugin (snimoli) to 3rd parties -- in the
form of source code files or as a source code package, but without embedding it
into another larger software unit.
\item[covers] OSUC-08S\footnote{For details $\rightarrow$ OSLiC, pp.\
\pageref{OSUC-08S-DEF}}
\item[requires] the following tasks in order to fulfill the license conditions:
\begin{itemize}
  
  \item \textbf{[mandatory:]} Ensure that the complete PHP license -- esp.\
  the copyright notice, the PHP conditions, and the PHP disclaimer -- are
  retained in your package in the form you have received them.
  
  \item \textbf{[mandatory:]} Let the documentation of your distribution and/or
  your additional material also contain a line of acknowledgment in the form
  \begin{footnotesize}"This product includes PHP, freely available from
  $\langle$http://www.php.net/$\rangle$"\end{footnotesize}.
    
  \item \textbf{[voluntary:]} Let the documentation of your distribution and/or
  your additional material also contain the original copyright notice, the PHP
  conditions, and the PHP disclaimer.
     
  \item \textbf{[voluntary:]} Mark your modifications in the source code.
  
\end{itemize}

\item[prohibits] to endorse or promote your product by mentioning PHP, esp. not
by make the string 'PHP' part of its name.

\end{description}


\subsection{PHP-C7: Passing a modified library as independent binary}
\label{OSUC-08B-PHP}

\begin{description}

\item[means] that you are going to distribute a modified version of the received
PHP code snippet, module, library, or plugin (snimoli) to 3rd parties -- in the
form of binary files or as a binary package but without embedding it into
another larger software unit.

\item[covers] OSUC-08B\footnote{For details $\rightarrow$ OSLiC, pp.\
\pageref{OSUC-08B-DEF}}

\item[requires] the following tasks in order to fulfill the license conditions:
\begin{itemize}
  
  \item \textbf{[mandatory:]} Let the documentation of your distribution and/or
  your additional material also contain a line of acknowledgment in the form
  \begin{footnotesize}"This product includes PHP, freely available from
  $\langle$http://www.php.net/$\rangle$"\end{footnotesize}.
    
  \item \textbf{[mandatory:]} Let the documentation of your distribution and/or
  your additional material also contain the original copyright notice, the PHP
  conditions, and the PHP disclaimer.

  \item \textbf{[voluntary:]} Ensure that the complete PHP license -- esp.\ the
  copyright notice, the PHP conditions, and the PHP disclaimer -- are
  \textbf{reproduced} by your package -- as far as this can be done by the
  library itself.
  
  \item \textbf{[voluntary:]} Mark your modifications in the source code,
  even if you do want to distribute the code.

\end{itemize}

\item[prohibits] to endorse or promote your product by mentioning PHP, esp. not
by make the string 'PHP' part of its name.

\end{description}

\subsection{PHP-C8: Passing a modified library as embedded source code}
\label{OSUC-10S-PHP}
\begin{description}
\item[means] that you are going to distribute a modified version of the received
PHP code snippet, module, library, or plugin (snimoli) to 3rd parties -- in the
form of source code files or an integrated source code package together with
another larger software unit which contains this code snippet, module, library,
or plugin as an embedded component.
\item[covers] OSUC-10S\footnote{For details $\rightarrow$ OSLiC, pp.\
\pageref{OSUC-10S-DEF}}
\item[requires] the following tasks in order to fulfill the license conditions:
\begin{itemize}
  
  \item \textbf{[mandatory:]} Ensure that the complete PHP license -- esp.\
  the copyright notice, the PHP conditions, and the PHP disclaimer -- are
  retained in your package in the form you have received them.
  
  \item \textbf{[mandatory:]} Let the documentation of your distribution and/or
  your additional material also contain a line of acknowledgment in the form
   \begin{footnotesize}"This product includes PHP, freely available from
  $\langle$http://www.php.net/$\rangle$"\end{footnotesize}.
    
  \item \textbf{[mandatory:]} Let the documentation of your distribution and/or
  your additional material also contain the original copyright notice, the PHP
  conditions, and the PHP disclaimer.

  \item \textbf{[voluntary:]} It is a good practice of the open source
  community, to let the copyright notice which is shown by the running program
  also state that the program uses a copmponent licensed under the PHP license.
  So, let the copyright screen of the overarching program \emph{reproduce} the
  complete PHP license including the copyright notice, the PHP conditions, and
  the PHP disclaimer -- as it is required for binary
  distributions\footnote{Following distributors of compiled versions will
  appreciate your prepatory work.}
  
  \item \textbf{[voluntary:]} Mark your modifications in the source code.
  
  \item \textbf{[voluntary:]} Arrange your source code distribution so that the
  licensing elements -- esp.\ the PHP license text, the specific copyright
  notice of the original author(s), and the PHP disclaimer -- clearly refer
  only to the embedded library and do not disturb the licensing of your own
  overarching work. It's a good tradition to keep the embedded components like
  libraries, modules, snippets, or plugins in specific directory which contains
  also all additional licensing elements.
  
\end{itemize}

\item[prohibits] to endorse or promote your product by mentioning PHP, esp. not
by make the string 'PHP' part of its name.

\end{description}


\subsection{PHP-C9: Passing a modified library as embedded binary}
\label{OSUC-10B-PHP}

\begin{description}
\item[means] that you are going to distribute a modified version of the received
PHP code snippet, module, library, or plugin to 3rd parties -- in the form of
binary files or as a binary package together with another larger software unit
which contains this code snippet, module, library, or plugin as an embedded
component.
\item[covers] OSUC-10B\footnote{For details $\rightarrow$ OSLiC, pp.\
\pageref{OSUC-10B-DEF}}
\item[requires] the following tasks in order to fulfill the license conditions:
\begin{itemize}

  \item \textbf{[mandatory:]} Let the documentation of your distribution and/or
  your additional material also contain a line of acknowledgment in the form
  \begin{footnotesize}"This product includes PHP, freely available from
  $\langle$http://www.php.net/$\rangle$"\end{footnotesize}.
    
  \item \textbf{[mandatory:]} Let the documentation of your distribution and/or
  your additional material also contain the original copyright notice, the PHP
  conditions, and the PHP disclaimer.

  \item \textbf{[voluntary:]} Ensure that the complete PHP license -- esp.\ the
  copyright notice, the PHP conditions, and the PHP disclaimer -- are
  \textbf{reproduced} by your package, esp. by the copyright screen of your
  overaching program which uses the library.
    
  \item \textbf{[voluntary:]} Mark your modifications in the source code,
  even if you do not you want to distribute the code.
  
  \item \textbf{[voluntary:]} Arrange your binary distribution so that the
  licensing elements -- esp.\ the PHP license text, the specific copyright
  notice of the original author(s), and the PHP disclaimer -- clearly refer
  only to the embedded library and do not disturb the licensing of your own
  overarching work. It's a good tradition to keep the libraries, modules,
  snippet, or plugins in specific directories which contain also all licensing
  elements.
\end{itemize}

\item[prohibits] to endorse or promote your product by mentioning PHP, esp. not
by make the string 'PHP' part of its name.

\end{description}

\subsection{Discussions and Explanations}

\label{sec:PhpDiscussions}

First of all, it might surprise some readers that the OSLiC also describes the
open source use cases which concern the distribution of binary files although it
deals with the PHP license. PHP is a script language. Thus, delivering the
source code seems to be a must. But one has to consider that the PHP license
could also be applied to works which are based on other languages constituted on
the compiler paradigm. Or there might be used PHP compiler.

It might also surprise some readers that in case of the binary distribution of
modifications the condition to repoduce the php license in the documentation is
a \emph{must}, while its reproduction in a copyright screen of the program is a
\emph{should}. This is directly evoked by the binary-conditon of the php license
which expressly requires that \enquote{Redistributions in binary form must
reproduce the above copyright notice, this list of conditions and the following
disclaimer in the documentation and/or other materials provided with the
distribution}\footcite[cf.][wp. §2]{Php30OsiLicense2013a}. But of course, to
implement the \emph{must} and the \emph{should} is the best.

%\bibliography{../../../bibfiles/oscResourcesEn}
