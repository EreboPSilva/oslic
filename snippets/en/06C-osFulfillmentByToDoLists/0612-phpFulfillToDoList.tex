% Telekom osCompendium 'for being included' snippet template
%
% (c) Karsten Reincke, Deutsche Telekom AG, Darmstadt 2011
%
% This LaTeX-File is licensed under the Creative Commons Attribution-ShareAlike
% 3.0 Germany License (http://creativecommons.org/licenses/by-sa/3.0/de/): Feel
% free 'to share (to copy, distribute and transmit)' or 'to remix (to adapt)'
% it, if you '... distribute the resulting work under the same or similar
% license to this one' and if you respect how 'you must attribute the work in
% the manner specified by the author ...':
%
% In an internet based reuse please link the reused parts to www.telekom.com and
% mention the original authors and Deutsche Telekom AG in a suitable manner. In
% a paper-like reuse please insert a short hint to www.telekom.com and to the
% original authors and Deutsche Telekom AG into your preface. For normal
% quotations please use the scientific standard to cite.
%
% [ Framework derived from 'mind your Scholar Research Framework' 
%   mycsrf (c) K. Reincke 2012 CC BY 3.0  http://mycsrf.fodina.de/ ]
%


%% use all entries of the bibliography
%\nocite{*}

\section{PHP-3.0 licensed software}

\begin{license}{PHP} % ends at end of file
\licensename{PHP-3.0}
\licensespec{PHP 3.0 License}
\licenseversion{3.0}
\licenseabbrev{PHP}


The PHP-3.0 license contains a few more conditions than the MIT license and
additionally distinguishes the \enquote{redistribution of source
code}\citePHP{} from the \enquote{redistribution in binary form}.\citePHP{}
Nevertheless, the PHP-3.0 license focusses only on the redistribution or---as we
call it in the \oslic---\emph{the 2others use
cases.} Thus, the PHP-3.0 specific finder can be simplified:

\tikzstyle{nodv} = [font=\small, ellipse, draw, fill=gray!10, 
    text width=2cm, text centered, minimum height=2em]


\tikzstyle{nods} = [font=\footnotesize, rectangle, draw, fill=gray!20, 
    text width=1.2cm, text centered, rounded corners, minimum height=3em]

\tikzstyle{nodb} = [font=\footnotesize, rectangle, draw, fill=gray!20, 
    text width=2.2cm, text centered, rounded corners, minimum height=3em]
    
\tikzstyle{leaf} = [font=\tiny, rectangle, draw, fill=gray!30, 
    text width=1.2cm, text centered, minimum height=6em]

\tikzstyle{edge} = [draw, -latex']

\begin{tikzpicture}[]

\node[nodv] (l71) at (3.5,10) {PHP-3.0};

\node[nodb] (l61) at (0,8.6) {\textit{recipient:} \\ \textbf{4yourself}};
\node[nodb] (l62) at (6.5,8.6) {\textit{recipient:} \\ \textbf{2others}};

\node[nodb] (l51) at (2.5,7) {\textit{state:} \\ \textbf{unmodified}};
\node[nodb] (l52) at (9.3,7) {\textit{state:} \\ \textbf{modified}};

\node[nods] (l41) at (1.8,5.4) {\textit{form:} \textbf{source}};
\node[nods] (l42) at (3.6,5.4) {\textit{form:} \textbf{binary}};
\node[nodb] (l43) at (6.5,5.4) {\textit{type:} \\ \textbf{proapse}};
\node[nodb] (l44) at (12,5.4) {\textit{type:} \\ \textbf{snimoli}};


\node[nods] (l31) at (5.4,3.8) {\textit{form:} \textbf{source}};
\node[nods] (l32) at (7.2,3.8) {\textit{form:} \textbf{binary}};
\node[nodb] (l33) at (10,3.8) {\textit{context:} \\ \textbf{independent}};
\node[nodb] (l34) at (13.5,3.8) {\textit{context:} \\ \textbf{embedded}};

\node[nods] (l21) at (9,2.2) {\textit{form:} \textbf{source}};
\node[nods] (l22) at (10.8,2.2) {\textit{form:} \textbf{binary}};
\node[nods] (l23) at (12.6,2.2) {\textit{form:} \textbf{source}};
\node[nods] (l24) at (14.4,2.2) {\textit{form:} \textbf{binary}};

\node[leaf] (l11) at (0,0) {\textbf{PHP-3.0-C1} \textit{using software only
for yourself}};

\node[leaf] (l12) at (1.8,0) { \textbf{PHP-3.0-C2} \textit{ distributing unmodified
software as sources}};

\node[leaf] (l13) at (3.6,0) { \textbf{PHP-3.0-C3}  \textit{ distributing unmodified
software as binaries}};

\node[leaf] (l14) at (5.4,0) { \textbf{PHP-3.0-C4}  \textit{ distributing modified
program as sources}};

\node[leaf] (l15) at (7.2,0) { \textbf{PHP-3.0-C5}  \textit{ distributing modified
program as binaries}};

\node[leaf] (l16) at (9,0) { \textbf{PHP-3.0-C6}  \textit{ distributing modified
library as independent sources}};

\node[leaf] (l17) at (10.8,0) { \textbf{PHP-3.0-C7} \textit{distributing modified
library as independent binaries}};

\node[leaf] (l18) at (12.6,0) { \textbf{PHP-3.0-C8}  \textit{distributing
modified library as embedded sources}};

\node[leaf] (l19) at (14.4,0) { \textbf{PHP-3.0-C9}  \textit{ distributing modified
library as embedded binaries}};

\path [edge] (l71) -- (l61);
\path [edge] (l71) -- (l62);
\path [edge] (l61) -- (l11);
\path [edge] (l62) -- (l51);
\path [edge] (l62) -- (l52);
\path [edge] (l51) -- (l41);
\path [edge] (l51) -- (l42);
\path [edge] (l52) -- (l43);
\path [edge] (l52) -- (l44);
\path [edge] (l41) -- (l12);
\path [edge] (l42) -- (l13);
\path [edge] (l43) -- (l31);
\path [edge] (l43) -- (l32);
\path [edge] (l44) -- (l33);
\path [edge] (l44) -- (l34);
\path [edge] (l31) -- (l14);
\path [edge] (l32) -- (l15);
\path [edge] (l33) -- (l21);
\path [edge] (l33) -- (l22);
\path [edge] (l34) -- (l23);
\path [edge] (l34) -- (l24);
\path [edge] (l21) -- (l16);
\path [edge] (l22) -- (l17);
\path [edge] (l23) -- (l18);
\path [edge] (l24) -- (l19);

\end{tikzpicture}

%% =============================================================================
%% Common building blocks
%%

% ------------------------------------------------------------------------------
% Preserve the license elements

\newcommand{\keepPHPLicense}{Ensure that the complete PHP-3.0 license
  (especially the copyright notice, the PHP-3.0 conditions, and the PHP-3.0
  disclaimer) are retained in your package in the form you have received them.}

% ------------------------------------------------------------------------------
% Reproduce the license elements
\newcommand{\reproducePHPLicense}{Ensure that the complete PHP-3.0 license
  (especially the copyright notice, the PHP-3.0 conditions, and the PHP-3.0
  disclaimer) are \emph{reproduced} by your package in the form you have
  received them. If you compile the binary file from the source code package and
  if this process does not also generate and integrate the licensing files then
  create the copyright notice, the PHP-3.0 conditions, and the PHP-3.0
  disclaimer in the form present in the source code package and insert these
  files into your distribution manually.}

% ------------------------------------------------------------------------------
% Mark modifications in the source

\newcommand{\markModifications}{Mark your modifications in the source code.}
\newcommand{\markUndistributedModifications}{Mark your modifications in the
  source code, even if you do not want to distribute the code.}

% ------------------------------------------------------------------------------
% Acknowledge PHP in documentation and link to homepage

\newcommand{\acknowledgePHPInDocumentation}{Let the documentation of your
  distribution or your additional material also contain a line of acknowledgment
  in the form:
  \enquote{This product includes PHP, freely available from http://www.php.net/}}

% ------------------------------------------------------------------------------
% Add license to documentation

\newcommand{\addLicenseToDocumentation}{Let the documentation of your
  distribution and/or your additional material also contain the original
  copyright notice, the PHP-3.0 conditions, and the PHP-3.0 disclaimer.}

% ------------------------------------------------------------------------------
% Create a copyright dialog or add elements to it

\newcommand{\auxCDIntro}[1]{It is a good practice of the open source community
  to let the copyright notice, which is shown by the running program, also state
  that the program #1 under the PHP-3.0 license.}
\newcommand{\auxCDElements}{\emph{reproduce} the complete PHP-3.0 license
  including the copyright notice, the PHP-3.0 conditions, and the PHP-3.0
  disclaimer (as it is required for binary distributions.)} 

\newcommand{\createCopyrightDialog}{%
  \auxCDIntro{is licensed}.
  Because you are already modifying the program you can also add such a hint if
  the original copyright notice lacks such a statement. If such a notice is
  missing in the copyright screen, consider, if it is possible to let it
  \auxCDElements} 

\newcommand{\listEmbeddedLibraryInCopyrightDialog}{%
  \auxCDIntro{uses a component licensed}
  So, let the copyright screen of the enclosing program \auxCDElements}

% ------------------------------------------------------------------------------
% Separate embedded library from enclosing program

\newcommand{\auxKeepSeparate}[1]{Arrange your #1 distribution so that the
  licensing elements (especially the PHP-3.0 license text, the specific
  copyright notice of the original author(s), and the PHP-3.0 disclaimer)
  clearly refer only to the embedded library and do not affect the licensing of
  your own overarching work. It's a good tradition to keep embedded components
  like libraries, modules, snippets, or plugins in separate directories, which
  contain also all additional licensing elements.}

\newcommand{\keepSourceSeparate}{\auxKeepSeparate{source code}}
\newcommand{\keepBinarySeparate}{\auxKeepSeparate{binary}}

% ------------------------------------------------------------------------------
% Forbid to use the name PHP

\newcommand{\toUseTheNamePHPForServices}{to endorse or promote any service you
  establish based on this software by the name `PHP.'}
\newcommand{\toUseTheNamePHP}{to endorse or promote your product by mentioning
  PHP, especially not by making the string `PHP' part of its name.}

%% =============================================================================
%% Use Cases

\subsection{PHP-3.0-C1: Using the software only for yourself}
\begin{lsuc}{PHP-3.0-C1}
  \linkosuc{01} 
  \linkosuc{03} 
  \linkosuc{06}
  \linkosuc{09}
  
  \lsucmeans{that you received PHP-3.0 licensed software, that you will use it
    only for yourself, and that you do not hand it over to any third party in
    any sense.}

  \coversOsucs{OSUC-01, OSUC-03, OSUC-06, and OSUC-09}{01}{09}
  
  \begin{lsucrequiresnothing}
    \lsucitem{You are allowed to use any kind of PHP-3.0 software in any sense
      and in any context without any obligations as long as you do not give the
      software to third parties.}
  \end{lsucrequiresnothing}

  \begin{lsucprohibits}
    \lsucitem{\toUseTheNamePHPForServices}
  \end{lsucprohibits}
\end{lsuc}

% ------------------------------------------------------------------------------
\subsection{PHP-3.0-C2: Passing the unmodified software as source code}
\begin{lsuc}{PHP-3.0-C2}
  \linkosuc{02S}
  \linkosuc{05S} 
  \linkosuc{07S} 

  \lsucmeans{that you received PHP-3.0 licensed software which you are now going
    to distribute to third parties in the form of unmodified source code files
    or as unmodified source code package. In this case it makes no difference if
    you distribute a program, an application, a server, a snippet, a module, a
    library, or a plugin as an independent or as an embedded unit.}

  \coversOsucs{OSUC-02S, OSUC-05S, OSUC-07S}{02S}{07S}

  \begin{lsucrequires}
    \lsucmandatory{\keepPHPLicense}
    \lsucmandatory{\acknowledgePHPInDocumentation}
    \lsucoptional{\addLicenseToDocumentation}
  \end{lsucrequires}

  \begin{lsucprohibits}
    \lsucitem{\toUseTheNamePHP}
  \end{lsucprohibits} 
\end{lsuc}

% ------------------------------------------------------------------------------
\subsection{PHP-3.0-C3: Passing the unmodified software as binary}
\begin{lsuc}{PHP-3.0-C3}
  \linkosuc{02B} 
  \linkosuc{05B} 
  \linkosuc{07B} 

  \lsucmeans{that you received PHP-3.0 licensed software which you are now going
    to distribute to third parties in the form of unmodified binary files or as
    unmodified binary package. In this case it does not matter if you distribute
    a program, an application, a server, a snippet, a module, a library, or a
    plugin as an independent or an embedded unit.}

  \coversOsucs{OSUC-02B, OSUC-05B, OSUC-07B}{02B}{07B}

  \begin{lsucrequires}
    \lsucmandatory{\reproducePHPLicense}%
    \footnote{Because you are distributing an unmodified binary, you could
      assume that the copright screens of the application do already what they
      have to do.}%
    \passingFilesCorrectly
    \lsucmandatory{\acknowledgePHPInDocumentation}
    \lsucoptional{\addLicenseToDocumentation}
  \end{lsucrequires}

  \begin{lsucprohibits}
    \lsucitem{\toUseTheNamePHP}
  \end{lsucprohibits}
\end{lsuc}

% ------------------------------------------------------------------------------
\subsection{PHP-3.0-C4: Passing a modified program as source code}
\begin{lsuc}{PHP-3.0-C4}
  \linkosuc{04S}

  \lsucmeans{that you received a PHP-3.0 licensed program, application, or
    server (proapse), that you modified it, and that you are now going to
    distribute this modified version to third parties in the form of source code
    files or as a source code package.}

  \mapsToOsuc{04S}

  \begin{lsucrequires}
    \lsucmandatory{\keepPHPLicense}
    \lsucmandatory{\acknowledgePHPInDocumentation}
    \lsucoptional{\addLicenseToDocumentation}
    \lsucoptional{\createCopyrightDialog}% 
    \footnote{Following distributors of compiled versions will appreciate your
      prepatory work.} 
    \lsucoptional{\markModifications}
  \end{lsucrequires}

  \begin{lsucprohibits}
    \lsucitem{\toUseTheNamePHP}
  \end{lsucprohibits}
\end{lsuc}

% ------------------------------------------------------------------------------
\subsection{PHP-3.0-C5: Passing a modified program as binary}
\begin{lsuc}{PHP-3.0-C5}
  \linkosuc{04B}

  \lsucmeans{that you received a PHP-3.0 licensed program, application, or
    server (proapse), that you modified it, and that you are now going to
    distribute this modified version to third parties in the form of binary
    files or as a binary package.}

  \mapsToOsuc{04B}

  \begin{lsucrequires}
    \lsucmandatory{\acknowledgePHPInDocumentation}
    \lsucmandatory{\addLicenseToDocumentation}
    \lsucoptional{\reproducePHPLicense}
    \lsucoptional{\markUndistributedModifications}
  \end{lsucrequires}

  \begin{lsucprohibits}
    \lsucitem{\toUseTheNamePHP}
  \end{lsucprohibits}
\end{lsuc}

% ------------------------------------------------------------------------------
\subsection{PHP-3.0-C6: Passing a modified library as independent source code}
\begin{lsuc}{PHP-3.0-C6}
  \linkosuc{08S}

  \lsucmeans{that you received a PHP-3.0 licensed code snippet, module, library,
    or plugin (snimoli), that you modified it, and that you are now going to
    distribute this modified version to third parties in the form of source code
    files or as a source code package, but without embedding it into another
    larger software unit.}

  \mapsToOsuc{08S}

  \begin{itemize}
    \lsucmandatory{\keepPHPLicense}
    \lsucmandatory{\acknowledgePHPInDocumentation}
    \lsucoptional{\addLicenseToDocumentation}
    \lsucoptional{\markModifications}
  \end{itemize}

  \begin{lsucprohibits}
    \lsucitem{\toUseTheNamePHP}
  \end{lsucprohibits}
\end{lsuc}

% ------------------------------------------------------------------------------
\subsection{PHP-3.0-C7: Passing a modified library as independent binary}
\begin{lsuc}{PHP-3.0-C7}
  \linkosuc{08B}

  \lsucmeans{that you received a PHP-3.0 licensed code snippet, module, library,
    or plugin (snimoli), that you modified it, and that you are now going to
    distribute this modified version to third parties in the form of binary files
    or as a binary package but without embedding it into another larger software
    unit.}

  \mapsToOsuc{08B}

  \begin{lsucrequires}
    \lsucmandatory{\acknowledgePHPInDocumentation}
    \lsucmandatory{\addLicenseToDocumentation}
    \lsucoptional{\reproducePHPLicense}
    \lsucoptional{\markUndistributedModifications}
  \end{lsucrequires}

  \begin{lsucprohibits}
    \lsucitem{\toUseTheNamePHP}
  \end{lsucprohibits}
\end{lsuc}

% ------------------------------------------------------------------------------
\subsection{PHP-3.0-C8: Passing a modified library as embedded source code}
\begin{lsuc}{PHP-3.0-C8}
  \linkosuc{10S}

  \lsucmeans{that you received a PHP-3.0 licensed code snippet, module, library,
    or plugin (snimoli), that you modified it, and that you are now going to
    distribute this modified version to third parties in the form of source code
    files or as a source code package together with another larger software unit
    which contains this code snippet, module, library, or plugin as an embedded
    component.}

  \mapsToOsuc{10S}

  \begin{lsucrequires}
    \lsucmandatory{\keepPHPLicense}
    \lsucmandatory{\acknowledgePHPInDocumentation}
    \lsucmandatory{\addLicenseToDocumentation}
    \lsucoptional{\listEmbeddedLibraryInCopyrightDialog}%
    \footnote{Following distributors of compiled versions will appreciate your
      prepatory work.} 
    \lsucoptional{\markModifications}
    \lsucoptional{\keepSourceSeparate}
  \end{lsucrequires}

  \begin{lsucprohibits}
    \lsucitem{\toUseTheNamePHP}
  \end{lsucprohibits}
\end{lsuc}

% ------------------------------------------------------------------------------
\subsection{PHP-3.0-C9: Passing a modified library as embedded binary}
\begin{lsuc}{PHP-3.0-C9}
  \linkosuc{10B}

  \lsucmeans{that you received a PHP-3.0 licensed code snippet, module, library,
    or plugin (snimoli), that you modified it, and that you are now going to
    distribute this modified version to third parties in the form of binary
    files or as a binary package together with another larger software unit
    which contains this code snippet, module, library, or plugin as an embedded
    component.}

  \mapsToOsuc{10B}

  \begin{lsucrequires}
    \lsucmandatory{\acknowledgePHPInDocumentation}
    \lsucmandatory{\addLicenseToDocumentation}
    \lsucoptional{\reproducePHPLicense}
    \lsucoptional{\markUndistributedModifications}
    \lsucoptional{\keepBinarySeparate}
  \end{lsucrequires}

  \begin{lsucprohibits}
    \lsucitem{\toUseTheNamePHP}
  \end{lsucprohibits}
\end{lsuc}

% ------------------------------------------------------------------------------
\subsection{Discussions and Explanations}
\label{sec:PhpDiscussions}

First of all, it might surprise some readers that the \oslic{} also describes
the open source use cases which concern the distribution of binary files
although it deals with the PHP-3.0 license. PHP is a script language. Thus,
delivering the source code seems to be a must. But one has to consider that the
PHP-3.0 license could also be applied to works which are based on other
languages constituted on the compiler paradigm. Or there might a PHP compiler be
used. 

It might also surprise some readers that in case of the binary distribution of
modifications the condition to repoduce the php license in the documentation is
a \emph{must,} while its reproduction in a copyright screen of the program is a
\emph{should.} This is directly caused by the binary-condition of the php license
which expressly requires that \enquote{Redistributions in binary form must
reproduce the above copyright notice, this list of conditions and the following
disclaimer in the documentation and/or other materials provided with the
distribution.}\citePHP{} But of course, implementing the \emph{must} and the
\emph{should} is best. 

% ------------------------------------------------------------------------------
\end{license}

%\bibliography{../../../bibfiles/oscResourcesEn}

% Local Variables:
% mode: latex
% fill-column: 80
% End:
