% Telekom osCompendium 'for being included' snippet template
%
% (c) Karsten Reincke, Deutsche Telekom AG, Darmstadt 2011
%
% This LaTeX-File is licensed under the Creative Commons Attribution-ShareAlike
% 3.0 Germany License (http://creativecommons.org/licenses/by-sa/3.0/de/): Feel
% free 'to share (to copy, distribute and transmit)' or 'to remix (to adapt)'
% it, if you '... distribute the resulting work under the same or similar
% license to this one' and if you respect how 'you must attribute the work in
% the manner specified by the author ...':
%
% In an internet based reuse please link the reused parts to www.telekom.com and
% mention the original authors and Deutsche Telekom AG in a suitable manner. In
% a paper-like reuse please insert a short hint to www.telekom.com and to the
% original authors and Deutsche Telekom AG into your preface. For normal
% quotations please use the scientific standard to cite.
%
% [ Framework derived from 'mind your Scholar Research Framework' 
%   mycsrf (c) K. Reincke 2012 CC BY 3.0  http://mycsrf.fodina.de/ ]
%


%% use all entries of the bibliography
%\nocite{*}

\section{EPL-1.0 licensed software}
\begin{license}{EPL}
\licensename{EPL}
\licensespec{Eclipse Public License}
\licenseabbrev{EPL}

The Eclipse Public License clearly distinguishes the distribution in the form of
source code from that in the form of binaries: First, it allows to
\enquote{distribute} Eclipse licensed programs \enquote{in source code and in
object code}.\citeEPL{§3} Then it specifies under which conditions one may
distribute the program as a set of binaries.\citeEPL{§3 top area} One of
these conditions is---roughly speaking---that the distributor makes the sources
available too.\citeEPL{§3 mid area} More precisely, the EPL-1.0 has to be taken as a
license with weak copyleft ($\rightarrow$ \oslic, p.\ \protectionpageref{EPL}).
The other conditions refer to the distribution in general---no matter what form
or state is used.\citeEPL{§3 bottom area} So, taken as whole, the EPL-1.0 mainly
focusses on the distribution of software.  Thus, for finding the relevant, easy
to process task lists, the following EPL-1.0 specific open source use case 
structure%
  \footnote{For details of the general OSUC finder $\rightarrow$ \oslic,
    pp.\ \pageref{OsucTokens} and \pageref{OsucDefinitionTree}}
can be used:
 
\tikzstyle{nodv} = [font=\small, ellipse, draw, fill=gray!10, 
    text width=2cm, text centered, minimum height=2em]

\tikzstyle{nods} = [font=\footnotesize, rectangle, draw, fill=gray!20, 
    text width=1.2cm, text centered, rounded corners, minimum height=3em]

\tikzstyle{nodb} = [font=\footnotesize, rectangle, draw, fill=gray!20, 
    text width=2.2cm, text centered, rounded corners, minimum height=3em]
    
\tikzstyle{leaf} = [font=\tiny, rectangle, draw, fill=gray!30, 
    text width=1.2cm, text centered, minimum height=6em]

\tikzstyle{edge} = [draw, -latex']

\begin{tikzpicture}[]

\node[nodv] (l71) at (4,10) {EPL-1.0};

\node[nodb] (l61) at (0,8.6) {\textit{recipient:} \\ \textbf{4yourself}};
\node[nodb] (l62) at (6.5,8.6) {\textit{recipient:} \\ \textbf{2others}};

\node[nodb] (l51) at (2.5,7) {\textit{state:} \\ \textbf{unmodified}};
\node[nodb] (l52) at (9.3,7) {\textit{state:} \\ \textbf{modified}};

\node[nods] (l41) at (1.8,5.4) {\textit{form:} \textbf{source}};
\node[nods] (l42) at (3.6,5.4) {\textit{form:} \textbf{binary}};
\node[nodb] (l43) at (6.5,5.4) {\textit{type:} \\ \textbf{proapse}};
\node[nodb] (l44) at (12,5.4) {\textit{type:} \\ \textbf{snimoli}};


\node[nods] (l31) at (5.4,3.8) {\textit{form:} \textbf{source}};
\node[nods] (l32) at (7.2,3.8) {\textit{form:} \textbf{binary}};
\node[nodb] (l33) at (10,3.8) {\textit{context:} \\ \textbf{independent}};
\node[nodb] (l34) at (13.5,3.8) {\textit{context:} \\ \textbf{embedded}};

\node[nods] (l21) at (9,2.2) {\textit{form:} \textbf{source}};
\node[nods] (l22) at (10.8,2.2) {\textit{form:} \textbf{binary}};
\node[nods] (l23) at (12.6,2.2) {\textit{form:} \textbf{source}};
\node[nods] (l24) at (14.4,2.2) {\textit{form:} \textbf{binary}};

\node[leaf] (l11) at (0,0) {\textbf{EPL-1.0-C1} \textit{using software only
for yourself}};

\node[leaf] (l12) at (1.8,0) { \textbf{EPL-1.0-C2} \textit{ distributing unmodified
software as sources}};

\node[leaf] (l13) at (3.6,0) { \textbf{EPL-1.0-C3}  \textit{ distributing unmodified
software as binaries}};

\node[leaf] (l14) at (5.4,0) { \textbf{EPL-1.0-C4}  \textit{ distributing modified
program as sources}};

\node[leaf] (l15) at (7.2,0) { \textbf{EPL-1.0-C5}  \textit{ distributing modified
program as binaries}};

\node[leaf] (l16) at (9,0) { \textbf{EPL-1.0-C6}  \textit{ distributing modified
library as independent sources}};

\node[leaf] (l17) at (10.8,0) { \textbf{EPL-1.0-C7} \textit{distributing modified
library as independent binaries}};

\node[leaf] (l18) at (12.6,0) { \textbf{EPL-1.0-C8}  \textit{distributing
modified library as embedded sources}};

\node[leaf] (l19) at (14.4,0) { \textbf{EPL-1.0-C9}  \textit{ distributing modified
library as embedded binaries}};


\path [edge] (l71) -- (l61);
\path [edge] (l71) -- (l62);
\path [edge] (l61) -- (l11);
\path [edge] (l62) -- (l51);
\path [edge] (l62) -- (l52);
\path [edge] (l51) -- (l41);
\path [edge] (l51) -- (l42);
\path [edge] (l52) -- (l43);
\path [edge] (l52) -- (l44);
\path [edge] (l41) -- (l12);
\path [edge] (l42) -- (l13);
\path [edge] (l43) -- (l31);
\path [edge] (l43) -- (l32);
\path [edge] (l44) -- (l33);
\path [edge] (l44) -- (l34);
\path [edge] (l31) -- (l14);
\path [edge] (l32) -- (l15);
\path [edge] (l33) -- (l21);
\path [edge] (l33) -- (l22);
\path [edge] (l34) -- (l23);
\path [edge] (l34) -- (l24);
\path [edge] (l21) -- (l16);
\path [edge] (l22) -- (l17);
\path [edge] (l23) -- (l18);
\path [edge] (l24) -- (l19);

\end{tikzpicture}

%%
%% Common Building Blocks
%%

% ------------------------------------------------------------------------------
% Forbid to change copyright notices
\newcommand{\dontChangeCopyrightNotices}{to remove or to alter any copyright
  notices that were contained in the software package when you received it.} 
  
% ------------------------------------------------------------------------------
% Forbid patent litigation
\newcommand{\noPatentLitigation}{%
  to institute a patent litigation against anyone alleging that the software
  constitutes patent infringement.}
  

% ------------------------------------------------------------------------------
% Require to keep licensing elements
\newcommand{\keepLicensingElements}{Ensure that the licensing elements
  (particularly all copyright notices and the disclaimer of warranty and
  disclaimer of liability) are retained in your package in exactly the form you
  have received them.}
\newcommand{\addWhenCompiling}{If you compile the binary from the sources,
  ensure that all these licensing elements are also incorporated into the
  package.}

% ------------------------------------------------------------------------------
% Give recipient a copy of the license
\newcommand{\giveLicenseFile}{Give the recipient a copy of the EPL-1.0 license.
  If it is not already part of the software package, add it. If the licensing
  statement in the licensing file of the package does still not clearly state
  that the package is licensed under the EPL-1.0, additionally insert your own
  correct EPL-1.0 licensing file.}

% ------------------------------------------------------------------------------
% No warranty
\newcommand{\noWarranty}[1]{If still not existing, integrate an explicit, very 
  prominently placed `No warranty' statement into the distributed #1 package.
  Let this statement clearly say that all (other) contributors to the software
  do not accept any responsibility for the quality of the software. Then, copy
  the no-warranty clause and the disclaimer of liability from the EPL-1.0 itself
  into that file.} 

\newcommand{\includeInCopyrightScreen}{Let the copyright screen of your own
  overarching program show the same information as a specification for the
  embedded component.}

\newcommand{\updateCopyrightScreen}{Update an existing copyright screen
  presented by the program so that it shows the same information.}

% ------------------------------------------------------------------------------
% Add license info to documentation
\newcommand{\auxAddToDoc}[1]{%
  Let the documentation of your distribution or your additional material
  reproduce the content of an existing \emph{copyright notice text files}, a
  hint to the software name, a link to its homepage, and a link to the EPL-1.0
  license#1.}   

\newcommand{\addToDocumentation}{\auxAddToDoc{}}
\newcommand{\addToYourCopyrightNotice}{%
  \auxAddToDoc{, preferably as a subsection of your own copyright notice}}

% ------------------------------------------------------------------------------
% Publish source code
\newcommand{\auxPublishSourceCode}[2]{Make the source code of #1 accessible
  through a repository under your own control#2: Push the source code package
  into an internet repository and enable the download function. Ensure that this
  respository is available for a reasonable period of time.}

\newcommand{\publishUnmodifiedSourceCode}[1]{%
  \auxPublishSourceCode{#1}{, even if you did not modify it}}
\newcommand{\publishSourceCode}[1]{%
  \auxPublishSourceCode{#1}{}}

% ------------------------------------------------------------------------------
% Include link to source repository
\newcommand{\linkToRepo}{Insert a prominent hint to the download repository
  into your distribution or your additional material and explain how the code
  can be obtained.}
 
% ------------------------------------------------------------------------------
% Create a modification text file...
\newcommand{\describeModifications}{Create a \emph{modification text file} if
  such a file does not exist. \emph{Add} a general description of your
  modifications to the \emph{modification text file.} Incorporate it into your
  distribution package.} 

% ------------------------------------------------------------------------------
% mark all modifications in the source
\newcommand{\markAllModifications}{Mark all modifications of the source code of
  the program thoroughly; namely within the modified source code.}

% ------------------------------------------------------------------------------
% Organize your sources
\newcommand{\organizeYourModifications}{Organize your modifications in a way
  that they are covered by the existing EPL-1.0 licensing statements.}
\newcommand{\addHeaderToNewFiles}{If you add new source code files, insert a
  header containing your copyright line and an EPL-1.0 adequate licensing the
  statement.} 

% ------------------------------------------------------------------------------
% Use separate directories
\newcommand{\useSeparateDirectory}[1]{Arrange your #1 distribution so that the
  integrated EPL-1.0 and the \emph{licensing files} clearly refer only to the
  embedded library and do not disturb the licensing of your own overarching
  work. It's a good tradition to keep the embedded components like libraries,
  modules, snippets, or plugins in separate directories which also contains all 
  additional licensing elements.}

% ------------------------------------------------------------------------------
% EPL-C1
% ------------------------------------------------------------------------------
\subsection{EPL-1.0-C1: Using the software only for yourself}
\begin{lsuc}{EPL-C1}
  \linkosuc{01} 
  \linkosuc{03} 
  \linkosuc{06} 
  \linkosuc{09}

  \lsucmeans{that you received EPL-1.0 licensed software, that you will use it
  only for yourself, and that you do not hand it over to any 3rd party in any
  sense.} 

  \coversOsucs{OSUC-01, OSUC-03, OSUC-06, and OSUC-09}{01}{09}

  \begin{lsucrequiresnothing}
    \item You are allowed to use any kind of EPL-1.0 software in any sense and in
    any context without being obliged to do anything as long as you do not
    give the software to third parties.
  \end{lsucrequiresnothing}

  \begin{lsucprohibits}
    \lsucitem{\dontChangeCopyrightNotices}
    \lsucitem{\noPatentLitigation}
  \end{lsucprohibits}

\end{lsuc}

% ------------------------------------------------------------------------------
% EPL-C2
% ------------------------------------------------------------------------------
\subsection{EPL-1.0-C2: Passing the unmodified software as source code}
\begin{lsuc}{EPL-C2}
  \linkosuc{02S} 
  \linkosuc{05S}
  \linkosuc{07S} 

  \lsucmeans{that you received EPL-1.0 licensed software which you are now
  going to distribute to third parties in the form of unmodified source code
  files or as unmodified source code package. In this case it makes no
  difference if you distribute a program, an application, a server, a snippet, a
  module, a library, or a plugin as an independent or as an embedded unit.} 

  \coversOsucs{OSUC-02S, OSUC-05S, OSUC-07S}{02S}{07S}

  \begin{lsucrequires}
    \lsucmandatory{\keepLicensingElements}
    \lsucmandatory{\giveLicenseFile}\passingFilesCorrectly
    \lsucmandatory{\noWarranty{source code}}  
    \lsucoptional{\addToDocumentation}
  \end{lsucrequires}

  \begin{lsucprohibits}
    \lsucitem{\dontChangeCopyrightNotices}
    \lsucitem{\noPatentLitigation}
  \end{lsucprohibits}

\end{lsuc}

% ------------------------------------------------------------------------------
% EPL-C3
% ------------------------------------------------------------------------------
\subsection{EPL-1.0-C3: Passing the unmodified software as binaries} 
\begin{lsuc}{EPL-C3}
  \linkosuc{02B} 
  \linkosuc{05B} 
  \linkosuc{07B} 

  \lsucmeans{that you received EPL-1.0 licensed software which you are now
  going to distribute to third parties in the form of unmodified binary files or
  as unmodified binary package. In this case it does not matter if you distribute
  a program, an application, a server, a snippet, a module, a library, or a
  plugin as an independent or an embedded unit.}

  \coversOsucs{OSUC-02B, OSUC-05B, OSUC-07B}{02B}{07B}

  \begin{lsucrequires}
  
    \lsucmandatory{\keepLicensingElements\ \addWhenCompiling}
    \lsucmandatory{\noWarranty{binary}}
    \lsucmandatory{\publishUnmodifiedSourceCode{the software}}
    \lsucmandatory{\linkToRepo}
    \lsucsourcedist{EPL-C2}
    \lsucoptional{\addToDocumentation}
  \end{lsucrequires}

  \begin{lsucprohibits}
    \lsucitem{\dontChangeCopyrightNotices}
    \lsucitem{\noPatentLitigation}
  \end{lsucprohibits}

\end{lsuc}

% ------------------------------------------------------------------------------
% EPL-C4
% ------------------------------------------------------------------------------
\subsection{EPL-1.0-C4: Passing a modified program as source code}
\begin{lsuc}{EPL-C4}
  \linkosuc{04S} 

  \lsucmeans{that you received an EPL-1.0 licensed program, application, or
  server (proapse), that you modified it, and that you are now going to
  distribute this modified version to third parties in the form of source code files or as
  a source code package.}

  \mapsToOsuc{04S}

  \begin{lsucrequires}
    \lsucmandatory{\keepLicensingElements}
    \lsucmandatory{\describeModifications}
    \lsucmandatory{\markAllModifications}
    \lsucmandatory{\giveLicenseFile}\passingFilesCorrectly
    \lsucmandatory{\organizeYourModifications\ \addHeaderToNewFiles}
    \lsucmandatory{\noWarranty{source code} \updateCopyrightScreen}
    \lsucoptional{\addToDocumentation}
  \end{lsucrequires}
 
  \begin{lsucprohibits}
    \lsucitem{\dontChangeCopyrightNotices}
    \lsucitem{\noPatentLitigation}
  \end{lsucprohibits}

\end{lsuc}

% ------------------------------------------------------------------------------
% EPL-C5
% ------------------------------------------------------------------------------
\subsection{EPL-1.0-C5: Passing a modified program as binary}
\begin{lsuc}{EPL-C5}
  \linkosuc{04B}

  \lsucmeans{that you received an EPL-1.0 licensed program, application, or
  server (proapse), that you modified it, and that you are now going to
  distribute this modified version to third parties in the form of binary files or as a
  binary package.}

  \mapsToOsuc{04B}

  \begin{lsucrequires}
    \lsucmandatory{\keepLicensingElements\ \addWhenCompiling}
    \lsucmandatory{\describeModifications}
    \lsucmandatory{\markAllModifications}
    \lsucmandatory{\organizeYourModifications}
    \lsucmandatory{\noWarranty{binary} \updateCopyrightScreen}
    \lsucmandatory{\publishSourceCode{the program}}
    \lsucmandatory{\linkToRepo}
    \lsucsourcedist{EPL-C4}
    \lsucoptional{\addToYourCopyrightNotice}
  \end{lsucrequires}

  \begin{lsucprohibits}
    \lsucitem{\dontChangeCopyrightNotices}
    \lsucitem{\noPatentLitigation}
  \end{lsucprohibits}

\end{lsuc}

% ------------------------------------------------------------------------------
% EPL-C6
% ------------------------------------------------------------------------------
\subsection{EPL-1.0-C6: Passing a modified library as independent source code}
\begin{lsuc}{EPL-C6}
  \linkosuc{08S}

  \lsucmeans{that you received an EPL-1.0 licensed code snippet, module, library,
  or plugin (snimoli), that you modified it, and that you are now going to
  distribute this modified version to third parties in the form of source code
  files or as a source code package, but without embedding it into another
  larger software unit.}

  \mapsToOsuc{08S}

  \begin{lsucrequires}
    \lsucmandatory{\keepLicensingElements}
    \lsucmandatory{\describeModifications}
    \lsucmandatory{\markAllModifications}
    \lsucmandatory{\giveLicenseFile}\passingFilesCorrectly
    \lsucmandatory{\organizeYourModifications\ \addHeaderToNewFiles}
    \lsucmandatory{\noWarranty{source code}}
    \lsucoptional{\addToDocumentation}
  \end{lsucrequires}

  \begin{lsucprohibits}
    \lsucitem{\dontChangeCopyrightNotices}
    \lsucitem{\noPatentLitigation}
  \end{lsucprohibits}

\end{lsuc}

% ------------------------------------------------------------------------------
% EPL-C7
% ------------------------------------------------------------------------------
\subsection{EPL-1.0-C7: Passing a modified library as independent binary}
\begin{lsuc}{EPL-C7}
  \linkosuc{08B}

  \lsucmeans{that you received an EPL-1.0 licensed code snippet, module, library,
  or plugin (snimoli), that you modified it, and that you are now going to
  distribute this modified version to third parties in the form of binary files
  or as a binary package but without embedding it into another larger software
  unit.}

  \mapsToOsuc{08B}

  \begin{lsucrequires}
    \lsucmandatory{\keepLicensingElements\ \addWhenCompiling}
    \lsucmandatory{\describeModifications}
    \lsucmandatory{\markAllModifications}
    \lsucmandatory{\organizeYourModifications}
    \lsucmandatory{\noWarranty{binary}}
    \lsucmandatory{\publishSourceCode{the modified library}}
    \lsucmandatory{\linkToRepo}
    \lsucsourcedist{EPL-6}
    \lsucoptional{\addToDocumentation}
  \end{lsucrequires}

  \begin{lsucprohibits}
    \lsucitem{\dontChangeCopyrightNotices}
    \lsucitem{\noPatentLitigation}
  \end{lsucprohibits}

\end{lsuc}

% ------------------------------------------------------------------------------
% EPL-C8
% ------------------------------------------------------------------------------
\subsection{EPL-1.0-C8: Passing a modified library as embedded source code}
\begin{lsuc}{EPL-C8}
  \linkosuc{10S}

  \lsucmeans{that you received an EPL-1.0 licensed code snippet, module, library,
  or plugin (snimoli), that you modified it, and that you are now going to
  distribute this modified version to third parties in the form of source code
  files or as a source code package together with another larger software unit
  which contains this code snippet, module, library, or plugin as an embedded
  component.}

  \mapsToOsuc{10S}

  \begin{lsucrequires}
    \lsucmandatory{\keepLicensingElements}
    \lsucmandatory{\describeModifications}
    \lsucmandatory{\markAllModifications}
    \lsucmandatory{\giveLicenseFile}\passingFilesCorrectly
    \lsucmandatory{\noWarranty{source code} \includeInCopyrightScreen}
    \lsucmandatory{\organizeYourModifications\ \addHeaderToNewFiles}
    \lsucoptional{\useSeparateDirectory{source code}}
    \lsucoptional{\addToYourCopyrightNotice}
  \end{lsucrequires}

  \begin{lsucprohibits}
    \lsucitem{\dontChangeCopyrightNotices}
    \lsucitem{\noPatentLitigation}
  \end{lsucprohibits}

\end{lsuc}

% ------------------------------------------------------------------------------
% EPL-C9
% ------------------------------------------------------------------------------
\subsection{EPL-1.0-C9: Passing a modified library as embedded binary}
\begin{lsuc}{EPL-C9}
  \linkosuc{10B}

  \lsucmeans{that you received an EPL-1.0 licensed code snippet, module, library,
  or plugin (snimoli), that you modified it, and that you are now going to
  distribute this modified version to third parties in the form of binary files
  or as a binary package together with another larger software unit which
  contains this code snippet, module, library, or plugin as an embedded component.}

  \mapsToOsuc{10B}

  \begin{lsucrequires}
    \lsucmandatory{\keepLicensingElements\ \addWhenCompiling}
    \lsucmandatory{\describeModifications}
    \lsucmandatory{\markAllModifications}
    \lsucmandatory{\noWarranty{binary} \includeInCopyrightScreen}
    \lsucmandatory{\publishSourceCode{the embedded library}}
    \lsucmandatory{\linkToRepo}
    \lsucmandatory{\organizeYourModifications}
    \lsucsourcedist{EPL-C8}
    \lsucoptional{\useSeparateDirectory{binary}}
    \lsucoptional{\addToYourCopyrightNotice}
  \end{lsucrequires}

  \begin{lsucprohibits}
    \lsucitem{\dontChangeCopyrightNotices}
    \lsucitem{\noPatentLitigation}
  \end{lsucprohibits}

\end{lsuc}

% ------------------------------------------------------------------------------
\subsection{Discussions and Explanations}
\label{EPLDiscussion}

The EPL-1.0 contains a succinct section \enquote{Requirements}\citeEPL{§3}
complemented by some definitions concerning a \enquote{Commercial
Distribution}\citeEPL{§4}: First, it describes what a distributor must do for
correctly distributing an Eclipse licensed program as a set of binaries. Then,
it explains, what must be done to comply with the license when distributing the
software as source code.  Finally, it lists two conditions which must be
fulfilled in any case.\citeEPL{§3}  
With respect to this structure, we can discover the following tasks:

\begin{itemize}

  \item The EPL-1.0 generally requires that \enquote{Contributors may not remove or
    alter any copyright notices contained within the Program}\citeEPL{§3} where
    the word `Contributor' has to be read as \enquote{any person or entity that
    distributes the Program}, and the word `Program' denotes the
    \enquote{initial contribution} and all its modifications.\citeEPL{§1} 
    Similar to the EUPL and at least in a very strict reading, the EPL-1.0 does not
    limit these requirements to the distribution of the software
    (\emph{2others}). But in practice it will be difficult to control the
    compliant use of the software in those cases where one uses the software
    only for oneself. But opposite to, for example, the EUPL, the EPL-1.0 clearly
    contains this interdiction. The \oslic{} solves this practical inconsistence
    duplicating the message: On the one hand, it rewrites the negative condition
    as a mandatory positive assertion for the \emph{2others} use cases (EPL-1.0-C2 --
    EPL-1.0-C9). This should emphasize the \emph{activity} to retain the copyright
    notes in exact the form one has received them. On the other hand, the \oslic{}
    inserts the corresponding interdiction into the `prohibits' section of the
    \emph{4yourself} use cases (EPL-1.0-C1 -- EPL-1.0-C9).
  
  \item Furthermore, the EPL-1.0 requires that \enquote{each Contributor must
    identify itself as the originator of its Contributions [\ldots] in a manner
    that reasonably allows subsequent Recipients to identify the originator of
    the Contribution},\citeEPL{§3} In this case, `Contribution' has to be read
    as the \enquote{initial code and documention} together with all subsequent
    modifications of these parts.\citeEPL{§1} To fulfill this condition
    faithfully, a developer must mark and describe his modifications of the
    source code within this source code; and the distributor must describe these
    modifications on the more general level of software features in a file
    sometimes called CHANGES. At a first glance, the requirement to document the
    source code modifications within the source code seems to be restricted to
    the use cases which concern the distribution of a modified EPL-1.0 software in
    the form of source code. But the EPL-1.0 allows the distribution in the form of
    binaries only if the distributor also states where one can obtain the
    correspoding code.\citeEPL{§3} So, distributing the binaries implies the
    distribution of the source code.  Therefore the \oslic{} inserts the two
    requirements as mandatory clauses into all the use cases concerning the
    distribution of a modified EPL-1.0 software (EPL-1.0-C4 -- EPL-1.0-C9).
  
  \item For all distributions in the form of source code the EPL-1.0 requires that
    the software \enquote{[\ldots] must be made available under this (Eclipse
    Public License 1.0) Agreement} and that \enquote{[\ldots] a copy of this
    Agreement must be included with each copy of the Program.}\citeEPL{§3} 
    Thus, the \oslic{} inserts a respective mandatory clause into the use cases
    (EPL-1.0-C4, EPL-1.0-C6, EPL-1.0-C8). But the EPL-1.0 is a license with a weak copyleft%
    \footnote{($\rightarrow$ \oslic, p.\ \protectionpageref{EPL})}. 
    Therefore, this conditions does not cover the overarching program which uses
    the embedded library (EPL-1.0-C8).
    
  \item Additionally, the EPL-1.0 allows to distribute the software in the form
    of binaries if the distributor \enquote{[\ldots] effectively disclaims on
    behalf of all Contributors all warranties and conditions [\ldots] (and)
    effectively excludes on behalf of all Contributors all liability for
    damages [\ldots]} in the broadest sense.\citeEPL{§3} This limitation of
    liability is very important to the EPL-1.0. Thus, it further specifies and
    explains this aspect once more in another section titled \enquote{Commercial
    Distribution}. There, this aspect is no longer focussed only on a
    distribution in the form of binaries.\citeEPL{§4} So the \oslic{} inserts a
    mandatory clause into all use cases concerning the distribution that the
    paragraph of \enquote{No Warranty}\citeEPL{§5} and the \enquote{Disclaimer
    of Liability}\citeEPL{§6} of the EPL-1.0 must explicitly be present in the
    documentation of distribution package and---if technically possible---%
    presented by the copyright screen.   
  
  \item Aside from that, the EPL-1.0 allows the distribution of the software in the
    form of binaries only if the distributor clearly \enquote{[\ldots] states that
    the source code for the program is available from such Contributor
    (distributor) [\ldots]} and if he additionally \enquote{[\ldots] informs
    licensees how to obtain it in a reasonable manner [\ldots]}\citeEPL{§3} 
    This requirement can only be fulfilled seriously if the distributor himself
    offers the source code via a repository. It is not sufficient to point to
    any external download repository in the world wide web. Thus,---for all use
    cases concerning the distribution in the form of binaries---the \oslic{}
    follows the respective requirement introduced by the EPL-1.0 (EPL-1.0-C3, EPL-1.0-C5,
    EPL-1.0-C7, EPL-1.0-C9).  
  
  \item Moreover, one has clearly to state that the previous rule implies a real
    source code distribution which therefore must follow the rules of
    distributing the software. Thus, the \oslic{} requires in all cases of a binary 
    distribution to execute also the task-lists of the respective source code
    use cases. 
 
 	\item Finally, the EPL-1.0 contains a patent clause stating that \enquote{if
 	any recipient institutes patent litigation against any entity [\ldots]
 	alleging that the Program itself [\ldots] infringes such Recipient's
 	patent(s), then such Recipient's rights granted [\ldots by the EPL-1.0] shall
 	terminate [\ldots]}\citeEPL{§7}. Based on this fact, the \oslic{} generally
 	(EPL-1.0-C1 -- EPL-1.0-C9) interdicts to legally fight against patents linked to the software.
\end{itemize}
\end{license}

%\bibliography{../../../bibfiles/oscResourcesEn}

% Local Variables:
% mode: latex
% fill-column: 80
% End:
