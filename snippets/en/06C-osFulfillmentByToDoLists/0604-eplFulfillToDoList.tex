% Telekom osCompendium 'for being included' snippet template
%
% (c) Karsten Reincke, Deutsche Telekom AG, Darmstadt 2011
%
% This LaTeX-File is licensed under the Creative Commons Attribution-ShareAlike
% 3.0 Germany License (http://creativecommons.org/licenses/by-sa/3.0/de/): Feel
% free 'to share (to copy, distribute and transmit)' or 'to remix (to adapt)'
% it, if you '... distribute the resulting work under the same or similar
% license to this one' and if you respect how 'you must attribute the work in
% the manner specified by the author ...':
%
% In an internet based reuse please link the reused parts to www.telekom.com and
% mention the original authors and Deutsche Telekom AG in a suitable manner. In
% a paper-like reuse please insert a short hint to www.telekom.com and to the
% original authors and Deutsche Telekom AG into your preface. For normal
% quotations please use the scientific standard to cite.
%
% [ Framework derived from 'mind your Scholar Research Framework' 
%   mycsrf (c) K. Reincke 2012 CC BY 3.0  http://mycsrf.fodina.de/ ]
%


%% use all entries of the bibliography
%\nocite{*}

\section{EPL licensed software}

The Eclipse Public License clearly distinguishes the distribution in the form of
source code from that in the form of binaries: First, it allows to
\enquote{distribute} Eclipse licensed programs \enquote{in source code and in
object code}\footcite[cf.][\nopage wp.\ §3]{Epl10OsiLicense2005a}. Then it
specifies under which conditions one may distribute the program as a set of
binaries\footcite[cf.][\nopage wp.\ §3 top area]{Epl10OsiLicense2005a}. One of
these conditions is -- roughly spoken -- that the distributor makes the sources
available too\footnote{\cite[cf.][\nopage wp.\ §3 mid
area]{Epl10OsiLicense2005a}. More precisely, the EPL has to be taken as a
license with weak copyleft ($\rightarrow$ OSLiC, p.\
\protectionpageref{EPL})}. The other conditions refer to the
distribution in general -- no matter what form or state is
used\footcite[cf.][\nopage wp.\ §3 bottom area]{Epl10OsiLicense2005a}. So, taken
as whole, the EPL mainly focusses on the distribution of software. Thus, for
finding the relevant, simply processable task lists, also the following EPL
specific open source use case structure\footnote{For details of the general OSUC
finder $\rightarrow$ OSLiC, pp.\ \pageref{OsucTokens} and
\pageref{OsucDefinitionTree}} can be used:
 
\tikzstyle{nodv} = [font=\small, ellipse, draw, fill=gray!10, 
    text width=2cm, text centered, minimum height=2em]

\tikzstyle{nods} = [font=\footnotesize, rectangle, draw, fill=gray!20, 
    text width=1.2cm, text centered, rounded corners, minimum height=3em]

\tikzstyle{nodb} = [font=\footnotesize, rectangle, draw, fill=gray!20, 
    text width=2.2cm, text centered, rounded corners, minimum height=3em]
    
\tikzstyle{leaf} = [font=\tiny, rectangle, draw, fill=gray!30, 
    text width=1.2cm, text centered, minimum height=6em]

\tikzstyle{edge} = [draw, -latex']

\begin{tikzpicture}[]

\node[nodv] (l71) at (4,10) {EPL};

\node[nodb] (l61) at (0,8.6) {\textit{recipient:} \\ \textbf{4yourself}};
\node[nodb] (l62) at (6.5,8.6) {\textit{recipient:} \\ \textbf{2others}};

\node[nodb] (l51) at (2.5,7) {\textit{state:} \\ \textbf{unmodified}};
\node[nodb] (l52) at (9.3,7) {\textit{state:} \\ \textbf{modified}};

\node[nods] (l41) at (1.8,5.4) {\textit{form:} \textbf{source}};
\node[nods] (l42) at (3.6,5.4) {\textit{form:} \textbf{binary}};
\node[nodb] (l43) at (6.5,5.4) {\textit{type:} \\ \textbf{proapse}};
\node[nodb] (l44) at (12,5.4) {\textit{type:} \\ \textbf{snimoli}};


\node[nods] (l31) at (5.4,3.8) {\textit{form:} \textbf{source}};
\node[nods] (l32) at (7.2,3.8) {\textit{form:} \textbf{binary}};
\node[nodb] (l33) at (10,3.8) {\textit{context:} \\ \textbf{independent}};
\node[nodb] (l34) at (13.5,3.8) {\textit{context:} \\ \textbf{embedded}};

\node[nods] (l21) at (9,2.2) {\textit{form:} \textbf{source}};
\node[nods] (l22) at (10.8,2.2) {\textit{form:} \textbf{binary}};
\node[nods] (l23) at (12.6,2.2) {\textit{form:} \textbf{source}};
\node[nods] (l24) at (14.4,2.2) {\textit{form:} \textbf{binary}};

\node[leaf] (l11) at (0,0) {\textbf{EPL-C1} \textit{using software only
for yourself}};

\node[leaf] (l12) at (1.8,0) { \textbf{EPL-C2} \textit{ distributing unmodified
software as sources}};

\node[leaf] (l13) at (3.6,0) { \textbf{EPL-C3}  \textit{ distributing unmodified
software as binaries}};

\node[leaf] (l14) at (5.4,0) { \textbf{EPL-C4}  \textit{ distributing modified
program as sources}};

\node[leaf] (l15) at (7.2,0) { \textbf{EPL-C5}  \textit{ distributing modified
program as binaries}};

\node[leaf] (l16) at (9,0) { \textbf{EPL-C6}  \textit{ distributing modified
library as independent sources}};

\node[leaf] (l17) at (10.8,0) { \textbf{EPL-C7} \textit{distributing modified
library as independent binaries}};

\node[leaf] (l18) at (12.6,0) { \textbf{EPL-C8}  \textit{distributing
modified library as embedded sources}};

\node[leaf] (l19) at (14.4,0) { \textbf{EPL-C9}  \textit{ distributing modified
library as embedded binaries}};


\path [edge] (l71) -- (l61);
\path [edge] (l71) -- (l62);
\path [edge] (l61) -- (l11);
\path [edge] (l62) -- (l51);
\path [edge] (l62) -- (l52);
\path [edge] (l51) -- (l41);
\path [edge] (l51) -- (l42);
\path [edge] (l52) -- (l43);
\path [edge] (l52) -- (l44);
\path [edge] (l41) -- (l12);
\path [edge] (l42) -- (l13);
\path [edge] (l43) -- (l31);
\path [edge] (l43) -- (l32);
\path [edge] (l44) -- (l33);
\path [edge] (l44) -- (l34);
\path [edge] (l31) -- (l14);
\path [edge] (l32) -- (l15);
\path [edge] (l33) -- (l21);
\path [edge] (l33) -- (l22);
\path [edge] (l34) -- (l23);
\path [edge] (l34) -- (l24);
\path [edge] (l21) -- (l16);
\path [edge] (l22) -- (l17);
\path [edge] (l23) -- (l18);
\path [edge] (l24) -- (l19);

\end{tikzpicture}


\subsection{EPL-C1: Using the software only for yourself}
\label{OSUC-01-EPL} \label{OSUC-03-EPL} 
\label{OSUC-06-EPL} \label{OSUC-09-EPL}

\begin{description}

\item[means] that you are going to use a received EPL licensed software only
for yourself and that you do not hand it over to any 3rd party in any sense.

\item[covers] OSUC-01, OSUC-03, OSUC-06, and OSUC-09\footnote{For details 
$\rightarrow$ OSLiC, pp.\ \pageref{OSUC-01-DEF} - \pageref{OSUC-09-DEF}}

\item[requires] no tasks in order to fulfill the conditions of the EPL 1.0
license with respect to this use case:
  \begin{itemize}
    \item You are allowed to use any kind of EPL software in any sense and in
    any context without being obliged to do anything as long as you do not
    give the software to 3rd parties.
  \end{itemize}

\item[prohibits] \ldots
\begin{itemize}
  \item to remove or to alter any copyright notices contained within the
  received software package.
\end{itemize}

\end{description}

\subsection{EPL-C2: Passing the unmodified software as source code}
\label{OSUC-02S-EPL} \label{OSUC-05S-EPL} \label{OSUC-07S-EPL} 

\begin{description}

\item[means] that you are going to distribute an unmodified version of the
received EPL software to 3rd parties -- in the form of source code files or as a
source code package. In this case it is not discriminating to distribute a
program, an application, a server, a snippet, a module, a library, or a plugin
as an independent or an embedded unit.

\item[covers] OSUC-02S, OSUC-05S, OSUC-07S\footnote{For details $\rightarrow$
OSLiC, pp.\ \pageref{OSUC-02S-DEF} - \pageref{OSUC-07S-DEF}}

\item[requires] the following tasks in order to fulfill the license conditions:
\begin{itemize}
  
  \item \textbf{[mandatory:]} Ensure that the licensing elements -- esp.\ all
  copyright notices and the disclaimer of warranty and liability -- are retained
  in your package in exact the form you have received them.
  
  \item \textbf{[mandatory:]} Give the recipient a copy of the EPL 1.0 license.
  If it is not already part of the software package, add it\footnote{For
  implementing the handover of files correctly $\rightarrow$ OSLiC, p.
  \pageref{DistributingFilesHint}}. If the licensing statement in the licensing
  file of the package does still not clearly state that the package is licensed
  under the EPL, additionally insert your own correct EPL licensing file.
  
  \item \textbf{[mandatory:]} If still not existing, integrate an explicit, very
  prominently placed 'No warranty' statement into the distributed source code
  package. Let this statement clearly say that all (other) contributors to the
  software do not take over any responsibility for the quality of the software.
  Then, add the no-warranty clause and the disclaimer of the liability of the
  EPL itself into that file.
  
  \item \textbf{[voluntary:]} Let the documentation of your distribution and/or
  your additional material also reproduce the content of the existing
  \emph{copyright notice text files}, a hint to the software name, a link to its
  homepage, and a link to the EPL 1.0 license.
\end{itemize}

\item[prohibits] \ldots
\begin{itemize}
  \item to remove or to alter any copyright notices contained within the
  received software package.
\end{itemize}

\end{description}


\subsection{EPL-C3: Passing the unmodified software as binaries} 
\label{OSUC-02B-EPL} \label{OSUC-05B-EPL} \label{OSUC-07B-EPL} 

\begin{description}
\item[means] that you are going to distribute an unmodified version of the
received EPL software to 3rd parties -- in the form of binary files or as a
bi\-na\-ry package. In this case it is not discriminating to distribute a
program, an application, a server, a snippet, a module, a library, or a plugin
as an independent or an embedded unit.

\item[covers] OSUC-02B, OSUC-05B, OSUC-07B\footnote{For details $\rightarrow$
OSLiC, pp.\ \pageref{OSUC-02B-DEF} - \pageref{OSUC-07B-DEF}}

\item[requires] the following tasks in order to fulfill the license conditions:
\begin{itemize}
  
  \item \textbf{[mandatory:]} Ensure that the licensing elements -- esp.\ all
  copyright notices and the disclaimer of warranty and liability -- are retained
  in your package in exact the form you have received them. If you compile the
  binary from the sources, ensure that all these licensing elements are also
  incorporated into the package.
  
  \item \textbf{[mandatory:]} If still not existing, integrate an explicit, very
  prominently placed 'No warranty' statement into the distributed source code
  package. Let this statement clearly say that all (other) contributors to the
  software do not take over any responsibility for the quality of the software.
  Then, add the no-warranty clause and the disclaimer of the liability of the
  EPL itself into that file.

  \item \textbf{[mandatory:]} Make the source code of the software accessible
  via a repository under your own control -- even if you did not modified it:
  Push the source code package into an internet repository and enable its
  download function. Integrate an easily to find description into your
  distribution package which explains how the code can be received from where.
  Ensure, that this repository is usable reasonably long enough.
  
  \item \textbf{[mandatory:]} Insert a prominent hint to the download repository
  into your distribution and/or your additional material.
  
  \item \textbf{[mandatory:]} Execute the to-do list of use case EPL-C2\footnote{
  Making the code accessible via a repository means distributing the software in
  the form of source code. Hence, you must also fulfill all tasks of the
  corresponding use case.}.
  
  \item \textbf{[voluntary:]} Let the documentation of your distribution and/or
  your additional material also reproduce the content of the existing
  \emph{copyright notice text files}, a hint to the software name, a link to its
  homepage, and a link to the EPL 1.0 license.
    
\end{itemize}

\item[prohibits] \ldots
\begin{itemize}
  \item to remove or to alter any copyright notices contained within the
  received software package.
\end{itemize}

\end{description}

\subsection{EPL-C4: Passing a modified program as source code}
\label{OSUC-04S-EPL} 

\begin{description}
\item[means] that you are going to distribute a modified version of the received
EPL licensed program, application, or server (proapse) to 3rd parties -- in the
form of source code files or as a source code package.
\item[covers] OSUC-04S\footnote{For details $\rightarrow$ OSLiC, pp.\
\pageref{OSUC-04S-DEF}}
\item[requires] the tasks in order to fulfill the license conditions:
\begin{itemize}
  
  \item \textbf{[mandatory:]} Ensure that the licensing elements -- esp.\ all
  copyright notices and the disclaimer of warranty and liability -- are retained
  in your package in exact the form you have received them.
  
  \item \textbf{[mandatory:]} Create a \emph{modification text file}, if such a
  notice file still does not exist. \emph{Expand} the \emph{modification text
  file} by a more general description of your modifications. Incorporate it into
  your distribution package.
  
  \item \textbf{[mandatory:]} Mark all modifications of the source code of the
  program (proapse) thoroughly -- namely within the
  modfied source code.
  
  \item \textbf{[mandatory:]} Give the recipient a copy of the EPL 1.0 license.
  If it is not already part of the software package, add it\footnote{For
  implementing the handover of files correctly $\rightarrow$ OSLiC, p.
  \pageref{DistributingFilesHint}}. If the licensing statement in the licensing
  file of the package does still not clearly state that the package is licensed
  under the EPL, additionally insert your own correct EPL licensing file.

  \item \textbf{[mandatory:]} Organize your modifications in a way that they are
  covered by the existing EPL licensing statements. If you add new source code
  files, insert a header containing your copyright line and an EPL adequate
  licensing the statement.
  
  \item \textbf{[mandatory:]} If still not existing, integrate an explicit, very
  prominently placed 'No warranty' statement into the distributed source code
  package. Let this statement clearly say that all (other) contributors to the
  software do not take over any responsibility for the quality of the software.
  Then, add the no-warranty clause and the disclaimer of the liability of the
  EPL itself into that file. Update an existing copyright screen presented by
  the program so that it shows the same information.

  \item \textbf{[voluntary:]} Let the documentation of your distribution and/or
  your additional material also reproduce the content of the existing
  \emph{copyright notice text files}, a hint to the software name, a link to its
  homepage, and a link to the EPL 1.0 license.
  
 \end{itemize}
 
\item[prohibits] \ldots
\begin{itemize}
  \item to remove or to alter any copyright notices contained within the
  received software package.
\end{itemize}

\end{description}

\subsection{EPL-C5: Passing a modified program as binary}
\label{OSUC-04B-EPL}

\begin{description}
\item[means] that you are going to distribute a modified version of the received
EPL licensed pro\-gram, application, or server (proapse) to 3rd parties -- in
the form of binary files or as a binary package.
\item[covers] OSUC-04B\footnote{For details $\rightarrow$ OSLiC, pp.\
\pageref{OSUC-04B-DEF}}
\item[requires] the following tasks in order to fulfill the license conditions:
\begin{itemize}

  \item \textbf{[mandatory:]} Ensure that the licensing elements -- esp.\ all
  copyright notices and the disclaimer of warranty and liability -- are retained
  in your package in exact the form you have received them. If you compile the
  binary from the sources, ensure that all these licensing elements are also
  incorporated into the package.

  \item \textbf{[mandatory:]} Create a \emph{modification text file}, if such a
  notice file still does not exist. \emph{Expand} the \emph{modification text
  file} by a more general description of your modifications. Incorporate it into
  your distribution package.

  \item \textbf{[mandatory:]} Mark all modifications of the source code of the
  program (proapse) thoroughly -- namely within the
  modfied source code.
  
  \item \textbf{[mandatory:]} Organize your modifications in a way that they are
  covered by the (existing) EPL licensing statements.

  \item \textbf{[mandatory:]} If still not existing, integrate an explicit, very
  prominently placed 'No warranty' statement into the distributed source code
  package. Let this statement clearly say that all (other) contributors to the
  software do not take over any responsibility for the quality of the software.
  Then, add the no-warranty clause and the disclaimer of the liability of the
  EPL itself into that file. Update an existing copyright screen presented by
  the program so that it shows the same information.

  \item \textbf{[mandatory:]} Make the source code of the modifed program
  accessible via a repository under your own control: Push the source code
  package into an internet repository and enable its download function.
  Integrate an easily to find description into your distribution package which
  explains how the code can be received from where. Ensure, that this repository
  is usable reasonably long enough.
  
  \item \textbf{[mandatory:]} Insert a prominent hint to the download repository
  into your distribution and/or your additional material.

  \item \textbf{[mandatory:]} Execute the to-do list of use case EPL-C4\footnote{
  Making the code accessible via a repository means distributing the software in
  the form of source code. Hence, you must also fulfill all tasks of the
  corresponding use case.}.
 
  \item \textbf{[voluntary:]} Let the documentation of your distribution and/or
  your additional material  also reproduce the content of the existing
  \emph{copyright notice text files}, a hint to the software name, a link to its
  homepage, and a link to the EPL 1.0 license -- especially as a subsection of
  your own copyright notice.


\end{itemize}  

\item[prohibits] \ldots
\begin{itemize}
  \item to remove or to alter any copyright notices contained within the
  received software package.
\end{itemize}

\end{description}

\subsection{EPL-C6: Passing a modified library as independent source code}
\label{OSUC-08S-EPL}

\begin{description}
\item[means] that you are going to distribute a modified version of the received
EPL licensed code snippet, module, library, or plugin (snimoli) to 3rd parties
-- in the form of source code files or as a source code package, but without
embedding it into another larger software unit.
\item[covers] OSUC-08S\footnote{For details $\rightarrow$ OSLiC, pp.\
\pageref{OSUC-08S-DEF}} 
\item[]
\item[requires] the following tasks in order to fulfill the license conditions:
\begin{itemize}

  \item \textbf{[mandatory:]} Ensure that the licensing elements -- esp.\ all
  copyright notices and the disclaimer of warranty and liability -- are retained
  in your package in exact the form you have received them.

  \item \textbf{[mandatory:]} Create a \emph{modification text file}, if such a
  notice file still does not exist. \emph{Expand} the \emph{modification text
  file} by a more general description of your modifications. Incorporate it into
  your distribution package.

  \item \textbf{[mandatory:]} Mark all modifications of the source code of the
  library (snimoli) thoroughly -- namely within
  the modfied source code.
  
  \item \textbf{[mandatory:]} Give the recipient a copy of the EPL 1.0 license.
  If it is not already part of the software package, add it\footnote{For
  implementing the handover of files correctly $\rightarrow$ OSLiC, p.
  \pageref{DistributingFilesHint}}. If the licensing statement in the licensing
  file of the package does still not clearly state that the package is licensed
  under the EPL, additionally insert your own correct EPL licensing file.
  
  \item \textbf{[mandatory:]} Organize your modifications in a way that they are
  covered by the existing EPL licensing statements. If you add new source code
  files, insert a header containing your copyright line and an EPL adequate
  licensing the statement.
  
  \item \textbf{[mandatory:]} If still not existing, integrate an explicit, very
  prominently placed 'No warranty' statement into the distributed source code
  package. Let this statement clearly say that all (other) contributors to the
  software do not take over any responsibility for the quality of the software.
  Then, add the no-warranty clause and the disclaimer of the liability of the
  EPL itself into that file.
  
  \item \textbf{[voluntary:]} Let the documentation of your distribution and/or
  your additional material  also reproduce the content of the existing
  \emph{copyright notice text files}, a hint to the software name, a link to its
  homepage, and a link to the EPL 1.0 license.

\end{itemize}

\item[prohibits] \ldots
\begin{itemize}
  \item to remove or to alter any copyright notices contained within the
  received software package.
\end{itemize}

\end{description}


\subsection{EPL-C7: Passing a modified library as independent binary}
\label{OSUC-08B-EPL}

\begin{description}
\item[means] that you are going to distribute a modified version of the received
EPL licensed code snippet, module, library, or plugin (snimoli) to 3rd parties
-- in the form of binary files or as a binary package but without embedding it
into another larger software unit.
\item[covers] OSUC-08B\footnote{For details $\rightarrow$ OSLiC, pp.\
\pageref{OSUC-08B-DEF}}
\item[requires] the following tasks in order to fulfill the license conditions:
\begin{itemize}

  \item \textbf{[mandatory:]} Ensure that the licensing elements -- esp.\ all
  copyright notices and the disclaimer of warranty and liability -- are retained
  in your package in exact the form you have received them. If you compile the
  binary from the sources, ensure that all these licensing elements are also
  incorporated into the package.

  \item \textbf{[mandatory:]} Create a \emph{modification text file}, if such a
  notice file still does not exist. \emph{Expand} the \emph{modification text
  file} by a more general description of your modifications. Incorporate it into
  your distribution.

  \item \textbf{[mandatory:]} Mark all modifications of the source code of the
  library (snimoli) thoroughly -- namely within the modfied source code.

  \item \textbf{[mandatory:]} Organize your modifications in a way that they are
  covered by the existing EPL licensing statements.
  
  \item \textbf{[mandatory:]} If still not existing, integrate an explicit, very
  prominently placed 'No warranty' statement into the distributed source code
  package. Let this statement clearly say that all (other) contributors to the
  software do not take over any responsibility for the quality of the software.
  Then, add the no-warranty clause and the disclaimer of the liability of the
  EPL itself into that file.
  
  \item \textbf{[mandatory:]} Make the source code of the modified library
  accessible via a repository under your own control: Push the source code
  package into an internet repository and enable its download function.
  Integrate an easily to find description into your distribution package which
  explains how the code can be received from where. Ensure, that this repository
  is usable reasonably long enough.
  
  \item \textbf{[mandatory:]} Insert a prominent hint to the download repository
  into your distribution and/or your additional material.
  
  \item \textbf{[mandatory:]} Execute the to-do list of use case EPL-6\footnote{
  Making the code accessible via a repository means distributing the software in
  the form of source code. Hence, you must also fulfill all tasks of the
  corresponding use case.}.
  
 
  \item \textbf{[voluntary:]} Let the documentation of your distribution and/or
  your additional material  also reproduce the content of the existing
  \emph{copyright notice text files}, a hint to the software name, a link to its
  homepage, and a link to the EPL 1.0 license -- especially as a subsection of
  your own copyright notice.
  
\end{itemize}

\item[prohibits] \ldots
\begin{itemize}
  \item to remove or to alter any copyright notices contained within the
  received software package.
\end{itemize}

\end{description}

\subsection{EPL-C8: Passing a modified library as embedded source code}
\label{OSUC-10S-EPL}

\begin{description}
\item[means] that you are going to distribute a modified version of the received
EPL licensed code snippet, module, library, or plugin (snimoli) to 3rd parties
-- in the form of source code files or as a source code package together with
another larger software unit which contains this code snippet, module, library,
or plugin as an embedded component.
\item[covers] OSUC-10S\footnote{For details $\rightarrow$ OSLiC, pp.\
\pageref{OSUC-10S-DEF}}
\item[requires] the following tasks in order to fulfill the license conditions:
\begin{itemize}

  \item \textbf{[mandatory:]} Ensure that the licensing elements -- esp.\ all
  copyright notices and the disclaimer of warranty and liability -- are retained
  in your package in exact the form you have received them.

  \item \textbf{[mandatory:]} Create a \emph{modification text file}, if such a
  notice file still does not exist. \emph{Expand} the \emph{modification text
  file} by a more general description of your modifications. Incorporate it into
  your distribution package.
  
  \item \textbf{[mandatory:]} Mark all modifications of the source code of the
  embedded library (snimoli) thoroughly -- namely within the source code.
   
  \item \textbf{[mandatory:]} Give the recipient a copy of the EPL 1.0 license.
  If it is not already part of the software package, add it\footnote{For
  implementing the handover of files correctly $\rightarrow$ OSLiC, p.
  \pageref{DistributingFilesHint}}. If the licensing statement in the licensing
  file of the package does still not clearly state that the embedded library is
  licensed under the EPL, additionally insert your own correct EPL licensing file.

  \item \textbf{[mandatory:]} If still not existing, integrate an explicit, very
  prominently placed 'No warranty' statement into the distributed source code
  package. Let this statement clearly say that all (other) contributors to the
  software do not take over any responsibility for the quality of the software.
  Then, add the no-warranty clause and the disclaimer of the liability of the
  EPL itself into that file. Let the copyright screen of your own overarching
  program show the same information -- as a specification for the embedded
  component.

  \item \textbf{[mandatory:]} Organize your modifications of the embedded
  library in a way that they are covered by the existing EPL licensing
  statements. If you add new source code files into the scope of the library,
  insert a header containing your copyright line and an EPL adequate licensing
  the statement.
       
  \item \textbf{[voluntary:]} Arrange your source code distribution so that the
  integrated EPL and the \emph{licensing files} clearly refer only to the
  embedded library and do not disturb the licensing of your own overarching
  work. It's a good tradition to keep the embedded components like libraries,
  modules, snippets, or plugins in specific directory which contains also all
  additional licensing elements.
  
  \item \textbf{[voluntary:]} Let the documentation of your distribution and/or
  your additional material also reproduce the content of the existing
  \emph{copyright notice text files}, a hint to the name of the used EPL
  licensed component, a link to its homepage, and a link to the EPL 1.0 license
  -- especially as subsection of your own copyright notice.
 
\end{itemize}

\item[prohibits] \ldots
\begin{itemize}
  \item to remove or to alter any copyright notices contained within the
  received software package.
\end{itemize}

\end{description}


\subsection{EPL-C9: Passing a modified library as embedded binary}
\label{OSUC-10B-EPL}

\begin{description}
\item[means] that you are going to distribute a modified version of the received
EPL licensed code snippet, module, library, or plugin to 3rd parties -- in the
form of binary files or as a binary package together with another larger
software unit which contains this code snippet, module, library, or plugin as an
embedded component.
\item[covers] OSUC-10B\footnote{For details $\rightarrow$ OSLiC, pp.\
\pageref{OSUC-10B-DEF}}
\item[requires] the following tasks in order to fulfill the license conditions:
\begin{itemize}

  \item \textbf{[mandatory:]} Ensure that the licensing elements -- esp.\ all
  copyright notices and the disclaimer of warranty and liability -- are retained
  in your package in exact the form you have received them. If you compile the
  binary from the sources, ensure that all these licensing elements are also
  incorporated into the package.
  
  \item \textbf{[mandatory:]} Create a \emph{modification text file}, if such a
  notice file still does not exist. \emph{Expand} the \emph{modification text
  file} by a more general description of your modifications. Incorporate it into
  your distribution package.

  \item \textbf{[mandatory:]} Mark all modifications of the source code of the
  embedded library (snimoli) thoroughly -- namely within the source code.

  \item \textbf{[mandatory:]} If still not existing, integrate an explicit, very
  prominently placed 'No warranty' statement into the distributed source code
  package. Let this statement clearly say that all (other) contributors to the
  software do not take over any responsibility for the quality of the software.
  Then, add the no-warranty clause and the disclaimer of the liability of the
  EPL itself into that file. Let the copyright screen of your own overarching
  program show the same information  -- as a specification for the embedded
  component.
  
  \item \textbf{[mandatory:]} Make the source code of the embedded library
  accessible via a repository under your own control: Push the source code
  package into an internet repository and enable its download function.
  Integrate an easily to find description into your distribution package which
  explains how the code can be received from where. Ensure, that this repository
  is usable reasonably long enough.
  
  \item \textbf{[mandatory:]} Insert a prominent hint to the download repository
  into your distribution and/or your additional material.
  
  \item \textbf{[mandatory:]} Execute the to-do list of use case EPL-C8\footnote{
  Making the code accessible via a repository means distributing the software in
  the form of source code. Hence, you must also fulfill all tasks of the
  corresponding use case.}.
  
  \item \textbf{[voluntary:]} Arrange your binary distribution so that the
  integrated EPL and the \emph{licensing files} clearly refer only to the
  embedded library and do not disturb the licensing of your own overarching
  work. It's a good tradition to keep the embedded components like libraries,
  modules, snippets, or plugins in specific directory which contains also all
  additional licensing elements.
  
  \item \textbf{[mandatory:]} Organize your modifications of the embedded
  library in a way that they are covered by the existing EPL licensing
  statements. 
  
  \item \textbf{[voluntary:]} Let the documentation of your distribution and/or
  your additional material  also reproduce the content of the existing
  \emph{copyright notice text files}, a hint to the name of the used EPL
  licensed component, a link to its homepage, and a link to the EPL 1.0 license
  -- especially as subsection of your own copyright notice.
  
\end{itemize}

\item[prohibits] \ldots
\begin{itemize}
  \item to remove or to alter any copyright notices contained within the
  received software package.
\end{itemize}

\end{description}

\subsection{Discussions and Explanations}

The EPL offers a lean section
\enquote{Requirements}\footcite[cf.][\nopage wp.\ §3]{Epl10OsiLicense2005a}
completed by some definitions concerning a \enquote{Commercial
Distribution}\footcite[cf.][\nopage wp.\ §4]{Epl10OsiLicense2005a}: First it
describes, what a distributor must do for correctly distributing an Eclipse
licensed program as a set of binaries. Then it describes, what must be done for
compliantly distributing the software as source code. Finally it lists two
conditions which must be fulfilled in any case\footcite[cf.][\nopage wp.\
§3]{Epl10OsiLicense2005a}. With respect to this structure, we can detect the
following tasks:

\begin{itemize}

  \item The EPL generally requires that \enquote{Contributors may not remove or
  alter any copyright notices contained within the
  Program}\footcite[cf.][\nopage wp.\ §3]{Epl10OsiLicense2005a} whereas -- on the
  one hand -- the word 'Contributor' has to be read as \enquote{any person or
  entity that distributes the Program}, and -- on the other hand -- the word
  'Program' denotes the \enquote{initial contribution} and all its
  modifications\footcite[cf.][\nopage wp.\ §1]{Epl10OsiLicense2005a}. Similar to
  the EUPL and at least in a very strict reading, also the EPL does not limit
  these requirements to the distribution of the software (\emph{2others}). But
  practically, it will be difficult to control the compliant use of the software
  in those cases where one uses the software only for oneself. But in opposite
  to -- for example -- the EUPL, the EPL clearly contains this interdiction. The
  OSLiC solves this practical inconsistence duplicating the message: On the one
  hand, it rewrites the negative condition as a mandatory positive assertion for
  the \emph{2others} use cases (EPL-C2 - EPL-C9). This should emphasize the
  \emph{activity} to retain the copyright notes in exact the form one has
  received them. On the other hand, the OSLiC inserts the interdictios into the
  'prohibits' section of the \emph{4yourself} use cases (EPL-C1 - EPL-C9).
  
  \item Furthermore, the EPL requires that \enquote{each Contributor must
  identify itself as the originator of its Contributions [\ldots] in a manner
  that reasonably allows subsequent Recipients to identify the originator of the
  Contribution}\footcite[cf.][\nopage wp.\ §3]{Epl10OsiLicense2005a}, In this
  case, 'Contribution' has to be read as the \enquote{initial code and
  documention} together with all subsequent modifications of these
  parts\footcite[cf.][\nopage wp.\ §1]{Epl10OsiLicense2005a}. For fulfilling this
  condition faithfully, a developer must mark and describe his modifications of
  a source code within this source code; and the distributor must describe these
  modifications on the more general level of software features in a file
  sometimes called CHANGES. On a first glance, the requirement to document the
  source code modifications within the source code seems to be restricted to the
  use cases which concern the distribution of a modified EPL software in the
  form of source code. But the EPL allows the distribution in the form of
  binaries only if the distributor also states where one can obtain the
  correspoding code\footcite[cf.][\nopage wp.\ §3]{Epl10OsiLicense2005a}. So,
  distributing the binaries implies the distribution of the source code.
  Therefore the OSLiC inserts the two requirements as mandatory clauses into all
  the use cases concerning the distribution of a modified EPL software (EPL-C4 -
  EPL-C9).
  
  \item For all distributions in the form of source code the EPL requires that
  the software \enquote{[\ldots] must be made available under this (Eclipse
  Public License 1.0) Agreement} and that \enquote{[\ldots] a copy of this
  Agreement must be included with each copy of the
  Program}\footcite[cf.][\nopage wp.\ §3]{Epl10OsiLicense2005a}. Thus, the OSLiC
  inserts a respective mandatory clause into the use cases (EPL-C4, EPL-C6,
  EPL-C8). But the EPL is a license with a weak copyleft\footnote{($\rightarrow$
  OSLiC, p.\ \protectionpageref{EPL})}. Therefore, this conditions
  does not cover the overarching program which uses the embedded library (EPL-C8)
  
  \item Additionally, the EPL allows to distribute the software in the form
  of binaries if the distributor \enquote{[\ldots] effectively disclaims on
  behalf of all Contributors all warranties and conditions [\ldots] (and)
  effectively excludes on behalf of all Contributorsall liability for damages
  [\ldots]} -- namely in a very broad sense\footcite[cf.][\nopage wp.\
  §3]{Epl10OsiLicense2005a}. This delimitation is very important for the EPL.
  Thus, it subspecifies and explains this aspect once more in a special section
  titled \enquote{Commercial Distribution}. There, this aspect is no longer only
  focussed on a distribution in the form of binaries\footcite[cf.][\nopage wp.\
  §4]{Epl10OsiLicense2005a}. So the OSLiC inserts a mandatory clause into all
  use cases concerning the distribution that the paragraph of \enquote{No
  Warranty}\footcite[cf.][\nopage wp.\ §5]{Epl10OsiLicense2005a} and the
  \enquote{Disclaimer of Liability}\footcite[cf.][\nopage wp.\
  §6]{Epl10OsiLicense2005a} of the EPL must explicitly be presented in and by
  the documentation of distribution package
  and -- if technically possible -- by the copyright screen.
  
  \item Aside from that, the EPL allows the distribution of the software in the
  form of binaries only if the distributor clearly \enquote{[\ldots] states that
  the source code for the program is available from such Contributor
  (distributor) [\ldots]} and if he additionally \enquote{[\ldots] informs
  licensees how to obtain it in a reasonable manner
  [\ldots]}\footcite[cf.][\nopage wp.\ §3]{Epl10OsiLicense2005a}. This
  requirement can only be fulfilled seriously if the distributor himself offers
  the source code via repository. It is not sufficient to point to any external
  download repository in the world wide web. Thus, -- for all use cases
  concerning the distribution in the form of binaries -- the OSLiC follows the
  respective requirement introduced by the EPL (EPL-C3, EPL-C5, EPL-C7, EPL-C9).
  
  \item Finally, one has clearly to state that this rule above evokes a real
  source code distribution which therefore must follow the rules of distributing
  the software. Thus, the OSLiC requires in all cases of a binary distribution
  to execute also the task-lists of the respective source code use cases.
 
\end{itemize}








%\bibliography{../../../bibfiles/oscResourcesEn}
