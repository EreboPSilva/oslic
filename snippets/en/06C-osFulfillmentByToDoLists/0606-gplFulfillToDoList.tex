% Telekom osCompendium 'for being included' snippet template
%
% (c) Karsten Reincke, Deutsche Telekom AG, Darmstadt 2011
%
% This LaTeX-File is licensed under the Creative Commons Attribution-ShareAlike
% 3.0 Germany License (http://creativecommons.org/licenses/by-sa/3.0/de/): Feel
% free 'to share (to copy, distribute and transmit)' or 'to remix (to adapt)'
% it, if you '... distribute the resulting work under the same or similar
% license to this one' and if you respect how 'you must attribute the work in
% the manner specified by the author ...':
%
% In an internet based reuse please link the reused parts to www.telekom.com and
% mention the original authors and Deutsche Telekom AG in a suitable manner. In
% a paper-like reuse please insert a short hint to www.telekom.com and to the
% original authors and Deutsche Telekom AG into your preface. For normal
% quotations please use the scientific standard to cite.
%
% [ Framework derived from 'mind your Scholar Research Framework' 
%   mycsrf (c) K. Reincke 2012 CC BY 3.0  http://mycsrf.fodina.de/ ]
%


%% use all entries of the bibliography
%\nocite{*}

\section{GPL licensed software}

Both versions of the GNU General Public License explicitly distinguish the
distribution of the source code from that of the binaries: On the one hand, the
GPL-2.0 mainly talks about copying and distributing the source
code,\citeGPLtwo{§1, §2} but also mentions the specific conditions for
\enquote{[\ldots] (copying) and (distributing) the Program [\ldots] in object
code or executable form [\ldots]}\citeGPLtwo{§3} On the other hand, the GPL-3.0
describes the \enquote{Basic Permissions} and the conditions for
\enquote{Conveying Verbatim Copies} or for \enquote{Conveying Modified Source
Versions}\citeGPLtwo{§2, §4, §5} before it explains the rules for
\enquote{Conveying Non-Source-Forms}.\citeGPLtwo{§2, §4, §5}  

GPL-2.0 and GPL-3.0 mainly talk about copying \emph{and} distributing the
software; private use is nearly completely unspecified: The GPL-2.0 lists its
`restrictions' only with respect to the act of copying \emph{and} distributing
\enquote{copies of the program}\citeGPLtwo{§1, §2, §4 et passim; emphasize by
KR} while the GPL-3.0 explicitly specifies that one \enquote{[\ldots] may
make, run and propagate covered works that (one does) not convey, without
conditions so long as (the) license otherwise remains in
force.}\citeGPLthree{§2} 

As licenses with a strong copyleft, they require that any application that
contains a GPL-licensed library must itself be licensed under the same
conditions as the library.
 
Finally, the GPL-2.0 and the GPL-3.0 aim for the same results and share the
same spirit by requiring nearly the same task to be performed for fulfilling the
license conditions.  Therefore it is appropriate to cover both versions in the
same chapter and to offer a common specialized GPL open source use case
structure for quickly finding the appropriate task list.%
  \footnote{For details of the general OSUC finder $\rightarrow$ \oslic,
    pp.\ \pageref{OsucTokens} and \pageref{OsucDefinitionTree}}
However, the task lists themselves will be kept separate.

In the following diagram, GPL-*-C1 (GPL-*-C2, \ldots, GPL-*-CB) is either
GPL-2.0-C1 (and so forth), if you are looking for the GPL-2.0 use case, or
GPL-3.0-C1, \ldots for the GPL-3.0 use case.

\tikzstyle{nodv} = [font=\scriptsize, ellipse, draw, fill=gray!10, 
    text width=2cm, text centered, minimum height=2em]

\tikzstyle{nods} = [font=\tiny, rectangle, draw, fill=gray!20, 
    text width=1cm, text centered, rounded corners, minimum height=3em]

\tikzstyle{nodb} = [font=\tiny, rectangle, draw, fill=gray!20, 
    text width=1.5cm, text centered, rounded corners, minimum height=3em]
    
\tikzstyle{leaf} = [font=\tiny, rectangle, draw, fill=gray!30, 
    text width=1.2cm, text centered, minimum height=6em]

\tikzstyle{slimleaf} = [font=\tiny, rectangle, draw, fill=gray!30, 
    text width=1cm, text centered, minimum height=6em]


\tikzstyle{edge} = [draw, -latex']

\begin{tikzpicture}[]

\node[nodv] (l801) at (4,11.8) {GPL};

\node[nodv] (l701) at (0,10.2) {2.0};
\node[nodv] (l702) at (7.5,10.2) {3.0};


\node[nodb] (l601) at (0,8.6) {\textit{recipient:} \\ \textbf{4yourself}};
\node[nodb] (l602) at (7.5,8.6) {\textit{recipient:} \\ \textbf{2others}};

\node[nodb] (l501) at (4,7) {\textit{state:} \\ \textbf{unmodified}};
\node[nodb] (l502) at (11,7) {\textit{state:} \\ \textbf{modified}};

\node[nodb] (l401) at (2.25,5.4) {\textit{type:} \\ \textbf{proapse or snimoli}};
\node[nodb] (l402) at (5.4,5.4) {\textit{type:} \\ \textbf{snimoli}};
\node[nodb] (l403) at (8.4,5.4) {\textit{type:} \\ \textbf{proapse}};
\node[nodb] (l404) at (12.8,5.4) {\textit{type:} \\ \textbf{snimoli}};


\node[nodb] (l301) at (2.25,3.8) {\textit{context:} \\ \textbf{independent}};
\node[nodb] (l302) at (5.4,3.8) {\textit{context:} \\ \textbf{embedded}};
\node[nodb] (l303) at (8.4,3.8) {\textit{context:} \\ \textbf{independent}};
\node[nodb] (l304) at (11.3,3.8) {\textit{context:} \\ \textbf{independent}};
\node[nodb] (l305) at (14.3,3.8) {\textit{context:} \\ \textbf{embedded}};

\node[nods] (l201) at (1.45,2.2) {\textit{form:} \textbf{source}};
\node[nods] (l202) at (3.0,2.2) {\textit{form:} \textbf{binary}};
\node[nods] (l203) at (4.6,2.2) {\textit{form:} \textbf{source}};
\node[nods] (l204) at (6.2,2.2) {\textit{form:} \textbf{binary}};
\node[nods] (l205) at (7.7,2.2) {\textit{form:} \textbf{source}};
\node[nods] (l206) at (9.1,2.2) {\textit{form:} \textbf{binary}};
\node[nods] (l207) at (10.5,2.2) {\textit{form:} \textbf{source}};
\node[nods] (l208) at (11.9,2.2) {\textit{form:} \textbf{binary}};
\node[nods] (l209) at (13.4,2.2) {\textit{form:} \textbf{source}};
\node[nods] (l210) at (15.0,2.2) {\textit{form:} \textbf{binary}};

\node[slimleaf] (l101) at (0,0) {
  \textbf{GPL-*-C1} 
  \textit{using software only for yourself}};

\node[leaf] (l102) at (1.45,0) { 
  \textbf{GPL-*-C2} 
  \textit{distributing unmodified software as independent sources}};

\node[leaf] (l103) at (3.0,0) { 
  \textbf{GPL-*-C3}  
  \textit{distributing unmodified software as independent binaries}};

\node[leaf] (l104) at (4.6,0) { 
  \textbf{GPL-*-C4} 
  \textit{distributing unmodified library as embedded sources}};

\node[leaf] (l105) at (6.2,0) { 
  \textbf{GPL-*-C5}  
  \textit{distributing unmodified library as embedded binaries}};

\node[slimleaf] (l106) at (7.7,0) { 
  \textbf{GPL-*-C6}  
  \textit{distributing modified program as sources}};

\node[slimleaf] (l107) at (9.1,0) { 
  \textbf{GPL-*-C7}  
  \textit{distributing modified program as binaries}};

\node[slimleaf] (l108) at (10.5,0) { 
  \textbf{GPL-*-C8}  
  \textit{distributing modified library as independent sources}};

\node[slimleaf] (l109) at (11.9,0) { 
  \textbf{GPL-*-C9}
  \textit{distributing modified library as independent binaries}};

\node[leaf] (l110) at (13.4,0) { 
  \textbf{GPL-*-CA}  
  \textit{distributing modified library as embedded sources}};

\node[leaf] (l111) at (15,0) { 
  \textbf{GPL-*-CB}  
  \textit{ distributing modified library as embedded binaries}};

\path [edge] (l801) -- (l701);
\path [edge] (l801) -- (l702);
\path [edge] (l701) -- (l601);
\path [edge] (l701) -- (l602);
\path [edge] (l702) -- (l601);
\path [edge] (l702) -- (l602);

\path [edge] (l602) -- (l501);
\path [edge] (l602) -- (l502);

\path [edge] (l501) -- (l401);
\path [edge] (l501) -- (l402);
\path [edge] (l502) -- (l403);
\path [edge] (l502) -- (l404);

\path [edge] (l401) -- (l301);
\path [edge] (l402) -- (l302);
\path [edge] (l403) -- (l303);
\path [edge] (l404) -- (l304);
\path [edge] (l404) -- (l305);

\path [edge] (l301) -- (l201);
\path [edge] (l301) -- (l202);
\path [edge] (l302) -- (l203);
\path [edge] (l302) -- (l204);
\path [edge] (l303) -- (l205);
\path [edge] (l303) -- (l206);
\path [edge] (l304) -- (l207);
\path [edge] (l304) -- (l208);
\path [edge] (l305) -- (l209);
\path [edge] (l305) -- (l210);

\path [edge] (l601) -- (l101);
\path [edge] (l201) -- (l102);
\path [edge] (l202) -- (l103);
\path [edge] (l203) -- (l104);
\path [edge] (l204) -- (l105);
\path [edge] (l205) -- (l106);
\path [edge] (l206) -- (l107);
\path [edge] (l207) -- (l108);
\path [edge] (l208) -- (l109);
\path [edge] (l209) -- (l110);
\path [edge] (l210) -- (l111);

\end{tikzpicture}

%% ============================================================================= 
%% Common Building Blocks

% ------------------------------------------------------------------------------
% Common description of license specific use cases

\newcommand{\useCaseOne}[1]{\lsucmeans{that you received GPL-#1 licensed
    software that you will use it only for yourself and that you do not hand 
    over to any third party in any sense.}}

\newcommand{\useCaseTwo}[1]{\lsucmeans{that you received GPL-#1 licensed
    software that you are now going to distribute to third parties as an
    independent unit and in the form of unmodified source code files or as an
    unmodified source code package. In this case it makes no difference if you
    distribute a program, an application, a server, a snippet, a module, a
    library, or a plugin.}}

\newcommand{\useCaseThree}[1]{\lsucmeans{that you received GPL-#1 licensed
    software that you are now going to distribute to third parties as an
    independent unit and in the form of unmodified binary files or as an
    unmodified binary package. In this case it does not matter if you distribute
    a program, an application, a server, a snippet, a module, a library, or a
    plugin.}}

\newcommand{\useCaseFour}[1]{\lsucmeans{that you received a GPL-#1 licensed
    snippet, module or library that you are now going to distribute to third
    parties as an embedded component of a larger unit and in the form of
    unmodified source code files or as an unmodified source code package.}}

\newcommand{\useCaseFive}[1]{\lsucmeans{that you received a GPL-#1 licensed
    snippet, module or library that you are now going to distribute to third
    parties as an embedded component of a larger unit and in the form of
    unmodified binary files or as unmodified binary package.}}

\newcommand{\useCaseSix}[1]{\lsucmeans{that you received a GPL-#1 licensed
    program, application, or server (proapse), that you modified it, and that
    you are now going to distribute this modified version to third parties in
    the form of source code files or as a source code package.}}

\newcommand{\useCaseSeven}[1]{\lsucmeans{that you received a GPL-#1 licensed
    program, application, or server (proapse), that you modified it, and that
    you are now going todistribute this modified version to third parties in the
    form of binary files or as a binary package.}}

\newcommand{\useCaseEight}[1]{\lsucmeans{that you received a GPL-#1 licensed
    code snippet, module, library, or plugin (snimoli), that you modified it,
    and that you are now going to distribute this modified version to third
    parties in the form of source code files or as a source code package, but
    without embedding it into another larger software unit.}}

\newcommand{\useCaseNine}[1]{\lsucmeans{that you received a GPL-#10 licensed
    code snippet, module, library, or plugin (snimoli), that you modified it,
    and that you are now going to distribute this modified version to third
    parties in the form of binary files or as a binary package but without
    embedding it into another larger software unit.} }

\newcommand{\useCaseA}[1]{\lsucmeans{that you received a GPL-#1 licensed code
    snippet, module, library, or plugin (snimoli), that you modified it, and
    that you are now going to distribute this modified version to third parties
    in the form of source code files or as a source code package together with
    another larger software unit which contains this code snippet, module,
    library, or plugin as an embedded component.}}

\newcommand{\useCaseB}[1]{\lsucmeans{that you received a GPL-#1 licensed code
    snippet, module, library, or plugin (snimoli), that you modified it, and
    that you are now going to distribute this modified version to third
    partiesin the form of binary files or as a binary package together with
    another larger software unit which contains this code snippet, module,
    library, or plugin as an embedded component.}}

\newcommand{\coversOne}{\coversOsucs{OSUC-01, OSUC-03, OSUC-06, and OSUC-09}{01}{09}}
\newcommand{\coversTwo}{\coversOsucs{OSUC-02S, OSUC-05S}{02S}{05S}}
\newcommand{\coversThree}{\coversOsucs{OSUC-02B, OSUC-05B}{02B}{05B}}
\newcommand{\coversFour}{\mapsToOsuc{07S}}
\newcommand{\coversFive}{\mapsToOsuc{07B}}
\newcommand{\coversSix}{\mapsToOsuc{04S}}
\newcommand{\coversSeven}{\mapsToOsuc{04B}}
\newcommand{\coversEight}{\mapsToOsuc{08S}}
\newcommand{\coversNine}{\mapsToOsuc{08B}}
\newcommand{\coversA}{\mapsToOsuc{10S}}
\newcommand{\coversB}{\mapsToOsuc{10B}}

% ------------------------------------------------------------------------------
% Keep license elements 

\newcommand{\keepLicenseElements}{Ensure that the licensing elements (especially
  all notices that refer to the GPL-\ver{} and to the absence of any warranty) are
  retained in your package in the form you have received them.}

% ------------------------------------------------------------------------------
% Make sure that the copyright notice and disclaimer of warranty are present

% GPL-3.0
\newcommand{\auxThreeEnsureCopyrightNotice}[1]{Ensure that the
  distributed #1 package contains a conspicuous, easy to find copyright notice.
  If this element is missing, add a new file containing the main copyright
  notice.} 

% GPL-2.0
\newcommand{\auxTwoEnsureCopyrightNotice}[1]{Ensure that the distributed
  #1 package contains a conspicuous, easy to find copyright notice and
  disclaimer of warranty. If these elements are missing, add a new file
  containing the main copyright notice and the disclaimer of warranty in the
  form which is textually defined by the GPL-2.0 license itself. (Yes, repeat
  the disclaimer although it is also part of the license itself and although you
  are required to hand the license itself over to the receiver.)}

\newcommand{\gpltwoEnsureCopyrightNoticeSource}{%
  \auxTwoEnsureCopyrightNotice{source code}}

\newcommand{\gpltwoEnsureCopyrightNoticeBinary}{%
  \auxTwoEnsureCopyrightNotice{binary}}

\newcommand{\gplthreeEnsureCopyrightNoticeSource}{%
  \auxThreeEnsureCopyrightNotice{source code}}

\newcommand{\gplthreeEnsureCopyrightNoticeBinary}{%
  \auxThreeEnsureCopyrightNotice{binary}}

% ------------------------------------------------------------------------------
% Give a copy of the license to the recipient of the software

\newcommand{\giveLicense}{Give the recipient a copy of the GPL-\ver{} license.
  If it is not already part of the software package, add it.}

% ------------------------------------------------------------------------------
% Keep all copyright notices intact

\newcommand{\retainCopyrightNotices}{Retain all existing copyright notices.}

% ------------------------------------------------------------------------------
%

\newcommand{\addToDocumentation}{Let the documentation of your distribution
  and/or your additional material also reproduce the content of the existing
  copyright notices, a hint to the software name, a link to its homepage, the
  respective disclaimer of warranty, and a link to the GPL-\ver.}

% ------------------------------------------------------------------------------
% Publish the source code

\newcommand{\auxSourceRepository}{Push the source code package into a
  repository under your control and make it downloadable via the internet.
  Ensure, that this repository is online for at least 3 years after you ceased
  distributing the software package.}

\newcommand{\makeUnmodifiedSourceAvailable}{Make the source code of the
  distributed software publicly available (even though you did not modify it):
  \auxSourceRepository} 

\newcommand{\makeModifiedSourceAvailable}{Make the source code of the
  distributed software publicly available: \auxSourceRepository} 

\newcommand{\makeAllSourcesAvailable}{Make the complete source code of the
  program embedding the library publicly available (and, therefore, also the
  source code of the library itself): \auxSourceRepository}

% ------------------------------------------------------------------------------
% Explain where to find the sources

\newcommand{\describeHowToGetSource}{Insert a easy to find description into the
  distribution package that explains how and where the code can be retrieved.}

% ------------------------------------------------------------------------------
% Create and update the modification text file

\newcommand{\createChangelog}{Create a \emph{modification text file,} if such a
  file does not yet exist. \emph{Add} a description of your modifications on a
  functional level to the \emph{modification text file.}}

% ------------------------------------------------------------------------------
%

\newcommand{\auxCopyrightDialogContent}{Let it reproduce the content of the
  existing copyright notices, the software name, a link to its homepage, the
  respective disclaimer of warranty, and a link to the GPL-\ver.}

\newcommand{\addToCopyrightDialogLib}{Let the copyright dialog of the on-top
  development clearly say, that it uses the GPL licensed library and that it is
  itself licensed under the GPL-\ver, too. \auxCopyrightDialogContent}

\newcommand{\addToCopyrightDialogApp}{Let the copyright dialog of the program
  clearly say that it is a GPL licensed program. 
  \auxCopyrightDialogContent\ 
  If these conditions are not already met, add the missing elements.}

% ------------------------------------------------------------------------------
% Make sure, licensing statements apply to enclosing program

\newcommand{\auxArrange}[1]{Arrange the #1 of the on-top development in a way
  that they are covered by the GPL-\ver{} licensing statements.} 

\newcommand{\arrangeEnclosingBinaries}{%
  \auxArrange{the binaries of the on-top development}}

\newcommand{\arrangeEnclosingSources}{%
  \auxArrange{the sources of the on-top development}}

% ------------------------------------------------------------------------------
% make sure licensing statements apply to your modifications

\newcommand{\auxArrangeChanges}[1]{Arrange your modifications of #1 in a way
  that they are covered by existing GPL licensing statements. If you add new
  source code files to the #1, insert a header containing your copyright line
  and a licensing statement in the form recommended by the GPL.}

\newcommand{\arrangeProgramChanges}{\auxArrangeChanges{program}}
\newcommand{\arrangeLibraryChanges}{\auxArrangeChanges{library}}
\newcommand{\arrangeEmbeddedChanges}{\auxArrangeChanges{embedded library}}

\newcommand{\howToApplyTheseTerms}{%
  \footnote{For details see section `How to Apply These Terms to Your New
    Programs' in the GPL-\ver{} license.}} 


% ------------------------------------------------------------------------------
%

\newcommand{\auxMarkChanges}[1]{Mark all modifications of the source code #1
  thoroughly within the source code and include the date of the modification.}

\newcommand{\markEmbeddedModifications}{%
  \auxMarkChanges{of the embedded library (snimoli)}}

\newcommand{\markLibraryModifications}{%
  \auxMarkChanges{of the library (snimoli)}}

\newcommand{\markProgramModifications}{
  \auxMarkChanges{the program (proapse)}}

% ------------------------------------------------------------------------------
% Forbid patent litigation

\newcommand{\noPatentLitigation}{%
  to institute a patent litigation against anyone alleging that the software
  constitutes patent infringement.} 

%% ============================================================================= 
%% GPL-2.0 Use Cases

\newcommand{\ver}{2.0}

\begin{license}{GPL2} 
\licensename{GPL-2.0}
\licensespec{General Public License Version 2}
\licenseabbrev{GPL}
%\licenseversion{2.0}

% ------------------------------------------------------------------------------
\subsection{GPL-\ver-C1: Using the software only for yourself}
\begin{lsuc}{GPL-\ver-C1}
  \linkosuc{01} 
  \linkosuc{03}
  \linkosuc{06} 
  \linkosuc{09}

  \useCaseOne{\ver}
  \coversOne

  \begin{lsucrequiresnothing}
    \lsucitem{You are allowed to use any kind of GPL software in any sense and in
      any context without being obliged to do anything as long as you do not
      give the software to third parties.}
  \end{lsucrequiresnothing}

  \lsucprohibitsnothing
\end{lsuc}

% ------------------------------------------------------------------------------
\subsection{GPL-\ver-C2: Passing the unmodified software as independent sources}
\begin{lsuc}{GPL-\ver-C2}
  \linkosuc{02S}
  \linkosuc{05S}

  \useCaseTwo{\ver}
  \coversTwo

  \begin{lsucrequires}
    \lsucmandatory{\keepLicenseElements}
    \lsucmandatory{\gpltwoEnsureCopyrightNoticeSource}
    \lsucmandatory{\giveLicense}\passingFilesCorrectly
    \lsucmandatory{\retainCopyrightNotices}
    \lsucoptional{\addToDocumentation}
  \end{lsucrequires}

  \lsucprohibitsnothing
\end{lsuc}

% ------------------------------------------------------------------------------
\subsection{GPL-\ver-C3: Passing the unmodified software as independent binaries} 
\begin{lsuc}{GPL-\ver-C3}
  \linkosuc{02B} 
  \linkosuc{05B}

  \useCaseThree{\ver}
  \coversThree

  \begin{lsucrequires}
    \lsucmandatory{\keepLicenseElements}
    \lsucmandatory{\gpltwoEnsureCopyrightNoticeBinary}
    \lsucmandatory{\giveLicense}\passingFilesCorrectly  
    \lsucmandatory{\makeUnmodifiedSourceAvailable}
    \lsucmandatory{\describeHowToGetSource}
    \lsucmandatory{\retainCopyrightNotices}
    \lsucsourcedist{GPL-\ver-C2}
    \lsucoptional{\addToDocumentation}
  \end{lsucrequires}

  \lsucprohibitsnothing
\end{lsuc}

% ------------------------------------------------------------------------------
\subsection{GPL-\ver-C4: Passing the unmodified library as embedded sources}
\begin{lsuc}{GPL-\ver-C4}
  \linkosuc{07S} 

  \useCaseFour{\ver}
  \coversFour

  \begin{lsucrequires}
    \lsucmandatory{\keepLicenseElements}
    \lsucmandatory{\gpltwoEnsureCopyrightNoticeSource}
    \lsucmandatory{\giveLicense}\passingFilesCorrectly
    \lsucmandatory{\retainCopyrightNotices}
    \lsucmandatory{\addToCopyrightDialogLib}
    \lsucmandatory{\arrangeEnclosingSources}
    \lsucoptional{\addToDocumentation}
  \end{lsucrequires}

  \lsucprohibitsnothing
\end{lsuc}

% ------------------------------------------------------------------------------
\subsection{GPL-\ver-C5: Passing the unmodified library as embedded binaries} 
\begin{lsuc}{GPL-\ver-C5}
  \linkosuc{07B} 

  \useCaseFive{\ver}
  \coversFive

  \begin{lsucrequires}
    \lsucmandatory{\keepLicenseElements}
    \lsucmandatory{\gpltwoEnsureCopyrightNoticeBinary}
    \lsucmandatory{\giveLicense}\passingFilesCorrectly
    \lsucmandatory{\makeAllSourcesAvailable}
    \lsucmandatory{\describeHowToGetSource}
    \lsucmandatory{\addToCopyrightDialogLib}
    \lsucmandatory{\arrangeEnclosingBinaries}
    \lsucmandatory{\retainCopyrightNotices}
    \lsucsourcedist{GPL-\ver-C4}
    \lsucoptional{\addToDocumentation}
  \end{lsucrequires}

  \lsucprohibitsnothing
\end{lsuc}

% ------------------------------------------------------------------------------
\subsection{GPL-\ver-C6: Passing a modified program as source code}
\begin{lsuc}{GPL-\ver-C6}
  \linkosuc{04S} 

  \useCaseSix{\ver}
  \coversSix

  \begin{lsucrequires}
    \lsucmandatory{\keepLicenseElements}
    \lsucmandatory{\gpltwoEnsureCopyrightNoticeSource}
    \lsucmandatory{\giveLicense}\passingFilesCorrectly
    \lsucmandatory{\retainCopyrightNotices}
    \lsucmandatory{\addToCopyrightDialogApp}
    \lsucmandatory{\markProgramModifications}
    \lsucmandatory{\arrangeProgramChanges}\howToApplyTheseTerms
    \lsucoptional{\createChangelog}
    \lsucoptional{\addToDocumentation}
  \end{lsucrequires}

  \lsucprohibitsnothing
\end{lsuc}

% ------------------------------------------------------------------------------
\subsection{GPL-\ver-C7: Passing a modified program as binary}
\begin{lsuc}{GPL-\ver-C7}
  \linkosuc{04B}

  \useCaseSeven{\ver}
  \coversSeven

  \begin{lsucrequires}
    \lsucmandatory{\keepLicenseElements}
    \lsucmandatory{\gpltwoEnsureCopyrightNoticeBinary}
    \lsucmandatory{\giveLicense}\passingFilesCorrectly
    \lsucmandatory{\retainCopyrightNotices}
    \lsucmandatory{\markProgramModifications}
    \lsucmandatory{\addToCopyrightDialogApp}
    \lsucmandatory{\arrangeProgramChanges}\howToApplyTheseTerms
    \lsucmandatory{\makeModifiedSourceAvailable}
    \lsucmandatory{\describeHowToGetSource}
    \lsucsourcedist{GPL-\ver-C6}
    \lsucoptional{\createChangelog}
    \lsucoptional{\addToDocumentation}
  \end{lsucrequires}

  \lsucprohibitsnothing
\end{lsuc}

% ------------------------------------------------------------------------------
\subsection{GPL-\ver-C8: Passing a modified library as independent source code}
\begin{lsuc}{GPL-\ver-C8}
  \linkosuc{08S}

  \useCaseEight{\ver}
  \coversEight

  \begin{lsucrequires}
    \lsucmandatory{\keepLicenseElements}
    \lsucmandatory{\gpltwoEnsureCopyrightNoticeSource}
    \lsucmandatory{\giveLicense}\passingFilesCorrectly
    \lsucmandatory{\retainCopyrightNotices}
    \lsucmandatory{\markLibraryModifications}
    \lsucmandatory{\arrangeLibraryChanges}\howToApplyTheseTerms
    \lsucoptional{\createChangelog}
    \lsucoptional{\addToDocumentation}
  \end{lsucrequires}

  \lsucprohibitsnothing
\end{lsuc}

% ------------------------------------------------------------------------------
\subsection{GPL-\ver-C9: Passing a modified library as independent binary}
\begin{lsuc}{GPL-\ver-C9}
  \linkosuc{08B}

  \useCaseNine{\ver}
  \coversNine

  \begin{lsucrequires}
    \lsucmandatory{\keepLicenseElements}
    \lsucmandatory{\gpltwoEnsureCopyrightNoticeSource}  
    \lsucmandatory{\giveLicense}\passingFilesCorrectly
    \lsucmandatory{\retainCopyrightNotices}
    \lsucmandatory{\makeModifiedSourceAvailable}
    \lsucmandatory{\describeHowToGetSource}
    \lsucsourcedist{GPL-\ver-C8}
    \lsucmandatory{\markLibraryModifications}
    \lsucmandatory{\arrangeLibraryChanges}\howToApplyTheseTerms
    \lsucoptional{\createChangelog}
    \lsucoptional{\addToDocumentation}
  \end{lsucrequires}

  \lsucprohibitsnothing
\end{lsuc}

% ------------------------------------------------------------------------------
\subsection{GPL-\ver-CA: Passing a modified library as embedded source code}
\begin{lsuc}{GPL-\ver-CA}
  \linkosuc{10S}

  \useCaseA{\ver}
  \coversA

  \begin{lsucrequires}
    \lsucmandatory{\keepLicenseElements}
    \lsucmandatory{\gpltwoEnsureCopyrightNoticeSource}
    \lsucmandatory{\giveLicense}\passingFilesCorrectly
    \lsucmandatory{\retainCopyrightNotices}
    \lsucmandatory{\addToCopyrightDialogLib}
    \lsucmandatory{\markEmbeddedModifications}
    \lsucmandatory{\arrangeEmbeddedChanges}\howToApplyTheseTerms
    \lsucmandatory{\arrangeEnclosingSources}
    \lsucoptional{\createChangelog}
    \lsucoptional{\addToDocumentation}
  \end{lsucrequires}

  \lsucprohibitsnothing
\end{lsuc}

% ------------------------------------------------------------------------------
\subsection{GPL-\ver-CB: Passing a modified library as embedded binary}
\begin{lsuc}{GPL-\ver-CB}
  \linkosuc{10B}

  \useCaseB{\ver}
  \coversB

  \begin{lsucrequires}
    \lsucmandatory{\keepLicenseElements}
    \lsucmandatory{\gpltwoEnsureCopyrightNoticeBinary}
    \lsucmandatory{\giveLicense}\passingFilesCorrectly
    \lsucmandatory{\retainCopyrightNotices}
    \lsucmandatory{\makeAllSourcesAvailable}
    \lsucmandatory{\describeHowToGetSource}
    \lsucsourcedist{GPL-\ver-CA}
    \lsucmandatory{\addToCopyrightDialogLib}
    \lsucmandatory{\markEmbeddedModifications}
    \lsucmandatory{\arrangeEmbeddedChanges}\howToApplyTheseTerms
    \lsucmandatory{\arrangeEnclosingBinaries}
    \lsucoptional{\createChangelog}
    \lsucoptional{\addToDocumentation}
  \end{lsucrequires}

  \lsucprohibitsnothing
\end{lsuc}

% ------------------------------------------------------------------------------
\end{license}

%% =============================================================================
%% GPL-3.0 Use Cases

\renewcommand{\ver}{3.0}

\begin{license}{GPL3} 
\licensename{GPL-3.0}
\licensespec{General Public License Version 3}
\licenseabbrev{GPL}
%\licenseversion{3.0}

% ------------------------------------------------------------------------------
\subsection{GPL-\ver-C1: Using the software only for yourself}
\begin{lsuc}{GPL-\ver-C1}
  \linkosuc{01} 
  \linkosuc{03}
  \linkosuc{06} 
  \linkosuc{09}

  \useCaseOne{\ver}
  \coversOne

  \begin{lsucrequiresnothing}
    \lsucitem{You are allowed to use any kind of GPL software in any sense and in
      any context without being obliged to do anything as long as you do not
      give the software to third parties.}
  \end{lsucrequiresnothing}

  \begin{lsucprohibits}
    \lsucitem{\noPatentLitigation}
  \end{lsucprohibits}
\end{lsuc}

% ------------------------------------------------------------------------------
\subsection{GPL-\ver-C2: Passing the unmodified software as independent sources}
\begin{lsuc}{GPL-\ver-C2}
  \linkosuc{02S}
  \linkosuc{05S}

  \useCaseTwo{\ver}
  \coversTwo

  \begin{lsucrequires}
    \lsucmandatory{\keepLicenseElements}
    \lsucmandatory{\gplthreeEnsureCopyrightNoticeSource}
    \lsucmandatory{\giveLicense}\passingFilesCorrectly
    \lsucmandatory{\retainCopyrightNotices}
    \lsucoptional{\addToDocumentation}
  \end{lsucrequires}

  \begin{lsucprohibits}
    \lsucitem{\noPatentLitigation}
  \end{lsucprohibits}
\end{lsuc}

% ------------------------------------------------------------------------------
\subsection{GPL-\ver-C3: Passing the unmodified software as independent binaries} 
\begin{lsuc}{GPL-\ver-C3}
  \linkosuc{02B} 
  \linkosuc{05B}

  \useCaseThree{\ver}
  \coversThree

  \begin{lsucrequires}
    \lsucmandatory{\keepLicenseElements}
    \lsucmandatory{\gplthreeEnsureCopyrightNoticeBinary}
    \lsucmandatory{\giveLicense}\passingFilesCorrectly  
    \lsucmandatory{\makeUnmodifiedSourceAvailable}
    \lsucmandatory{\describeHowToGetSource}
    \lsucmandatory{\retainCopyrightNotices}
    \lsucsourcedist{GPL-\ver-C2}
    \lsucoptional{\addToDocumentation}
  \end{lsucrequires}

  \begin{lsucprohibits}
    \lsucitem{\noPatentLitigation}
  \end{lsucprohibits}
\end{lsuc}

% ------------------------------------------------------------------------------
\subsection{GPL-\ver-C4: Passing the unmodified library as embedded sources}
\begin{lsuc}{GPL-\ver-C4}
  \linkosuc{07S} 

  \useCaseFour{\ver}
  \coversFour

  \begin{lsucrequires}
    \lsucmandatory{\keepLicenseElements}
    \lsucmandatory{\gplthreeEnsureCopyrightNoticeSource}
    \lsucmandatory{\giveLicense}\passingFilesCorrectly
    \lsucmandatory{\retainCopyrightNotices}
    \lsucmandatory{\addToCopyrightDialogLib}
    \lsucmandatory{\arrangeEnclosingSources}
    \lsucoptional{\addToDocumentation}
  \end{lsucrequires}

  \begin{lsucprohibits}
    \lsucitem{\noPatentLitigation}
  \end{lsucprohibits}
\end{lsuc}

% ------------------------------------------------------------------------------
\subsection{GPL-\ver-C5: Passing the unmodified library as embedded binaries} 
\begin{lsuc}{GPL-\ver-C5}
  \linkosuc{07B} 

  \useCaseFive{\ver}
  \coversFive

  \begin{lsucrequires}
    \lsucmandatory{\keepLicenseElements}
    \lsucmandatory{\gplthreeEnsureCopyrightNoticeBinary}
    \lsucmandatory{\giveLicense}\passingFilesCorrectly
    \lsucmandatory{\makeAllSourcesAvailable}
    \lsucmandatory{\describeHowToGetSource}
    \lsucmandatory{\addToCopyrightDialogLib}
    \lsucmandatory{\arrangeEnclosingBinaries}
    \lsucmandatory{\retainCopyrightNotices}
    \lsucsourcedist{GPL-\ver-C4}
    \lsucoptional{\addToDocumentation}
  \end{lsucrequires}

  \begin{lsucprohibits}
    \lsucitem{\noPatentLitigation}
  \end{lsucprohibits}
\end{lsuc}

% ------------------------------------------------------------------------------
\subsection{GPL-\ver-C6: Passing a modified program as source code}
\begin{lsuc}{GPL-\ver-C6}
  \linkosuc{04S} 

  \useCaseSix{\ver}
  \coversSix

  \begin{lsucrequires}
    \lsucmandatory{\keepLicenseElements}
    \lsucmandatory{\gplthreeEnsureCopyrightNoticeSource}
    \lsucmandatory{\giveLicense}\passingFilesCorrectly
    \lsucmandatory{\retainCopyrightNotices}
    \lsucmandatory{\addToCopyrightDialogApp}
    \lsucmandatory{\markProgramModifications}
    \lsucmandatory{\arrangeProgramChanges}\howToApplyTheseTerms
    \lsucoptional{\createChangelog}
    \lsucoptional{\addToDocumentation}
  \end{lsucrequires}

  \begin{lsucprohibits}
    \lsucitem{\noPatentLitigation}
  \end{lsucprohibits}
\end{lsuc}

% ------------------------------------------------------------------------------
\subsection{GPL-\ver-C7: Passing a modified program as binary}
\begin{lsuc}{GPL-\ver-C7}
  \linkosuc{04B}

  \useCaseSeven{\ver}
  \coversSeven

  \begin{lsucrequires}
    \lsucmandatory{\keepLicenseElements}
    \lsucmandatory{\gplthreeEnsureCopyrightNoticeBinary}
    \lsucmandatory{\giveLicense}\passingFilesCorrectly
    \lsucmandatory{\retainCopyrightNotices}
    \lsucmandatory{\markProgramModifications}
    \lsucmandatory{\addToCopyrightDialogApp}
    \lsucmandatory{\arrangeProgramChanges}\howToApplyTheseTerms
    \lsucmandatory{\makeModifiedSourceAvailable}
    \lsucmandatory{\describeHowToGetSource}
    \lsucsourcedist{GPL-\ver-C6}
    \lsucoptional{\createChangelog}
    \lsucoptional{\addToDocumentation}
  \end{lsucrequires}

  \begin{lsucprohibits}
    \lsucitem{\noPatentLitigation}
  \end{lsucprohibits}
\end{lsuc}

% ------------------------------------------------------------------------------
\subsection{GPL-\ver-C8: Passing a modified library as independent source code}
\begin{lsuc}{GPL-\ver-C8}
  \linkosuc{08S}

  \useCaseEight{\ver}
  \coversEight

  \begin{lsucrequires}
     \lsucmandatory{\keepLicenseElements}
    \lsucmandatory{\gplthreeEnsureCopyrightNoticeSource}
    \lsucmandatory{\giveLicense}\passingFilesCorrectly
    \lsucmandatory{\retainCopyrightNotices}
    \lsucmandatory{\markLibraryModifications}
    \lsucmandatory{\arrangeLibraryChanges}\howToApplyTheseTerms
    \lsucoptional{\createChangelog}
    \lsucoptional{\addToDocumentation}
  \end{lsucrequires}

  \begin{lsucprohibits}
    \lsucitem{\noPatentLitigation}
  \end{lsucprohibits}
\end{lsuc}

% ------------------------------------------------------------------------------
\subsection{GPL-\ver-C9: Passing a modified library as independent binary}
\begin{lsuc}{GPL-\ver-C9}
  \linkosuc{08B}

  \useCaseNine{\ver}
  \coversNine

  \begin{lsucrequires}
    \lsucmandatory{\keepLicenseElements}
    \lsucmandatory{\gplthreeEnsureCopyrightNoticeSource}  
    \lsucmandatory{\giveLicense}\passingFilesCorrectly
    \lsucmandatory{\retainCopyrightNotices}
    \lsucmandatory{\makeModifiedSourceAvailable}
    \lsucmandatory{\describeHowToGetSource}
    \lsucsourcedist{GPL-\ver-C8}
    \lsucmandatory{\markLibraryModifications}
    \lsucmandatory{\arrangeLibraryChanges}\howToApplyTheseTerms
    \lsucoptional{\createChangelog}
    \lsucoptional{\addToDocumentation}
  \end{lsucrequires}

  \begin{lsucprohibits}
    \lsucitem{\noPatentLitigation}
  \end{lsucprohibits}
\end{lsuc}

% ------------------------------------------------------------------------------
\subsection{GPL-\ver-CA: Passing a modified library as embedded source code}
\begin{lsuc}{GPL-\ver-CA}
  \linkosuc{10S}

  \useCaseA{\ver}
  \coversA

  \begin{lsucrequires}
    \lsucmandatory{\keepLicenseElements}
    \lsucmandatory{\gplthreeEnsureCopyrightNoticeSource}
    \lsucmandatory{\giveLicense}\passingFilesCorrectly
    \lsucmandatory{\retainCopyrightNotices}
    \lsucmandatory{\addToCopyrightDialogLib}
    \lsucmandatory{\markEmbeddedModifications}
    \lsucmandatory{\arrangeEmbeddedChanges}\howToApplyTheseTerms
    \lsucmandatory{\arrangeEnclosingSources}
    \lsucoptional{\createChangelog}
    \lsucoptional{\addToDocumentation}
  \end{lsucrequires}

  \begin{lsucprohibits}
    \lsucitem{\noPatentLitigation}
  \end{lsucprohibits}
\end{lsuc}

% ------------------------------------------------------------------------------
\subsection{GPL-\ver-CB: Passing a modified library as embedded binary}
\begin{lsuc}{GPL-\ver-CB}
  \linkosuc{10B}

  \useCaseB{\ver}
  \coversB

  \begin{lsucrequires}
    \lsucmandatory{\keepLicenseElements}
    \lsucmandatory{\gplthreeEnsureCopyrightNoticeBinary}
    \lsucmandatory{\giveLicense}\passingFilesCorrectly
    \lsucmandatory{\retainCopyrightNotices}
    \lsucmandatory{\makeAllSourcesAvailable}
    \lsucmandatory{\describeHowToGetSource}
    \lsucsourcedist{GPL-\ver-CA}
    \lsucmandatory{\addToCopyrightDialogLib}
    \lsucmandatory{\markEmbeddedModifications}
    \lsucmandatory{\arrangeEmbeddedChanges}\howToApplyTheseTerms
    \lsucmandatory{\arrangeEnclosingBinaries}
    \lsucoptional{\createChangelog}
    \lsucoptional{\addToDocumentation}
  \end{lsucrequires}

  \begin{lsucprohibits}
    \lsucitem{\noPatentLitigation}
  \end{lsucprohibits}
\end{lsuc}

% ------------------------------------------------------------------------------
\end{license}

%% =============================================================================
%% Discussion

\subsection{Discussions and Explanations}

\newcommand{\gplTwoAndThree}[2]{\footnote{%
    For GPL-2.0 see \cite[cf.][\nopage wp.\ #1]{Gpl20OsiLicense1991a}.\par\noindent 
    For GPL-3.0 see \cite[cf.][\nopage wp.\ #2]{Gpl30OsiLicense2007a}.}} 

The GPL-2.0 allows to \enquote{[\ldots] copy and (to) distribute verbatim copies
of the Program's complete source code as you receive it [...] provided that
you [a] conspicuously and appropriately publish on each copy an appropriate
copyright notice and disclaimer of warranty; [b] keep intact all the notices
that refer to this License and to the absence of any warranty; and [c]
distribute a copy of this License along with the Program.}\citeGPLtwo{§1} The
GPL-2.0 also allows to \enquote{[\ldots] copy and distribute [\ldots]
modifications (of the Program or any portion of it) [\ldots] under the terms
of Section~1}\citeGPLtwo{§2} while it allows to distribute binaries
\enquote{under the terms of Sections 1 and~2}.\citeGPLtwo{§4} But the GPL-2.0
does not require any tasks if you are using the work only for yourself. Thus,
the quoted conditions of \enquote{Section~1} are mandatory for all use cases
concerning the distribution of an GPL-2.0 licensed work (GPL-2.0-C2 --
GPL-2.0-CB)
  
\label{Gpl3ConditionsDistri}
The GPL-3.0 uses a similar structure to establish the same requirements: In §4
it allows to \enquote{[\ldots] convey verbatim copies of the Program's source
code as you receive it [\ldots] provided that you conspicuously and
appropriately publish on each copy an appropriate copyright notice; keep
intact all notices stating that this License and any non-permissive terms
added in accord with section 7 apply to the code; keep intact all notices of
the absence of any warranty; and give all recipients a copy of this License
along with the Program}. §5 also allows to \enquote{[\ldots] convey [\ldots]
modifications [\ldots] under the terms of section 4 [\ldots]} and §6 gives
permission to \enquote{[\ldots] convey a covered work in object form under the
terms of sections of 4 and 5}.\citeGPLthree{§4, §5, §6} In contrast to the
GPL-2.0, the GPL-3.0 explicitly states that one \enquote{[\ldots] may make, run
and propagate covered works that (one) (does) not convey [distribute], without
conditions so long as (the GPL-3.0) license otherwise remains in
force.}\citeGPLthree{§2}
% TODO: rephrase, including hosting the software, and name the conditions:
% exclusively for 'you' and under 'your' direction and control, and no copies
% of your own copyrighted material
Moreover, giving a package to a third party for getting a modified version back
has not to be taken as a case of distribution if the modification has only been
executed on behalf and only for the purpose of the purchaser and if the modified
version is not distributed to any third party.\citeGPLthree{§2} If one collects
all these GPL-3.0 statements together, than one may conclude that the tasks
which fulfill the corresponding GPL-2.0 requirements together also fit the
GPL-3.0 conditions.
  
The GPL-2.0 allows to \enquote{[\ldots] copy and (to) distribute the Program (or
a work based on it [\ldots]) in object code or executable form [\ldots] provided
that you accompany it with the complete corresponding machine-readable source
code [\ldots] on a medium customarily used for software
interchange}.\citeGPLtwo{§3, §3a} As a substitution for this basic condition,
the GPL-2.0 allows to \enquote{accompany} the binary distribution package
\enquote{[\ldots] with a written offer, valid for at least three years, to give
any third party, for a charge no more than your cost of physically performing
source distribution, a complete machine-readable copy of the corresponding
source code [\ldots] on a medium customarily used for software
interchange}.\citeGPLtwo{§3b} The \oslic{} construes the common technique to
download files from the internet as a distribution \emph{on a medium [being
today] customarily used for software interchange}. Therefore, the \oslic{} requires
for all open source use cases that refer to the distribution of binaries
(GPL-2.0-C3, GPL-2.0-C7, GPL-2.0-C9, GPL-2.0-CB) to make the corresponding
source code of the library itself accessible via an internet repository under
your own control. 
  
\label{Gpl3CondCopyleft}
The GPL-3.0 also explicitly requires to make the source code accessible in case
of distributing binaries. But opposite to the GPL-2.0, the GPL-3.0 explicitly
offers the option of giving \enquote{[\ldots] access to copy the Corresponding
Source from a network server at no charge} as a means to fulfill the
conditions.\citeGPLthree{§6 and §6b} So again, the tasks which ensure to act in
accordance to the GPL-2.0 license in case of distributing binaries, also fulfill
the conditions of the GPL-3.0.

The weakness that in this case \enquote{third parties [which have received the
binaries] are not compelled to copy the source code [\ldots]} is a concession
made by the GPL-2.0.\citeGPLtwo{§3, at the end} But the necessity to offer the
source code via a repository controlled by yourself may generally not be
circumvented: The GPL-2.0 allows to redistribute a link to an external source
code repository only in case of \enquote{noncommercial
distributions}.\citeGPLtwo{§3c} 
  
Both, the GPL-2.0 and the GPL-3.0 allow you to \enquote{[\ldots] modify your
copy or copies of the Program or any portion of it [\ldots] and (to) copy and
distribute such modifications [\ldots]} only under very similar restrictions and
conditions:\citeGPLtwo{§2} 
\begin{itemize}
\item First, modified files must be marked as modifications and the date of the
  modification.\gplTwoAndThree{§2}{§5} These conditions must be respected by all
  open source use cases concerning the distribution of the modified work
  [GPL-2.0-C6/GPL-3.0C6 -- GPL-2.0-C9/GPL-3.0-C9], because even if one primarily
  intends to distribute binaries, one has also to deliver the source code. The
  \oslic{} captures this requirement in the mandatory condition to mark each
  modified file and the voluntary condition to update / generate a general
  changelog.
    
\item Second, both versions of the GPL require that all copies of the modified
  software which are using an interactive interface or a method to display
  messages must \enquote{[\ldots] print or display an announcement including an
  appropriate copyright notice and a notice that there is no warranty [\ldots]
  and that users may redistribute the program under these conditions, and
  telling the user how to view a copy of this License.}\gplTwoAndThree{§2c}{§5d}
  The \oslic{} rewrites this condition in the form that the work shall let its
  copyright dialog clearly reproduce the content of the existing copyright
  notices, the software name, a link to its homepage, the respective disclaimer
  of warranty, and a link to the GPL-2.0-file (or GPL-3.0-file, resp.), which
  has to be delivered together with the software.
  % TODO: actually, the task does not refer to the _file_
  These conditions have to be respected if one redistributes the received and
  then modified programs (GPL-2.0-C6, GPL-2.0-C7, GPL-3.0-C6, GPL-3.0-C7) or if
  one distributes one's own programs which are using (modified) libraries as
  embedded components (GPL-2.0-CA, GPL-2.0-CB, GPL-3.0-CA, GPL-3.0-CB). For
  those open source use cases that concern the redistribution of received and
  modified libraries, etc., the \oslic{} does not mention these requirements
  because libraries, plugins, or snippets normally do not have their own
  copyright dialogs.  
    
\item Third, the GPL requires to \enquote{ [\ldots] cause any work (being
  distributed or published), that in whole or in part contains or is derived
  from the Program or any part thereof, to be licensed as a whole at no charge
  to all third parties under the terms of this (GPL.)}\gplTwoAndThree{§2b}{§5c}
  This requirement does not depend of the form in which the software is
  distributed. The \oslic{} adopts this statement in the following way:
  \begin{itemize}
  \item For all open source use cases which concern the distribution (GPL-2.0-C2
    \ldots GPL-2.0-CB, GPL-3.0-C2 \ldots GPL-3.0-CB), the \oslic{} rewrites this
    condition as the mandatory requirement to retain all existing licensing
    elements.
      
  \item For all use cases which deal with the distribution of a modified version
    of the software (GPL-2.0-C6 \ldots GPL-2.0-CB, GPL-3.0-C6 \ldots
    GPL-3.0-CB), the OSliC adds the requirement to organize the modifications in
    a way that they are covered by the respective GPL-2.0 or GPL-3.0 licensing
    statements.
      
  \item For the use case which deal with the distribution of an embedded library
    (GPL-2.0-C4, GPL-2.0-C5, GPL-2.0-CA, GPL-2.0-CB, GPL-3.0-C4, GPL-3.0-C5,
    GPL-3.0-CA, GPL-3.0-CB) the \oslic{} requires also to license the on-top
    development under the terms of the respective GPL-2.0 or GPL-3.0 license.
    \end{itemize}
   
\item Finally, as parts of those task lists which concern the distribution in
  the form of binaries, the \oslic{} reminds the reader also to execute the
  corresponding source code use cases because distributing the binaries without
  making the corresponding sources accessible is not allowed by the GPL.
\end{itemize}
  
%\bibliography{../../../bibfiles/oscResourcesEn}

% Local Variables:
% mode: latex
% fill-column: 80
% End:
