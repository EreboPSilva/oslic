% Telekom osCompendium 'for being included' snippet template
%
% (c) Karsten Reincke, Deutsche Telekom AG, Darmstadt 2011
%
% This LaTeX-File is licensed under the Creative Commons Attribution-ShareAlike
% 3.0 Germany License (http://creativecommons.org/licenses/by-sa/3.0/de/): Feel
% free 'to share (to copy, distribute and transmit)' or 'to remix (to adapt)'
% it, if you '... distribute the resulting work under the same or similar
% license to this one' and if you respect how 'you must attribute the work in
% the manner specified by the author ...':
%
% In an internet based reuse please link the reused parts to www.telekom.com and
% mention the original authors and Deutsche Telekom AG in a suitable manner. In
% a paper-like reuse please insert a short hint to www.telekom.com and to the
% original authors and Deutsche Telekom AG into your preface. For normal
% quotations please use the scientific standard to cite.
%
% [ Framework derived from 'mind your Scholar Research Framework' 
%   mycsrf (c) K. Reincke 2012 CC BY 3.0  http://mycsrf.fodina.de/ ]
%


%% use all entries of the bibliography
%\nocite{*}

\section{GPL licensed software}

Both versions of the GNU General Public License explicitly distinguish the
distribution of the source code from that of the binaries: On the one hand, the
GPL-V2 mainly talks about copying and distributing the source
code\footcite[cf.][\nopage wp.\ §1, §2]{Gpl20OsiLicense1991a}, but also mentions
the specific conditions for \enquote{[\ldots] (copying) and (distributing) the
Program [\ldots] in object code or executable form
[\ldots]}\footcite[cf.][\nopage wp.\ §3]{Gpl20OsiLicense1991a}. On the other
hand, the GPL-V3 describes the \enquote{Basic Permissions} and the conditions
for \enquote{Conveying Verbatim Copies} or for \enquote{Conveying Modified
Source Versions}\footcite[cf.][\nopage wp.\ §2, §4, §5]{Gpl30OsiLicense2007a}
before it explains the rules for \enquote{Conveying
Non-Source-Forms}\footcite[cf.][\nopage wp.\ §2, §4, §5]{Gpl30OsiLicense2007a}.
Additionally, GPL-V2 and GPL-V3 mainly talk about copying \emph{and}
distributing the software; the private use is nearly complete
unspecified\footnote{The GPL-V2 lists its 'restrictions' only with respect to
the act of copying \emph{and} distributing \enquote{copies of the program}
(\cite[cf.][\nopage wp.\ §1, §2, §4 et passim; emphasizings by
KR]{Gpl20OsiLicense1991a}) while the GPL-V3 explicitly specifies that one
\enquote{[\ldots] may make, run and propagate covered works that (one does) not
convey, without conditions so long as (the) license otherwise remains in force}
(\cite[cf.][\nopage wp.\ §2]{Gpl30OsiLicense2007a}).}. Finally, the GPL-V2 and
the GPL-V3 are aiming for the same results and the same spirit by requiring
nearly the same license fulfilling tasks.  But with respect to the use of an
unmodified library as embedded component and in opposite to the power of the
other licenses, a license with strong copy left evokes also that the application
which is using the unmodified library has also to be licensed under the same
conditions. Therefore it is appropriate to cover both versions by the same
chapter and to offer the same more sophisticated GPL specific open source use
case structure\footnote{For details of the general OSUC finder $\rightarrow$
OSLiC, pp.\ \pageref{OsucTokens} and \pageref{OsucDefinitionTree}} for finding
the easily processable corresponding task lists:
 
\tikzstyle{nodv} = [font=\scriptsize, ellipse, draw, fill=gray!10, 
    text width=2cm, text centered, minimum height=2em]


\tikzstyle{nods} = [font=\tiny, rectangle, draw, fill=gray!20, 
    text width=1cm, text centered, rounded corners, minimum height=3em]

\tikzstyle{nodb} = [font=\tiny, rectangle, draw, fill=gray!20, 
    text width=1.5cm, text centered, rounded corners, minimum height=3em]
    
\tikzstyle{leaf} = [font=\tiny, rectangle, draw, fill=gray!30, 
    text width=1.2cm, text centered, minimum height=6em]

\tikzstyle{slimleaf} = [font=\tiny, rectangle, draw, fill=gray!30, 
    text width=1cm, text centered, minimum height=6em]


\tikzstyle{edge} = [draw, -latex']

\begin{tikzpicture}[]

\node[nodv] (l801) at (4,11.8) {GPL};

\node[nodv] (l701) at (0,10.2) {version 2.0};
\node[nodv] (l702) at (7.5,10.2) {version 3.0};


\node[nodb] (l601) at (0,8.6) {\textit{recipient:} \\ \textbf{4yourself}};
\node[nodb] (l602) at (7.5,8.6) {\textit{recipient:} \\ \textbf{2others}};

\node[nodb] (l501) at (4,7) {\textit{state:} \\ \textbf{unmodified}};
\node[nodb] (l502) at (11,7) {\textit{state:} \\ \textbf{modified}};

\node[nodb] (l401) at (2.25,5.4) {\textit{type:} \\ \textbf{proapse or snimoli}};
\node[nodb] (l402) at (5.4,5.4) {\textit{type:} \\ \textbf{snimoli}};
\node[nodb] (l403) at (8.4,5.4) {\textit{type:} \\ \textbf{proapse}};
\node[nodb] (l404) at (12.8,5.4) {\textit{type:} \\ \textbf{snimoli}};


\node[nodb] (l301) at (2.25,3.8) {\textit{context:} \\ \textbf{independent}};
\node[nodb] (l302) at (5.4,3.8) {\textit{context:} \\ \textbf{embedded}};
\node[nodb] (l303) at (8.4,3.8) {\textit{context:} \\ \textbf{independent}};
\node[nodb] (l304) at (11.3,3.8) {\textit{context:} \\ \textbf{independent}};
\node[nodb] (l305) at (14.3,3.8) {\textit{context:} \\ \textbf{embedded}};

\node[nods] (l201) at (1.45,2.2) {\textit{form:} \textbf{source}};
\node[nods] (l202) at (3.0,2.2) {\textit{form:} \textbf{binary}};
\node[nods] (l203) at (4.6,2.2) {\textit{form:} \textbf{source}};
\node[nods] (l204) at (6.2,2.2) {\textit{form:} \textbf{binary}};
\node[nods] (l205) at (7.7,2.2) {\textit{form:} \textbf{source}};
\node[nods] (l206) at (9.1,2.2) {\textit{form:} \textbf{binary}};
\node[nods] (l207) at (10.5,2.2) {\textit{form:} \textbf{source}};
\node[nods] (l208) at (11.9,2.2) {\textit{form:} \textbf{binary}};
\node[nods] (l209) at (13.4,2.2) {\textit{form:} \textbf{source}};
\node[nods] (l210) at (15.0,2.2) {\textit{form:} \textbf{binary}};

\node[slimleaf] (l101) at (0,0) {\textbf{GPL-01} \textit{using software only
for yourself}};

\node[leaf] (l102) at (1.45,0) { \textbf{GPL-02} \textit{ distributing unmodified
software as independent sources}};

\node[leaf] (l103) at (3.0,0) { \textbf{GPL-03}  \textit{ distributing unmodified
software as independent binaries}};

\node[leaf] (l104) at (4.6,0) { \textbf{GPL-04} \textit{ distributing unmodified
library as embedded sources}};

\node[leaf] (l105) at (6.2,0) { \textbf{GPL-05}  \textit{ distributing unmodified
library as embedded binaries}};

\node[slimleaf] (l106) at (7.7,0) { \textbf{GPL-06}  \textit{ distributing modified
program as sources}};

\node[slimleaf] (l107) at (9.1,0) { \textbf{GPL-07}  \textit{ distributing modified
program as binaries}};

\node[slimleaf] (l108) at (10.5,0) { \textbf{GPL-08}  \textit{ distributing modified
library as independent sources}};

\node[slimleaf] (l109) at (11.9,0) { \textbf{GPL-09} \textit{distributing modified
library as independent binaries}};

\node[leaf] (l110) at (13.4,0) { \textbf{GPL-10}  \textit{distributing
modified library as embedded sources}};

\node[leaf] (l111) at (15,0) { \textbf{GPL-11}  \textit{ distributing modified
library as embedded binaries}};

\path [edge] (l801) -- (l701);
\path [edge] (l801) -- (l702);
\path [edge] (l701) -- (l601);
\path [edge] (l701) -- (l602);
\path [edge] (l702) -- (l601);
\path [edge] (l702) -- (l602);

\path [edge] (l602) -- (l501);
\path [edge] (l602) -- (l502);

\path [edge] (l501) -- (l401);
\path [edge] (l501) -- (l402);
\path [edge] (l502) -- (l403);
\path [edge] (l502) -- (l404);

\path [edge] (l401) -- (l301);
\path [edge] (l402) -- (l302);
\path [edge] (l403) -- (l303);
\path [edge] (l404) -- (l304);
\path [edge] (l404) -- (l305);

\path [edge] (l301) -- (l201);
\path [edge] (l301) -- (l202);
\path [edge] (l302) -- (l203);
\path [edge] (l302) -- (l204);
\path [edge] (l303) -- (l205);
\path [edge] (l303) -- (l206);
\path [edge] (l304) -- (l207);
\path [edge] (l304) -- (l208);
\path [edge] (l305) -- (l209);
\path [edge] (l305) -- (l210);

\path [edge] (l601) -- (l101);
\path [edge] (l201) -- (l102);
\path [edge] (l202) -- (l103);
\path [edge] (l203) -- (l104);
\path [edge] (l204) -- (l105);
\path [edge] (l205) -- (l106);
\path [edge] (l206) -- (l107);
\path [edge] (l207) -- (l108);
\path [edge] (l208) -- (l109);
\path [edge] (l209) -- (l110);
\path [edge] (l210) -- (l111);

\end{tikzpicture}


\subsection{GPL-01: Using the software only for yourself}
\label{OSUC-01-GPL} \label{OSUC-03-GPL}
\label{OSUC-06-GPL} \label{OSUC-09-GPL}

\begin{description}

\item[means] that you are going to use a received GPL licensed software only
for yourself and that you do not hand it over to any 3rd party in any sense.

\item[covers] OSUC-01, OSUC-03, OSUC-06, and OSUC-09\footnote{For details
$\rightarrow$ OSLiC, pp.\ \pageref{OSUC-01-DEF} - \pageref{OSUC-09-DEF}}

\item[requires] no tasks in order to fulfill the conditions of the GPL-V2 or
the GPL-V3 with respect to this use case:
  \begin{itemize}
    \item You are allowed to use any kind of GPL software in any sense and in
    any context without being obliged to do anything as long as you do not
    give the software to 3rd parties.
  \end{itemize}

\item[prohibits] nothing explictly with respect to this use case.
\end{description}


\subsection{GPL-02: Passing the unmodified software as independent sources}
\label{OSUC-02S-GPL} \label{OSUC-05S-GPL}

\begin{description}

\item[means] that you are going to distribute an unmodified version of the
received GPL software to 3rd parties -- as an independent unit and in the form
of source code files or as a source code package. In this case, it doesn't
matter whether you distribute a program, an application, a server, a snippet, a
module, a library, or a plugin.

\item[covers] OSUC-02S, OSUC-05S\footnote{For details $\rightarrow$
OSLiC, pp.\ \pageref{OSUC-02S-DEF} - \pageref{OSUC-05S-DEF}}

\item[requires] the following tasks in order to fulfill the license conditions:
\begin{itemize}
 
  \item \textbf{[mandatory:]} Ensure that the licensing elements -- esp.\ all
  notices that refer to the GPL-V2 or GPL-V3 and to the absence of any
  warranty -- are retained in your package in the form you have received them.

  \item \textbf{[mandatory:]} Ensure that the distributed source code package
  contains a conspicuously and appropriately designed, easily to find copyright
  notice and a disclaimer of warranty. If these elements are missed, add a new
  file containing the main copyright notice and the disclaimer of warranty in the
  form which is textually defined by the license GPL-V2 itself resp. by the
  GPL-V3 itself. (Yes, repeat the disclaimer although it is also part of the
  license itself and although you are required to hand the license itself over
  to the receiver.)
  
  \item \textbf{[mandatory:]} Give the recipient a copy of the GPL-V2 resp.
  GPL-V3 license. If it is not already part of the software package, add
  it\footnote{For implementing the handover of files correctly $\rightarrow$
  OSLiC, p. \pageref{DistributingFilesHint}}.

  \item \textbf{[mandatory:]} Retain all existing copyright notices and
  licensing elements.
    
  \item \textbf{[voluntary:]} Let the documentation of your distribution and/or
  your additional material also reproduce the content of the existing
  copyright notices, a hint to the software name, a link to its homepage,
  the respective disclaimer of warranty, and a link to the GPL-V2 resp.
  GPL-V3.

\end{itemize}

\item[prohibits] nothing explictly with respect to this use case.

\end{description}


\subsection{GPL-03: Passing the unmodified software as independent binaries} 
\label{OSUC-02B-GPL} \label{OSUC-05B-GPL}

\begin{description}


\item[means] that you are going to distribute an unmodified version of the
received GPL software to 3rd parties -- as an independent unit and in the form
of binary files or as a binary package. In this case, it doesn't matter whether
you distribute a program, an application, a server, a snippet, a module, a
library, or a plugin.


\item[covers] OSUC-02B, OSUC-05B\footnote{For details $\rightarrow$ OSLiC, pp.\
\pageref{OSUC-02B-DEF} - \pageref{OSUC-05B-DEF}}

\item[requires] the following tasks in order to fulfill the license conditions:
\begin{itemize}
  
  \item \textbf{[mandatory:]} Ensure that the licensing elements -- esp.\ all
  notices that refer to the GPL-V2 or GPL-V3 and to the absence of any
  warranty -- are retained in your package in the form you have received them.

  \item \textbf{[mandatory:]} Ensure that the distributed software binary
  package contains a conspicuously and appropriately designed, easily to find
  copyright notice and a disclaimer of warranty. If these elements are missed,
  add a new file containing the main copyright notice and the disclaimer of
  warranty in the form which is textually defined by the license GPL-V2 itself
  resp. by the GPL-V3 itself. (Yes, repeat the disclaimer although it is also
  part of the license itself and although you are required to hand the license
  itself over to the receiver.)
  
  \item \textbf{[mandatory:]} Give the recipient a copy of the GPL-V2 resp.
  GPL-V3 license. If it is not already part of the software package, add
  it\footnote{For implementing the handover of files correctly $\rightarrow$
  OSLiC, p. \pageref{DistributingFilesHint}}.
  
  \item \textbf{[mandatory:]} Make the source code of the distributed software
  accessible via a repository under your own control (even if you do not
  modified it): Push the source code package into a repository, make it
  downloadable via the internet, and integrate an easily to find description
  into the distribution package which explains how the code can be received from
  where. Ensure, that this repository is online for at least 3 years after
  having distributed the last instance of your software package.
  
  \item \textbf{[mandatory:]} Insert a prominent hint to the download repository
  into your distribution and/or your additional material.

  \item \textbf{[mandatory:]} Retain all existing copyright notices and
  licensing elements.
    
  \item \textbf{[mandatory:]} Execute the to-do list of use case GPL-02\footnote{
  Making the code accessible via a repository means distributing the software in
  the form of source code. Hence, you must also fulfill all tasks of the
  corresponding use case.}.

  \item \textbf{[voluntary:]} Let the documentation of your distribution and/or
  your additional material also reproduce the content of the existing
  copyright notices, a hint to the software name, a link to its homepage,
  the respective disclaimer of warranty, and a link to the GPL-V2 resp.
  GPL-V3.

\end{itemize}

\item[prohibits] nothing explictly with respect to this use case.

\end{description}

\subsection{GPL-04: Passing the unmodified library as embedded sources}
\label{OSUC-07S-GPL} 

\begin{description}

\item[means] that you are going to distribute an unmodified version of the
received GPL licensed snippet, module or library to 3rd parties -- as embedded
component of a larger unit and in the form of source code files or as a source
code package.

\item[covers] OSUC-07S\footnote{For details $\rightarrow$
OSLiC, pp.\ \pageref{OSUC-07S-DEF}}

\item[requires] the following tasks in order to fulfill the license conditions:
\begin{itemize}
 
  \item \textbf{[mandatory:]} Ensure that the licensing elements -- esp.\ all
  notices that refer to the GPL-V2 or GPL-V3 and to the absence of any
  warranty -- are retained in your package in the form you have received them.

  \item \textbf{[mandatory:]} Ensure that the distributed source code package
  contains a conspicuously and appropriately designed, easily to find copyright
  notice and a disclaimer of warranty. If these elements are missed, add a new
  file containing the main copyright notice and the disclaimer of warranty in the
  form which is textually defined by the license GPL-V2 itself resp. by the
  GPL-V3 itself. (Yes, repeat the disclaimer although it is also part of the
  license itself and although you are required to hand the license itself over
  to the receiver.)
  
  \item \textbf{[mandatory:]} Give the recipient a copy of the GPL-V2 resp.
  GPL-V3 license. If it is not already part of the software package, add
  it\footnote{For implementing the handover of files correctly $\rightarrow$
  OSLiC, p. \pageref{DistributingFilesHint}}.

  \item \textbf{[mandatory:]} Retain all existing copyright notices and
  licensing elements.
    
  \item \textbf{[mandatory:]} Let the copyright dialog of the on-top development
  clearly say, that it uses the GPL licensed library and that it is itself
  licensed under the GPL-V2 resp. GPL-V3 too. Let it reproduce the content of
  the existing copyright notices, a hint to the software name, a link to its
  homepage, the respective disclaimer of warranty, and a link to the GPL-V2
  resp. GPL-V3.    
    
 \item \textbf{[mandatory:]} Organize the sources of the on-top development in
  a way that they are also covered by the GPL-V2 resp. GPL-V3 licensing
  statements. 

  \item \textbf{[voluntary:]} Let the documentation of your distribution and/or
  your additional material also reproduce the content of the existing
  copyright notices, a hint to the software name, a link to its homepage,
  the respective disclaimer of warranty, and a link to the GPL-V2 resp.
  GPL-V3.

\end{itemize}

\item[prohibits] nothing explictly with respect to this use case.

\end{description}


\subsection{GPL-05: Passing the unmodified library as embedded binaries} 
\label{OSUC-07B-GPL} 

\begin{description}
\item[means] that you are going to distribute an unmodified version of the
received GPL licensed snippet, module or library to 3rd parties -- as embedded
component of a larger unit and in the form of binary files or as a bi\-na\-ry
package.


\item[covers] OSUC-07B\footnote{For details $\rightarrow$ OSLiC, pp.\
\pageref{OSUC-07B-DEF}}

\item[requires] the following tasks in order to fulfill the license conditions:
\begin{itemize}
  
  \item \textbf{[mandatory:]} Ensure that the licensing elements -- esp.\ all
  notices that refer to the GPL-V2 or GPL-V3 and to the absence of any
  warranty -- are retained in your package in the form you have received them.

  \item \textbf{[mandatory:]} Ensure that the distributed software binary
  package contains a conspicuously and appropriately designed, easily to find
  copyright notice and a disclaimer of warranty. If these elements are missed,
  add a new file containing the main copyright notice and the disclaimer of
  warranty in the form which is textually defined by the license GPL-V2 itself
  resp. by the GPL-V3 itself. (Yes, repeat the disclaimer although it is also
  part of the license itself and although you are required to hand the license
  itself over to the receiver.)
  
  \item \textbf{[mandatory:]} Give the recipient a copy of the GPL-V2 resp.
  GPL-V3 license. If it is not already part of the software package, add
  it\footnote{For implementing the handover of files correctly $\rightarrow$
  OSLiC, p. \pageref{DistributingFilesHint}}.
  
  \item \textbf{[mandatory:]} Make the source code of the embedded library
  \textbf{\emph{and}} the source code of your overarching program accessible via
  a repository under your own control: Push the source code package into a
  repository and make it downloadable via the internet. Integrate an easily to
  find description into the distribution package which explains how the code can
  be received from where. Ensure, that this repository is online for for at
  least 3 years after having distributed the last instance of your software
  package.
  
  \item \textbf{[mandatory:]} Insert a prominent hint to the download repository
  into your distribution and/or your additional material.


  \item \textbf{[mandatory:]} Let the copyright dialog of the on-top development
  clearly say, that it uses the GPL licensed library and that it is itself
  licensed under the GPL-V2 resp. GPL-V3 too. Let it reproduce the content of
  the existing copyright notices, a hint to the software name, a link to its
  homepage, the respective disclaimer of warranty, and a link to the GPL-V2
  resp. GPL-V3.
  
  \item \textbf{[mandatory:]} Organize the binaries of the on-top development in
  a way that they are also covered by the GPL-V2 resp. GPL-V3 licensing
  statements.
  
  \item \textbf{[mandatory:]} Retain all existing copyright notices and
  licensing elements.
    
  \item \textbf{[mandatory:]} Execute the to-do list of use case GPL-04\footnote{
  Making the code accessible via a repository means distributing the software in
  the form of source code. Hence, you must also fulfill all tasks of the
  corresponding use case.}.

  \item \textbf{[voluntary:]} Let the documentation of your distribution and/or
  your additional material also reproduce the content of the existing
  copyright notices, a hint to the used software name, a link to its homepage,
  the respective disclaimer of warranty, and a link to the GPL-V2 resp.
  GPL-V3.

\end{itemize}

\item[prohibits] nothing explictly with respect to this use case.

\end{description}


\subsection{GPL-06: Passing a modified program as source code}
\label{OSUC-04S-GPL} 

\begin{description}
\item[means] that you are going to distribute a modified version of the received
GPL licensed program, application, or server (proapse) to 3rd parties -- in the
form of source code files or as a source code package.
\item[covers] OSUC-04S\footnote{For details $\rightarrow$ OSLiC, pp.\
\pageref{OSUC-04S-DEF}}

\item[requires] the following tasks in order to fulfill the license conditions:
\begin{itemize}
  
  \item \textbf{[mandatory:]} Ensure that the licensing elements -- esp.\ all
  notices that refer to the GPL-V3 and to the absence of any
  warranty -- are retained in your package in the form you have received them.

  \item \textbf{[mandatory:]} Ensure that the distributed source code package
  contains a conspicuously and appropriately designed, easily to find copyright
  notice and a disclaimer of warranty. If these elements are missed, add a new
  file containing the main copyright notice and the disclaimer of warranty in the
  form which is textually defined by the license GPL-V3 itself. (Yes, repeat
  the disclaimer although it is also part of the license itself and although you
  are required to hand the license itself over to the receiver.)
  
  \item \textbf{[mandatory:]} Give the recipient a copy of the GPL-V3 license.  
  
  \item \textbf{[mandatory:]} Retain all existing copyright notices and
  licensing elements.
  
  \item \textbf{[mandatory:]} Let the copyright dialog of the program clearly
  say that it is a GPL licensed program. Let it reproduce the content of the
  existing copyright notices, a hint to the software name, a link to its
  homepage, the respective disclaimer of warranty, and a link to the GPL-V2
  resp. GPL-V3. If these conditions are not already fulfilled, add the missed
  elements.

  \item \textbf{[mandatory:]} Mark all modifications of source code of the
  program (proapse) thoroughly -- namely within the source code and including
  the date of the modification.
  
  \item \textbf{[mandatory:]} Organize your modifications in a way that they are
  covered by the existing GPL licensing statements.
  
  \item \textbf{[voluntary:]} Create a \emph{modification text file}, if such a
  notice file still does not exist. \emph{Expand} the \emph{modification text
  file} by a description of your modifications on a more functional level.
    
  \item \textbf{[voluntary:]} Let the documentation of your distribution and/or
  your additional material also reproduce the content of the existing
  copyright notices, a hint to the software name, a link to its homepage,
  the respective disclaimer of warranty, and a link to the GPL-V3.
  
 \end{itemize}
 
\item[prohibits] nothing explictly with respect to this use case.

\end{description}

\subsection{GPL-07: Passing a modified program as binary}
\label{OSUC-04B-GPL}

\begin{description}
\item[means] that you are going to distribute a modified version of the received
GPL licensed pro\-gram, application, or server (proapse) to 3rd parties -- in
the form of binary files or as a binary package.
\item[covers] OSUC-04B\footnote{For details $\rightarrow$ OSLiC, pp.\
\pageref{OSUC-04B-DEF}}

\item[requires] the following tasks in order to fulfill the license conditions:
\begin{itemize}

  \item \textbf{[mandatory:]} Ensure that the licensing elements -- esp.\ all
  notices that refer to the GPL-V3 and to the absence of any
  warranty -- are retained in your package in the form you have received them.

  \item \textbf{[mandatory:]} Ensure that the distributed software binary
  package contains a conspicuously and appropriately designed, easily to find
  copyright notice and a disclaimer of warranty. If these elements are missed,
  add a new file containing the main copyright notice and the disclaimer of
  warranty in the form which is textually defined by the license GPL-V3 itself.
  (Yes, repeat the disclaimer although it is also part of the license itself and
  although you are required to hand the license itself over to the receiver.)
  
  \item \textbf{[mandatory:]} Give the recipient a copy of the GPL-V3 license.
  If it is not already part of the software package, add it\footnote{For
  implementing the handover of files correctly $\rightarrow$ OSLiC, p.
  \pageref{DistributingFilesHint}}.
  
  \item \textbf{[mandatory:]} Retain all existing copyright notices and
  licensing elements.
  
  \item \textbf{[mandatory:]} Mark all modifications of source code of the
  program (proapse) thoroughly namely within the source code and including
  the date of the modification.

  \item \textbf{[mandatory:]} Let the copyright dialog of the program clearly
  say that it is a GPL licensed program. Let it reproduce the content of the
  existing copyright notices, a hint to the software name, a link to its
  homepage, the respective disclaimer of warranty, and a link to the GPL-V2
  resp. GPL-V3. If these conditions are not already fulfilled, add the missed
  elements.
  
  \item \textbf{[mandatory:]} Organize your modifications in a way that they are
  covered by the existing GPL licensing statements.
  
  \item \textbf{[mandatory:]} Make the source code of the distributed software
  accessible via a via a repository under your own control: Push the source code
  package into a repository, make it downloadable via the internet, and
  integrate an easily to find description into the distribution package which
  explains how the code can be received from where. Ensure, that this repository
  is online for for at least 3 years after having distributed the last instance
  of your software package.
  
  \item \textbf{[mandatory:]} Insert a prominent hint to the download repository
  into your distribution and/or your additional material.
  
  \item \textbf{[mandatory:]} Execute the to-do list of use case GPL-06\footnote{
  Making the code accessible via a repository means distributing the software in
  the form of source code. Hence, you must also fulfill all tasks of the
  corresponding use case.}.

  
  \item \textbf{[voluntary:]} Create a \emph{modification text file}, if such a
  notice file still does not exist. \emph{Expand} the \emph{modification text
  file} by a description of your modifications on a more functional level.
      
  \item \textbf{[voluntary:]} Let the documentation of your distribution and/or
  your additional material also reproduce the content of the existing
  copyright notices, a hint to the software name, a link to its homepage,
  the respective disclaimer of warranty, and a link to the GPL-V3.


\end{itemize}

\item[prohibits] nothing explictly with respect to this use case.

\end{description}

\subsection{GPL-08: Passing a modified library as independent source code}
\label{OSUC-08S-GPL}

\begin{description}
\item[means] that you are going to distribute a modified version of the received
GPL licensed code snippet, module, library, or plugin (snimoli) to 3rd parties
-- in the form of source code files or as a source code package, but without
embedding it into another larger software unit.
\item[covers] OSUC-08S\footnote{For details $\rightarrow$ OSLiC, pp.\
\pageref{OSUC-08S-DEF}}
\item[requires] the tasks in order to fulfill the license conditions:
\begin{itemize}
 
  \item \textbf{[mandatory:]} Ensure that the licensing elements -- esp.\ all
  notices that refer to the GPL-V2 or GPL-V3 and to the absence of any
  warranty -- are retained in your package in the form you have received them.

  \item \textbf{[mandatory:]} Ensure that the distributed source code package
  contains a conspicuously and appropriately designed, easily to find copyright
  notice and a disclaimer of warranty. If these elements are missed, add a new
  file containing the main copyright notice and the disclaimer of warranty in the
  form which is textually defined by the license GPL-V2 itself resp. by the
  GPL-V3 itself. (Yes, repeat the disclaimer although it is also part of the
  license itself and although you are required to hand the license itself over
  to the receiver.)
  
  \item \textbf{[mandatory:]} Give the recipient a copy of the GPL-V2 resp.
  GPL-V3 license. If it is not already part of the software package, add
  it\footnote{For implementing the handover of files correctly $\rightarrow$
  OSLiC, p. \pageref{DistributingFilesHint}}.
  
  \item \textbf{[mandatory:]} Retain all existing copyright notices and
  licensing elements.
  
  \item \textbf{[mandatory:]} Mark all modifications of source code of the
  library (snimoli) thoroughly -- namely within the source code and including
  the date of the modification.
  
  \item \textbf{[mandatory:]} Organize your modifications in a way that they are
  covered by the existing GPL licensing statements.
    
  \item \textbf{[voluntary:]} Create a \emph{modification text file}, if such a
  notice file still does not exist. \emph{Expand} the \emph{modification text
  file} by a description of your modifications.
  
  \item \textbf{[voluntary:]} Let the documentation of your distribution and/or
  your additional material also reproduce the content of the existing
  copyright notices, a hint to the software name, a link to its homepage,
  the respective disclaimer of warranty, and a link to the GPL-V2 resp.
  GPL-V3.

\end{itemize}

\item[prohibits] nothing with respect to this use case

\end{description}


\subsection{GPL-09: Passing a modified library as independent binary}
\label{OSUC-08B-GPL}

\begin{description}
\item[means] that you are going to distribute a modified version of the received
GPL licensed code snippet, module, library, or plugin (snimoli) to 3rd parties
-- in the form of binary files or as a binary package but without embedding it
into another larger software unit.
\item[covers] OSUC-08B\footnote{For details $\rightarrow$ OSLiC, pp.\
\pageref{OSUC-08B-DEF}}
\item[requires] the tasks in order to fulfill the license conditions:
\begin{itemize}

  \item \textbf{[mandatory:]} Ensure that the licensing elements -- esp.\ all
  notices that refer to the GPL-V2 or GPL-V3 and to the absence of any
  warranty -- are retained in your package in the form you have received them.

  \item \textbf{[mandatory:]} Ensure that the distributed software binary
  package contains a conspicuously and appropriately designed, easily to find
  copyright notice and a disclaimer of warranty. If these elements are missed,
  add a new file containing the main copyright notice and the disclaimer of
  warranty in the form which is textually defined by the license GPL-V2 itself
  resp. by the GPL-V3 itself. (Yes, repeat the disclaimer although it is also
  part of the license itself and although you are required to hand the license
  itself over to the receiver.)
  
  \item \textbf{[mandatory:]} Give the recipient a copy of the GPL-V2 resp.
  GPL-V3 license. If it is not already part of the software package, add
  it\footnote{For implementing the handover of files correctly $\rightarrow$
  OSLiC, p. \pageref{DistributingFilesHint}}.
  
  \item \textbf{[mandatory:]} Retain all existing copyright notices and
  licensing elements.

  \item \textbf{[mandatory:]} Make the source code of the distributed software
  accessible via a repository under your own control: Push the source code
  package into a repository, make it downloadable via the internet, and
  integrate an easily to find description into the distribution package which
  explains how the code can be received from where. Ensure, that this repository
  is online for for at least 3 years after having distributed the last instance
  of your software package.
  
  \item \textbf{[mandatory:]} Insert a prominent hint to the download repository
  into your distribution and/or your additional material.
  
  \item \textbf{[mandatory:]} Execute the to-do list of use case GPL-08\footnote{
  Making the code accessible via a repository means distributing the software in
  the form of source code. Hence, you must also fulfill all tasks of the
  corresponding use case.}.

  \item \textbf{[mandatory:]} Mark all modifications of source code of the
  library (snimoli) thoroughly -- namely within the source code and including
  the date of the modification.

  \item \textbf{[mandatory:]} Organize your modifications in a way that they are
  covered by the existing GPL licensing statements.
    
  \item \textbf{[voluntary:]} Create a \emph{modification text file}, if such a
  notice file still does not exist. \emph{Expand} the \emph{modification text
  file} by a description of your modifications.

  \item \textbf{[voluntary:]} Let the documentation of your distribution and/or
  your additional material also reproduce the content of the existing
  copyright notices, a hint to the software name, a link to its homepage,
  the respective disclaimer of warranty, and a link to the GPL-V2 resp.
  GPL-V3.
  
\end{itemize}

\item[prohibits] nothing with respect to this use case
\end{description}

\subsection{GPL-10: Passing a modified library as embedded source code}
\label{OSUC-10S-GPL}

\begin{description}
\item[means] that you are going to distribute a modified version of the received
GPL licensed code snippet, module, library, or plugin (snimoli) to 3rd parties
-- in the form of source code files or as a source code package together with
another larger software unit which contains this code snippet, module, library,
or plugin as an embedded component.
\item[covers] OSUC-10S\footnote{For details $\rightarrow$ OSLiC, pp.\
\pageref{OSUC-10S-DEF}}
\item[requires] the tasks in order to fulfill the license conditions:
\begin{itemize}


  \item \textbf{[mandatory:]} Ensure that the licensing elements -- esp.\ all
  notices that refer to the GPL-V2 or GPL-V3 and to the absence of any
  warranty -- are retained in your package in the form you have received them.

  \item \textbf{[mandatory:]} Ensure that the distributed source code package
  contains a conspicuously and appropriately designed, easily to find copyright
  notice and a disclaimer of warranty. If these elements are missed, add a new
  file containing the main copyright notice and the disclaimer of warranty in
  the form which is textually defined by the license GPL-V2 itself resp. by the
  GPL-V3 itself. (Yes, repeat the disclaimer although it is also part of the
  license itself and although you are required to hand the license itself over
  to the receiver.)
  
  \item \textbf{[mandatory:]} Give the recipient a copy of the GPL-V2 resp.
  GPL-V3 license. If it is not already part of the software package, add
  it\footnote{For implementing the handover of files correctly $\rightarrow$
  OSLiC, p. \pageref{DistributingFilesHint}}.
  
  \item \textbf{[mandatory:]} Retain all existing copyright notices and
  licensing elements.
  
  \item \textbf{[mandatory:]} Let the copyright dialog of the on-top development
  clearly say, that it uses the GPL licensed library and that it is itself
  licensed under the GPL-V2 resp. GPL-V3 too. Let it reproduce the content of
  the existing copyright notices, a hint to the software name, a link to its
  homepage, the respective disclaimer of warranty, and a link to the GPL-V2
  resp. GPL-V3.
    
  \item \textbf{[mandatory:]} Mark all modifications of source code of the
  embedded library (snimoli) thoroughly -- namely within the source code and
  including the date of the modification.
  
  \item \textbf{[mandatory:]} Organize your modifications of the embedded
  library in a way that they are covered by the existing GPL licensing
  statements. 
  
  \item \textbf{[mandatory:]} Organize the sources of the on-top development in
  a way that they are also covered by the GPL-V2 resp. GPL-V3 licensing
  statements.
     
  \item \textbf{[voluntary:]} Create a \emph{modification text file}, if such a
  notice file still does not exist. \emph{Expand} the \emph{modification text
  file} by a description of your modifications.
  
  \item \textbf{[voluntary:]} Let the documentation of your distribution and/or
  your additional material also reproduce the content of the existing
  copyright notices, a hint to the software name, a link to its homepage,
  the respective disclaimer of warranty, and a link to the GPL-V2 resp.
  GPL-V3.
  
\end{itemize}

\item[prohibits] nothing with respect to this use case

\end{description}

\subsection{GPL-11: Passing a modified library as embedded binary}
\label{OSUC-10B-GPL}

\begin{description}
\item[means] that you are going to distribute a modified version of the received
GPL licensed code snippet, module, library, or plugin to 3rd parties -- in the
form of binary files or as a binary package together with another larger
software unit which contains this code snippet, module, library, or plugin as an
embedded component.
\item[covers] OSUC-10B\footnote{For details $\rightarrow$ OSLiC, pp.\
\pageref{OSUC-10B-DEF}}
\item[requires] the tasks in order to fulfill the license conditions:
\begin{itemize}

  \item \textbf{[mandatory:]} Ensure that the licensing elements -- esp.\ all
  notices that refer to the GPL-V2 or GPL-V3 and to the absence of any
  warranty -- are retained in your package in the form you have received them.

  \item \textbf{[mandatory:]} Ensure that the distributed software binary
  package contains a conspicuously and appropriately designed, easily to find
  copyright notice and a disclaimer of warranty. If these elements are missed,
  add a new file containing the main copyright notice and the disclaimer of
  warranty in the form which is textually defined by the license GPL-V2 itself
  resp. by the GPL-V3 itself. (Yes, repeat the disclaimer although it is also
  part of the license itself and although you are required to hand the license
  itself over to the receiver.)
  
  \item \textbf{[mandatory:]} Give the recipient a copy of the GPL-V2 resp.
  GPL-V3 license. If it is not already part of the software package, add
  it\footnote{For implementing the handover of files correctly $\rightarrow$
  OSLiC, p. \pageref{DistributingFilesHint}}.
  
  \item \textbf{[mandatory:]} Retain all existing copyright notices and
  licensing elements.

  \item \textbf{[mandatory:]} Make the source code of the embedded library and
  the source code of your overarching program accessible via a repository under
  your own control: Push the source code package into a repository and make it
  downloadable via the internet. Integrate an easily to find description into
  the distribution package which explains how the code can be received from
  where. Ensure, that this repository is online for for at least 3 years after
  having distributed the last instance of your software package.

  \item \textbf{[mandatory:]} Insert a prominent hint to the download repository
  into your distribution and/or your additional material.
    
  \item \textbf{[mandatory:]} Execute the to-do list of use case GPL-10\footnote{
  Making the code accessible via a repository means distributing the software in
  the form of source code. Hence, you must also fulfill all tasks of the
  corresponding use case.}.

  \item \textbf{[mandatory:]} Let the copyright dialog of the on-top development
  clearly say, that it uses the GPL licensed library and that it is itself
  licensed under the GPL-V2 resp. GPL-V3 too. Let it reproduce the content of
  the existing copyright notices, a hint to the software name, a link to its
  homepage, the respective disclaimer of warranty, and a link to the GPL-V2
  resp. GPL-V3.
  
  \item \textbf{[mandatory:]} Mark all modifications of source code of the
  embedded library (snimoli) thoroughly -- namely within the source code and
  including the date of the modification.
  
  \item \textbf{[mandatory:]} Organize your modifications of the embedded
  library in a way that they are covered by the existing GPL licensing
  statements. 
  
  \item \textbf{[mandatory:]} Organize the binaries of the on-top development in
  a way that they are also covered by the GPL-V2 resp. GPL-V3 licensing
  statements.
       
  \item \textbf{[voluntary:]} Create a \emph{modification text file}, if such a
  notice file still does not exist. \emph{Expand} the \emph{modification text
  file} by a description of your modifications.
    
  \item \textbf{[voluntary:]} Let the documentation of your distribution and/or
  your additional material also reproduce the content of the existing copyright
  notices, a hint to the software name, a link to its homepage, the respective
  disclaimer of warranty, and a link to the GPL-V2 resp.
  GPL-V3.
  
\end{itemize}

\item[prohibits] nothing with respect to this use case

\end{description}

\subsection{Discussions and Explanations}

\begin{itemize}
  
  \item The GPL-V2 allows to \enquote{[\ldots] to copy \emph{and} (to)
  distribute verbatim copies of the Program's complete source code as you
  receive it [...] provided that you [a] conspicuously and appropriately publish
  on each copy an appropriate copyright notice and disclaimer of warranty; [b]
  keep intact all the notices that refer to this License and to the absence of
  any warranty; and [c] distribute a copy of this License along with the
  Program}\footcite[cf.][\nopage wp.\ §1, emphasizes by
  KR]{Gpl20OsiLicense1991a}. Additionally, the GPL-V2 allows to
  \enquote{[\ldots] copy and distribute [\ldots] modifications (of the Program
  or any portion of it) [\ldots] under the terms of Section
  1}\footcite[cf.][\nopage wp.\ §2]{Gpl20OsiLicense1991a} while it allows to
  distribute binaries \enquote{under the terms of Sections 1 and
  2}\footcite[cf.][\nopage wp.\ §4]{Gpl20OsiLicense1991a}. But the GPL does not
  require any tasks if you are using the work only for yourself. Thus, the
  quoted conditions of \enquote{Section 1} are mandatory for all use cases
  concerning the distribution of an GPL licensed work (GPL-02 - GPL-11)\footnote{
  \label{Gpl3ConditionsDistri}
  The GPL-V3 uses a similar structure to establish the same requirements:
  In §4 it allows to \enquote{[\ldots] convey verbatim copies of the Program's
  source code as you receive it [\ldots] provided that you conspicuously and
  appropriately publish on each copy an appropriate copyright notice; keep
  intact all notices stating that this License and any non-permissive terms
  added in accord with section 7 apply to the code; keep intact all notices of
  the absence of any warranty; and give all recipients a copy of this License
  along with the Program}. Additionally in §5 it also allows to
  \enquote{[\ldots] convey [\ldots] modifications [\ldots] under the terms of
  section 4 [\ldots]} and in §6 it allows to \enquote{[\ldots] convey a covered
  work in object form under the terms of sections of 4 and 5}
  (\cite[cf.][\nopage wp.\ §4, §5, §6]{Gpl30OsiLicense2007a}). In opposite to
  the GPL-V2, the GPL-V3 explicitly states that one \enquote{[\ldots] may
  make, run and propagate covered works that (one) (does) not convey
  [distribute], without conditions so long as (the GPL-V3) license otherwise
  remains in force} (\cite[cf.][\nopage wp.\ §2]{Gpl30OsiLicense2007a}).
  Moreover, giving a package to a third party for getting a modified version
  back has not to be taken as a case of distribution if the modification has
  only been executed on behalf and only for the purpose of the purchaser and if
  the modified version is not distributed to any third party (\cite[cf.][\nopage
  wp.\ §2]{Gpl30OsiLicense2007a}). If one collects all these GPL-V3 statements
  together, than one may conclude that the tasks which fulfill the corresponding
  GPL-V2 requirements together also fit the GPL-V3 conditions.}.
  
  \item The GPL-V2 allows to \enquote{[\ldots] copy and (to) distribute the
  Program (or a work based on it [\ldots]) in object code or executable form
  [\ldots] provided that you accompany it with the complete corresponding
  machine-readable source code [\ldots] on a medium customarily used for
  software interchange}\footcite[cf.][\nopage wp.\ §3,
  §3a]{Gpl20OsiLicense1991a}. As a substitution for this basic condition, the
  GPL-V2 allows to \enquote{accompany} the binary distribution package
  \enquote{[\ldots] with a written offer, valid for at least three years, to
  give any third party, for a charge no more than your cost of physically
  performing source distribution, a complete machine-readable copy of the
  corryponding source code [\ldots] on a medium customarily used for software
  interchange}\footcite[cf.][\nopage wp.\ §3b]{Gpl20OsiLicense1991a}. The OSLiC
  construes the common technique to download files from the internet as a
  distribution \emph{on a medium [being today] customarily used for software
  interchange}. Therefore, the OSLiC requires for all open source use cases
  which refer to the distribution of binaries (GPL-03, GPL-07, GPL-09, GPL-11) to
  make the corresonding source code of the library itself accessible via an
  internet repository under your own control\footnote{\label{Gpl3CondCopyleft}
  Also the GPL-V3 explicitly requires to make the source code accessible in
  case of distributing binaries. But in opposite to the GPL-V2, the GPL-V3
  explicitly offers the option that giving the \enquote{[\ldots] access to copy
  the Corresponding Source from a network server at no charge} would fulfill the
  conditions (\cite[cf.][\nopage wp.\ §6 and §6b]{Gpl30OsiLicense2007a}). So
  again, the tasks which ensure to act in accordance to the GPL-V2 license in
  case of distributing binaries, also fulfill the conditions of the GPL-V3}.
  The weakness that in this case \enquote{third parties [which have received the
  binaries] are not compelled to copy the source code [\ldots]} is mediately
  accepted by the GPL\footcite[cf.][\nopage wp.\ §3, at the
  end]{Gpl20OsiLicense1991a}. But the necessity to offer the source code via a
  repository being controled by yourself (mostly) may not be circumvented: The
  GPL-V2 allows to redistribute a link to an external source code repository
  only in case of \enquote{noncommercial distributions}\footcite[cf.][\nopage
  wp.\ §3c]{Gpl20OsiLicense1991a}.
  
  \item Both, the GPL-V2 and the GPL-V3 allow to \enquote{[\ldots] modify your
  copy or copies of the Program or any portion of it [\ldots] and (to) copy and
  distribute such modifications [\ldots]} only under very similar restrictions
  and conditions\footcite[cf.][\nopage wp.\ §2]{Gpl20OsiLicense1991a}:
  \begin{itemize}
    \item First, modified files must be marked as modifications and marked by
    the date of the modification\footnote{For GPL-V2 see \cite[cf.][\nopage wp.\
    §2]{Gpl20OsiLicense1991a}. For GPL-V3 see \cite[cf.][\nopage wp.\
    §5]{Gpl30OsiLicense2007a}}. These conditions must be respected by all open
    source use cases concerning the distribution of the modified work [GPL-06 -
    GPL -9], because even if one primarily intends to distribute binaries, one
    has lateron also to deliver the source code. The OSLiC rewrites this
    requirement by the mandatory condition to mark each modified file and by the
    voluntary condition to update / generate a general changing file.
    
    \item Second, the GPL requires that all copies of the modified software
    which are using an interactive interface or a method to display messages
    must \enquote{[\ldots] print or display an announcement including an
    appropriate copyright notice and a notice that there is no warranty [\ldots]
    and that users may redistribute the program under these conditions, and
    telling the user how to view a copy of this License}\footnote{For GPL-V2
    see \cite[cf.][\nopage wp.\ §2c]{Gpl20OsiLicense1991a}. For GPL-V3 see
    \cite[cf.][\nopage wp.\ §5d]{Gpl30OsiLicense2007a}}. The OSLiC rewrites this
    condition in the form that the work shall let its copyright dialog clearly
    reproduce the content of the existing copyright notices, a hint to the
    software name, a link to its homepage, the respective disclaimer of
    warranty, and a link to the GPL-V2-file resp. the GPL-V3-file which has to
    be delivered together with the software. These conditions have to be
    respected if one redistributes the received and then modified programs
    (GPL-06, GPL-07) or if one distributes own programs which are using (modified)
    libraries as embedded components (GPL-10, GPL-11). For those open source use
    cases which concern the redistribution of  received and modified libraries
    etc., the OSLiC does not mention these requirements because libraries,
    plugins, or snippets normally do not offer their own copyright dialogs.
    
    \item Third, the GPL requires to \enquote{ [\ldots] cause any work (being
    distributed or published), that in whole or in part contains or is derived
    from the Program or any part thereof, to be licensed as a whole at no charge
    to all third parties under the terms of this (GPL)}\footnote{For GPL-V2 see
    cite[cf.][\nopage wp.\ §2b]{Gpl20OsiLicense1991a}. For GPL-V3 see
    cite[cf.][\nopage wp.\ §5c]{Gpl30OsiLicense2007a}}. This requirement does
    not depend of the form in which the software is distributed. The OSLiC
    adopts this statement in the following way:
    \begin{itemize}
      \item For all open source use cases which concern the distribution (GPL-02
      \ldots GPL-11), the OSLiC rewrites this condition as the mandatory
      requirement to retain all existing licensing elements.
      
      \item For all use cases which deal with the distribution of a modified
      version of the software (GPL-06 \ldots GPL-11), the OSliC adds the
      requirement to organize the modifications in a way that they are covered
      by the respective GPL-V2 oder GPL-V3 licensing statements.
      
      \item For the use case which deal with the distribution of an embedded
      library (GLP-04,GPL-05,GPL-10,GPL-11) the OSLiC requires also to license
      the on-top development under the terms of the respective GPL-V2 or GPL-V3
      license.
    \end{itemize}
   
    \item Finally, as parts of those task lists which concern the distribution
    in the form of binaries, the OSLiC reminds the reader also to execute the
    corresponding source code use cases because distributing the binaries
    without making the corresponding sources accessible is not allowed by the
    GPL.
  \end{itemize}
  

\end{itemize}





%\bibliography{../../../bibfiles/oscResourcesEn}
