% Telekom osCompendium 'for being included' snippet template
%
% (c) Karsten Reincke, Deutsche Telekom AG, Darmstadt 2011
%
% This LaTeX-File is licensed under the Creative Commons Attribution-ShareAlike
% 3.0 Germany License (http://creativecommons.org/licenses/by-sa/3.0/de/): Feel
% free 'to share (to copy, distribute and transmit)' or 'to remix (to adapt)'
% it, if you '... distribute the resulting work under the same or similar
% license to this one' and if you respect how 'you must attribute the work in
% the manner specified by the author ...':
%
% In an internet based reuse please link the reused parts to www.telekom.com and
% mention the original authors and Deutsche Telekom AG in a suitable manner. In
% a paper-like reuse please insert a short hint to www.telekom.com and to the
% original authors and Deutsche Telekom AG into your preface. For normal
% quotations please use the scientific standard to cite.
%
% [ Framework derived from 'mind your Scholar Research Framework' 
%   mycsrf (c) K. Reincke 2012 CC BY 3.0  http://mycsrf.fodina.de/ ]
%


%% use all entries of the bibliography
%\nocite{*}

\section{Microsoft Public License}


The MS-PL license is also one of the most permissive licenses. Thus, the
MS-PL specific finder can be simplified:


\tikzstyle{nodv} = [font=\small, ellipse, draw, fill=gray!10, 
    text width=2cm, text centered, minimum height=2em]

\tikzstyle{nods} = [font=\footnotesize, rectangle, draw, fill=gray!20, 
    text width=1.2cm, text centered, rounded corners, minimum height=3em]

\tikzstyle{nodb} = [font=\footnotesize, rectangle, draw, fill=gray!20, 
    text width=2.2cm, text centered, rounded corners, minimum height=3em]
    
\tikzstyle{leaf} = [font=\tiny, rectangle, draw, fill=gray!30, 
    text width=1.2cm, text centered, minimum height=6em]

\tikzstyle{edge} = [draw, -latex']

\begin{tikzpicture}[]

\node[nodv] (l71) at (4,10) {MS-PL};

\node[nodb] (l61) at (0,8.6) {\textit{recipient:} \\ \textbf{4yourself}};
\node[nodb] (l62) at (6.5,8.6) {\textit{recipient:} \\ \textbf{2others}};

\node[nodb] (l51) at (2.5,7) {\textit{state:} \\ \textbf{unmodified}};
\node[nodb] (l52) at (9.3,7) {\textit{state:} \\ \textbf{modified}};

\node[nodb] (l43) at (6.5,5.4) {\textit{type:} \\ \textbf{proapse}};
\node[nodb] (l44) at (12,5.4) {\textit{type:} \\ \textbf{snimoli}};


\node[nods] (l31) at (5.4,3.8) {\textit{form:} \textbf{source}};
\node[nods] (l32) at (7.2,3.8) {\textit{form:} \textbf{binary}};
\node[nodb] (l33) at (10,3.8) {\textit{context:} \\ \textbf{independent}};
\node[nodb] (l34) at (13.5,3.8) {\textit{context:} \\ \textbf{embedded}};

\node[nods] (l21) at (9,2.2) {\textit{form:} \textbf{source}};
\node[nods] (l22) at (10.8,2.2) {\textit{form:} \textbf{binary}};
\node[nods] (l23) at (12.6,2.2) {\textit{form:} \textbf{source}};
\node[nods] (l24) at (14.4,2.2) {\textit{form:} \textbf{binary}};

\node[leaf] (l11) at (0,0) {\textbf{MS-PL-C1} \textit{using software only
for yourself}};

\node[leaf] (l12) at (2.5,0) { \textbf{MS-PL-C2} \textit{ distributing unmodified
software}};

\node[leaf] (l14) at (5.4,0) { \textbf{MS-PL-C3}  \textit{ distributing modified
program as sources}};

\node[leaf] (l15) at (7.2,0) { \textbf{MS-PL-C4}  \textit{ distributing modified
program as binaries}};

\node[leaf] (l16) at (9,0) { \textbf{MS-PL-C5}  \textit{ distributing modified
library as independent sources}};

\node[leaf] (l17) at (10.8,0) { \textbf{MS-PL-C6} \textit{distributing modified
library as independent binaries}};

\node[leaf] (l18) at (12.6,0) { \textbf{MS-PL-C7}  \textit{distributing
modified library as embedded sources}};

\node[leaf] (l19) at (14.4,0) { \textbf{MS-PL-C8}  \textit{ distributing modified
library as embedded binaries}};


\path [edge] (l71) -- (l61);
\path [edge] (l71) -- (l62);
\path [edge] (l61) -- (l11);
\path [edge] (l62) -- (l51);
\path [edge] (l62) -- (l52);
\path [edge] (l52) -- (l43);
\path [edge] (l52) -- (l44);
\path [edge] (l51) -- (l12);


\path [edge] (l43) -- (l31);
\path [edge] (l43) -- (l32);
\path [edge] (l44) -- (l33);
\path [edge] (l44) -- (l34);
\path [edge] (l31) -- (l14);
\path [edge] (l32) -- (l15);
\path [edge] (l33) -- (l21);
\path [edge] (l33) -- (l22);
\path [edge] (l34) -- (l23);
\path [edge] (l34) -- (l24);
\path [edge] (l21) -- (l16);
\path [edge] (l22) -- (l17);
\path [edge] (l23) -- (l18);
\path [edge] (l24) -- (l19);

\end{tikzpicture}



\subsection{MS-PL-C1: Using the software only for yourself}
\label{OSUC-01-MS-PL} 
\label{OSUC-03-MS-PL} 
\label{OSUC-06-MS-PL}
\label{OSUC-09-MS-PL}
  
\begin{description}
  \item[means] that you are going to use a received MS-PL software only for
  yourself and that you do not hand it over to any 3rd party in any sense.
  \item[covers] OSUC-01, OSUC-03, OSUC-06, and OSUC-09\footnote{For details see
  pp.\ \pageref{OSUC-01-DEF} - \pageref{OSUC-09-DEF}}
  \item[requires] no tasks in order to fulfill the conditions of the MS-PL 
license with respect to this use case:
  \begin{itemize}
    \item You are allowed to use any kind of MS-PL licensed software in any
    sense and in any context without any other obligations if you do not
    handover the software to 3rd parties.
  \end{itemize}
\item[prohibits] to use any contributors' name, logo, or trademarks (without an
additional or general legally based approval).
\end{description}


\subsection{MS-PL-C2: Passing the unmodified software}
\label{OSUC-02S-MS-PL} \label{OSUC-05S-MS-PL} \label{OSUC-07S-MS-PL} 
\label{OSUC-02B-MS-PL} \label{OSUC-05B-MS-PL} \label{OSUC-07B-MS-PL} 

\begin{description}
\item[means] that you are going to distribute an unmodified version of the
received MS-PL software to 3rd parties -- in the form of binaries or as source
code files. In this case it is not discriminating to distribute a
program, an application, a server, a snippet, a module, a library, or a plugin
as an independent package.

\item[covers] OSUC-02S, OSUC-02B, OSUC-05S,  OSUC-05B, OSUC-07S,
OSUC-07B\footnote{For details $\rightarrow$ OSLiC, pp.\ \pageref{OSUC-02B-DEF} -
\pageref{OSUC-07B-DEF}}

\item[requires] the following tasks in order to fulfill the license conditions:
\begin{itemize}
  \item \textbf{[mandatory:]} Ensure that all licensing elements -- esp.\ all
  copyright, patent, trademark, and attribution notices that are part of the
  version you received -- are completely retained in your package.
  
  \item \textbf{[mandatory:]} Incorporate a complete copy of the MS-PL license
  into your package, regardless whether you distribute a source code or a binary
  package\footnote{$\rightarrow$ OSLiC, p.\ \pageref{MsplSourceBinHint}}.
  
  \item \textbf{[voluntary:]} It's a good tradition to let the documentation of
  your distribution and/or your additional material also contain a link to the
  original software (project) and its homepage.
\end{itemize}

\item[prohibits] to use any contributors' name, logo, or trademarks (without an
additional or general legally based approval).

\end{description}

\subsection{MS-PL-C3: Passing a modified program as source code}
\label{OSUC-04S-MS-PL}

\begin{description}

\item[means] that you are going to distribute a modified version of the received
MS-PL licensed program, application, or server (proapse) to 3rd parties - in
the form of source code files or source code package.

\item[covers] OSUC-04S\footnote{For details $\rightarrow$ OSLiC, pp.\
\pageref{OSUC-04S-DEF}}

\item[requires] the following tasks in order to fulfill the license conditions:
\begin{itemize}
  \item \textbf{[mandatory:]} Ensure that all licensing elements -- esp.\ all
  copyright, patent, trademark, and attribution notices that are part of the
  version you received -- are completely retained in your package.
 
  \item \textbf{[mandatory:]} Incorporate a complete copy of the MS-PL license
  into your package.
  
  \item \textbf{[mandatory:]} If you do not want to publish your modifications
  under the MS-PL too, then cleanly separate your own sources and licensing
  documents from original elements of the adopted work.
  
  \item \textbf{[voluntary:]} Mark your modifications in the sourcecode.
  
  \item \textbf{[voluntary:]} It's a good tradition to let the documentation of
  your distribution and/or your additional material also contain a link to the
  original software (project) and its homepage (as far as this does not clashes
  with the prohibitions stated below).
  
  \item \textbf{[voluntary:]} You are allowed to expand an existing copyright
  notice of the program to mention your own contributions.
  
  \item \textbf{[voluntary:]} It is a good practice of the open source
  community, to let the copyright notice which is shown by the running program
  also state that the program is licensed under the MS-PL license (as far as
  this does not clashes with the prohibitions stated below). Because you are
  already modifying the program, you can also add such a hint, if the 
  original copyright notice lacks such a statement.
    
\end{itemize}

\item[prohibits] to use any contributors' name, logo, or trademarks (without an
additional or general legally based approval).

\end{description}


\subsection{MS-PL-C4: Passing a modified program as binary}
\label{OSUC-04B-MS-PL}

\begin{description}

\item[means] that you are going to distribute a modified version of the received
MS-PL licensed program, application, or server (proapse) to 3rd parties -- in
the form of binary files or as bianry package.

\item[covers] OSUC-04B\footnote{For details $\rightarrow$ OSLiC, pp.\
\pageref{OSUC-04B-DEF}}

\item[requires] the following tasks in order to fulfill the license conditions:
\begin{itemize}
  
  \item \textbf{[voluntary:]} Mark your modifications in the source code even if
  you do not intend to distribute it.
  
  \item \textbf{[voluntary:]} It's a good tradition to let the documentation of
  your distribution and/or your additional material also contain a link to the
  original software (project) and its homepage (as far as this does not clashes
  with with the prohibitions stated below).
  
  \item \textbf{[voluntary:]} It is a good practice of the open source
  community, to let the copyright notice which is shown by the running program
  also state that the derivative work is based on a version originally licensed
  under the MS-PL license (as far as this does not clashes with the prohibitions
  stated below) -- perhaps by linking to the project homepage of the original.
  Because you are already modifying the program, you can also add such a hint,
  if the original copyright notice lacks such a statement.
    
\end{itemize}

\item[prohibits] to use any contributors' name, logo, or trademarks (without an
additional or general legally based approval)

\end{description}



\subsection{MS-PL-C5: Passing a modified library independently as source code}
\label{OSUC-08S-MS-PL}
\begin{description}
\item[means] that you are going to distribute a modified version of the received
MS-PL code snippet, module, library, or plugin (snimoli) to 3rd parties -- in
the form of source code but without embedding it into another larger software
unit.
\item[covers] OSUC-08S\footnote{For details $\rightarrow$ OSLiC, pp.\
\pageref{OSUC-08S-DEF}}
\item[requires] the following tasks in order to fulfill the license conditions:

\begin{itemize}
  \item \textbf{[mandatory:]} Ensure that all licensing elements -- esp.\ all
  copyright, patent, trademark, and attribution notices that are part of the
  version you received -- are completely retained in your package.
 
  \item \textbf{[mandatory:]} Incorporate a complete copy of the MS-PL license
  into your package.
  
  \item \textbf{[mandatory:]} If you do not want to publish your modifications
  under the MS-PL too, then cleanly separate your own sources and licensing
  documents from original elements of the adopted part(s).
  
  \item \textbf{[voluntary:]} Mark your modifications in the sourcecode.
  
  \item \textbf{[voluntary:]} It's a good tradition to let the documentation of
  your distribution and/or your additional material also contain a link to the
  original software (project) and its homepage (as far as this does not clashes
  with with the prohibitions stated below).
  
\end{itemize}

\item[prohibits] to use any contributors' name, logo, or trademarks (without an
additional or general legally based approval)

\end{description}

\subsection{MS-PL-C6: Passing a modified library independently as binary}
\label{OSUC-08B-MS-PL}
\begin{description}
\item[means] that you are going to distribute a modified version of the received
MS-PL code snippet, module, library, or plugin (snimoli) to 3rd parties -- in
the form of binary files but without embedding it into another larger software
unit.
\item[covers] OSUC-08B\footnote{For details $\rightarrow$ OSLiC, pp.\
\pageref{OSUC-08B-DEF}}
\item[requires] the following tasks in order to fulfill the license conditions:

\begin{itemize}
  
  \item \textbf{[voluntary:]} Mark your modifications in the source code even if
  do not want to distribute it.
  
  \item \textbf{[voluntary:]} It's a good tradition to let the documentation of
  your distribution and/or your additional material also contain a link to the
  original software (project) and its homepage (as far as this does not clashes
  with with the prohibitions stated below).
    
\end{itemize}

\item[prohibits] to use any contributors' name, logo, or trademarks (without an
additional or general legally based approval)

\end{description}

\subsection{MS-PL-C7: Passing a modified library as embedded source code}
\label{OSUC-10S-MS-PL}
\begin{description}
\item[means] that you are going to distribute a modified version of the received
MS-PL licensed code snippet, module, library, or plugin (snimoli) to 3rd parties
-- in the form of source code files or as a source code package together with
another larger software unit which contains this code snippet, module, library,
or plugin as an embedded component.

\item[covers] OSUC-10S\footnote{For details $\rightarrow$ OSLiC, pp.\
\pageref{OSUC-10S-DEF}}
\item[requires] the following tasks in order to fulfill the license conditions:
\begin{itemize}
 
 \item \textbf{[mandatory:]} Ensure that all licensing elements -- esp.\ all
  copyright, patent, trademark, and attribution notices that are part of the
  version you received -- are completely retained in your package.
 
  \item \textbf{[mandatory:]} Incorporate a complete copy of the MS-PL license
  into your package.
  
  \item \textbf{[mandatory:]} If you do not want to publish your modifications
  and/or your overarching application under the MS-PL too, then cleanly separate
  your own sources and licensing documents from original elements of the adopted
  work.
  
  \item \textbf{[voluntary:]} Mark your modifications in the sourcecode.
  
  \item \textbf{[voluntary:]} It's a good tradition to let the documentation of
  your distribution and/or your additional material also contain a link to the
  original software (project) and its homepage (as far as this does not clashes
  with with the prohibitions stated below).

  \item \textbf{[voluntary:]} It is a good practice of the open source
  community, to let the copyright notice shown by your overarching program also
  state that it is based on a component originally licensed under the MS-PL
  license -- perhaps by linking the project homepage of the original (as far as
  this does not clashes with the prohibitions stated below).
  
\end{itemize}

\item[prohibits] to use any contributors' name, logo, or trademarks (without an
additional or general legally based approval)

\end{description}

\subsection{MS-PL-C8: Passing a modified library as embedded binary}
\label{OSUC-10B-MS-PL}
\begin{description}
\item[means] that you are going to distribute a modified version of the received
MS-PL licensed code snippet, module, library, or plugin (snimoli) to 3rd parties
-- in the form of a binary package together with another larger software unit
which contains this code snippet, module, library, or plugin as an embedded
component.

\item[covers] OSUC-10B\footnote{For details $\rightarrow$ OSLiC, pp.\
\pageref{OSUC-10B-DEF}}
\item[requires] the following tasks in order to fulfill the license conditions:
\begin{itemize}
 
  \item \textbf{[voluntary:]} Mark your modifications in the source code even if
  do not want to distribute it.
  
  \item \textbf{[voluntary:]} It's a good tradition to let the documentation of
  your distribution and/or your additional material also contain a link to the
  original software (project) and its homepage (as far as this does not clashes
  with with the prohibitions stated below).
  
  \item \textbf{[voluntary:]} It is a good practice of the open source
  community, to let the copyright notice shown by your own overarching program
  also state that it is based on a component originally licensed under the MS-PL
  license -- perhaps by linking the project homepage of the original (as far as
  this does not clashes with the prohibitions stated below).
  
\end{itemize}

\item[prohibits] to use any contributors' name, logo, or trademarks (without an
additional or general legally based approval)

\end{description}

\subsection{Discussions and Explanations}

The MS-PL is also a very permissive and short license. It requires to do:
(a) You must preserve existing licensing elements. (b) You must distribute
the source code as whole or \enquote{portions} of the source code under the
MS-PL. (c) You must add a copy of the license if you distribute (parts of) the
source code. (d) If you distribute a binary package, you must distribute (the
parts of) the work under a license \enquote{that complies with this (MS-PL)
license}\footcite[cf.][\nopage wp]{MsplOsiLicense2013a}.

The most confusing clause is probably the condition, to \enquote{[\ldots]
distribute any portion of the software in compiled or object code form [\ldots]
only [\ldots] under a license that complies with this license}. But a closer
examination is lighting the situation: The only other conditions of the license
which refer to the context of distributing binaries are the requirements a) not
to abuse trademarks, b) not to bring a patent claim against any contributor, and
c) not to expect any warranties or guarantees with respect to the distributed
portion\footcite[cf.][\nopage wp.\ §3A, §3B, §3E]{MsplOsiLicense2013a}.

Based on these readings we decided \ldots

\label{MsplSourceBinHint} 
\begin{itemize}
  \item \ldots to let you incorporate a copy of the license into your
  distribution even if it only contains the binaries of the unmodified version:
  if you have not modified it, you do not lose any advantage if you add the
  license, too. So, this is the best method to fulfill the \emph{MSL-PL binary
  condition}.
  \item \ldots to erase all mandatory conditions in case of the binary
  distributions: the patent restriction of the MS-PL itself is already covered
  by the MS-PL patent section of the OSLiC\footnote{$\rightarrow$ OSLiC, p.\
  \pageref{subsec:MsplPatentClause}} and the no warranty clause of the MS-PL by
  the OSLiC section concerning the power of the MS-PL\footnote{$\rightarrow$
  OSLiC, p.\ \pageref{sec:ProtectingPowerOfMspl}} while the trademark
  restrictions are explicitly added into the prohibition section.
  \item \ldots to erase the hints to a voluntarily updated copyright dialog in
  case of distributing a snimoli independently because the copyright dialog
  normally is designed by the overarching work which uses the library, not by
  the library itself.
\end{itemize}


%\bibliography{../../../bibfiles/oscResourcesEn}
