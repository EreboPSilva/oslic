% Telekom osCompendium 'for being included' snippet template
%
% (c) Karsten Reincke, Deutsche Telekom AG, Darmstadt 2011
%
% This LaTeX-File is licensed under the Creative Commons Attribution-ShareAlike
% 3.0 Germany License (http://creativecommons.org/licenses/by-sa/3.0/de/): Feel
% free 'to share (to copy, distribute and transmit)' or 'to remix (to adapt)'
% it, if you '... distribute the resulting work under the same or similar
% license to this one' and if you respect how 'you must attribute the work in
% the manner specified by the author ...':
%
% In an internet based reuse please link the reused parts to www.telekom.com and
% mention the original authors and Deutsche Telekom AG in a suitable manner. In
% a paper-like reuse please insert a short hint to www.telekom.com and to the
% original authors and Deutsche Telekom AG into your preface. For normal
% quotations please use the scientific standard to cite.
%
% [ Framework derived from 'mind your Scholar Research Framework' 
%   mycsrf (c) K. Reincke 2012 CC BY 3.0  http://mycsrf.fodina.de/ ]
%


%% use all entries of the bibliography
%\nocite{*}

\section{Microsoft Public License}


The MS-PL license is also one of the most permissive licenses. Thus, the
MS-PL specific finder can be simplified:

\begin{center}
\begin{footnotesize}
\pstree[levelsep=*1,treesep=0.2]{\Toval{MS-PL}}{
  \pstree{
    \Tr{\Ovalbox{\shortstack{recipient: \textit{4yourself}\\
    \textbf{\textit{used by yourself}}}}} 
  }{
    \Tr{\doublebox{\shortstack{\tiny{\textbf{MS-PL-1}:}\\
    \tiny{\textit{using the}}\\\tiny{\textit{software}}\\
    \tiny{\textit{only for}}\\\tiny{\textit{yourself}} }}} 
  }
  \pstree[levelsep=*0.2,treesep=0.2]{
    \Tr{\Ovalbox{\shortstack{recipient: \textit{4others}\\
      \textbf{\textit{distributed to 3rd parties}}}}} 
  }{ 
    \pstree[levelsep=*0.2,treesep=0.2]{
      \Tr{\Ovalbox{\shortstack{state:\\\textbf{\textit{unmodified}}}}}
    }{
        \Tr{\doublebox{\shortstack{\tiny{\textbf{MS-PL-2}:}\\
        \tiny{\textit{distributing an}}\\
        \tiny{\textit{unmodified pkg}} }}}

    }

    \pstree[levelsep=*0.2,treesep=0.2]{
      \Tr{\Ovalbox{\shortstack{state:\\\textbf{\textit{modified}}}}}
    }{ 
      \pstree{
        \Tr{\Ovalbox{\shortstack{type:\\\textbf{\textit{proapse}}}}}
      }{
          \Tr{\doublebox{\shortstack{\tiny{\textbf{MS-PL-3}:}\\
          \tiny{\textit{distributing}}\\\tiny{\textit{a modified}}\\
          \tiny{\textit{program}} }}} 
           
      }
      \pstree{
        \Tr{\Ovalbox{\shortstack{type:\\\textbf{\textit{snimoli}}}}}
      }{
        \pstree{
          \Tr{\Ovalbox{\shortstack{context:\\\textbf{\textit{independent}}}}}
        }{        
            \Tr{\doublebox{\shortstack{\tiny{\textbf{MS-PL-4}:}\\
            \tiny{\textit{distributing}}\\\tiny{\textit{a modified}}\\
            \tiny{\textit{library as}}\\\tiny{\textit{independent}}
            \tiny{\textit{pkg}} }}} 
           
        }
        \pstree{
          \Tr{\Ovalbox{\shortstack{context:\\\textbf{\textit{embedded}}}}}
        }{        
            \Tr{\doublebox{\shortstack{\tiny{\textbf{MS-PL-5}:}\\
            \tiny{\textit{distributing}}\\\tiny{\textit{a modified}}\\
            \tiny{\textit{library as}}\\\tiny{\textit{embedded}}
            \tiny{\textit{pkg}} }}} 
           
        }


      }
 
    }
   }   
}
\end{footnotesize}
\end{center}

\subsection{MS-PL-1: Using the software only for yourself}
\label{OSUC-01-MS-PL} 
\label{OSUC-03-MS-PL} 
\label{OSUC-06-MS-PL}
\label{OSUC-09-MS-PL}
  
\begin{description}
  \item[means] that you are going to use a received MS-PL software only for
  yourself and that you do not hand it over to any 3rd party in any sense.
  \item[covers] OSUC-01, OSUC-03, OSUC-06, and OSUC-09\footnote{For details see
  pp.\ \pageref{OSUC-01-DEF} - \pageref{OSUC-09-DEF}}
  \item[requires] the following tasks in order to fulfill the conditions of the MS-PL
  license:
  \begin{itemize}
    \item You are allowed to use any kind of MS-PL licensed software in any
    sense and in any context without any other obligations if you do not
    handover the software to 3rd parties.
  \end{itemize}
\item[prohibits] nothing explicitly.
\end{description}


\subsection{MS-PL-2: Passing the unmodified software}
\label{OSUC-02-MS-PL} \label{OSUC-05-MS-PL} \label{OSUC-07-MS-PL} 

\begin{description}
\item[means] that you are going to distribute an unmodified version of the
received MS-PL software to 3rd parties -- regardless whether you distribute it
in the form of binaries or as source code files\footnote{In this case it also
doesn't matter whether you distribute a program, an application, a server, a
snippet, a module, a library, or a plugin as an independent package}

\item[covers] OSUC-02, OSUC-05, OSUC-07\footnote{For details $\rightarrow$ OSLiC, pp.\
\pageref{OSUC-02-DEF} - \pageref{OSUC-07-DEF}}

\item[requires] the following tasks in order to fulfill the license conditions:
\begin{itemize}
  \item \textbf{[mandatory:]} Ensure that all licensing elements -- esp.\ all
  copyright, patent, trademark, and attribution notices that are present in the
  version you received -- are completely retained in your package.
  
  \item \textbf{[mandatory:]} Incorporate a complete copy of the MS-PL license
  into your package, regardless whether you distribute a source code or a binary
  package\footnote{$\rightarrow$ OSLiC, p.\ \pageref{MsplSourceBinHint}}.
  
  \item \textbf{[voluntary:]} It's a good tradition to let the documentation of
  your distribution and/or your additional material also contain a link to the
  original software (project) and its homepage.
\end{itemize}

\item[prohibits] to use any contributors' name, logo, or trademarks (without an
additional or general legally based approval).

\end{description}

\subsection{MS-PL-3: Passing a modified program as source code}
\label{OSUC-04-MS-PL}

\begin{description}

\item[means] that you are going to distribute a modified version of the received
MS-PL licensed program, application, or server (proapse) to 3rd parties as a
source code package.

\item[covers] OSUC-04\footnote{For details $\rightarrow$ OSLiC, pp.\ \pageref{OSUC-04-DEF}}

\item[requires] the following tasks in order to fulfill the license conditions:
\begin{itemize}
  \item \textbf{[mandatory:]} Ensure that all licensing elements -- esp.\ all
  copyright, patent, trademark, and attribution notices that are present in the
  version you received -- are completely retained in your package.
 
  \item \textbf{[mandatory:]} Incorporate a complete copy of the MS-PL license
  into your package.
  
  \item \textbf{[mandatory:]} If you do not want to publish your modifications
  under the MS-PL too, then cleanly separate your own sources and licensing
  documents from original elements of the adopted work.
  
  \item \textbf{[voluntary:]} Mark your modifications in the sourcecode.
  
  \item \textbf{[voluntary:]} It's a good tradition to let the documentation of
  your distribution and/or your additional material also contain a link to the
  original software (project) and its homepage (as far as this does not clashes
  with the prohibitions stated below).
  
  \item \textbf{[voluntary:]} You are allowed to expand an existing copyright
  notice of the program to mention your own contributions.
  
  \item \textbf{[voluntary:]} It is a good practice of the open source
  community, to let the copyright notice which is shown by the running program
  also state that the program is licensed under the MS-PL license (as far as
  this does not clashes with the prohibitions stated below). Because you are
  already modifying the program, you can also add such a hint, if the 
  original copyright notice lacks such a statement.
    
\end{itemize}

\item[prohibits] to use any contributors' name, logo, or trademarks (without an
additional or general legally based approval).

\end{description}


\subsection{MS-PL-3: Passing a modified program as binary}

\begin{description}

\item[means] that you are going to distribute a modified version of the received
MS-PL licensed program, application, or server (proapse) to 3rd parties as a
package of binaries.

\item[covers] OSUC-04\footnote{For details $\rightarrow$ OSLiC, pp.\ \pageref{OSUC-04-DEF}}

\item[requires] the following tasks in order to fulfill the license conditions:
\begin{itemize}
  
  \item \textbf{[voluntary:]} Mark your modifications in the source code even if
  you do not intend to distribute it.
  
  \item \textbf{[voluntary:]} It's a good tradition to let the documentation of
  your distribution and/or your additional material also contain a link to the
  original software (project) and its homepage (as far as this does not clashes
  with with the prohibitions stated below).
  
  \item \textbf{[voluntary:]} It is a good practice of the open source
  community, to let the copyright notice which is shown by the running program
  also state that the derivative work is based on a version originally licensed
  under the MS-PL license (as far as this does not clashes with the prohibitions
  stated below) -- perhaps by linking to the project homepage of the original.
  Because you are already modifying the program, you can also add such a hint,
  if the original copyright notice lacks such a statement.
    
\end{itemize}

\item[prohibits] to use any contributors' name, logo, or trademarks (without an
additional or general legally based approval)

\end{description}



\subsection{MS-PL-4: Passing a modified library independently as source code}
\label{OSUC-08-MS-PL}
\begin{description}
\item[means] that you are going to distribute a modified version of the received
MS-PL code snippet, module, library, or plugin (snimoli) to 3rd parties without
embedding it into another larger software unit.
\item[covers] OSUC-08\footnote{For details $\rightarrow$ OSLiC, pp.\ \pageref{OSUC-08-DEF}}
\item[requires] the following tasks in order to fulfill the license conditions:

\begin{itemize}
  \item \textbf{[mandatory:]} Ensure that all licensing elements -- esp.\ all
  copyright, patent, trademark, and attribution notices that are present in the
  version you received -- are completely retained in your package.
 
  \item \textbf{[mandatory:]} Incorporate a complete copy of the MS-PL license
  into your package.
  
  \item \textbf{[mandatory:]} If you do not want to publish your modifications
  under the MS-PL too, then cleanly separate your own sources and licensing
  documents from original elements of the adopted part(s).
  
  \item \textbf{[voluntary:]} Mark your modifications in the sourcecode.
  
  \item \textbf{[voluntary:]} It's a good tradition to let the documentation of
  your distribution and/or your additional material also contain a link to the
  original software (project) and its homepage (as far as this does not clashes
  with with the prohibitions stated below).
  
\end{itemize}

\item[prohibits] to use any contributors' name, logo, or trademarks (without an
additional or general legally based approval)

\end{description}

\subsection{MS-PL-4: Passing a modified library independently as binary}
\label{OSUC-08-MS-PL}
\begin{description}
\item[means] that you are going to distribute a modified version of the received
MS-PL code snippet, module, library, or plugin (snimoli) to 3rd parties without
embedding it into another larger software unit.
\item[covers] OSUC-08\footnote{For details $\rightarrow$ OSLiC, pp.\ \pageref{OSUC-08-DEF}}
\item[requires] the following tasks in order to fulfill the license conditions:

\begin{itemize}
  
  \item \textbf{[voluntary:]} Mark your modifications in the source code even if
  do not want to distribute it.
  
  \item \textbf{[voluntary:]} It's a good tradition to let the documentation of
  your distribution and/or your additional material also contain a link to the
  original software (project) and its homepage (as far as this does not clashes
  with with the prohibitions stated below).
    
\end{itemize}

\item[prohibits] to use any contributors' name, logo, or trademarks (without an
additional or general legally based approval)

\end{description}


\subsection{MS-PL-5: Passing a modified library as embedded source code}
\label{OSUC-10-MS-PL}
\begin{description}
\item[means] that you are going to distribute a modified version of the received
MS-PL licensed code snippet, module, library, or plugin (snimoli) to 3rd parties
in the form of source code files or as a source code package together with
another larger software unit which contains this code snippet, module, library,
or plugin as an embedded component.

\item[covers] OSUC-10\footnote{For details $\rightarrow$ OSLiC, pp.\ \pageref{OSUC-10-DEF}}
\item[requires] the following tasks in order to fulfill the license conditions:
\begin{itemize}
 
 \item \textbf{[mandatory:]} Ensure that all licensing elements -- esp.\ all
  copyright, patent, trademark, and attribution notices that are present in the
  version you received -- are completely retained in your package.
 
  \item \textbf{[mandatory:]} Incorporate a complete copy of the MS-PL license
  into your package.
  
  \item \textbf{[mandatory:]} If you do not want to publish your modifications
  and/or your overarching application under the MS-PL too, then cleanly separate
  your own sources and licensing documents from original elements of the adopted
  work.
  
  \item \textbf{[voluntary:]} Mark your modifications in the sourcecode.
  
  \item \textbf{[voluntary:]} It's a good tradition to let the documentation of
  your distribution and/or your additional material also contain a link to the
  original software (project) and its homepage (as far as this does not clashes
  with with the prohibitions stated below).

  \item \textbf{[voluntary:]} It is a good practice of the open source
  community, to let the copyright notice shown by your overarching program also
  state that it is based on a component originally licensed under the MS-PL
  license -- perhaps by linking the project homepage of the original (as far as
  this does not clashes with the prohibitions stated below).
  
\end{itemize}

\item[prohibits] to use any contributors' name, logo, or trademarks (without an
additional or general legally based approval)

\end{description}

\subsection{MS-PL-5: Passing a modified library as embedded binary}
\label{OSUC-10-MS-PL}
\begin{description}
\item[means] that you are going to distribute a modified version of the received
MS-PL licensed code snippet, module, library, or plugin (snimoli) to 3rd parties
in the form of a binary package together with another larger software unit which
contains this code snippet, module, library, or plugin as an embedded component.

\item[covers] OSUC-10\footnote{For details $\rightarrow$ OSLiC, pp.\ \pageref{OSUC-10-DEF}}
\item[requires] the following tasks in order to fulfill the license conditions:
\begin{itemize}
 
  \item \textbf{[voluntary:]} Mark your modifications in the source code even if
  do not want to distribute it.
  
  \item \textbf{[voluntary:]} It's a good tradition to let the documentation of
  your distribution and/or your additional material also contain a link to the
  original software (project) and its homepage (as far as this does not clashes
  with with the prohibitions stated below).
  
  \item \textbf{[voluntary:]} It is a good practice of the open source
  community, to let the copyright notice shown by your own overarching program
  also state that it is based on a component originally licensed under the MS-PL
  license -- perhaps by linking the project homepage of the original (as far as
  this does not clashes with the prohibitions stated below).
  
\end{itemize}

\item[prohibits] to use any contributors' name, logo, or trademarks (without an
additional or general legally based approval)

\end{description}

\subsection{Discussions and Explanations}

The MS-PL is also a very permissive and short license. It requires to do:
(a) You must preserve existing licensing elements. (b) You must distribute
the source code as whole or \enquote{portions} of the source code under the
MS-PL. (c) You must add a copy of the license if you distribute (parts of) the
source code. (d) If you distribute a binary package, you must distribute (the
parts of) the work under a license \enquote{that complies with this (MS-PL)
license}\footcite[cf.][\nopage wp]{MsplOsiLicense2013a}.

The most confusing clause is probably the condition, to \enquote{[\ldots]
distribute any portion of the software in compiled or object code form [\ldots]
only [\ldots] under a license that complies with this license}. But a closer
examination is lighting the situation: The only other conditions of the license
which refer to the context of distributing binaries are the requirements a) not
to abuse trademarks, b) not to bring a patent claim against any contributor, and
c) not to expect any warranties or guarantees with respect to the distributed
portion\footcite[cf.][\nopage wp\ §3A, §3B, §3E]{MsplOsiLicense2013a}.

Based on these readings we decided \ldots

\label{MsplSourceBinHint} 
\begin{itemize}
  \item \ldots to let you incorporate a copy of the license into your
  distribution even if it only contains the binaries of the unmodified version:
  if you have not modified it, you do not lose any advantage if you add the
  license, too. So, this is the best method to fulfill the \emph{MSL-PL binary
  condition}.
  \item \ldots to erase all mandatory conditions in case of the binary
  distributions: the patent restriction of the MS-PL itself is already covered
  by the MS-PL patent section of the OSLiC\footnote{$\rightarrow$ OSLiC, p.\
  \pageref{subsec:MsplPatentClause}} and the no warranty clause of the MS-PL by
  the OSLiC section concerning the power of the MS-PL\footnote{$\rightarrow$
  OSLiC, p.\ \pageref{sec:ProtectingPowerOfMspl}} while the trademark
  restrictions are explicitly added into the prohibition section.
  \item \ldots to erase the hints to a voluntarily updated copyright dialog in
  case of distributing a snimoli independently because the copyright dialog
  normally is designed by the overarching work which uses the library, not by
  the library itself.
\end{itemize}


%\bibliography{../../../bibfiles/oscResourcesEn}
