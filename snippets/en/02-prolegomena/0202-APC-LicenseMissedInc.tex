% Telekom osCompendium 'for beeing included' snippet
%
% (c) Karsten Reincke, Deutsche Telekom AG, Darmstadt 2011
%
% This LaTeX-File is licensed under the Creative Commons Attribution-ShareAlike
% 3.0 Germany License (http://creativecommons.org/licenses/by-sa/3.0/de/): Feel
% free 'to share (to copy, distribute and transmit)' or 'to remix (to adapt)'
% it, if you '... distribute the resulting work under the same or similar
% license to this one' and if you respect how 'you must attribute the work in
% the manner specified by the author ...':
%
% In an internet based reuse please link the reused parts to www.telekom.com and
% mention the original authors and Deutsche Telekom AG in a suitable manner. In
% a paper-like reuse please insert a short hint to www.telekom.com and to the
% original authors and Deutsche Telekom AG into your preface. For normal
% quotations please use the scientific standard to cite.
%
% [ File structure derived from 'mind your Scholar Research Framework' 
%   mycsrf (c) K. Reincke CC BY 3.0  http://mycsrf.fodina.de/ ]

%
Without Open Source Licenses there is no Open Source movement. Nevertheless in
dealing with Open Source Licenses, this is sometimes neglected. Take the
\emph{Apache Web Server} as an example: No doubt, it's one of the most important
pieces of Open Source Software\endnote{To prove that the \textit{Apache} is
really a piece of Open Source Software one must execute a set of steps: Firstly
you have to note, that \emph{Apache} is something like a meta project, covered
by the \emph{Apache Software Foundation}, also known as \emph{ASF} (cf.
\texttt{http://www.apache.org/}, wp.). Therefor you can not directly jump into
the \emph{Apache License}. First of all you have to visit the project site (cf.
\texttt{http://httpd.apache.org/}, wp.) even if at the end its' license link
leads you back to the general \emph{Apache License subsite} (cf.
\texttt{http://www.apache.org/licenses/}, wp.) which announces, that \glqq{}all
software produced by The Apache Software Foundation or any of its projects or
subjects is licensed according to the terms of the documents listed
below\grqq{}. Only now you can use the offered link for switching to the
\emph{Apache License}, Version 2.0, if you want to check your rights and duties.
But that is difficult. There does not exist any simple list what you have to do
for fulfilling the license. Even the faq (cf.
\texttt{http://httpd.apache.org/docs/2.2/faq/}, wp.) - meanwhile being moved to
a wiki - only says that the server \glqq{}[\ldots] comes with an unrestrictive
license\grqq{} and that you are allowed to put the code on a CD (cf.
\texttt{http://wiki.apache.org/httpd/FAQ}, wp.). Hence, from the viewpoint of
the ASF the license itself shall answer all questions. [Reference download for
all urls: 2011-08-31] } with a specific license\footcite[cf.][\nopage
wp.][]{AsfApacheLicense20a}. Moreover: the success of the Open Source movement
in the commercial world depends directly on the decision of IBM to replace its
corresponding own component in the \textit{IBM WebSphere Application Server}
with the free \textit{Apache Web Server}\footcite[cf.][287ff]{Moody2001a}.
Meanwhile many companies use the \textit{Apache Web Server} to act as a web
provider. Currently the \emph{Apache http server} - as it has to be named
correctly - is used more than twice as much as all the other http server
software together\footcite[cf.][\nopage wp]{Netcraft2011a}. Hence many business
models depends on the Apache License. Another aspect is that even the famous
\emph{Apache Cookbook}, which explains the installation, the configuration and
the maintaining of an Apache Web Server detailedly\footcite[cf.][\nopage et
passim]{CoaBow2004a}, does not mention anything about the license which allows
for installation, configuration and maintenance. Neither the index lists the
word 'license'\footcite[cf.][245ff, esp. p. 250]{CoaBow2004a}, nor the chapters
'Installation'\footcite[cf.][1ff]{CoaBow2004a} or the chapter
'Miscellaneous'\footcite[cf.][219ff]{CoaBow2004a} mentions the license question
in a serious way. There's only one short hint as to the advantage of Open Source
Software, i.e. that everybody is allowed to install it\footcite[cf.][1: \glqq{}
\ldots einer der Vorzüge von Open Source Software besteht darin, dass
je\-der\-mann die Erlaubnis zur Erzeugung eines eigenen Installationskits hat
\grqq{}]{CoaBow2004a}. Can you be sure that you are allowed to do what you are
doing on the base of such a phrase?

Naturally, the \emph{Apache Cookbook} is not a book for lawyers, it's a book for
administrators and developers, They do not want to get bogged down by
legalities, they want to set up an Apache Web Server as fast as possible and get
down to work. Indeed, the Apache Cookbook offers a good support. But not only as
a company you have to ask yourself whether you are really allowed to do what you
are doing. Can you find the answer in the \emph{Apache Cookbook}? No. Can you
find it in the license itself? Yes, but it is difficult\endnote{And do we really
want our developers and maintainers to read the original licenses? Do we really
want them to discover that they also have to check the licenses of the used
modules?}. So again: Can you find your answer in another book, which is
\emph{Amazon's} current top recommendation for the request \emph{'apache
server'}\endnote{Tested on \texttt{http://www.amazon.de/} at 2011-08-31.}? Not
really: Sascha Kersken's Apache 2.2 Handbook offers a license chapter, but only
two pages long\footcite[cf.][111f]{Kersken2009a}. Moreover the rights and duties
are condensed into just 5 bullet points which taken together do not explain when
the software and the license has to be handed over to a customer and when you
are allowed to hide your improvements\footcite[cf.][112]{Kersken2009a}.

This brings us to the question of what prevents us from using something like a
\emph{'general license cookbook'} which explains all the necessary details and which
offers  quick access to the relevant points:

%\bibliography{../bibfiles/oscResourcesEn}
