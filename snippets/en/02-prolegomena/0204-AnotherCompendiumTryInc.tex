% Telekom osCompendium 'for beeing included' snippet template
%
% (c) Karsten Reincke, Deutsche Telekom AG, Darmstadt 2011
%
% This LaTeX-File is licensed under the Creative Commons Attribution-ShareAlike
% 3.0 Germany License (http://creativecommons.org/licenses/by-sa/3.0/de/): Feel
% free 'to share (to copy, distribute and transmit)' or 'to remix (to adapt)'
% it, if you '... distribute the resulting work under the same or similar
% license to this one' and if you respect how 'you must attribute the work in
% the manner specified by the author ...':
%
% In an internet based reuse please link the reused parts to www.telekom.com and
% mention the original authors and Deutsche Telekom AG in a suitable manner. In
% a paper-like reuse please insert a short hint to www.telekom.com and to the
% original authors and Deutsche Telekom AG into your preface. For normal
% quotations please use the scientific standard to cite.
%
% [ File structure derived from 'mind your Scholar Research Framework' 
%   mycsrf (c) K. Reincke CC BY 3.0  http://mycsrf.fodina.de/ ]

%
While we were searching for an existing Open Source compendium we found an
article with the title 'Compendium for the Publication of Open Source
Software'\endnote{approximately translated}. It aims to be a 'pragmatic
guidebook' and an 'assistance' for 'publishing software under the conditions of
an Open Source License'\footcite[cf.][166f (originally: ein
\glqq{}pragmatischer Ratgeber\grqq{} zur \glqq{}Veröffentlichung einer Software
unter den Rahmenbedingungen einer Open-Source-Lizenz\grqq{}) ]{BreGlaGra2008a}.
Moreover, at the end of this article its' authors formulate ambitiously that
their 'guide' should be carried out, section by section - for getting a legally
water tight process of publishing Open Source software\footcite[cf.][186
(originally: ein \glqq{}Ratgeber\grqq{}, der es erlaubt \glqq{} (\ldots) die zu
berücksichtigende Aspekte (strukturiert abzuarbeiten) (\ldots) \glqq{} und einen
\glqq{}rechtlich nicht angreifbaren Veröffentlichungsprozess\grqq{} zu
ermöglichen) ]{BreGlaGra2008a}.

The authors of this article describe something close to what we were looking
for. Indeed, the article lists important aspects which have to be taken in
consideration if you want to deal Open Source Software correctly: It announces
that no obligation exists to publish code either if you embed GPL code into your
proprietary code or if you modify the GPL code. It is only if you hand over your
binary to other persons that you have to distribute the code too, but only to
them and not to the general public\footcite[cf.][170 and
181]{BreGlaGra2008a}. Additionally the articles explains exactly that software
- at least in Germany - can only be acknowledged as Open Source Software by
transferring the rights to use - the 'Nutzungsrechte' - to other people, while
the copyright itself - the 'Urheberpersönlichkeitsrecht' - is not transferable
and belongs to the author\footcite[cf.][173]{BreGlaGra2008a}. Moreover,
besides other aspects the articles discusses briefly and deeply the problem of
the No-Warranty-Clauses which are not valid in Germany and which will therefore
automatically be replaced by the liability rules for a
donation\footcite[cf.][177]{BreGlaGra2008a}. And last but not least this
article actually summarizes the idea of Copyleft and the differences between
LGPL and GPL\footcite[cf.][181]{BreGlaGra2008a}.

However some gaps remain. The article does not analyze in which cases a
University or a company perhaps \emph{must} publish its' developments based
upon Open Source Software. It does not discern between different licenses
and conditions. It also does not discuss what Universities or companies,
which (re-)use and/or distribute Open Source Software (internally), must do to
fulfill the touched Open Source Licenses. And finally this article
does not offer the step by step list as promised.

We did, however, feel supported by this article, in two ways. Firstly it was a
well written summary of some main problems. Secondly it stated the necessity to
have a compendium for being able to establish a legally 'water-tight' process of
publishing Open Source software\footcite[cf.][186]{BreGlaGra2008a}. We
seemed to be justified in our assumptions. But the Open Source Compendium we
were looking for had to be more practical, more processable, more distinguishing
and more elaborated.

%\bibliography{../bibfiles/oscResourcesEn}
