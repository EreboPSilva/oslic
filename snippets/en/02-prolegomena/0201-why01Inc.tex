% Telekom osCompendium 'for beeing included' snippet template
%
% (c) Karsten Reincke, Deutsche Telekom AG, Darmstadt 2011
%
% This LaTeX-File is licensed under the Creative Commons Attribution-ShareAlike
% 3.0 Germany License (http://creativecommons.org/licenses/by-sa/3.0/de/): Feel
% free 'to share (to copy, distribute and transmit)' or 'to remix (to adapt)'
% it, if you '... distribute the resulting work under the same or similar
% license to this one' and if you respect how 'you must attribute the work in
% the manner specified by the author ...':
%
% In an internet based reuse please link the reused parts to www.telekom.com and
% mention the original authors and Deutsche Telekom AG in a suitable manner. In
% a paper-like reuse please insert a short hint to www.telekom.com and to the
% original authors and Deutsche Telekom AG into your preface. For normal
% quotations please use the scientific standard to cite.
%
% [ Framework derived from 'mind your Scholar Research Framework' 
%   mycsrf (c) K. Reincke 2012 CC BY 3.0  http://mycsrf.fodina.de/ ]
%


%% use all entries of the bibliography
%\nocite{*}

\section{Why}

Do we need another book about Open Source? Do \emph{you} need another book about
Open Source Software? Let us address this question from the viewpoint of what we
already know, what we instinctively believe and what we may have heard. For
example you may presume one or more of the following statements is correct. Or
you may even have experienced similar perceptions from your peers or managers.
Or you have been told they describe 'Open Source':

\begin{itemize}
  \item The Open Source Definition offers rules to use Open Source Software.
  \item Modified Open Source Software must be published.
  \item Modified Open Source Software must be given back to the community.
  \item All generations of Open Source Software will remain open for ever.
  \item Software can either be Open Source Software or proprietary software.
  \item The opposite of Open Source Software is commercial software.
  \item Open Source Software prohibits to earn money.
  \item Modifications of Open Source Software must be marked explicitly.
  \item Modifiers of Open Source Software must identify themselves.
  \item When distributing an Open Source binary it’s enough point to a download
  page to obtain the source code.
  \item The aim of Open Source Software is to improve the world ethically.
  \item Open Source Software is viral and infectious.
\end{itemize}

Do these conceptions sound familiar to you? Unfortunately, whatever we might
believe or wish for, these concepts are incorrect. Naturally we will discuss
this issue later on. For the moment let us assume they are indeed
incorrect\footnote{For those who want directly verify our argumentation, we have
generated a condensed summary of the arguments and citations. You can find this
summary in our appendices.}.

So, again: Do \emph{we} need another book about Open Source Software? \emph{We},
that is - in this case and at least initially - the large German company
\textit{Deutsche Telekom AG}. Arguing from the perspective of a large company
requires not only identifying the common misconceptions, but catering for the
unique needs of a large Enterprise. And indeed the very size of the company
brings its own problems.

Large companies use more Open Source Software in more varied contexts than small
companies. There is an important question that every company should ask:
\emph{'Are we sure that we respect all those requirements of Open Source
Software we have to respect?'}. But large companies can not answer this question
as easily as small companies: the large number of diverse Open Source
deployments in different contexts mean that case by case governance, a model
that may work in small concerns, is far from appropriate for our needs. This
leads to wasting both time and money. Further, the chances of success are small:
training at least one employee in each software team as an Open Source Software
License expert is unrealistic in terms of cost-efficiency and reliability.

Nevertheless even large companies want to and try to fulfill the rules of Open
Source Software thoroughly - especially \emph{Deutsche Telekom AG}. When this
company realized that the question \textit{Are we sure that we respect all those
rules of Open Source Software correctly which we have to respect} could be
problematic, it directly asked some of its' employees who were known as Open
Source enthusiasts - to establish a service and a process for answering this
question.

So, it is no surprising that we, the initial authors of this \textit{Open Source
License Compendium}, were asked by our employer \emph{Deutsche Telekom AG}.
Naturally we were proud to work on an Open Source topic officially. But while we
were doing our job we had to ask ourselves if \emph{we} perhaps needed another
book on Open Source. Our answer was \textit{Yes, we do!} Let us shortly explain,
why:

Firstly, we already knew that there exist supporting software. These
meta-pro\-grams take the code of any other application and try to list those
Open Source Software being 'covered' by that application\footnote{As general
examples let us mention Palamida (\texttt{http://www.palamida.com/}) and
BlackDuck (\texttt{http://www.blackducksoftware.com/}).}. But we had also
already realised that this supporting software did not always match the way we
thought the problem should be solved. Secondly we recognized fairly quickly that
we need a reliable guide. We personally were asked to give the \emph{ok} for
projects of our company. We could not answer such requests on the base of
\textit{'Oh yes, I read this in the \emph{Heise-Ticker} a few days ago'} - even
if the \emph{Heise-Ticker} had described the situation completely correctly. We
ourselves had to be reliable than this. Naturally we already knew a great deal
about Open Source Software. Even so, our knowledge was not as systematic as
necessary. We looked for an Open Source Compendium which adequately described
what a project or product development team had to do to fulfill the criteria of
its Open Source Licenses. We wanted to use that compendium to the basis of our
recommendations.

We were very thorough but we did not find what we were looking for. Our 'little'
bibliography attest our seriousness. What we found was a lot of information
releating to individual issues spread over many sources. We did not find answers
for our question even in the specific literature. Let us describe three little
steps to increase the understanding of the issue:


%\bibliography{../../../bibfiles/oscResourcesEn}
