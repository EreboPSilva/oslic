% Telekom osCompendium 'for beeing included' snippet template
%
% (c) Karsten Reincke, Deutsche Telekom AG, Darmstadt 2011
%
% This LaTeX-File is licensed under the Creative Commons Attribution-ShareAlike
% 3.0 Germany License (http://creativecommons.org/licenses/by-sa/3.0/de/): Feel
% free 'to share (to copy, distribute and transmit)' or 'to remix (to adapt)'
% it, if you '... distribute the resulting work under the same or similar
% license to this one' and if you respect how 'you must attribute the work in
% the manner specified by the author ...':
%
% In an internet based reuse please link the reused parts to www.telekom.com and
% mention the original authors and Deutsche Telekom AG in a suitable manner. In
% a paper-like reuse please insert a short hint to www.telekom.com and to the
% original authors and Deutsche Telekom AG into your preface. For normal
% quotations please use the scientific standard to cite.
%
% [ Framework derived from 'mind your Scholar Research Framework' 
%   mycsrf (c) K. Reincke 2012 CC BY 3.0  http://mycsrf.fodina.de/ ]
%


%% use all entries of the bibliography
%\nocite{*}

\section{What}

Now we can briefly explain how you can use this compendium:

\begin{description}
  \item[Open Source: Idea and Concepts] :- Here you will find background
  information to help you interpret Open Source Licenses in the sense of the
  \emph{Free Software movement}\footcite[At least at this place you are perhaps
  expecting that we use the logograms FLOSS, F/OSS, F/LOSS, or whatever. As you
  will read later on the word \textit{Free} is ambiguous and has strained the
  use of the concept \textit{Free Software}. Later on we will also talk about
  the invention of the concept \textit{Open Source} designed as a 'replacement'
  and acting as a 'splitter'. The mentioned logograms are introduced to
  re-establish or - at least - to underline the common history and the common
  center of 'both' movements, whereby the word \textit{Libre} shall resolve the
  ambiguity of the word \textit{Free}. For a first survey cf.] [\nopage
  wp.]{wpFloss2011a}, the \emph{Open Source Software movement}\footcite[For
  another brief and informative introduction cf.][231ff esp. p.
  232f.]{Fogel2006a} or the GNU-Project\footnote{ We ourselves will stay with the
  concept \textit{Open Source} because the OSD specifies the scope of our
  analysis. But we do it with a deep obeisance to Stallmann and the FSF - even
  if we know that this will not protect us from the thunderbolt of RMS.}. If
  you are familiar with the evolution of the Open Source Initiative, with the
  character of the Open Source Definition as a set of necessary but insufficient
  criteria and if you know about the history and meaning of free software as
  older and broader concept then you can ignore this chapter.
  \item[The Same Idea, Different Licenses] :- In this chapter we discuss
  different ways to cluster Open Source Licenses. Finally we present our own
  taxonomy based on the labels 'protecting the developer', 'protecting the
  licensed code' and 'protecting the on-top-developments'. If you are familiar
  with the methods of grouping different Open Source Licenses and particular
  if you know that you can not authorize your doings on the base of descriptions
  of such license groups than it's enough, in order to understand our line of
  thought, to briefly note our taxonomy and its wording.
  \item[The Problem of Derivated Works] :- This chapter is important. In the
  spirit of software developer we try to explain which kinds of programming
  evoke a derivated work and which not. Our to-do lists will refer to this
  analysis.
  \item[The Problem of Combining Different Licenses] :- You should
  not ignore this chapter. We will explain why and how combining software
  of different licenses is not as dangerous as it's often told. The results of
  this chapter influence the structure of our to-do lists.
  \item[Open Source Software and Money] :- Here we will shortly
  discuss ways in which money is no problem. If you already know that it is only
  prohibited to require payment for the act of licensing a piece of Open Source
  Software to second or third parties and if you already know that this is only
  forbidden by some licenses, and not by all, than you can postpone the reading
  of this chapter.
  \item[The Problem of Implicitly Freeing Patents] :- Here we
  will illuminate some aspects of software patents and how the are handled by
  some Open Source Licenses. You should know what licenses implicitly do with
  your patents. But it's not our intention to write a software patent
  compendium.
  \item[Open Source: Use Cases as Principle of Classification] :- This is an
  important chapter. We explain our categories 'Use as it is', 'Modify the
  Code', 'With Redistribution', 'Without Redistribution', 'Isolated Initial
  Development', 'On-Top-Development': we develop and discuss our taxonomy with
  respect to the side effects of 'combining different licenses' and 'generating
  derivated works'. This taxonomy will determine the following chapters.
  \item[Open Source Licenses: Find Your Specific To-do Lists] :- This is a kind
  of summary which joins the relevant aspects and elaborates the 'finder
  for your to-do lists'. This is that chapter which you probably will reuse
  multiply, even if you do not want to read any of our explanations.
  \item[Open Source License Fulfillment: Classified To-do Lists] :- This chapter
  offers all classified to-do lists. The structure of its' subchapters will
  match the structure of our finder and the structure of our taxonomy.
  \item[Open Source Licenses and Their Legal Environments] :- Here we discuss
  that using Open Source Software in a regular manner is not only a question of
  the licenses themselves but of the kind of the surrounding legal system.
  \item[Appendices: Some Widespread Open Source Myths] :- Here we make good on
  our promise to explain why all the propositions mentioned at the beginning of
  this chapter are wrong. You might read this chapter as a special introduction
  or a reminder epilogue whenever you want to do.
\end{description}


%\bibliography{../../../bibfiles/oscResourcesEn}
