% Telekom osCompendium 'for being included' snippet template
%
% (c) Karsten Reincke, Deutsche Telekom AG, Darmstadt 2011
%
% This LaTeX-File is licensed under the Creative Commons Attribution-ShareAlike
% 3.0 Germany License (http://creativecommons.org/licenses/by-sa/3.0/de/): Feel
% free 'to share (to copy, distribute and transmit)' or 'to remix (to adapt)'
% it, if you '... distribute the resulting work under the same or similar
% license to this one' and if you respect how 'you must attribute the work in
% the manner specified by the author ...':
%
% In an internet based reuse please link the reused parts to www.telekom.com and
% mention the original authors and Deutsche Telekom AG in a suitable manner. In
% a paper-like reuse please insert a short hint to www.telekom.com and to the
% original authors and Deutsche Telekom AG into your preface. For normal
% quotations please use the scientific standard to cite.
%
% [ Framework derived from 'mind your Scholar Research Framework' 
%   mycsrf (c) K. Reincke 2012 CC BY 3.0  http://mycsrf.fodina.de/ ]
%


%% use all entries of the bibliography
%\nocite{*}

\section{Excursion: Why linking is a secondary criteria}
\label{sec:LinkingSecondary}
\footnotesize
\begin{quote}\itshape
Distributing statically or dynamically linked software is often discussed as a
problem (and sometimes as a solution) for acting compliantly. In this chapter,
we briefly discuss why this aspect can mostly be ignored and why it does not
help to determine the existence of a derivative work.
\end{quote}
\normalsize

In some earlier versions of the OSLiC, its finder subclassified some use cases
with respect to the way an application was 'composed' as a larger unit: In the
previous form for gathering the necessary information, the OSLiC user had to
answer whether he \emph{was going to combine the received open source software
with other software components by linking all together statically, by linking
them dynamically, or by textually including (parts of) the open source software
into [his] larger unit}. Today, this question has totally been erased. The
authors could convince themselves that it is not necessary to consider this
aspect.

Of course, we know that being linked statically or dynamically is often and
deeply discussed by license experts\footnote{Even on the \emph{European Legal
and Licensing Workshop, 2013} in Amsterdam, there was given an excellent lecture
concerning the nature and concequences of linking elf files.}. It seems to be an
important aspect:

[TBD: Discussion of the literature]
%TODO Discus statically dynamicall discussion.

So, let us start with some undeniable facts: The OSLiC deals with the Apache-2.0
license\footcite [cf.][\nopage wp]{Apl20OsiLicense2004a}, the BSD-2-Clause
license\footcite [cf.][\nopage wp]{BsdLicense2Clause}, the BSD-3-Clause
license\footcite [cf.][\nopage wp]{BsdLicense3Clause}, the MIT license\footcite
[cf.][\nopage wp]{MitLicense2012a}, the MS-PL\footcite[cf.][\nopage
wp]{MsplOsiLicense2013a}, the PgL\footcite[cf.][\nopage wp]{PglOsiLicense2013a}
and the PHP license\footcite[cf.][\nopage wp]{Php30OsiLicense2013a} as instances
of the permissive licenses.
Additionally, the OSLiC treats the EPL\footcite[cf.][\nopage
wp]{Epl10OsiLicense2005a}, the EUPL\footcite[cf.][\nopage
wp]{Eupl11OsiLicense2007a}, the LGPL\footnote{For LGPL-2.1 see \cite
[cf.][\nopage wp]{Lgpl21OsiLicense1999a}. For LGPL-3.0 see \cite [cf.][\nopage
wp]{Lgpl30OsiLicense2007a} }, and the MPL\footcite [cf.][\nopage
wp]{Mpl20OsiLicense2013a} as licenses with weak copyleft. Finally, the OSLiC
thoroughly discusses the GPL\footnote{For GPL-2.0 see \cite [cf.][\nopage
wp]{Gpl20OsiLicense1991a}. For GPL-3.0 see \cite [cf.][\nopage
wp]{Gpl30OsiLicense2007a} } and the AGPL\footcite[cf.][\nopage
wp]{Agpl30OsiLicense2007a} as licenses with strong copyleft\footnote{You can
find html based instances of these licenses in the OSLiC directory 'licenses'.
They have been downloaded from the OSI pages. All of the following statements
refer to these files.}.

Only three of these licenses mention the word \emph{linking} (or variants of
it): Using the command \texttt{grep -i link * | grep -v
"<link\textbackslash{}|links\textbackslash{}|skip-link"} in a shell -- executed
as an operation on a set of html formatted license files -- directly shows that
only the AGPL-3.0, the ApL-2.0, the GPL-2.0, the GPL-3.0, the LGPL-2.1 and the
LGPL-3.0 are using mutations of the word \emph{linking}. Additionally, the
results of the command \texttt{grep -i statical *} show that only the LGPL-2.1
uses the word 'statical', while using the command \texttt{grep -i dynamical *}
only hints to the AGPL-3.0 and the GPL-3.0. Finally, the command \texttt{grep -i
"shared" *} -- executed on the same set of files -- shows that the term
\emph{shared libary} is also only used by these licenses.

This analysis already indicates that being statically or dynamically linked
might not be as important for acting compliantly as it is often suggested.
% 
If one reads the concrete statements, then one can see, that acting compliantly
depends only slightly and only rarely on the kind of being 'combined':

\begin{description}

  \item[ApL-2.0:] This version of the Apache license uses the word \emph{link}
  only once for stating that \enquote{[\ldots] Derivative Works shall not
  include works that [\ldots] link [\ldots] to the interfaces of, the Work and
  Derivative Works thereof}\footcite [cf.][\nopage wp.\
  §0]{Apl20OsiLicense2004a}. Thus, the ApL does not use the criteria \emph{being
  linked} for determining a derivative work, neither \emph{being linked} in
  general, nor \emph{being statically linked}, nor being \emph{dynamically
  linked}. Hence, for acting in accordance to the ApL, this class of attributes
  can completely be ignored.

  \item[GPL-3.0:] The GPL-3.0 uses the word \emph{link} three times: First, it
  defines the \enquote{\enquote{Corresponding Source} for a work in object code
  form [\ldots as] all the source code needed to generate, install, and [\ldots]
  run the object code and to modify the work [\ldots]}. Additionally the GPL-3.0
  also explains in this context that this definition shall include
  \enquote{[\ldots] the source code for shared libraries and dynamically linked
  subprograms that the work is specifically designed to
  require}\footcite[cf.][\nopage wp.\ §0]{Gpl30OsiLicense2007a}. Second, the
  GPL-3.0 allows \enquote{[\ldots] to link or combine any covered work with a
  work licensed under version 3 of the GNU Affero General Public License into a
  single combined work, and to convey the resulting work}\footcite[cf.][\nopage
  wp.\ §13]{Gpl30OsiLicense2007a}. Finally, the GPL-3.0 explains that
  \enquote{the GNU General Public License [itself] does not permit incorporating
  your program into proprietary programs} and that the LGPL might be a better
  license for those licensors who have written a \enquote{subroutine library
  [\ldots] and may consider it more useful to permit linking proprietary
  applications with the library [\ldots]}\footcite[cf.][\nopage wp.\ last
  parapgraph]{Gpl30OsiLicense2007a}.
  
  So, also in this text, the features \emph{statically linked} or
  \emph{dynamically linked} are not used to trigger any license fulfilling
  actions. The conditions for \enquote{Conveying Modified [\ldots] Versions}
  refer to the \enquote{work based on the Program}\footcite[cf.][\nopage wp.\
  §5]{Gpl30OsiLicense2007a} which itself denotes a \enquote{\enquote{modified
  version} of the earlier work}\footcite[cf.][\nopage wp.\
  §0]{Gpl30OsiLicense2007a}. Moreover, the licensee -- as modifier, distributor,
  and subsequent licensor -- is required by the GPL-3.0 \enquote{[\ldots] to
  license the entire work [which has been developed on the base of a GPL-3.0
  component], as a whole, under this License to anyone who comes into possession
  of a copy}\footcite[cf.][\nopage wp.\ §5]{Gpl30OsiLicense2007a}. The GPL-3.0
  does not limit this claim -- especially not by referring to a mode of being
  linked. Hence, also with respect to the GPL-3.0, one can completely ignore
  these features of the software, its use and its distribution for determining
  how to use the software compliantly.

  \item[AGPL-3.0:] Concerning the use and the meaning of the words
  \emph{dynamically} and \emph{linking}, the AGPL-3.0 exactly follows the
  structure of the GPL-3.0: first the terms arise in the context of defining the
  \enquote{Corresponding Source}\footcite[cf.][\nopage wp.\
  §0]{Agpl30OsiLicense2007a}; then the word \emph{link} helps to say that AGPL
  and GPL are compatible licenses\footcite[cf.][\nopage wp.\
  §13]{Agpl30OsiLicense2007a}; and finally the word \emph{link} is used to hint
  to the LGPL\footcite[cf.][\nopage wp.\ §5]{Agpl30OsiLicense2007a}. So, again,
  one can ignore the feature of being statically or dynamically linked if one
  wants to determine how to use the software compliantly.

  \item[GPL-2.0:] In the GPL-2.0, the word \emph{link} only arise in the context
  of hinting to the LGPL\footcite [cf.][\nopage wp.\ last
  paragraph]{Gpl20OsiLicense1991a}. Moreover, the words \emph{statical} and
  \emph{dynamical} are not used in this text -- not at all and in no sense: the
  copy left feature of the GPL depends 'only' on a specification which refers to
  a \enquote{work based on the Program [\ldots] that in whole or in part
  contains or is derived from the Program or any part thereof [\ldots]}\footcite
  [cf.][\nopage wp.\ §2]{Gpl20OsiLicense1991a}. Thus, even in this old version
  of the GPL, the criteria of being linked -- in which way ever -- does not
  trigger any task for using the software compliantly.

  \item[LGPL-3.0:] In this license, variants of the word \emph{link} are used to
  define the concept of a \enquote{Combined Work} which shall be the name for a
  \enquote{[\ldots] work produced by combining or linking an Application with
  the Library}\footcite [cf.][\nopage wp.\ §0]{Lgpl30OsiLicense2007a}. In the
  end the LGPL-3.0 allows to \enquote{[\ldots] convey a Combined Work under
  terms of [his own] choice [\ldots]}, provided that one distributes also all
  material (including the object files of the overarching on-top developments)
  being necessary for enabling the receiver to relink the whole product with a
  later incoming newer version of the library or that one presupposes the use of
  a \enquote{suitable shared library mechanism} so that the receiver can update
  the library simply by replacing the binary library file\footcite[cf.][\nopage
  wp.\ §4]{Lgpl30OsiLicense2007a}. For fulfilling these conditions it is
  sufficient to require that a distributor shall \emph{either distribute the
  on-top development and the library in the form of dynamically linkable parts
  or distribute the statically linked application together with a written offer,
  valid for at least three years, to give the user all object-files of the
  on-top development and the library, so that he can relink the application on
  its own behalf}.

  \item[LGPL-2.1:] Even if the LGPL-2.1 is more sophistically arguing than all
  the other licenses, in its preamble this license clearly states what it wants
  to evoke: \enquote{If you link other code with the library, you must provide
  complete object files to the recipients, so that they can relink them with the
  library after making changes to the library and recompiling it
  [\ldots]}\footcite[cf.][\nopage wp.\ preamble]{Lgpl21OsiLicense1999a}. For
  that purpose, the LGPL-2.1 defines in the beginning that if \enquote{a program
  is linked with a library, whether statically or using a shared library, [then]
  the combination of the two is legally speaking a combined work, a derivative
  of the original library}\footcite[cf.][\nopage wp.\
  preamble]{Lgpl21OsiLicense1999a}: On the one hand a \enquote{work that uses
  the Libary} -- which is only \enquote{[\ldots] designed to work with the
  Library by being compiled or linked with it [\ldots]} -- \enquote{[\ldots] in
  isolation, is not a derivative work of the library [\ldots]}. On the other
  hand, it is no question for the LGPL-2.1, that \enquote{linking a
  \enquote{work that uses the Library} with the Library creates an executable
  that is a derivative of the Library (because it contains portions of the
  Library)}\footcite[cf.][\nopage wp.\ §5]{Lgpl21OsiLicense1999a}. But then --
  \enquote{as an exeption} -- the LGPL-2.1 allows to \enquote{[\ldots] combine
  or link a "work that uses the Library" with the Library to produce a work
  containing portions of the Library, and distribute that work under terms of
  your choice}. The right to do this is granted provided that the distributor
  either presupposes the use of a \enquote{suitable shared library mechanism} or
  that he distributes also the complete material (including the object files of
  the overarching on-top developments) which is necessary to enable the receiver
  to relink the whole product with a later incoming newer version of the
  library\footcite[cf.][\nopage wp.\ §6, §6b and §6c together with
  §6c]{Lgpl21OsiLicense1999a}. Again, for fulfilling all these conditions it is
  sufficient to require that a distributor shall \emph{either distribute the
  on-top development and the library in the form of dynamically linkable parts
  or distribute the statically linked application together with a written offer,
  valid for at least three years, to give the user all object-files of the
  on-top development and the library, so that he can relink the application on
  its own behalf}.

\end{description}

Thus, with respect to this analysis, we can conclude that -- in general -- there
is no need to gather more or less complicately whether one wants to distributed
software in the form of statically or dynamically linked binaries for deriving
the necessary tasks to distribute this software compliantly. Instead of this, we
can directly incorporate those doings into the task lists of the LGPL what has
been discovered as sufficient doings. Moreover, it is also sufficient to insert
this statement only in the task list of the LGPL. There is no need to generalize
this discussion. So, we could simplify our form offered to gather the
information to find the adequate license fulfilling task list.


%\bibliography{../../../bibfiles/oscResourcesEn}
