% Telekom osCompendium 'for being included' snippet template
%
% (c) Karsten Reincke, Deutsche Telekom AG, Darmstadt 2011
%
% This LaTeX-File is licensed under the Creative Commons Attribution-ShareAlike
% 3.0 Germany License (http://creativecommons.org/licenses/by-sa/3.0/de/): Feel
% free 'to share (to copy, distribute and transmit)' or 'to remix (to adapt)'
% it, if you '... distribute the resulting work under the same or similar
% license to this one' and if you respect how 'you must attribute the work in
% the manner specified by the author ...':
%
% In an internet based reuse please link the reused parts to www.telekom.com and
% mention the original authors and Deutsche Telekom AG in a suitable manner. In
% a paper-like reuse please insert a short hint to www.telekom.com and to the
% original authors and Deutsche Telekom AG into your preface. For normal
% quotations please use the scientific standard to cite.
%
% [ Framework derived from 'mind your Scholar Research Framework' 
%   mycsrf (c) K. Reincke 2012 CC BY 3.0  http://mycsrf.fodina.de/ ]
%


%% use all entries of the bibliography
%\nocite{*}

\section{Excursion: What is a 'Derivative Work' - the basic idea of open source}
\footnotesize \begin{quote}\itshape This chapter briefly discusses aspects of
being a derivated pieces of software which have to be known for using open
source software compliantly. As used in all its other parts, the OSLiC
only tries to find one safe interpretation. The authors know that there
exist many other ways to consider this topic. So, if you feel, that the
viewpoint of the OSLiC does not fit the specific circumstances of your
particular case, do not hesitate to ask your own lawyer. But if you agree with
the OSLiC, be aware that you dealing with this topic from the viewpoint of a
benevolent user.
\end{quote}
\normalsize
Let us start with the outline of an arguing structure:

\begin{description}
  \item[The meaning 'derivative work' must be known!] Many open source licenses
  are using the term 'derivative work'\footnote{cite the sources}, either
  directly or indirectly in form of the work 'modification'\footnote{cite the
  sources}[Write a table as survey]. And nearly all licenses, which are using
  the term 'derivative work' etc., are linking tasks which must be executed to
  comply with the corresponding license, to the precondition, that something is
  a derivative work [table survey]. \textbf{Hence, for acting in accordance to
  such a license, it has to be known, what a derivate work is}
  \item[Unfortunately the meaning is not clearly fixed]. The exist some
  different readings of the term 'derivative work' [specify the differences and
  cite the sources] \textbf{Hence, it is not as clear as wished what a derivative
  work is}
  \item[So, let us argue from the viewpoint of a benevolent developer]: Open
  source licenses are written for software developers, mostly to preserve their
  freedom, to develop software. And sometimes these licenses are also written by
  software developers -- or at least by the assistance of. So, one should be
  able to answer the question under which circumstances a piece of software is a
  'derivative work' of another piece of software on the base of two principles:
  \begin{itemize}
  \item Let us argue on the base of a benevolent neutral software developer
  without hidden interests or a hidden agenda.
  \item In case of doubts let us preferably assume that the two pieces
  interrelate as source and derivative work -- so that the OSLiC rather recommends
  to execute the required tasks than to put them away.
\end{itemize}
\end{description}

Basically we generalize a specific viewpoint of the LGPL: It uses three terms:

\begin{description}
  \item[\enquote{library}] is defined as \enquote{a collection of software
  functions and/or data prepared so as to be conveniently linked with
  application programs}\footcite[cf.][\nopage wp §0]{Lgpl21OsiLicense1999a}.
  \item[\enquote{work based on the library}] is defined as \enquote{either the
  library or any derivative work}\footcite[cf.][\nopage wp
  §0]{Lgpl21OsiLicense1999a}.
  \item[\enquote{work that uses the library}] is defined as something which
  initially \enquote{[\ldots] is not a derivative work of the library [\ldots]}
  but can become a derivative work by being combined / linked to the library it
  uses\footcite[cf.][\nopage wp §5]{Lgpl21OsiLicense1999a}.
\end{description}

Following these specifications, one has to conclude that there can be derived
\emph{derivative works} of the library in two different ways: First, the library
itself can be enhanced without changing the character of being a library. Then,
of course, the resulting library is a derivative work of the initial library.
Second, an overaching program can use the library by calling functions, methods
or data, offered by the library. In this case, the overarching program
functionally depends on the library and is a derivative work (as soon as it is
linked to the library).

This viewpoint can be generalized: also snippets, modules, plugins can be
enhanced and used by overarching programs or even by more complex libraries.
Based on this viewpoint - which should finally be formulated as the viewpoint of
a benevolent impartial developer - the OSLiC uses the following rules by which
the OSLiC decides to take something as derivative Work:
\label{sec:BenevolentDerivativeWorkUnderstanding}

\begin{description}
  \item[Copy-Case] Copying a piece of code from a source file and pasting it
  into a target file makes the target file a derivatve work of the source
  file\footnote{ Be careful: this case must still be distinguished from the case
  of an automatically inclusion (header files, script libraries) during the
  compilation / execution: Header files allon should not evekoe a derivative
  work.}.
  \item[Modify-Case] Inserting any new content or deleting any existing content
  of a source file makes the resulting target file being a derivate work of the
  source file.
  \item[Call-Case] Inserting into a target file the call of function which is
  defined inside of and delivered by a sourcefile makes the target file
  depending on the source file and therefore a derivative work of the delivering
  source file.
\end{description}

And here are some applications of these rules:

\begin{itemize}
  \item \textbf{Enlarging an existing source file by an external text evokes a
  derivative work!} Why? \emph{Because you are going to reuse the
  external code for simplifying our life.} [see 'Copy Case']
  \item \textbf{Reducing a source file evokes a derivative work!} Why?
  \emph{Because you are going to prepare the given file(s) for a better reuse.}
  [see 'Modify-Case']
  \item \textbf{Replacing something in a source file evokes a derivative work!}
  Why? \emph{Because you are going to reuse parts of the existing code for
  simplifying our life.} [see 'Modify Case']
  \item \textbf{Integrating a foreign snippet into an existing source code
  evokes a derivative work!} Why? \emph{Because you are going to simplify
  your life by reusing both, the foreign snippet and the original file.} [see
  'Copy Case' and 'Modify-Case']
  \item \textbf{Refactoring a given work by extracting a function / method into
  an autonmous file evokes a derivative work in two respects!} Why?
  \emph{Because first you are going to let depend all modified / generated
  files on the original file and  second because you are going to let
  depend those files with function calls on the function defining file itself.}
  [see 'Modify-Case' and 'Call-Case]
  \item \textbf{Calling a function - served by a defining module - let the
  calling file become a derivative work of the serving module!} Why?
  \emph{Because you are going to simplify your life by reusing an already
  prepared work (often offered by other developers).} [see 'Call-Case']
  \item \textbf{Calling elements - served by a defining library - let the
  calling file become a derivative work of the serving library!} Why? 
  \emph{Because you are going to simplify your life by reusing an already
  prepared work (often offered by other developers).} [see
  'Call-Case']\footnote{In this context, you may sometimes read that one has to
  differentiate the defining file (for exmaple the C-code) and the declaring
  file (for example the C-Header). But in our view, it is not so important to
  make such a difference: The using file, which includes the declaring header
  file depends on the defining source code file ('Call-Case'). So, one can
  ignore the formal dependance on the declaring header file ('Copy-Case').}
\end{itemize}

And now some additional 'ideas' which might invite to be discussed:

\begin{itemize}
  \item \textbf{Does a plugin depend on its framework? No.} Why? \emph{
  Because it is like a module: it offers a function (normally without using
  a function, offered by the framework itself).}
  \item \textbf{Does a framework depend on its plugin? Let us try to answer:
  Sometimes yes, sometimes no.} Why? \emph{If the framwork crashes when it is
  missing its plugin, then it clearly depends on the plugin. Not doubt. It is
  simply not autonomous. But if it does not crash, then it perfectly does for
  which it has been designed: it is offering a place which might be filled by
  the plugin, but not necessarily. This kind of a framework is like an
  application listing to a port for getting data which it shall process and
  which are served by another application.}
  \item \textbf{Does a program using inter process communication depend on its
  IO-partners? Definitely not!} Why? \emph{Because we otherwise need not discuss
  all these cases, every thing depends on everything -- in each running system.}
\end{itemize}

[\ldots TBD \ldots]

%\bibliography{../../../bibfiles/oscResourcesEn}
