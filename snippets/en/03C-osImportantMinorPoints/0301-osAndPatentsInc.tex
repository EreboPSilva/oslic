% Telekom osCompendium 'for being included' snippet template
%
% (c) Karsten Reincke, Deutsche Telekom AG, Darmstadt 2011
%
% This LaTeX-File is licensed under the Creative Commons Attribution-ShareAlike
% 3.0 Germany License (http://creativecommons.org/licenses/by-sa/3.0/de/): Feel
% free 'to share (to copy, distribute and transmit)' or 'to remix (to adapt)'
% it, if you '... distribute the resulting work under the same or similar
% license to this one' and if you respect how 'you must attribute the work in
% the manner specified by the author ...':
%
% In an internet based reuse please link the reused parts to www.telekom.com and
% mention the original authors and Deutsche Telekom AG in a suitable manner. In
% a paper-like reuse please insert a short hint to www.telekom.com and to the
% original authors and Deutsche Telekom AG into your preface. For normal
% quotations please use the scientific standard to cite.
%
% [ Framework derived from 'mind your Scholar Research Framework' 
%   mycsrf (c) K. Reincke 2012 CC BY 3.0  http://mycsrf.fodina.de/ ]
%

%% use all entries of the bibliography
%\nocite{*}

\section{The problem of implicitly releasing patents}
\footnotesize \begin{quote}\itshape In this chapter, we briefly analyze
the effects of patent clauses in open source licenses---not in general, but with
respect to the license fulfilling tasks they require, also known as the
`implicit acceptance of a patent use' by distributing open source software.
\end{quote}
\normalsize

At least the free software movement frowns on the existence of software
patents.%
  \footnote{For an early and elaborate description on the effects of
  software patents based on the viewpoint of the free software movement
  \cite[see][\nopage wp]{Stallman2001a}. This lecture seems to have been given
  more than once and printed later on (\cite[cf.][\nopage wp]{Stallman2002a}).
  Within the first decade of 2000, the focus switched to a more political fight
  against software patents (\cite[cf.][\nopage wp]{Stallman2004a}). But recently
  there seems to have appeared another turn in dealing with software patents:
  Not fighting against the patents, but mitigating their effects. The proposal is
  `[...] (to legislate) that developing, distributing, or running a program on
  generally used computing hardware does not constitute patent infringement'
  (\cite[cf.][\nopage wp]{Stallman2012a})}
One of the best known witnesses for that attitude is the GPL itself. Its
preamble purports that \enquote{[\ldots] any free program is threatened
constantly by software patents.}\citeGPLtwo{} One can read that the open
source community fears three risks: First, they are apprehensive of people who
hijack the idea of a piece of open source software they do not have developed,
register a corresponding patent, and finally try to earn money by preventing the
use of the software or by involving its users in patent
ligitations.\footcite[cf.][234]{JaeMet2011a} Second, they fear a bramble of
general software patents which practically prohibits them to develop open source
software legally.\footcite[cf.][234]{JaeMet2011a} Third, they anticipate the
possibility that (not quite benevolent) open source developers could try to
register patents with the intention of undermining the open source
principles.\footcite[cf.][235]{JaeMet2011a}

Howsoever, regardless whether one tries to fight against software patents or not,
software patents have become a reality. To abide by the law requires managing the
constraints of patents properly. Open source licenses know and respect this
necessity. Moreover, at least some of them try to manage the effect of software
patents by specific patent clauses\citeAPL[pars pro toto cf.]{§3} or by several
sentences distributed in the license text.\citeEPL[pars pro toto cf.]{wp} But why
does the \oslic{} have to deal with this topic, if the \oslic{} does not want to
participate in general discussions?

Opposite to the other conditions of the open source licenses, their patent
clauses or propositions in general do not directly refer to a specific set of
actions which have to be executed for acting in accordance with the licenses. Open
source patent clauses normally do not join in the game `paying by doing.' So,
actually, it does not seem to be necessary to mention the patent clauses here.

Unfortunately, although the patent clauses do not directly say \emph{`do this or
that in these or those circumstances,'} some of them nevertheless have side
effects which imply that the distributors of open source software already
have something done if they actually distribute a piece of open
source software. This implicit effect makes it necessary to deal with the patent
clauses even in an only pragmatic \oslic.

Patent clauses in open source licenses can have two different directions of
impact. They use two methods to protect the users of the open source software---%
and sometimes these methods are combined:

\begin{itemize}
  \item First, an open source license can assure that all contributors to and
  distributors of a piece of open source software grant to all users/%
  recipients not only the right to use the open source software itself, but
  automatically and implicitly also the right to use all those patents 
  belonging to the contributors/distributors which as patents are necessary
  to use the software legally.%
    \footnote{There might arise a legal discussion
    whether even a distributor who does not contribute to the software development
    has to grant the necessary rights of his patent
    portfolio. The \oslic{} does not want to participate in this discussion. We take a
    simple and pragmatic position: to be sure that you are acting according to
    an open source license with such a patent clause you should simply assume that
    you have to do so. If this default position is not reasonable for you it might
    be a good idea to consult legal experts which---perhaps---may find another
    way for you to use the software legally.} 
  So, let us---a little simplifying and therefore only on the following few
  pages---name such licenses the \emph{granting licenses}.
  \item Second, an open source license can try to automatically terminate the
  right to use, to modify, and to distribute the software if its user initiates
  litigations against any of the contributors/distributors with respect to an
  infringement of patent. That can be seen as a revocation of rights granted 
  earlier. So, let us name these license the \emph{revoking licenses.}
\end{itemize}

Later on, we will summarize the concrete patent clauses of all the licenses
discussed in the \oslic{} as a proof for the following classification:

\begin{small}

\begin{center}
\begin{tikzpicture}
\label{PATTAX}


\node[ellipse,minimum height=8.5cm,minimum width=14.2cm,draw,fill=gray!10] (l0100) at (6.7,6.8)
{  };

\draw [-,dotted,line width=0pt,white,
    decoration={text along path,
              text align={center},
              text={|\itshape|open source licenses}},
              postaction={decorate}] (-0.8,6.5) arc (218:322:9.5cm);
              
\node[ellipse,minimum height=6.2cm,minimum width=5cm,draw,fill=gray!20] (l0100)
at (2.5,6.8) {  };

\draw [-,dotted,line width=0pt,white,
    decoration={text along path,
              text align={center},
              text={|\itshape| without granting patent clauses}},
              postaction={decorate}] (0.75,7.5) arc (180:0:1.8cm);

\node[rectangle,draw,text width=1.2cm, text height=0.36cm, fill=gray!40, text
centered] (l0101) at (2.5,8.6) {\footnotesize \textit{MIT}};
\node[rectangle,draw,text width=1.2cm, text height=0.36cm, fill=gray!40, text
centered] (l0102) at (1.7,7.6) {\footnotesize \textit{BSD-X-Clause}};
\node[rectangle,draw,text width=1.2cm, text height=0.36cm, fill=gray!40, text
centered] (l0103) at (3.4,6.4) {\footnotesize \textit{LGPL-2.1}};
\node[rectangle,draw,text width=1.2cm, text height=0.36cm, fill=gray!40, text
centered] (l0104) at (3.4,7.6) {\footnotesize \textit{GPL-2.0}};
\node[rectangle,draw,text width=1.2cm, text height=0.36cm, fill=gray!40, text
centered] (l0105) at (1.7,6.4) {\footnotesize \textit{PHP-3.X}};
\node[rectangle,draw,text width=1.4cm, text height=0.36cm, fill=gray!40, text
centered] (l0106) at (2.5,5.2) {\footnotesize \textit{Post-greSQL}};

\node[ellipse,minimum height=6cm,minimum width=8.5cm,draw,fill=gray!20] (l0200)
at (9.4,6.5) {  };

\draw [-,dotted,line width=0pt,white,
    decoration={text along path,
              text align={center},
              text={|\itshape| with granting patent clauses}},
              postaction={decorate}] (2.2,2) arc (180:0:7cm);


\node[ellipse,minimum height=4.5cm,minimum width=5.6cm,draw,fill=gray!30]
(l0210) at (8.4,6.1) {  };

\draw [-,dotted,line width=0pt,white,
    decoration={text along path,
              text align={center},
              text={|\itshape| granting + revoking}},
              postaction={decorate}] (4.4,3.8) arc (180:0:4cm);

\node[rectangle,draw, text width=2cm, text height=0.34cm, fill=gray!40, text
centered] (l0212) at (7.2,6.9) {  \footnotesize \textit{Apache-2.0}};

\node[rectangle,draw, text width=2cm, text height=0.34cm, fill=gray!40, text
centered] (l0211) at (9.6,6.9) {  \footnotesize  \textit{EPL-1.X}};

\node[rectangle,draw, text width=2cm, text height=0.34cm, fill=gray!40, text
centered] (l0213) at (7.2,6.1) {  \footnotesize  \textit{MPL-X.Y}};

\node[rectangle,draw, text width=2cm, text height=0.34cm, fill=gray!40, text
centered] (l0214) at (9.6,6.1) {  \footnotesize  \textit{MS-PL}};

\node[rectangle,draw, text width=2cm, text height=0.34cm, fill=gray!40, text
centered] (l0213) at (7.2,5.3) {  \footnotesize  \textit{LGPL-3.X}};

\node[rectangle,draw, text width=2cm, text height=0.34cm, fill=gray!40,
text centered] (l0213) at (9.6,5.3) {  \footnotesize  \textit{GPL-3.0}};

\node[rectangle,draw,text width=1.6cm, text height=0.34cm, fill=gray!40, text
centered] (l0214) at (8.4,4.5) {  \footnotesize  \textit{AGPL-3.0}};
 
 
\node[rectangle,draw, text width=1.8cm, text height=0.34cm, fill=gray!40, text
centered] (l0221) at (11.6,8) {  \footnotesize  \textit{EUPL-1.X}};

\end{tikzpicture}
\end{center}

\end{small}


But regardless of the final textual form a license uses to express its
granting or revoking positions, in any case one has to consider some aspects: 

\begin{itemize}
  
  \item Overall, one has to keep in mind that of course no licensor, contributor
  and/or distributor can release the right to use any patents he does not own---%
  not even if he \emph{tries} to release them by an open source patent
  clause.%
    \footnote{The EPL is one of the licenses which insists on this aspect:
    It the second half of its patent clause, the EPL underlines that
    \enquote{[\ldots] no assurances are provided by any Contributor that the
    Program does not infringe the patent or other intellectual property rights of
    any other entity.} Moreover, it explicitly adds that \enquote{[\ldots] if a
    third party patent license is required to allow Recipient to distribute the
    Program, it is Recipient's responsibility to acquire that license before
    distributing the Program} (\cite[cf.][\nopage wp §2c]{Epl10OsiLicense2005a}).}
  Implictly touched patents of third parties not having contributed to the
  development and/or participated in the distribution can never be implicitly
  and automatically released on the base of such an (open source) patent clause:
  no rights, no right to release.% 
    \footnote{This is an important aspect which is sometimes not considered by
    programmers. Inside of DTAG we had a fruitful discussion evoked by Mr. Stephan
    Altmeyer who---as patent lawyer---patiently explained this constraint to us.} 
  Hence: even for those open source licenses which try to protect the users,
  finally the users themselves must nevertheless ensure that they do not violate
  the patents of third parties being unwillingly touched by the way the code
  works or the processes in which the software is used.%
    \footnote{Sometimes, this problem of willingly or
    unwillingly violated third party patents is seen as a weakness of open source
    software. But that is not true. It is a weakness of every software. Even a
    commercial licensor (developer) has only the right to license the use of those
    patents he really owns or he has `bought' for relicensing. Moreover, even
    commercial licensors can willingly or unwillingly violate patents of other
    persons.}
  
  \item In the context of a granting license, one has also to consider that
  contributing to and distributing a piece of software implicitly evokes that
  all patents of the contributor and/or distributor are `given free' which are
  necessary to use the software as whole---including the more or less deeply
  embedded libraries. So, if one wants to check whether some of the core patents
  of one's patent portfolio are afflicted by a patent clause (and whether one
  therefore better should not use/distribute the corresponding piece of open
  source software), one should not forget to check the embedded libraries, too.
  
  \item Finally, one has to consider in the context of a granting license that
  its patent clause only releases the use of the patents in the meaning of
  `allowed to be used for enabling the use of the distributed software.' The
  patent clause does not release the patents generally. Thus, the threat of
  (unwillingly) releasing patents by open source software is not as large as
  sometimes feared: the use of the patent is only granted in combination with
  the software. On the one hand, you may not use the open source software
  without having the right to use the patent because the use of the patent is
  inherently necessary for using the software---regardless, whether the open
  source software is embedded into a larger process or not. On the other hand,
  you are not allowed to use patents---released by the patent clause of an open
  source license---without exactly that open source software which has been
  licensed under this open source license, because the patent clause only refers
  to the use of just that open source software.
  % TODO: this is not completely accurat. OSS license grant downstream patent
  % licenses even for modified software.  The extent to which the software may
  % be modified varies between diffenrent licenses. (RPD)

  \item Summarized, one has to consider that the granting open source licenses
  automatically and implicitly force you to grant all the rights which are
  necessary to use the software legally. Open source contributors and
  distributors should know that.\footnote{Again: It might be debatable whether
  this is also valid for the distributors which do not contribute anything to
  the development. That's a legal discussion the \oslic{} does not wish to participate
  in. From the viewpoint of an open source user who only wants to have one
  reliable and secure way to use open source software compliantly, one should
  perhaps assume that there is no difference.}

  \item With respect to the revoking licenses, one has to consider that their
  patent clauses contain negative conditions which may be read as interdictions.
  The \oslic{} will integrate these conditions into specific `prohibits'-sections
  of its to-do lists.
  
  \item Finally one should mention that in some cases, the form of the
  revocation used by the revoking license refers to the use of the software, in
  other cases to the use of the patents. But nevertheless, one can reason that%
  ---from the pragmatic viewpoint of a benevolent open source software user---%
  this second case of patent revocation also implicitly terminates the right to
  use the software: If the use of a patent is necessary to use a piece of
  software legally, one is not allowed to use the software without having the
  right to use the patent, too; and if the use of the patent is not necessary
  for using the software, then the patent is not covered by the patent clause.
  So, in any case, this kind of patent clauses seems to terminate the right to
  use, distribute or modify the software. Hence, single users as well
  as companies or organizations should also respect such patent clauses if they
  want to be sure to use open source software compliantly.
\end{itemize}

The \oslic{} wants to support its readers not only to act according to the licenses
in general, but also according to its patent clause. Thus, we now briefly cite
and summarize the meaning of particular patent clauses:

\subsection{AGPL statements concerning patents}
\patentlabel{AGPL}

(prelimiary text)

The AGPL-3.0 is a license derived from the GPL-3.0: apart from the preamble and
the paragraphs §11 and §13, they contain nearly the same text.%
  \footnote{compare \cite[][\nopage]{Agpl30OsiLicense2007a} and
  \cite[][\nopage]{Gpl30OsiLicense2007a} in both §1 \ldots §11}
In §13, the AGPL explictly refers to the focus on a \enquote{remote network
interaction} which shall also be able to trigger the delivery of the
corresponding source code; and in §11, the AGPL establishes its specific patent
clause \cite[cf.][\nopage §11 and §13]{Agpl30OsiLicense2007a}.

Like the GPL-3.0, the AGPL-3.0 tries to protect all licensees against patent
claims. This kind of protection is then established by three steps:

First, the AGPL-3.0 assures that \enquote{each contributor grants a non
exclusive, worldwide, royalty free patent license under the contributor’s
essential patent claims, to make, use, sell offer for sale, import and
otherwise run, modify and propagate the contents of its contributor
version.}\citeAGPL{§11} Furthermore, the patent license defines that this patent
license granted by the contributor is automatically extended to all downstream
recipients who later on receive any version of the work even if they indirectly
receive them by third parties and even if they receive a covered work or work
based on the program.\citeAGPL{§11}

Second, the AGPL enforces not only the grant of patent licenses by the
\enquote{contributors,} the license even requires the same from licensees who
distributes the program unchanged: \enquote{If, pursuant to or in connection
with a single transaction or arrangement, you convey, or propagate by procuring
conveyance of, a covered work, and grant a patent license to some of the parties
receiving the covered work authorizing them to use, propagate, modify or convey
a specific copy of the covered work, then the patent license you grant is
automatically extended to all recipients of the covered work and works based on
it.}\citeAGPL{§11}

Finally, the AGPL-3.0 introduces an revoking clause by stating that a licensee
\enquote{[\ldots] may not initiate litigation (including a cross-claim or
counterclaim in a lawsuit) alleging that any patent claim is infringed by
making, using, selling, offering for sale, or importing the Program or any
portion of it}\citeAGPL{§10} and that this licensee \enquote{automatically}
loses the rights granted by the AGPL-3.0 \enquote{including any patent
licenses} if he tries to propagate or modify a covered work against the
regulations of the AGPL-3.0.\citeAGPL{§8} 

According to that, the AGPL-3.0 is like the GPL-3.0 a granting and a revoking
license: At first, one is granted the right to use all patents of all
contributors which are necessary to use the software legally. But if one
installs any litigation regarding an infringement of patents, then the rights
granted to him are revoked.


\subsection{Apache-2.0 statements concerning patents}
\patentlabel{APL}

Titled by the headline \enquote{Grant of Patent License}, the Apache License~2.0
contains a specific patent clause being comprised of two very long and condensed
sentences.\citeAPL{§3} Outside of this patent clause, the word \emph{patent} is
only used once again---for requiring that one \enquote{[\ldots] must retain, in
the (sources) [\ldots] all [\ldots] patent [\ldots] notices [\ldots]}\citeAPL{§4.3}

The one core message of the Apache-2.0 patent clause is that
\enquote{[\ldots] each Contributor hereby grants to You a perpetual, worldwide,
non-exclusive, no-charge, royalty-free, irrevocable [\ldots] patent license to
make, have made, use, offer to sell, sell, import, and otherwise transfer the
Work [\ldots]}%
  \footnote{\cite[cf.][\nopage wp §3]{Apl20OsiLicense2004a}. The
  \enquote{Contributor,} \enquote{Work,} and \enquote{You} are defined in §1:
  \emph{Contributor} refers to the original licensor and to all others whose
  contributions have been incorporated into the Work. The \emph{Work} denotes
  the result of the development process regardless of its form. \emph{You}
  denotes the licensees.}

The second core message of the Apache-2.0 patent clause is the statement that
\enquote{if You institute patent litigation against any entity [\ldots] alleging
that the Work [\ldots] constitutes [\ldots] patent infringement, then any patent
licenses granted to You [\ldots] shall terminate [\ldots]}\citeAPL{§3}

The third message of the Apache-2.0 patent clause is the statement, that the
\enquote{[\ldots] license applies only to those patent claims licensable by such
Contributor that are necessarily infringed by their Contribution(s) alone or by
combination of their Contribution(s) with the Work to which such Contribution(s)
was submitted}.\citeAPL{§3}

Thus, the Apache-2.0 is---as we use to say in this chapter---a granting and a
revoking license: At first you are granted to use all patents of all
contributors which are necessary to use the software legally. But if you---with
respect to the software---install any litigation concerning the infringement of
patents, then the rights granted to you are revoked.

\subsection{CDDL statements concerning patents}
\patentlabel{CDDL}

The patent clauses of the CDDL are similiar in spirit to the Apache License: 
The license grants rights to each contributors patents that are neccessarily
infringed by distributing or using the software. The license also revokes all
rights granted to someone who files a patent litigation with respect to the
software against any contributor.  The CDDL differs from other licenses in that
the litigant does not lose his rights automatically and immediately but gets a
grace period of 60 days. If he withdraws his claims during this period, the
license granted to him will not be terminated.

The actual wording used in the CDDL is complicated by the fact that the CDDL
distinguished between the \enquote{Initial Developer} and other
\enquote{Contributors.}  A \enquote{Contributor} receives a version of the
software to which he then adds some \enquote{Modifications} thus creating the
\enquote{Contributor Version.} For all practical purposes we can treat the
\enquote{Initial Developer} as another contributor who happens to not receive
any software and whose \enquote{Contributor Version} (officially called
\enquote{Original Software}) equals his \enquote{Modifications.}

The patent licenses are granted in the clause (b) of the sections titled
\enquote{The Initial Developer Grant}\citeCDDL{§2.1(b)} and \enquote{Contributor
  Grant.}\citeCDDL{§2.2(b)} Each contributor grants the licensee \enquote{a
  world-wide, royalty-free, non-exclusive license under Patent Claims infringed
  by the making, using, or selling of Modifications made by that Contributor
  either alone and/or in combination with its Contributor Version [\ldots], to
  make, use, sell, offer for sale, have made, and/or otherwise dispose of: (1)
  Modifications made by that Contributor [\ldots]; and (2) the combination of
  Modifications made by that Contributor with its Contributor Version [\ldots]} 
This limits the patent license to patents infringed by code present in the
contributor version. And clause (d) limits the grant even further to exclude
\enquote{infringements caused by[\ldots]third party modifications of Contributor
Version}\citeCDDL{§2.2(d)} or {Covered Software in the absence of Modifications
made by that Contributor.}\citeCDDL{§2.2(d)}
This ensures that no contributor is required to tolerate an infringement of his
patents caused by code modified after he made his contribution and, in
particular, it is not possible to remove the contributors modifications completely
without also removing all other causes of infringement of the patent claims
because the patent license does not carry over to such a use of the software.

The section titled \enquote{TERMINATION} contains the usual defense
against patent infringement claims by declaring that any such claim
against a \enquote{Participant%
  \footnote{The \enquote{Contributor} or \enquote{Initial Developer} against
  whom the claim is made}
[\ldots] alleging that the Participant Software [\ldots] directly or indirectly
infringes any patent, then any and all rights granted directly or indirectly to 
You\footnote{The party making the patent infringement claim}
[\ldots] under Sections 2.1 and/or 2.2 of this
License shall, upon 60 days notice from Participant terminate prospectively and
automatically at the expiration of such 60 day notice period, unless [\ldots] 
You withdraw Your claim [\ldots] against such Participant either unilaterally or
pursuant to a written agreement with Participant.}

Thus, not only has the Participant to actively initiate the termination of the
licenses, the licensee also has 60 days to either settle the case by an
agreement with the Participant or to withdraw his claims.


\subsection{EPL statements concerning patents}
\patentlabel{EPL}

The Eclipse Public License treats the patents necessary to use the program
in the same section and under the same headline \enquote{Grant of Rights} like
all the other rights: First, the EPL clearly states that \enquote{[\ldots] each
Contributor [\ldots] grants (the recipient) a non-exclusive, worldwide,
royalty-free patent license under Licensed Patents to make, use, sell, offer to
sell, import and otherwise transfer the Contribution of such Contributor, if
any, in source code and object code form.}\citeEPL{§2.b} Then the EPL delimits
the extend of this act of granting: Neither hardware patents of the contributors
are covered by this releasing patent clause, nor patents that concern aspects
out of the area of the initially intended software combination.\citeEPL{§2.b}
Finally, the EPL hints to the general fact that 3$^{rd}$ party patents not
belonging to the contributors can never be implicity be released by such a
patent clause. Moreover, it gives the example that \enquote{[\ldots] if a third
party patent license is required to allow Recipient to distribute the Program,
it is Recipient's responsibility to acquire that license before distributing the
Program.}\citeEPL{§2.c}

Like other open source licenses, the EPL announces at its end that
\enquote{if (a) Recipient institutes patent litigation against any entity
[\ldots] alleging that the Program [\ldots] infringes such Recipient's
patent(s), then such (granted) Recipient's rights [\ldots] shall terminate
[\ldots]}\citeEPL{§7}

Thus, the EPL, too, is a granting and a revoking license: 
At first you are granted the use of all patents of all
contributors which are necessary to use the software legally. But if you---with
respect to the software---install any litigation concerning an infringement of
patents, then the rights granted to you are revoked.

\subsection{EUPL statements concerning patents}
\patentlabel{EUPL}

The European Union Public License contains a very brief patent clause. It only
states, that \enquote{the Licensor grants to the Licensee royalty-free, non
exclusive usage rights to any patents held by the Licensor, to the extent
necessary to make use of the rights granted on the Work under this
Licence.}\citeEUPL{end of §2}
Furthermore the EUPL does not contain any patent specific revoking clause, but
only an abstract clause requiring that all \enquote{[\ldots] the rights granted
hereunder will terminate automatically upon any breach by the Licensee of the
terms of the Licence}\citeEUPL{§12}. Thus, the EUPL is---as we are using to say
in this chapter---a granting license but not a revoking license.

\subsection{GPL statements concerning patents}

Although the GPL versions 2.0 and 3.0 are aiming for the same results, they
differ heavily with respect to textual and arguing structure. Therefore, it
should be helpful to treat these two licenses separately.

\subsubsection{GPL-2.0}
\patentlabel{GPLa}

The GPL-2.0 does not contain any specific patent clause by which it would grant
(and revoke) the rights to use those patents belonging to the contributors and 
being necessary to use the software in accordance with the legal patent system.

Instead of this, the preamble of the GPL-2.0 alleges that \enquote{[\ldots] any
free program is threatened constantly by software patents} and that the authors
of the GPL---for tackling this threat---\enquote{[\ldots] had made it clear
that any patent must be licensed for everyone's free use or not licensed at
all}\citeGPLtwo{Preamble}. Unfortunately, this specification is only an indirect
claim which needs a lot of arguing for establishing a protective effect against
patent disputes. Howsoever, this paragraph of the GPL-2.0 does not directly
grant any rights to the software users to use necessary patents, too.

With respect to the patent problem, the GPL-2.0 also states that a licensee has
to fulfill the conditions of the GPL-2.0 completely, even if an existing patent
infringement---being \enquote{imposed} on the GPL licensee---\enquote{[\ldots]
contradicts the conditions of this license} so, that a waiver of the use of the
software is the only way to fulfill both constraints.\citeGPLtwo{§11} And
finally the GPL-2.0 allows the original copyright holder to \enquote{add an
explicit geographical distribution limitation excluding [\ldots] countries}
provided that these countries \enquote{[\ldots] (restict) the distribution
and/or use of the library [\ldots] by patents [\ldots]}\citeGPLtwo{§12}
Based on these statements, one cannot infer that the GPL-2.0 grants any patent
rights to the software user, neither directly, nor indirectly.

Thus, the GPL-2.0 is neither a granting nor a revoking license.

\subsubsection{GPL-3.0}
\patentlabel{GPLb}

Initially, the GPL-3.0 regrets that \enquote{[\ldots] every program is
threatened constantly by software patents} what should be seen as the
\enquote{[\ldots] danger that patents applied to a free program could make it
effectively proprietary}. And therefore---as the GPL-3.0 itself summarizes its
patent rules---\enquote{[\ldots] the GPL assures that patents cannot be used to
render the program non-free.}\citeGPLthree{Preamble}. This kind of protection is
then established by three steps. First, the GPL-3.0 stipulates that
\enquote{each contributor grants [\ldots the licensees] a non-exclusive,
worldwide, royalty-free patent license under the contributor's essential patent
claims, to make, use, sell, offer for sale, import and otherwise run, modify and
propagate the contents of its contributor version.}\citeGPLthree{§11}
Second, the GPL-3.0 defines that this patent license granted by the contributor
\enquote{[\ldots] is automatically extended to all recipients} who later on
receive any version of the work, even if they indirectly receive them by third
parties and even if they receive a \enquote{covered work} or \enquote{works
based on it.}\citeGPLthree{§11} Moreover, the GPL-3.0 also specifies that those
distributors of a \enquote{covered work} who have the right to use a patent
necessary for the use of the distributed software but who are not allowed to
relicense this patent to third parties must solve this problem by making the
source code available nevertheless, by \enquote{depriving} themselves or by
\enquote{extending the patent license to downstream recipients.}\citeGPLthree{§11} 
And finally, the GPL-3.0 also introduces a revoking clause by stating that a
licensee \enquote{[\ldots] may not initiate litigation [\ldots] alleging that
any patent claim is infringed by making, using, selling, offering for sale, or
importing the Program or any portion of it}\citeGPLthree{§10} and that this
licensee \enquote{automatically} loses the rights granted by the GPL-3.0
\enquote{including any patent licenses} if he tries to propagate or modify a
covered work against the rules of the GPL-3.0.\citeGPLthree{§8}

Thus, GPL-3.0 is a granting and a revoking license: At first, one is granted the
right to use all patents of all contributors which are necessary to use the
software legally. But if you---with respect to the software---install any
litigation concerning an infringement of patents, then the rights granted to you
are revoked. 


\subsection{LGPL statements concerning patents}

As already mentioned above, the LGPL versions 2.1 and~3.0 differ heavily with
respect to textual and arguing structure. Therefore, they should be treated
separately.

\subsubsection{LGPL-2.1}
\patentlabel{LGPLa}

Like the GPL-2.0, the LGPL-2.1 does not contain any specific patent clause by
which it would grant (and revoke) the rights to use those patents belonging to
the contributors and being necessary to use the software in accordance with the
legal patent system.

Instead of this, the preamble of the LGPL-2.1 says that \enquote{[\ldots]
software patents pose a constant threat to the existence of any free program}
and that the authors of the LGPL---for tackling this threat---%
\enquote{[\ldots] insist that any patent license obtained for a version of the
library must be consistent with the full freedom of use specified in this
license.}\citeLGPLtwo{Preamble}
Unfortunately, this specification is again only an indirect claim which needs a
lot of arguing to establish a protective effect against patent disputes.
Howsoever, this paragraph of the LGPL-2.1 does not directly grant any rights to
the software users to use necessary patents.

With respect to the patent problem, the LGPL-2.1 also states that a licensee has
to fulfill the conditions of the LGPL-2.1 completely, even if an existing patent
infringement---being \enquote{imposed} on the LGPL licensee---%
\enquote{[\ldots] contradicts the conditions of this license} so that a waiving
of the use of the software is the only way to fulfill both
constraints.\citeLGPLtwo{§11} And finally the LGPL-2.1 allows the original
copyright holder to \enquote{add an explicit geographical distribution limitation
excluding [\ldots] countries} provided that these countries \enquote{[\ldots]
(restict) the distribution and/or use of the library [\ldots] by patents
[\ldots]}\citeLGPLtwo{§12} Based on these statements, one cannot infer that 
the LGPL grants any patent rights to the software user, neither directly, nor
indirectly.

Thus, the LGPL-2.1 is neither a granting nor revoking license.

\subsubsection{LGPL-3.0}
\patentlabel{LGPLb}

The LGPL-3.0 is an extension of the GPL-3.0. Before starting with a section
\enquote{Additional Definitions}, the LGPL-3.0 states that it \enquote{[\ldots]
incorporates the terms and conditions of version~3 of the GNU General Public
License} and then \enquote{supplements} this GPL-3.0 content by some
\enquote{additional permissions.}\citeLGPLthree{wp} The LGPL-3.0 itself does not
contain the word `patent,' but the GPL-3.0 does.\citeGPLthree{§11}
So, the LGPL-3.0 inherits its patent clause from the GPL-3.0 which is---as we
already described\footnote{$\rightarrow$ \oslic{}, p.\
\patentpageref{GPLb}}---a granting and a revoking license.
 
\subsection{MPL statements concerning patents}
\patentlabel{MPL}

The MPL distributes its statements concerning the tolerated use of the patents
over three paragraphs: First, it clearly says that \enquote{each Contributor
[\ldots] grants [\ldots the licensee] a world-wide, royalty-free,
non-exclusive license [\ldots] under Patent Claims of such Contributor to
make, use, sell, offer for sale, have made, import, and otherwise transfer
either its Contributions or its Contributor Version}\citeMPL{§2.1,
esp. §2.1.b} Second, it hihlights some \enquote{limitations.}\citeMPL{§2.3}
And finally, the MPL introduces a revoking clause which signifies that the
rights, granted to the licensee \enquote{[\ldots] by any and all Contributors
[\ldots] shall terminate} if the licensee \enquote{initiates litigation
against any entity by asserting a patent infringement claim [\ldots] alleging
that a Contributor Version directly or indirectly infringes any patent
[\ldots]}\citeMPL{§5.2}

Thus, the MPL is a granting license and a revoking license.

\subsection{MS-PL statements concerning patents}
\patentlabel{MSPL}

First, the MS-PL contains a statement, by which \enquote{[\ldots] each 
contributor grants (the software users) a non-exclusive, worldwide, royalty-free 
license under its licensed patents to make, have made, use, sell, offer for 
sale, import, and/or otherwise dispose of its contribution in the software or 
derivative works of the contribution in the software.}\citeMSPL{§2.B} Second,
the MS-PL says that \enquote{if you bring a patent claim against any
contributor[\ldots] your patent license from such contributor to the software
ends automatically.}\citeMSPL{§3.B} 

Thus, the MS-PL is a granting and a revoking license: At first you are granted
to use all patents of all contributors which are necessary to use the software
legally. But if you install any litigation concerning an infringement of
patents with respect to the software, then the rights granted to you are revoked. 

% \bibliography{../../../bibfiles/oscResourcesEn}

% Local Variables:
% mode: latex
% fill-column: 80
% End:
