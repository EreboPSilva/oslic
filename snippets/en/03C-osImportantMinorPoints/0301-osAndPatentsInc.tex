% Telekom osCompendium 'for being included' snippet template
%
% (c) Karsten Reincke, Deutsche Telekom AG, Darmstadt 2011
%
% This LaTeX-File is licensed under the Creative Commons Attribution-ShareAlike
% 3.0 Germany License (http://creativecommons.org/licenses/by-sa/3.0/de/): Feel
% free 'to share (to copy, distribute and transmit)' or 'to remix (to adapt)'
% it, if you '... distribute the resulting work under the same or similar
% license to this one' and if you respect how 'you must attribute the work in
% the manner specified by the author ...':
%
% In an internet based reuse please link the reused parts to www.telekom.com and
% mention the original authors and Deutsche Telekom AG in a suitable manner. In
% a paper-like reuse please insert a short hint to www.telekom.com and to the
% original authors and Deutsche Telekom AG into your preface. For normal
% quotations please use the scientific standard to cite.
%
% [ Framework derived from 'mind your Scholar Research Framework' 
%   mycsrf (c) K. Reincke 2012 CC BY 3.0  http://mycsrf.fodina.de/ ]
%


%% use all entries of the bibliography
%\nocite{*}

\section{The Problem of Implicitly Releasing Patent Utilization}
\footnotesize \begin{quote}\itshape In this chapter, we are briefly analyzing
the effect of patent clauses in open source licenses -- not in general, but with
respect to those license fulfilling tasks they require, also known as the
'implicit acceptance of a patent use' by distributing Open Source Software.
\end{quote}
\normalsize

At least the free software movement frowns on the existence of software
patents\footnote{For an early and elaborated description on the effects of
software patents based on the viewpoint of the free software movement
\cite[see][\nopage wp]{Stallman2001a}. It is a lecture which has been given more
than onetime and which has also been printed lateron (\cite[cf.][\nopage
wp]{Stallman2002a}). Within the first decade of 2000, the focus switched to a
more political fight against software patents (\cite[cf.][\nopage
wp]{Stallman2004a}). But recently there seemed to appear another turn in dealing
with software patents: Not fighting against them, but mitigating there effects:
The proposal is '[...] (to legislate) that developing, distributing, or running
a program on generally used computing hardware does not constitute patent
infringement' (\cite[cf.][\nowpage wp]{Stallman2012a})}. One of the most known
witnesses for that attitude is the preamble of the GPL itself which purports
that \enquote{[\ldots] any free program is threatened constantly by software
patents}\footcite[cf.][wp]{Gpl20OsiLicense1991a}. It is said, that there are
three threats feared by the open source community: First, they are apprehensive
of the patent announcement by people who do not develop the open source software
and which then try to prevent the use of the developed software by installing
patent ligitations\footcite[cf.][234]{JaeMet2011a}. Second, they fear a bramble
of general software patents which practically prohibits to develop open source
software  legally\footcite[cf.][234]{JaeMet2011a}. Third, they anticipate the
possibility that (not benevolent) open source developers could try to register
patents for undermining the open source
principles\footcite[cf.][235]{JaeMet2011a}.

Howsoever, regardless wether one tries to fight against software patents or not,
software patents have come true. To act law-abindingly requires to manage the
constraints of patents properly. Open source licenses know and respect this
necessity. Moreover, at least some of them try to manage the effect of software
patents by a specific patent clause\footcite[pars pro toto cf.][\nopage wp.
§3]{Apl20OsiLicense2004a} or by several distributed sentences\footcite[pars pro
toto cf.][\nopage wp.]{Epl10OsiLicense2005a}. But why does the OSLiC have to
deal with this topic if the OSLiC does not want to participate in general
discussions, but only wants to list one possible way to fulfill the open soure
license with respect to the specific use cases?

In opposite to the other conditions of the open source licenses, their patent
clauses or propositions in general do not directly refer to a specific set of
actions which has to be executed for acting in accordance to the licenses. Open
source patent clauses normally do not join in the game 'paying by doing'. So,
actually, it seems to be not necessary to mention the patent clauses.

Unfortunately, although the patent clauses do not directly say \emph{'do this or
that in these or those circumstances'}, they have nevertheless some side effects
which evoke that the distributors - in all the open source software use cases
which include the distribution of open source software - implictly having
already something done if they are using and distributing a piece of open source
software. This implicit effect makes it necessary to deal with the patent
clauses even in the only pragmatically oriented OSLIC.

Patent clauses in open source licenses can have two different directions of
impact. They use two methods to protect the users of the open source software --
and sometimes these methods are combined:

\begin{itemize}
  \item First, an open source license can assure that all contributors /
  distributors to / of a piece of open source software grant to all users /
  receivers not only the right to use the open source software itself, but
  automatically and implicitly also the right to use all those patents --
  belonging to the contributors / distributors -- which as patents are necessary
  to use the software legally\footnote{There might arise a legal discussion
  whether a distributor who does not contribute to the software development
  really imlictly also has to grant the necessary rights of his patent
  portfolio. The OSLiC doesn't want to participate in this discussion. We take a
  simple and pragmatic position: for being sure that you are acting according to
  such an open source license and its patent clause you should simply assume
  that you have to do so. If this default position is not reasonable for you it
  might be a good idea to consult legal experts which -- perhaps -- may find
  another way for you to use the software legally.}. So, let us - only on the
  following few pages -- name such licenses the \emph{granting licenses}.
  \item Second, an open source license can try to automatically terminate the
  right to use, to modify , and to distribute the software if its user litigates
  against any of the contributors / distributors with respect to patent. That
  can be seen as a revocation of earlier granted rights. So, let us name these
  license the \emph{revoking licenses}.
\end{itemize}

Later on, we will summarize the concrete patent clauses of the licenses as proof
of the following classifying picture. But in which definite form ever these
positions arise in a specific license, one has to consider some aspects:

\begin{itemize}
  
  \item  Overall, one has to keep in mind, that - even as part of a granting
  license - of course no patent clause can release the right of to use patents,
  which are not owned by the licensors, contributors and/or
  distributors\footnote{This is an important aspect which is often not seen.
  Inside of DTAG we had a fruitful discussion evoked by Mr. Stephan Altmeyer who
  -- as patent lawyer -- patiently explained it to us.}. Hence: even for those
  open source licenses which try to protect the users, the user itself must
  nevertheless ensure that he is not violating the patents of third parties
  which can be covered by the patent clauses\footnote{Sometimes, this problem of
  willingly or unwillingly vioaleted patents of third parties is seen as a
  weakness of open source software. But that is not true. It is a weakness of
  every software. Even a commercial licensor (developer) has only the right to
  license the use of those patents he really owns or he has 'bought' for
  relicensing them. Moreover, also commercial licensors can willingly or
  unwillingly violate patents of other persons.}
  
  \item In the context of a granting license, one has also to consider, that
  contributing to and distributing of a piece of software implictly evokes that
  all patents of the contributor and/or distributor are 'given free' which are
  necessary to use the software as whole -- including the more or less deeply
  embedded libraries. This might evoke a problem for those persons,
  organizations, or companies which see some of these patents as core patents of
  their patent portfolio.
  
  \item Finally, one has to consider in the context of a granting license, that
  its patent clause give free the use of the patents only in the meaning
  'allowed to be used for enabling the use of the distributed software'. They do
  not release the patent use generally. So, the threat of (unwillingly)
  releasing patents by open source software is not as large as often feared:
  granted is only the combination of the use of the patent and the software. On
  the one hand, you may not use the software without having the right to use the
  patent because the use of the patent is inherently necessary for using use the
  sofwtare. On the other hand, you are not allowed to use the patent without the
  software because the patent clauses only refer to the use of that specific
  open source software licensed by the corresponding open source license.

  \item Summarized one has to consider that granting open source licenses
  automatically and implicitly enforce you to grant all the rights to the user
  of the software which are necessary to use the software legally. Open source
  contributors and distributors should know that\footnote{It might be debatable
  wether this is also valid for not contributing distributors. From the
  viewpoint of an open source user who wants to have one reliable way you should
  assume that there is no difference.}.

  \item With respect to the revoking licenses, one has to consider that it
  clearly contains a negative condition in form of an interdiction. We will
  integrate these conditions into a 'prohibits'-section of the to-do lists.
  
  \item Finally one should mention, that the form of the revocation used in the
  revoking license in some cases refers the use of the software, in other cases
  the use of the patents. But one has to consider that -- from the pragmatic
  viewpoint of a benevolent open source software user -- also this second case
  of patent revocation implictly terminates the right to use the software: If
  the use of a patent is necessary to use a piece of software, one is not
  allowed to use the software; and if the use of the patent is not necessary for
  using the software, the patent is not covered by the patent clause. So, in any
  case, this kind of patent clause terminates the right to use / to distribute
  and/or to modify the software. Hence, single users as well as companies or
  organizations should also respect such patent clauses if they want to use open
  source software compliantly.
\end{itemize}

For supporting you by using the open source software not only according to
license in general, but according to its patent clause, we now briefly cite and
summarize the meaning of the particular patent clauses:

\subsection{Patent clause of the ApL}

\subsection{Patent clause of the ApL}

\subsection{Patent clause of the BSD Licenses}

\subsection{Patent clause of the EPL}

\subsection{Patent clause of the EUPL}

\subsection{Patent clause of the GPL}

\subsection{Patent clause of the LGPL}
 
\subsection{Patent clause of the MIT License}

\subsection{Patent clause of the MPL}         

\subsection{Patent clause of the MS-PL}
  
\subsection{Patent clause of the PgL}

\subsection{Patent clause of the PHP License}



% \bibliography{../../../bibfiles/oscResourcesEn}
