% Telekom osCompendium 'for being included' snippet template
%
% (c) Karsten Reincke, Deutsche Telekom AG, Darmstadt 2011
%
% This LaTeX-File is licensed under the Creative Commons Attribution-ShareAlike
% 3.0 Germany License (http://creativecommons.org/licenses/by-sa/3.0/de/): Feel
% free 'to share (to copy, distribute and transmit)' or 'to remix (to adapt)'
% it, if you '... distribute the resulting work under the same or similar
% license to this one' and if you respect how 'you must attribute the work in
% the manner specified by the author ...':
%
% In an internet based reuse please link the reused parts to www.telekom.com and
% mention the original authors and Deutsche Telekom AG in a suitable manner. In
% a paper-like reuse please insert a short hint to www.telekom.com and to the
% original authors and Deutsche Telekom AG into your preface. For normal
% quotations please use the scientific standard to cite.
%
% [ Framework derived from 'mind your Scholar Research Framework' 
%   mycsrf (c) K. Reincke 2012 CC BY 3.0  http://mycsrf.fodina.de/ ]
%


%% use all entries of the bibliography
%\nocite{*}

\section{The Problem of Implicitly Releasing Patent Utilization}
\footnotesize \begin{quote}\itshape In this chapter, we are briefly analyzing
the effect of patent clauses in open source licenses -- not in general, but with
respect to the license fulfilling tasks they require, also known as the
'implicit acceptance of a patent use' by distributing open source software.
\end{quote}
\normalsize

At least the free software movement frowns on the existence of software
patents\footnote{For an early and elaborated description on the effects of
software patents based on the viewpoint of the free software movement
\cite[see][\nopage wp]{Stallman2001a}. This lecture seems to be given more
than onetime and seems also to have been printed lateron (\cite[cf.][\nopage
wp]{Stallman2002a}). Within the first decade of 2000, the focus switched to a
more political fight against software patents (\cite[cf.][\nopage
wp]{Stallman2004a}). But recently there seemed to appear another turn in dealing
with software patents: Not fighting against them, but mitigating their effects:
The proposal is '[...] (to legislate) that developing, distributing, or running
a program on generally used computing hardware does not constitute patent
infringement' (\cite[cf.][\nopage wp]{Stallman2012a})}. One of the most known
witnesses for that attitude is the preamble of the GPL itself which purports
that \enquote{[\ldots] any free program is threatened constantly by software
patents}\footcite[cf.][wp]{Gpl20OsiLicense1991a}. It is said, that there are
three threats feared by the open source community: First, they are apprehensive
of people who hjack the idea of a piece of open source software they do not have
developed, register a corresponding patent, and finally try to earn money by
preventing the use of the software or by envolving its users into patent
ligitations\footcite[cf.][234]{JaeMet2011a}. Second, they fear a bramble of
general software patents which practically prohibits to develop open source
software legally\footcite[cf.][234]{JaeMet2011a}. Third, they anticipate the
possibility that (not quite benevolent) open source developers could try to
register patents for undermining the open source
principles\footcite[cf.][235]{JaeMet2011a}.

Howsoever, regardless wether one tries to fight against software patents or not,
software patents have come true. To act law-abindingly requires to manage the
constraints of patents properly. Open source licenses know and respect this
necessity. Moreover, at least some of them try to manage the effect of software
patents by specific patent clauses\footcite[pars pro toto cf.][\nopage wp.
§3]{Apl20OsiLicense2004a} or by several sentences distributed in the license
text\footcite[pars pro toto cf.][\nopage wp.]{Epl10OsiLicense2005a}. But why
does the OSLiC have to deal with this topic if the OSLiC does not want to
participate in general discussions?

In opposite to the other conditions of the open source licenses, their patent
clauses or propositions in general do not directly refer to a specific set of
actions which has to be executed for acting in accordance to the licenses. Open
source patent clauses normally do not join in the game 'paying by doing'. So,
actually, it seems to be not necessary to mention the patent clauses.

Unfortunately, although the patent clauses do not directly say \emph{'do this or
that in these or those circumstances'}, some of them have nevertheless side
effects which evoke that the distributors of open source software implictly
having already something done if they actually are distributing a piece of open
source software. This implicit effect makes it necessary to deal with the patent
clauses even in the only pragmatically oriented OSLIC.

Patent clauses in open source licenses can have two different directions of
impact. They use two methods to protect the users of the open source software --
and sometimes these methods are combined:

\begin{itemize}
  \item First, an open source license can assure that all contributors /
  distributors to / of a piece of open source software grant to all users /
  receivers not only the right to use the open source software itself, but
  automatically and implicitly also the right to use all those patents --
  belonging to the contributors / distributors -- which as patents are necessary
  to use the software legally\footnote{There might arise a legal discussion
  whether a distributor who does not contribute to the software development
  really imlictly also has to grant the necessary rights of his patent
  portfolio. The OSLiC doesn't want to participate in this discussion. We take a
  simple and pragmatic position: for being sure that you are acting according to
  an open source license with such a patent clause you should simply assume
  that you have to do so. If this default position is not reasonable for you it
  might be a good idea to consult legal experts which -- perhaps -- may find
  another way for you to use the software legally.}. So, let us - a little
  simplifying and therefore following few pages -- name such licenses the
  \emph{granting licenses}.
  \item Second, an open source license can try to automatically terminate the
  right to use, to modify , and to distribute the software if its user litigates
  against any of the contributors / distributors with respect to a software
  patent. That can be seen as a revocation of earlier granted rights. So, let us
  name these license the \emph{revoking licenses}.
\end{itemize}

Later on, we will summarize the concrete patent clauses of all the licenses
discussed in the OSLiC as a proof for the following classification:


\begin{center}

\begin{tikzpicture}
\label{LICTAX}
\small

\node[ellipse,minimum height=8.5cm,minimum width=14cm,draw,fill=gray!10] (l0100) at (6.8,6.8)
{  };

\draw [-,dotted,line width=0pt,white,
    decoration={text along path,
              text align={center},
              text={|\itshape|open source licenses}},
              postaction={decorate}] (-0.8,6.5) arc (218:322:9.5cm);
              
\node[ellipse,minimum height=6.2cm,minimum width=4.4cm,draw,fill=gray!20] (l0100) at (2.6,6.8)
{  };

\draw [-,dotted,line width=0pt,white,
    decoration={text along path,
              text align={center},
              text={|\itshape| without patent clauses}},
              postaction={decorate}] (0.9,7.4) arc (180:0:1.8cm);

\node[circle,draw,text width=1cm, fill=gray!40, text centered] (l0101) at (2,8)
{  \footnotesize \bfseries \textit{?}};
\node[circle,draw,text width=1cm, fill=gray!40, text centered] (l0102) at (3.5,8)
{  \footnotesize \bfseries \textit{BSD}};
\node[circle,draw,text width=1cm, fill=gray!40, text centered] (l0103) at (2,6.5)
{  \footnotesize \bfseries \textit{MIT}};
\node[circle,draw,text width=1cm, fill=gray!40, text centered] (l0104) at (3.5,6.5)
{  \scriptsize \bfseries \textit{?}};
\node[circle,draw,text width=1cm, fill=gray!40, text centered] (l0105) at (2,5)
{  \footnotesize \bfseries \textit{?}};
\node[circle,draw,text width=1cm, fill=gray!40, text centered] (l0106) at (3.5,5)
{  \footnotesize \bfseries \textit{?}};

\node[ellipse,minimum height=6cm,minimum width=8.5cm,draw,fill=gray!20] (l0200) at (9.2,6.5)
{  };

\draw [-,dotted,line width=0pt,white,
    decoration={text along path,
              text align={center},
              text={|\itshape| with patent clauses}},
              postaction={decorate}] (7.5,8.5) arc (120:60:4cm);


\node[ellipse,minimum height=4.5cm,minimum width=6cm,draw,fill=gray!30] (l0220) at (10.4,6.5)
{  };
\node[ellipse,minimum height=4.5cm,minimum width=5cm,draw,fill=gray!30] (l0210) at (7.45,6.5)
{  };
\node[ellipse,minimum height=4.5cm,minimum width=6cm,draw] (l0220) at (10.4,6.5)
{  };


\draw [-,dotted,line width=0pt,white,
    decoration={text along path,
              text align={center},
              text={|\itshape| granting}},
              postaction={decorate}] (5.4,6.2) arc (180:0:2cm);

\node[circle,draw,text width=1cm, fill=gray!40, text centered] (l0211) at (6.7,7)
{  \footnotesize \bfseries \textit{?}};
\node[circle,draw,text width=1cm, fill=gray!40, text centered] (l0212) at (8.4,7)
{  \footnotesize \bfseries \textit{ApL}};
\node[circle,draw,text width=1cm, fill=gray!40, text centered] (l0213) at (6.7,5.5)
{  \footnotesize \bfseries \textit{?}};
\node[circle,draw,text width=1cm, fill=gray!40, text centered] (l0214) at (8.2,5.5)
{  \footnotesize \bfseries \textit{?}};

 
% line width=0pt,white,
\draw [-,dotted,line width=0pt,white,
    decoration={text along path,
              text align={center},
              text={|\itshape| revoking}},
              postaction={decorate}] (10.4,7) arc (180:0:1cm);

\node[circle,draw,text width=1cm, fill=gray!40, text centered] (l0221) at (11.4,7)
{  \footnotesize \bfseries \textit{?}};
\node[circle,draw,text width=1cm, fill=gray!40, text centered] (l0222) at (11.4,5.5)
{  \footnotesize \bfseries \textit{?}};


\end{tikzpicture}
\end{center}

But regardless of the final textual form a license is using to express its
granting or revoking positions, the reader has to consider some aspects:

\begin{itemize}
  
  \item Overall, one has to keep in mind that of course no licensor, contributor
  and/or distributor can release the right to use any patents he does not own --
  even not if he tries to release them by an open source patent clause.
  Implictly touched patents of third parties not having contributed to the
  development and/or participated in the distribution can never be implicitly
  and automatically released on the base of such an (open source) patent clause:
  no rights, no right to release\footnote{This is an important aspect which is
  sometimes not considered by programmers. Inside of DTAG we had a fruitful
  discussion evoked by Mr. Stephan Altmeyer who -- as patent lawyer -- patiently
  explained this constraint to us.}. Hence: even for those open source licenses
  which try to protect the users, finally the user itself must nevertheless
  ensure that he does not violate the patents of third parties being unwillingly
  touched by the way the code works\footnote{Sometimes, this problem of
  willingly or unwillingly violated third party patents is seen as a weakness of
  open source software. But that is not true. It is a weakness of every
  software. Even a commercial licensor (developer) has only the right to license
  the use of those patents he really owns or he has 'bought' for relicensing
  them. Moreover, also commercial licensors can willingly or unwillingly violate
  patents of other persons.}.
  
  \item In the context of a granting license, one has also to consider that
  contributing to and distributing of a piece of software implictly evokes that
  all patents of the contributor and/or distributor are 'given free' which are
  necessary to use the software as whole -- including the more or less deeply
  embedded libraries. So, if one wants to check wether some of the core patents
  of one's patent portfolio are affected (and wether one therefore better should
  not use / distribute this piece of open source software), one should not
  forget to check the embedded libraries, too.
  
  \item Finally, one has to consider in the context of a granting license that
  its patent clause only releases the use of the patents in the meaning 'allowed
  to be used for enabling the use of the distributed software'. The patent
  clause does not genreally release the use of the patents. Thus, the threat of
  (unwillingly) releasing patents by open source software is not as large as
  sometimes feared: granted is only the combination of the use of the patent and
  the software. On the one hand, you may not use the software without having the
  right to use the patent because the use of the patent is inherently necessary
  for using use the software. On the other hand, you are not allowed to use the
  patent without the software because the patent clause only refers to the use
  of that specific open source software licensed by the corresponding open
  source license.

  \item Summarized, one has to consider that the granting open source licenses
  automatically and implicitly enforce you to grant all the rights to the user
  of the software which are necessary to use the software legally. Open source
  contributors and distributors should know that\footnote{Again: It might be
  debatable wether this is also valid for the distributors which do not
  contribute anything to the development. From the viewpoint of an open source
  user who wants only to have one reliable way one can assume that there is no
  difference.}.

  \item With respect to the revoking licenses, one has to consider that it
  clearly contains a negative condition in form of an interdiction. We will
  integrate these conditions into a 'prohibits'-section of the to-do lists.
  
  \item Finally one should mention, that the form of the revocation used in the
  revoking license in some cases refers to the use of the software, in other
  cases to the use of the patents. But one has to consider that -- from the
  pragmatic viewpoint of a benevolent open source software user -- also this
  second case of patent revocation implictly terminates the right to use the
  software: If the use of a patent is necessary to use a piece of software
  legally, one is not allowed to use the software without having the right to
  use the patent, too; and if the use of the patent is not necessary for using
  the software, the patent is not covered by the patent clause. So, in any case,
  this kind of patent clauses seems to terminate the right to use / to
  distribute and/or to modify the software. Hence, single users as well as
  companies or organizations should also respect such patent clauses if they
  want to be sure to use open source software compliantly.
\end{itemize}

The OSLiC wants to support its readers not only to act according to the licenses
in general, but also according to its patent clause. Thus, we now briefly cite
and summarize the meaning of particular patent clauses:

\subsection{Patent clause of the ApL}

\subsection{Patent clause of the ApL}

\subsection{Patent clause of the BSD Licenses}

\subsection{Patent clause of the EPL}

\subsection{Patent clause of the EUPL}

\subsection{Patent clause of the GPL}

\subsection{Patent clause of the LGPL}
 
\subsection{Patent clause of the MIT License}

\subsection{Patent clause of the MPL}         

\subsection{Patent clause of the MS-PL}
  
\subsection{Patent clause of the PgL}

\subsection{Patent clause of the PHP License}



% \bibliography{../../../bibfiles/oscResourcesEn}
