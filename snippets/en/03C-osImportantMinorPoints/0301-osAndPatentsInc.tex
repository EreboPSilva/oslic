% Telekom osCompendium 'for being included' snippet template
%
% (c) Karsten Reincke, Deutsche Telekom AG, Darmstadt 2011
%
% This LaTeX-File is licensed under the Creative Commons Attribution-ShareAlike
% 3.0 Germany License (http://creativecommons.org/licenses/by-sa/3.0/de/): Feel
% free 'to share (to copy, distribute and transmit)' or 'to remix (to adapt)'
% it, if you '... distribute the resulting work under the same or similar
% license to this one' and if you respect how 'you must attribute the work in
% the manner specified by the author ...':
%
% In an internet based reuse please link the reused parts to www.telekom.com and
% mention the original authors and Deutsche Telekom AG in a suitable manner. In
% a paper-like reuse please insert a short hint to www.telekom.com and to the
% original authors and Deutsche Telekom AG into your preface. For normal
% quotations please use the scientific standard to cite.
%
% [ Framework derived from 'mind your Scholar Research Framework' 
%   mycsrf (c) K. Reincke 2012 CC BY 3.0  http://mycsrf.fodina.de/ ]
%


%% use all entries of the bibliography
%\nocite{*}

\section{The problem of implicitly releasing patents}
\footnotesize \begin{quote}\itshape In this chapter, we are briefly analyzing
the effect of patent clauses in open source licenses -- not in general, but with
respect to the license fulfilling tasks they require, also known as the
'implicit acceptance of a patent use' by distributing open source software.
\end{quote}
\normalsize

At least the free software movement frowns on the existence of software
patents\footnote{For an early and elaborated description on the effects of
software patents based on the viewpoint of the free software movement
\cite[see][\nopage wp]{Stallman2001a}. This lecture seems to have been given
more than once and printed later on (\cite[cf.][\nopage wp]{Stallman2002a}).
Within the first decade of 2000, the focus switched to a more political fight
against software patents (\cite[cf.][\nopage wp]{Stallman2004a}). But recently
there seems to have been appeared another turn in dealing with software patents:
Not fighting against the patents, but mitigating their effects: The proposal is
'[...] (to legislate) that developing, distributing, or running a program on
generally used computing hardware does not constitute patent infringement'
(\cite[cf.][\nopage wp]{Stallman2012a})}. One of the most known witnesses for
that attitude is the GPL itself. Its preamble purports that \enquote{[\ldots]
any free program is threatened constantly by software
patents}\footcite[cf.][wp]{Gpl20OsiLicense1991a}. One can read that the open
source community fears three risks: First, they are apprehensive of people who
hjack the idea of a piece of open source software they do not have developed,
register a corresponding patent, and finally try to earn money by preventing the
use of the software or by envolving its users into patent
ligitations\footcite[cf.][234]{JaeMet2011a}. Second, they fear a bramble of
general software patents which practically prohibits to develop open source
software legally\footcite[cf.][234]{JaeMet2011a}. Third, they anticipate the
possibility that (not quite benevolent) open source developers could try to
register patents for undermining the open source
principles\footcite[cf.][235]{JaeMet2011a}.

Howsoever, regardless wether one tries to fight against software patents or not,
software patents have come true. To act law-abidingly requires to manage the
constraints of patents properly. Open source licenses know and respect this
necessity. Moreover, at least some of them try to manage the effect of software
patents by specific patent clauses\footcite[pars pro toto cf.][\nopage wp.
§3]{Apl20OsiLicense2004a} or by several sentences distributed in the license
text\footcite[pars pro toto cf.][\nopage wp.]{Epl10OsiLicense2005a}. But why
does the OSLiC have to deal with this topic, if the OSLiC does not want to
participate in general discussions?

In opposite to the other conditions of the open source licenses, their patent
clauses or propositions in general do not directly refer to a specific set of
actions which has to be executed for acting in accordance to the licenses. Open
source patent clauses normally do not join in the game 'paying by doing'. So,
actually, it does not seem to be not necessary to mention the patent clauses,
here.

Unfortunately, although the patent clauses do not directly say \emph{'do this or
that in these or those circumstances'}, some of them nevertheless trigger side
effects which evoke that the distributors of open source software implictly
having already something done if they actually are distributing a piece of open
source software. This implicit effect makes it necessary to deal with the patent
clauses even in an only pragmatic OSLIC.

Patent clauses in open source licenses can have two different directions of
impact. They use two methods to protect the users of the open source software --
and sometimes these methods are combined:

\begin{itemize}
  \item First, an open source license can assure that all contributors /
  distributors to / of a piece of open source software grant to all users /
  recipients not only the right to use the open source software itself, but
  automatically and implicitly also the right to use all those patents --
  belonging to the contributors / distributors -- which as patents are necessary
  to use the software legally\footnote{There might arise a legal discussion
  whether a distributor who does not contribute to the software development
  really imlictly also has to grant the necessary rights of his patent
  portfolio. The OSLiC doesn't want to participate in this discussion. We take a
  simple and pragmatic position: for being sure that you are acting according to
  an open source license with such a patent clause you should simply assume that
  you have to do so. If this default position is not reasonable for you it might
  be a good idea to consult legal experts which -- perhaps -- may find another
  way for you to use the software legally.}. So, let us - a little simplifying
  and therefore only on the following few pages -- name such licenses the
  \emph{granting licenses}.
  \item Second, an open source license can try to automatically terminate the
  right to use, to modify , and to distribute the software if its user litigates
  against any of the contributors / distributors with respect to a software
  patent. That can be seen as a revocation of earlier granted rights. So, let us
  name these license the \emph{revoking licenses}.
\end{itemize}

Later on, we will summarize the concrete patent clauses of all the licenses
discussed in the OSLiC as a proof for the following classification:

% \begin{minipage}
% \end{minipage}
\begin{small}

\begin{center}
\begin{tikzpicture}
\label{LICTAX}


\node[ellipse,minimum height=8.5cm,minimum width=14cm,draw,fill=gray!10] (l0100) at (6.8,6.8)
{  };

\draw [-,dotted,line width=0pt,white,
    decoration={text along path,
              text align={center},
              text={|\itshape|open source licenses}},
              postaction={decorate}] (-0.8,6.5) arc (218:322:9.5cm);
              
\node[ellipse,minimum height=6.2cm,minimum width=4.4cm,draw,fill=gray!20] (l0100) at (2.6,6.8)
{  };

\draw [-,dotted,line width=0pt,white,
    decoration={text along path,
              text align={center},
              text={|\itshape| without patent clauses}},
              postaction={decorate}] (0.9,7.4) arc (180:0:1.8cm);

\node[circle,draw,text width=1cm, fill=gray!40, text centered] (l0101) at (2,8)
{  \footnotesize \bfseries \textit{?}};
\node[circle,draw,text width=1cm, fill=gray!40, text centered] (l0102) at (3.5,8)
{  \footnotesize \bfseries \textit{BSD}};
\node[circle,draw,text width=1cm, fill=gray!40, text centered] (l0103) at (2,6.5)
{  \footnotesize \bfseries \textit{MIT}};
\node[circle,draw,text width=1cm, fill=gray!40, text centered] (l0104) at (3.5,6.5)
{  \scriptsize \bfseries \textit{?}};
\node[circle,draw,text width=1cm, fill=gray!40, text centered] (l0105) at (2,5)
{  \footnotesize \bfseries \textit{?}};
\node[circle,draw,text width=1cm, fill=gray!40, text centered] (l0106) at (3.5,5)
{  \footnotesize \bfseries \textit{?}};

\node[ellipse,minimum height=6cm,minimum width=8.5cm,draw,fill=gray!20] (l0200) at (9.2,6.5)
{  };

\draw [-,dotted,line width=0pt,white,
    decoration={text along path,
              text align={center},
              text={|\itshape| with patent clauses}},
              postaction={decorate}] (7.5,8.5) arc (120:60:4cm);


\node[ellipse,minimum height=5cm,minimum width=8cm,draw,fill=gray!40] (l0220) at (9.2,6.2)
{  };
\node[ellipse,minimum height=4cm,minimum width=4.5cm,draw,fill=gray!30] (l0210) at (7.8,6.2)
{  };



\draw [-,dotted,line width=0pt,white,
    decoration={text along path,
              text align={center},
              text={|\itshape| granting + revoking}},
              postaction={decorate}] (4.7,4.7) arc (180:0:3cm);

\node[ellipse,draw,text width=1cm, fill=gray!40, text centered] (l0211) at (6.7,6.5)
{  \footnotesize \bfseries \textit{EpL}};
\node[ellipse,draw,text width=1cm, fill=gray!40, text centered] (l0212) at (8.8,6.7)
{  \footnotesize \bfseries \textit{ApL}};
\node[ellipse,draw,text width=1cm, fill=gray!40, text centered] (l0213) at (6.7,5.5)
{  \footnotesize \bfseries \textit{?}};
\node[ellipse,draw,text width=1.1cm, fill=gray!40, text centered] (l0214) at (8.4,5)
{  \scriptsize \bfseries \textit{MS-PL}};

 
% line width=0pt,white,
\draw [-,dotted,line width=0pt,white,
    decoration={text along path,
              text align={center},
              text={|\itshape| granting}},
              postaction={decorate}] (8.5,6) arc (180:0:2cm);

\node[ellipse,draw,text width=1cm, fill=gray!40, text centered] (l0221) at (11.4,7)
{  \footnotesize \bfseries \textit{??}};
\node[ellipse,draw,text width=1cm, fill=gray!40, text centered] (l0222) at (11.4,5.5)
{  \footnotesize \bfseries \textit{??}};

\end{tikzpicture}
\end{center}

\end{small}


But regardless of the final textual form a license is using to express its
granting or revoking positions, in any case one has to consider some aspects: 

\begin{itemize}
  
  \item Overall, one has to keep in mind that of course no licensor, contributor
  and/or distributor can release the right to use any patents he does not own --
  even not if he \emph{tries} to release them by an open source patent
  clause\footnote{The EPL is one of the licenses which insists on this aspect:
  It the second half of its patent clause, the EPL underlines that
  \enquote{[\ldots] no assurances are provided by any Contributor that the
  Program does not infringe the patent or other intellectual property rights of
  any other entity}. Moreover, it explicitly adds that \enquote{[\ldots] if a
  third party patent license is required to allow Recipient to distribute the
  Program, it is Recipient's responsibility to acquire that license before
  distributing the Program} (\cite[cf.][\nopage wp.
  §2c]{Epl10OsiLicense2005a}).}. Implictly touched patents of third parties not
  having contributed to the development and/or participated in the distribution
  can never be implicitly and automatically released on the base of such an
  (open source) patent clause: no rights, no right to release\footnote{This is
  an important aspect which is sometimes not considered by programmers. Inside
  of DTAG we had a fruitful discussion evoked by Mr. Stephan Altmeyer who -- as
  patent lawyer -- patiently explained this constraint to us.}. Hence: even for
  those open source licenses which try to protect the users, finally the user
  itself must nevertheless ensure that he does not violate the patents of third
  parties being unwillingly touched by the way the code
  works\footnote{Sometimes, this problem of willingly or unwillingly violated
  third party patents is seen as a weakness of open source software. But that is
  not true. It is a weakness of every software. Even a commercial licensor
  (developer) has only the right to license the use of those patents he really
  owns or he has 'bought' for relicensing them. Moreover, also commercial
  licensors can willingly or unwillingly violate patents of other persons.}.
  
  \item In the context of a granting license, one has also to consider that
  contributing to and distributing of a piece of software implictly evokes that
  all patents of the contributor and/or distributor are 'given free' which are
  necessary to use the software as whole -- including the more or less deeply
  embedded libraries. So, if one wants to check wether some of the core patents
  of one's patent portfolio are afflicted by a patent clauses (and wether one
  therefore better should not use / distribute the corresponding piece of open
  source software), one should not forget to check the embedded libraries, too.
  
  \item Finally, one has to consider in the context of a granting license that
  its patent clause only releases the use of the patents in the meaning of
  'allowed to be used for enabling the use of the distributed software'. The
  patent clause does not release the patents generally. Thus, the threat of
  (unwillingly) releasing patents by open source software is not as large as
  sometimes feared: the use of the patent is only granted in combination with
  the software. On the one hand, you may not use the software without having the
  right to use the patent because the use of the patent is inherently necessary
  for using use the software. On the other hand, you are not allowed to use the
  patent without the software because the patent clause only refers to the use
  of that specific open source software licensed by the corresponding open
  source license.

  \item Summarized, one has to consider that the granting open source licenses
  automatically and implicitly enforce you to grant all the rights which are
  necessary to use the software legally. Open source contributors and
  distributors should know that\footnote{Again: It might be debatable wether
  this is also valid for the distributors which do not contribute anything to
  the development. That's a legal discussion the OSLiC do no wish to participate
  in. From the viewpoint of an open source user who only wants to have one
  reliable and secure way to use open sopurce software compliantly, one should
  perhaps assume that there is no difference.}.

  \item With respect to the revoking licenses, one has to consider that their
  patent clauses contain a negative conditions which may be read as
  interdictions. The OSLiC will integrate these conditions into specific
  'prohibits'-sections of its to-do lists.
  
  \item Finally one should mention that in some cases, the form of the
  revocation used by the revoking license refers to the use of the software, in
  other cases to the use of the patents. But nevertheless, one can reason that
  -- from the pragmatic viewpoint of a benevolent open source software user --
  also this second case of patent revocation implictly terminates the right to
  use the software: If the use of a patent is necessary to use a piece of
  software legally, one is not allowed to use the software without having the
  right to use the patent, too; and if the use of the patent is not necessary
  for using the software, then the patent is not covered by the patent clause.
  So, in any case, this kind of patent clauses seems to terminate the right to
  use / to distribute and/or to modify the software. Hence, single users as well
  as companies or organizations should also respect such patent clauses if they
  want to be sure to use open source software compliantly.
\end{itemize}

The OSLiC wants to support its readers not only to act according to the licenses
in general, but also according to its patent clause. Thus, we now briefly cite
and summarize the meaning of particular patent clauses:

\subsection{ApL statements concerning patents}\label{subsec:ApLPatentClause}

Titled by the headline \enquote{Grant of Patent License}, the Apache License 2.0
contains a specific patent clause being comprised of two very long and condensed
sentences\footcite[cf.][\nopage wp. §3]{Apl20OsiLicense2004a}. Outside of this
patent clause, the word \emph{patent} is only used once again -- for requiring
that one \enquote{[\ldots] must retain, in the (sources) [\ldots] all [\ldots]
patent [\ldots] notices [\ldots]}\footcite[cf.][\nopage wp.
§4.3]{Apl20OsiLicense2004a}.

The one core message of the ApL patent clause is the statement that
\enquote{[\ldots] each Contributor hereby grants to You a perpetual, worldwide,
non-exclusive, no-charge, royalty-free, irrevocable [\ldots] patent license to
make, have made, use, offer to sell, sell, import, and otherwise transfer the
Work [\ldots]}\footcite[cf.][\nopage wp. §3. \enquote{Contributor},
\enquote{Work} and \enquote{You} are defined §1: \emph{Contributor} refers to
the original licensor and to all others whose contributions have been
incorporated into the Work. The \emph{Work} denotes the result of the
development process regardless of its form. \emph{You} denote the
licensees.]{Apl20OsiLicense2004a}.

The second core message of the ApL patent clause is the statement that
\enquote{if You institute patent litigation against any entity [\ldots] alleging
that the Work [\ldots] constitutes [\ldots] patent infringement, then any patent
licenses granted to You [\ldots] shall terminate [\ldots]}\footcite[cf.][\nopage
wp. §3]{Apl20OsiLicense2004a}.

The third message of the ApL patent clause is the statement, that the
\enquote{[\ldots] license applies only to those patent claims licensable by such
Contributor that are necessarily infringed by their Contribution(s) alone or by
combination of their Contribution(s) with the Work to which such Contribution(s)
was submitted}\footcite[cf.][\nopage wp. §3]{Apl20OsiLicense2004a}.

Thus, the ApL is - as we are using to say in this chapter - a granting and a
revoking license: At first you are granted to use all patents of all
contributors which are necessary to use the software legally. But if you -- with
respect to the software -- install any litigation concerning an infringement of
patents, then the rights granted to you are revoked.


\subsection{EPL statements concerning patents}\label{subsec:EpLPatentClause}

The Eclipse Public License treats the patents being necessary to use the program
in the same section and under the same headline \enquote{Grant of Rights} like
all the other rights: First, the EPL clearly states that \enquote{[\ldots] each
Contributor [\ldots] grants (the recipient) a non-exclusive, worldwide,
royalty-free patent license under Licensed Patents to make, use, sell, offer to
sell, import and otherwise transfer the Contribution of such Contributor, if
any, in source code and object code form}\footcite[cf.][\nopage wp.
§2.b]{Epl10OsiLicense2005a}. Then the EPL delimits the extend of this act of
granting: Neither hardware patents of the contributors are covered by this
releasing patent clause, nor patents that concern aspects out of the area of the
initially intended software combination\footcite[cf.][\nopage wp.
§2.b]{Epl10OsiLicense2005a}. Finally, the EPL hints to the general fact that
3rd. party patents not belonging to the contributors can never be implicity be
released by such a patent clause. Moreover, it alleges the example that
\enquote{[\ldots] if a third party patent license is required to allow Recipient
to distribute the Program, it is Recipient's responsibility to acquire that
license before distributing the Program}\footcite[cf.][\nopage wp.
§2.c]{Epl10OsiLicense2005a}.

Like other open source licenses, also the EPL announces at its end that
\enquote{if (a) Recipient institutes patent litigation against any entity
[\ldots] alleging that the Program [\ldots] infringes such Recipient's
patent(s), then such (granted) Recipient's rights [\ldots] shall terminate
[\ldots]}\footcite[cf.][\nopage wp. §7]{Epl10OsiLicense2005a}.

Thus, also the EpL is - as we are using to say in this chapter - a granting and
a revoking license: At first you are granted to use all patents of all
contributors which are necessary to use the software legally. But if you -- with
respect to the software -- install any litigation concerning an infringement of
patents, then the rights granted to you are revoked.


\subsection{EUPL statements concerning patents [tbd]}

\subsection{GPL statements concerning patents [tbd]}

\subsection{LGPL statements concerning patents [tbd]}
 
\subsection{MPL statements concerning patents [tbd]}         

\subsection{MS-PL statements concerning patents}\label{subsec:MsplPatentClause}

First, the MS-PL contains a statement, by which \enquote{[\ldots] each
contributor grants (the software users) a non-exclusive, worldwide, royalty-free
license under its licensed patents to make, have made, use, sell, offer for
sale, import, and/or otherwise dispose of its contribution in the software or
derivative works of the contribution in the software}\footcite[cf.][\nopage wp.
§2.B]{MsplOsiLicense2013a}. Second, the MS-PL says that \enquote{if you bring a
patent claim against any contributor[\ldots] your patent license from such
contributor to the software ends automatically}\footcite[cf.][\nopage wp.
§3.B]{MsplOsiLicense2013a}.

Thus, the MS-PL is - as we are using to say in this chapter - a granting and a
revoking license: At first you are granted to use all patents of all
contributors which are necessary to use the software legally. But if you -- with
respect to the software -- install any litigation concerning an infringement of
patents, then the rights granted to you are revoked.




% \bibliography{../../../bibfiles/oscResourcesEn}
