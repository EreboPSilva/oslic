% Telekom osCompendium 'for being included' snippet template
%
% (c) Karsten Reincke, Deutsche Telekom AG, Darmstadt 2011
%
% This LaTeX-File is licensed under the Creative Commons Attribution-ShareAlike
% 3.0 Germany License (http://creativecommons.org/licenses/by-sa/3.0/de/): Feel
% free 'to share (to copy, distribute and transmit)' or 'to remix (to adapt)'
% it, if you '... distribute the resulting work under the same or similar
% license to this one' and if you respect how 'you must attribute the work in
% the manner specified by the author ...':
%
% In an internet based reuse please link the reused parts to www.telekom.com and
% mention the original authors and Deutsche Telekom AG in a suitable manner. In
% a paper-like reuse please insert a short hint to www.telekom.com and to the
% original authors and Deutsche Telekom AG into your preface. For normal
% quotations please use the scientific standard to cite.
%
% [ Framework derived from 'mind your Scholar Research Framework' 
%   mycsrf (c) K. Reincke 2012 CC BY 3.0  http://mycsrf.fodina.de/ ]
%


%% use all entries of the bibliography
%\nocite{*}

\section{Excursion: What is a 'Derivated Work' - the kernel of Open Source}
\footnotesize
\begin{quote}\itshape
We will shortly discuss existing attempts to define the derivated works of
technical aspects, like dynamical or statical linking or not. We will
prove that linking can not deliver a definite criteria: 1) modules are only
unzipped libraries. 2) you can distribute software as modules added by a script,
which statically(sic!) links all modules before executing the program. 3) The
criteria of pipe-communication is good, but not sufficient. 4) All these
attempts do not match the constituting features of script languages. Therefore we
will follow Moglen(?) and will argue from the viewpoint of a developer: it's
only a question of a function, method or anything else which calls (jumps into)
a piece of code which has been licensed by a license protecting
on-top-developments and you have a derivated work.
\end{quote}
\normalsize
\ldots


%\bibliography{../../../bibfiles/oscResourcesEn}
