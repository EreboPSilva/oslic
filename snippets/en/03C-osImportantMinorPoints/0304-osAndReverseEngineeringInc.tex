% Telekom osCompendium 'for being included' snippet template
%
% (c) Karsten Reincke, Deutsche Telekom AG, Darmstadt 2011
%
% This LaTeX-File is licensed under the Creative Commons Attribution-ShareAlike
% 3.0 Germany License (http://creativecommons.org/licenses/by-sa/3.0/de/): Feel
% free 'to share (to copy, distribute and transmit)' or 'to remix (to adapt)'
% it, if you '... distribute the resulting work under the same or similar
% license to this one' and if you respect how 'you must attribute the work in
% the manner specified by the author ...':
%
% In an internet based reuse please link the reused parts to www.telekom.com and
% mention the original authors and Deutsche Telekom AG in a suitable manner. In
% a paper-like reuse please insert a short hint to www.telekom.com and to the
% original authors and Deutsche Telekom AG into your preface. For normal
% quotations please use the scientific standard to cite.
%
% [ Framework derived from 'mind your Scholar Research Framework' 
%   mycsrf (c) K. Reincke 2012 CC BY 3.0  http://mycsrf.fodina.de/ ]
%


%% use all entries of the bibliography
%\nocite{*}

\section{Excursion: Reverse Engineering and Open Source}

Reverse engineering is a special challenge. If one gets a program in form
of a binary its license must allow to decompile this binary. Otherwise it is
forbidden. The right to decompile the program is a right like the other rights
which must escpecially be licensed -- like the right to use it, to study it, to
modify it, or to distribute it.

Generally, closed software is forwarded with the intention not to give the
recipients access to the source code. Therefore closed software is mostly
delivered in form of binaries. Hence, in case of closed software it is mostly
not allowed to decompile the distributed binaries.

As opposed to this, normally open source software is not affected by the problem
of reverse engineering - at least if the source code of the software is also
accessible: Licenses with strong or weak copyleft require that a distributed
binary must be accompanied either by the corresponding source code itself or by
an offer to get it. Permissive licenses allow also to distribute the software
only in form of binaries. If one wants to decompile a program licensed under
such a permissive license this license must explictly or imlicitly also grant
the right to decompile the binary.

If one scans the texts of the licenses analysed by the OSLiC one has
surprisingly to consider that (a) none of the licenses is talking about
decompiling and that (b) only the LGPL-2.1 and the LGPL-3.0 deal with the
terminus 'reverse engineering'. In the LGPL-2.1, one can read

\begin{quote}\emph{
\enquote{As an exception to the Sections above, you may also combine or
link a \enquote{work that uses the Library} with the Library to produce a work
containing portions of the Library, and distribute that work under terms of your
choice, \emph{provided that the terms permit} modification of the work for the
customer's own use and \emph{reverse engineering} for debugging such
modifications.}\footcite[cf.][\nopage wp]{Lgpl21OsiLicense1999a}}
\end{quote}


And in the LGPL-3.0, one can find the sentence

\begin{quote}\emph{
\enquote{You may convey a Combined Work under terms of your choice that,
taken together, effectively \emph{do not restrict} modification of the portions
of the Library contained in the Combined Work and \emph{reverse engineering} for
debugging such modifications, if you also do each of the following
[\ldots]}\footcite[cf.][\nopage wp]{Lgpl30OsiLicense2007a}}
\end{quote}

Based on these predications one can find the statement that those developers who
compliantly integrate an LGPL licensed library as component into an
'overarching' program, must also grant to all recipients of this program the
right to decompile the overarching program. At least one German publication
explicitly says that the LGPL-2.1 generally requires that a distributor of a
program which accesses the LGPL-2.1 licensed library, must grant his customer
also the right to modify the accessing program and hence to execute a reverse
engineering\footnote{Originally the German text says: \enquote{Ziffer 6 LGPLv2.1
knüpft die Erlaubnis, das zugreifende Programm unter beliebigen
Lizenzbestimmungen verbreiten zu drüfen, an eine Reihe von Verpflichtungen, die
in der Praxis oft übersehen werden: Zunächst muss dem Kunden, dem die Software
geliefert wird, die Veränderung des zugreifenden Programms gestattet werden und
zu diesem Zweck auch ein Reverse Engineering zur Fehlerbehebung. Dies dürfte
alle Formen des Debugging und das Dekompilieren des zugreifenden Programms
umfassen.}\cite[cf.][81]{JaeMet2011a} }. In other words: it is sometimes argued
that the LGPL allows the reverse engineering of all works that use the
library\footnote{In general, the situation seems to be not as clear as possible.
For example, an important American author states, that the sections 5 and 6 of
the LGPL \enquote{[\ldots] are an impenetrable maze of technological babble} and
that they \enquote{[\ldots] should not be in a general-purpose software
license}\cite[cf.][124]{Rosen2005a}. And he does not discuss the granting of
the LGPL to execute a reverse engineering.}.

Despite of these substantial expressions and despite of the reputation of all
these witnesses, the OSLiC wants to show, that the LGPL actually requires to
allow a reverse engineering only in case of distributing a statically linked
program. For accepting this conclusion, one has to follow the sentences of the
LGPL-2.1 license very strictly:

First of all, the LGPL-2.1 distinguishes between a \enquote{work based the
Library} and a \enquote{work that uses the Library}. And a \enquote{Library} is
defined as \enquote{a collection of software functions and/or data prepared so
as to be conveniently linked with application programs (which use some of those
functions and data) to form executables}\footcite[cf.][\nopage
wp §0]{Lgpl21OsiLicense1999a}. This definition already contains an important
stipulation: neither the library itself (which delivers functions etc.) nor the
program (which wants to use the delivered elements) is an executable.
Executables are formed by linking the library and the program. Based on this
viewpoint, the LGPL-2.1 additionally determines that 

\begin{quote}\emph{\enquote{
A \enquote{work based on the Library} means either the Library or any derivative
work under copyright law: that is to say, a work containing the Library or a portion
of it, either verbatim or with modifications and/or translated straightforwardly
into another language.}\footcite[cf.][\nopage wp §0]{Lgpl21OsiLicense1999a} }
\end{quote}

Hence, in the wording of the LGPL-2.1, you have created a \enquote{derivative
work} of the library whenever you have expanded or reduced \enquote{the
collection of software functions and/or data}, whenever you have
\enquote{modified} some of its \enquote{functions and/or data}, and whenever you
have copied \enquote{portions} of the library into another work. And this
process of generating a derivative work does not depend on the format, neither
on that of the library nor on that of the other work.\footcite[cf.][\nopage wp
§0]{Lgpl21OsiLicense1999a}.

But the LGPL-2.1 knows that there nevertheless might exist software which - in
this sense - is not a derivative work of the Library. If such \enquote{a
program contains no derivative of any portion of the Library, but is
designed to work with the Library by being compiled or linked with it, [then
it] is called a \enquote{work that uses the Library}}\footcite[cf.][\nopage wp
§5]{Lgpl21OsiLicense1999a}.

Based on this distinction, the LGPL-2.1 can clearly assert, that \enquote{such
a work, in isolation, is not a derivative work of the Library, and therefore
falls outside the scope of this License}\footcite[cf.][\nopage wp
§5]{Lgpl21OsiLicense1999a}: if the isolated work is designed to work with the
library but still does not contain any derivative of any portion of the library,
it is not derived from the library.

From the view of a programmer -- especially from the view of a
C-programmer\footnote{When the GPL and the LGPL were designed, the programming
language C was the paradigm to generate executable software} -- this definition
causes a problem. Designing a program to use a library means including the
header files of the library. Such header files, which are developed and
delivered by the developer of the Library, may not contain declarations of
functions and data types, but also generally usable variants and inline
functions. Using these declarations is the common way to design a program to
work with a library. So lately such a designed program contains elements of the
library. So, a C-programmer might argue that designing a program to work with
the library let the program become a work based on the Library.

Thankfully, the LGPL-2.1 addresses this problem and generates a solution:

The LGPL-2.1 clearly states that the compiled version of a work that uses the
library -- in opposite to its source code version -- can indeed become a
derivative work of the library: 

\begin{quote}\emph{\enquote{ When a \enquote{work that uses the Library} uses
material from a header file that is part of the Library, the object code for the
work may be a derivative work of the Library even though the source code is
not.}\footcite[cf.][\nopage wp §5]{Lgpl21OsiLicense1999a} }
\end{quote}

From a viewpoint of a (C-) programmer this statement is adequate and clearly
comprehensible: the source code of the \enquote{work that uses the Library}
itself normally containes only pure include directives which refer to the names
of header files of the library. The act of compiling this source code can indeed 
enrich the work that uses the library by parts of the library: the preprocessor
will expand inline functions and data types (and therefore copy code of the
library into the \emph{work that therefore no longer only uses the Library}, the
compilation will add variables and references to variables into the object file,
and so on and so on. But if the include files only contain
fucntions declartions, then compiled onject code of the \enquote{work that uses
the Library} still does not contain elements of the library.

And finally, when both parts will be linked, the LGPL-2.1 regards the
result of the linking process indeed as a derivative work:
\begin{quote}\enquote{linking a \enquote{work that uses the Library} with the
Library creates an executable that is a derivative of the Library (because it
contains portions of the Library), rather than a \enquote{work that uses the
library}}\footcite[cf.][\nopage wp §5]{Lgpl21OsiLicense1999a}.
\end{quote}

So, the concrete status of the compiled version of the \enquote{work that uses
the Library} which still has not been linked to the library is vague. To give
clearness back to all users the LGPL defines that if the compiled but still
unlinked version of \enquote{work that uses the Library} indeed already contains
elements of the library, then is a derivative work of the libary. But if these
adopted elements are normal elements which are offered to design a work to work
with the library, then being a derivative work shall not have any effect. Or in
the words of the LGPL-2.1

\begin{quote}\enquote{ If such an object file [the compilation of the
\enquote{work that uses the Library}] uses only numerical parameters, data structure
layouts and accessors, and small macros and small inline functions (ten lines or
less in length), then the use of the object file is unrestricted, regardless of
whether it is legally a derivative work }\footcite[cf.][\nopage wp
§5]{Lgpl21OsiLicense1999a}.
\end{quote}

Hence, generally the LGPL-2.1 regards the \enquote{work that uses the Library}
as an independ unit which is not covered by the rules of the LGPL-2.1 license --
as long as both parts have not become an integrated entity as for example the
process of linking generates.

Based on these specifications one can clearly show that the LGPL-2.1 only
requires the permisson of reverse engineering in case of distributing a
statically linked integrated entity containing the \enquote{work that uses the
Library} and the library:

\begin{itemize}
\item First, anyone is allowed to distribute the LGPL-2.1 licensed library to
any third party. This follows from the freedom of free software.

\item Second, the copyright owners are allowed to distribute their \enquote{work
that uses the Library} to any third party, too. This follows from being the copyright
owner.

\item Third, if the copyright owner distribute their \enquote{work that
uses the Library} as object file which is not linked to the LGPL-Library, then
the LGPL does not oblige them to do anything. This follows from specicifation of
the LGPL under which conditions the use of the object file is unrestricted.

\item Hence, the recipient of the isolatedly distributed \enquote{work
that uses the Library} and the library can link these parts on his own machine
and under his own responsibility. [Follows from 1-3]
\end{itemize}

Now, we meet the question, whether the third party user which links the
seperatedly received parts on his own machine and on its behalf gets the right
to decompile the object file of the enquote{work that uses the Library} because it
is linked with the LGLP license library.

The answer is: No, he does not get the right to do so. The LGPL-2.1 clearly says
the right of reverse engineering is only given to third party, if 

%\bibliography{../../../bibfiles/oscResourcesEn}

% Local Variables:
% mode: latex
% fill-column: 80
% End:
