% Telekom osCompendium 'for being included' snippet template
%
% (c) Karsten Reincke, Deutsche Telekom AG, Darmstadt 2011
%
% This LaTeX-File is licensed under the Creative Commons Attribution-ShareAlike
% 3.0 Germany License (http://creativecommons.org/licenses/by-sa/3.0/de/): Feel
% free 'to share (to copy, distribute and transmit)' or 'to remix (to adapt)'
% it, if you '... distribute the resulting work under the same or similar
% license to this one' and if you respect how 'you must attribute the work in
% the manner specified by the author ...':
%
% In an internet based reuse please link the reused parts to www.telekom.com and
% mention the original authors and Deutsche Telekom AG in a suitable manner. In
% a paper-like reuse please insert a short hint to www.telekom.com and to the
% original authors and Deutsche Telekom AG into your preface. For normal
% quotations please use the scientific standard to cite.
%
% [ Framework derived from 'mind your Scholar Research Framework' 
%   mycsrf (c) K. Reincke 2012 CC BY 3.0  http://mycsrf.fodina.de/ ]
%

For simplifying our discussion let us now replace the meaningful terminal
phrases of our form by some logical variables:

\begin{description}
  \item[$\Gamma$] :- (you may *join a work that uses the Library with the
  Library to produce a work containing portions of the Library) 
  \item[$\Delta$] :- (you may distribute that work containing portions of the
  Library under terms of your choice)
  \item[$\Phi$] :- (the terms of your choice permit modification of the work 
  containing portions of the Library for the customer's own use)
  \item[$\Sigma$] :- (the terms of your choice permit reverse engineering for
  debugging modifications of the work containing portions of the Library)
  \item[$\Theta$] :- \emph{$\Gamma$ and $\Delta$}
  \item[$\Omega$] :- \emph{$\Phi$ and $\Sigma$}
\end{description}

Based on these definitions, we can syntactically reduce the
LGPL2-RefEng-Sentence to the formula \emph{$(\Gamma$ and $\Delta)$ provided that
$(\Phi$ and $\Sigma)$} or -- even shorter -- to \emph{$(\Theta$ provided that
$\Omega)$}.

Now, we have to clarify the meaning of the conjunction \emph{'provided that'}:

Obviously, \emph{provided that} means something like \emph{under the condition
that}. So, one might try to take this conjunction as another more stylish
version of the common \emph{if(\ldots)then(\ldots)}-formula, sometimes also
identified as a (logical) implication\footnote{Actually the logical implication
and the computational if-then-construct are not equivalent. Fortunately, we
later on can show, that in the context of this discussion the difference can be
ignored.}. Thus, we have to consider the process of sequencing the linguistic
form into a logical formula: if we indeed take the conjunction \emph{provided
that} as another form of the logical implication, it is not evident, which part
of the linguistic sentence must become the premise, and which the conclusion:
Does \emph{($\Theta$ provided that $\Omega$)} mean \emph{(if $\Theta$ then
$\Omega$)} or \emph{(if $\Omega$ then $\Theta$)}?

Apparently, \emph{provided that} wants to establish something like a
precondition. So, one might conclude that \emph{$(\Theta$ provided that
$\Omega)$} means \emph{(if $\Omega$ then $\Theta)$} or -- more logically notated
-- \emph{$((\Phi$ $\wedge$ $\Sigma)$ $\rightarrow$ $(\Gamma$ $\wedge$
$\Delta))$}. If this interpretation is adequate, it must of course fulfill the
intended purpose of the corresponding LGPL-v2-section, which wants to regulate
the distribution of works containing portions of LGPL libraries.

For facilitating the understanding of our argumentation, let us first check
whether this logical interpretation of the linguistic conjunction fits the
purpose of the LGPL -- by unfolding the slightly reduced version \emph{$(\Sigma$
$\rightarrow$ $\Delta)$} back to the corresponding verbal form:

\begin{quote}\noindent\emph{\textbf{if (} [\ldots] the terms permit reverse
engineering for debugging modifications of the work containing portions of the
Library, \textbf{) then (} [\ldots] you may distribute that work containing
portions of the Library under the terms of your choice.\textbf{)}}\end{quote}

Now we can better see the problem: An implication as a whole is false only if
the premise is true and the conclusion is false. In all other cases it is true.
Especially, it is true, if the premise is false: If the premise is false, then
the truth value of the conclusion does not matter in any sense. Thus, if we take
this implication as a rule, which shall determine our behaviour, then this
implication only supports us, if we already have decided to permit reverse
engineering. In this case the rule successfully tells us that we are allowed to
distribute the work containing portions of the Library. But from the converse
decision that we will not permit reverse engineering, follows nothing - because
a false premise does not influence the truth value of the conclusion.
Especially, the rule does not tell us that we may not distribute the work
containing portions of the Library. So -- from the viewpoint of the formal logic
-- this translation of the original conjunction \emph{'provided that'} says,
that if the terms of your own license do not permit reverse engineering for
debugging modifications of the work containing portions of the
Library\footnote{The premise is false.}, then \textbf{you may or may not}
distribute that work containing portions of the Library under the terms of your
choice\footnote{The truth value of the conlusion is undetermined by the rule.}.
Hence, we must state that this interpretation does not fulfill the purpose of
the LGPL-V2: if reverse engineering is not allowed, the distribution of the work
containing portions of the Library is not regulated. We have to conclude, that
this sequencing the LGPL2-RefEng-Sentence as a logical implication is wrong.

But we deduced this consequence from a slighty reduced form of the
LGPL2-RefEng-Sentence. Thus, we still have to ask, whether we have to derive
this conclusion also on the base of the completely unfolded formula
\emph{$((\Phi$ $\wedge$ $\Sigma)$ $\rightarrow$ $(\Gamma$ $\wedge$ $\Delta))$}?
The answer is yes: the premise \emph{$((\Phi$ $\wedge$ $\Sigma)$} contains a
logical conjunction. So the truth value of the whole premise depends on the
truth value of each of its terminal statements, particularly on that of the
statement $\Sigma$: If we decide not to permit reverse engineering, then the
premise as whole is false, regardless we forbid or allow modifications.
Consequently, the premise does not influence the truth value of the conclusion.
So, there is no way, to conclude that we have to allow or that we do not have to
allow reverse engineering. Hence we can transfer our result, deduced for the
slightly reduced formula to the unfolded complete formula: assuming that
\emph{$(\Theta$ provided that $\Omega)$} means \emph{(if $\Omega$ then
$\Theta)$} is wrong.

So, let us test the other combination. Let us ask, whether \emph{($\Theta$
provided that $\Omega$)} means \emph{(if $\Theta$ then $\Omega$)} or -- more
logically notated -- \emph{$((\Gamma$ $\wedge$ $\Delta)$ $\rightarrow$ $(\Phi$
$\wedge$ $\Sigma))$}. If we again for a moment focus on the reduced version
\emph{$(\Delta$ $\rightarrow$ $\Sigma)$} and dissolve our replacements, then we
get back the rule:

\begin{quote}\noindent\emph{\textbf{if (} [\ldots] you may distribute that work
containing portions of the Library under the terms of your choice, \textbf{)
then (} [\ldots] the terms permit reverse engineering for debugging
modifications of the work containing portions of the Library.
\textbf{)}}\end{quote}

Now we can see, that this version perfectly regulates the distribution of works
containing portions of LGPL libraries: If we are allowed to do so or -- in other
words: if we are compliantly distributing works containing portions of LGPL
libraries\footnote{The premise is true.}, then we have to permit reverse
engineering\footnote{The conclusion must be true, too!}. This follows from
applying \emph{Modus Ponens} to the implication\footnote{A true premise evokes a
true conclusion based on the given truth of the implication / rule itself.}. And
if we do not permit reverse engineering\footnote{The conclusion is false.}, then
we are not allowed to distribute works containing portions of LGPL
libraries\footnote{The premise must be false, too!}. This follows from applying
\emph{Modus Tollens} to the implication\footnote{A false conclusion evokes a
false premise based on the given truth of the implication / rule itself.}

But -- again -- we have to consider that we have deduced this consequence from a
slighty reduced version of our LGPL2-RefEng-Sentence. Thus, we still have to
show that our result also holds for the completely unfolded formula
\emph{$((\Gamma$ $\wedge$ $\Delta)$ $\rightarrow$ $(\Phi$ $\wedge$ $\Sigma))$}:
If we want to distribute works containing portions of the Library which have
been produced by joining the Library and the work using the
Library\footnote{Premise is true.}, then our terms must permit the modification
\emph{and} reverse engineering of the distributed product\footnote{Conclusion
must become true by Modus Ponens.}. And if we do not allow its modification
\emph{or} reverse engineering\footnote{Conclusion is false.}, then we do not
compliantly distribute works containing portions of the Library which have been
produced by joining the Library and the work using the Library\footnote{Premise
must become false by Modus Tollens.} Thus, we may generally state, that the
logical explication \emph{$((\Gamma$ $\wedge$ $\Delta)$ $\rightarrow$ $(\Phi$
$\wedge$ $\Sigma))$} perfectly regulates the distribution of works containing
portions of LGPL libraries.

Based on this clarification, we can reasonably replace the more stylish
conjunction \emph{'provided that'} by its more known equivalent
\emph{'implication'}\footnote{Here we can also see, that the difference between
the if-then-command as part of a procedural computer language and the logical
implication does not influence our results: In the context of a procedural
if-then-command the truth of the premise triggers the execution of the
conclusion. In our discussion, this aspect is totally covered by the Modus
Ponens derivation of the logical interpretation. And the Modus Tollens
derivation of the logical interpretation on the other side does not play any
role in a procedural if-then-command. So, it was the right decision to
understand the LGPL2-RefEng-Sentence logically and not as procedual command.},
which we indicate by the commonly used character for a logical implication, the
sign '\emph{$\rightarrow$}':

\begin{description}
  \item[\#]  $\Theta$ provided that $\Omega$
  \item[$\equiv$] $\Theta$ $\rightarrow$ $\Omega$
  \item[$\equiv$] ($\Phi$ $\wedge$ $\Sigma$) $\rightarrow$ ($\Gamma$ $\wedge$
  $\Delta$)
  \item[$\equiv$]
\begin{alltt}   
  ( ( [\(\Phi\)] you may 
       *join a work that uses the Library with the Library
       to produce a work containing portions of the Library )
  \(\wedge\)
  ( [\(\Sigma\)] you may 
        distribute that work containing portions of the 
        Library under terms of your choice 
) )
\(\rightarrow\)
( ( [\(\Gamma\)] the terms of your choice permit 
        modification of the work containing portions of 
        the Library for the customer's own use )
  \(\wedge\)
  ( [\(\Delta\)] the terms of your choice permit
        reverse engineering for debugging modifications 
        of the work containing portions of the Library   
) )
\end{alltt}
\end{description}

%% use all entries of the bibliography
%\nocite{*}

