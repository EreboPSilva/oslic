% Telekom osCompendium 'for being included' snippet template
%
% (c) Karsten Reincke, Deutsche Telekom AG, Darmstadt 2011
%
% This LaTeX-File is licensed under the Creative Commons Attribution-ShareAlike
% 3.0 Germany License (http://creativecommons.org/licenses/by-sa/3.0/de/): Feel
% free 'to share (to copy, distribute and transmit)' or 'to remix (to adapt)'
% it, if you '... distribute the resulting work under the same or similar
% license to this one' and if you respect how 'you must attribute the work in
% the manner specified by the author ...':
%
% In an internet based reuse please link the reused parts to www.telekom.com and
% mention the original authors and Deutsche Telekom AG in a suitable manner. In
% a paper-like reuse please insert a short hint to www.telekom.com and to the
% original authors and Deutsche Telekom AG into your preface. For normal
% quotations please use the scientific standard to cite.
%
% [ Framework derived from 'mind your Scholar Research Framework' 
%   mycsrf (c) K. Reincke 2012 CC BY 3.0  http://mycsrf.fodina.de/ ]
%

We can now simplify this formula once more by considering some empirical facts
and explicating some underlying understandings:

The first sentence $\Phi$ explains that the \emph{work that uses the Library}
and the used \emph{Library} itself together are joined and thereby transformed
into a \emph{work containing portions of the Library}. So, formally, one might
ask, whether this newly generated \emph{work containing portions of the Library}
also still \emph{uses the Library}?

Unfortunately, it is empirically possible, that such a process for combining the
two components could (a) copy all original portions of the library into a
something like a 'dead end section' of the program where they are never excuted,
and could (b) replace all original portions of the library by functionally
equivalent portions of any other library. Thus, the resulting \emph{work
containing portions of the Library} would indeed still contain portions of the
Library, although it would not use it any longer. And because of this
possibility, we are not allowed to say, that every work containing portions of a
library also uses the library\footnote{\ldots even if we think that this is a
really silly way to organize the joining process!}.

But, fortunately, the normal computational process of \emph{combining and
linking a work that uses the Library with the Library to produce a work
containing portions of the Library} inherently preserves the utilization of the
joined library: It is the general purpose of a software library to offer
functions and/or data (structures) for really being used by applications. And
vice versa, software developers refer to a specific library because they prefer
its service: They use readily prepared libraries (or classes or anything else)
because they want to simplify their own work while they conserve the quality
level of their work. Thus, they chose a library based on the assertion, that the
standard compiling and linking process guarantees, that indeed the chosen
library is used (and not secretly substituted by a mysterious 'equivalent').
With respect to this praxis of programming we are allowed to say that a
\emph{work containing portions of the Library} which has been \textbf{built by
the normal development processes} of combining, compiling, and linking source
and object files, indeed also uses the intended library.

Now, we are able to consider an empirical correlation between the first sentence
$\Phi$ and the second sentence $\Sigma$:

It seems to be evident, that we must already have done $\Phi$, in other words:
that we must already have \emph{*joined -- respectively: combined or linked -- a
work that uses the Library with the Library to produce a work containing
portions of the Library}, if we are going to compliantly \emph{distribute that
work containing portions of the Library under terms of your choice}. Or briefly
spoken: It seems to be conclusive that $\Sigma$ \emph{\textbf{empirically}
implies} $\Phi$\footnote{but not vice versa.}.

But is this conclusion correct? Let us check this statement by assuming the
opposite: If the contrary was true, there had to exist a \emph{work containing
portions of the Library} which had been gained without having linked or combined
the work and the Library in any sense. But from the inference above we already
know that \emph{works containing portions of the Library}, which have been
produced by the standard computational processes of \emph{combining and linking
a work that uses the Library with the Library}, indeed also \emph{use the
Library}. Thus, it would be self-contradictory to talk about a \emph{work
containing portions of the Library}, which was produced by the standard
combining and linking processes, and similarily to state, that exactly this work
is not combined with the library in any sense. And from a proof by contradiction
we may infer the truth of the logical opposite:

So, with respect to the meaning of \emph{being standardly combined or linked
with}, we may now say, that
\begin{itemize}
  \item it is necessarily true that a computional work, which is standardly
  produced on the base of \emph{a work that uses the Library} and \emph{the
  Library} and which therefore literally contains more or less
  \emph{portions of a library}, indeed uses the \emph{the Library} and \emph{is}
  therefore \emph{combined with the library}.
  \item  $\Sigma$\footnote{distributing \emph{a work that uses the Library and
  contains portions of a library}} empirically implies $\Phi$\footnote{A work
  that uses the Library has been *joined with the Library to produce a work
  containing portions of the Library} (in the standardized world of software
  development), because $\Phi$ must ever have been executed when $\Sigma$ is
  going to be realized.
\end{itemize}

Thus, we can now reduce the LGPL2-RefEng-Sentence to its' real core, the
LGPL2-RefEng-Rule:

\begin{quote}
\begin{alltt}   
(   [\(\Sigma\)] you may
        distribute (a) work containing portions of the 
        Library\(\footnote{which previously has been prepared for being distributed by standardly combining and
linking the work that uses the Library with the Library in a way that this prepared work indeed
also uses the Library}\) under terms of your choice )   
\(\rightarrow\)
( ( [\(\Gamma\)] the terms of your choice permit 
        modification of the work containing portions of 
        the Library for the customer's own use )
  \(\wedge\)
  ( [\(\Delta\)] the terms of your choice permit
        reverse engineering for debugging modifications 
        of the work containing portions of the Library   
) )
\end{alltt}
\end{quote}

This is indeed the essence of the LGPL2-RefEng-Sentence. It logically explains
us that we have to \emph{allow reverse engineering} and modification of a
\emph{work containing portions of the Library} if we distribute it (Modus
Ponens) and that we are \emph{not allowed to distribute a work containing
portions of the Library}, if we do \emph{not allow} its modification or
\emph{reverse engineering} (Modus Tollens).

Thus, for applying this rule correctly, we now only must know whether a work
indeed contains portions of the Library or not.

%% use all entries of the bibliography
%\nocite{*}

