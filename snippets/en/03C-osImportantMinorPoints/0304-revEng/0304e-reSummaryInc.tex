% Telekom osCompendium 'for being included' snippet template
%
% (c) Karsten Reincke, Deutsche Telekom AG, Darmstadt 2011
%
% This LaTeX-File is licensed under the Creative Commons Attribution-ShareAlike
% 3.0 Germany License (http://creativecommons.org/licenses/by-sa/3.0/de/): Feel
% free 'to share (to copy, distribute and transmit)' or 'to remix (to adapt)'
% it, if you '... distribute the resulting work under the same or similar
% license to this one' and if you respect how 'you must attribute the work in
% the manner specified by the author ...':
%
% In an internet based reuse please link the reused parts to www.telekom.com and
% mention the original authors and Deutsche Telekom AG in a suitable manner. In
% a paper-like reuse please insert a short hint to www.telekom.com and to the
% original authors and Deutsche Telekom AG into your preface. For normal
% quotations please use the scientific standard to cite.
%
% [ Framework derived from 'mind your Scholar Research Framework' 
%   mycsrf (c) K. Reincke 2012 CC BY 3.0  http://mycsrf.fodina.de/ ]
%

So, finally we can compile all our results into one single result:

\begin{itemize}
  \item \emph{With respect to any open source Library\footnote{$\rightarrow$ p.
  \pageref{RevEngOslicOsLisences}}, you are not required to allow reverse
  engineering, if you [A] develop your work using the Library, on the base of a
  standard version of the Library containing the interfaces as the original
  developers have designed it, if you [B] compile your work using this Library,
  as a discret (set of) dynamically linkable or combinable file(s), if you [C]
  use only the standard compilation methods which preserve the upstream approved
  interfaces\footnote{and which therefore do not to exceed limits, prescribed by
  the owners of the Library}, and if you [D] distribute the produced unlinked
  object code or bytecode files before they are linked as an executable.}
  \item \emph{In all other cases of distributing your work using such a Library,
  you are probably required to allow reverse engineering of your work. By all
  means, you have at least to fear that you are implictly allowing reverse
  engineering of your work using such a Library -- especially, \ldots}
  \begin{itemize}
    \item \emph{if you distribute the work using the Library and the Library
    together as a statically linked program or as an integrated package
    containing both parts, the work using the library and the Library
    itself\footnote{This holds also if you distribute a script language based
    program or package, notwithstanding the fact, that one does not need the
    permission of reverse engineering to understand script language based
    applications}.}
    \item \emph{if you distribute a work containing manually copied portions of
    the Library.}
  \end{itemize}
\end{itemize}
 
And, so, we can reformulate our result as a slightly modified \enquote{rule of
thumbs} originally offered by an open source expert who analyzed the problem of
protecting your own work from an other viewport:

\begin{itemize}
  \item \enquote{DO NOT statically link [or combine] [open source] code if you
  wish to keep your program proprietary [and if you want to protect it against reverse
  engineering]}\footcite[cf.][6; bracketed text KR.]{Ilardi2010a}.
  \item \enquote{DO dynamically link to [any open source code, not only to] LGPL
  code}\footcite[cf.][6; bracketed text KR.]{Ilardi2010a}.
\end{itemize}

\textbf{\textsf{q.e.d}}

%% use all entries of the bibliography
%\nocite{*}

