% Telekom osCompendium 'for being included' snippet template
%
% (c) Karsten Reincke, Deutsche Telekom AG, Darmstadt 2011
%
% This LaTeX-File is licensed under the Creative Commons Attribution-ShareAlike
% 3.0 Germany License (http://creativecommons.org/licenses/by-sa/3.0/de/): Feel
% free 'to share (to copy, distribute and transmit)' or 'to remix (to adapt)'
% it, if you '... distribute the resulting work under the same or similar
% license to this one' and if you respect how 'you must attribute the work in
% the manner specified by the author ...':
%
% In an internet based reuse please link the reused parts to www.telekom.com and
% mention the original authors and Deutsche Telekom AG in a suitable manner. In
% a paper-like reuse please insert a short hint to www.telekom.com and to the
% original authors and Deutsche Telekom AG into your preface. For normal
% quotations please use the scientific standard to cite.
%
% [ Framework derived from 'mind your Scholar Research Framework' 
%   mycsrf (c) K. Reincke 2012 CC BY 3.0  http://mycsrf.fodina.de/ ]
%

The rest of our way is simple: First, we can ascertain, that none of the other
open source licenses we consider\footnote{Just as the OSLiC (), also this
article only focuses on the most important open source licenses: the Apache
license, the BSD licenses, the MIT license, the MS-PL, the PostgreSXQL, the PHP
license, the CDDL, the EPL, the EUPL, the MPL, the LGPLs the MPL, the GPLs and
the AGPL}, contain the phrase 'reverse engineering'. Moreover, they even do not
contain one of the single words\footnote{This can be verified by (a) loading
down all license texts from the osi page
(http://opensource.org/licenses/alphabetical) and by executing the command $grep
-i "engineering" *$ respectively $grep -i "reverse"$ in the directory into which
the license texts have been loaded down: grep finds the word reverse and
engineering only in the texts of the LGPLs.}. So, we may infer, that these most
important other open source licenses could at most require the permission of
reverse engineering indirectly. Second, we know already that distributing script
code let the allowance to reverse engineer, become irrelevant: script code can
directly be read and understood, if one knows the script language. Third, from
the definition of strong copleft we may derive, that distributing software
licensed under a strong copyleft license let the permission of reverse
engineering become unimportant, because the source code of the work using the
libraries must also be made accessible.

So -- overally -- we may conclude, that we have only to consider those cases,
where a piece of software is distributed in form of binaries oder byte code,
which uses libraries licensed under permissive open source license or under a
weak copyleft license.

From the definition of being a permissive license or a weak copyleft license we
know already that the used components do not directly influence the permission
to use, to distribute, to investigate, or to modify the work using the
components. 

So, if we distribute such a work in form of dynamically linkable, but still not
linked binaries or byte code, then there is no way to reasonably derive that the
work using the compnents, may be reverse engineered: the open source licenses
only cover the components themselves. But these components still are not
part of still unlinked work\footnote{The only way to infer that the licenses of
the components operates also on the using work, is to argue that the using work
must at least contain elements (identifiers etc.) of the interfaces declared
(but not defined) by the libraries and that therefore at least these elements
may be investigated or modified. Fortunately, it is a feature of libraries to
be used in accordance to an interface, the developers of the library have
designed to make their library usable. So, if the licenses do not address the
issue of the implicitly included portions of the library in case of
still unlinked binaries or byte code files, then one might infer from the
common sense of computing software, that the developers implictly allowed such
integration without any requirements, simply, because their library would not
be usable, if this was not so.}

On the other side, if we distribute the work using the components as a
statically linked binary or byte code file, which therefore already contains the
complete used components (and not only the declared lements of the interfaces),
then we have to add the open source license text of the component, we have to
distribute the copyright line, and we are not allowed to modify one of the
licensing assertions integrated into the source code of the components. This
follows from all the open source licenses. So, the receiver of the statically
linked work probably is allowed to modify the embedded open source components -
even if he had to edit the binary or byte code files. Methods, to do so, are
known as reverse engineering. Hence, if we distribute a statically linked work
using open source licensed components, we at least have to fear that our
receivers could indirectly have also got the permission to reverse engineer our
complete product.



%% use all entries of the bibliography
%\nocite{*}

