% Telekom osCompendium 'for being included' snippet template
%
% (c) Karsten Reincke, Deutsche Telekom AG, Darmstadt 2011
%
% This LaTeX-File is licensed under the Creative Commons Attribution-ShareAlike
% 3.0 Germany License (http://creativecommons.org/licenses/by-sa/3.0/de/): Feel
% free 'to share (to copy, distribute and transmit)' or 'to remix (to adapt)'
% it, if you '... distribute the resulting work under the same or similar
% license to this one' and if you respect how 'you must attribute the work in
% the manner specified by the author ...':
%
% In an internet based reuse please link the reused parts to www.telekom.com and
% mention the original authors and Deutsche Telekom AG in a suitable manner. In
% a paper-like reuse please insert a short hint to www.telekom.com and to the
% original authors and Deutsche Telekom AG into your preface. For normal
% quotations please use the scientific standard to cite.
%
% [ Framework derived from 'mind your Scholar Research Framework' 
%   mycsrf (c) K. Reincke 2012 CC BY 3.0  http://mycsrf.fodina.de/ ]
%

For fulfilling our rule, to read the text strictly and deduce our
interpretations reasonably, let us firstly only highlight the syntactical
conjunctions for simplifying the understanding:

\begin{quote}\noindent\emph{\enquote{[\ldots] you may [\ldots] combine
\textbf{or} link a \enquote{work that uses the Library} with the Library to
produce a work containing portions of the Library \textbf{and} distribute that
work under terms of your choice, \textbf{provided that} the terms permit
modification of the work for the customer's own use \textbf{and} \emph{reverse
engineering} for debugging such modifications.}\footnote{\cite[cf.][\nopage wp.,
§6, emphasis KR.]{Lgpl21OsiLicense1999a}.}}
\end{quote}

It is evident that the conjunction \emph{'provided that'} is splitting the
sentence into two parts: you are allowed to do something \emph{provided that} a
condition is fulfilled. Additionally, both parts of the sentence -- the one
before the conjunction \emph{'provided that'} and the part after it -- are
syntactically condensed embedded phrases which also contain subordinated
conjunctions and elliptical constructions\footnote{cf.
http://en.wikipedia.org/wiki/Ellipsis\_\%28linguistics\%29, wp.
}. These syntactical interconnections must be disbanded:

Let us firstly \textbf{dissolve the syntactical compression \underline{before}
the conjunction \emph{'provided that'}}: It is established by using the two
other conjunctions \emph{and} and \emph{or} and introduced by the subordinating
phrase \emph{you may [\ldots]}. Unfortunately, from a formal point of view, one
can read the phrase \emph{you may (X or Y and Z)} as two different groupings:
either as \emph{you may ((X or Y) and Z)} or as \emph{you may (X or (Y and Z))}.

But, fortunately, we know from the semantic point of view that speaking about
\emph{\enquote{[\ldots] combining \textbf{or} linking [\ldots something] to
produce a work containing portions of the Library}} denotes two different
methods which both can \emph{join} the components \emph{\enquote{[\ldots] to
produce a work containing portions of the Library}}. So, let us -- only for a
moment\footnote{Later on we will re-insert th original phrase!} -- simply
replace the string \emph{\enquote{combine or link}} by the string
\emph{\enquote{*join}}\footnote{When the LGPL and the GPL were initially
defined, the C programming language was the predominant model of software
development. Knowing this method eases the understanding of these licenses.
Thus, it is not totally wrong to take this token *join also as a curtsey to the
C programming language.}. This reduces the syntactical structure of the sentence
back to the simple phrase \emph{you may (W and Z)} in which \emph{W} stands for
\emph{(X or Y)}.

Now, we can directly state that the phrase \emph{you may (W and Z)} itself is a
condensed version of the explicit phrase \emph{ (you may W) and (you may Z)}.

Finally we have to note, that the phrase before the conjunction \emph{'provided
that'} contains also a linguistic ellipsis\footnote{cf.
http://en.wikipedia.org/wiki/Ellipsis\_\%28linguistics\%29, wp.
}: It says that you may *join the components \enquote{to produce \textbf{a work
containing portions of the Library} \textbf{and} distribute \textbf{that work}
under terms of your choice}. With respect to the English grammar, we may
conclude that the second term \emph{that work} refers back to the previously
introduced specification of \emph{a work containing portions of the Library}: if
a complete phrase has just been introduced explicitly, then the English language
allows to reduce its' next occurence syntactically while its' complete meaning
is retained. Hence, conversely, we are allowed to unfold the reduced form to
restore the complete phrase.

So -- overall -- we may understand the phrase before the conjunction
\emph{'provided that'} as a phrase with the structure \emph{(you may W) and (you
may Z')}:

\begin{quote}\noindent\emph{\textbf{((}you may [\ldots] \emph{*join} a
\enquote{work that uses the Library} with the Library to produce a work
containing portions of the Library\textbf{) and (}you may [\ldots] distribute
that work containing portions of the Library under terms of your
choice\textbf{))}} \textbf{provided that} [\ldots]\end{quote}

Theoretically, a reader could reject our first dissolution of the
LGPL2-RefEng-Sentence. But for reasonably denying our interpretation he has to
deliver other resolutions of the lingustic elliptical subphrases or other
dissolvations of the conjunctions. Fortunately, it seems to be evident that such
attempts must violate the English grammar.

Let us secondly \textbf{dissolve the part \underline{after} the conjunction
\emph{'provided that'}}: With respect to the subordinated conjunction
\emph{'and'}, the subphrase \emph{the terms permit} syntactically refers to
both, the \emph{modification} and the \emph{reverse engineering}: An embedded
conjunction \emph{'and'} allows to use a more stylish grammatical compaction.
So, it should be clear, that saying

\begin{quote}\noindent\textbf{provided that} \emph{the terms permit modification
of the work for the customer's own use \emph{\textbf{and}} reverse engineering
for debugging such modifications}\end{quote}

means

\begin{quote}\noindent\textbf{provided that} \emph{the terms permit
\textbf{(} modification of the work for the customer's own use \emph{\textbf{and}}
reverse engineering for debugging such modifications\textbf{)}}\end{quote}

and is totally equivalent to the sentence 

\begin{quote}\noindent[\ldots] \textbf{provided that} \emph{\textbf{((}the terms
permit modification of the work for the customer's own use\textbf{)}
\emph{\textbf{and}} \textbf{(}the terms permit reverse engineering for debugging
such modifications\textbf{))}}.
\end{quote}

We believe that there is no other possibility to understand this part of the
LGPL2-RefEng-Sentence with respect to the rules of the English language.
Nevertheless, this is a next point where our reader may formally disagree with
us. If he really wants to object our dissolution, he must deliver another valid
interpreation of the scope of the conjunction \emph{and} or he must deliver
another resolutions of the linguistic ellipsis. But we reckon, that one can not
reasonably argue for such alternatives.

Finally, there are other deeply embedded ellipses, which need to be resolved
as well:

\begin{enumerate}
  \item  In the part before the splitting conjunction \emph{'provided that'} we
  already had to expand the abridging \emph{'that work'} by its intended
  explicated version \emph{'that work containing portions of the Library'}.  In
  the part after the splitting conjunction the first subphrase also contains the
  term \emph{'the work'}. Formally, this term can either refer to \emph{'the
  work that uses the library'} as one of the components which are joined, or it
  can refer to \emph{'the work containing portions of the Library'} as the
  result of joining the components. We decide to constantly dissolve the
  elliptic abridgement by the phrase \emph{'the work containing portions of the
  Library'}.
  \item The first clause of the part after the splitting conjunction
  \emph{'provided that'} talks about the purpose of \enquote{permitting
  modification of the work} which we just had to unfold to the phrase
  \emph{'permitting modification of the work containing portions of the
  Library'}. The second clause talks about the purpose of \enquote{permitting
  reverse engineering}: it shall support the \enquote{debugging [of] such
  modifications}. The pronoun \emph{'such'} indicates that the word
  \emph{'modifications'} refers back to the just unfolded phrase
  \emph{modification of the work containing portions of the Library}. So, even
  the second sentence has to be expanded to that explicit phrase.
  \item Finally and only for being complete, we also have to unfold the clause
  \enquote{the terms} to the form which is predetermined by the first referred
  instance \enquote{the terms of your choice}
\end{enumerate}

So -- overall -- we are allowed to rewrite the LGPL2-RevEng-Sentence 
in the following form, namely without having changed its
meaning\footnote{Recollect that '*join' still stands for 'combine or link'.}:

\begin{verbatim}
( ( you may 
       *join a work that uses the Library with the Library
        to produce a work containing portions of the Library )
  AND 
  ( you may 
        distribute that work containing portions of the Library
        under terms of your choice 
) )
PROVIDED THAT
( ( the terms of your choice permit 
        modification of the work containing portions of 
        the Library for the customer's own use )
  AND
  ( the terms of your choice permit
        reverse engineering for debugging modifications 
        of the work containing portions of the Library   
) )
\end{verbatim}

At this point we must recommend all our readers to verify that this
'structurally explicated presentation' does exactly mean the same as the
intially quoted LGPL2-RefEng-Sentence. We are now going to discuss some of its'
logical aspects by some formal transformations. For accepting these operations
and linking the results back to the original LGPL2-RefEng-Sentence, it is very
helpful to know that one already has accepted the equivalence of this explicated
form and the more condensed original version. For reviewing the equivalence the
reader could -- for example -- ask himself which of our rewritings are wrong,
why they are wrong and which alternatives can reasonably be offered for solving
the syntactical issues which disposed us to chose our solutions. Again, we
ourselves -- of course -- are profoundly convinced that both versions are
completely equivalent.

%% use all entries of the bibliography
%\nocite{*}

