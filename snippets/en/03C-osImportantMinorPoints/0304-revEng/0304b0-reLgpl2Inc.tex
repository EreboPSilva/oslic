% Telekom osCompendium 'for being included' snippet template
%
% (c) Karsten Reincke, Deutsche Telekom AG, Darmstadt 2011
%
% This LaTeX-File is licensed under the Creative Commons Attribution-ShareAlike
% 3.0 Germany License (http://creativecommons.org/licenses/by-sa/3.0/de/): Feel
% free 'to share (to copy, distribute and transmit)' or 'to remix (to adapt)'
% it, if you '... distribute the resulting work under the same or similar
% license to this one' and if you respect how 'you must attribute the work in
% the manner specified by the author ...':
%
% In an internet based reuse please link the reused parts to www.telekom.com and
% mention the original authors and Deutsche Telekom AG in a suitable manner. In
% a paper-like reuse please insert a short hint to www.telekom.com and to the
% original authors and Deutsche Telekom AG into your preface. For normal
% quotations please use the scientific standard to cite.
%
% [ Framework derived from 'mind your Scholar Research Framework' 
%   mycsrf (c) K. Reincke 2012 CC BY 3.0  http://mycsrf.fodina.de/ ]
%


%% use all entries of the bibliography
%\nocite{*}

The LGPL-v2.1 contains one sentence which literally refers to the issues of
\emph{reverse engineering}:

\begin{quote}\noindent\emph{\enquote{[\ldots] you may [\ldots] combine or link a
\enquote{work that uses the Library} with the Library to produce a work
containing portions of the Library, and distribute that work under terms of your
choice, provided that the terms permit modification of the work for the
customer's own use and \emph{reverse engineering} for debugging such
modifications.}\footnote{\cite[cf.][\nopage wp., §6. ]{Lgpl21OsiLicense1999a}.
The first ellipse in this citation -- notated by the string '[\ldots]' -- refers
to the phrase \enquote{As an exception to the Sections above,}, the second to
the phrase \enquote{also}. These words together want to indicate, that the LGPL
offers its §6-way-of-distribution as an exception to the intended default way of
distributing such a Library. So, the nature of the extraordinary way itself is
not affected by this hint. Thus, we feel free to erase this contextual
link.}}\end{quote}

Hereinafter, we will sometimes denote these lines by
the word \emph{LGPL2-RefEng-Sentence}.
