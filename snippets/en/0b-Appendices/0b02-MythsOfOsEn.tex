% Telekom osCompendium 'for beeing included' snippet template
%
% (c) Karsten Reincke, Deutsche Telekom AG, Darmstadt 2011
%
% This LaTeX-File is licensed under the Creative Commons Attribution-ShareAlike
% 3.0 Germany License (http://creativecommons.org/licenses/by-sa/3.0/de/): Feel
% free 'to share (to copy, distribute and transmit)' or 'to remix (to adapt)'
% it, if you '... distribute the resulting work under the same or similar
% license to this one' and if you respect how 'you must attribute the work in
% the manner specified by the author ...':
%
% In an internet based reuse please link the reused parts to www.telekom.com and
% mention the original authors and Deutsche Telekom AG in a suitable manner. In
% a paper-like reuse please insert a short hint to www.telekom.com and to the
% original authors and Deutsche Telekom AG into your preface. For normal
% quotations please use the scientific standard to cite.
%
% [ File structure derived from 'mind your Scholar Research Framework' 
%   mycsrf (c) K. Reincke CC BY 3.0  http://mycsrf.fodina.de/ ]

%


%% use all entries of the bibliography
%\nocite{*}


\section{Some Widespread Open Source Myths}

From the viewpoint of an internet student we have to consider that the web
offers a mass of rumors concerning the nature of Open Source Software
(Licenses). Here are some of the myths\footcite[At least one time even a
scientific legally discussing book is talking about the \glqq{}Myth around Open
Source Licenses\grqq{} - although only as part of  the title: cf][1ff,
especially 209ff]{GuiOvd2006a}.
we met:
 

\begin{description}
  \item[Open Source tries to improve the world ethically] :- no, there's a clear
  ban to exclude persons, groups, purposes
  \item[Changed Open Source Software must be re-published] :- no, in a double
  sense! There are OS licenses which a allow the proprietarization of the
  modified code. And even the LGPL and the GPL, which clearly try to prevent
  the proprietarization, do not require generally that a modified code must be
  (re-)published. Only if you give your modfied (L)GPL licensed application as
  binary to anyboday then you have to handover the modified code too.
  \item[Modified Open Source Software must be given back to the whole community]
  :- No. Again, there are OS licenses which a allow the proprietarization of the
  modified code. And even the LGPL and the GPL - which clearly require, that you
  also publish the modified code, if you give the modified binary to anybody -
  do not require that you distribute your modification around the world. LGPL and
  GPL clearly say that you have to hand over the code to those persons which you
  give the binary. And if you only give your improvement only one person or a
  group of person, then you must handover your the code only to that persons or
  only to all members of that group.
  \item[Published Open Source Software is open for ever] :- No. The Copyright
  holder ever holds the Copyright. The can change the licence of next release of
  its software.
  \item[Software can either be Open Source Software or proprietary software] :-
  The Copyright holders can ever distribute the code under other conditions, een
  additionally. That's not a question of the licence, but of the Copyright.
  \item[The opposite of Open Source Software is commercial Software] :- No.
  Firstly you are also allowed to use the Open Source software in any commercial
  purpose. There's only one point which is excluded in OSS: you are not allowed
  to ask for a licence fee if you distribute 'Open Source Software'. Secondly
  there are many other forms like Freeware, Public Domain Software or anything
  else which is neither Open Source Software nor Commercial Software. It's
  senseless to take the question of money as mark for distinguish Open Source
  Software and its opposite. Moreover: Proprietary Software as opposite of Open
  Source Software should be defined ex negativo: all kind of software, which
  does not fit the OSD is proprietary.
  \item[Open Source Software prohibits to earn money] :- No,
  you are allowed to invent each business model you want. There's only one
  exception: you are not allowed to ask for a licence fee if you distribute
  'Open Source Software [Achtung: sollte eigentlich nur für GPL gelten].
  Historically this mistake might be evoked by Debian: The GNU project missed
  its kernel while the Linux kernel was already distributed as part of
  collections which also include GNU software. Then, in 1983? Ian Murdock was
  supported by RMS and its FSF to build a really free distribution (Debian)
  containg GNU software and the Linux kernel. But Ian Murdock states also, that
  debian does not want to earn money. (clear details)
% TODO: check, wether OSD requires license fee free distribution
  \item[Modifications of Open Source Software must be marked] :- No. This is not
  a defining postulation of the OSD. The OSD allows licenses to require the mark
  of modifacations. But it does not require from all licenses to rquire the mark
  modifications for being an Open Source License.
  \item[Modifications of Open Source Software must be marked by your personal
  data] :- No, it's only required to mark modifications so that a reader could
  distinguish the modifications from the original code. It's required for saving
  the integrity of the original author. And therefor it's not required as a
  constitutiv criteria by the OSD. It might be, that a license additionally
  requires your name. But's not featue of Open Source Software in general. And
  at least the licenses discussed by us do not require to insert your name.
% TODO: check wether any of our licenses reuire that you mark modifications by
% your personal data / real name  
  \item[The Open Source Definition determines the conditions to use Open Source
  Software] :- No. The Open Source Definition determines which licenses are Open
  Source Licenses, nothing more. The OSD is a set of necessary conditions to be
  an Open Source License. It determines the freedom and the responsibilities of
  a user as a set of more or less abstract rules. But it does not constitute a
  set of sufficient tasks which a user has to do for fulfilling any Open Source
  License. Open Source Licenses may differ by instatiating the OSD criteria.
  So, if you want to know what you have to do to fulfill a license you have to
  go back to the real license of that software you are using.
\end{description}

%\bibliography{../bibfiles/oscResourcesEn}
