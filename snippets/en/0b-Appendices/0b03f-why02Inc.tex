% Telekom osCompendium 'for being included' snippet template
%
% (c) Karsten Reincke, Deutsche Telekom AG, Darmstadt 2011
%
% This LaTeX-File is licensed under the Creative Commons Attribution-ShareAlike
% 3.0 Germany License (http://creativecommons.org/licenses/by-sa/3.0/de/): Feel
% free 'to share (to copy, distribute and transmit)' or 'to remix (to adapt)'
% it, if you '... distribute the resulting work under the same or similar
% license to this one' and if you respect how 'you must attribute the work in
% the manner specified by the author ...':
%
% In an internet based reuse please link the reused parts to www.telekom.com and
% mention the original authors and Deutsche Telekom AG in a suitable manner. In
% a paper-like reuse please insert a short hint to www.telekom.com and to the
% original authors and Deutsche Telekom AG into your preface. For normal
% quotations please use the scientific standard to cite.
%
% [ Framework derived from 'mind your Scholar Research Framework' 
%   mycsrf (c) K. Reincke 2012 CC BY 3.0  http://mycsrf.fodina.de/ ]
%


%% use all entries of the bibliography
%\nocite{*}

So again: Did we need a new book about Open Source Software? We had looked for a
reliable integrated Open Source Compendium. But we found separate pieces of
information and - as we know today - some rumors. Our answer was clear:
naturally we did not need a new general book about Open Source. But what was
lacking was a description what responsible developers, project managers or
product developers require to fulfill Open Source Licenses. We needed an
\textit{Open Source License Compendium}.

At the best such an \textit{Open Source License Compendium} would contain a set
of simply to process \textit{'For-Fulfilling-The-License-To-Do-Lists'}.
Additionally it should offer an intuitively user-friendly search option for
these lists. In any case, it should share developers and project managers the
effort of having to become Open Source License experts. For the other users, it
should also clearly explain why one has to do this and not that. Hence a
reliable \textit{Open Source License Compendium} should not only list what one
has to do, but should offer both, thoroughly verified reliable details and
clearly condensed guidance.

Although we did not find such an Open Source Compendium we were familiar with
the spirit of the Open Source Community. Hence we followed one of its' most
simple rules: \emph{'what you miss you must develop ion your own'}. Some
principles should help us to achieve our targets:

\begin{description}
  \item[To-do lists as the core, discussions around them]: Our work should be
  split into two parts. As it core we wanted to offer a
  set of To-Do-Lists. Each of these lists should be relevant to one specific
  Open Source License and should be clustered by the Open Source specific use
  cases. Around this all those aspects of Open Source Software which influence the
  interpretation of the licenses and the rules core should be precisely
  characterized. Nevertheless, the users should be able to skip
  details and go directly to the section they require.
  \item[Quotations with thoroughly specified sources]: Even if our users should
  not be obliged to read every part of the compendium they should not be
  required to believe us without question. We wanted to be revisable. Because
  our sources and our conclusions should be easily verifiable, we decided to use
  the academic citations and list bibliographic data extensively on the basis
  that our task should be to collect information, not to invent new 'facts'.
  \item[Not the internet alone, also books and articles]: We wanted to go back
  to the originals even if the internet was full of more or less modified
  copies. We wished to get reliable facts and descriptions. Therefore we decided
  to evaluate not only the internet but also scientific sources - for example -
  offered by university libraries.
  \item[Not clearing out the forest land, but cutting out a swathe]: Even if we
  had to deal with licenses and their legal aspects we did not want to get lost
  in detailed discussions. It should not be our task to find out whether a
  specific kind of handling would still be legal or already forbidden.
  We did not want to fight against the licenses. We did not want to stretch
  their ambit or to test their boundary. We wished to accept Open Source
  Licenses as they are: rules written from developers for developers. And even
  if some parts of these licenses would not be valid with respect to a legal
  system\footcite[And indeed for example for the GPL one can argue in this way:
  Even if you take the GPL as a contract of the type 'donation' respectively
  \enquote{Schenkung}, it is presented in the form of AGBs respectively
  \enquote{Allgemeine Geschäftsbedingungen} and must therefore follow the
  general AGB rules.'Regrettably' in Germany these general AGB rules do not
  allow to exclude each type of warranty. If we follow Oberhem, §11 and §12 of
  the GPL must be invalid in Germany because of these general AGB rules.
  Moreover, for Oberhem even §5 - the important clause of the GPL by which you
  can only get the right to use and to distribute GPL software if you respect
  the rules of the GPL - seems also to be invalid respectively
  \enquote{unwirksam}. But the good message is that the GPL as whole is not
  invalid even if it contains invalid clauses.][128, 133ff, 150ff, esp. 146,
  159]{Oberhem2008a}), we wanted to take them as our guideline - at least while
  they do not violate more general laws\footnote{what they clearly do not do!}.
  We simply wanted to \emph{find one proven way} to cross the maybe slightly
  unsure forest of Open Source Licenses. Even if indeed some clauses of the
  licenses finally were not enforceable against us we wanted to respect them
  'voluntarily'. We wanted to deliver a set of rules which support users and
  remove the possibility of becoming involved in license disputes with Open
  Source developers or the Free Software Foundation.
  \item[Take the text seriously]: On the other side we wanted to take our
  license texts as they were. If they lacked anything\footcite[The systematical
  underdetermination of licenses is a problem being also known in the Open
  Source respectively Free Software movement. Following the biography of RMS his
  main judicial counselor Moglen has stated, that \enquote{there is uncertainty
  in every legal process (\ldots) } and that it seemed to be silly to try
  \enquote{(\ldots) to take out all the bugs (\ldots)}. Nevertheless - so
  Moglen resp. Williams - the goal of Richard Stallman was \enquote{the complete
  opposite}: He tried \enquote{(\ldots) to remove uncertainty which is
  inherently impossible}. But - and that's the nub of this analysis -
  Moglen had to follow Stallmann because of RMS character. And he had to
  summarize their work so, that \enquote{(\ldots) the resulting elegance (of the
  GPL; KR.), the resulting simplicity (of the GPL; KR.) in design almost
  achieves what it has to achieve}. Hence we are asked to take the license
  texts themselves seriously. cf.][177f]{Williams2002a}, we would interpret the
  open issues in the spirit of the Open Source idea. But where the text was
  clear and definite we wanted to take its propositions as a definite decision -
  even if that meaning stood against well known Open Source 'facts'.
  \item[Trust the swarm]: We did not want to use our own research alone as a
  basis. We knew that the swarm is ever stronger than a set of some randomly
  selected experts. Therefore we decided to publish our text as a still
  unfinished work, starting with an early release 0.2. And then we wanted
  to invite the community to complete the compendium together with us. We would elaborate our Open
  Source Compendium as a set of LaTeX- and BibTeX files which could be developed
  and managed in GIT or any other version control system. And finally we would
  publish our text under a Creative Commons Attribution-Share Alike German 3.0
  license% Telekom osCompendium License Include Module
%
% (c) Karsten Reincke, Deutsche Telekom AG, Darmstadt 2011
%
% This LaTeX-File is licensed under the Creative Commons Attribution-ShareAlike
% 3.0 Germany License (http://creativecommons.org/licenses/by-sa/3.0/de/): Feel
% free 'to share (to copy, distribute and transmit)' or 'to remix (to adapt)'
% it, if you '... distribute the resulting work under the same or similar
% license to this one' and if you respect how 'you must attribute the work in
% the manner specified by the author ...':
%
% In an internet based reuse please link the reused parts to www.telekom.com and
% mention the original authors and Deutsche Telekom AG in a suitable manner. In
% a paper-like reuse please insert a short hint to www.telekom.com and to the
% original authors and Deutsche Telekom AG into your preface. For normal
% quotations please use the scientific standard to cite.
%
% [ File structure derived from 'mind your Scholar Research Framework' 
%   mycsrf (c) K. Reincke CC BY 3.0  http://mycsrf.fodina.de/ ]
%
\footnote{
This text is licensed under the Creative Commons Attribution-ShareAlike 3.0 Germany
License (http://creativecommons.org/licenses/by-sa/3.0/de/): Feel free \enquote{to
share (to copy, distribute and transmit)} or \enquote{to remix (to
adapt)} it, if you \enquote{[\ldots] distribute the resulting work under the
same or similar license to this one} and if you respect how \enquote{you
must attribute the work in the manner specified by the author(s)
[\ldots]}):
\newline
In an internet based reuse please mention the initial authors in a suitable
manner, name their sponsor \textit{Deutsche Telekom AG} and link it to
\texttt{http://www.telekom.com}. In a paper-like reuse please insert a short
hint to \texttt{http://www.telekom.com}, to the initial authors, and to their
sponsor \textit{Deutsche Telekom AG} into your preface. For normal quotations
please use the scientific standard to cite.
\newline
{ \tiny \itshape [derived from myCsrf (= 'mind your Scholar Research Framework') 
\copyright K. Reincke CC BY 3.0  http://mycsrf.fodina.de/)] }}, to allow other people to correct
  us, to help us or even to take our results for their own purposes.
\end{description}

And so we did. Here is the result. Feel free to use it - according to our
licensing.

%\bibliography{../../../bibfiles/oscResourcesEn}
