% Telekom osCompendium 'for being included' snippet template
%
% (c) Karsten Reincke, Deutsche Telekom AG, Darmstadt 2011
%
% This LaTeX-File is licensed under the Creative Commons Attribution-ShareAlike
% 3.0 Germany License (http://creativecommons.org/licenses/by-sa/3.0/de/): Feel
% free 'to share (to copy, distribute and transmit)' or 'to remix (to adapt)'
% it, if you '... distribute the resulting work under the same or similar
% license to this one' and if you respect how 'you must attribute the work in
% the manner specified by the author ...':
%
% In an internet based reuse please link the reused parts to www.telekom.com and
% mention the original authors and Deutsche Telekom AG in a suitable manner. In
% a paper-like reuse please insert a short hint to www.telekom.com and to the
% original authors and Deutsche Telekom AG into your preface. For normal
% quotations please use the scientific standard to cite.
%
% [ Framework derived from 'mind your Scholar Research Framework' 
%   mycsrf (c) K. Reincke 2012 CC BY 3.0  http://mycsrf.fodina.de/ ]
%

\section{Some Additional Remarks on the OSLiC Quotation Style}\label{sec:QuotationAppendix}

We have already characterized the general tone of our
footnotes\footnote{$\rightarrow$ p. \pageref{QuotationPrinciple} }. Let us now
briefly explain a little peculiarity of our bibliography:

Modern times have also changed the humanities. Formerly a book or an article
must be printed for being ripe to be quoted. Our statements relied on static,
readily prepared works. Nowadays even university libraries sometimes offer those
books and articles as PDF files which are printed in the original. As a scholar,
now you must rely on the equality of the printed version and the PDF file - at
least with respect to the page numbers and the appearance. You can not verify the
equivalence - at least to a certain degree.

Moreover: in case of such 'e-books' and 'e-articles' the libraries often do not
offer the pdf files themselves but links to the download pages of the publisher.
Formerly as a scholar you could trust that your readers would be able to
retrieve the quoted work if they want to verify your citations. It's one task of
our libraries to hold available our scientific sources. But now they do not buy
any longer the books, but the right to download files over the university net.
In this case these PDF files are not stored on the serves of the university
library. By using the link provided by the publisher each student or each reader
downloads his own file - case by case. Therefore - as a scholar - you now have
to trust that the publisher, who provides the link, will not change that pdf
file that you have cited.

But it gets even worse: While it might be that publishers modify their work
secretly (even it is not very likely that they do it), it's a definite feature
of the web that its' pages are fre\-quen\-tly changed. Hence we must ask
ourselves: Can we seriously argue on the basis of statements and documents which
might disappear? Can we quote such possibly volatile sources? The problem is: we
must do it, especially if we write about an internet topic - and even if we want
to write a really reliable compendium.

So, what can we do? Firstly we must confide in our readers, that they either
will retrieve our sources or - if they can not find them - that they
believe that we really have found and read what we have written and
quoted. Secondly we store all these e-wares\footnote{Take this little word as
(new) generalization of 'e-book', 'e-article', 'e-paper' and so on.} we
read\footnote{But because of the copyright we ourselves are naturally not
allowed to offer a download link for them or to send a copy of it to those who
want to verify our quotes.}. And thirdly we should lay open to our readers the
different levels of reliableness of our sources. Therefore we use
the following markers in our bibliographic data\footnote{And another hint: Nowadays sometimes
even scientific libraries doesn't offer exact 'e-copies' of the original. In
some cases one can get only html-versions of articles which formerly were
printed as part of journals. In these case the scholar has to use sources which
lost their original page-numbers. The same can happen to articles of proceedings
etc. which are now only offered as autonomous pdf files with an internal paging.
If we quote such kind of articles we try to specify the number of the quoted
article in the original row of articles, added - if possible - by an internal
page number. But naturally we also try to follow the bibliographic data
delivered by that organization which distributes these kind of copies.}:

\begin{itemize}
  \item Print / Copy:- The source is printed and we saw either the printed work
  really or we get an official copy by our library. Hence you should also be able
  to get the work in a library, at least in those we used (UB Frankfurt or ULB
  Darmstadt).
  \item BibWeb/[PDF/\ldots] :- The source might be printed, but we read only the
  electronic version (PDF or other type of format), offered by and over the
  net of our university libraries (UB Frankfurt or ULB Darmstadt).
  \item FreeWeb/[PDF/\ldots] :- We read the electronic version offered by the
  free web. In this case we add the url\footnote{Please note: Long urls often
  destroy the pleasing appearance of a text because it's difficult to wrap the
  lines acceptably. Hence we wished to make it easier for LaTeX to do this job.
  Therefor we sometime split the urls and inserted blanks. So you have to erase
  all blanks if you want to verify our urls.} and the date when we downloaded /
  saw the text.
\end{itemize}


%\bibliography{../../../bibfiles/oscResourcesEn}
