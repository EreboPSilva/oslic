% Telekom osCompendium 'for being included' snippet template
%
% (c) Karsten Reincke, Deutsche Telekom AG, Darmstadt 2011
%
% This LaTeX-File is licensed under the Creative Commons Attribution-ShareAlike
% 3.0 Germany License (http://creativecommons.org/licenses/by-sa/3.0/de/): Feel
% free 'to share (to copy, distribute and transmit)' or 'to remix (to adapt)'
% it, if you '... distribute the resulting work under the same or similar
% license to this one' and if you respect how 'you must attribute the work in
% the manner specified by the author ...':
%
% In an internet based reuse please link the reused parts to www.telekom.com and
% mention the original authors and Deutsche Telekom AG in a suitable manner. In
% a paper-like reuse please insert a short hint to www.telekom.com and to the
% original authors and Deutsche Telekom AG into your preface. For normal
% quotations please use the scientific standard to cite.
%
% [ Framework derived from 'mind your Scholar Research Framework' 
%   mycsrf (c) K. Reincke 2012 CC BY 3.0  http://mycsrf.fodina.de/ ]
%


%% use all entries of the bibliography
%\nocite{*}


\chapter{Introduction}

% Abstract
\footnotesize \begin{quote}\itshape
This chapter briefly describes the idea behind the \oslic, the way it should be
used and the way it can be read---which is indeed not quite the same.
\end{quote}
\normalsize{}

% Content
This book focuses on just one issue: \emph{What needs to be done in order to act
in accordance with the licenses of those \emph{open source software} we use?}
The \emph{Open Source License Compendium} aims at reliably answering this
question---in a simple and easy to understand manner. However, it is not just
another book on \emph{open source} in ge\-ne\-ral.\footnote{Meanwhile, there are
tons of literature dealing with open source. Trying to expand your knowledge by
means of books and articles might let you get lost in literature: our list of
secondary literature may adumbrate this `danger of being overwhelmed'. But
nevertheless, our bibliography at the end of the \oslic{} is not complete.
Moreover, it is not intended to be complete. It is only an extract representing
the background information we did not directly cite in the \oslic. If we were
forced to indicate two books for attaining a good overview on the topic of
\emph{open source (licenses)} we would name (a) the `Rebel Code' (\cite[for a
German version cf.][\nopage passim]{Moody2001a}---\cite[for an English version
cf.][passim]{Moody2002a}) and (b) the `legal basic conditions'
(\cite[cf.][\nopage passim]{JaeMet2011a}). But fortunately, we are not forced to
do so.} The intention is, rather, for it to be a tool for simplifying the
activities for achieving license conformity.


This compendium was created out of necessity at \emph{Deutsche Telekom AG} to
counter a challenge some of its software developers and project managers were
facing: Of course, the company itself wants to behave as license compliantly as
its employees, but, unfortunately, they could not find a reference text which
simply lists what precisely must be done in order to comply with the license of
that piece of open source being used.

As some of these co-workers in Telekom projects, even we---the initial authors
of the \oslic---did not want to become open source license experts only for being
able to use open source software in accordance with their respective licenses. We
did not want to become lawyers. We just wanted to do more efficiently, what
in those days claimed much time and many resources. We were searching for clear
guidance instead of having to determine a correct way through the jungle of open
source licenses---over and over again, project for project. We loved using the
high-quality open source software to improve our performance. We liked using it
legally. But we did not like to laboriously discuss the legal constraints of the
many and different open source licenses.

What we needed was an easy-to-use handout which would lead us without any
detours to executable lists of work items. We wished to obtain to-do lists,
tailored to our usecases and our licenses. We needed reliable lists of tasks we
only had to execute for being sure that we were acting in accordance with the
open source license. When we started out, such a compendium did not exist.

For solving this problem our company took three decisions:

The first decision our company arrived at was to support a small group of
employees to act as \emph{a board of open source license experts}: They should
offer a service for the whole company. Projects, managers, and developers should
be able to ask this board what they have to do for complying with a specific
open Source License under specific circumstances. And this board should answer
with authoritative to-do lists whose executions would assure that the requestors
are acting according to the corresponding open source licenses. The idea behind
this decision was simple. It would save cost and increase quality if one had one
central group of experts instead of being obliged to select (and to train)
developers---over and over again, for every new project. So, the \emph{OSRB} 
(the \emph{Telekom Open Source Review Board}) was founded as an internal expert
group---as a self-organizing, bottom-up driven community.

The second decision our company took was to allow this \emph{Telekom OSRB} to
collect their results systematically---in the form of a reusable compendium.
The idea behind this decision was also simple: The more the internal service
became known, the more the workload would increase: the more work, the more
resources, the more costs. So, such a compendium should save costs and enable
the requestors to find answers by themselves without becoming license experts:
For all default cases, they should find an answer in the compendium instead of
having to request that their work is analyzed by the OSRB. Thus, the planned
\emph{Telekom Open Source License Compendium} will prevent the need
to increase the size of the OSRB in the future.

The third decision our company reached was to allow the \emph{Telekom OSRB} to
create the compendium in the same mode of cooperation that open source projects
usually use. Again, a simple reason evoked this ruling: If in the future--as a
rule---not a reviewing OSRB, but a simple manual should assure the open 
source license compliant behavior of projects, programmers, and managers, this
book had of course to be particularly reliable. There is a known feature of the
open source working model: the ongoing review by the cooperating community
increases the quality. Therefore, the decision not only to write an internal
`Telekom handout,' but to enable the whole community to use, modify, and 
redistribute a broader \emph{Open Source License Compendium} was a decision for
improving quality. Consequently, the \emph{OSRB} decided to publish the
\emph{\oslic} as a set of \LaTeX sources, publicly available via the open
repository GitHub.\footnote{Get the code by using the link
\texttt{https://github.com/dtag-dbu/oslic}; get project information by
\texttt{http://dtag-dbu.github.com/oslic/} or by
\texttt{http://www.oslic.org/}.}  And it licensed the \oslic{} under Creative
Commons Attribution-ShareAlike 3.0 Germany License.\footnote{ This text is
licensed under the Creative Commons Attribution-ShareAlike 3.0 Germany License
(\texttt{http://creativecommons.org/licenses/by-sa/3.0/de/}): Feel free
\enquote{to share (to copy, distribute and transmit)} or \enquote{to remix (to
adapt)} it, if you \enquote{[\ldots] distribute the resulting work under the
same or similar license to this one} and if you respect how \enquote{you must
attribute the work in the manner specified by the author(s) [\ldots]}):
In an internet based reuse please mention the initial authors in a suitable
manner, name their sponsor \textit{Deutsche Telekom AG} and link it to
\texttt{http://www.telekom.com}. In a paper-like reuse please insert a short
hint to \texttt{http://www.telekom.com}, to the initial authors, and to their
sponsor \textit{Deutsche Telekom AG} into your preface. For normal citations
please use the scientific standard.}

But to publish the \emph{\oslic} as a free book has another important connotation --
at least for the \emph{Telekom OSRB}: It is also intended to be an appreciative
\emph{giving back} to the \emph{open source community} which has enriched and
simplified the life of so many employees and companies over so many years.

Altogether, the \oslic{} follows five principles:

\begin{description}
  \item[To-do lists as the core, discussions around them] Based on a simple
  form for gathering information concerning the use of a piece of open
  source software and its license, the \oslic{} shall offer an easy to use finder
  taking the requestor to the respective to-do list for ensuring license
  conformity. In addition, all these elements of the \oslic{} should comprehensibly
  be introduced and discussed without disturbing the usage itself.

  \item[Quotations with thoroughly specified sources]\label{QuotationPrinciple}
  The \oslic{} shall be revisable and reliable. It shall comprehensibly argue and
  explicitly specify why it adopts which information, from whom, in which
  version, and why.\footnote{For that purpose, we are using an `old-fashioned'
  bibliographic style with footnotes, instead of endnotes or inline-hints.
  We want to enable the users to review or to ignore our comments and hints just
  as they prefer---but on all accounts without being disturbed by large inline
  comments or frequent page turnings. We know that modern writer guides prefer
  less `noisy' styles (\cite[pars pro toto cf.][\nopage passim]{Mla2009a}). But
  for a reliable usage---challenged by the often modified internet sources---%
  these methods are still a little imprecise (for details $\rightarrow$ \oslic,
  pp.\ \pageref{sec:QuotationAppendix}. For a short motivation of the style used
  in the \oslic{} \cite[cf.][\nopage passim]{Reincke2012a}. For a more elaborated
  legitimizing version \cite[cf.][\nopage passim]{Reincke2012b}).}

  \item[Not clearing the forest, but cutting a swath] The \oslic{} has to deal with
  licenses and their legal aspects, no doubt. But it shall not discuss all
  details of every aspect. It shall focus on one possible way to act according
  to a license in a specific usecase---even if it is known that there might be
  alternatives.\footnote{The \oslic{} shall not counsel projects with respect to
  their specific needs. This must remain the task for lawyers and legal experts.
  The \oslic{} cannot and shall not replace a legal review or a legal advice by
  lawyers. It shall inspire developers, managers, open source experts, and
  companies to find good solutions, which they finally should have reviewed by
  legal counselors. For the German readers let us repeat again: The
  \oslic{} naturally respects the German \emph{Rechtsdienstleistungsgesetz}. It only
  contains legal reflections addressed to a general public. Its content may only
  be read as a \enquote{nur an die Allgemeinheit gerichtete Darstellung und
  Erörterung von Rechtsfragen}.}
  
  \item[Take the license text seriously!] The \oslic{} shall not give general
  lectures on legal discussions, much less shall it participate in them. It
  shall only find one dependable way for each license and each usecase to comply
  with the license. The main source for this analysis shall be the exact reading
  of the open source licenses themselves---based on and supported by the
  interpretation of benevolent lawyers and rationally arguing software
  developers. The \oslic{} shall respect that open source licenses are written for
  software developers (and sometimes by developers).
  
  \item[Trust the swarm!] The \oslic{} shall be open for improvements and
  adjustments encouraged and stimulated also by other people than employees of
  \emph{Deutsche Telekom AG}.
\end{description}

Based on these principles the \oslic{} offers two methods for being used:

First and foremost the readers expect to simply and quickly find those to-do
lists fitting their needs. Here is the respective process:%
  \footnote{For the well known `quick and dirty hackers'---as we tend to be, 
  too---we have integrated a shortcut: If you already know the license of the 
  open source package you want to use and if you are very familiar with the 
  meaning of the open source use cases we defined, then you might directly 
  jump to the corresponding license specific chapter, without `struggling' 
  with \textit{OSLiC 5 query form} ($\rightarrow$ \oslic{} p. 
  \pageref{OSLiCStandardFormForGatheringInformation}), the taxonomic
  \textit{Open Source Use Case Finder} ($\rightarrow$
  \pageref{OSLiCUseCaseFinder}) or the \textit{O}pen \textit{S}ource \textit{U}se
  \textit{C}ase page ($\rightarrow$ \pageref{OSUCList}ff.): Some of the chapters
  dedicated to specific open source licenses start with a license specific
  finder offering a set of license specific use cases---which, according to the
  complexity of the license, in some cases could be stripped down. But the
  disadvantage of this method is that you have to apply your knowledge about the
  use cases and their side effects by yourself without being systematically guided
  by the \oslic{} process.}

\tikzstyle{decision} = [diamond, draw, fill=gray!20, 
    text width=4.5em, text badly centered, node distance=4cm, inner sep=0pt]

\tikzstyle{preparation} = [rectangle, draw, fill=gray!30, 
    text width=11.5em, text centered, rounded corners, minimum height=3em]
 
\tikzstyle{lprocs} = [rectangle, draw, fill=gray!40, 
    text width=11.5em, text centered, rounded corners, minimum height=3em]
    
\tikzstyle{processing} = [rectangle, draw, fill=gray!40, node distance=2.4cm,
    text width=15em, text centered, rounded corners, minimum height=4em]
    
\tikzstyle{line} = [draw, -latex']

\tikzstyle{cloud} = [draw, ellipse, text centered, fill=gray!10]
 
    
\begin{tikzpicture}[node distance =1.5cm, auto]
\footnotesize
    % Place nodes
    
  \node [cloud, anchor=south, text width=7em] (start) at (1,10) 
    {$\forall$ open source \\ components};
  \node [preparation, below of=start] (select) 
    {select next open source component};     
  \node [preparation,  below of=select] (analyze) 
    {analyze its role as part of software system};
     
  \node [preparation,  below of=analyze] (determine) 
    {determine usage of final software product / service};     
    
  \node [preparation,  below of=determine] (detect) 
    {detect respective open source license};
    
  \node [lprocs,  below of=detect] (fillin)
    {\textbf{fill in the 5 query form} ($\rightarrow$ p.
    \pageref{OSLiCStandardFormForGatheringInformation})};
    
  \node [decision, right of=fillin] (success) {success?};
  
  \node [processing,  below of=success] (traverse)
    {\textbf{traverse} taxonomic \textbf{Open Source Use Case Finder}
    ($\rightarrow$ \pageref{OSLiCUseCaseFinder}) \& jump to indicated
    \textbf{O}pen \textbf{S}ource \textbf{U}se \textbf{C}ase page ($\rightarrow$
    \pageref{OSUCList}ff.)};
    
  \node [processing,  below of=traverse] (find)
    {\textbf{Determine} page of \textbf{license and use case specific to-do
    list} being presented in license specific chapter};
 
  \node [processing,  below of=find] (process)
    {Jump to indicated page \& \textbf{process license and use case specific
    to-do list} ($\rightarrow$ \pageref{OSUCToDoLists}ff.)};
    
  \node [decision, right of=process] (other) {more?};
  \node [cloud, right of=other, anchor=west] (stop) {stop};

  \path [line] (start) -- (select);  
     
  \path [line] (select) -- (analyze);      
  \path [line] (analyze) -- (determine);
  \path [line] (determine) -- (detect);       
  \path [line] (detect) -- (fillin);
  \path [line] (fillin) -- (success);
  
  \path [line] (success) |- node [near start] {no} (analyze);
  \path [line] (success) -- node [near start] {yes} (traverse);             
  
  \path [line] (traverse) -- (find);              
  \path [line] (find) -- (process);
  \path [line] (process) -- (other);

  \path [line] (other) |- node [near start] {yes} (select);
  \path [line] (other) -- node [near start] {no} (stop);                      

\end{tikzpicture}

Second, the readers might wish to comprehend the whole analysis. So, we briefly
discuss open source license taxonomies as the basis for a license compliant
behavior.\footnote{$\rightarrow$ OSLIC \enquote{\nameref{sec:LicenseTaxonomies}},
pp.\ \pageref{sec:LicenseTaxonomies}}  We consider some side effects of acting
according to the open source licenses.\footnote{$\rightarrow$ OSLiC
\enquote{\nameref{sec:SideEffects}}, pp.\ \pageref{sec:SideEffects}} Finally,
we study the structure of open source use cases.\footnote{$\rightarrow$ OSLiC
\enquote{\nameref{sec:OSUCdeduction}}, pp.\ \pageref{sec:OSUCdeduction}}

So, let us close our introduction by using, modifying, and (re)distributing a
well known wish of a well known man: Happy (Legally) Hacking.

%\bibliography{../../../bibfiles/oscResourcesEn}

% Local Variables:
% mode: latex
% fill-column: 80
% End:
