% Telekom osCompendium 'for being included' snippet template
%
% (c) Karsten Reincke, Deutsche Telekom AG, Darmstadt 2011
%
% This LaTeX-File is licensed under the Creative Commons Attribution-ShareAlike
% 3.0 Germany License (http://creativecommons.org/licenses/by-sa/3.0/de/): Feel
% free 'to share (to copy, distribute and transmit)' or 'to remix (to adapt)'
% it, if you '... distribute the resulting work under the same or similar
% license to this one' and if you respect how 'you must attribute the work in
% the manner specified by the author ...':
%
% In an internet based reuse please link the reused parts to www.telekom.com and
% mention the original authors and Deutsche Telekom AG in a suitable manner. In
% a paper-like reuse please insert a short hint to www.telekom.com and to the
% original authors and Deutsche Telekom AG into your preface. For normal
% quotations please use the scientific standard to cite.
%
% [ Framework derived from 'mind your Scholar Research Framework' 
%   mycsrf (c) K. Reincke 2012 CC BY 3.0  http://mycsrf.fodina.de/ ]
%


%% use all entries of the bibliography
%\nocite{*}

\chapter{Open Source Use Cases: Concept and Taxonomy}\label{sec:OSUCdeduction}

\footnotesize
\begin{quote}\itshape
This chapter establishes our concept of \emph{Open Source Use Cases} as a
classification system for to-do lists. The conditions of a specific license, in
the context of a par\-ti\-cu\-lar \emph{Open Source Use Case}, will be fulfilled
by following the corresponding to-do list. Additionally this chapter introduces a
taxonomy for these \emph{Open Source Use Cases}. Later on, this taxonomy will
organize the \emph{Open Source Use Case Finder}.
\end{quote}
\normalsize{}


After all these introductory remarks, we can summarize our idea. We know that
the right to use Open Source Software depends on the tasks required by the Open
Source Licenses. As opposed to commercial licenses, you can not buy the right to
use a piece of Open Source Software using money. It is embedded into the Open
Source Definition that the right to use the software may not be sold. The OSD
states firstly that an Open Source License may \enquote{[\ldots] not restrict any
party from selling or giving away the software as a component of (any) aggregate
software distribution}, and adds secondly in the same context that an Open
Source License \enquote{[\ldots] shall not require a royalty or other fee for such
sale}\footcite[cf.][\nopage wp. §1]{OSI2012a}.

However, it would be wrong to conclude that you are automatically allowed to use
Open Source Software without any service in return: generally you have to do
something to gain the right to use the software. In other words: Open Source
Software is covered by the idea of ’paying by doing’. As such, Open Source
Li\-cen\-ses describe specific circumstances under which the user must execute
some tasks in order to be compliant with the licenses. So, if we want to offer
to-do lists for fulfilling license conditions, we must consider these tasks and
circumstances.

In practice, such circumstances are not linear and simple. They contain
combinations of (sometimes context sensitive) conditions which can be grouped
into classes of tokens. Such a class of tokens might denote a feature of the
software itself - such as being an application or a library. Or it can refer to
the circumstances of using the software, such as 'using the software only for
yourself' or 'distributing the software also to third parties'.

At the end, we want to determine a set of specific OSUCs - the \emph{Open Source
Use Cases}. And we want to deliver for each of these OSUCs and for each of the
considered Open Source Licenses one list of actions which fulfills the license
in that context\footnote{Fortunately, sometimes one task list fulfills the
conditions of more than one use case - a welcome reduction of complexity}.

Such an \emph{Open Source Use Case} shall be considered as a set of tokens
describing the circumstances of a specific usage. Hence, to begin, we must
specify the relevant classes of tokens, before we can determine the valid
combinations of these tokens - our \emph{Open Source Use Cases}. Finally, based
on the tokens, we generate a taxonomy in form of a tree. This tree will become
the base of the \emph{Open Source Use Case Finder} which will be offered by the
next chapter, and which leads you to your specific OSUC by evaluating just a few
questions and answers.

There are only a handful of tokens which are relevant to the circumstances of
Open Source Software Licenses:

\label{OsucTokens}
\begin{itemize}
  \item The \textbf{\underline{type} of the Open Source Software}: On one hand
  there are code snippets, modules, libraries and plugins, and on the other
  hand, autonomous applications, programs and servers. We’ll coin the word
  ’snimolis’ for the first set, and ’proapses’ for the second. This is
  necessary, as we are not only talking about libraries and applications in the
  everyday sense, but rather in the broadest sense\footnote{Of course, our newly
  introduced concepts 'snimoli' and 'proapse' are not absolutely one of the most
  elegant words. So, initially we tried to talk about 'applications' and
  'libraries', although in our context these words should denote more, than they
  traditionally do. But we couldn't minimize the irritations of our
  interlocutors. Too often we had to amend that we were not only talking about
  applications and libraries in the strict sense of the words. Finally we
  decided to find our own words - and to stay open for better proposals ;-) }.
  More specifically, we will ask you, whether the Open Source Software, you
  want to use, is an includable code snippet, a linkable module or library, or a
  loadable plugin, or whether it is an autonomous application or server which
  can be executed or processed. In the first case, the answer should be 'it's a
  \underline{snimoli}', in the second 'it's a \underline{proapse}'.

  \item The \textbf{\underline{state} of the of the Open Source Software}: It
  might be used, as one has got it. Or it can be modified, before being used.
  More specifically, we will ask you, whether you want to leave the Open Source
  Software as you have got it, or whether you want to modify it before using
  and/or distributing it to 3rd parties. In the first case, the answer should be
  '\underline{unmodified}', in the second '\underline{modified}'.
  
  \item The \textbf{usage \underline{context} of the of the Open Source
  Software}: On the hand you might use the received Open Source Software as a
  readily prepared application. On the other hand you might embed the received
  Open Source into a larger application as one of its' components. More
  specifically, we will ask you, whether you are you using the Open Source
  Software as an autonomous piece of software, or whether you are using it as an
  embedded part of a larger, more complex piece of software. In the first case,
  the answer should be '\underline{independent}', in the second
  '\underline{embedded}'.
  
  \item The \textbf{\underline{recipient} of the of the Open Source Software}:
  Sometimes you might wish to use the received Open Source Software only for
  yourself. In other cases you might intend to hand over the software (also) to
  other people. More specifically, we will ask you, whether you are going to use
  the Open Source Software only for yourself, or whether you plan to
  (re)distribute it (also) to third parties. In the first case, the answer
  should be '\underline{4yourself}', in the second '\underline{4others}'.
 
  \item The \textbf{\underline{mode} of combination}: In this case, we will ask
  you, whether you are going to combine or to embed the Open Source Software
  with other software components by linking them statically or dynamically, or
  by textually including (parts of) the Open Source Software into your larger
  product. In the first case, the answer should be '\underline{statically
  linked}', in the second '\underline{dynamically linked}', in the third
  '\underline{textually included}'
  
\end{itemize}

From a more programmatic point-of-view, we can summarize these tokens as
follows:

\begin{itemize}
  \item \texttt{type::snimoli} \emph{or} \texttt{type::proapse}
  \item \texttt{state::unmodified} \emph{or} \texttt{state::modified}
  \item \texttt{context::independent} \emph{or} \texttt{context::embedded}
  \item \texttt{recipient::4yourself} \emph{or} \texttt{recipient::4others}
  \item \texttt{mode::statically-linked} \emph{or} \texttt{mode::dynamically-linked}
   \emph{or} \\ \texttt{mode::textually-included}
\end{itemize}

We already defined an Open Source Use Case as a combination of these tokens. If
we simply combine all these tokens of all these classes with all the tokens of
the other classes\footnote{in the sense of the cross product TYPE $\times$ STATE
$\times$ CONTEXT $\times$ RECIPIENT $\times$ MODE}, we get 2*2*2*2*3 = 48 sets
of tokens - or 48 \emph{Open Source Use Cases}. Fortunately, some of the
generated sets are invalid from an empirical or logical view, and some of these
sets are context sensitive:
\label{InvalidFinderTokenCombinations}

\begin{enumerate}
  
  \item It makes no sense to ask you
  whether you are going to combine the received software with other software
  components by linking them statically or dynamically, or by including it
  textually into a larger unit, if you already have answered that the received
  Open Source Software is a \emph{proapse} or that it shall be used
  \emph{independently}: A readily prepared application or server can't be linked
  to another application or server which also contains a
  \texttt{main}-function. And using a \emph{proapse} or \emph{snimoli}
  \emph{independently} includes that it is used \emph{not in combination} with
  other units, simply because they are tokens of the same class.
  
  \item If you already have specified that the used Open Source Software is a
  \emph{proapse} - hence an autonomous program, an application, or a server -,
  then your answer includes that the software is used independently and is not
  embedded with other components into a larger unit - simply because of the
  nature of all \emph{proapses}. But if you have specified that the used Open
  Source Software is a \emph{snimoli} - hence a snippet of code, a module, a
  plugin, or a library -, then it can indeed be used as an embedded component of
  a constructed larger application or server, or it can be used independently in
  case you 'only' re-distribute it to 3rd. parties.
  
  \item If you already have specified that the used Open Source Software is a
  \emph{snimoli} - hence a snippet of code, a module, a plugin, or a library -,
  and that this \emph{snimoli} shall be used only by yourself (not distributed
  to other 3rd. parties), then your answer must also imply that this
  \emph{snimoli} is used in combination, as an embedded part of a larger unit.
  It makes no sense to 'try' to use a library autonomously, without using it
  as component of another application. In this case, it would simply sit on the
  disk and would do nothing more than occupying space.

\end{enumerate}

Does this sound complex? We thought so, too. We spent much time explaining these
constraints to ourselves, and only when we had transposed all the combinations
and rules into a tree, did the situation become clearer. The following diagrams
shall summarize this way of clarification:

\section{The OS Use Case Dimensions: Classes and Tokens}


\begin{footnotesize}

\pstree[treemode=R, levelsep=*0.2, treesep=0.6]{\Toval{tokens}}{ 
    \pstree[]{\Tr{\fbox{type?}}}{
      \Toval{proapse} 
      \Toval{snimoli}
    }
    \pstree[]{\Tr{\fbox{state?}}}{
      \Toval{unmodified} 
      \Toval{modified}
    }
    \pstree[]{\Tr{\fbox{context?}}}{
      \Toval{independent} 
      \pstree[]{\Toval{embedded}}{
        \pstree[]{\Tr{\fbox{mode?}}}{
          \Toval{statically linked}
          \Toval{dynamically linked}
          \Toval{textually included}
        }
      }
    }
    \pstree[]{\Tr{\fbox{recipient?}}}{
      \Toval{4yourself} 
      \Toval{4others}
    }
  }

\end{footnotesize}  
  

\section{The OS Use Case Taxonomy}

This is one tree, 'collecting' the tokens and offering the \emph{Open Source Use
Cases} as their leafs\footnote{ Each of the invalid use cases (= sets of tokens)
[for details s. p. \pageref{InvalidFinderTokenCombinations}] is marked by an
\lightning{} and leads to an empty set (= $\varnothing$): A proapse can not be
embedded with another software unit, also containing a main-function. Using
a software library only for yourself and independent (not in combination with larger
software unit), is like having an unused heap of bytes on your disc.}:

\label{OsucDefinitionTree}
\begin{tiny}
\pstree[treemode=R,levelsep=*0.2, treesep=0.2]{\Toval{OSS}}{ 
    \pstree[]{\Tr{\fbox{type?}}}{
      \pstree[]{\Tr{\parbox{4em}{\texttt{\{\underline{proapse}\}}}}}{
        \pstree[]{\Tr{\fbox{state?}}}{
          \pstree[]{\Tr{\parbox{5.2em}{
                             \texttt{\{proapse,\\
                             \hspace*{0.5em}\underline{unmodified}\}}}}}{
            \pstree[]{\Tr{\fbox{context?}}}{
              \pstree[]{\Tr{\parbox{5.2em}{
                             \texttt{\{proapse,\\
                             \hspace*{0.5em}unmodified,\\
                             \hspace*{0.5em}\underline{independent}\}}}}}{
                \pstree[]{\Tr{\fbox{recipient?}}}{
                
                  \pstree[]{\Toval{\bfseries{OSUC-01}}}{
              
                    \Tr{\parbox{5.2em}{
                             \texttt{\{proapse,\\
                             \hspace*{0.5em}unmodified,\\
                             \hspace*{0.5em}independent,\\
                             \hspace*{0.5em}\underline{4yourself}\}}}}
                   }                            
                  \pstree[]{\Toval{\bfseries{OSUC-02}}}{
                             
                    \Tr{\parbox{5.2em}{
                             \texttt{\{proapse,\\
                             \hspace*{0.5em}unmodified,\\
                             \hspace*{0.5em}independent,\\
                             \hspace*{0.5em}\underline{4others}\}}}}
                    }         
                }
              }
              \pstree[]{\Tr{\parbox{5.4em}{
                             \texttt{\{proapse,\\
                             \hspace*{0.5em}unmodified,\\
                             \hspace*{0.5em}\underline{embedded}
                             \bfseries{\lightning}\}}}}}{
                \Tr{$\varnothing$}
              }              
            }
          }
          \pstree[]{\Tr{\parbox{5.2em}{
                             \texttt{\{proapse,\\
                             \hspace*{0.5em}\underline{modified}\}}}}}{
            \pstree[]{\Tr{\fbox{context?}}}{
              \pstree[]{\Tr{\parbox{5.2em}{
                             \texttt{\{proapse,\\
                             \hspace*{0.5em}modified,\\
                             \hspace*{0.5em}\underline{independent}\}}}}}{
                \pstree[]{\Tr{\fbox{recipient?}}}{
                  \pstree[]{\Toval{\bfseries{OSUC-03}}}{
                    \Tr{\parbox{5.2em}{
                             \texttt{\{proapse,\\
                             \hspace*{0.5em}modified,\\
                             \hspace*{0.5em}independent,\\
                             \hspace*{0.5em}\underline{4yourself}\}}}}
                   }
                   \pstree[]{\Toval{\bfseries{OSUC-04}}}{         
                    \Tr{\parbox{5.2em}{
                             \texttt{\{proapse,\\
                             \hspace*{0.5em}modified,\\
                             \hspace*{0.5em}independent,\\
                             \hspace*{0.5em}\underline{4others}\}}}}
                   }         
                }
              }
              \pstree[]{\Tr{\parbox{5.4em}{
                             \texttt{\{proapse,\\
                             \hspace*{0.5em}modified,\\
                             \hspace*{0.5em}\underline{embedded}
                             \bfseries{\lightning}\}}}}}{                          
                \Tr{$\varnothing$}
              }              
            }
          }
        }
      } 
      \pstree[]{\Tr{\parbox{4em}{\texttt{\{\underline{snimoli}\}}}}}{
        \pstree[]{\Tr{\fbox{state?}}}{
          \pstree[]{\Tr{\parbox{5.2em}{
                             \texttt{\{snimoli,\\
                             \hspace*{0.5em}\underline{unmodified}\}}}}}{
            \pstree[]{\Tr{\fbox{context?}}}{
              \pstree[]{\Tr{\parbox{5.2em}{
                             \texttt{\{snimoli,\\
                             \hspace*{0.5em}unmodified,\\
                             \hspace*{0.5em}\underline{independent}\}}}}}{
                \pstree[]{\Tr{\fbox{recipient?}}}{
                  \pstree[]{
                    \Tr{\parbox{5.8em}{
                             \texttt{\{snimoli,\\
                             \hspace*{0.5em}unmodified,\\
                             \hspace*{0.5em}independent,\\
                             \hspace*{0.5em}\underline{4yourself}
                             \bfseries{\lightning}\}}}}
                     }{\Tr{$\varnothing$}}        
                             
                             
                   \pstree[]{\Toval{\bfseries{OSUC-05}}}{        
                    \Tr{\parbox{5.2em}{
                             \texttt{\{snimoli,\\
                             \hspace*{0.5em}unmodified,\\
                             \hspace*{0.5em}independent,\\
                             \hspace*{0.5em}\underline{4others}\}}}}
                    }         
                }
              }
              
              
              \pstree[]{\Tr{\parbox{5.2em}{
                             \texttt{\{snimoli,\\
                             \hspace*{0.5em}unmodified,\\
                             \hspace*{0.5em}\underline{embedded}\}}}}}{
                \pstree[]{\Tr{\fbox{recipient?}}}{
                  \pstree[]{\Toval{\bfseries{OSUC-06}}}{
                    \pstree[]{\Tr{\parbox{5.2em}{
                             \texttt{\{snimoli,\\
                             \hspace*{0.5em}unmodified,\\
                             \hspace*{0.5em}embedded,\\
                             \hspace*{0.5em}\underline{4yourself}\}}}}}{
                               \pstree[]{\Tr{\fbox{$\cup$ mode?}}}{
                                 \Tr{\parbox{5.6em}{\underline{\textit{OSUC-06a}}\\
                                 \texttt{\{statically\\
                                  \hspace*{0.5em}linked\}}}}
                                \Tr{\parbox{5.6em}{\underline{\textit{OSUC-06b}}\\
                                \texttt{\{dynamically\\
                                  \hspace*{0.5em}linked\}}}}
                                \Tr{\parbox{5.6em}{\underline{\textit{OSUC-06c}}\\
                                \texttt{\{textually\\
                                  \hspace*{0.5em}included\}}}}
                               }
                             }
                    }
                    \pstree[]{\Toval{\bfseries{OSUC-07}}}{        
                      \pstree[]{\Tr{\parbox{5.2em}{
                             \texttt{\{snimoli,\\
                             \hspace*{0.5em}unmodified,\\
                             \hspace*{0.5em}embedded,\\
                             \hspace*{0.5em}\underline{4others}\}}}}}{
                               \pstree[]{\Tr{\fbox{$\cup$ mode?}}}{
                                \Tr{\parbox{5.6em}{\underline{\textit{OSUC-07a}}\\
                                \texttt{\{statically\\
                                  \hspace*{0.5em}linked\}}}}
                                \Tr{\parbox{5.26em}{\underline{\textit{OSUC-07b}}\\
                                \texttt{\{dynamically\\
                                  \hspace*{0.5em}linked\}}}}
                                \Tr{\parbox{5.6em}{\underline{\textit{OSUC-07c}}\\
                                \texttt{\{textually\\
                                  \hspace*{0.5em}included\}}}}
                               }
                             }
                       }     
                 }
              }
              
                            
            }
          }
          \pstree[]{\Tr{\parbox{5.2em}{
                             \texttt{\{snimoli,\\
                             \hspace*{0.5em}\underline{modified}\}}}}}{
            \pstree[]{\Tr{\fbox{context?}}}{
              
              \pstree[]{\Tr{\parbox{5.2em}{
                             \texttt{\{snimoli,\\
                             \hspace*{0.5em}modified,\\
                             \hspace*{0.5em}\underline{independent}\}}}}}{
                \pstree[]{\Tr{\fbox{recipient?}}}{
                  \pstree[]{
                    \Tr{\parbox{5.8em}{
                             \texttt{\{snimoli,\\
                             \hspace*{0.5em}modified,\\
                             \hspace*{0.5em}independent,\\
                             \hspace*{0.5em}\underline{4yourself}
                             \bfseries{\lightning}\}}}}
                     }{\Tr{$\varnothing$}}         
                             
                   \pstree[]{\Toval{\bfseries{OSUC-08}}}{              
                    \Tr{\parbox{5.2em}{
                             \texttt{\{snimoli,\\
                             \hspace*{0.5em}modified,\\
                             \hspace*{0.5em}independent,\\
                             \hspace*{0.5em}\underline{4others}\}}}}
                    }
                }
              }
              
              \pstree[]{\Tr{\parbox{5.2em}{
                             \texttt{\{snimoli,\\
                             \hspace*{0.5em}modified,\\
                             \hspace*{0.5em}\underline{embedded}\}}}}}{
                \pstree[]{\Tr{\fbox{recipient?}}}{
                  \pstree[]{\Toval{\bfseries{OSUC-09}}}{     
                    \pstree[]{\Tr{\parbox{5.2em}{
                             \texttt{\{snimoli,\\
                             \hspace*{0.5em}modified,\\
                             \hspace*{0.5em}embedded,\\
                             \hspace*{0.5em}\underline{4yourself}\}}}}}{
                               \pstree[]{\Tr{\fbox{$\cup$ mode?}}}{
                                 \Tr{\parbox{5.6em}{\underline{\textit{OSUC-09a}}\\
                                 \texttt{\{statically\\
                                  \hspace*{0.5em}linked\}}}}
                                \Tr{\parbox{5.6em}{\underline{\textit{OSUC-09b}}\\
                                \texttt{\{dynamically\\
                                \hspace*{0.5em}linked\}}}}
                                \Tr{\parbox{5.6em}{\underline{\textit{OSUC-09c}}\\
                                \texttt{\{textually\\
                                  \hspace*{0.5em}included\}}}}
                               }
                             }
                           }
                   \pstree[]{\Toval{\bfseries{OSUC-10}}}{           
                    \pstree[]{\Tr{\parbox{5.2em}{
                             \texttt{\{snimoli,\\
                             \hspace*{0.5em}modified,\\
                             \hspace*{0.5em}embedded,\\
                             \hspace*{0.5em}\underline{4others}\}}}}}{
                               \pstree[]{\Tr{\fbox{$\cup$ mode?}}}{
                                \Tr{\parbox{5.6em}{\underline{\textit{OSUC-10a}}\\
                                \texttt{\{statically\\
                                  \hspace*{0.5em}linked\}}}}
                                \Tr{\parbox{5.6em}{\underline{\textit{OSUC-10b}}\\
                                \texttt{\{dynamically\\
                                  \hspace*{0.5em}linked\}}}}
                                \Tr{\parbox{5.2em}{\underline{\textit{OSUC-10c}}\\
                                \texttt{\{textually\\
                                  \hspace*{0.5em}included\}}}}
                               }
                             }
                      }
                 }
              }             
            }
          }
        }
      }      
    }
  }
\end{tiny}


%\bibliography{../../../bibfiles/oscResourcesEn}
