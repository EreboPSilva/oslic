% Telekom osCompendium cloak file English text
%
% (c) Karsten Reincke, Deutsche Telekom AG, Darmstadt 2011
%
% This LaTeX-File is licensed under the Creative Commons Attribution-ShareAlike
% 3.0 Germany License (http://creativecommons.org/licenses/by-sa/3.0/de/): Feel
% free 'to share (to copy, distribute and transmit)' or 'to remix (to adapt)'
% it, if you '... distribute the resulting work under the same or similar
% license to this one' and if you respect how 'you must attribute the work in
% the manner specified by the author ...':
%
% In an internet based reuse please link the reused parts to www.telekom.com and
% mention the original authors and Deutsche Telekom AG in a suitable manner. In
% a paper-like reuse please insert a short hint to www.telekom.com and to the
% original authors and Deutsche Telekom AG into your preface. For normal
% quotations please use the scientific standard to cite.
%
% [ File structure derived from 'mind your Scholar Research Framework' 
%   mycsrf (c) K. Reincke CC BY 3.0  http://mycsrf.fodina.de/ ]

%\documentclass[DIV=calc,BCOR=5mm,12pt,headings=small,oneside,toc=bib,draft]{scrbook}
\documentclass[DIV=calc,BCOR=5mm,12pt,headings=small,oneside,toc=bib]{scrartcl}

%%% (1) general configurations %%%
\usepackage[utf8]{inputenc}

%%% (2) language specific configurations %%%
\usepackage[]{a4}
\usepackage[english]{babel}
\selectlanguage{english}
\usepackage[automark]{scrpage2}

\usepackage{microtype}

%language specific quoting signs
%default for language emglish is american style of quotes
%\usepackage[english=british]{csquotes}
\usepackage[english=american]{csquotes}

% jurabib configuration
\usepackage[see]{jurabib}
\bibliographystyle{jurabib}
% do not comment the litrature any longer
% Telekom osCompendium English Jurabib Configuration Include Module 
%
% (c) Karsten Reincke, Deutsche Telekom AG, Darmstadt 2011
%
% This LaTeX-File is licensed under the Creative Commons Attribution-ShareAlike
% 3.0 Germany License (http://creativecommons.org/licenses/by-sa/3.0/de/): Feel
% free 'to share (to copy, distribute and transmit)' or 'to remix (to adapt)'
% it, if you '... distribute the resulting work under the same or similar
% license to this one' and if you respect how 'you must attribute the work in
% the manner specified by the author ...':
%
% In an internet based reuse please link the reused parts to www.telekom.com and
% mention the original authors and Deutsche Telekom AG in a suitable manner. In
% a paper-like reuse please insert a short hint to www.telekom.com and to the
% original authors and Deutsche Telekom AG into your preface. For normal
% quotations please use the scientific standard to cite.
%
% [ File structure derived from 'mind your Scholar Research Framework' 
%   mycsrf (c) K. Reincke CC BY 3.0  http://mycsrf.fodina.de/ ]

% the first time cite with all data, later with shorttitle
\jurabibsetup{citefull=first}

%%% (1) author / editor list configuration
%\jurabibsetup{authorformat=and} % uses 'und' instead of 'u.'
% therefore define your own abbreviated conjunction: 
% an 'and before last author explicetly written conjunction

% for authors in citations
\renewcommand*{\jbbtasep}{ a.\ } % bta = between two authors sep
\renewcommand*{\jbbfsasep}{, } % bfsa = between first and second author sep
\renewcommand*{\jbbstasep}{, a.\ }% bsta = between second and third author sep
% for editors in citations
\renewcommand*{\jbbtesep}{ a.\ } % bta = between two authors sep
\renewcommand*{\jbbfsesep}{, } % bfsa = between first and second author sep
\renewcommand*{\jbbstesep}{, a.\ }% bsta = between second and third author sep

% for authors in literature list
\renewcommand*{\bibbtasep}{ a.\ } % bta = between two authors sep
\renewcommand*{\bibbfsasep}{, } % bfsa = between first and second author sep
\renewcommand*{\bibbstasep}{, a.\ }% bsta = between second and third author sep
% for editors  in literature list
\renewcommand*{\bibbtesep}{ a.\ } % bte = between two editors sep
\renewcommand*{\bibbfsesep}{, } % bfse = between first and second editor sep
\renewcommand*{\bibbstesep}{, a.\ }% bste = between second and third editor sep

% use: name, forname, forname lastname u. forname lastname
\jurabibsetup{authorformat=firstnotreversed}
\jurabibsetup{authorformat=italic}

%%% (2) title configuration
% in every case print the title, let it be seperated from the 
% author by a colon and use the slanted font
\jurabibsetup{titleformat={all,colonsep}}
%\renewcommand*{\jbtitlefont}{\textit}

%%% (3) seperators in bib data
% separate bibliographical hints and page hints by a comma
\jurabibsetup{commabeforerest}

%%% (4) specific configuration of bibdata in quotes / footnote
% use a.a.O if possible
\jurabibsetup{ibidem=strict}
% replace ugly a.a.O. by translation of ders., a.a.O.
\AddTo\bibsgerman{
  \renewcommand*{\ibidemname}{Id., l.c.}
  \renewcommand*{\ibidemmidname}{id., l.c.}
}
\renewcommand*{\samepageibidemname}{Id., ibid.}
\renewcommand*{\samepageibidemmidname}{id., ibid.}


%%% (5) specific configuration of bibdata in bibliography
% ever an in: before journal and collection/book-tiltes 
\renewcommand*{\bibbtsep}{in: }
\renewcommand*{\bibjtsep}{in: }
% ever a colon after author names 
\renewcommand*{\bibansep}{: }
% ever a semi colon after the title
% \AddTo\bibsgerman{\renewcommand*{\urldatecomment}{Referenzdownload: }}
\renewcommand*{\bibatsep}{; }
% ever a comma before date/year
\renewcommand*{\bibbdsep}{, }

% let jurabib insert the S. and p. information
% no S. necessary in bib-files and in cites/footcites
\jurabibsetup{pages=format}

% use a compressed literature-list using a small line indent
\jurabibsetup{bibformat=compress}
\setlength{\jbbibhang}{1em}

% which follows the design of the cites and offers comments
\jurabibsetup{biblikecite}

% print annotations into bibliography
% \jurabibsetup{annote}
% \renewcommand*{\jbannoteformat}[1]{{ \itshape #1 }}

%refine the prefix of url download
\AddTo\bibsgerman{\renewcommand*{\urldatecomment}{reference download: }}

% we want to have the year of articles in brackets
\renewcommand*{\bibaldelim}{(}
\renewcommand*{\bibardelim}{)}

% in english version Nr. must be replaced by No.
\renewcommand*{\artnumberformat}[1]{\unskip,\space No.~#1}
\renewcommand*{\pernumberformat}[1]{\unskip\space No.~#1}%
\renewcommand*{\revnumberformat}[1]{\unskip\space No.~#1}%


%Reformatierung Seriestitels and Seriesnumber
\DeclareRobustCommand{\numberandseries}[2]{%
\unskip\unskip%,
\space\bibsnfont{(=~#2}%
\ifthenelse{\equal{#1}{}}{)}{, [Vol./No.]~#1)}%
}%


% Local Variables:
% mode: latex
% fill-column: 80
% End:
 %no annotations

% language specific hyphenation
% Telekom osCompendium osHyphenation Include Module
%
% (c) Karsten Reincke, Deutsche Telekom AG, Darmstadt 2011
%
% This LaTeX-File is licensed under the Creative Commons Attribution-ShareAlike
% 3.0 Germany License (http://creativecommons.org/licenses/by-sa/3.0/de/): Feel
% free 'to share (to copy, distribute and transmit)' or 'to remix (to adapt)'
% it, if you '... distribute the resulting work under the same or similar
% license to this one' and if you respect how 'you must attribute the work in
% the manner specified by the author ...':
%
% In an internet based reuse please link the reused parts to www.telekom.com and
% mention the original authors and Deutsche Telekom AG in a suitable manner. In
% a paper-like reuse please insert a short hint to www.telekom.com and to the
% original authors and Deutsche Telekom AG into your preface. For normal
% quotations please use the scientific standard to cite.
%
% [ File structure derived from 'mind your Scholar Research Framework' 
%   mycsrf (c) K. Reincke CC BY 3.0  http://mycsrf.fodina.de/ ]
%


\hyphenation{rein-cke}
\hyphenation{OS-LiC}
\hyphenation{ori-gi-nal}


%%% (3) layout page configuration %%%

% select the visible parts of a page
% S.31: { plain|empty|headings|myheadings }
%\pagestyle{myheadings}
\pagestyle{scrheadings}

% select the wished style of page-numbering
% S.32: { arabic,roman,Roman,alph,Alph }
\pagenumbering{arabic}
\setcounter{page}{1}

% select the wished distances using the general setlength order:
% S.34 { baselineskip| parskip | parindent }
% - general no indent for paragraphs
\setlength{\parindent}{0pt}
\setlength{\parskip}{1.2ex plus 0.2ex minus 0.2ex}

%%% (4) general package activation %%%
%\usepackage{utopia}
%\usepackage{courier}
%\usepackage{avant}

% graphic
\usepackage{graphicx,color}
\usepackage{array}
\usepackage{shadow}
\usepackage{fancybox}
\usepackage{alltt}

%- start(footnote-configuration)
%  flush the cite numbers out of the vertical line and let
%  the footnote text directly start in the left vertical line
% \usepackage[marginal,hang]{footmisc}
% \renewcommand\footnotemargin{1.5em}

% formatting the footnote with koma script tools
% \deffootnote[1em]{1.5em}{1em}{\textsuperscript{\thefootnotemark}}
\deffootnote[1.5em]{1.5em}{1.5em}{\textsuperscript{\thefootnotemark)\ }}

%\deffootnote[0em]{1.5em}{1em}{\textsuperscript{\thefootnotemark}}
%- end(footnote-configuration)

% - additional packages

\usepackage{tikz}
\usetikzlibrary{arrows}
\usetikzlibrary{shapes,snakes}
\usetikzlibrary{positioning}
\usetikzlibrary{decorations.text}
\usetikzlibrary{trees}

\usepackage{multirow}

%RPD%%%\usepackage{blindtext}
\usepackage{caption}

\usetikzlibrary{matrix}

\usepackage{amsmath}
\usepackage{amsfonts}
\usepackage{amssymb}
\usepackage{wasysym}
\usepackage{chngcntr}
\usepackage{nameref}

%\counterwithout{footnote}{chapter}

\usepackage[intoc]{nomencl}
\let\abbr\nomenclature
% Modify Section Title of nomenclature
\renewcommand{\nomname}{Periodicals, Shortcuts, and Abbreviations}
%\renewcommand{\nomname}{Periodika, ihre Kurzformen und generelle Abkürzungen}

% insert point between abbrewviation and explanation
\setlength{\nomlabelwidth}{.24\hsize}
\renewcommand{\nomlabel}[1]{#1 \dotfill}
% reduce the line distance
\setlength{\nomitemsep}{-\parsep}
\makenomenclature

% depth of contents
\setcounter{secnumdepth}{5}
\setcounter{tocdepth}{5}

% Hyperlinks
\usepackage{hyperref}
\hypersetup{bookmarks=true,breaklinks=true,colorlinks=true,citecolor=blue,draft=false}

% Compatibility command if hyperref cannot be used
%\newcommand{\texorpdfstring}[2]{#1}

% Abbreviations
\newcommand{\oslic}{OSLiC}

\begin{document}

%% use all entries of the bliography
%\nocite{*}

%%-- start(titlepage)
\titlehead{Version 1.0}

\subject{\small \itshape How to Achieve Open Source License Compliance} 

\title{Open Source Software and Reverse Engineering}

\author{
Karsten Reincke\thanks{Deutsche Telekom AG, Products \& Innovation, 
T-Online-Allee 1, 64295 Darmstadt}
}

\maketitle
%%-- end(titlepage)

\normalsize

\begin{abstract}\noindent\emph{Sometimes, it is stated that software using LGPL
licensed components, may only be distributed by permitting reverse engineering
of the software which uses the LGPL components. By reading the licenses very
strictly, this article proves in detail that we are not obliged to permit
reverse engineering as long as we distribute our work as dynamically linkable
files. Particularly, the article shows, that we also can  compliantly distribute
software using LGPL components if we distribute our work in form of compiled,
but still not linked files.}
\end{abstract}

\section{Preface}

Reverse Engineering is a challenging issue: Sometimes, developers must protect
their business relevant work. Thus, they only incorporate open source components
published under a permissive or a weak copyleft license: Of course, they want to
use the open source software compliantly. But nevertheless, they want be able to
distribute their own work as closed software for not making their secrets known.
So, being obliged by the embedded components to permit reverse engineering of
the work using the components, would subvert this strategie. Unfortunately, it
is often said, that one of the most important weak copyleft license, the LGPL,
requires to permit reverse engineering.

During the last two years, I was repeatedly asked to explain and defend my
viewpoint in respect of reverse engineering and open source software: I am
profoundly convinced that there is no problem at all as long as you distribute
dynamically linkable programs. This position had often astonished my dialog
partners. Some of them were still looking for an exit from the inconsistent
situation, while the other already had given up and could rather believe that
there was such a simple solution. So, --- at the end of our discussions -- they
encouraged me to prove my position in detail.

For example, by my brave colleague Helene Tamer constantly insisted, that
Deutsche Telekom AG could not give up its restrictions to use LGPL libraries
until I had offered a reliable proof that the LGPL does not require reverse
engineering generally. Or I could enthousiasticly discuss the permission of
reverse engineering with the well informed German open source attorney Hendrik
Schöttle\footnote{$\rightarrow$
http://www.osborneclarke.com/lawyers/schottle-hendrik/} who boosted me to
understand that I have to argue more subtler than I had done in my first
answering eMail. And finally Karen Copenhaver\footnote{$\rightarrow$
http://www.choate.com/people/karen-copenhaver} suggested me additional
literature and emphatically asked me to share my thoughts with the community.

So, I thankfully\footnote{Following the spirit of the open source movement, this
article is published under the terms of CC-BY-SA 3.0 ($\rightarrow$ p.
\pageref{License}). But as a document concerning legal issues, it must  be
published under a specific proviso: ($\rightarrow$ p. \pageref{Disclaimer}). And
finally, it is first and foremost developed as a chapter of the \textbf{O}pen
\textbf{S}ource \textbf{Li}cense \textbf{C}ompendium ($\rightarrow$
http://www.oslic.org/. But for offering version which can more simply be
distributed, we also produced this extract.} can now offer a thoroughly
elaborated proof for the assertion that there - in general - is a way to
distribute open source software compliantly without permitting reverse
engineering, and that this way -- in particular -- is also usable to compliantly
distribute LGPL licensed software without allowing reverse engineering.

\footnotesize
\tableofcontents

%%%%%%%%%%%%

\section{Reverse Engineering and Open Source as Challenge}
% Telekom osCompendium 'for being included' snippet template
%
% (c) Karsten Reincke, Deutsche Telekom AG, Darmstadt 2011
%
% This LaTeX-File is licensed under the Creative Commons Attribution-ShareAlike
% 3.0 Germany License (http://creativecommons.org/licenses/by-sa/3.0/de/): Feel
% free 'to share (to copy, distribute and transmit)' or 'to remix (to adapt)'
% it, if you '... distribute the resulting work under the same or similar
% license to this one' and if you respect how 'you must attribute the work in
% the manner specified by the author ...':
%
% In an internet based reuse please link the reused parts to www.telekom.com and
% mention the original authors and Deutsche Telekom AG in a suitable manner. In
% a paper-like reuse please insert a short hint to www.telekom.com and to the
% original authors and Deutsche Telekom AG into your preface. For normal
% quotations please use the scientific standard to cite.
%
% [ Framework derived from 'mind your Scholar Research Framework' 
%   mycsrf (c) K. Reincke 2012 CC BY 3.0  http://mycsrf.fodina.de/ ]
%


%% use all entries of the bibliography
%\nocite{*}

Beyond any doubt, the LGPL mentions \enquote{reverse engineering}
literally\footnote{For the LGPL-v2 \cite[cf.][\nopage wp., 
§6]{Lgpl21OsiLicense1999a}; for the LGPL-v3 \cite[cf.][\nopage wp., 
§4]{Lgpl30OsiLicense2007a} } for indicating that reverse engineering in any
respect must be allowed to use and distribute LGPL software compliantly:

\begin{quote}\noindent\emph{\enquote{[\ldots] you may [\ldots] distribute a work
(containing portions of the Library) under terms of your choice, provided that
the terms permit [\ldots] \emph{reverse engineering} [\ldots]}
\footnote{\cite[cf.][\nopage wp, §6]{Lgpl21OsiLicense1999a}. The LGPL-v2 uses
the capitalized word \enquote{Library} for denoting a library which
\enquote{[\ldots] has been distributed under (the) terms} of the LGPL-v2.
(\cite[cf.][\nopage wp, §0]{Lgpl21OsiLicense1999a}). Inside of our LGPL
chapter(s) we follow this interpretation. } }
\end{quote}

There are three strategies for dealing with such provisions: one can try to
fully honor its meaning, one can mitigate its meaning, or one can avoid to
discuss this requirement altogether:

A first group of well known open source experts take the sentence of the LGPL-v2
as a strict rule which requires that one has to allow reverse engineering of the
whole software product if one embeds any LGPL licensed component into that
product\footnote{For example, a very trustworthy German expert states that the
LGPL-2.1 generally requires that a distributor of a program which accesses a
LGPL-2.1 licensed library, must grant his customer also the right to modify the
accessing program and hence also the right to execute reverse engineering.
Literally the German text says:
\begin{quote}\enquote{Ziffer 6 LGPLv2.1 knüpft die Erlaubnis, das zugreifende
Programm unter beliebigen Lizenzbestimmungen verbreiten zu drüfen, an eine Reihe
von Verpflichtungen, die in der Praxis oft übersehen werden: Zunächst muss dem
Kunden, dem die Software geliefert wird, die Veränderung des zugreifenden
Programms gestattet werden und zu diesem Zweck auch ein Reverse Engineering zur
Fehlerbehebung. \emph{Dies dürfte alle Formen des Debugging und das
Dekompilieren des zugreifenden Programms umfassen}.} (\cite[cf.][81; emphasis
KR]{JaeMet2011a}).\end{quote} At first glance, also \enquote{copyleft.org} --
the \enquote{[...] collaborative project to create and disseminate useful
information, tutorial material, and new policy ideas regarding all forms of
copyleft licensing} (\cite[cf.][\nopage wp.]{CopyLeftOrg2014a}) -- could be
taken as another witness for such an attitude of strict reading: Some of its
contributors elucidate in a chapter dealing with \enquote{special topics in
compliance} that \enquote{the license of the whole work must [sic!] permit
\enquote{reverse engineering for debugging such modifications} to the library}
and that one therefore \enquote{ should take care that the EULA used for the
Application does not contradict this
permission}(\cite[cf.][86]{KuhSebGin2014a}}.

A second group of well known and knowledgeable open source experts signify that
the LGPL-v2 indeed literally contains a strict rule, but that this rule actually
is not meant as it sounds: For example, two of these experts explain that
\enquote{these requirements on the licensed combination require that the license
chosen not prohibit the customer’s modification and reverse engineering for
debugging these modifications in the work as a whole}. But then they directly
add the limitation, that \enquote{in practice, enforcement history suggests, it
means that the license terms chosen may not prohibit modification and reverse
engineering for debugging of modification in the LGPL’d code included in the
combination}\footnote{\cite[cf.][\nopage wp., chapter LGPLv2.1, section
6]{MogCho2014a}. Such a mitigation can also be found in the tutorial of
copyleft.org: After they have summarized the LGPL-v2 sentence as a strict rule,
they directly continue, that one \enquote{[\ldots] must refrain from license
terms on works based on the licensed work that prohibit replacement of the
licensed components of the larger non-LGPL'd work, or prohibit decompilation or
reverse engineering in order to enhance or fix bugs in the LGPL'd components}
(\cite[cf.][86]{KuhSebGin2014a}). This added specification indicates, that one
only has to facilitate the modification of the library and that reverse
engineering can be ignored as long as there are other ways to improve the
embedded library.}.

Finally, a third group of experts prefers not to discuss the problem of reverse
engineering, although this technique is literally mentioned in the license and
although they want explain how to use GPL/LGPL licensed software
compliantly\footnote{An article of Terry J. Ilardi might be taken as a first
witness of this third strategy: he profoundly explains the essence of the LGPL,
he especially discusses §6, and he delivers applicable rules like \enquote{DO
NOT statically link to LGPL [\ldots] code if you wish to keep your program
proprietary}. But he does not discuss \emph{reverse engineering}
(\cite[cf.][5f]{Ilardi2010a}). Similarily argues Rosen
(\cite[cf.][121ff]{Rosen2005a}). And -- despite their comments on reverse
engineering in the specific chapter \emph{special topics in compliance} -- the
copyleft.org document can also be taken as an instance of this attitude:
Although its' authors recommend to \enquote{study chapter 10 carefully} for
establishing an adequate \enquote{compliance with LGPLv2.1}
(\cite[cf.][86]{KuhSebGin2014a}), this chapter 10 -- dedicated to the meaning of
the \enquote{Lesser GPL} -- does not deal with reverse engineering, although it
discusses the §6 of the LGPLv2.1 in depth (\cite[cf.][56ff, esp.
60f]{KuhSebGin2014a}).}.

This situation must bother companies and people who want to use open source
software compliantly and who therefore are looking for guidance. Particularly it
disturbs those who want to protect their business relevant software. At the end,
they might consider that this sentence is not consistently understood by the
open source community itself. And -- as far as we know -- at least some of these
companies preventively prohibit their developers to embed LGPL licensed
components into programs which contain business relevant techniques.
Unfortunately, this consequence does not only obstruct access to a large set of
well written free software, but it is scarcely possible to obey such an
interdiction consequently: The glibc, which enables the programms to talk with
the kernel of the GNU/Linux system\footnote{cf.
http://www.gnu.org/software/libc/}, is licensed under the LGPL\footnote{cf.
http://en.wikipedia.org/wiki/GNU\_C\_Library}. And hence, this library is
indirectly linked to or combined with any program running on the GNU/Linux
system. So, if the LGPL-v2 indeed required, that reverse engineering of every
program must be allowed, which contains portions of any LGPL Library, then every
GNU/Linux user would be allowed to examine every program running on GNU/Linux by
\emph{reverse engineering}, simply, because finally every 'GNU/Linux program' is
linked to or combined with the glibc\footnote{This conclusion might surprise.
But it is inferred with exactly the same arguments as the conclusion, that
without a licence offering a weaker copyleft every program would have been
licensed under the GPL. The copyleft.org document explains this argumentation in
great detail (\cite[cf.][56f]{KuhSebGin2014a}).}. In other words: if the LGPL
indeed required the permission of reverse engineering, then
every program executed on GNU/Linux may be reverse engineered.

But an exhaustive reading of the LGPL-v2 strongly indicates, that there must be
another valid, more 'liquid' understanding of the LGPL: The preamble explains
the reason for offering another weaker license beside the GPL. It says that
\enquote{[\ldots] on rare occasions, there may be a special need to encourage
the widest possible use of a certain library, so that it becomes a de-facto
standard} and that therefore it could be strategicly necessary to \enquote{allow
[\ldots] non-free programs [\ldots] to use the library} without enforcing that
these programs become free software too\footcite[cf.][\nopage wp,
§preamble]{Lgpl21OsiLicense1999a}.

So, if the LGPL had indeed determined that every program linked to or combined
with any LGPL library may be reverse engineered, then the LGPL would have an
effect contrary to its own intention. It would have introduced something like
\emph{'security by obscurity'}: First, the LGPL would allow to protect the
internals of your own work against investigation because the code of the
non-free programm using the library, does not necessarily have to be published
as well\footnote{The weak copyleft has been introduced for encouring the widest
possible use of the library.}. But then -- in the end -- the LGPL would also
allow the user to reverse engineer the received binary and hence would enable
him to nevertheless discover all internals\footnote{It would only cost a little
more effort - as security by obscurity indicates.}. By this means, the LGPL-v2
would undermine its' own raison d'$\grave{e}$tre introduced by its' inventors:
under such circumstances there probably would be less hope that any LGPL library
could have become a defacto standard.

We know that the inventors of the GNU licenses and GNU software are very
sophisticated experts. They never would have published such an inconsistent
document. So, this dissent read in(to) the document is a strong indicator for
the fact, that there must be a better way to understand the license. And thus,
it is up to us, the followers, to explicate a more adequate interpretation. Of
course, such an interpretation must be grounded on the written text. We, the
scholars, are not allowed to add our own wishes. We must read the license very
strictly. We have to deduce 'understandings' only by matching the
interpretations explicitly, strictly, and reasonably back to the license text
itself.


NEU FORMULIEREN: start

Encouraged by the indication that a better understanding of the license may
exist and contrary to the other strategies, we are going to prove that, in
fact, none of the open source licenses\footnote{being discussed in the OSLiC} in
general require to allow reverse engineering of software containing a component
licensed under that open source license. In particular, we will prove, that even
the LGPL does not claim this permission generally: We want to explain, why the
LGPL only requires to permit reverse engineering if and only if the LGPL
licensed component is embedded into a statically linked and distributed piece of
software. Moreover, we want to show that in all other cases the LGPL allows 
to distribute packages without granting permission to reverse engineer the
software which uses the LGPL licensed library\footnote{By the way, our analysis
should also provide proof that the LGPL is not something like a 'poisoned'
license containing \enquote{an imprenetrable maze of technology babble} which
\enquote{[\ldots] should not be in a general-purpose software license}
(\cite[cf.][124]{Rosen2005a}). The challenge of the today's descendants is to
understand the former inventors of the GNU licenses and their way to think about
computing - including all the hassle the computing language C might provoke.}.
We hope that our analysis, grounded on the license text itself, will support
companies and people to compliantly use open source software more often and with
less scruples.

NEU FORMULIEREN: ende

Hence, let us prove our position 'bottom up'. Let us firstly show that it is
true for the LGPL-v2 -- by explicating the license text lingually, then
logically, and finally empirically, before we infer the correct understanding.
Then let us show that it is also true for the LGPL-v3. And in the end let us
show that it is true for all other licenses\footnote{analysed by the OSLiC.}.


\section{Reverse Engineering in the LGPL-v2}
% Telekom osCompendium 'for being included' snippet template
%
% (c) Karsten Reincke, Deutsche Telekom AG, Darmstadt 2011
%
% This LaTeX-File is licensed under the Creative Commons Attribution-ShareAlike
% 3.0 Germany License (http://creativecommons.org/licenses/by-sa/3.0/de/): Feel
% free 'to share (to copy, distribute and transmit)' or 'to remix (to adapt)'
% it, if you '... distribute the resulting work under the same or similar
% license to this one' and if you respect how 'you must attribute the work in
% the manner specified by the author ...':
%
% In an internet based reuse please link the reused parts to www.telekom.com and
% mention the original authors and Deutsche Telekom AG in a suitable manner. In
% a paper-like reuse please insert a short hint to www.telekom.com and to the
% original authors and Deutsche Telekom AG into your preface. For normal
% quotations please use the scientific standard to cite.
%
% [ Framework derived from 'mind your Scholar Research Framework' 
%   mycsrf (c) K. Reincke 2012 CC BY 3.0  http://mycsrf.fodina.de/ ]
%


%% use all entries of the bibliography
%\nocite{*}

The LGPL-v2.1 contains one sentence which literally refers to the issues of
\emph{reverse engineering}:

\begin{quote}\noindent\emph{\enquote{[\ldots] you may [\ldots] combine or link a
\enquote{work that uses the Library} with the Library to produce a work
containing portions of the Library, and distribute that work under terms of your
choice, provided that the terms permit modification of the work for the
customer's own use and \emph{reverse engineering} for debugging such
modifications.}\footnote{\cite[cf.][\nopage wp., §6. ]{Lgpl21OsiLicense1999a}.
The first ellipse in this citation -- notated by the string '[\ldots]' -- refers
to the phrase \enquote{As an exception to the Sections above,}, the second to
the phrase \enquote{also}. These words together want to indicate, that the LGPL
offers its §6-way-of-distribution as an exception to the intended default way of
distributing such a Library. So, the nature of the extraordinary way itself is
not affected by this hint. Thus, we feel free to erase this contextual
link.}}\end{quote}

Hereinafter, we will sometimes denote these lines by
the word \emph{LGPL2-RefEng-Sentence}.

\subsection{Linguistical Clarification}
% Telekom osCompendium 'for being included' snippet template
%
% (c) Karsten Reincke, Deutsche Telekom AG, Darmstadt 2011
%
% This LaTeX-File is licensed under the Creative Commons Attribution-ShareAlike
% 3.0 Germany License (http://creativecommons.org/licenses/by-sa/3.0/de/): Feel
% free 'to share (to copy, distribute and transmit)' or 'to remix (to adapt)'
% it, if you '... distribute the resulting work under the same or similar
% license to this one' and if you respect how 'you must attribute the work in
% the manner specified by the author ...':
%
% In an internet based reuse please link the reused parts to www.telekom.com and
% mention the original authors and Deutsche Telekom AG in a suitable manner. In
% a paper-like reuse please insert a short hint to www.telekom.com and to the
% original authors and Deutsche Telekom AG into your preface. For normal
% quotations please use the scientific standard to cite.
%
% [ Framework derived from 'mind your Scholar Research Framework' 
%   mycsrf (c) K. Reincke 2012 CC BY 3.0  http://mycsrf.fodina.de/ ]
%

For fulfilling our rule, to read the text strictly and deduce our
interpretations reasonably, let us firstly only highlight the syntactical
conjunctions for simplifying the understanding:

\begin{quote}\noindent\emph{\enquote{[\ldots] you may [\ldots] combine
\textbf{or} link a \enquote{work that uses the Library} with the Library to
produce a work containing portions of the Library \textbf{and} distribute that
work under terms of your choice, \textbf{provided that} the terms permit
modification of the work for the customer's own use \textbf{and} \emph{reverse
engineering} for debugging such modifications.}\footnote{\cite[cf.][\nopage wp.,
§6, emphasis KR.]{Lgpl21OsiLicense1999a}.}}
\end{quote}

It is evident that the conjunction \emph{'provided that'} is splitting the
sentence into two parts: you are allowed to do something \emph{provided that} a
precondition is fulfilled. Additionally, both parts of the sentence --
the one before the conjunction \emph{'provided that'} and the part after it --
are syntactically condensed embedded phrases which also contain subordinated 
conjunctions and elliptical constructions\footnote{cf.
http://en.wikipedia.org/wiki/Ellipsis\_\%28linguistics\%29, wp.
}. These syntactical interconnections must be disbanded:

Let us firstly \textbf{dissolve the syntactical compression \underline{before}
the conjunction \emph{'provided that'}}: It is established by using the two
other conjunctions \emph{and} and \emph{or} and introduced by the subordinating
phrase \emph{you may [\ldots]}. Unfortunately, from a formal point of view, one
can read the phrase \emph{you may (X or Y and Z)} as two different groupings:
either as \emph{you may ((X or Y) and Z)} or as \emph{you may (X or (Y and Z))}.

But, fortunately, we know from the semantic point of view that speaking about
\emph{\enquote{[\ldots] combining \textbf{or} linking [\ldots something] to
produce a work containing portions of the Library}} denotes two different
methods which both can \emph{join} the components \emph{\enquote{[\ldots] to
produce a work containing portions of the Library}}. So, let us -- only for a
moment\footnote{Later on we will re-insert th original phrase!} -- simply
replace the string \emph{\enquote{combine or link}} by the string
\emph{\enquote{*join}}\footnote{When the LGPL and the GPL were initially
defined, the C programming language was the predominant model of software
development. Knowing this method eases the understanding of these licenses.
Thus, it is not totally wrong to take this token *join also as a curtsey to the
C programming language.}. This reduces the syntactical structure of the sentence
back to the simple phrase \emph{you may (W and Z)} in which \emph{W} stands for
\emph{(X or Y)}.

Now, we can directly state that the phrase \emph{you may (W and Z)} itself is a
condensed version of the explicit phrase \emph{ (you may W) and (you may Z)}.

Finally we have to note, that the phrase before the conjunction \emph{'provided
that'} contains also a linguistic ellipsis\footnote{cf.
http://en.wikipedia.org/wiki/Ellipsis\_\%28linguistics\%29, wp.
}: It says that you may *join the components \enquote{to produce \textbf{a work
containing portions of the Library} \textbf{and} distribute \textbf{that work}
under terms of your choice}. With respect to the English grammar, we may
conclude that the second term \emph{that work} refers back to the previously
introduced specification of \emph{a work containing portions of the Library}: if
a complete phrase has just been introduced explicitly, then the English language
allows to reduce its' next occurence syntactically while its' complete meaning
is retained. Hence, conversely, we are allowed to unfold the reduced form to
restore the complete phrase.

So -- overall -- we may understand the phrase before the conjunction
\emph{'provided that'} as a phrase with the structure \emph{(you may W) and (you
may Z')}:

\begin{quote}\noindent\emph{\textbf{((}you may [\ldots] \emph{*join} a
\enquote{work that uses the Library} with the Library to produce a work
containing portions of the Library\textbf{) and (}you may [\ldots] distribute
that work containing portions of the Library under terms of your
choice\textbf{))}} \textbf{provided that} [\ldots]\end{quote}

Theoretically, a reader could reject our first dissolution of the
LGPL2-RefEng-Sentence. But for reasonably denying our interpretation he has to
deliver other resolutions of the lingustic elliptical subphrases or other
dissolvations of the conjunctions. Fortunately, it seems to be evident that such
attempts must violate the English grammar.

Let us secondly \textbf{dissolve the part \underline{after} the conjunction
\emph{'provided that'}}: With respect to the subordinated conjunction
\emph{'and'}, the subphrase \emph{the terms permit} syntactically refers to
both, the \emph{modification} and the \emph{reverse engineering}: An embedded
conjunction \emph{'and'} allows to use a more stylish grammatical compaction.
So, it should be clear, that saying

\begin{quote}\noindent\textbf{provided that} \emph{the terms permit modification
of the work for the customer's own use \emph{\textbf{and}} reverse engineering
for debugging such modifications}\end{quote}

means

\begin{quote}\noindent\textbf{provided that} \emph{the terms permit
\textbf{(} modification of the work for the customer's own use \emph{\textbf{and}}
reverse engineering for debugging such modifications\textbf{)}}\end{quote}

and is totally equivalent to the sentence 

\begin{quote}\noindent[\ldots] \textbf{provided that} \emph{\textbf{((}the terms
permit modification of the work for the customer's own use\textbf{)}
\emph{\textbf{and}} \textbf{(}the terms permit reverse engineering for debugging
such modifications\textbf{))}}.
\end{quote}

We believe that there is no other possibility to understand this part of the
LGPL2-RefEng-Sentence with respect to the rules of the English language.
Nevertheless, here we reached a next point where our reader may formally
disagree with us. If he really wants to object our dissolution, he must deliver
another valid interpreation of the scope of the conjunction \emph{and} or he
must deliver another resolutions of the linguistic ellipsis. But we reckon, that
one can not reasonably argue for such alternatives.

Finally, there are other deeply embedded ellipses, which need to be resolved
as well:

\begin{enumerate}
  \item  In the part before the splitting conjunction \emph{'provided that'} we
  already had to expand the abridging \emph{'that work'} by its intended
  explicated version \emph{'that work containing portions of the Library'}.  In
  the part after the splitting conjunction the first subphrase also contains the
  term \emph{'the work'}. Formally, this term can either refer to \emph{'the
  work that uses the library'} as one of the components which are joined, or it
  can refer to \emph{'the work containing portions of the Library'} as the
  result of joining the components. We decide to constantly dissolve the
  elliptic abridgement by the phrase \emph{'the work containing portions of the
  Library'}.
  \item The first clause of the part after the splitting conjunction
  \emph{'provided that'} talks about the purpose of \enquote{permitting
  modification of the work} which we just had to unfold to the phrase
  \emph{'permitting modification of the work containing portions of the
  Library'}. The second clause talks about the purpose of \enquote{permitting
  reverse engineering}: it shall support the \enquote{debugging [of] such
  modifications}. The pronoun \emph{'such'} indicates that the word
  \emph{'modifications'} refers back to the just unfolded phrase
  \emph{modification of the work containing portions of the Library}. So, even
  the second sentence has to be expanded to that explicit phrase.
  \item Finally and only for being complete, we also have to unfold the clause
  \enquote{the terms} to the form which is predetermined by the first referred
  instance \enquote{the terms of your choice}
\end{enumerate}

So -- overall -- we are allowed to rewrite the LGPL2-RevEng-Sentence 
in the following form, namely without having changed its
meaning\footnote{Recollect that '*join' still stands for 'combine or link'.}:

\begin{verbatim}
( ( you may 
       *join a work that uses the Library with the Library
        to produce a work containing portions of the Library )
  AND 
  ( you may 
        distribute that work containing portions of the Library
        under terms of your choice 
) )
PROVIDED THAT
( ( the terms of your choice permit 
        modification of the work containing portions of 
        the Library for the customer's own use )
  AND
  ( the terms of your choice permit
        reverse engineering for debugging modifications 
        of the work containing portions of the Library   
) )
\end{verbatim}

At this point we must recommend all our readers to verify that this
'structurally explicated presentation' does exactly mean the same as the
intially quoted LGPL2-RefEng-Sentence. We are now going to discuss some of its'
logical aspects by some formal transformations. For accepting these operations
and linking the results back to the original LGPL2-RefEng-Sentence, it is very
helpful to know that one already has accepted the equivalence of this explicated
form and the more condensed original version. For reviewing the equivalence the
reader could -- for example -- ask himself which of our rewritings are wrong,
why they are wrong and which alternatives can reasonably be offered for solving
the syntactical issues which disposed us to chose our solutions. Again, we
ourselves -- of course -- are profoundly convinced that both versions are
completely equivalent.

%% use all entries of the bibliography
%\nocite{*}


\subsection{Logical Clarification}
% Telekom osCompendium 'for being included' snippet template
%
% (c) Karsten Reincke, Deutsche Telekom AG, Darmstadt 2011
%
% This LaTeX-File is licensed under the Creative Commons Attribution-ShareAlike
% 3.0 Germany License (http://creativecommons.org/licenses/by-sa/3.0/de/): Feel
% free 'to share (to copy, distribute and transmit)' or 'to remix (to adapt)'
% it, if you '... distribute the resulting work under the same or similar
% license to this one' and if you respect how 'you must attribute the work in
% the manner specified by the author ...':
%
% In an internet based reuse please link the reused parts to www.telekom.com and
% mention the original authors and Deutsche Telekom AG in a suitable manner. In
% a paper-like reuse please insert a short hint to www.telekom.com and to the
% original authors and Deutsche Telekom AG into your preface. For normal
% quotations please use the scientific standard to cite.
%
% [ Framework derived from 'mind your Scholar Research Framework' 
%   mycsrf (c) K. Reincke 2012 CC BY 3.0  http://mycsrf.fodina.de/ ]
%

For simplifying our discussion let us now replace the meaningful terminal
phrases of our form by some logical variables:

\begin{description}
  \item[$\Gamma$] :- (you may *join a work that uses the Library with the
  Library to produce a work containing portions of the Library) 
  \item[$\Delta$] :- (you may distribute that work containing portions of the
  Library under terms of your choice)
  \item[$\Phi$] :- (the terms of your choice permit modification of the work 
  containing portions of the Library for the customer's own use)
  \item[$\Sigma$] :- (the terms of your choice permit reverse engineering for
  debugging modifications of the work containing portions of the Library)
  \item[$\Theta$] :- \emph{$\Gamma$ and $\Delta$}
  \item[$\Omega$] :- \emph{$\Phi$ and $\Sigma$}
\end{description}

Based on these definitions, we can syntactically reduce the
LGPL2-RefEng-Sentence to the formula \emph{$(\Gamma$ and $\Delta)$ provided that
$(\Phi$ and $\Sigma)$} or -- even shorter -- to \emph{$(\Theta$ provided that
$\Omega)$}.

Now, we have to clarify the meaning of the conjunction \emph{'provided that'}:

Obviously, \emph{provided that} means something like \emph{under the condition
that}. So, one might try to take this conjunction as another more stylish
version of the common \emph{if(\ldots)then(\ldots)}-formula, sometimes also
identified as a (logical) implication\footnote{Actually the logical implication
and the computational if-then-construct are not equivalent. Fortunately, we
later on can show, that in the context of this discussion the difference can be
ignored.}. Thus, we have to consider the process of sequencing the linguistic
form into a logical formula: if we indeed take the conjunction \emph{provided
that} as another form of the logical implication, it is not evident, which part
of the linguistic sentence must become the premise, and which the conclusion:
Does \emph{($\Theta$ provided that $\Omega$)} mean \emph{(if $\Theta$ then
$\Omega$)} or \emph{(if $\Omega$ then $\Theta$)}?

Apparently, \emph{provided that} wants to establish something like a
precondition. So, one might conclude that \emph{$(\Theta$ provided that
$\Omega)$} means \emph{(if $\Omega$ then $\Theta)$} or -- more logically notated
-- \emph{$((\Phi$ $\wedge$ $\Sigma)$ $\rightarrow$ $(\Gamma$ $\wedge$
$\Delta))$}. If this interpretation is adequate, it must of course fulfill the
intended purpose of the corresponding LGPL-v2-section, which wants to regulate
the distribution of works containing portions of LGPL libraries.

For facilitating the understanding of our argumentation, let us first check
whether this logical interpretation of the linguistic conjunction fits the
purpose of the LGPL -- by unfolding the slightly reduced version \emph{$(\Sigma$
$\rightarrow$ $\Delta)$} back to the corresponding verbal form:

\begin{quote}\noindent\emph{\textbf{if (} [\ldots] the terms permit reverse
engineering for debugging modifications of the work containing portions of the
Library, \textbf{) then (} [\ldots] you may distribute that work containing
portions of the Library under the terms of your choice.\textbf{)}}\end{quote}

Now we can better see the problem: An implication as a whole is false only if
the premise is true and the conclusion is false. In all other cases it is true.
Especially, it is true, if the premise is false: If the premise is false, then
the truth value of the conclusion does not matter in any sense. Thus, if we take
this implication as a rule, which shall determine our behaviour, then this
implication only supports us, if we already have decided to permit reverse
engineering. In this case the rule successfully tells us that we are allowed to
distribute the work containing portions of the Library. But from the converse
decision that we will not permit reverse engineering, follows nothing - because
a false premise does not influence the truth value of the conclusion.
Especially, the rule does not tell us that we may not distribute the work
containing portions of the Library. So -- from the viewpoint of the formal logic
-- this translation of the original conjunction \emph{'provided that'} says,
that if the terms of your own license do not permit reverse engineering for
debugging modifications of the work containing portions of the
Library\footnote{The premise is false.}, then \textbf{you may or may not}
distribute that work containing portions of the Library under the terms of your
choice\footnote{The truth value of the conlusion is undetermined by the rule.}.
Hence, we must state that this interpretation does not fulfill the purpose of
the LGPL-V2: if reverse engineering is not allowed, the distribution of the work
containing portions of the Library is not regulated. We have to conclude, that
this sequencing the LGPL2-RefEng-Sentence as a logical implication is wrong.

But we deduced this consequence from a slighty reduced form of the
LGPL2-RefEng-Sentence. Thus, we still have to ask, whether we have to derive
this conclusion also on the base of the completely unfolded formula
\emph{$((\Phi$ $\wedge$ $\Sigma)$ $\rightarrow$ $(\Gamma$ $\wedge$ $\Delta))$}?
The answer is yes: the premise \emph{$((\Phi$ $\wedge$ $\Sigma)$} contains a
logical conjunction. So the truth value of the whole premise depends on the
truth value of each of its terminal statements, particularly on that of the
statement $\Sigma$: If we decide not to permit reverse engineering, then the
premise as whole is false, regardless we forbid or allow modifications.
Consequently, the premise does not influence the truth value of the conclusion.
So, there is no way, to conclude that we have to allow or that we do not have to
allow reverse engineering. Hence we can transfer our result, deduced for the
slightly reduced formula to the unfolded complete formula: assuming that
\emph{$(\Theta$ provided that $\Omega)$} means \emph{(if $\Omega$ then
$\Theta)$} is wrong.

So, let us test the other combination. Let us ask, whether \emph{($\Theta$
provided that $\Omega$)} means \emph{(if $\Theta$ then $\Omega$)} or -- more
logically notated -- \emph{$((\Gamma$ $\wedge$ $\Delta)$ $\rightarrow$ $(\Phi$
$\wedge$ $\Sigma))$}. If we again for a moment focus on the reduced version
\emph{$(\Delta$ $\rightarrow$ $\Sigma)$} and dissolve our replacements, then we
get back the rule:

\begin{quote}\noindent\emph{\textbf{if (} [\ldots] you may distribute that work
containing portions of the Library under the terms of your choice, \textbf{)
then (} [\ldots] the terms permit reverse engineering for debugging
modifications of the work containing portions of the Library.
\textbf{)}}\end{quote}

Now we can see, that this version perfectly regulates the distribution of works
containing portions of LGPL libraries: If we are allowed to do so or -- in other
words: if we are compliantly distributing works containing portions of LGPL
libraries\footnote{The premise is true.}, then we have to permit reverse
engineering\footnote{The conclusion must be true, too!}. This follows from
applying \emph{Modus Ponens} to the implication\footnote{A true premise evokes a
true conclusion based on the given truth of the implication / rule itself.}. And
if we do not permit reverse engineering\footnote{The conclusion is false.}, then
we are not allowed to distribute works containing portions of LGPL
libraries\footnote{The premise must be false, too!}. This follows from applying
\emph{Modus Tollens} to the implication\footnote{A false conclusion evokes a
false premise based on the given truth of the implication / rule itself.}

But -- again -- we have to consider that we have deduced this consequence from a
slighty reduced version of our LGPL2-RefEng-Sentence. Thus, we still have to
show that our result also holds for the completely unfolded formula
\emph{$((\Gamma$ $\wedge$ $\Delta)$ $\rightarrow$ $(\Phi$ $\wedge$ $\Sigma))$}:
If we want to distribute works containing portions of the Library which have
been produced by joining the Library and the work using the
Library\footnote{Premise is true.}, then our terms must permit the modification
\emph{and} reverse engineering of the distributed product\footnote{Conclusion
must become true by Modus Ponens.}. And if we do not allow its modification
\emph{or} reverse engineering\footnote{Conclusion is false.}, then we do not
compliantly distribute works containing portions of the Library which have been
produced by joining the Library and the work using the Library\footnote{Premise
must become false by Modus Tollens.} Thus, we may generally state, that the
logical explication \emph{$((\Gamma$ $\wedge$ $\Delta)$ $\rightarrow$ $(\Phi$
$\wedge$ $\Sigma))$} perfectly regulates the distribution of works containing
portions of LGPL libraries.

Based on this clarification, we can reasonably replace the more stylish
conjunction \emph{'provided that'} by its more known equivalent
\emph{'implication'}\footnote{Here we can also see, that the difference between
the if-then-command as part of a procedural computer language and the logical
implication does not influence our results: In the context of a procedural
if-then-command the truth of the premise triggers the execution of the
conclusion. In our discussion, this aspect is totally covered by the Modus
Ponens derivation of the logical interpretation. And the Modus Tollens
derivation of the logical interpretation on the other side does not play any
role in a procedural if-then-command. So, it was the right decision to
understand the LGPL2-RefEng-Sentence logically and not as procedual command.},
which we indicate by the commonly used character for a logical implication, the
sign '\emph{$\rightarrow$}':

\begin{description}
  \item[\#]  $\Theta$ provided that $\Omega$
  \item[$\equiv$] $\Theta$ $\rightarrow$ $\Omega$
  \item[$\equiv$] ($\Phi$ $\wedge$ $\Sigma$) $\rightarrow$ ($\Gamma$ $\wedge$
  $\Delta$)
  \item[$\equiv$]
\begin{alltt}   
  ( ( [\(\Phi\)] you may 
       *join a work that uses the Library with the Library
       to produce a work containing portions of the Library )
  \(\wedge\)
  ( [\(\Sigma\)] you may 
        distribute that work containing portions of the 
        Library under terms of your choice 
) )
\(\rightarrow\)
( ( [\(\Gamma\)] the terms of your choice permit 
        modification of the work containing portions of 
        the Library for the customer's own use )
  \(\wedge\)
  ( [\(\Delta\)] the terms of your choice permit
        reverse engineering for debugging modifications 
        of the work containing portions of the Library   
) )
\end{alltt}
\end{description}

%% use all entries of the bibliography
%\nocite{*}


\subsection{Empirical Clarification}
% Telekom osCompendium 'for being included' snippet template
%
% (c) Karsten Reincke, Deutsche Telekom AG, Darmstadt 2011
%
% This LaTeX-File is licensed under the Creative Commons Attribution-ShareAlike
% 3.0 Germany License (http://creativecommons.org/licenses/by-sa/3.0/de/): Feel
% free 'to share (to copy, distribute and transmit)' or 'to remix (to adapt)'
% it, if you '... distribute the resulting work under the same or similar
% license to this one' and if you respect how 'you must attribute the work in
% the manner specified by the author ...':
%
% In an internet based reuse please link the reused parts to www.telekom.com and
% mention the original authors and Deutsche Telekom AG in a suitable manner. In
% a paper-like reuse please insert a short hint to www.telekom.com and to the
% original authors and Deutsche Telekom AG into your preface. For normal
% quotations please use the scientific standard to cite.
%
% [ Framework derived from 'mind your Scholar Research Framework' 
%   mycsrf (c) K. Reincke 2012 CC BY 3.0  http://mycsrf.fodina.de/ ]
%

We can now simplify this formula once more by considering some empirical facts
and explicating some underlying understandings:

The first sentence $\Phi$ explains that the \emph{work that uses the Library}
and the used \emph{Library} itself together are joined and thereby transformed
into a \emph{work containing portions of the Library}. So, formally, one might
ask, whether this newly generated \emph{work containing portions of the Library}
also still \emph{uses the Library}?

Unfortunately, it is empirically possible, that such a process for combining the
two components could (a) copy all original portions of the library into a
something like a 'dead end section' of the program where they are never excuted,
and could (b) replace all original portions of the library by functionally
equivalent portions of any other library. Thus, the resulting \emph{work
containing portions of the Library} would indeed still contain portions of the
Library, although it would not use it any longer. And because of this
possibility, we are not allowed to say, that every work containing portions of a
library also uses the library\footnote{\ldots even if we think that this is a
really silly way to organize the joining process!}.

But, fortunately, the normal computational process of \emph{combining and
linking a work that uses the Library with the Library to produce a work
containing portions of the Library} inherently preserves the utilization of the
joined library: It is the general purpose of a software library to offer
functions and/or data (structures) for really being used by applications. And
vice versa, software developers refer to a specific library because they prefer
its service: They use readily prepared libraries (or classes or anything else)
because they want to simplify their own work while they conserve the quality
level of their work. Thus, they chose a library based on the assertion, that the
standard compiling and linking process guarantees, that indeed the chosen
library is used (and not secretly substituted by a mysterious 'equivalent').
With respect to this praxis of programming we are allowed to say that a
\emph{work containing portions of the Library} which has been \textbf{built by
the normal development processes} of combining, compiling, and linking source
and object files, indeed also uses the intended library.

Now, we are able to consider an empirical correlation between the first sentence
$\Phi$ and the second sentence $\Sigma$:

It seems to be evident, that we must already have done $\Phi$, in other words:
that we must already have \emph{*joined -- respectively: combined or linked -- a
work that uses the Library with the Library to produce a work containing
portions of the Library}, if we are going to compliantly \emph{distribute that
work containing portions of the Library under terms of your choice}. Or briefly
spoken: It seems to be conclusive that $\Sigma$ \emph{\textbf{empirically}
implies} $\Phi$\footnote{but not vice versa.}.

But is this conclusion correct? Let us check this statement by assuming the
opposite: If the contrary was true, there had to exist a \emph{work containing
portions of the Library} which had been gained without having linked or combined
the work and the Library in any sense. But from the inference above we already
know that \emph{works containing portions of the Library}, which have been
produced by the standard computational processes of \emph{combining and linking
a work that uses the Library with the Library}, indeed also \emph{use the
Library}. Thus, it would be self-contradictory to talk about a \emph{work
containing portions of the Library}, which was produced by the standard
combining and linking processes, and similarily to state, that exactly this work
is not combined with the library in any sense. And from a proof by contradiction
we may infer the truth of the logical opposite:

So, with respect to the meaning of \emph{being standardly combined or linked
with}, we may now say, that
\begin{itemize}
  \item it is necessarily true that a computional work, which is standardly
  produced on the base of \emph{a work that uses the Library} and \emph{the
  Library} and which therefore literally contains more or less
  \emph{portions of a library}, indeed uses the \emph{the Library} and \emph{is}
  therefore \emph{combined with the library}.
  \item  $\Sigma$\footnote{distributing \emph{a work that uses the Library and
  contains portions of a library}} empirically implies $\Phi$\footnote{A work
  that uses the Library has been *joined with the Library to produce a work
  containing portions of the Library} (in the standardized world of software
  development), because $\Phi$ must ever have been executed when $\Sigma$ is
  going to be realized.
\end{itemize}

Thus, we can now reduce the LGPL2-RefEng-Sentence to its' real core, the
LGPL2-RefEng-Rule:

\label{RevEngEssentialLgplSection6Meaning}
\begin{quote}
\begin{alltt}   
(   [\(\Sigma\)] you may
        distribute (a) work containing portions of the 
        Library\(\footnote{which previously has been prepared for being distributed by standardly combining and
linking the work that uses the Library with the Library in a way that this prepared work indeed
also uses the Library}\) under terms of your choice )   
\(\rightarrow\)
( ( [\(\Gamma\)] the terms of your choice permit 
        modification of the work containing portions of 
        the Library for the customer's own use )
  \(\wedge\)
  ( [\(\Delta\)] the terms of your choice permit
        reverse engineering for debugging modifications 
        of the work containing portions of the Library   
) )
\end{alltt}
\end{quote}

This is indeed the essence of the LGPL2-RefEng-Sentence. It logically explains
us that we have to \emph{allow reverse engineering} and modification of a
\emph{work containing portions of the Library} if we distribute it (Modus
Ponens) and that we are \emph{not allowed to distribute a work containing
portions of the Library}, if we do \emph{not allow} its modification or
\emph{reverse engineering} (Modus Tollens).

Thus, for applying this rule correctly, we now only must know whether a work
indeed contains portions of the Library or not.

%% use all entries of the bibliography
%\nocite{*}


\subsection{Final Conclusion}
% Telekom osCompendium 'for being included' snippet template
%
% (c) Karsten Reincke, Deutsche Telekom AG, Darmstadt 2011
%
% This LaTeX-File is licensed under the Creative Commons Attribution-ShareAlike
% 3.0 Germany License (http://creativecommons.org/licenses/by-sa/3.0/de/): Feel
% free 'to share (to copy, distribute and transmit)' or 'to remix (to adapt)'
% it, if you '... distribute the resulting work under the same or similar
% license to this one' and if you respect how 'you must attribute the work in
% the manner specified by the author ...':
%
% In an internet based reuse please link the reused parts to www.telekom.com and
% mention the original authors and Deutsche Telekom AG in a suitable manner. In
% a paper-like reuse please insert a short hint to www.telekom.com and to the
% original authors and Deutsche Telekom AG into your preface. For normal
% quotations please use the scientific standard to cite.
%
% [ Framework derived from 'mind your Scholar Research Framework' 
%   mycsrf (c) K. Reincke 2012 CC BY 3.0  http://mycsrf.fodina.de/ ]
%

Unfortunately, there are more than one software developing scenarios, which must
be considered for answering this question in detail. We see three general types
of developing computer software:

\begin{enumerate}
  \item You can produce software by using script languages. Source files which
  contain script language commands are distributed and executed by an
  interpreter without priorly being transformed into another 'more' machine
  specific language.
  \item You can develop software by using languages which are designed for being
  compiled into a machine independent bytecode. Later on, this independent
  bytecode is executed by a machine specific virtual machine.
  \item You can write traditional software files. Sometimes, these files are
  remastered by a preprocessor before the real process starts. The traditional
  sources themselves or the output files of the preprocessor are then compiled
  and linked as machine specific binary file(s).
\end{enumerate}
  
You may take 'php' is an example for the first environment, 'Java' an example
for the second, and 'C/C++' an example for the third.

Fortunately, the nature of these environments simplifies the answer to the
question under which conditions the work using the Library contains portions of
the Library:

\paragraph{Distributing works with manually copied portions of the Library
evokes the copyleft effect:}
\label{RevEngCopyCodeManually}
Manually copying code from the sources of the Library into the overarching
work that uses the Library, is not the standard way of combining both
components, neither in the world of script programming, nor in the world of
bytecode programming, nor in the world of programming machine specific code:

Normally, the work which uses the Library is joined to the intended Library by
an include statement, an input statement, an import statement, a package
statement, or anything else. These *join-statements are inserted into the code
of the work. They denote the file(s) which deliver(s) the used functions,
methods, classes, or data. It is an integrated feature of the normal development
tools that inserting such *join-statements does not directly augment the work
using the library by some code of the Library: The development processes are
designed to offer an automatic augmention as part of the standard compilation
which is started after the actual development loop has been terminated.

Nevertheless, developers can circumvent these standard methods for using a
Library. Technically, they can directly copy code from the Library into their
own work. Consequently, these manually copied extensions of the code will be
compiled and/or executed together with the 'own' code of the work. Thus, it is
clear, that in this case the work that uses the Library, already contains
portions of the Library, paticularly before the normal *join-processes of the
environment are executed.

Hence, if you are going to distribute works that contain literal copies of the
Library source code, then you have to allow reverse engineering, even if they
have already been compiled (but still not linked) on the base of such augmented
files\footnote{This directly follows from the LGPL2-RefEng-Rule by Modus Ponens.
But nevertheless, one might reply here, that even the result of manually copying
code from the Library to the work using the Library is covered by the limits of
tolerance, introduced by the LGPL-v2-§5. Formally, this argument seems to be
appropriate. And indeed, also we have to consider these limits of
tolerance later on. But in the context of copying code from the Library into the
work manually, a closer look reduces its impact. You have to discriminate three
cases:
\begin{enumerate}
  \item Developers can  \textbf{manually copy / transfer some or at most all
  elements of the Library header files into the code of the work} which the
  preporcessor itself would copy / transfer into that code automatically. But
  developers will not do that. Some simple include commands would cause the same
  effect. And Developers want to save recourses, especially their own working
  time. So, why should they manually do, what they can delegate to the standard
  development process. Thus, it is reasonable to assume, that developers, who
  nevertheless copy portions from Library into their work, do not want to repeat
  the service of the preprocessor manually, but to transfer more than only these
  elements. Hence, it is reasonable to assume, that their work is covered by the
  LGPL2-RefEng-Rule.
  \item Developers can \textbf{manually copy / transfer more than only the
  elements of the Library header files from the Library sources into the code of
  the work using the Library and they can nevertheless let the work being linked
  to the Library}. But again developers will not do that, because -- again --
  some simple include and linking commands would cause the same effect. So it is
  reasonable to start from the premise, that copying developers in fact do more
  than this. Thus, it is reasonable to assume, that their work is covered by the
  LGPL2-RefEng-Rule.
  \item Developers can \textbf{manually copy / transfer more than only the
  elements of the Library header files from the Library sources into the code of
  the work using the Library without linking it to the Library}. This is a
  reasonable step of work, because it spares the developers to link their work
  to the Library. But -- by definition -- such an augmented work contains more elements
  of the Library than LGPL-v2-§5 tolerates. Thus it is -- again -- reasonable to
  assume, that such a work is covered by the LGPL2-RefEng-Rule.
\end{enumerate}
Hence -- overall and from a practical point of view -- we can indeed say, that
manually copying code from the Library into the work using the Library
requires to allow reverse engineering.}.

But, if we manually copy code from the Library to our work using the Library, we
also have to consider, that the LGPL-v2 directly regulates this kind of using
the Library: It says, that \enquote{you may modify your copy or copies of the
Library or any portion of it [\ldots] provided that you also [\ldots] cause the
whole of the work to be licensed [\ldots] under the terms of this
License}\footcite[cf.][\nopage wp., §2, escpcially §2c]{Lgpl21OsiLicense1999a}.
Thus, there are strong arguments for the proposition, that the LGPL causes the
copyleft effect in case of literally copying code from the Library into the work
using the Library: The code of the work using the library has to be made
accessible, as well.

So, overall, we might say, that 'manually' copying code from an LGPL-v2 Library
into a work using that Library as a bypass of the standard software combining
processes and distributing the result indeed requires to additionally permit its
reverse engineering -- even if this permission is probably not very important
for the recipient, because he probably must have a direct access to the code.

\paragraph{Distributing scripts does not need reverse engineering:}
\label{RevEngDistributeScripts}
Computer programs written in a script language are distributed as they have been
developed. They are not transformed into another kind of code\footnote{Java
script is often offered as compressed code. Roughly spoken, this means that at
first all white space signs have been replaced by blanks and then all rows of
blanks have been reduced to at most one single blank. So, even then, the code
itself is directly readable and comprehensible -- even if only for very
sophisticated experts.}. The interpreter takes the script file as it is and
directly executes it. Thus, there is no special technique of reverse engineering
for understanding these kind of software: you can directly read it if you know
the script language.

So, again, we might conclude, that a script using a script Library perhaps
requires to permit its reverse engineering -- but probably this permission is
not very important for the recipient, because he can directly read the code.

\paragraph{Distributing statically combined bytecode requires the
permission of reverse engineering:}
\label{RevEngDistributeStaticallyCombinedByteCode}
In Java -- the prototype for languages which are compiled to machine independent
portable bytecode -- each class is compiled as a seperate class file. These
class files have to be stored somewhere in the classpath. A side from that,
classes can also be collected and distributed in form of packages which then can
be used like 'traditional' Libraries. These packages must also be stored
somewhere in the classpath. A single class is made known to the work, that want
to use it, by an import statement which contains the class name; a whole Java
library is made usable by integrating a package statement into the code.

The code which follows such import- or package statements, can then use the
definitions offered by the classes. It denotes the elements of the classes by
the (qualified) names of its public or protected member variables or methods.
Thus, -- from a strict viewpoint -- the code of such a Java work using a Library
indeed contains portions of that library, even if these portions are only
identifying names or data structures containing identifying names. The Java
compilation process which generates the bytecode, preserves these denoting
names. It does not replace the referring names by the referred code of the
methods and so on. Only just at the end, when the java virtual machine itself
tries to execute the work using the Library, it collects all necessary commands
of all 'joined' classes.

So, one might tend to argue that answering the question, whether a distributed
java bytecode already contains portions of the used Library, depends on the
interpretation, whether a denoting identifier of a Library indeed is a portion
of the Library. We will discuss this case together with the corresponding
C/C++-Case. 

But there is another Java specific aspect, which has to be considered as well.
As already mentioned, in Java you can also join your work containing the
denoting indentifiers and the denoted Library by building a new package, which
then contains both, the work using the Library and the used Library. Hence, one
can say, that this package is quasi statically linked: if you distribute such an
integrated package, then you are distributing both components together. Thus, if
you distribute a complete package, in other words: a quasi statically linked
work containing the work using the Library and (all portions of) the Library,
then you have to permit reverse engineering\footnote{This directly follows from
the LGPL2-RefEng-Rule by Modus Ponens}.

So, preliminarily we conclude, that with respect to Java programming you (a)
have to permit reverse engineering, if you distribute your work using the
Library and the Library itself as a (statically linked) integrated
package\footnote{This follows from the LGPL2-RevEng-Rule by Modus Ponens.} and
that (b) in all other cases your obligation to permit reverse engineering
depends on the interpretation whether the identifiers declared by a Library are
indeed portions of the Library.

Fortunately, we can reasonably decide the issue of case (b) soon.

\paragraph{Distributing statically combined binaries require the
permission of reverse engineering:}
\label{RevEngDistributeStaticallyLinkedBinaries}
Similar to Java, in C/C++ -- the prototype of those languages, which are
compiled as machine specific code -- a C/C++ Library is also explicitly made
known to the work that want to use it, namely by some include statements. These
include statements denote the header files offered by and distributed with the
Library. They contain the declarations of those elements which the Library wants
to publish. Or briefly worded: the Library contains the definitions in form of
code, the header files the corresponding declarations.

The C/C++ code following such include statements can refer to the definitions
offered by the Library by using the declarations anounced by the header files.
So, again, -- from a strict viewpoint -- the code of such a C/C++ work using the
Library indeed contains portions of the library, even if these portions are only
identifying names or data structures published by the header files.

Beyond that conceptual relation, the C/C++ development process finally compiles
the work using the library as an object file containing machine specific code.
Just as the Java compilation, also this process does not replace the referring
names by the referred code of the Library; it still preserves the denoting
names. The resulting file, which has been compiled into machine specific code,
but still contains the denoting identifiers, is also known as 'object code
file'.

The C/C++ compilation process is (mostly) managed by a make file, which is
executed by the make command\footnote{Sometimes there additionally exist a
complete meta environment which generates such make files. The GNU build system
for example offers a complex set of configure scripts and make file templates
(cf. http://en.wikipedia.org/wiki/GNU\_build\_system, wp.).}. This development
tool calls the compiler for each source file, makes known the directories which
contain the compiled target object files, and finally calls the linker.
The linker recursively scans the compiled object files and replaces each
embedded identifier by a truely executable jump command into that set of Library
commands which are denoted by the identifier and which shall be executed as part
of the work using the Library. So, only just at the end, the linker collects all
necessary commands of all 'joined' object files and Libraries and produces the
really executable work.

But -- notwithstanding the above -- the linker can either be called as
integrated step of the developement process itself. Or the linker can
seperatedly be called, especially on another machine: In the first case, the
development process generates a \emph{statically linked executable} which
already contains all necessary portions of all used Libraries. In the second
case, the development process generates a \emph{dynamically linkable program} by
collecting the (set of) still unlinked object code file(s) as a distributable
package. Thus, if you distribute a statically linked executable, it definitely
contains 'portions' of the library; if you distribute a dynamically linkable
program you have to decide whether the embedded identifying names of a Library
have indeed to be interpreted as portions of the Library.

Unfortunately, we still have to consider a little complication, based on the
nature of the a C/C++ development process: In contrary to the Java development
environment, a C/C++ development process inherently uses a preprocessor engine.
This engine takes the header files delivered by the Library, verifies the
syntactically correct use of the Library and can indeed replace some tokens of
the work using the Library by commands and/or lines from the Library. This
technique is known as \emph{inline functions} or \emph{macros}. They have been
invented for those cases where expanding the stack of commands during the
compilation by a real function call is more expensive than writing the embedded
commands of the function more than one time into the whole code. Hence, in the
C/C++ development process the compiled object files can indeed contain more than
only the referring names which denote portions of the Library: beside the
denoting identifiers, they can also already contain real, functionally relevant
portions of the Library.
 
Thus, -- again and similar to Java compilation -- we may conclude, that with
respect to C/C++ programming you (a) have to permit reverse engineering, if you
distribute your work using the Library together with the Library as a statically
linked program\footnote{This follows from the LGPL2-RevEng-Rule by Modus
Ponens.} and that (b) in all other cases your obligation to permit reverse
engineering depends on the interpretation whether the used identifiers or
dissolved inline functions and macros, which have been declared by the Library
and which therefore have automatically and standard conformably been embedded
into an object file, are indeed portions of the Library.

Obviously, it is time to answer this crucial question:

\paragraph{Distributing dynamically combinable bytecode and linkable object code
does not require the permission of reverse engineering:} 
\label{RevEngDistributeDynamicallyLinkedCode}
Of course, there is only one instance, that can answer the question, whether
indentifiers and dissolved inline-functions or macros, which are -- according to
the development standard -- embedded into a work using the Library, indeed are
portions of the Library. This instance is the LGPL-v2 itself. And -- fortunately
-- this license supports us in a very clear way to answer this question, even if
not by its §6 which deals with the reverse engineering, but by its §5:

The LGPL simply specifies that \enquote{linking a \enquote{work that uses the
Library} with the Library creates an executable that is a derivative of the
Library (because it contains portions of the Library) [\ldots]} and that
\enquote{the executable is therefore covered by this
License}\footcite[cf.][\nopage wp. §5]{Lgpl21OsiLicense1999a}. Additionally, it
talks about compiled, but still unlinked \enquote{object files}, which therefore
are not executables . Such an unexecutable \enquote{object file} -- for example
that of the \enquote{work using the Library} --, which \enquote{[\ldots] uses
only numerical parameters, data structure layouts and accessors, and small
macros and small inline functions (ten lines or less in length)} shall
practically not be covered by the license of the Library, because
\enquote{[\ldots] the use of the object file is unrestricted regardless of
whether it is legally a derivative work}\footcite[cf.][\nopage wp.
§5]{Lgpl21OsiLicense1999a} - as long as it does not exceed the given limits.

Obviously, the answer of the LGPL to our question is this: (a) yes, such
object files containing names and snippets offered by the used Library, could
contain portions of the Library. But it is not necessary to clarify the details,
because (b) -- up to a specific limit of sizes -- these kind of 'little'
portions being embedded into the object file by the standard compilation
processes do not evoke any requirements: they especially do not evoke the
obligation to allow reverse engineering. In other words: These little portions
of a Library which are embedded by the standard development process and which do
not contain more than the specified size of code may be regarded as another type
of portions compared to the normal, real portions which indeed evoke the
obligation to allow reverse engineering. From the viewpoint of the LGPL, they
are \emph{pseudo portions} of the Library, because they do not restrict the
containg object file in any respect.

So, from the LGPL-RevEng-Rule we can now indirectly conclude, that distributing
dynamically linkable or combinable bytecode or object code files which contain
\enquote{only numerical parameters, data structure layouts and accessors, and
small macros and small inline functions (ten lines or less in length)} being
delivered by a Library does not require to allow reverse
engineering\footnote{From the decision not to allow reverse engineering follows
by Modus Tollens applied to the LGPL2-RevEng-Rule, that the distribution of the
work using the Library must not contain real portions of the Library. From
LGPL-v2-§5 and the limit of the standard proccesses follows that here the work
using the Library does not contain normal, real portions. So, we know, that this
case is not covered by the LGPL2-RevEng-Rule and thus we are allowed, to
distribute a work using the Library without allowing its reverse engineering.}.

Unfortunately, there might be a practical objection which seems to disturb our
simple result: For applying this rule correctly, we apparently have to assure
that a compiled work that uses the Library but is still not *joined to it,
indeed has only been expanded by \enquote{small macros or small inline functions
(ten lines or less in length)}. Thus, seemingly, we have to study all header
files of all used Libraries in detail, if we want to compliantly distribute a
work using a Library without permitting reverse engineering. This could be a lot
of work -- up to a bulk which practically can not be managed.

Fortunately, there is a simple solution for this challenge, a rule of thumb,
based on the principle \enquote{trust the upstream}\footnote{On the ELLW 2013,
we were told about this principle for the first time. We do not know, whether
Armijn Hemel invented it. But we can respectfully affirm that he has
persuasively explained the spirit and purpose of the principle \enquote{trust
the upstream}.}:

The Library developers of course publish the header files or the public members
and functions of the classes in exactly that form they want these elements to be
used. And they want their Library to be used as an LGPL library, otherwise they
would have chosen another License. So, they wish that improvements of the
Libraries shall be made accessible as well, but that the works using the Library
shall not necessarily be published in form of source code\footnote{The meaning
of the weak copyleft.}. Thus, as long as we use a Library exactly in that form,
the original authors have published, as long as we load down the Library from
the official repository, and as long as we do not modify the intended interfaces
defined and published by the original header and class files, we may justifiably
assume that we are using the Libraries just as their copyright owners want them
to be used. And thus, -- in other words: as long as we trust the upstream -- we
might assume that the header and class files of our Libraries fit the
restrictions of the LGPL-v2.

\paragraph{LGPL-v2 compliance with or without permitting reverse engineering:} 
\label{RevEngLgpl2ComplianceByRenverseEngine}

Now, we have reached our target. Our last clarification can directly be applied
to the both open cases: to the case of distributing Java bytecode as well, as to
the case of distribution C/C++ object code. We now know, that the LGPL-v2
wishes, that not all portions of a Library covered by a work using the Library,
trigger the permission of reverse engineering. And we now know that the limits
-- given by the LGPL-v2-§5 -- up to which such pseudo portions indeed do not
trigger the obligation to permit reverse engineering, are respected, if we use
\emph{'upstream approved'} C/C++ and Java libraries in standard development
environments. Thus, we indeed finally may conclude, that the
LGPL-RevEng-Sentence

\begin{quote}\noindent\emph{\enquote{[\ldots] you may [\ldots] combine
\textbf{or} link a \enquote{work that uses the Library} with the Library to
produce a work containing portions of the Library \textbf{and} distribute that
work under terms of your choice, \textbf{provided that} the terms permit
modification of the work for the customer's own use \textbf{and} \emph{reverse
engineering} for debugging such modifications.}\footcite[cf.][\nopage wp., §6, 
emphasis KR.]{Lgpl21OsiLicense1999a}}
\end{quote}

means 'nothing else' than

\begin{itemize}
  \item \emph{With respect to a LGPL-v2 licensed Library, you are not required
  to allow reverse engineering, if you [A] develop your work using the Library,
  on the base of a standard version of the Library containing the interfaces as
  the original developers have designed it, if you [B] compile your work using
  this Library, as a discret (set of) dynamically linkable or combinable
  file(s), if you [C] use only the standard compilation methods which preserve
  the upstream approved interfaces\footnote{and which therefore do not to exceed
  the LGPL-v2 limits!}, and if you [D] distribute the produced unlinked object
  code or bytecode files before they are linked as an executable.}
  \item \emph{In all other cases of distributing a work using such a Library,
  you are required to allow reverse engineering of the work using this Library
  -- especially, \ldots}
  \begin{itemize}
    \item \emph{if you distribute the work using the Library and the Library
    together as a statically linked program or as an integrated package
    containing both parts, the work using the library and the Library
    itself\footnote{This holds also if you distribute a script language based
    program or package, notwithstanding the fact, that one does not need the
    permission of reverse engineering to understand script language based
    applications.}.}
    \item \emph{if you distribute a work containing manually copied portions of
    the Library.}
  \end{itemize}
\end{itemize}

%% use all entries of the bibliography
%\nocite{*}


\subsection{Final Securing}
% Telekom osCompendium 'for being included' snippet template
%
% (c) Karsten Reincke, Deutsche Telekom AG, Darmstadt 2011
%
% This LaTeX-File is licensed under the Creative Commons Attribution-ShareAlike
% 3.0 Germany License (http://creativecommons.org/licenses/by-sa/3.0/de/): Feel
% free 'to share (to copy, distribute and transmit)' or 'to remix (to adapt)'
% it, if you '... distribute the resulting work under the same or similar
% license to this one' and if you respect how 'you must attribute the work in
% the manner specified by the author ...':
%
% In an internet based reuse please link the reused parts to www.telekom.com and
% mention the original authors and Deutsche Telekom AG in a suitable manner. In
% a paper-like reuse please insert a short hint to www.telekom.com and to the
% original authors and Deutsche Telekom AG into your preface. For normal
% quotations please use the scientific standard to cite.
%
% [ Framework derived from 'mind your Scholar Research Framework' 
%   mycsrf (c) K. Reincke 2012 CC BY 3.0  http://mycsrf.fodina.de/ ]
%

So far, we have done a lot of work: At first, we unfolded and dissolved some
stylisch condensed formulations of the original LGPL2-RevEng-Sentence by their
linguistically explicit version. At second, we explicated the logical structure
of the sentence. At third, we empirically carved out the real meaning of the
sentence. And finally we mapped the triggering part of that rule to some
verifiable facts. Indeed, a lot of work for understanding one sentence
correctly\footnote{Here, some readers might ask why the original authors have
encapsulated their clear ideas in such a sophisticate sentence. Here are two
answers: First, this question is practically irrelevant: The authors of the
LGPL-v2 did, what they have done. And many developers have already licensed
their works under the terms of the LGPL-v2. Thus, we simply have to live with
the results -- just until the last software being published under the terms of
the LGPL-v2 is relicensed by a better version. Probably this won't happen during
our life time. Secondly, we appreciate the foresight of the LGPL-v2 authors.
They wrote a license which have successfully worked for more than twenty years.
They chosed a formulation which had also to cover 'uninvented' techniques. So,
it is not so surprizing, that we -- today -- have to do a lot of work to
understand all details the original authors want to be understood.}. So, it is a
good securing to verify that the derived result fits the spirit and the goals of
the LGPL-v2 perfectly:

The LGPL-v2 clearly describes its goals. It wants to enable the community to let
an LGPL Library \enquote{[\ldots] become a de-facto standard}. And the LGPL
knows, that \enquote{to achieve this [goal], non-free programs must be allowed
to use the library}, because the \enquote{[\ldots] permission to use a particular
library in non-free programs enables a greater number of people to use a large
body of free software}. But the LGPL also asserts in this context, that
\enquote{although the Lesser General Public License is Less protective of the
users' freedom, it does ensure that the user of a program that is linked with
the Library has the freedom and the wherewithal to run that program using a
modified version of the Library}\footcite[cf.][\nopage wp., preamble, emphasis
KR.]{Lgpl21OsiLicense1999a}.

So -- as a last check of our derivation -- we can analyze, whether our derived
result violates this goal. If it does, then we probably made a tremendous fault;
if not, then we are allowed to trust in the consistence our analysis:

If you receive a work using the Library in form of a discret (set of)
dynamically linkable or combinable file(s) and if -- hence -- your provider
assumed that the delivered files are linked on your target machine which --
therefore -- has also to deliver the necessary dynamically linkable Libraries,
than you have the freedom systematically to replace the dynamically linked
Libraries by updated versions\footnote{In GNU/Linux -- for example -- you must
(only) copy the dynamically linkable new version of the Library into the
lib/-directory and replace the existing link by a version pointing to the newer
version. Sometimes you should additionally verify the ld.so.conf files and call
ldconfig tool.}. And as long as the newer versions of the Libraries preserve the
defined and declared interfaces, you can do that successfully.That's, what the
LGPL-v2 want to ensure.

In all other cases, you must have the permission of reverse engineering or you
have a direct access to the source code. So, you can use the corresponding tools
and techniques to replace the embedded version of the Library by a newer
version; especially if you have received a statically linked package. Hence,
also the second part of our interpretation respects the spirit of the LGPL-v2.

So, finally we can say, everything is fine: The LGPL2-RevEng-Rule -- together
with the meaning of being a portion of a Library -- does not only verifiably
exeplicate the meaning of the LGPL2-RevEng-Sentence, but also fits the spirit
and the purpose of the LGPL-v2 as it has been announced by its preamble.


%% use all entries of the bibliography
%\nocite{*}


\section{Reverse Engineering in the LGPL-v3}
% Telekom osCompendium 'for being included' snippet template
%
% (c) Karsten Reincke, Deutsche Telekom AG, Darmstadt 2011
%
% This LaTeX-File is licensed under the Creative Commons Attribution-ShareAlike
% 3.0 Germany License (http://creativecommons.org/licenses/by-sa/3.0/de/): Feel
% free 'to share (to copy, distribute and transmit)' or 'to remix (to adapt)'
% it, if you '... distribute the resulting work under the same or similar
% license to this one' and if you respect how 'you must attribute the work in
% the manner specified by the author ...':
%
% In an internet based reuse please link the reused parts to www.telekom.com and
% mention the original authors and Deutsche Telekom AG in a suitable manner. In
% a paper-like reuse please insert a short hint to www.telekom.com and to the
% original authors and Deutsche Telekom AG into your preface. For normal
% quotations please use the scientific standard to cite.
%
% [ Framework derived from 'mind your Scholar Research Framework' 
%   mycsrf (c) K. Reincke 2012 CC BY 3.0  http://mycsrf.fodina.de/ ]
%

Based on our experiences how to successfully carve out the meaning of a natural
sentence, we can shorten the way to understand the one LGPL3-RevEng-Sentence
referring to \emph{reverse engineering}:

\begin{quote}\emph{ \enquote{You may convey a Combined Work under terms of your
choice that, taken together, effectively do not restrict modification of the
portions of the Library contained in the Combined Work and reverse engineering
for debugging such modifications, if you also do each of the following
[\ldots]}\footnote{\cite[cf.][\nopage wp., §4]{Lgpl30OsiLicense2007a}. The
ellipsis at the end of the sentence denotes a set of tasks which we do not
listen here for saving recourses, but which have to be considered as an
integrated part of this sentence.}}
\end{quote}

Reusing our method of disambiguation, we first can exemplify the meaning of the
LGPL3-RevEng-Sentence by the following text:

\begin{alltt}   
( \([\Theta:]\)
  ( You \(\emph{compliantly distribute}\) a Combined Work 
    under terms of your choice 
    (   (that together effectively, do not restrict modification of 
        the portions of the Library contained in the Combined Work)
    \(\textbf{AND}\) 
        (that together effectively, do not restrict reverse
        engineering for debugging modifications of the portions
        of the Library contained in the Combined Work)
  )  )
  \(\textbf{IF}\)
  \([\Omega:]\) 
  ( you also do each of the following \([\ldots]\))
)
\end{alltt}

But our next step, the logical serialization, let us running into a problem:

If we serialized \emph{($\Theta$ IF $\Omega$)} as \emph{($\Omega$ $\rightarrow$
$\Theta$)}, then from not respecting $\Theta$ would follow by Modus Tollens,
that we are not allowed to realize $\Omega$ -- in other words:
that we may not do even one of the single tasks covered by the ellipsis -- which
is a silly result. 

If we serialized \emph{($\Theta$ IF $\Omega$)} as \emph{($\Theta$ $\rightarrow$
$\Omega$)} then from doing $\Theta$ would successfully follow by Modus Ponens
that we also have to do $\Omega$. And from not respecting $\Omega$ would
successfully follow by Modus Tollens, that we must not do $\Theta$. But
unfortunately, we can respect this second interdiction also \emph{by
distributing a Combined Work under terms} that restrict modifications and/or
reverse engineering (instead of not restricting these techniques) -- which,
again, is a silly result.

Obviously, a simple serialization based on a intutively unclear reading fails.
In fact, the LGPL3-RevEng-Sentence must have a more sophisticated underlying
structure. It must be logically serialized in a form, that integrates the
requirements, not to restrict modifications and reverse enigneering, as really
triggable conditions. Thus, the meaning of the sentence can logically be
explicated as the \emph{LGPL3-RevEng-Rule}:

\begin{alltt}
( \([\Sigma:]\)
  ( You \(\emph{compliantly distribute}\) a Combined Work 
    under terms of your choice 
  ) 
  \(\rightarrow\)  
  (    \([\Gamma:]\)
     ( the terms of your choice together effectively do 
       not restrict modification of the portions of the 
       Library contained in the Combined Work) 
     \(\wedge\) \([\Delta:]\)
     ( the terms of your choice together effectively, do 
       not restrict reverse engineering for debugging 
       modifications of the portions of the Library 
       contained in the Combined Work)
     \(\wedge\) \([\Omega:]\) 
     ( you also do each of the following \([\ldots]\))
) )
\end{alltt}  

This LGPL3-RevEng-Rule indeed successfully regulates how to compliantly
distribute a Combined Work by telling us,

\begin{itemize}
  \item that we have to respect $\Gamma$, $\Delta$ \textbf{and} all single parts
  of $\Omega$, if we distribute a Combined Work compliantly\footnote{follows by
  Modus Ponens. Thus, in this case especially our terms \enquote{[\ldots]
  together effectively \textbf{[must] not restrict reverse engineering} for
  debugging modifications of the portions of the Library contained in the
  Combined Work}}.
  \item that we do not distribute a Combined Work compliantly, if we do not
  respect one of the requirements $\Gamma$, $\Delta$ or one single part of
  $\Omega$\footnote{follows by Modus Tollens. Thus, especially we are not
  distributing a Combined Work compliantly, if our terms \enquote{[\ldots]
  together effectively \textbf{do restrict reverse engineering} for debugging
  modifications of the portions of the Library contained in the Combined Work}}
\end{itemize}

Now, we can directly see, that the LGPLv3 does not enforce us, not to obstruct
reverse engineering in all respects! The required reverse engineering is limited
to the purpose of supporting the debugging of modifications and focused to the
Combined Work containing portions of the Library. In other words: our terms may
obstruct other purposes of reverse engineering or may restrict reverse
engineering of other forms of our work which which can not be specified as
Combined Work or do not contain portions of the Library. Thus, the first crucial
question is, what the LGPL-v3 means if it talks about a \enquote{Combined Work}.
The second question is, what the LGPL-v3 specifies as a portion of the Library.

Again, fortunately, the LGPL-v3 answers clearly: \enquote{A \enquote{Combined
Work} is a work produced by combining or linking an Application with the
Library}\footcite[cf.][\nopage wp., §0]{Lgpl30OsiLicense2007a}. From our LGPL-v2
analysis we know the ways how works that uses a Library can technically be
linked or combined with the Library:

\begin{itemize}
  \item Copying code from the Library into the work using the
  Library\footnote{The LGPL-v3 designates the work using the Library as
  \enquote{Application} and defines that it \enquote{[\ldots] makes use of an
  interface provided by the Library [\ldots]} (\cite[cf.][\nopage wp.,
  §0]{Lgpl30OsiLicense2007a}).} causes that the application respectively the
  work using the Library indeed contains portions of the
  Library\footnote{$\rightarrow$ p. \pageref{RevEngCopyCodeManually}}.
  \item Combining script language based applications and Libraries may evoke
  that the resulting application contains portions of the Library. But the
  details can be neglected with respect to the reverse engineering, because
  script code is distributed as it has been developed and can therefore directly
  be understood\footnote{$\rightarrow$ p. \pageref{RevEngDistributeScripts}}.
  \item Combining java classes and libraries as integrated quasi statically
  linked packages causes, that the resulting package already contains all
  functionally necessary code of the Library\footnote{$\rightarrow$ p.
  \pageref{RevEngDistributeStaticallyCombinedByteCode}}.
  \item Compiling java classes without combining them with the referred Library
  classes causes, that the compiled classes at least contain identifiers having
  been declared by the Library\footnote{$\rightarrow$ p.
  \pageref{RevEngDistributeDynamicallyLinkedCode}}.
  \item Combiling C/C++ files or classes and linking them with the referred
  Libaries statically causes, that the resulting executable indeed contains all
  functional relevant code of all used Libraries\footnote{$\rightarrow$ p.
  \pageref{RevEngDistributeStaticallyLinkedBinaries}}.
  \item Combiling C/C++ files or classes without linking them to the referred
  Libaries causes, that the resulting object file can dynamically be linked on
  another machine and contains identifiers offered by the Library and sometimes
  some functional code injected by dissolving some inline functions or
  macros\footnote{$\rightarrow$ p.
  \pageref{RevEngDistributeDynamicallyLinkedCode}}.
\end{itemize}

So -- overall -- the situation is this: The LGPL3-RevEng-Rule tells us that we
have to allow reverse engineering of the portions of the Library
contained in the Combined Work. The LGPL3 additionally specifies, that a Combined Work
is simply the result of technically combining the work using the Library (the
application) and the Library. Finally the praxis tells us, that (a) combining
both components statically indeed causes that the resulting Combined Work contains
portions of the Library\footnote{So, it is triggering the LGPL3-RevEng-Rule.},
and that (b) we -- in case of preparing the both parts as dynamically
combinable components -- still have to clarify whether the resulting work
already contains portions of the Library.

Just as the LGPL-v2, the LGPL-v3 supports us to answer this question by its §3
whose linguistic conjunctions we thoroughly have to consider:

\begin{quote}\emph{The object code form of an Application may incorporate
material from a header file that is part of the Library. \textbf{You may convey}
such object code under terms of your choice, \emph{provided that}, \textbf{[}
\textbf{if} the incorporated material is \textbf{not} limited to numerical
parameters, data structure layouts and accessors, or small macros, inline
functions and templates (ten or fewer lines in length), \textbf{you do both} of
the following: \textbf{a)} Give prominent notice with each copy of the object
code that the Library is used in it and that the Library and its use are covered
by this License. \textbf{b)} Accompany the object code with a copy of the GNU
GPL and this license document\textbf{]}\footcite[cf.][\nopage wp., §3; emphasis
and additional braces KR.]{Lgpl30OsiLicense2007a}.}
\end{quote}

The first sentence of this paragraph tells us that he is dedicated to object
files which are compiled and not linked to the used Library, but which
nevertheless can contain portions of the Library. The second sentence regulates
the distribution of such object files and can be logically serialized:

\begin{alltt}
( \([\Lambda:]\)
  ( You \(\emph{compliantly distribute}\) object code [incorporating 
    material from the Library] under terms of your choice ) 
  \(\rightarrow\)  
  \([\Xi:]\)
  ( \([\omega:]\)
    ( the incorporated material is not limited to numerical
      parameters, data structure layouts and accessors, or 
      small macros, inline functions and templates 
      [ten or fewer lines in length] ) 
    \(\rightarrow\) 
    ( \([\alpha:]\) ( you do [a] \(\ldots]\) )
    \(\wedge\) \([\beta:]\) ( you do [b] \(\ldots]\) )
) ) )
\end{alltt}  

We see, that this LGPL3-sentence concerning the distribution of object files
contains a main rule (\emph{($\Lambda$ $\rightarrow$ $\Xi$)}) and that the
conclusion $\Xi$ itself has the form of an embedded sub rule (\emph{($\omega$
$\rightarrow$ ( $\alpha$ $\wedge$ $\beta$)}).

Firstly, the main rule enforces us to respect the sub rule if we want to
distribute the object code compliantly\footnote{follows by Modus Ponens to
\emph{($\Lambda$ $\rightarrow$ $\Xi$)}.}. Secondly, the main rule tells us that
we do not distribute the object code compliantly if we do not respect the sub
rule \footnote{follows by Modus Ponens to \emph{($\Lambda$ $\rightarrow$
$\Xi$)}.}.

We have two ways to respect the sub rule, and one way not to respect it:
\begin{itemize}
  \item If the object code contains more and/or larger elements of the Library
  than the limit specifies, then \textbf{we do respect the sub rule}, if we do
  $\alpha$ and $\beta$\footnote{follows by Modus Ponens to \emph{($\omega$
  $\rightarrow$ ($\alpha$ $\wedge$ $\beta$))}.}.
  \item If the object code contains elements of the Library at most up to
  specified limits, then \textbf{we do respect the sub rule} without having to
  do some additionally tasks\footnote{follows by definition of an implication:
  if the premise of this sub rule is false, the sub rule is as whole is true and
  hence respected.}
  \item But if the object code contains more and/or larger elements of the
  Library than the limit specifies \textbf{and} if we do not do $\alpha$ and
  $\beta$, then \textbf{we do not respect the sub rule}\footnote{follows from
  definition of an implication: if the premise is true and the conclusion is
  false, the the implication as whole is false, as well.}.
\end{itemize}

Thus, -- at the end and based on the additional object code specification and
the known empirical background knowledge concerning the software programming --
the LGPL3-RevEng-Rule delivers the same result as the
LGPL2-RevEng-Rule\footnote{$\rightarrow$
\pageref{RevEngLgpl2ComplianceByRenverseEngine}}:

\begin{itemize}
  \item \emph{You are not required to allow reverse engineering if you compile
  the application using the Library as a discret (set of) dynamically linkable
  or combinable file(s) on the base of a standard version of the Library and by
  using the standard compilation methods which preserve the upstream approved
  published interfaces\footnote{and which therefore do not to exceed the LGPL-v2
  limits} and if you distribute the produced unlinked object code or bytecode
  files before they are linked as an executable.}
  \item \emph{In all other cases, you are required to allow reverse engineering
  of a work using a Library -- especially, \ldots}
  \begin{itemize}
    \item \emph{if you distribute the work using the Library and the Library
    together as a statically linked program or as an integrated package
    containing both parts, the work using the library and the Library
    itself\footnote{This holds also if you distribute a script language based
    program or package, notwithstanding the fact, that one does not need the
    permission of reverse engineering to understand script language based
    applications}.}
    \item \emph{if you distribute a work containing manually copied portions of
    the Library.}
  \end{itemize}
\end{itemize}


%% use all entries of the bibliography
%\nocite{*}


\section{Reverse Engineering in the other Open Source Licenses}
% Telekom osCompendium 'for being included' snippet template
%
% (c) Karsten Reincke, Deutsche Telekom AG, Darmstadt 2011
%
% This LaTeX-File is licensed under the Creative Commons Attribution-ShareAlike
% 3.0 Germany License (http://creativecommons.org/licenses/by-sa/3.0/de/): Feel
% free 'to share (to copy, distribute and transmit)' or 'to remix (to adapt)'
% it, if you '... distribute the resulting work under the same or similar
% license to this one' and if you respect how 'you must attribute the work in
% the manner specified by the author ...':
%
% In an internet based reuse please link the reused parts to www.telekom.com and
% mention the original authors and Deutsche Telekom AG in a suitable manner. In
% a paper-like reuse please insert a short hint to www.telekom.com and to the
% original authors and Deutsche Telekom AG into your preface. For normal
% quotations please use the scientific standard to cite.
%
% [ Framework derived from 'mind your Scholar Research Framework' 
%   mycsrf (c) K. Reincke 2012 CC BY 3.0  http://mycsrf.fodina.de/ ]
%

The rest of our way is simple: First, we can ascertain, that none of the other
open source licenses we consider\footnote{$\rightarrow$ p.
\pageref{RevEngOslicOsLisences} }, contain the phrase 'reverse engineering'.
Moreover, they even do not contain one of the single words\footnote{One can
verify this negative statement by (a) loading down all licenses from the OSI
homepage (http://opensource.org/licenses/alphabetical) and by (b) executing the
command \texttt{grep -i "engineering" *} respectively \texttt{grep -i "reverse"
*} in the directory into which the license files have been stored: grep will
find the words \emph{reverse} and \emph{engineering} only in the texts of the
LGPLs.}. So, we may infer, that these most important other open source licenses
could at most indirectly require the permission of reverse engineering. Second,
we know already that distributing script code let the allowance to reverse
engineer, become irrelevant: script code can directly be read and understood, if
one knows the script language\footnote{$\rightarrow$ p.
\pageref{RevEngDistributeScripts}}.
Third, from the definition of strong copleft we may derive, that distributing
software licensed under a strong copyleft license let the permission of reverse
engineering become unimportant, because the source code of the work using the
libraries licensed under a copleft license, must also be made
accessible\footcite[cf.][\nopage wp]{Stallman1996c}.

So -- overally -- we may conclude, that we have only to consider those cases,
where a piece of software is distributed in form of binaries or bytecode, which
uses libraries licensed under permissive open source licenses or under weak
copyleft licenses.

From the definition of being a permissive license or a weak copyleft license we
know already that the licenses of the open source components do not directly
influence the permission or interdiction to use the overarching work which uses
the open source software components\footcite[cf.][20ff.]{Reincke2015a}.

So, if we distribute such a work in form of dynamically linkable, but still not
linked binaries or bytecode files, then there is no way to reasonably derive
that the work using the components, may be reverse engineered: The permissive or
weak copyleft open source licenses mainly concern the open source components,
not the work using the components. On the one side, these licenses indeed
require that we add the license texts and the copyright lines of all the open
source components our work wants to use, to the distributed package containing
our work. And the lisenses prohibit to modify the licensing assertions being
integrated into the open source components our work wants to use\footnote{These
requirements are part of all the open source licenses we consider here. For
details \cite[cf.][chapter 6.]{Reincke2015a}}. But -- on the other side -- the
freedom to use, to study, to modify, or to distribute the software, which is
established by the open source licenses, concerns only the open source
components themselves, not the open source based work itself. But these
components, which may be studied or modified, still are not part of the compiled
work as long as it is not linked to or combined with the open source components
in accordance to the standard compilation and computation methods\footnote{The
only way to infer that the licenses of the components operates also on the using
work, is to argue that the using work must at least contain elements
(identifiers etc.) of the interfaces declared (but not defined) by the libraries
and that therefore at least these elements may be investigated or modified. This
challenge is explicitly addressed by the LGPL\footnote{$\rightarrow$ p.
\pageref{RevEngDistributeDynamicallyLinkedCode}}. Fortunately, it is a general
feature of software libraries that they must and shall be used in accordance to
the interfaces, the developers of the libraries have designed to make their
libraries practically usable. So, if the licenses -- in contrary to the LGPLs --
do not explicitly address the issue of implicitly included portions of the
library in case of unlinked binaries or bytecode files which have been compiled
in accordance to the standard methods and which therefore use open source
software by reffering to their standard interfaces, then one has to infer from
the nature of computation, that the developers have implictly allowed without
any requirements such an integration of declared, but not defined interface
elements, because they have designed the interface as they did and because they
have licensed their work as they did. If they had not wished to use these
elements without any requirements, hey had designed another interface. And if
they had wished to incorporate any copyleft effect or permission of reverse
engineering, then they would have selected another license. But again: this
conclusion holds only for the standard methods to use a software library.}.

On the other side, if we compliantly distribute the work using the components,
as a statically linked binary or bytecode file -- which therefore already
contains all the necessary components\footnote{instead of only the declared
interface elements!} and can directly be executed --, then we are also obliged
to add all the open source license texts and all the copyright lines to our
package, and we are not allowed to modify one of the licensing assertions
integrated into the original open source components\footcite[cf.][chapter
6.]{Reincke2015a}. Thus, one might conclude, that the freedom to use and to
modify the open source components themselves, survive if we distribute software
statically linked to or combined with the open source components. So, the
receiver of the statically linked work probably is allowed to modify the
embedded open source components - even if he had to edit the binary or bytecode
files. Methods to develop binary files reversely, are known as reverse
engineering. Hence, if we distribute a statically linked work using open source
licensed components, we have at least to fear that our receivers indirectly have
also got the permission to reverse engineer our complete product. And we have to
fear so even if the statically linked libraries are licensed under any
permissive or weak copleft license.

So, again, we can summarize the result in the following form:

\begin{itemize}
  \item \emph{With respect to a Library licensed under any permissive or weak
  copyleft license, you are not required to allow reverse engineering, if you
  [A] develop your work using the Library, on the base of a standard version of
  the Library containing the interfaces as the original developers have designed it,
  if you [B] compile your work using this Library, as a discret (set of)
  dynamically linkable or combinable file(s), if you [C] use only the standard
  compilation methods which preserve the upstream approved interfaces, and if
  you [D] distribute the produced unlinked object code or bytecode files before
  they are linked as an executable.}
  \item \emph{In all other cases of distributing a work using such a Library,
  you have at least to fear that you are implictly allowing reverse engineering
  of the work using this Library -- especially, \ldots}
  \begin{itemize}
    \item \emph{if you distribute the work using the Library and the Library
    together as a statically linked program or as an integrated package
    containing both parts, the work using the library and the Library
    itself\footnote{This holds also if you distribute a script language based
    program or package, notwithstanding the fact, that one does not need the
    permission of reverse engineering to understand script language based
    applications}.}
    \item \emph{if you distribute a work containing manually copied portions of
    the Library.}
  \end{itemize}
\end{itemize}


%% use all entries of the bibliography
%\nocite{*}


\section{Reverse Engineering in Open Source Licenses: Summary}
% Telekom osCompendium 'for being included' snippet template
%
% (c) Karsten Reincke, Deutsche Telekom AG, Darmstadt 2011
%
% This LaTeX-File is licensed under the Creative Commons Attribution-ShareAlike
% 3.0 Germany License (http://creativecommons.org/licenses/by-sa/3.0/de/): Feel
% free 'to share (to copy, distribute and transmit)' or 'to remix (to adapt)'
% it, if you '... distribute the resulting work under the same or similar
% license to this one' and if you respect how 'you must attribute the work in
% the manner specified by the author ...':
%
% In an internet based reuse please link the reused parts to www.telekom.com and
% mention the original authors and Deutsche Telekom AG in a suitable manner. In
% a paper-like reuse please insert a short hint to www.telekom.com and to the
% original authors and Deutsche Telekom AG into your preface. For normal
% quotations please use the scientific standard to cite.
%
% [ Framework derived from 'mind your Scholar Research Framework' 
%   mycsrf (c) K. Reincke 2012 CC BY 3.0  http://mycsrf.fodina.de/ ]
%

So, finally we can compile all our results into one single result:
%\pageref{RevEngOsiLicenses}
\begin{itemize}
  \item \emph{With respect to any open source Library\footnote{$\rightarrow$ p.
  }, you are not required to allow reverse
  engineering, if you [A] develop your work using the Library, on the base of a
  standard version of the Library containing the interfaces as the original
  developers have designed it, if you [B] compile your work using this Library,
  as a discret (set of) dynamically linkable or combinable file(s), if you [C]
  use only the standard compilation methods which preserve the upstream approved
  interfaces\footnote{and which therefore do not to exceed limits, prescribed by
  the owners of the Library}, and if you [D] distribute the produced unlinked
  object code or bytecode files before they are linked as an executable.}
  \item \emph{In all other cases of distributing your work using such a Library,
  you are probably required to allow reverse engineering of your work. By all
  means, you have at least to fear that you are implictly allowing reverse
  engineering of your work using such a Library -- especially, \ldots}
  \begin{itemize}
    \item \emph{if you distribute the work using the Library and the Library
    together as a statically linked program or as an integrated package
    containing both parts, the work using the library and the Library
    itself\footnote{This holds also if you distribute a script language based
    program or package, notwithstanding the fact, that one does not need the
    permission of reverse engineering to understand script language based
    applications}.}
    \item \emph{if you distribute a work containing manually copied portions of
    the Library.}
  \end{itemize}
\end{itemize}
 
And, so, we can reformulate our result as a slightly modified \enquote{rule of
thumbs} originally offered by an open source expert who analyzed the problem of
protecting your own work from an other viewport:

\begin{itemize}
  \item \enquote{DO NOT statically link [or combine] [open source] code if you
  wish to keep your program proprietary [and if you want to protect it against reverse
  engineering]}\footcite[cf.][6; bracketed text KR.]{Ilardi2010a}.
  \item \enquote{DO dynamically link to [any open source code, not only to] LGPL
  code}\footcite[cf.][6; bracketed text KR.]{Ilardi2010a}.
\end{itemize}

$\qed$

%% use all entries of the bibliography
%\nocite{*}



\label{Disclaimer}
\section{Disclaimer}
% Telekom osCompendium 'for beeing included' snippet template
%
% (c) Karsten Reincke, Deutsche Telekom AG, Darmstadt 2011
%
% This LaTeX-File is licensed under the Creative Commons Attribution-ShareAlike
% 3.0 Germany License (http://creativecommons.org/licenses/by-sa/3.0/de/): Feel
% free 'to share (to copy, distribute and transmit)' or 'to remix (to adapt)'
% it, if you '... distribute the resulting work under the same or similar
% license to this one' and if you respect how 'you must attribute the work in
% the manner specified by the author ...':
%
% In an internet based reuse please link the reused parts to www.telekom.com and
% mention the original authors and Deutsche Telekom AG in a suitable manner. In
% a paper-like reuse please insert a short hint to www.telekom.com and to the
% original authors and Deutsche Telekom AG into your preface. For normal
% quotations please use the scientific standard to cite.
%
% [ File structure derived from 'mind your Scholar Research Framework' 
%   mycsrf (c) K. Reincke CC BY 3.0  http://mycsrf.fodina.de/ ]

%

%%% \chapter*{Disclaimer} %%%

This article is thoroughly developed. Finally -- and preferably with the help
of the open source community --, it shall deliver reliable information. But
nevertheless, it can not offer more than the opinion(s) of its authors and
contributors. It is only one voice of the chorus discussing the open source
licenses. For protecting the authors and contributors from charges and claims of
indemnification we adopt the lightly modified GPL3 disclaimer:

{ \newcommand\gummi{\hspace{0pt plus 1pc}} THERE IS NO WARRANTY FOR THIS
ATRTICLE, TO THE EXTENT PERMITTED BY APPLICABLE LAW. THE COPYRIGHT HOLDERS
AND/OR OTHER PARTIES PROVIDE THE TEXT “AS IS” WITHOUT WARRANTY OF ANY KIND,
EITHER EXPRESSED OR IMPLIED, INCLUDING, BUT NOT LIMITED TO, THE IMPLIED
WARRANTIES OF MERCHANTABILITY AND FITNESS FOR A PARTICULAR PURPOSE. THE ENTIRE
RISK AS TO THE QUALITY AND PERFORMANCE OF THE \oslic{} IS WITH YOU. \gummi
SHOULD THE \oslic{} PROVE DEFECTIVE, \gummi YOU ASSUME THE COST OF ALL NECESSARY
SERVICING, REPAIR OR CORRECTION.

IN NO EVENT \gummi UNLESS REQUIRED BY APPLICABLE LAW \gummi OR AGREED TO IN
WRITING WILL ANY COPYRIGHT HOLDER, OR ANY OTHER PARTY WHO MODIFIES AND/OR
CONVEYS THE \oslic{} AS PERMITTED ABOVE, BE LIABLE TO YOU FOR DAMAGES, INCLUDING
ANY GENERAL, \gummi SPECIAL, \gummi INCIDENTAL \gummi OR CONSEQUENTIAL DAMAGES
ARISING OUT OF THE USE OR INABILITY TO USE THIS ARTICLE (INCLUDING BUT NOT
LIMITED TO LOSS OF DATA OR DATA BEING RENDERED INACCURATE OR LOSSES SUSTAINED BY
YOU OR THIRD PARTIES OR A FAILURE OF THE \oslic{} TO COOPERATE WITH ANY OTHER
TOOLS), EVEN IF SUCH HOLDER OR OTHER PARTY HAS BEEN ADVISED OF THE POSSIBILITY
OF SUCH DAMAGES.
}

% \textit{Particularly, it must be highlighted that - referred to your solitary
% case - this work can not and shall not replace a legal review or a legal advice
% by lawyers. It may inspire developers, managers, open source experts, and
% companies to find good own solutions. But it only contains legal reflections
% addressed to a general public\footnote{For German readers: Of course, also this
% work respects the German 'Rechtsdienstleistungsgesetz' and may only be read as
% an \enquote{nur an die Allgemeinheit gerichtete Darstellung und Erörterung von
% Rechtsfragen.}}



\label{License}
\section{License}
% Telekom osCompendium License Include Module
%
% (c) Karsten Reincke, Deutsche Telekom AG, Darmstadt 2011
%
% This LaTeX-File is licensed under the Creative Commons Attribution-ShareAlike
% 3.0 Germany License (http://creativecommons.org/licenses/by-sa/3.0/de/): Feel
% free 'to share (to copy, distribute and transmit)' or 'to remix (to adapt)'
% it, if you '... distribute the resulting work under the same or similar
% license to this one' and if you respect how 'you must attribute the work in
% the manner specified by the author ...':
%
% In an internet based reuse please link the reused parts to www.telekom.com and
% mention the original authors and Deutsche Telekom AG in a suitable manner. In
% a paper-like reuse please insert a short hint to www.telekom.com and to the
% original authors and Deutsche Telekom AG into your preface. For normal
% quotations please use the scientific standard to cite.
%
% [ File structure derived from 'mind your Scholar Research Framework' 
%   mycsrf (c) K. Reincke CC BY 3.0  http://mycsrf.fodina.de/ ]
%


This text is licensed under the Creative Commons Attribution-ShareAlike 3.0 Germany
License (http://creativecommons.org/licenses/by-sa/3.0/de/): Feel free \enquote{to
share (to copy, distribute and transmit)} or \enquote{to remix (to
adapt)} it, if you \enquote{[\ldots] distribute the resulting work under the
same or similar license to this one} and if you respect how \enquote{you
must attribute the work in the manner specified by the author(s)
[\ldots]}):
\newline
In an internet based reuse please mention the initial authors in a suitable
manner, name their sponsor \textit{Deutsche Telekom AG} and link it to
\texttt{http://www.telekom.com}. In a paper-like reuse please insert a short
hint to \texttt{http://www.telekom.com}, to the initial authors, and to their
sponsor \textit{Deutsche Telekom AG} into your preface. For normal quotations
please use the scientific standard to cite.
% In an internet based reuse please link the reused parts to www.telekom.com
% and mention the original authors and Deutsche Telekom AG in a suitable manner. In
% a paper-like reuse please insert a short hint to www.telekom.com and to the
% original authors and Deutsche Telekom AG into your preface. For normal
% quotations please use the scientific standard to cite.
\newline
{ \tiny \itshape [derived from myCsrf (= 'mind your Scholar Research Framework') 
\copyright K. Reincke CC BY 3.0  http://mycsrf.fodina.de/)] }



\footnotesize
\bibliography{../bibfiles/oscResourcesEn}

\end{document}

