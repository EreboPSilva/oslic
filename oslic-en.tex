% Telekom osCompendium cloak file English text
%
% (c) Karsten Reincke, Deutsche Telekom AG, Darmstadt 2011
%
% This LaTeX-File is licensed under the Creative Commons Attribution-ShareAlike
% 3.0 Germany License (http://creativecommons.org/licenses/by-sa/3.0/de/): Feel
% free 'to share (to copy, distribute and transmit)' or 'to remix (to adapt)'
% it, if you '... distribute the resulting work under the same or similar
% license to this one' and if you respect how 'you must attribute the work in
% the manner specified by the author ...':
%
% In an internet based reuse please link the reused parts to www.telekom.com and
% mention the original authors and Deutsche Telekom AG in a suitable manner. In
% a paper-like reuse please insert a short hint to www.telekom.com and to the
% original authors and Deutsche Telekom AG into your preface. For normal
% quotations please use the scientific standard to cite.
%
% [ File structure derived from 'mind your Scholar Research Framework' 
%   mycsrf (c) K. Reincke CC BY 3.0  http://mycsrf.fodina.de/ ]

\documentclass[DIV=calc,BCOR=5mm,12pt,headings=small,oneside,toc=bib]{scrbook}

%%% (1) general configurations %%%
\usepackage[utf8]{inputenc}

%%% (2) language specific configurations %%%
\usepackage[]{a4,ngerman}
\usepackage[ngerman, english]{babel}
\selectlanguage{english}

%language specific quoting signs
%default for language emglish is american style of quotes
\usepackage[english=british]{csquotes}

% jurabib configuration
\usepackage[see]{jurabib}
\bibliographystyle{jurabib}
% Telekom osCompendium English Jurabib Configuration Include Module 
%
% (c) Karsten Reincke, Deutsche Telekom AG, Darmstadt 2011
%
% This LaTeX-File is licensed under the Creative Commons Attribution-ShareAlike
% 3.0 Germany License (http://creativecommons.org/licenses/by-sa/3.0/de/): Feel
% free 'to share (to copy, distribute and transmit)' or 'to remix (to adapt)'
% it, if you '... distribute the resulting work under the same or similar
% license to this one' and if you respect how 'you must attribute the work in
% the manner specified by the author ...':
%
% In an internet based reuse please link the reused parts to www.telekom.com and
% mention the original authors and Deutsche Telekom AG in a suitable manner. In
% a paper-like reuse please insert a short hint to www.telekom.com and to the
% original authors and Deutsche Telekom AG into your preface. For normal
% quotations please use the scientific standard to cite.
%
% [ File structure derived from 'mind your Scholar Research Framework' 
%   mycsrf (c) K. Reincke CC BY 3.0  http://mycsrf.fodina.de/ ]

% the first time cite with all data, later with shorttitle
\jurabibsetup{citefull=first}

%%% (1) author / editor list configuration
%\jurabibsetup{authorformat=and} % uses 'und' instead of 'u.'
% therefore define your own abbreviated conjunction: 
% an 'and before last author explicetly written conjunction

% for authors in citations
\renewcommand*{\jbbtasep}{ a.\ } % bta = between two authors sep
\renewcommand*{\jbbfsasep}{, } % bfsa = between first and second author sep
\renewcommand*{\jbbstasep}{, a.\ }% bsta = between second and third author sep
% for editors in citations
\renewcommand*{\jbbtesep}{ a.\ } % bta = between two authors sep
\renewcommand*{\jbbfsesep}{, } % bfsa = between first and second author sep
\renewcommand*{\jbbstesep}{, a.\ }% bsta = between second and third author sep

% for authors in literature list
\renewcommand*{\bibbtasep}{ a.\ } % bta = between two authors sep
\renewcommand*{\bibbfsasep}{, } % bfsa = between first and second author sep
\renewcommand*{\bibbstasep}{, a.\ }% bsta = between second and third author sep
% for editors  in literature list
\renewcommand*{\bibbtesep}{ a.\ } % bte = between two editors sep
\renewcommand*{\bibbfsesep}{, } % bfse = between first and second editor sep
\renewcommand*{\bibbstesep}{, a.\ }% bste = between second and third editor sep

% use: name, forname, forname lastname u. forname lastname
\jurabibsetup{authorformat=firstnotreversed}
\jurabibsetup{authorformat=italic}

%%% (2) title configuration
% in every case print the title, let it be seperated from the 
% author by a colon and use the slanted font
\jurabibsetup{titleformat={all,colonsep}}
%\renewcommand*{\jbtitlefont}{\textit}

%%% (3) seperators in bib data
% separate bibliographical hints and page hints by a comma
\jurabibsetup{commabeforerest}

%%% (4) specific configuration of bibdata in quotes / footnote
% use a.a.O if possible
\jurabibsetup{ibidem=strict}
% replace ugly a.a.O. by translation of ders., a.a.O.
\AddTo\bibsgerman{
  \renewcommand*{\ibidemname}{Id., l.c.}
  \renewcommand*{\ibidemmidname}{id., l.c.}
}
\renewcommand*{\samepageibidemname}{Id., ibid.}
\renewcommand*{\samepageibidemmidname}{id., ibid.}


%%% (5) specific configuration of bibdata in bibliography
% ever an in: before journal and collection/book-tiltes 
\renewcommand*{\bibbtsep}{in: }
\renewcommand*{\bibjtsep}{in: }
% ever a colon after author names 
\renewcommand*{\bibansep}{: }
% ever a semi colon after the title
% \AddTo\bibsgerman{\renewcommand*{\urldatecomment}{Referenzdownload: }}
\renewcommand*{\bibatsep}{; }
% ever a comma before date/year
\renewcommand*{\bibbdsep}{, }

% let jurabib insert the S. and p. information
% no S. necessary in bib-files and in cites/footcites
\jurabibsetup{pages=format}

% use a compressed literature-list using a small line indent
\jurabibsetup{bibformat=compress}
\setlength{\jbbibhang}{1em}

% which follows the design of the cites and offers comments
\jurabibsetup{biblikecite}

% print annotations into bibliography
\jurabibsetup{annote}
\renewcommand*{\jbannoteformat}[1]{{ \itshape #1 }}

%refine the prefix of url download
\AddTo\bibsgerman{\renewcommand*{\urldatecomment}{reference download: }}

% we want to have the year of articles in brackets
\renewcommand*{\bibaldelim}{(}
\renewcommand*{\bibardelim}{)}

% in english version Nr. must be replaced by No.
\renewcommand*{\artnumberformat}[1]{\unskip,\space No.~#1}
\renewcommand*{\pernumberformat}[1]{\unskip\space No.~#1}%
\renewcommand*{\revnumberformat}[1]{\unskip\space No.~#1}%


%Reformatierung Seriestitels and Seriesnumber
\DeclareRobustCommand{\numberandseries}[2]{%
\unskip\unskip%,
\space\bibsnfont{(=~#2}%
\ifthenelse{\equal{#1}{}}{)}{, [Vol./No.]~#1)}%
}%


% Local Variables:
% mode: latex
% fill-column: 80
% End:


% language specific hyphenation
% Telekom osCompendium osHyphenation Include Module
%
% (c) Karsten Reincke, Deutsche Telekom AG, Darmstadt 2011
%
% This LaTeX-File is licensed under the Creative Commons Attribution-ShareAlike
% 3.0 Germany License (http://creativecommons.org/licenses/by-sa/3.0/de/): Feel
% free 'to share (to copy, distribute and transmit)' or 'to remix (to adapt)'
% it, if you '... distribute the resulting work under the same or similar
% license to this one' and if you respect how 'you must attribute the work in
% the manner specified by the author ...':
%
% In an internet based reuse please link the reused parts to www.telekom.com and
% mention the original authors and Deutsche Telekom AG in a suitable manner. In
% a paper-like reuse please insert a short hint to www.telekom.com and to the
% original authors and Deutsche Telekom AG into your preface. For normal
% quotations please use the scientific standard to cite.
%
% [ File structure derived from 'mind your Scholar Research Framework' 
%   mycsrf (c) K. Reincke CC BY 3.0  http://mycsrf.fodina.de/ ]
%


\hyphenation{rein-cke}
\hyphenation{OS-LiC}
\hyphenation{ori-gi-nal}


%%% (3) layout page configuration %%%

% select the visible parts of a page
% S.31: { plain|empty|headings|myheadings }
%\pagestyle{myheadings}
\pagestyle{headings}

% select the wished style of page-numbering
% S.32: { arabic,roman,Roman,alph,Alph }
\pagenumbering{arabic}
\setcounter{page}{1}

% select the wished distances using the general setlength order:
% S.34 { baselineskip| parskip | parindent }
% - general no indent for paragraphs
\setlength{\parindent}{0pt}
\setlength{\parskip}{1.2ex plus 0.2ex minus 0.2ex}

%%% (4) general package activation %%%
%\usepackage{utopia}
%\usepackage{courier}
%\usepackage{avant}
\usepackage[dvips]{epsfig}

% graphic
\usepackage{graphicx,color}
\usepackage{array}
\usepackage{shadow}
\usepackage{fancybox}

%- start(footnote-configuration)
%  flush the cite numbers out of the vertical line and let
%  the footnote text directly start in the left vertical line
% \usepackage[marginal,hang]{footmisc}
% \renewcommand\footnotemargin{1.5em}

% formatting the footnote with koma script tools
% \deffootnote[1em]{1.5em}{1em}{\textsuperscript{\thefootnotemark}}
\deffootnote[1.5em]{1.5em}{1.5em}{\textsuperscript{\thefootnotemark)\ }}


%\deffootnote[0em]{1.5em}{1em}{\textsuperscript{\thefootnotemark}}
%- end(footnote-configuration)


% %- start(endnote-configuration) uncomment to activate
% % Let all notes being marked with \endnote instead of \footnote
% % become endnotes. This set of endnotes replaces the next 
% % arising command \theendnotes - even if it is not located
% % at the end of the text.
% 
% \usepackage{endnotes}
% 
% % Format endnotes as Block with indention - Solution 1
% %\renewcommand\enoteformat{%
% %   \noindent\theenmark.) \ \hangindent .7\parindent%
% %}
% 
% % Format endnotes as Block with indention - Solution 2
% \makeatletter
% \def\enoteformat{\rightskip\z@ \leftskip0em \parindent=0em \parskip=0em
% \leavevmode\llap{\hbox{\@theenmark.~}}}
% \makeatother
% 
% \renewcommand\notesname{Annotations}
% % additionally we shall active a special jurabib option
% % if we want to get all jurabib footnotes as endnotes
% \jurabibsetup{citetoend=true}
% %- end(footnote-configuration)

% - additional packages

\usepackage{tikz}
\usetikzlibrary{arrows}
\usetikzlibrary{shapes,snakes}
\usetikzlibrary{positioning}
\usetikzlibrary{decorations.text}

\usepackage{multirow}

\usepackage{blindtext}
\usepackage{caption}

\usetikzlibrary{matrix}

\usepackage{amsmath,amsfonts}
\usepackage{amssymb}
\usepackage{wasysym}
\usepackage{pstricks, pst-node, pst-tree}
\usepackage{chngcntr}
\usepackage{nameref}



\counterwithout{footnote}{chapter}

\usepackage[intoc]{nomencl}
\let\abbr\nomenclature
% Modify Section Title of nomenclature
\renewcommand{\nomname}{Periodicals, Shortcuts, and Abbreviations}
%\renewcommand{\nomname}{Periodika, ihre Kurzformen und generelle Abkürzungen}

% insert point between abbrewviation and explanation
\setlength{\nomlabelwidth}{.24\hsize}
\renewcommand{\nomlabel}[1]{#1 \dotfill}
% reduce the line distance
\setlength{\nomitemsep}{-\parsep}
\makenomenclature

% depth of contents
\setcounter{secnumdepth}{5}
\setcounter{tocdepth}{5}
%%%%%%%%%%%%%%
\begin{document}

%% use all entries of the bliography
%% \nocite{*}

%%-- start(titlepage)
\titlehead{Version 0.98.1
 % -- February, 2013

}
\subject{\small \itshape A Practical Guide for Developers, Managers, OS Experts, 
and Companies} 

\title{Open Source License Compendium}

\subtitle{How to Achieve Open Source License Compliance% Telekom osCompendium License Include Module
%
% (c) Karsten Reincke, Deutsche Telekom AG, Darmstadt 2011
%
% This LaTeX-File is licensed under the Creative Commons Attribution-ShareAlike
% 3.0 Germany License (http://creativecommons.org/licenses/by-sa/3.0/de/): Feel
% free 'to share (to copy, distribute and transmit)' or 'to remix (to adapt)'
% it, if you '... distribute the resulting work under the same or similar
% license to this one' and if you respect how 'you must attribute the work in
% the manner specified by the author ...':
%
% In an internet based reuse please link the reused parts to www.telekom.com and
% mention the original authors and Deutsche Telekom AG in a suitable manner. In
% a paper-like reuse please insert a short hint to www.telekom.com and to the
% original authors and Deutsche Telekom AG into your preface. For normal
% quotations please use the scientific standard to cite.
%
% [ File structure derived from 'mind your Scholar Research Framework' 
%   mycsrf (c) K. Reincke CC BY 3.0  http://mycsrf.fodina.de/ ]
%
\footnote{
This text is licensed under the Creative Commons Attribution-ShareAlike 3.0 Germany
License (http://creativecommons.org/licenses/by-sa/3.0/de/): Feel free \enquote{to
share (to copy, distribute and transmit)} or \enquote{to remix (to
adapt)} it, if you \enquote{[\ldots] distribute the resulting work under the
same or similar license to this one} and if you respect how \enquote{you
must attribute the work in the manner specified by the author(s)
[\ldots]}):
\newline
In an internet based reuse please mention the initial authors in a suitable
manner, name their sponsor \textit{Deutsche Telekom AG} and link it to
\texttt{http://www.telekom.com}. In a paper-like reuse please insert a short
hint to \texttt{http://www.telekom.com}, to the initial authors, and to their
sponsor \textit{Deutsche Telekom AG} into your preface. For normal quotations
please use the scientific standard to cite.
\newline
{ \tiny \itshape [derived from myCsrf (= 'mind your Scholar Research Framework') 
\copyright K. Reincke CC BY 3.0  http://mycsrf.fodina.de/)] }}}
\author{
Karsten Reincke\thanks{Deutsche Telekom AG, Products \& Innovation, 
T-Online-Allee 1, 64295 Darmstadt}
\and
Greg Sharpe\thanks{Deutsche Telekom AG, Telekom Deutschland GmbH, 
Landgrabenweg, Bonn}}

\maketitle
%%-- end(titlepage)

\footnotesize
\begin{flushright} 

\parbox{100mm}{\itshape
The Open Source Community is a swarm: it is more powerful than a set of
arbritarily selected experts. We are proud to have its support. Gladly we thank
(in alphabetical order):
}

\parbox{50mm}{
\tiny
\begin{flushright}
Eitan Adler,\\
Stefan Altmeyer,\\
John Dobson, \\
Steffen Härtlein, \\
Ta'Id Holmes, \\
Michael Kern,\\
Michael Machado,\\
Thorsten Müller,\\
Thomas Quiehl,\\
Peter Schichl,\\
Helene Tamer,\\
Bernhard Tsai,\\
and all the others\ldots
\end{flushright}
}
\end{flushright}
\normalsize
\newpage

\footnotesize
\tableofcontents
\newpage
% Telekom osCompendium 'for being included' snippet template
%
% (c) Karsten Reincke, Deutsche Telekom AG, Darmstadt 2011
%
% This LaTeX-File is licensed under the Creative Commons Attribution-ShareAlike
% 3.0 Germany License (http://creativecommons.org/licenses/by-sa/3.0/de/): Feel
% free 'to share (to copy, distribute and transmit)' or 'to remix (to adapt)'
% it, if you '... distribute the resulting work under the same or similar
% license to this one' and if you respect how 'you must attribute the work in
% the manner specified by the author ...':
%
% In an internet based reuse please link the reused parts to www.telekom.com and
% mention the original authors and Deutsche Telekom AG in a suitable manner. In
% a paper-like reuse please insert a short hint to www.telekom.com and to the
% original authors and Deutsche Telekom AG into your preface. For normal
% quotations please use the scientific standard to cite.
%
% [ File structure derived from 'mind your Scholar Research Framework' 
%   mycsrf (c) K. Reincke CC BY 3.0  http://mycsrf.fodina.de/ ]

%


%% use all entries of the bibliography


\begin{table}
\footnotesize
\caption{History of the Open Source License Compendium}
\begin{center}
\begin{tabular}{|r|c|p{10cm}|}
\hline
\hline
    \texttt{2013-03-15}
  & \texttt{0.92.1} 
  & CeBIT release\newline
    $\vartriangleright$ to-do lists for the many important licenses added\newline
    $\vartriangleright$ branches merged and new master published\\
\hline
    \texttt{2013-03-08}
  & \texttt{0.90.1} 
  & CeBIT release\newline
    $\vartriangleright$ to-do lists for the some important licenses added\newline
    $\vartriangleright$ branches merged and new master published\\
\hline
    \texttt{2013-02-16}
  & \texttt{0.8.90} 
  & CeBIT pre release\newline
    $\vartriangleright$ new arguing structure focused on the topic license fulfillment\newline
    $\vartriangleright$ new classifying license review\newline   
    $\vartriangleright$ new top down introduction\\
\hline
    \texttt{2012-12-28}
  & \texttt{0.8.0} 
  & inmternal EOY release\newline
    $\vartriangleright$ many distributed improvents unified in branch kreinck\\
\hline
    \texttt{2012-08-25}
  & \texttt{0.5.2} 
  & expanded break through release\newline
    $\vartriangleright$ MIT license fulfilling to-do lists\newline
    $\vartriangleright$ using integrated Eclipse spell checking methods\\
\hline
    \texttt{2012-07-06}
  & \texttt{0.4.0} 
  & break through release\newline
    $\vartriangleright$ open source use case definition and taxonomy\newline 
    $\vartriangleright$ open source use case based general finder\newline 
    $\vartriangleright$ corresponding BSD specific mini finder\newline 
    $\vartriangleright$ BSD license fulfilling to-do lists\\
\hline
    \texttt{2012-03-22}
  & \texttt{0.2.1} 
  & $\vartriangleright$ framework published as first community edition\\
\hline
    \texttt{2012-01-31}
  & \texttt{0.1.8} 
  & $\vartriangleright$ renamed existing introduction as prolegomena\newline
    $\vartriangleright$ inserted a shorter top-down written introduction\newline
    $\vartriangleright$ inserted an OSLiC disclaimer\\
\hline
    \texttt{2012-01-21}
  & \texttt{0.1.7} 
  & $\vartriangleright$ oscCopiedButNotRead.bib expanded\newline 
  $\vartriangleright$ list of periodicals and shortcuts added\\
\hline
    \texttt{2011-12-29}
  & \texttt{0.1.6} 
  & $\vartriangleright$ many bibliographic data added\\
\hline
    \texttt{2011-10-17}
  & \texttt{0.1.5} 
  & $\vartriangleright$ bibliographic data updated\\
\hline
    \texttt{2011-09-29}
  & \texttt{0.1.4} 
  & $\vartriangleright$ document history integrated\newline
    $\vartriangleright$ typos erased\\
\hline
    \texttt{2011-09-28}
  & \texttt{0.1.3} 
  & $\vartriangleright$ review of english teacher integrated \\
\hline
    \texttt{2011-09-19}
  & \texttt{0.1.2} 
  & $\vartriangleright$ first comments of english teacher integrated \\
\hline
    \texttt{2011-09-15}
  & \texttt{0.1.1} 
  & $\vartriangleright$ improvements of John integrated\\
\hline
    \texttt{2011-09-12}
  & \texttt{0.1.0} 
  & $\vartriangleright$ introduction completed: purpose and methods \\
\hline
\hline 
\end{tabular}
\end{center}
\end{table}

\begin{footnotesize}
\begin{itemize}
  \item Insert taks lists for AGPL, CDDL, MS-RL \ldots
  \item Improve / complete the expositions of the concept of a derivative work
  in the conxtext of software development
  \item Discuss the license compatibility
  \item Explain the relationship between open source and earning money
  \item Complete the discussion of the dynamically and statically linked open
  source software by a summary of the respective secondary literature
  \item Improve / expand the integration of the used secondary literature
  \item 
  \item 
\end{itemize}
\end{footnotesize}

%\bibliography{../bibfiles/oscResourcesEn}

\normalsize

\chapter*{Disclaimer}
% Telekom osCompendium 'for beeing included' snippet template
%
% (c) Karsten Reincke, Deutsche Telekom AG, Darmstadt 2011
%
% This LaTeX-File is licensed under the Creative Commons Attribution-ShareAlike
% 3.0 Germany License (http://creativecommons.org/licenses/by-sa/3.0/de/): Feel
% free 'to share (to copy, distribute and transmit)' or 'to remix (to adapt)'
% it, if you '... distribute the resulting work under the same or similar
% license to this one' and if you respect how 'you must attribute the work in
% the manner specified by the author ...':
%
% In an internet based reuse please link the reused parts to www.telekom.com and
% mention the original authors and Deutsche Telekom AG in a suitable manner. In
% a paper-like reuse please insert a short hint to www.telekom.com and to the
% original authors and Deutsche Telekom AG into your preface. For normal
% quotations please use the scientific standard to cite.
%
% [ File structure derived from 'mind your Scholar Research Framework' 
%   mycsrf (c) K. Reincke CC BY 3.0  http://mycsrf.fodina.de/ ]

%

%%% \chapter*{Disclaimer} %%%

This book shall be thoroughly developed -- together with the open source
community. Finally, it shall deliver reliable information with respect to the
rule that the swarm knows more than the single fish.

But nevertheless it can not offer more than the opinion(s) of its authors and
contributors. It is only one voice of chorus discussing the topic of open source
licenses. For protecting the authors and contributors from charges and claims of
idemnification we adopt the lightly modified GPL3 disclaimer:

THERE IS NO WARRANTY FOR THE OSLiC, TO THE EXTENT PERMITTED BY APPLICABLE LAW.
THE COPYRIGHT HOLDERS AND/OR OTHER PARTIES PROVIDE THE TEXT “AS IS” WITHOUT
WARRANTY OF ANY KIND, EITHER EXPRESSED OR IMPLIED, INCLUDING, BUT NOT LIMITED
TO, THE IMPLIED WARRANTIES OF MERCHANTABILITY AND FITNESS FOR A PARTICULAR
PURPOSE. THE ENTIRE RISK AS TO THE QUALITY AND PERFORMANCE OF THE OSLiC IS
WITH YOU. SHOULD THE OSLiC PROVE DEFECTIVE, YOU ASSUME THE COST OF ALL
NECESSARY SERVICING, REPAIR OR CORRECTION.

IN NO EVENT UNLESS REQUIRED BY APPLICABLE LAW OR AGREED TO IN WRITING WILL ANY
COPYRIGHT HOLDER, OR ANY OTHER PARTY WHO MODIFIES AND/OR CONVEYS THE OSLiC AS
PERMITTED ABOVE, BE LIABLE TO YOU FOR DAMAGES, INCLUDING ANY GENERAL, SPECIAL,
INCIDENTAL OR CONSEQUENTIAL DAMAGES ARISING OUT OF THE USE OR INABILITY TO USE
THE PROGRAM (INCLUDING BUT NOT LIMITED TO LOSS OF DATA OR DATA BEING RENDERED
INACCURATE OR LOSSES SUSTAINED BY YOU OR THIRD PARTIES OR A FAILURE OF THE
PROGRAM TO OPERATE WITH ANY OTHER PROGRAMS), EVEN IF SUCH HOLDER OR OTHER PARTY
HAS BEEN ADVISED OF THE POSSIBILITY OF SUCH DAMAGES.

%%%%%%%%%%%%
% Telekom osCompendium 'for being included' snippet template
%
% (c) Karsten Reincke, Deutsche Telekom AG, Darmstadt 2011
%
% This LaTeX-File is licensed under the Creative Commons Attribution-ShareAlike
% 3.0 Germany License (http://creativecommons.org/licenses/by-sa/3.0/de/): Feel
% free 'to share (to copy, distribute and transmit)' or 'to remix (to adapt)'
% it, if you '... distribute the resulting work under the same or similar
% license to this one' and if you respect how 'you must attribute the work in
% the manner specified by the author ...':
%
% In an internet based reuse please link the reused parts to www.telekom.com and
% mention the original authors and Deutsche Telekom AG in a suitable manner. In
% a paper-like reuse please insert a short hint to www.telekom.com and to the
% original authors and Deutsche Telekom AG into your preface. For normal
% quotations please use the scientific standard to cite.
%
% [ Framework derived from 'mind your Scholar Research Framework' 
%   mycsrf (c) K. Reincke 2012 CC BY 3.0  http://mycsrf.fodina.de/ ]
%


%% use all entries of the bibliography
%\nocite{*}


\chapter{Introduction}

% Abstract
\footnotesize \begin{quote}\itshape This chapter shortly describes the idea of
the OSLiC, the way it should be used, and the way it can be read - what indeed
is not completely the same.
\end{quote}
\normalsize{}

% Content
This book focuses on only one issue: \emph{What do we have to do for acting
according to the licenses of those \emph{Open Source Software} we use?} The
\emph{Open Source License Compendium} wants to answer this question reliably -
in a simply to use and easily to understand manner. Thus, it is not another book
on \emph{Open Source} in ge\-ne\-ral\footnote{Meanwhile, there are tons of
literature dealing with Open Source. By improving your knowledge on the base of
such books and articles you might get lost in literature: our list of secondary
literature may adumbrate this 'danger of being overwhelmed'. But nevertheless,
our bibliography at the end of the OSLiC is not complete. Moreover, it's not
intended to be complete. It's only an extract which represents the background
knowledge we did not directly quoted in the OSLiC. If we were pressured to
indicate two books for getting a good survey on the topic \emph{Open Source
(Licenses)} we would name (a) the 'Rebel Code' (\cite[for a German version
cf.][\nopage passim]{Moody2001a} - \cite[for an English version
cf.][passim]{Moody2002a}) and (b) the 'legal basic conditions'
(\cite[cf.][\nopage passim]{JaeMet2011a}). But fortunately, we are not pressured
to do so.}. It shall be no more than a reliable tool to simplify the license
compliant behavior.

This compendium was evoked by a challenge of \emph{Deutsche Telekom AG} and some
of its software developers and project managers: Naturally, they want to behave
license compliantly. But they could not find a reference text which simply lists
what they concretely had to do to fulfill the licenses of that Open Source, they
currently used.

As some of these co-workers in Telekom projects, even we - the initial authors
of the OSLiC - did not want to become Open Source license experts only for being
able to use Open Source Software in a regular manner. We did not want to become
lawyers. We simply wanted to do in a simplier way, what in those days claimed
much time and many resources. We were searching for a clear guidance instead of
having to determine a correct way through the jungle of Open Source Licenses -
over and over again, project for project. We loved to use the high-quality Open
Source Software to improve our performance. We liked to use it legally. But we
did not like to laboriously discuss the juridical constraints of the many and
different Open Source Licenses.

What we needed, was a simply to use handout which would lead us without any
detours to executable lists of working items. We wished to get To-Do-Lists,
dedicated to our usecases and our licenses. We needed reliable lists of tasks we
only had to execute for being sure that we were acting according to the Open
Source License. When we started, such a compendium did not exist.

For solving this problem our company took three decisions:

The first decision our company came to, was to support a small group of
employees to act as \emph{a board of Open Source License experts}: They should
offer a service for the whole company. Projects, managers, and developers should
be enabled to ask this board what they have to do for fulfilling a specific Open
Source License under specific circumstances. And this board should answer with
reliable to-do lists whose executions would assure that the requestors are
acting according to the corresponding Open Source Licenses. The idea behind this
decision was simple. It would save costs and increases quality if one had a
central group of experts instead of being obliged to select (and to train)
developers - over and over again, for every new project. So, the \emph{OSRB} -
the \emph{Telekom Open Source Review Board} - was founded as an internal expert
group - as a self-organizing, bottom up driven community.

The second decision our company took, was to allow this \emph{Telekom OSRB} to
collect their results systematically . The idea behind this decision was also
simple: The more the internal service would become known, the more the workload
would increase: the more work, the more recources, the more costs. So, it was a
cost saving idea, to enable the requestors to find their answer by themselves -
but simply without becoming licenses experts: In the default cases, they should
find their answers in a compendium instead of making it necessary, to let their
work be analyzed by the OSRB. The planned \emph{Telekom Open Source License
Compendium} should preserve the Telekom to increase the number of OSRB members in
the future.

The third decision our company reached, was to allow the \emph{Telekom OSRB} to
elaborate the compendium in that mode of cooperation, Open Source projects used
to do. Again, a simple reason evoked this ruling: If in the future - as a rule -
not the reviewing OSRB, but a only manual should assure the Open Source License
compliant behavior of projects, programmers and managers, this book had of
course to be particularily reliable. There is a known feature of the Open Source
working model: the ongoing review by the cooperating community increases the
quality. Therefore, the decision, not only to write an internal 'Telekom
handout', but to enable the whole community to use, to modify and to
redistribute a broader \emph{Open Source License Compendium}, was a decision for
improving the quality. Thus, the \emph{OSRB} published the \emph{OSLiC} as a set
von LaTex Sources. It decided to make them accessible via the open repository
github\footnote{Get the code by using the link
\texttt{https://github.com/dtag-dbu/oslic}; get project information by
\texttt{http://dtag-dbu.github.com/oslic/} or by
\texttt{http://www.oslic.org/}.}. And it licensed the OSLiC under Creative
Commons Attribution-ShareAlike 3.0 Germany License\footnote{ This text is
licensed under the Creative Commons Attribution-ShareAlike 3.0 Germany License
(\texttt{http://creativecommons.org/licenses/by-sa/3.0/de/}): Feel free
\enquote{to share (to copy, distribute and transmit)} or \enquote{to remix (to
adapt)} it, if you \enquote{[\ldots] distribute the resulting work under the
same or similar license to this one} and if you respect how \enquote{you must
attribute the work in the manner specified by the author(s) [\ldots]}):
In an internet based reuse please mention the initial authors in a suitable
manner, name their sponsor \textit{Deutsche Telekom AG} and link it to
\texttt{http://www.telekom.com}. In a paper-like reuse please insert a short
hint to \texttt{http://www.telekom.com}, to the initial authors, and to their
sponsor \textit{Deutsche Telekom AG} into your preface. For normal quotations
please use the scientific standard to cite.}.

But to publish the \emph{OSLiC} as a free book has another important meaning -
at least for the \emph{Telekom OSRB}: It is also intended to be a thankful
\emph{Giving Back} to the \emph{Open Source Community} which has enriched and
simplified the life of so many employees and companies over so many years.

Howsoever, overall, the OSLiC follows five principles:

\begin{description}
  \item[To-do lists as the core, discussions around them]: Based on simply to
  gather information concerning the concrete use of a piece of Open Source
  Software and its license, the OSLiC shall offer a simply processable finder
  which leads the requestor to the relevant, license compliance assuring to-do
  list. Additionally, all these elements of the OSLiC should comprehensibly
  be introduced and discussed without disturbing the usage itself.

  \item[Quotations with thoroughly specified sources]\label{QuotationPrinciple}:
  The OSLiC shall be revisable and reliable. It shall comprehensibly argue and
  explictly specify why it overtakes which information from whom in which
  version and why\footnote{For that purpose, we are using an 'old-fashioned'
  bibliographic style presented footnotes, instead of endnotes or inline-hints.
  We want to enable the users to review or to ignore our comments and hints just
  as they prefer to do - but on all accounts without being disturbed by large
  inline comments or frequent page turnings. We know that modern writer guides
  prefer less 'noisy' styles (\cite[pars pro toto cf.][\nopage
  passim]{Mla2009a}). But for a reliable usage - challenged by the often
  modified internet sources - these methods are still a little imprecize (for
  details $\rightarrow$ \enquote{sec:QuotationAppendix}, pp.
  \pageref{sec:QuotationAppendix} . For a short motivation of the style used in
  the OSLiC \cite[cf.][\nopage passim]{Reincke2012a}. For a more elaborated
  legitimizing version \cite[cf.][\nopage passim]{Reincke2012b}) }.

 \item[Not clearing out the forest land, but cutting out a swathe]: The OSLiC
  has to deal with licenses and their legal aspects, no doubt. But it shall not
  discuss all details of all aspects. It shall focus on one possible way to act
  according a license in a specific usecase - even it is known that there might
  be alternatives\footnote{The OSLiC shall not counsel projects with respect to
  their specific needs. This must remain the tasks lawyers and legal experts.
  They can answer the question, whether a project under its' specfic conditions
  could also / better use another way to fulfill the Open Source license.}.
  
  \item[Take the license text seriously]: The OSLiC shall not give general
  lectures on legal discussions. Much less, it shall participate in them. It
  shall only find one reliable way for each license and each usecase to fulfill
  the license. The main source for this analyze shall be the exact reading of
  the Open Source Licenses themselves - based and supported by the
  interpretation of benevolent lawyers and rational argueing software
  developers. The OSLiC shall respect that Open Source Licenses are written for
  software developers (and sometimes by developers).
  
  \item[Trust the swarm]: The OSLiC shall be open for improvements and
  adjustments encouraged an stimulated also by other people than employees of
  \emph{Deutsche Telekom AG}.
\end{description}

And based on these principles the OSLiC offers two modes to be used:

First, and mostly the readers wish to find those to-do list simply and quickly
which fit their needs. Here is the corresponding process\footnote{For the well
known 'quick and dirty hackers' - as we tend to be, too - we have integrated a
shortcut: If you are already knowing the license of that Open Source package you
want to use and if you are very familiar with the meaning of the Open Source Use
Cases we have defined, then you might directly jump to the corresponding license
specific chapter, without 'struggling' with \textbf{OSLiC 5 query form}
($\rightarrow$ p. \pageref{OSLiCStandardFormForGatheringInformation}), the
taxonomic \textbf{Open Source Use Case Finder} ($\rightarrow$
\pageref{OSLiCUseCaseFinder}) or the \textbf{O}pen \textbf{S}ource \textbf{U}se
\textbf{C}ase page ($\rightarrow$ \pageref{OSUCList}ff.): Each of the Open
Source License specficic chapters starts with a license specific finder offering
a set of license specific use cases - which according to the complexity of the
license - in some cases could be stripped down. But the disdadvantage of this
method is that you have to apply your knowledge about the use cases and their
side effects by your own, without the systematic control of the full OSLiC
process as it is presented here.
}:

\tikzstyle{decision} = [diamond, draw, fill=gray!20, 
    text width=4.5em, text badly centered, node distance=4cm, inner sep=0pt]

\tikzstyle{preparation} = [rectangle, draw, fill=gray!30, 
    text width=10em, text centered, rounded corners, minimum height=4em]
 
\tikzstyle{lprocs} = [rectangle, draw, fill=gray!40, 
    text width=10em, text centered, rounded corners, minimum height=4em]
    
\tikzstyle{processing} = [rectangle, draw, fill=gray!40, node distance=2.4cm,
    text width=15em, text centered, rounded corners, minimum height=4em]
    
\tikzstyle{line} = [draw, -latex']

\tikzstyle{cloud} = [draw, ellipse, text width=6em, text centered, fill=gray!10]
 
    
\begin{tikzpicture}[node distance =1.9cm, auto]
\footnotesize
    % Place nodes
    
  \node [cloud] (start) at (1,10) 
    {$\forall$ your \\ Open Source \\ components};
  \node [preparation] (select) at (5,10) 
    {select your next Open Source component};     
  \node [preparation,  below of=select] (analyze) 
    {analyze its' role in the software architecture and 
    determine the use of the app as whole};  
  \node [preparation,  below of=analyze] (determine) 
    {determine its Open Source License};
  \node [lprocs,  below of=determine] (fillin)
    {\textbf{fill in the 5 query form} ($\rightarrow$ p.
    \pageref{OSLiCStandardFormForGatheringInformation})};
    
  \node [decision, right of=fillin] (success) {success?};
  
  \node [processing,  below of=success] (traverse)
    {\textbf{traverse the} taxonomic \textbf{Open Source Use Case Finder}
    ($\rightarrow$ \pageref{OSLiCUseCaseFinder}) and jump to the indicated
    \textbf{O}pen \textbf{S}ource \textbf{U}se \textbf{C}ase page ($\rightarrow$
    \pageref{OSUCList}ff.)};
    
  \node [processing,  below of=traverse] (find)
    {\textbf{Determine} the page of \textbf{the license and use case specific to-do list}
    being presenetd in the license specific chapter};
 
  \node [processing,  below of=find] (process)
    {Jump to the indicated page and \textbf{process the license and use case specific
    to-do list} ($\rightarrow$ \pageref{OSUCToDoLists}ff.)};
    
  \node [decision, right of=process] (other) {more?};
  \node [cloud, below of=other] (stop) {stop};

  \path [line] (start) -- (select);  
     
  \path [line] (select) -- (analyze);      
  \path [line] (analyze) -- (determine);         
  \path [line] (determine) -- (fillin);
  \path [line] (fillin) -- (success);
  
  \path [line] (success) |- node [near start] {no} (analyze);
  \path [line] (success) -- node [near start] {yes} (traverse);             
  
  \path [line] (traverse) -- (find);              
  \path [line] (find) -- (process);
  \path [line] (process) -- (other);

  \path [line] (other) |- node [near start] {yes} (select);
  \path [line] (other) -- node [near start] {no} (stop);                      

\end{tikzpicture}

Second, the readers might wish to comprehend the whole analysis. So, we
shortly discuss Open Source Licenses taxonomies as base of license compliant
behavior\footnote{$\rightarrow$ \enquote{\nameref{sec:LicenseTaxonomies}}, pp.
\pageref{sec:LicenseTaxonomies}}. We consider some side effects with
relation to act according to the Open Source Licenses\footnote{$\rightarrow$
\enquote{\nameref{sec:SideEffects}}, pp. \pageref{sec:SideEffects}}. And we
study the structure of Open Source Use Cases\footnote{$\rightarrow$
\enquote{\nameref{sec:OSUCdeduction}}, pp. \pageref{sec:OSUCdeduction}}.

So, let's close our introduction by using, modifying, and (re)distributing a well
known wish of a well known man: Happy (Legally) Hacking.

%\bibliography{../../../bibfiles/oscResourcesEn}

%% Telekom osCompendium 'for beeing included' snippet template
%
% (c) Karsten Reincke, Deutsche Telekom AG, Darmstadt 2011
%
% This LaTeX-File is licensed under the Creative Commons Attribution-ShareAlike
% 3.0 Germany License (http://creativecommons.org/licenses/by-sa/3.0/de/): Feel
% free 'to share (to copy, distribute and transmit)' or 'to remix (to adapt)'
% it, if you '... distribute the resulting work under the same or similar
% license to this one' and if you respect how 'you must attribute the work in
% the manner specified by the author ...':
%
% In an internet based reuse please link the reused parts to www.telekom.com and
% mention the original authors and Deutsche Telekom AG in a suitable manner. In
% a paper-like reuse please insert a short hint to www.telekom.com and to the
% original authors and Deutsche Telekom AG into your preface. For normal
% quotations please use the scientific standard to cite.
%
% [ File structure derived from 'mind your Scholar Research Framework' 
%   mycsrf (c) K. Reincke CC BY 3.0  http://mycsrf.fodina.de/ ]

%
\newpage
\section{Form [only to demo our lib style. will be replaced]}
\begin{itemize}
  \item first initially quoted book\footnote{\cite[cf.][123ff]{Grassmuck2002a}
  (expected: complete bibl. data)} using LaTeX \texttt{$\backslash$footnote}
  \item second initially quoted book\footcite[cf.][120 (expected: complete bibl.
  data)]{Fogel2006a} using jurabib \texttt{$\backslash$footcite} (same
  appereance)
  \item initially mentioned collection /
  proceedings\footnote{\cite[cf.][123ff]{DjoGehGraKreSpi2008a} (expected: complete
  bibl. data)}
  \item first initially mentioned article in an initially mentioned collection /
  proceedings\footnote{\cite[cf.][123ff]{Spielkamp2008a} (expected: complete
  bibl. data of article, short title data of collection)} using LaTeX
  \texttt{$\backslash$footnote}
  \item second initially mentioned article in an already mentioned collection
  / proceedings\footcite[cf.][123ff (expected: complete
  bibl. data of article, short title data of collection)]{Kreutzer2008a} using
  jurabib \texttt{$\backslash$footcite} 
  \item rementioned book\footnote{\cite[cf.][120]{Fogel2006a} (expected: short
  title)}
  \item directly rementioned same book same
  page\footnote{\cite[cf.][120]{Fogel2006a} (expected: id., ibid, / ders.,
  ebda.,)}
  \item directly rementioned same book different
  page\footnote{\cite[cf.][121]{Fogel2006a} (expected: id., lc., / ders.,
  a.a.O. \& page)}
  \item rementioned collection article\footnote{\cite[cf.][120 ]{Kreutzer2008a} (expected: short
  title)}
  \item directly rementioned collection article same
  page\footnote{\cite[cf.][120]{Kreutzer2008a} (expected: id., ibid, / ders.,
  ebda.,)}
  \item directly rementioned collection article different
  page\footnote{\cite[cf.][121]{Kreutzer2008a} (expected: id., lc., / ders.,
  a.a.O. \& page)}
\end{itemize}

%\bibliography{../bibfiles/oscResourcesEn}


%%%%%%%%%%%%%%%
% Telekom osCompendium 'for being included' snippet template
%
% (c) Karsten Reincke, Deutsche Telekom AG, Darmstadt 2011
%
% This LaTeX-File is licensed under the Creative Commons Attribution-ShareAlike
% 3.0 Germany License (http://creativecommons.org/licenses/by-sa/3.0/de/): Feel
% free 'to share (to copy, distribute and transmit)' or 'to remix (to adapt)'
% it, if you '... distribute the resulting work under the same or similar
% license to this one' and if you respect how 'you must attribute the work in
% the manner specified by the author ...':
%
% In an internet based reuse please link the reused parts to www.telekom.com and
% mention the original authors and Deutsche Telekom AG in a suitable manner. In
% a paper-like reuse please insert a short hint to www.telekom.com and to the
% original authors and Deutsche Telekom AG into your preface. For normal
% quotations please use the scientific standard to cite.
%
% [ Framework derived from 'mind your Scholar Research Framework' 
%   mycsrf (c) K. Reincke 2012 CC BY 3.0  http://mycsrf.fodina.de/ ]
%

\chapter{Open Source: The Same Idea, Different Licenses}\label{sec:LicenseTaxonomies}

%% use all entries of the bibliography
%\nocite{*}
\footnotesize \begin{quote}\itshape This chapter describes different license
models which follow the common idea of free open source software. We want to
discuss existing ways of grouping licenses to underline the limits of building
such clusters: These groups are often used as 'virtual prototypic licenses'
which shall deliver a simplified view at the conditions how to act according to
the respective real license instances. But one has to meet the requirements of a
specific license, not one's own generalized idea of a set of licenses.
Nonetheless, also we offer a new structuring view into the world of the open
source licenses. We will use a novel set of grouping criteria by referring
to the common intended purpose of licenses: each license is designed to protect
something or someone against something or someone. Following this pattern, we
can indeed summarize all Open Source Licenses in a comparable way.
\end{quote}
\normalsize{}

Grouping open source licenses is commonly done. Even the set of the \emph{open
source li\-cen\-ses}\footcite[cf.][\nopage wp]{OSI2012b} itself is already a
cluster being established by a set of grouping criteria: The
\enquote{distribution terms} of each software license that intends to become an
open source license, \enquote{[\ldots] must comply with the [\ldots] criteria}
of the \emph{Open Source De\-fi\-ni\-tion}\footcite[cf.][\nopage wp]{OSI2012a},
maintained by the \emph{Open Source Initiative}\footcite[cf.][\nopage
wp]{OSI2012c} and often abbreviated as \emph{OSD}. So, this \emph{OSD}
demarcates 'the group of [potential] open source licenses' against 'the group of
not open sources licenses'\footnote{More precisely: meeting the
OSD is only a necessary condition for becoming an \emph{open source license}. The
sufficient condition for becoming an \emph{open source license}, is the approval
by the OSI which offers a process for the officially approval of \emph{open
source license} (\cite[cf.][\nopage wp]{OSI2012d}).}.

Another way to cluster the \emph{Free Software Licenses} is specified by the
\enquote{Free Software Definition}. This \emph{FSD} contains four conditions
which must be met by any free software license: any FSD compliant license must
grant \enquote{the freedom to run a program, for any purpose [\ldots]},
\enquote{the freedom to study how it works, and adapt it to (one's) needs
[\ldots]}, \enquote{the freedom to redistribute copies [\ldots]}, and finally
\enquote{the freedom to improve the program, and release your improvements
[\ldots]}\footcite[cf.][41]{Stallman1996a}. Surprisingly this definition
implies that the requirement \emph{the sourcecode must be openly accessible},
is 'only' a derived condition. If the \enquote{freedom to make changes and the
freedom to publish improved versions} shall be \enquote{meaningful}, then the
\enquote{access to the source code of the program} is a prerequisite.
\enquote{Therefore, accessibility of source code is a necessary condition for
free software.}\footcite[cf.][41]{Stallman1996a}

The difference between the OSD and the FSD has often been described as a
difference of emphasis\footnote{This is also the viewpoint of Richard M.
Stallman: On the one hand, he clearly states that the \enquote{Free Software
movement} and the \enquote{open source movement} generally \enquote{[\ldots]
disagree on the basic principles, but agree more or less on the practical
recommendations} and that he \enquote{[\ldots] (does) not think of the open
source movement as an enemy}.  On the other hand, he delineates the two
movements by stating that \enquote{for the open source movement, the issue of
whether software should be open source is a practical question, not an ethical
one}, while \enquote{for the Free Software movement, non-free software is a
social problem and free software is the solution}
(\cite[cf.][55]{Stallman1998a}). \label{RmsFsPriority} Consequently, Richard
M. Stallman summarizes the positions in a simple way: \enquote{[\ldots] 'open
source' was designed not to raise [\ldots] the point that users deserve
freedom}. But he and his friends want \enquote{to spread the idea of freedom}
and therefore \enquote{[\ldots] stick to the term 'free software'}
(\cite[][59]{Stallman1998a}). For a brush-up of this position, expressing
again that \enquote{(o)pen source is a development methodolgy [and that] free
software is a social movement} with an \enquote{ethical imparative}
\cite[cf.][31]{Stallman2009a} }: Although both definitions \enquote{[\ldots]
(cover) almost exactly the same range of software}, the \emph{Free Software
Foundation} -- as it is said -- \enquote{prefers [\ldots] (to emphazise) the
idea of freedom [\ldots]} while the \emph{OSI} wants to underline the
philosophically indifferent \enquote{development methodolgy}\footcite[pars pro
toto: cf.][232]{Fogel2006a}.

A third method to group of free software and free software licenses is specified
by the \enquote{Debian Free Software Guideline} which is embedded into the
\enquote{Debian Social Contract}. This \enquote{DFSG} contains nine defining
criteria which -- as Debian itself says -- have been \enquote{[\ldots] adopted
by the free[sic!] software community as the basis of the Open Source
Definition}\footcite[cf.][wp]{DFSG2013a}.

A rough understanding of these methods might result in the conclusion that these
three definitions are extensionally equal and only differ intensionally.
But that is not true. To unveil the differences, let us compare the clusters
\emph{OSI approved licenses}, \emph{OSD compliant licenses}, \emph{DFSG
compliant licenses}, and \emph{FSD compliant licenses} extensionally, by asking
whether they \emph{could} establish different sets of licenses\footnote{Indeed,
for analyzing the extensional power of the definition we have to regard all
potentially covered licenses, not only the already existing licenses, because
the subset of really existing licenses still could be expanded be developing new
licenses which fit the definition.}.

First, the difference most easy to determine is that of an unidirectional
inclusion: By definition, the \emph{OSI approved licenses} and the \emph{OSD
compliant licenses} meet the requirements of the OSD\footcite[cf.][\nopage
wp]{OSI2012a}. But only the \emph{OSI approved licenses} have successfully
passed the OSI process\footcite[cf.][\nopage wp]{OSI2012a} and therefore are
officially listed as \emph{open source licenses}\footcite[cf.][\nopage
wp]{OSI2012b}. Hence, on the one hand, \emph{OSI approved licenses} are
\emph{open source licenses} and vice versa. On the other hand, both -- the
\emph{OSI approved licenses} and the \emph{open source licenses} -- are
\emph{OSD compliant licenses}, but not vice versa.

Second, a similar argumentation allows to distinguish the \emph{DFSG compliant
licenses} from the \emph{OSI approved licenses}. As it is stated, the OSD
\enquote{[\ldots] is based on the Debian Free Software Guideline and any
license that meets one definition almost meets the
other}\footcite[cf.][233]{Fogel2006a}. But then again, meeting the definition is
not enough for being an official open source license: the license has to be
approved by the OSI\footcite[cf.][\nopage wp]{OSI2012b}. Thus, it follows that
all \emph{OSI approved licenses} are also \emph{DFSG compliant licenses}, but
not vice versa.

Third, -- by ignoring the \enquote{few exceptions} which have appeared
\enquote{over the years}\footcite[cf.][233]{Fogel2006a} -- it can be said that,
because of their 'kinsmanlike' relation, at least the \emph{OSD compliant
licenses} are also \emph{DFSG compliant licenses} and vice versa.

Last but not least, it must be stated that the (potential) set of free software
licenses must be greater than all the other three sets: On the one side, the FSD
requires that a license of free software must not only allow to read the
software, but must also permit to use, to modify, and to distribute
it\footcite[cf.][41]{Stallman1996a}. These conditions are covered by at least
the first three paragraphs of the OSD concerning the topics \enquote{Free
Redistribution}, \enquote{Source Code}, and \enquote{Derived
Works}\footcite[cf.][\nopage wp]{OSI2012a}. On the other side, the OSD contains
at least some requirements which are not mentioned by the FSD and which
nevertheless must be met by a license in order to be qualified as an OSD
compliant license\footnote{For example, see the condition that \enquote{the
license must be technology-neutral} (\cite[cf.][\nopage wp]{OSI2012a}).}. It
follows then that there may exist licenses which fulfill all conditions of the
FSD and nevertheless do not fulfill at least some conditions of the
OSD\footnote{Again: we must consider the extensional potential of the
definitions, not the set of really existing licenses. In this context, it is
irrelevant that actually all existing Free Software Licenses like GPL, LGPL or
AGPL indeed are also classfied as open source licenses. We are referring to the
fact that there might be generated licenses which fulfill the FSD, but not the
OSD.}. So, the set of all (potential) \emph{Free Software Licenses} must be
greater than the set of all (potential) \emph{open source licenses} and greater
than the set of \emph{OSD compliant licenses}.

All in all, we can visualize the situation as follows:

\begin{center}

\begin{tikzpicture}
\label{LICTAX}
\small

\node[ellipse,minimum height=5.8cm,minimum width=11.6cm,draw,fill=gray!10] (l0210) at (5,5)
{ };

\draw [-,dotted,line width=0pt,white,
    decoration={text along path,
              text align={center},
              text={|\itshape|All Software Licenses}},
              postaction={decorate}] (0,6.1) arc (120:60:10cm);

\node[ellipse,minimum height=4.4cm,minimum width=10cm,draw,fill=gray!20] (l0210) at (5,5)
{ };

\draw [-,dotted,line width=0pt,white,
    decoration={text along path,
              text align={center},
              text={|\itshape|FSD Compliant Licenses}},
              postaction={decorate}] (0,5.4) arc (120:60:10cm);
              

\node[ellipse,minimum height=3cm,minimum width=8.4cm,draw,fill=gray!30] (l0210) at (5,5)
{ };


             
\draw [-,dotted,line width=0pt,white,
    decoration={text along path,
              text align={center},
              text={|\itshape|OSD Compliant Licenses}},
              postaction={decorate}] (0,4.7) arc (120:60:10cm);
              
\draw [-,dotted,line width=0pt,white,
    decoration={text along path,
              text align={center},
              text={|\itshape|DFSG Compliant Licenses}},
              postaction={decorate}] (0,5) arc (240:300:10cm);
          

\node[ellipse,text width=4.4cm, text centered,minimum height=1.6cm,minimum width=6cm,draw,fill=gray!40] (l0210) at (5,5)
{ \textit{OSI approved licenses} = \\ \textit{\textbf{open source licenses}}
};

\end{tikzpicture}
\end{center}

It should be clear without longer explanations that these clusters don't allow
to extrapolare to the correct compliant behaviour according to the \emph{open source
licenses}: On the one hand, all larger clusters do not talk about the \emph{open
source licenses}. On the other hand, the \emph{open source license cluster}
itself only collects its elements on the base of the OSD which does not stipulates
concrete license fulfilling actions for the licensee.

The next level of clustering \emph{open source licenses} concerns the inner
structure of these \emph{OSI approved licenses}. Even the OSI itself has recently
discussed whether a better kind of grouping the listed licenses would better fit
the needs of the visitors of the OSI site\footcite[cf.][\nopage wp]{OSI2013a}.
And finally the OSI ends up in the categories \enquote{popular and widely used
(licenses) or with strong communities}, \enquote{special purpose licenses},
\enquote{other/miscellaneous licenses}, \enquote{licenses that are redundant
with more popular licenses}, \enquote{non-reusable licenses},\enquote{superseded
licenses}, \enquote{licenses that have been voluntarily retired}, and \enquote{
uncategorized licenses}\footcite[cf.][\nopage wp]{OSI2013b}.

Another way to structure the field of open source licenses is to think in
\enquote{types of open source licenses} by grouping the \enquote{\emph{academic
licenses}, named as such because they were originally created by academic
institutions}\footcite[cf.][69]{Rosen2005a}, the \enquote{\emph{reciprocal
licenses}}, named as such because they \enquote{[\ldots] require the
distributors of derivative works to dis\-tri\-bu\-te those works under same
license including the requirement that the source code of those derivative works
be published}\footcite[cf.][70]{Rosen2005a}, the \enquote{\emph{standard
licenses}}, named as such because they refer to the reusability of
\enquote{industry standards}\footcite[cf.][70]{Rosen2005a}, and the
\enquote{\emph{content licenses}}, named as such because they refer to
\enquote{[\ldots] other than software, such as music art, film, literary works}
and so on\footcite[cf.][71]{Rosen2005a}.

Both kinds of taxonomy directly help to find the relevant licenses which should
be used for new (software) projects. But again: none of these categories 
allow to infer the license compliant behaviour, because the categories are
mostly defined based on license external criteria: whether a license is
published by a specific kind of organization or whether a license deals with
industry standards or other kind of works than software, inherently do not
evoke a license fulfilling behaviour.

Only the act of grouping into the \enquote{\emph{academic licenses}} and the
\enquote{\emph{reciprocal licenses}} touches the idea of license fulfilling
doings, if one -- as it has been done -- expands the definition of the
\enquote{\emph{academic licenses}} by the specification that these licenses
\enquote{[\ldots] allow the software to be used for any purpose whatsoever with
no obligation on the part of the licensee to distribute the source code of
derivative works}\footcite[cf.][71]{Rosen2005a}. With respect to this additional
specification, the clusters \enquote{\emph{academic licenses}} and the
\enquote{\emph{reciprocal licenses}} indeed might be referred as the
\enquote{main categories} of (open source)
licenses\footcite[cf.][179]{Rosen2005a}: By definition, they are constituting
not only a contrary, but contradictory opposite. However, it must  be kept in
mind that they constitute an inherent antagonism, an antinomy inside of the set
of open source licenses\footnote{Hence, it is at least a little confusing to say
that \enquote{the open source license (OSL) is a reciprocal license} and
\enquote{the Academic Free License (AFL) is the exact same license without the
reciprocity provisions} (\cite[cf.][180]{Rosen2005a}): If the BSD license is an
AFL and if an AFL is not an OSL and if the OSI approves only OSLs, then the BSD
license can not be an approved open source license. But in fact, it still is
(\cite[cf.][\nopage wp]{OSI2012b}).}.

Connatural to the clustering into \emph{academic licenses} and \emph{reciprocal
licenses} is the grouping into \emph{permissive licenses}, \emph{weak copyleft
licenses}, and \emph{strong copyleft licenses}: Even Wikipedia already uses the
term \enquote{permissive free software licence} in the meaning of \enquote{a
class of free software licence[s] with minimal requirements about how the
software can be redistributed} and \enquote{contrasts} them with the
\enquote{copyleft licences} as those \enquote{with reciprocity / share-alike
requirements}\footcite[cf.][\nopage wp]{wpPermLic2013a}. 

Some other authors name the set of \emph{academic licenses} the
\enquote{permissive licenses} and specify the \emph{reciprocal licenses} as
\enquote{restrictive licenses}, because in this case -- as a consequence of the
embedded \enquote{copyleft} effect -- the source code must be published in case
of modifications. They additionally introduce the subset of \enquote{strong
restrictive licenses} which additionally require that an (overarching)
derivative work must be published under the same license\footcite[pars pro toto
cf.][57]{Buchtala2007a}. The next refinement of such clustering concepts
directly uses the categories \enquote{[open source] licenses with a strict
copyleft clause}\footcite[Originally stated as \enquote{Lizenzen mit einer
strengen Copyleft-Klausel}. Cf.][24]{JaeMet2011a}, \enquote{[open source]
licenses with a restricted copyleft clause}\footcite[Originally stated as
\enquote{Lizenzen mit einer beschränkten Copyleft-Klausel}.
Cf.][71]{JaeMet2011a}, and \enquote{[open source] licenses without any copyleft
clause}\footcite[Originally stated as \enquote{Lizenzen ohne Copyleft-Klausel}.
Cf.][83]{JaeMet2011a}. Finally, this viewpoint can directly be mapped to the
categories \emph{strong copyleft} and \emph{weak copyleft}: While on the one
hand, \enquote{only changes to the weak-copylefted software itself become
subject to the copyleft provisions of such a license, [and] not changes to the
software that links to it}, on the other hand, the \enquote{strong copyleft}
states \enquote{[\ldots] that the copyleft provisions can be efficiently imposed
on all kinds of derived works}\footcite[cf.][\nopage wp]{wpCopyleft2013a}.

Based on this approach to an adequate clustering and labeling\footnote{Finally,
we should also mention that there still exists other classifications which might
become important in other contexts. For example, the ifross license subsumes
under the main category \enquote{Open Source Licenses} the subcategories
\enquote{Licenses without Copyleft Effect}, \enquote{Licenses with Strong
Copyleft}, \enquote{Licenses with Restricted Copyleft}, \enquote{Licenses with
Restricted Choice}, or \enquote{Licenses with Privilegs} -- and let finally
denote these categories also licenses which are not listed by the OSI
(\cite[cf.][\nopage wp]{ifross2011a}). This is well reasonable if one refers to
the meaning of the OSD (\cite[cf.][\nopage wp]{OSI2012a}). The OSLiC wants to
simplify its object of study by referring to the approved open source licenses
(\cite[cf.][\nopage wp]{OSI2012d}) listed by the OSI (\cite[cf.][\nopage
wp]{OSI2012b}).}, we can develop the following picture:

\begin{center}

\begin{tikzpicture}
\label{OSLICTAX}
\small



\node[ellipse,minimum height=8.5cm,minimum width=14cm,draw,fill=gray!10] (l0100) at (6.8,6.8)
{  };

\draw [-,dotted,line width=0pt,white,
    decoration={text along path,
              text align={center},
              text={|\itshape| OSI approved licenses}},
              postaction={decorate}] (-0.8,6.5) arc (142:38:9.5cm);

\draw [-,dotted,line width=0pt,white,
    decoration={text along path,
              text align={center},
              text={|\itshape|open source licenses}},
              postaction={decorate}] (-0.8,6.5) arc (218:322:9.5cm);
              
\node[ellipse,minimum height=6.2cm,minimum width=4cm,draw,fill=gray!20] (l0100) at (2.75,6.8)
{  };

\draw [-,dotted,line width=0pt,white,
    decoration={text along path,
              text align={center},
              text={|\itshape| permissive licenses}},
              postaction={decorate}] (0.9,7.4) arc (180:0:1.8cm);

\node[circle,draw,text width=1cm, fill=gray!40, text centered] (l0101) at (2,8)
{  \footnotesize \bfseries \textit{ApL}};
\node[circle,draw,text width=1cm, fill=gray!40, text centered] (l0102) at (3.5,8)
{  \footnotesize \bfseries \textit{BSD}};
\node[circle,draw,text width=1cm, fill=gray!40, text centered] (l0103) at (2,6.5)
{  \footnotesize \bfseries \textit{MIT}};
\node[circle,draw,text width=1cm, fill=gray!40, text centered] (l0104) at (3.5,6.5)
{  \scriptsize \bfseries \textit{MS-PL}};
\node[circle,draw,text width=1cm, fill=gray!40, text centered] (l0105) at (2,5)
{  \footnotesize \bfseries \textit{PgL}};
\node[circle,draw,text width=1cm, fill=gray!40, text centered] (l0106) at (3.5,5)
{  \footnotesize \bfseries \textit{PHP}};

\node[ellipse,minimum height=6cm,minimum width=8.5cm,draw,fill=gray!20] (l0200) at (9.2,6.5)
{  };

\draw [-,dotted,line width=0pt,white,
    decoration={text along path,
              text align={center},
              text={|\itshape| copyleft licenses}},
              postaction={decorate}] (7.5,8.5) arc (120:60:4cm);


\node[ellipse,minimum height=4.5cm,minimum width=4.2cm,draw,fill=gray!30] (l0210) at (7.45,6.5)
{  };

\draw [-,dotted,line width=0pt,white,
    decoration={text along path,
              text align={center},
              text={|\itshape| weak copyleft licenses}},
              postaction={decorate}] (5.4,6.2) arc (180:0:2cm);

\node[circle,draw,text width=1cm, fill=gray!40, text centered] (l0211) at (6.7,7)
{  \footnotesize \bfseries \textit{EPL}};
\node[circle,draw,text width=1cm, fill=gray!40, text centered] (l0212) at (8.2,7)
{  \footnotesize \bfseries \textit{EUPL}};
\node[circle,draw,text width=1cm, fill=gray!40, text centered] (l0213) at (6.7,5.5)
{  \footnotesize \bfseries \textit{LGPL}};
\node[circle,draw,text width=1cm, fill=gray!40, text centered] (l0214) at (8.2,5.5)
{  \footnotesize \bfseries \textit{MPL}};

\node[ellipse,minimum height=4.5cm,minimum width=3cm,draw,fill=gray!30] (l0220) at (11.4,6.5)
{  };
 
% line width=0pt,white,
\draw [-,dotted,line width=0pt,white,
    decoration={text along path,
              text align={center},
              text={|\itshape| strong copyleft}},
              postaction={decorate}] (10.4,7) arc (180:0:1cm);

\draw [-,dotted,line width=0pt,white,
    decoration={text along path,
              text align={center},
              text={|\itshape| licenses}},
              postaction={decorate}] (10.4,5.4) arc (180:360:1cm);        

\node[circle,draw,text width=1cm, fill=gray!40, text centered] (l0221) at (11.4,7)
{  \footnotesize \bfseries \textit{GPL}};
\node[circle,draw,text width=1cm, fill=gray!40, text centered] (l0222) at (11.4,5.5)
{  \footnotesize \bfseries \textit{AGPL}};


\end{tikzpicture}
\end{center}

This extensionally based clarification of a possible open source license
taxonomy is probably well-known and often -- more or less explicitly --
referred to\footnote{Even the FSF itself uses the term 'permissive non-copyleft
free software license' (\cite[pars pro toto: cf.][\nopage wp/section 'Original BSD
license']{FsfLicenseList2013a}) and contrasts it with the terms 'weak copyleft'
and 'strong copyleft' (\cite[pars pro toto: cf.][\nopage wp/section 'European
Union Public License']{FsfLicenseList2013a})}. Unfortunately, this taxonomy
still contains some misleading underlying messages:

\emph{Permissive} has a very positive connotation. So, the antinomy of
\emph{permissive licenses} versus \emph{copyleft licenses} implicitly signals,
that the \emph{permissive licenses} are in any meaning better, than the
\emph{copyleft licenses}. Naturally, this 'conclusion' is evoked by
confusing the extensionally definition and the intensional power of the labels.
But that is the way we -- the human beings -- like to think. 

Anyway, this underlying message is not necessarily 'wrong'. It might be
convenient for those people or companies who only want to use open source
software without being restricted by the \emph{obligation to give something
back} as it has been introduced by the 'copyleft'\footnote{De facto,
\emph{copyleft} is not \emph{copyleft}. Apart from the definition, its effect
depends on the par\-ti\-cu\-ar licenses which determine the conditions for
applying the copyleft 'method'. For example, in the GPL, the copyleft effect is
bound to the criteria 'being distributed'. Later on, we will collect these
conditions systematically (see chapter \emph{\nameref{sec:OSUCdeduction}}, pp.\
\pageref{sec:OSUCdeduction}). Therefore, here we still permit ourselves to use a
somewhat 'generalizing' mode of speaking.}. But there might be other people and
companies who emphasize the protecting effect of the copyleft licenses. And
indeed, at least the open source license\footnote{Although RMS naturally prefers
to specify it as a \emph{Free Software License} (s. p.\ \pageref{RmsFsPriority})
} \emph{GPL}\footnote{As the original source \cite[cf.][\nopage
wp]{Gpl20FsfLicense1991a}. Inside of the OSLiC, we constantly refer to the
license versions which are published by the OSI, because we are dealing with
officially approved open source licenses. For the 'OSI-GPL' \cite[cf.][\nopage
wp]{Gpl20OsiLicense1991a}} has initially been developed to protect the freedom,
to enable the developers to help their \enquote{neighbours} and to get the
modifications back\footnote{The history of the GNU project is multiply told. For
the GNU project and its initiator \cite[cf.\ pars pro toto][\nopage
passim]{Williams2002a}. For a broader survey \cite[cf.\ pars pro toto][\nopage
passim]{Moody2001a}. A very short version is delivered by Richard M. Stallman
himself where he states that -- in the years while the early free community were
destroyed -- he saw the \enquote{nondisclosure agreement} which must be signed ,
\enquote{[\ldots] even to get an executable copy} as a clear \enquote{[\ldots]
promise not to help your neighbour}: \enquote{A cooperating community was
forbidden.} (\cite[cf.][16]{Stallman1999a}).}: So, \enquote{Copyleft} is defined
as a \enquote{[\ldots] method for making a program free software and requiring
all modified and extended versions of the program to be free software as
well}\footcite[cf.][89]{Stallman1996c}. It is a method\footnote{Based on the
American legal copyright system, this method uses two steps: firstly one states,
\enquote{[\ldots] that it is copyrighted [\ldots]} and secondly one adds those
\enquote{[\ldots] distribution terms, which are a legal instrument that gives
everyone the rights to use, modify, and redistribute the program's code or any
program derived from it but only if the distribution terms are unchanged}
(\cite[cf.][89]{Stallman1996c}).} by which \enquote{[\ldots] the code and the
freedoms become legally inseparable}\footcite[cf.][89]{Stallman1996c}. Because
of these disparate interests of hoping not to be restricted and hoping to be
protected, it could be helpful to find a better label -- an impartial name for
the cluster of \emph{permissive licenses}. But until that time, we should at
least know that this taxonomy still contains an underlying declassing message.

The other misleading interpretation is -- counter-intuitively -- evoked by using
the concept of 'copyleft licenses'. By referring to a cluster of \emph{copyleft
licenses} as the opposite of the \emph{permissive licenses}, one implicitly also
sends two messages: First, that republishing one's own modifications is
sufficient to comply with the \emph{copyleft licenses}. And, secondly, that the
\emph{permissive licenses} do not require anything to be done for obtaining the
right to use the software. Even if one does not wish to evoke such an
interpretation, we -- the human beings -- tend to take the things as simple as
possible\footnote{And indeed, in the experience of the authors -- sometimes --
such simplifications gain their independent existence and determine decisions on
the management level. But that is not the fault of the managers. It is their
job, to aggregate, generalize and simplify information. It is the job of the
experts, to offer better viewpoints without overwhelming the others with
details.}. But because of several aspects, this understanding of the antinomy of
\emph{copyleft licenses} and \emph{permissive licenses} is too misleading for
taking it as a serious generalization:

On the one hand, even the 'strongly copylefted' GPL requires also other tasks
than just the republishing of derivative works. For example, it also calls for
to \enquote{[\ldots] give any other recipients of the [GPL licensed] Program a
copy of this License along with the Program}\footcite[cf.][\nopage wp\
§1]{Gpl20OsiLicense1991a}. Furthermore, the 'weakly copylefted' licenses require
also more and different criteria which has to be fulfilled for acting according
to these licenses. For example, the EUPL requires that the licensor who does not
directly deliver the binaries together with the sourcecode, must offer a
sourcecode version of his work free of charge\footnote{The German version of the
EUPL uses the phrase \enquote{problemlos und unentgeltlich(sic!) auf den
Quellcode (zugreifen können)} (\cite[cf.][3, section 3]{EuplLicense2007de})
while the English version contains the specification \enquote{the Source Code is
easily and freely accessible} (\cite[cf.][2, section 3]{EuplLicense2007en})},
while the MPL requires that under the same circumstances a recipient
\enquote{[\ldots] can obtain a copy of such Source Code Form [\ldots] at a
charge no more than the cost of distribution to the recipient
[\ldots]}\footcite[cf.][\nopage section 3.2.a]{Mpl20OsiLicense2013a}.
And last but not least, also the \emph{permissive licenses} require tasks which
must be fulfilled for a license compliant usage -- moreover, they also require
different things. For example, the BSD demands that \enquote{the
(re)distributions [\ldots] must (retain [and/or]) reproduce the above copyright
notice [\ldots]}. Because of the structure of the \enquote{copyright notice},
this compulsory notice implies that the authors / copyright holders of the
software must be publicly named\footcite[cf.][\nopage wp]{BsdLicense2Clause}. As
opposed to this, the Apache License requires that \enquote{if the Work includes
a "NOTICE" text file as part of its distribution, then any Derivative Works that
You distribute must include a readable copy of the attribution notices contained
within such NOTICE file} which often means that you have to present central
parts of such file publicly\footcite[cf.][\nopage wp\ section
4.4]{Apl20OsiLicense2004a} -- parts which can contain much more information than
only the names of the authors / copyright holders.

So, no doubt -- and contrary to the intuitive interpretation of this taxonomy --
each \emph{open source license} must be fulfilled by some actions, even the most
permissive one. And for ascertaining these tasks, one has to look into these
licenses themselves, not the generalized concepts of licenses taxonomies. Hence
again, we have to state that even this well known type of grouping of \emph{open
source licenses} does not allow to derive a specific license compliant behavior:
The taxonomy might be appropriate, if one wants to live with the implicite
messages and generalizations of some of its concepts. But the taxonomy is not an
adequate tool to determine, what one has to do for fulfilling an \emph{open
source license}. A license compliant behaviour for obtaining the right to use a
specific piece of \emph{open source software} must be based on the concrete
\emph{open source license} by which the licensor has licensed the software.
There is no shortcut.

Nevertheless, human beings need generalizing and structuring viewpoints for
enabling themselves to talk about a domain -- even if they finally have to
regard the single objects of the domain for specific purposes. We think that
there is a subtler method to regard and to structure the domain of \emph{open
source licenses}. So, we want to offer this other possibility to cluster the
\emph{open source licenses}\footnote{even if also we have to concede that,
ultimately, one has to always look into the license itself}:

We think that, in general, licenses have a common purpose: they should protect
someone or something against something. The structure of this task is based on
the nature of the word 'protect' which is a trivalent verb: it links someone or
something who protects, to someone or something who is protected and both
combined to something against which the protector protects and against the other
one is protected. Licenses in general do that. Moreover, to \enquote{protect}
the \enquote{rights} of the licensees is explicitly mentioned in the
GPL-2.0\footcite[cf.][\nopage wp, Preamble]{Gpl20OsiLicense1991a}, in the
LGPL-2.1\footcite[cf.][\nopage wp, Preamble]{Lgpl21OsiLicense1999a}, and the
GPL-3.0\footcite[cf.][\nopage wp, Preamble]{Gpl30OsiLicense2007a} - by which the
LGPL-3.0 inherits this purpose\footcite[cf.][\nopage wp,
prefix]{Lgpl30OsiLicense2007a}. Following this viewpoint, we want to generally
assume that open source licenses are designed to protect: They can protect
the user (recipient) of the software, its contributor resp.\ developer and/or
distributor, and the software itself. And they can protect them against
different threats:

\begin{itemize}
  \item First, we assume, that - in the context of open source software - the
  user can be protected against the loss of the right to use it, to modify it,
  and to redistribute it. Additionally, he can be protected against patent
  disputes.
  \item Second, we assume, that open source contributors and distributors can be
  protected against the loss of feedback in the form of code improvements and
  derivatives, against warranty claims, and against patent disputes.
  \item Third, we assume, that the open source programs and their specific forms
  -- may they be distributed or not, may they be modified or not, may they be
  distributed as binaries or as sources -- can be protected against the
  re-closing resp. against the re-privatizing of their further development.
  \item Fourth, we want to assume that new on-top developments being based on
  open source components can be protected against the privatizing for enlarging
  the world of freely usable software\footnote{In a more rigid version, this
  capability of a license could also be identified as the power to protect the
  community against a stagnation of the set of open source software -- but this
  description is at least a little to long to be used by the following pages}.
\end{itemize}

With respect to these viewpoints, one gets a subtler picture of the license
specific protecting power. Thus, we are going to describe and deduce the
protecting power of each of the open source licenses on the following pages.
Table \ref{tab:powerOfLicenses} summarizes the results as a quick
reference\footnote{$\rightarrow$ table \ref{tab:powerOfLicenses} on p.\
\pageref{tab:powerOfLicenses}}.

\begin{table}
\begin{minipage}{\textwidth}
\centering
\footnotesize
\caption{Open Source Licenses as Protectors}
\label{tab:powerOfLicenses}

\begin{tabular}{|c|c||c|c|c|c|c|c|c|c|c|c|c|c|c|c|c|}
\hline
  \multicolumn{2}{|c|}{\textit{Open}} &
  \multicolumn{13}{c|}{\textit{are protecting}}\\
\cline{3-15}
  \multicolumn{2}{|c|}{\textit{Source}} &
  \multicolumn{4}{c|}{ \textbf{Users}} &
  \multicolumn{3}{c|}{\textbf{Contributors}} &
  \multicolumn{5}{c|}{\textbf{Open Source Software}} &
  \multirow{4}{*}{\rotatebox{270}{\scriptsize{\textbf{On-Top Develop.\ }}}} 
  \\
\cline{10-14}
  \multicolumn{2}{|c|}{\textit{Licenses\footnote{'\checkmark' indicates that the
  license protects with respect to the meaning of the column, '$\neg$' indicates
  that the license does not protect with regard to the meaning of the column,
  and '--' indicates, that the corresponding statement must still be evaluated.
  \textit{Slanted names of licenses} indicate that these licenses are only
  listed in this table while the corresponding mindmap ($\rightarrow$ p.\
  \pageref{OSCLICMM}) does not cover them }}} &
  \multicolumn{4}{c|}{} &
  \multicolumn{3}{c|}{\tiny{(Distributors)}} &  
  not &
  \multicolumn{4}{c|}{distributed as} 
  & \\
\cline{3-9}\cline{11-14}
  \multicolumn{2}{|c|}{} &
  \multicolumn{4}{c|}{\scriptsize{\textit{who have already got}}} &
  \multicolumn{3}{c|}{\scriptsize{\textit{who spread open}}} & 
  distri- &
  \multicolumn{2}{c|}{unmodified} &
  \multicolumn{2}{c|}{modified} 
  & \\
  \cline{11-14}
  \multicolumn{2}{|c|}{} &
  \multicolumn{4}{c|}{\scriptsize{\textit{sources or binaries}}} &
  \multicolumn{3}{c|}{\scriptsize{\textit{source software}}} & 
  buted & 
 \rotatebox{270}{\footnotesize{sources\ }} &
 \rotatebox{270}{\footnotesize{binaries\ }} &
 \rotatebox{270}{\footnotesize{sources\ }} &
 \rotatebox{270}{\footnotesize{binaries\ }} 
 & \\
\cline{3-15}
  \multicolumn{2}{|c|}{} &
  \multicolumn{13}{c|}{\textit{against}}\\
\cline{3-15}
  \multicolumn{2}{|c|}{} &
  \multicolumn{3}{c|}{the loss of} & 
  \multirow{3}{*}{\rotatebox{270}{Patent Disputes}} &
  \multirow{3}{*}{\rotatebox{270}{Loss of Feedback}} & 
  \multirow{3}{*}{\rotatebox{270}{Warranty Claims}} & 
  \multirow{3}{*}{\rotatebox{270}{Patent Disputes}} & 
  \multicolumn{5}{c|}{}
  & \\
% no seperator line 
  \multicolumn{2}{|c|}{} &
  \multicolumn{3}{c|}{the right to} &
  & & & &
  \multicolumn{5}{c|}{\footnotesize{Re-Closings / Re-Privatizings}} &
  \multirow{3}{*}{\rotatebox{270}{Privatizings}}
   \\
\cline{3-5}
  \multicolumn{2}{|c|}{} & 
  \rotatebox{270}{use it} & 
  \rotatebox{270}{modify it} & 
  \rotatebox{270}{redistribute it\ } &
  &  &  &  &
  \multicolumn{5}{c|}{of already opened software}
  & \\
\hline
\hline
  ApL & 2.0 & \checkmark  & \checkmark  & \checkmark  &
  \checkmark & $\neg$ & \checkmark & \checkmark & $\neg$ &
   \checkmark  & $\neg$ & \checkmark & $\neg$ & $\neg$ \\
\hline
  \multirow{2}{*}{BSD} & 3-Cl & \checkmark & \checkmark  & \checkmark  & 
    $\neg$ & $\neg$ & \checkmark & $\neg$  &
    $\neg$ & \checkmark  & $\neg$ & \checkmark & $\neg$ & $\neg$ \\
\cline{2-15}
   & 2-Cl & \checkmark  & \checkmark  & \checkmark  & 
    $\neg$ & $\neg$ & \checkmark & $\neg$  &
    $\neg$ & \checkmark  & $\neg$ & \checkmark & $\neg$ & $\neg$ \\
\hline
  MIT & ~ & \checkmark  & \checkmark  & \checkmark  &
  $\neg$ & $\neg$ & \checkmark & $\neg$ & $\neg$ &
   \checkmark  & $\neg$ & \checkmark & $\neg$ & $\neg$ \\
\hline
  MS-PL & ~ & \checkmark  & \checkmark  & \checkmark  &
  \checkmark & $\neg$ & \checkmark & \checkmark & $\neg$ &
   \checkmark  & $\neg$ & \checkmark & $\neg$ & $\neg$ \\
\hline
  PgL & ~ & \checkmark  & \checkmark  & \checkmark  &
  $\neg$ & $\neg$ & \checkmark & $\neg$ & $\neg$ &
   \checkmark  & $\neg$ & \checkmark & $\neg$ & $\neg$ \\
\hline
  PHP & 3.0 & \checkmark  & \checkmark  & \checkmark  &
  $\neg$ & $\neg$ & \checkmark & $\neg$ & $\neg$ &
   \checkmark  & $\neg$ & \checkmark & $\neg$ & $\neg$ \\
\hline
\hline
  \textit{CDDL} & 1.0 & \checkmark & \checkmark & \checkmark &
  -- & -- & -- & -- & -- & -- & -- & -- & -- & -- \\
\hline
  EPL & 1.0 & \checkmark  & \checkmark  & \checkmark  &
  \checkmark  & \checkmark  & \checkmark & \checkmark & $\neg$ &
   \checkmark  & \checkmark & \checkmark & \checkmark & $\neg$ \\
\hline
  EUPL & 1.1 & \checkmark  & \checkmark  & \checkmark  &
  \checkmark  & \checkmark  & \checkmark & \checkmark & $\neg$ &
   \checkmark  & \checkmark & \checkmark & \checkmark & $\neg$ \\
\hline
  \multirow{2}{*}{LGPL} & 2.1 & \checkmark  & \checkmark  & \checkmark  &
   $\neg$ & \checkmark  & \checkmark & $\neg$ & $\neg$ &
   \checkmark  & \checkmark & \checkmark & \checkmark & $\neg$ \\
\cline{2-15}
   & 3.0 & \checkmark  & \checkmark  & \checkmark  &
   \checkmark & \checkmark  & \checkmark & \checkmark & $\neg$ &
   \checkmark  & \checkmark & \checkmark & \checkmark & $\neg$ \\
\hline
   MPL & 2.0 & \checkmark  & \checkmark  & \checkmark  &
  \checkmark  & \checkmark  & \checkmark & \checkmark & $\neg$ &
   \checkmark  & \checkmark & \checkmark & \checkmark & $\neg$ \\
\hline
  \textit{MS-RL} & ~ & \checkmark & \checkmark & \checkmark &
  -- & -- & -- & -- & -- & -- & -- & -- & -- & -- \\
\hline
\hline
  AGPL & 3.0 & \checkmark & \checkmark & \checkmark &
  -- & -- & -- & -- & -- & -- & -- & -- & -- & -- \\
\hline
  \multirow{2}{*}{GPL} & 2.1 & \checkmark & \checkmark & \checkmark &
   -- & -- & -- & -- & -- & -- & -- & -- & -- & -- \\
\cline{2-15}
  & 3.0 & \checkmark & \checkmark & \checkmark &
   -- & -- & -- & -- & -- & -- & -- & -- & -- & -- \\
\hline
\hline

\end{tabular}

\end{minipage}
\end{table}

\section{The protecting power of the Affero Gnu Public License (AGPL) [tbd]}
\begin{itemize} 
  \item The Affero Gnu Public License protects \ldots
  \item But the Affero Gnu Public License does not protect \ldots
\end{itemize}


\section{The protecting power of the Apache License (ApL)}
\label{sec:ProtectPowerOfApL}

As an approved \emph{open source license}\footcite[cf.][\nopage wp]{OSI2012b},
the Apache License\footnote{The Apache License, version 2.0 is maintained by the
Apache Software Foundation (\cite[cf.][\nopage wp]{AsfApacheLicense20a}).  Of
course, also the OSI is hosting a duplicate of the Apache license
(\cite[cf.][\nopage wp]{Apl20OsiLicense2004a}) and is listing it as an
officially approved open source license (\cite[cf.][\nopage wp]{OSI2012b}). The
Apache license 1.1 is classified by the OSI as \enquote{superseded
license}(\cite[cf.][\nopage wp]{OSI2013b}). In the same spirit, also the Apache
Software Foundation itself classifies the releases 1.0 and 1.1 as
\enquote{historic} (\cite[cf.][\nopage wp]{AsfLicenses2013a}). Thus, the OSLiC
only focuses on the most recent APL-2.0 version. For those, who have to fulfill
these earlier Apache licenses it could be helpful to read them as siblings of
the BSD-2CL and BSD-3CL licenses.} protects the user against the loss of the
right to use, to modify and/or to distribute the received copy of the source
code or the binaries\footcite[cf.][\nopage wp §2]{Apl20OsiLicense2004a}.
Furthermore, based on its patent clause\footnote{$\rightarrow$ OSLiC pp.\
\pageref{subsec:ApLPatentClause}}, the ApL protects the users against patent
disputes\footcite[cf.][\nopage wp §3]{Apl20OsiLicense2004a}. Because of this
patent clause and of its \enquote{disclaimer of warranty} together with its
\enquote{limitation of liability}, the Apache license also protects the
contributors / distributors against patents disputes and warranty
claims\footcite[cf.][\nopage wp §3, §7, §8]{Apl20OsiLicense2004a}. Finally, the
ApL protects the distributed sources themselves \emph{against} a change of the
license which would \emph{reset} the work \emph{as closed software}, because
first, one \enquote{[\ldots] must give any other recipients of the work or
derivative works a copy of (the Apache) license}, second, \enquote{in the source
form of any derivative works that (one) distributes}, one has \enquote{[\ldots]
to retain [\ldots] all copyright, patent, trademark, and attribution notices
[\ldots]}, and third, one must \enquote{[\ldots] include a readable copy [\ldots
of the] NOTICE file} being supplied by the original package one has
received\footcite[cf.][\nopage wp §4]{Apl20OsiLicense2004a}.

But the Apache License does not protect the contributors against the loss of
feedback because it does not 'copyleft' the software: the Apache license does
not contain any sentence requiring that one has also to publish the source code.
In the same spirit, the ApL does not protect the undistributed software or the
distributed binaries against re-closings -- neither in unmodified nor in
modified form -- because the Apache License allows to (re)distribute the
binaries without also supplying the sources -- even if the binaries rest upon
sources modified by the distributor. Finally, the ApL does not protect the
on-top developments against a privatizing.


\section{The protecting power of the BSD licenses}
\label{sec:ProtectingPowerOfBsd}

As approved \emph{open source licenses}\footcite[cf.][\nopage wp]{OSI2012b}, the
BSD Licenses\footnote{BSD has to be resolved as \emph{Berkely Software
Distribution}. For details of the BSD license release and namings
\cite[cf.][\nopage wp\ editorial]{BsdLicense3Clause}} protect the user against
the loss of the right to use, to modify and/or to distribute the received copy
of the source code or the binaries\footcite[cf.][\nopage wp §1ff]{OSI2012a}.
Additionally, they protect the contributors and/or distributors against warranty
claims of the software users, because these licenses contain a 'No Warranty
Clause'\footcite[one for all version cf.][\nopage wp]{BsdLicense2Clause}. And
finally they protect the distributed sources against a change of the license
which closes the sources, because each modification and \enquote{redistributions
of [the] source code must retain the [\ldots] copyright notice, this list of
conditions and the [\ldots] disclaimer}\footcite[cf.][\nopage
wp]{BsdLicense2Clause}: Therefore it is incorrect to distribute a BSD licensed
code under another license -- regardless, whether it closes the sources or
not\footnote{In common sense based discussions you may have heard that BSD
licenses allow to republish the work under another, an own license. Taking the
words of the BSD License seriously that is not valid under all circumstances:
Yes, it is true, you are not required to redistribute the sourcecode of a
modified (derivative) work. You are allowed to modify a received version and to
distribute the results only as binary code and to keep your improvements closed.
But if you distribute the source code of your modifications, you have retain the
licensing, because \enquote{Redistribution [\ldots] in source [\ldots], with or
without modification, are permitted provided that [\ldots] (the) redistributions
of source code [\ldots] retain the above copyright notice, this list of
conditions and the following disclaimer} (\cite[cf.][\nopage
wp]{BsdLicense2Clause})}.

But the BSD Licenses protect neither the users nor the contributors
and/or distributors against patent disputes (because they do not contain any
patent clause). They do not protect the contributors against the loss of
feedback (because they do not 'copyleft' the software). Moreover, they do not
protect the undistributed software or the distributed binaries against
re-closings -- neither in unmodified nor in modified form -- because they
allow to redistribute only the binaries without also supplying the source
code\footnote{see both, the BSD-2CL License (\cite[cf.][\nopage
wp]{BsdLicense2Clause}), and the BSD-3CL License (\cite[cf.][\nopage
wp]{BsdLicense3Clause})}. Finally, the BSD licenses do not protect the on-top
developments against a privatizing.


\section{The protecting power of the Eclipse Public License (EPL)}
\label{sec:ProtectingPowerOfEpl}

As an approved \emph{open source license}\footcite[cf.][\nopage wp]{OSI2012b},
the Eclipse Public License\footnote{ The Eclipse Public License, version 1.0 is
maintained by the Eclipse Software Foundation (\cite[cf.][\nopage
wp]{Epl10EclipseFoundation2005a}).  Of course, also the OSI is hosting a
duplicate (\cite[cf.][\nopage wp]{Epl10OsiLicense2005a}).} protects the user
against the loss of the right to use, to modify and/or to distribute the
received copy of the source code or the binaries\footcite[cf.][\nopage wp
§2a]{Epl10OsiLicense2005a}. Furthermore, based on its patent
clause\footnote{$\rightarrow$ OSLiC pp.\ \pageref{subsec:EpLPatentClause}}, the
EPL protects the users also against patent disputes\footcite[cf.][\nopage wp §2b
\& §2c]{Epl10OsiLicense2005a}. Besides this patent clause, the EPL contains the
sections \enquote{no warranty} and \enquote{disclaimer of
liability}\footcite[cf.][\nopage wp §5 \& §6]{Epl10OsiLicense2005a}. These three
elements together protect the contributors / distributors against patents
disputes and warranty claims. Finally, the EPL protects the distributed sources
themselves \emph{against} a change of the license which would \emph{reset} the
work \emph{as closed software}: First, the Eclipse Public Licenses requires that
if a work -- released under the EPL -- \enquote{[\ldots] is made available in
source code form [\ldots] (then) it must be made available under this (EPL)
agreement, too} while this act of 'making avalaible' \enquote{must} incorporate
a \enquote{copy} of the EPL into \enquote{each copy of the [distributed]
program} or program package\footcite[cf.][\nopage wp §3]{Epl10OsiLicense2005a}.
But in opposite to the permissive licenses, the EPL does not only protect the
distributed source code -- regardless whether it is modified or not. The EPL
also protects the distributed modified or unmodified binaries: The EPL allows
each modifying \enquote{contributor} and distributor \enquote{[\ldots] to
distribute the Program in object code form under (one's) own license agreement
[\ldots]} provided this license clearly states that the \enquote{source code for
the Program is available} and where the \enquote{licensees} can
\enquote{[\ldots] obtain it in a reasonable manner on or through a medium
customarily used for software exchange}\footcite[cf.][\nopage wp §3, esp.
§3.b.iv]{Epl10OsiLicense2005a}. Thus, one has to conclude that the EPL is a
copyleft license.

But the Eclipse Public License is not a license with strong copyleft; the EPL
uses 'only' a weak copyleft effect\footnote{Even if one can find contrary
specifications in the internet. \cite[Pars pro toto cf.][\nopage wp
]{ifross2011a}: This page is listing the EPL in the section \enquote{Other
Licenses with strong Copyleft Effect}}: Indeed, the EPL says that for each EPL
licensed \enquote{program} -- distributed in object form -- a place must be made
known where one can get the corresponding source code\footcite[cf.][\nopage wp
§3, esp. §3.b.iv]{Epl10OsiLicense2005a}. The term 'Program' is defined as any
\enquote{Contribution distributed in accordance with [\ldots] (the EPL)} while
the term 'Contribution' refers - besides other elements - to \enquote{changes to
the Program, and additions to the Program}\footcite[cf.][\nopage wp
§1]{Epl10OsiLicense2005a}. Unfortunately, this is a circular definition:
'Program' is defined by 'Contribution'; and 'Contribution' is defined by
'Program'. Nevertheless, one has to read the license benevolently.
Uncontroversial should be this: If one distributes any modified EPL licensed
program, library, module, or plugin, then one has to publish the modified source
code, too. If one \enquote{adds} some own plugins or additional libraries which
are used by an EPL licensed program (which on behalf of this use must have been
modified by adding [sic!] procedure calls) then one has to publish the code of
both parts: that of the program and that of the added elements. In this sense,
the EPL clearly protects the binaries against re-closings like other weak
copyleft using licenses. But if one distributes only an EPL licensed library
which is used as a component by another not EPL licensed on-top program, then
this library does not depend on the top development -- provided that the library
itself does not call any (program) functions or procedures delivered by the
overarching on-top development. Hence, nothing is added to the library; and
hence, no other code than that of the library must be published. Therefore, the
EPL does not use the strong copyleft effect in the meaning of -- for example --
the GPL.
 
\section{The protecting power of the European Union Public License (EUPL)}
\label{sec:ProtectingPowerOfEupl}

As an approved \emph{open source license}\footcite[cf.][\nopage wp]{OSI2012b},
the European Union Public License\footnote{ The European Union Public License,
version 1.1 is maintained by the European Union and hosted under the label
\enquote{Joinup} (\cite[cf.][\nopage wp]{EuplLicense2007en}).
This EUPL has officially been translated into many languages, among others into
German (\cite[cf.][\nopage wp]{EuplLicense2007de}). Because of this multi
lingual instances, the OSI does not offer its own version, but just a landing
page linked to the lading page of the European host \enquote{Joinup}
(\cite[cf.][\nopage wp]{Eupl11OsiLicense2007a}).} protects the user against the
loss of the right to use, to modify and/or to distribute the received copy of
the source code or the binaries\footcite[cf.][\nopage wp\
§2]{EuplLicense2007de}. Furthermore, based on its patent
clause\footnote{$\rightarrow$ OSLiC pp.\ \pageref{subsec:EupLPatentClause}}, the
EUPL protects the users against patent disputes\footcite[cf.][\nopage wp\ §2,
at its tail]{EuplLicense2007en}. Besides this patent clause, the EUPL
additionally contains a \enquote{Disclaimer of Warranty} and a
\enquote{Disclaimer of Liability}\footcite[cf.][\nopage wp\ §7 \&
§8]{EuplLicense2007en}. These three elements together protect the contributors /
distributors against patents disputes and warranty claims. Finally, the EUPL
also protects the distributed sources against a re-closing / re-privatizing
and the contributors against the loss of feedback. This protection is based on
two steps: First, the Europrean Public License contains a particular paragraph
titled \enquote{Copyleft clause} which stipulates that \enquote{copies of the
Original Work or Derivative Works based upon the Original Work} must be
distributed \enquote{under the terms of (the European Union Public)
License}\footcite[cf.][\nopage wp\ §5]{EuplLicense2007en}. Second, the EUPL
requires that each licensee -- as long as he \enquote{[\ldots] continues to
distribute and/or communicate the Work} -- has also to \enquote{[\ldots] provide
[\ldots] the Source Code}, either directly or by \enquote{[\ldots] (indicating)
a repository where this Source will be easily and freely available
[\ldots]}\footcite[cf.][\nopage wp\ §5]{EuplLicense2007en}. This condition
ssems to be so important for the EUPL that the license repeats its message: in
another paragraph the EUPL requires again that \enquote{if the Work is provided
as Executable Code, the Licensor provides in addition a machine-readable copy of
the Source Code of the Work along with each copy of the Work [\ldots] or
indicates, in a notice [\ldots], a repository where the Source Code is easily
and freely accessible for as long as the Licensor continues to distribute
[\ldots] the Work}\footcite[cf.][\nopage wp\ §3]{EuplLicense2007en}. Based on
the meaning of \enquote{Work} which is defined by the EUPL as \enquote{the
Original Work and/or its Derivative Works}\footcite[cf.][\nopage wp\
§1]{EuplLicense2007en} it must be concluded that the EUPL is a copyleft license.

But nevertheless, the European Union Public License is not a license with strong
copyleft: On the one hand, if one takes the core of the EUPL then the license
seems to protect not only the modifications of the original work against
re-closings and (re-)privatizings, but also the on-top developments because
normally you have to publish the source code in both cases. Understood in this
way, the EUPL would be a 'strong copyleft license'. But on the other hand, the
EUPL additionally contains a \enquote{Compatibility clause} stating that
\enquote{if the Licensee Distributes [\ldots] Derivative Works or copies thereof
based upon both the Original Work and another work licensed under a Compatible
Licence, this Distribution [\ldots] can be done under the terms of this
Compatible Licence}\footcite[cf.][\nopage wp\ §5]{EuplLicense2007en} -- while
the term \enquote{Compatible Licence} is explicitly defined by a list of
compatible licenses, for example the Eclipse Public
License\footcite[cf.][\nopage wp\ Appendix]{EuplLicense2007en}. Based on this
compatibility clause the obligation to publish the code of an on-top development
can be subverted: As first step, you could release a little, more or less futile
on-top application licensed under the Eclipse Public License\footnote{Taking the
license text very seriously, it is not even necessary that this little futile
application must depend on the EUPL library by calling functions of EUPL
library. The license text only says that \enquote{another [any other] work
licensed under a Compatible Licence} can be distributed together with
\enquote{derivative works}. By thisthe wording, the license itself is
establishing a contrast between the derivative work and the other work - what
indicates that the other work has not necessarily also to be a derivative work.}
which uses a library licensed under the EUPL. As second step, you add this 'EUPL
library' which you now may also distribute under the EPL instead of retaining
the EUPL licensing. So, finally you obtain the same work under the Eclipse
Public License which is a weak copyleft license\footnote{$\rightarrow$ OSLiC,
p.\ \pageref{sec:ProtectingPowerOfEpl}}. Hence the protection of the EUPL-1.1 is
not as comprehensive as one might assume on the base of the license text
itself\footnote{This kind of specifiying the protective power of the EUPL is
initially be presented by the FSF (\cite[cf.][wp\ section 'European Union
Public License']{FsfEuplStatement2013a}). The EU answers that publishing such a
trick will comprise its user in the eyes of the open source community
(\cite[cf.][wp]{FsfEuplRecomment2013}). That is undoubtely true. But
unfortunately, this argument does not close the hole in the protecting shield
put up by the EUPL}, it can at most be a weak copyleft license -- even if the
reader might get the impression that the authors of the EUPL wished to write a
strong copyleft license. Howsoever, the EUPL license does not protect the on-top
developments against a privatizing.

\section{The protecting power of the Gnu Public License (GPL)}
\label{sec:ProtectingPowerOfGpl}

% TODO erklären warum GPL20 und LGPL 2.1

Although the GPL versions 2.0 and 3.0 are aiming for the same results, they
heavily differ with respect to textual and arguing structure. Therefore, it
should be helpful to treat these two licenses separately.

\subsubsection {GPL-2.0 [tdb]} \label{subsec:ProtectingPowerOfGpl20}

\subsubsection {GPL-3.0 [tbd]}\label{subsec:ProtectingPowerOfLgpl30}

\section{The protecting power of the Lesser Gnu Public License (LGPL)}
\label{sec:ProtectingPowerOfLgpl}

The LGPL is maintained and offered by the Free Software Foundation 
%TDO license history LGPL. Says we are using OSI version

As already mentioned, the LGPL versions 2.1 and 3.0 heavily differ  with respect
to textual and arguing structure. Therefore, they should be treated separately.

\subsubsection {LGPL-2.1 [tbd]} \label{subsec:ProtectingPowerOfLgpl21}

Like the other versions of the GPL or LPGL, the LGPL-2.0 also explicitly
describes its purpose as the task to \enquote{protect [\ldots] rights}: it
states that generally all \enquote{[\ldots] the GNU General Public Licenses are
intended to guarantee your freedom to share and change free software
[\ldots]}\footcite[cf.][\nopage wp, Preamble]{Lgpl21OsiLicense1999a}. So, of
course the LGPL-2.1 is an an approved \emph{open source
license}\footcite[cf.][\nopage wp]{OSI2012b} which protects the user against the
loss of the right to use, to modify and/or to distribute the received copy of
the source code or the binaries\footcite[cf.][\nopage wp §1, §2,
§4]{Lgpl21OsiLicense1999a}. But the LGPL-2.1 does not offer any sentences to
infer that it grants any patent rights to the software
user\footnote{$\rightarrow$ OSLiC, p.\ \pageref{subsec:Lgpl21PatentClause}}. So,
it does not protect anayone against patent disputes, neither the users, nor the
contributors / distributors. Instead of this, the LGPL-2.1 contains a special
section \enquote{No Warranty} containing two paragraphs which together establish
the protection of the contributors and distributors against warranty
claims\footcite[cf.][\nopage wp §15, §16]{Lgpl21OsiLicense1999a}. Finally, the
LGPL-2.1 also protects the distributed sources against a re-closing /
re-privatizing and the contributors against the loss of feedback. For that
purpose, the LGPL-2.1 on the one hand states that one \enquote{[\ldots] may
modify (his) copy or copies of the Library or any portion of it [\ldots] and
copy and distribute such modifications [\ldots]} provided that the results of
these modifications are \enquote{[\ldots] licensed at no charge to all third
parties under the terms of (the LGPL-2.1)}\footcite[cf.][\nopage wp
§2]{Lgpl21OsiLicense1999a}. On the other hand, the LGPL allows to distribute
such modifications \enquote{in object code or executable form} provided that one
accompanies these entities \enquote{[\ldots] with the complete corresponding
machine-readable source code} which itself must be distributed under the terms
of the LGPL-2.1\footcite[cf.][\nopage wp §4]{Lgpl21OsiLicense1999a}.

But in opposite to the GPL, the LGPL does not require to publish the code of an
overarching program or any on-top development: It distinguishes the
\enquote{work that \emph{uses} the Library} from the \enquote{work \emph{based
on} the Library}: First it defines the \enquote{Library} as any
\enquote{software library or work} licensed under the LGPL-2.1 and adds that
\enquote{a 'work \emph{based on} the Library' means either the Library or any
derivative work under copyright law}\footcite[cf.][\nopage wp
§0]{Lgpl21OsiLicense1999a}. Second it defines the \enquote{work that \emph{uses}
the Library} as any \enquote{[\ldots] program that contains no derivative of an
portion of the Librabry, but is designed to work with the Library by beinging
compiled or linked with it} whereas this \enquote{work that \emph{uses} the
Library} -- taken \enquote{in isolation} -- clearly \enquote{[\ldots] is not a
derivative work of the Library [\ldots]}, but can \footcite[cf.][\nopage wp
§5]{Lgpl21OsiLicense1999a}




\subsubsection {LGPL-3.0 [tbd]}\label{subsec:ProtectingPowerOfLgpl30}



\section{The protecting power of the MIT license}
\label{sec:ProtectingPowerOfMit}

As an approved \emph{open source license}\footcite[cf.][\nopage wp]{OSI2012b},
the MIT License\footcite[MIT has to be resolved as \enquote{Massachusetts
Institute of Technology} (cf.][\nopage wp)]{wpMitLic2011a} protects the user
against the loss of the right to use, to modify and/or to distribute the
received copy of the source code or the binaries\footcite[cf.][\nopage wp
1ff]{OSI2012a}. Additionally, it protects the contributors and/or distributors
against warranty claims of the software users, because it contains a 'No
Warranty Clause'\footcite[cf.][\nopage wp]{MitLicense2012a}. And finally it
protects the distributed sources against a change of the license which would
close the sources, because the \enquote{permission [\ldots] to use, copy,
modify, [\ldots] distribute, [\ldots] (is granted) subject to the [\ldots]
conditions, [that] the [\ldots] copyright notice and this permission notice
shall be included in all copies or substantial portions of the
Software}\footnote{\cite[cf.][\nopage wp]{MitLicense2012a}. The argumentation
why the source code is protected, but not the binary form follows that of the
BSD licenses: By these requirements, one is not obliged to redistribute the
sourcecode of a modified (derivative) work. One is allowed to modify a received
version and to distribute the results only in binary form and to keep one's
improvements closed. But if one distribute the source code of the modifications,
the licensing is retained, simply because the MIT \enquote{[\ldots] permission
note shall be included in all copies or substantial portions of the software}.}

But the MIT License does not protect the users or the contributors and/or
distributors against patent disputes (because it does not contain any patent
clause). Additionally, it does not protect the contributors against the loss of
feedback (because it does not 'copyleft' the software). Moreover, the MIT
license does not protect the undistributed software or the distributed binaries
against re-closings -- neither in unmodified nor in modified form -- because it
allows to redistribute only the binaries without also supplying the source
code\footcite[cf.][\nopage wp]{MitLicense2012a}. Finally, the MIT license does
not protect the on-top developments against a privatizing.

\section{The protecting power of the Mozilla Public License (MPL)}
 \label{sec:ProtectingPowerOfMpl}
 
As an approved \emph{open source license}\footcite[cf.][\nopage wp]{OSI2012b},
the Mozilla Public License\footnote{In 2012, the Mozilla Public License 2.0
(\cite[cf.][\nopage wp]{Mpl20MozFoundation2012a}) has been released as a result
of a longer \enquote{Revision Process}(\cite[cf.][\nopage
wp]{Mpl11To20MozFoundation2013a}) by which the  Mozilla Public License 1.1
(\cite[cf.][\nopage wp]{Mpl11MozFoundation2013a}) has been ousted. The OSI is
also hosting its version of the MPL-2.0 (\cite[cf.][\nopage
wp]{Mpl20OsiLicense2013a}) and is listing it as an OSI approved license
(\cite[cf.][\nopage wp]{OSI2012b}) while it classifies the MPL-1.1 as a
\enquote{superseded license}(\cite[cf.][\nopage wp]{OSI2013b}). The Mozilla
Foundation itself says concerning the difference between the two licenses that
\enquote{the most important part of the license - the file-level copyleft - is
essentially the same in MPL 2.0 and MPL 1.1} (\cite[cf.][\nopage
wp]{Mpl11To20MozFoundation2013a}). By reading the MPL-1.1, one could get the
impression that fulfilling all conditions of the MPL-2.0 would imply also to act
in accordance to the MPL-1.1. Thus the OSLiC focuses on the MPL-2.0, at least
for the moment. Nevertheless, in this section we want to use the general label
'MPL' without any releasenumber for indicating that with respect to its
protecting power the MPL-2.0 and the MPL-1.1 can be taken as equipollent.}
protects the user against the loss of the right to use, to modify and/or to
distribute the received copy of the source code or the
binaries\footcite[cf.][\nopage wp\ §2.1.a]{Mpl20OsiLicense2013a}.
Furthermore, based on its split and distributed patent
clause\footnote{$\rightarrow$ OSLiC pp.\ \pageref{subsec:MplPatentClause}}, the
MPL protects the users against patent disputes\footcite[cf.][\nopage wp\
§2.1.b, §2.3, §5.2]{Mpl20OsiLicense2013a}. Besides this patent sections, the MPL
additionally contains a \enquote{Disclaimer of Warranty} and a
\enquote{Limitation of Liability}\footcite[cf.][\nopage wp\ §6 \&
§7]{Mpl20OsiLicense2013a}. These three elements together protect the
contributors / distributors against patents disputes and warranty claims.
Finally, the MPL also protects the distributed sources against a re-closing /
re-privatizing and the contributors against the loss of feedback: The MPL
clearly says that, on the one hand, \enquote{all distribution of Covered
Software in Source Code Form, including any Modifications[\ldots] must be under
the terms of this License}\footcite[cf.][\nopage wp\
§3.1]{Mpl20OsiLicense2013a} and that, on the other hand, an MPL licensed
software \enquote{[\ldots] (distributed) in Executable Form [\ldots] must also
be made available in Source Code Form [\ldots]}\footcite[cf.][\nopage wp\
§3.2]{Mpl20OsiLicense2013a}. So, it must be inferred that the MPL is a copyleft
license.

But nevertheless, the Mozilla Public License is not a license with strong
copyleft. It does not protect on-top developments against privatizings: First,
the MPL does not use the term \emph{derivative work}\footnote{
\cite[cf.][\nopage wp]{Mpl20OsiLicense2013a}. The MPL-1.1 uses the term
\emph{derivative work} only in the context of writing new \enquote{versions of
the license}, not in the context of licensing software (\cite[cf.][\nopage wp
§6.3]{Mpl11MozFoundation2013a}).}. Instead of this, the MPL denotes the
\enquote{[\ldots] (initial) Source Code Form [\ldots] and Modifications of such
Source Code Form} by the label \enquote{Covered Software}\footcite[cf.][\nopage
wp\ §1.4]{Mpl20OsiLicense2013a} -- while the term \enquote{Modifications}
refers to \enquote{any file in Source Code Form that results from an addition
to, deletion from, or modification of the contents of Covered Software or any
file in Source Code Form that results from an addition to, deletion from, or
modification of the contents of Covered Software}\footnote{\cite[cf.][\nopage
wp\ §1.10]{Mpl20OsiLicense2013a}. The Mozilla Foundation denotes this reading
by the term \enquote{file-level copyleft} (\cite[cf.][\nopage
wp]{Mpl11To20MozFoundation2013a}).}. Second, the MPL contrasts the source code
form and its modifications with the \enquote{Larger Work} by specifying that the
larger work is \enquote{[\ldots] material, in a seperate file or files, that is
not covered software}\footcite[cf.][\nopage wp\ §1.7]{Mpl20OsiLicense2013a}.
Finally, the MPL states, that \enquote{you may create and distribute a Larger
Work under terms of Your choice, provided that You also comply with the
requirements of this License for the Covered Software}\footcite[cf.][\nopage
wp\ §3.3]{Mpl20OsiLicense2013a}. Based on these specifications, one has to
reason that an on-top development which depends on MPL licensed libraries by
calling some of their functions, is undoubtably a derivative work\footnote{This
follows from the general meaning of a \emph{derivative work} as a benevolent
software developer would read this term ($\rightarrow$ OSLiC, pp.\
\pageref{sec:BenevolentDerivativeWorkUnderstanding}). But again: The MPL does
not focus on this general aspect; it uses its own concept of a \emph{larger
work}.}, but also only a larger work in the meaning of the MPL so that code of
this on-top application needs not to be published -- provided, that the library
and the on-top development are distributed as different files\footnote{It might
be discussed whether integrating a declaration of a function, class, or method
into the on-top development by including the corresponding header files indeed
means that one is \enquote{including portions (of the Source Code Form)} into a
file which therefore has to be taken as \enquote{Modification}
(\cite[cf.][\nopage wp\ §1.4]{Mpl11MozFoundation2013a}). From the viewpoint of
a benevolent developer it should be difficult to argue that the including of
declaring (header) files alone can evoke a derivative work. It is the call of
the function in one's code which establishes the dependency. But that is not the
point, the MPL focuses. The MPL aims on the textual reuse of (defining) code
snippets. Hence, one could ignore the textual integration of parts of the
declaring header files: it should not trigger that one's own work becomes a
modification in the eyes of the Mozilla Findation. But of course, one would
circumvent the idea of the MPL if one hides defining code in header files and
reuses that code by one's own compilation. This would undoubtably be an
incorporation of portions and therefore would make the incorporating file
becoming a modification of the MPL licensed initial work. }. Hence, the MPL is
license with a weak copyleft effect and does not protect the on-top developments
against privatizings.

\section{The protecting power of the Microsoft Public License (MS-PL)}
\label{sec:ProtectingPowerOfMspl}

As an approved \emph{open source license}\footcite[cf.][\nopage wp]{OSI2012b},
the Microsoft Public License protects the user against the loss of the right to
use, to modify and/or to distribute the received copy of the source code or the
binaries\footcite[cf.][\nopage wp
§2]{MsplOsiLicense2013a}. Furthermore, based on its patent
clause\footnote{$\rightarrow$ OSLiC pp.\ \pageref{subsec:MsplPatentClause}}, the
MS-PL protects the users against patent disputes\footcite[cf.][\nopage wp
§2.B and §3.B]{MsplOsiLicense2013a}. Because of this patent clause and of its
concise \emph{disclaimer of warranty}, the MS-PL also protects the contributors
/ distributors against patents disputes and warranty
claims\footcite[cf.][\nopage wp §2B, §3B, §3D]{MsplOsiLicense2013a}.
Finally, the Microsoft Public License protects the distributed sources
themselves - and even \enquote{portions of these sources} -- \emph{against} a
change of the license which would \emph{reset} the work \emph{as closed
software}, because first, one \enquote{[\ldots] must retain all copyright,
patent, trademark, and attribution notices that are present in the
software}\footcite[cf.][\nopage wp §3C]{MsplOsiLicense2013a}, and because
second, one must also incorporate \enquote{a complete copy of this license} into
one's own distribution premised one distributes the source
code\footcite[cf.][\nopage wp §3D]{MsplOsiLicense2013a}.

But the Microsoft Public License does not protect the contributors against the
loss of feedback because it does not 'copyleft' the software: The license does
not contain any sentence which requires that one has to publish the sources,
too\footnote{There seems to be some misunderstandings on the internet: The
English wikipedia specifies the MS-PL as a permissive license and the MS-RL as a
license with copyleft effect (\cite[cf.][\nopage wp]{wpMsSharedSources2013a}).
The German wikipedia says that the MS-PL is a license with a \enquote{schwachen
[weak] copyleft} (\cite[cf.][\nopage wp]{wpMspl2013a}). And it says also that
the \enquote{Microsoft Reciprocal License} (MS-RL) is a license with weak
copyleft, too (\cite[cf.][\nopage wp]{wpMsrl2013a}). But for the very
thoroughly working \enquote{ifross license center}, the MS-RL is a license with
restricted (weak) copyleft, while the MS-PL is a permissive license with some
selectable options (\cite[cf.][\nopage wp]{ifross2011a}). Based on the license
text itself and these other readings, we decided to take the MS-PL as a
permissive license in accordance to the English wikipedia page and the ifross
page.}. In the same spirit, the MS-PL does not protect the undistributed
software or the distributed binaries against re-closings -- neither in
unmodified nor in modified form -- because the MS-PL License allows to
(re)distribute the binaries without also supplying the sources -- even if the
binaries rest upon sources modified by the distributor. Finally, also the MS-PL
does not protect the on-top developments against a privatizing.


\section{The protecting power of the Postgres License (PgL)}
\label{sec:ProtectingPowerOfPgl}

As an approved \emph{open source license}\footcite[cf.][\nopage wp]{OSI2012b},
the PostgreSQL License protects the user against the loss of the right to use,
to modify and/or to distribute the received copy of the source code or the
binaries\footcite[cf.][\nopage wp]{PglOsiLicense2013a}.
Because of its \emph{disclaimer of warranty}, the PgL also protects the
contributors / distributors against warranty claims\footcite[cf.][\nopage
wp]{PglOsiLicense2013a}. Finally, the PgL protects the distributed sources
themselves \emph{against} a change of the license which would \emph{reset} the
work \emph{as closed software}, because the \enquote{copyright notice} and the
whole license must \enquote{[\ldots] appear in all copies}\footcite[cf.][\nopage
wp]{PglOsiLicense2013a}.

But the PostgreSQL License does not protect the contributors against the loss of
feedback because it does not 'copyleft' the software: The license does not
contain any sentence which requires that one has to publish the sources, too. 
In the same spirit, the PgL does not protect the undistributed software or the
distributed binaries against re-closings -- neither in unmodified nor in
modified form -- because the PgL allows to (re)distribute the binaries without
also supplying the sources -- even if the binaries rest upon sources modified by
the distributor. Finally, the PgL does not protect the on-top developments
against a privatizing.


\section{The protecting power of the PHP License}
\label{sec:ProtectingPowerOfPhp}

As an approved \emph{open source license}\footcite[cf.][\nopage wp]{OSI2012b},
the PHP-3.0 License protects the user against the loss of the right to use, to
modify and/or to distribute the received copy of the source code or the
binaries\footcite[cf.][\nopage wp]{Php30OsiLicense2013a}. Because of its
\emph{disclaimer of warranty}, the PHP license also protects the contributors /
distributors against warranty claims\footcite[cf.][\nopage
wp]{Php30OsiLicense2013a}. Finally, the PHP license protects the distributed
sources themselves \emph{against} a change of the license which would
\emph{reset} the work \emph{as closed software}, because
\enquote{redistributions of source code must retain the [\ldots] copyright
notice, this list of conditions and the [\ldots]
disclaimer}\footcite[cf.][\nopage wp]{Php30OsiLicense2013a}.

But the PHP-3.0 License does not protect the contributors against the loss of
feedback because it does not 'copyleft' the software: The license does not
contain any sentence which requires that one has to publish the sources, too. 
In the same spirit, the PHP license does not protect the undistributed software
or the distributed binaries against re-closings -- neither in unmodified nor in
modified form -- because the PHP license allows to (re)distribute the binaries
without also supplying the sources -- even if the binaries rest upon sources
modified by the distributor.
  
\section{Summary}

All these specifications cannot only be summarized by a
table\footnote{$\rightarrow$ OSLiC, p. \pageref{tab:powerOfLicenses}}, but also
by a mindmap:

\begin{tikzpicture}
\label{OSCLICMM}
\footnotesize

% (1.A) list of all licenses and their release numbers Level 5/6
\node[rectangle,draw,text width=1.4cm] (l0100) at (9,4)
{ \textit{BSD License} };
\node[text width=1.4cm] (l0101) at (8.25,3)
{ \scriptsize{3-Clauses} };
\node[text width=1.4cm] (l0102) at (10,3)
{ \scriptsize{2-Clauses} };
  
\node[rectangle,draw,text width=1.4cm] (l0200) at (10.2,5)
{ \textit{MIT License} }; 
  
\node[rectangle,draw,text width=1.4cm] (l0300) at (12,5.5)
{ \textit{\textbf{Ap}ache \textbf{L}icense}};
\node[text width=0.4cm] (l0301) at (12,4.5) {\scriptsize{2.0}};

\node[rectangle,draw,text width=1.4cm] (l0400) at (13,6.8)
{ \scriptsize{\textit{\textbf{M}icro\textbf{s}oft} \textbf{P}ublic \textbf{L}icense} };
  
\node[rectangle,draw,text width=1.4cm] (l0500) at (13,8)
{\textit{\textbf{P}ost\textbf{g}res \textbf{L}icense}};
  
\node[rectangle,draw,text width=1.4cm] (l0600) at (13,9)
{\textit{\textbf{PHP} License}};
\node[text width=0.4cm] (l0601) at (14.5,9){\scriptsize{3.0}};
  

\node[rectangle,draw,text width=1.4cm] (l0800) at (13,10.7)
{ \textit{\textbf{M}ozilla \textbf{P}ublic \textbf{L}icense}};
\node[text width=0.4cm] (l0801) at (14.5,10.2){\scriptsize{1.1}};
\node[text width=0.4cm] (l0802) at (14.5,11.2){\scriptsize{2.0}};

\node[rectangle,draw,text width=1.4cm] (l0900) at (13,12.25)
{\textit{\textbf{E}clipse \textbf{P}ublic \textbf{L}icense}};
\node[text width=0.4cm] (l0901) at (14.5,12.25) {\scriptsize{1.0}};
 
\node[rectangle,draw,text width=1.5cm] (l1000) at (13,13.8)
{\textit{\textbf{E}uropean \textbf{P}ublic \textbf{L}icense}}; 
\node[text width=0.4cm] (l1001) at (14.5,13.3){\scriptsize{1.1}};
\node[text width=0.4cm,style=dotted] (l1002) at (14.5,14.3){\scriptsize{\textit{1.2}}};
  
\node[rectangle,draw,text width=1.4cm] (l1100) at (13,15.5)
{\textit{\textbf{L}esser \textbf{G}NU \textbf{P}ublic \textbf{L}icense}};

\node[text width=0.4cm] (l1101) at (14.5,15){\scriptsize{2.1}};
\node[text width=0.4cm] (l1102) at (14.5,16){\scriptsize{3.0} };

\node[rectangle,draw,text width=1.4cm] (l1200) at (13,17.5)
{\textit{\textbf{G}NU \textbf{P}ublic \textbf{L}icense}};

\node[text width=0.4cm] (l1201) at (14.5,17){\scriptsize{2.1}};
\node[text width=0.4cm] (l1202) at (14.5,18){\scriptsize{3.0} };

\node[rectangle,draw,text width=1.4cm] (l1300) at (13,19.5)
{ \textit{\textbf{A}ffero \textbf{G}NU \textbf{P}ublic \textbf{L}icense}};
\node[text width=0.4cm] (l1302) at (14.5,19.5){\scriptsize{3.0}};

% 2. the clustering concepts of licenses (level 4)
\node[rectangle,draw,text width=2.3cm] (n0100) at (10,8)
 { \textit{protecting the user, the con\-tri\-butor \& the initial code}\\
   \tiny{Permissive Licenses}      
 };

\node[rectangle,draw,text width=2.3cm] (n0200) at (10,12.5)
{ \textit{protecting the user, the con\-tri\-butor, the
  initial code, \& all di\-rect de\-ri\-va\-tions}\\
  \tiny{Weak Copyleft}        
};

\node[rectangle,draw,text width=2.3cm] (n0300) at (10,16.5)
{ \textit{protecting the user, the con\-tri\-bu\-tor, the 
  initial code, all di\-rect de\-ri\-va\-tions \& the 
  (in\-di\-rect\-ly de\-ri\-ved) on-top-deve\-lop\-ments}\\ 
  \tiny{Strong Copyleft}    
 };

% 3. the threats (level 3)
\node[ellipse,draw,text width=1.6cm] (c110000) at (4.5,0)
{ \textbf{\textit{Patent Disputes}}};

\node[ellipse,draw,text width=1.6cm] (c120000) at (4.5,2)
{ \textbf{\textit{Loss of Rights}} };

\node[ellipse,draw,text width=1.6cm] (c210000) at (4.5,4)
{ \textbf{\textit{Warranty Claims}} };
 
\node[ellipse,draw,text width=1.6cm] (c220000) at (4.5,6)
{ \textbf{\textit{Loss of Feeback}}};

\node[ellipse,draw,text width=0.6cm] (c311000) at (6.2,8)
{ \tiny{\textit{\textbf{reclos\-ings}}}};

\node[ellipse,,draw,text width=0.6cm] (c321000) at (6.2,10)
{ \tiny{\textit{\textbf{reclos\-ings}}} };

\node[ellipse,,draw,text width=0.6cm] (c331000) at (6.2,12)
{ \tiny{\textit{\textbf{reclos\-ings}}} };

\node[ellipse,,draw,text width=0.6cm] (c341000) at (6.2,14)
{ \tiny{\textit{\textbf{reclos\-ings}}} };

\node[ellipse,,draw,text width=0.6cm] (c351000) at (6.8,16.2)
{ \tiny{\textit{\textbf{reclos\-ings}}} };

\node[ellipse,,draw,text width=0.7cm] (c361000) at (7.5,17.5)
{ \tiny{\textit{\textbf{privati\-zings}}} };

\node[ellipse,,draw,text width=1.6cm] (c411000) at (6.5,19)
{ \textit{\textbf{clos\-ings}} };


% 4. the subtypes of protected entities (level 2)
\node[ellipse,draw,text width=1.5cm] (c310000) at (3,8)
 { \scriptsize{un\-modified} \textbf{Sources}};

\node[ellipse,draw,text width=1.5cm] (c320000) at (3.25,10)
 { \scriptsize{un\-modified} \textbf{Binaries}};

\node[ellipse,draw,text width=1.2cm] (c330000) at (3.5,12)
 { \scriptsize{modified} \textbf{Sources}};

\node[ellipse,draw,text width=1.4cm] (c340000) at (3.25,14)
 { \scriptsize{modified} \textbf{Binaries}};

\node[ellipse,draw,text width=2cm] (c350000) at (3.6,16)
 { \tiny{\textbf{part of} On-Top-Developments}};

\node[ellipse,draw,text width=2.9cm] (c360000) at (3.4,17.5)
 { \tiny{\textbf{On-Top-Developments}}};


% 5. the protected entities (level 1)
\node[ellipse,draw,text width=1cm] (c100000) at (1,1)
 { \textbf{Users} };

\node[ellipse,draw,text width=0.8cm] (c200000) at (1,5)
 { \textbf{Con\-tribu\-tors}};

\node[ellipse,draw,text width=0.8cm] (c300000) at (1,12)
 { distri\-buted \textbf{Soft\-ware}};
 
\node[ellipse,draw,text width=2.2cm] (c400000) at (1,19)
 { un\-distri\-buted \textbf{Soft\-ware}}; 

% 6. main node (leve 0)
\node[ellipse,draw,text width=1.3cm] (c000000) at (0,8)
{ \textbf{open source license}};

% a linking Licenses to their release numbers (Linking level 5 to 6)
\foreach \father/\daughter in {
  l0100/l0101/,
  l0100/l0102/,
  l0300/l0301/,
  l0600/l0601/,
  l0800/l0801/,
  l0800/l0802/,
  l0900/l0901/,
  l1000/l1001/,
  l1000/l1002/,
  l1100/l1101/,
  l1100/l1102/,
  l1200/l1201/,
  l1200/l1202/,
  l1300/l1302/
  }
  \draw[dashed] (\father) to  (\daughter) ;

% b) linking Licenses to license concepts (Linking level 5 to 4)
\foreach \father/\daughter/\outangle/\inangle in {
  n0100/l0100/270/150,       
  n0100/l0200/280/155,
  n0100/l0300/290/160,
  n0100/l0400/300/165,
  n0100/l0500/310/150,
  n0100/l0600/340/160,
  n0200/l0800/300/160,
  n0200/l0900/340/170,
  n0200/l1000/20/190,
  n0200/l1100/60/200,
  n0300/l1200/40/180,
  n0300/l1300/80/180 
  }
  %\draw[dashed] (\father) to [out=\outangle,in=\inangle] (\daughter) ;
  \draw[dashed] (\father) to  (\daughter) ;

% c) linking license concepts to the threats against they protect
% c.1) strong copyleft licenses
\foreach \father/\daughter/\outangle/\inangle in {
  c361000/n0300/0/180,
  c351000/n0300/0/180,
  c341000/n0300/45/190,
  c331000/n0300/50/200,
  c321000/n0300/55/210,
  c311000/n0300/60/220,
  c220000/n0300/25/225,
  c210000/n0300/25/230,
  c120000/n0300/25/235
  }
  \draw[<-,color=blue] (\father) to [out=\outangle,in=\inangle] (\daughter) ;
% c.2) weak copyleft licenses
\foreach \father/\daughter/\outangle/\inangle in {
  c341000/n0200/330/170,
  c331000/n0200/0/180,
  c321000/n0200/0/180,
  c311000/n0200/20/190,
  c220000/n0200/15/220,
  c210000/n0200/15/230,
  c120000/n0200/15/235
  }
  \draw[<-,color=cyan] (\father) to [out=\outangle,in=\inangle] (\daughter) ;
% c.3) permissive licenses
\foreach \father/\daughter/\outangle/\inangle in {
  c331000/n0100/355/150,
  c311000/n0100/0/180,
  c210000/n0100/5/210,
  c120000/n0100/10/230
  }
  \draw[<-,color=red] (\father) to [out=\outangle,in=\inangle] (\daughter) ;
%c.4 agpl license
\foreach \father/\daughter/\outangle/\inangle in {
  c411000/l1300/0/180    
}
  \draw[<-,color=green] (\father) to [out=\outangle,in=\inangle] (\daughter) ;


%d linking protected entities, their subtypes and the the relations
\foreach \father/\daughter/\edgetext/\outangle/\inangle in {
  c000000/c100000/protecting/260/120,
  c100000/c110000/against/360/180,
  c100000/c120000/against/360/180,
  c000000/c200000/protecting/270/180,
  c200000/c110000/against/340/150,
  c200000/c210000/against/0/180,
  c200000/c220000/against/0/180,
  c000000/c300000/protecting/90/230,
  c300000/c310000/as/300/180,
  c300000/c320000/as/330/180,
  c300000/c330000/as/0/180,
  c300000/c340000/as/30/180,
  c300000/c350000/as/60/180,
  c300000/c360000/as/70/180,
  c000000/c400000/protecting/100/240,
  c400000/c411000/against/0/180        
}
  \draw[->,dotted,
    decoration={text along path,
              text align={center},
              text={|\itshape|\edgetext}},
              postaction={decorate},] (\father) to [out=\outangle,in=\inangle] (\daughter) ;

\foreach \father/\daughter/\edgetext/\outangle/\inangle in {
  c310000/c311000/against/0/180,
  c320000/c321000/against/0/180,
  c330000/c331000/against/0/180,
  c340000/c341000/against/00/180,
  c350000/c351000/against/0/180,
  c360000/c361000/against/0/180      
}
  \draw[->,dotted,
    decoration={text along path,
              text align={center},
              text={|\itshape \tiny|\edgetext}},
              postaction={decorate},] (\father) to [out=\outangle,in=\inangle] (\daughter) ;

%f linking the patent clauses
\foreach \father/\daughter/\outangle/\inangle in {
  c110000/l1302/0/305,
  c110000/l1202/0/303,
  c110000/l1201/0/301,
  c110000/l1102/0/299,
  c110000/l1101/0/297,
  c110000/l1002/0/295,
  c110000/l0901/0/290,
  c110000/l0802/0/285,
  c110000/l0400/0/275,
  c110000/l0301/0/270   
}
  \draw[<-,color=gray] (\father) to [out=\outangle,in=\inangle] (\daughter) ;

\end{tikzpicture}

Finally, one could generate new groups of open source license, new classes, like
'user protecting licenses'\footnote{all of them because all of them have to
fulfill the OSD}, 'patent disputes fending licenses', an so on.
% TODO nach Fertigstellung eigene Taxonomie entwickeln

However, it must kept in mind that all of these grouping viewpoints do not
permit the conclusion that all members of a group can be respected by fulfilling
the same requirements. This would only be possible if the grouping criteria
would directly refer to the fulfilling tasks. Indeed, nearly all open source
licenses do differ with respect to these criteria, and even if the differences
are very small, they can't be neglected\footnote{Pars pro toto: Both, the BSD
license and the Apache license require that you provide an indication to the
developers of the application. But in case of the BSD license you have to
publish the copyright notice / line, while in case of the Apache license you
have exactly to present the content of the notice file distributed together with
the application.}. So: reflecting on possible classes of open source licenses is
a good method to become familiar with the area of open source licenses. But it
is not a method to determine, what one needs to be done to obtain the right to
use the software. For that purpose every license must be considered
individually.



%\bibliography{../../../bibfiles/oscResourcesEn}



%%%%%%%%%%%%%%%

% Telekom osCompendium 'for being included' snippet template
%
% (c) Karsten Reincke, Deutsche Telekom AG, Darmstadt 2011
%
% This LaTeX-File is licensed under the Creative Commons Attribution-ShareAlike
% 3.0 Germany License (http://creativecommons.org/licenses/by-sa/3.0/de/): Feel
% free 'to share (to copy, distribute and transmit)' or 'to remix (to adapt)'
% it, if you '... distribute the resulting work under the same or similar
% license to this one' and if you respect how 'you must attribute the work in
% the manner specified by the author ...':
%
% In an internet based reuse please link the reused parts to www.telekom.com and
% mention the original authors and Deutsche Telekom AG in a suitable manner. In
% a paper-like reuse please insert a short hint to www.telekom.com and to the
% original authors and Deutsche Telekom AG into your preface. For normal
% quotations please use the scientific standard to cite.
%
% [ File structure derived from 'mind your Scholar Research Framework' 
%   mycsrf (c) K. Reincke CC BY 3.0  http://mycsrf.fodina.de/ ]
%

% Chapter Abstract
% ----------------
\chapter{Open Source: Some aspects with side effects}\label{sec:SideEffects}

\footnotesize
\begin{quote}\itshape
This chapter we shortly discusses some minor but although important issues.
\end{quote}
\normalsize{}


% Telekom osCompendium 'for being included' snippet template
%
% (c) Karsten Reincke, Deutsche Telekom AG, Darmstadt 2011
%
% This LaTeX-File is licensed under the Creative Commons Attribution-ShareAlike
% 3.0 Germany License (http://creativecommons.org/licenses/by-sa/3.0/de/): Feel
% free 'to share (to copy, distribute and transmit)' or 'to remix (to adapt)'
% it, if you '... distribute the resulting work under the same or similar
% license to this one' and if you respect how 'you must attribute the work in
% the manner specified by the author ...':
%
% In an internet based reuse please link the reused parts to www.telekom.com and
% mention the original authors and Deutsche Telekom AG in a suitable manner. In
% a paper-like reuse please insert a short hint to www.telekom.com and to the
% original authors and Deutsche Telekom AG into your preface. For normal
% quotations please use the scientific standard to cite.
%
% [ Framework derived from 'mind your Scholar Research Framework' 
%   mycsrf (c) K. Reincke 2012 CC BY 3.0  http://mycsrf.fodina.de/ ]
%


%% use all entries of the bibliography
%\nocite{*}

\section{The problem of implicitly releasing patents}
\footnotesize \begin{quote}\itshape In this chapter, we are briefly analyzing
the effect of patent clauses in open source licenses -- not in general, but with
respect to the license fulfilling tasks they require, also known as the
'implicit acceptance of a patent use' by distributing open source software.
\end{quote}
\normalsize

At least the free software movement frowns on the existence of software
patents\footnote{For an early and elaborated description on the effects of
software patents based on the viewpoint of the free software movement
\cite[see][\nopage wp]{Stallman2001a}. This lecture seems to be given more than
onetime and seems also to have been printed lateron (\cite[cf.][\nopage
wp]{Stallman2002a}). Within the first decade of 2000, the focus switched to a
more political fight against software patents (\cite[cf.][\nopage
wp]{Stallman2004a}). But recently there seemed to appear another turn in dealing
with software patents: Not fighting against them, but mitigating their effects:
The proposal is '[...] (to legislate) that developing, distributing, or running
a program on generally used computing hardware does not constitute patent
infringement' (\cite[cf.][\nopage wp]{Stallman2012a})}. One of the most known
witnesses for that attitude is the GPL itself. Its preamble purports that
\enquote{[\ldots] any free program is threatened constantly by software
patents}\footcite[cf.][wp]{Gpl20OsiLicense1991a}. One can read that the open
source community fears three risks: First, they are apprehensive of people who
hjack the idea of a piece of open source software they do not have developed,
register a corresponding patent, and finally try to earn money by preventing the
use of the software or by envolving its users into patent
ligitations\footcite[cf.][234]{JaeMet2011a}. Second, they fear a bramble of
general software patents which practically prohibits to develop open source
software legally\footcite[cf.][234]{JaeMet2011a}. Third, they anticipate the
possibility that (not quite benevolent) open source developers could try to
register patents for undermining the open source
principles\footcite[cf.][235]{JaeMet2011a}.

Howsoever, regardless wether one tries to fight against software patents or not,
software patents have come true. To act law-abidingly requires to manage the
constraints of patents properly. Open source licenses know and respect this
necessity. Moreover, at least some of them try to manage the effect of software
patents by specific patent clauses\footcite[pars pro toto cf.][\nopage wp.
§3]{Apl20OsiLicense2004a} or by several sentences distributed in the license
text\footcite[pars pro toto cf.][\nopage wp.]{Epl10OsiLicense2005a}. But why
does the OSLiC have to deal with this topic, if the OSLiC does not want to
participate in general discussions?

In opposite to the other conditions of the open source licenses, their patent
clauses or propositions in general do not directly refer to a specific set of
actions which has to be executed for acting in accordance to the licenses. Open
source patent clauses normally do not join in the game 'paying by doing'. So,
actually, it does not seem to be not necessary to mention the patent clauses,
here.

Unfortunately, although the patent clauses do not directly say \emph{'do this or
that in these or those circumstances'}, some of them nevertheless trigger side
effects which evoke that the distributors of open source software implictly
having already something done if they actually are distributing a piece of open
source software. This implicit effect makes it necessary to deal with the patent
clauses even in an only pragmatic OSLIC.

Patent clauses in open source licenses can have two different directions of
impact. They use two methods to protect the users of the open source software --
and sometimes these methods are combined:

\begin{itemize}
  \item First, an open source license can assure that all contributors /
  distributors to / of a piece of open source software grant to all users /
  recipients not only the right to use the open source software itself, but
  automatically and implicitly also the right to use all those patents --
  belonging to the contributors / distributors -- which as patents are necessary
  to use the software legally\footnote{There might arise a legal discussion
  whether a distributor who does not contribute to the software development
  really imlictly also has to grant the necessary rights of his patent
  portfolio. The OSLiC doesn't want to participate in this discussion. We take a
  simple and pragmatic position: for being sure that you are acting according to
  an open source license with such a patent clause you should simply assume that
  you have to do so. If this default position is not reasonable for you it might
  be a good idea to consult legal experts which -- perhaps -- may find another
  way for you to use the software legally.}. So, let us - a little simplifying
  and therefore only on the following few pages -- name such licenses the
  \emph{granting licenses}.
  \item Second, an open source license can try to automatically terminate the
  right to use, to modify , and to distribute the software if its user litigates
  against any of the contributors / distributors with respect to a software
  patent. That can be seen as a revocation of earlier granted rights. So, let us
  name these license the \emph{revoking licenses}.
\end{itemize}

Later on, we will summarize the concrete patent clauses of all the licenses
discussed in the OSLiC as a proof for the following classification:


\begin{center}

\begin{tikzpicture}
\label{LICTAX}
\small

\node[ellipse,minimum height=8.5cm,minimum width=14cm,draw,fill=gray!10] (l0100) at (6.8,6.8)
{  };

\draw [-,dotted,line width=0pt,white,
    decoration={text along path,
              text align={center},
              text={|\itshape|open source licenses}},
              postaction={decorate}] (-0.8,6.5) arc (218:322:9.5cm);
              
\node[ellipse,minimum height=6.2cm,minimum width=4.4cm,draw,fill=gray!20] (l0100) at (2.6,6.8)
{  };

\draw [-,dotted,line width=0pt,white,
    decoration={text along path,
              text align={center},
              text={|\itshape| without patent clauses}},
              postaction={decorate}] (0.9,7.4) arc (180:0:1.8cm);

\node[circle,draw,text width=1cm, fill=gray!40, text centered] (l0101) at (2,8)
{  \footnotesize \bfseries \textit{?}};
\node[circle,draw,text width=1cm, fill=gray!40, text centered] (l0102) at (3.5,8)
{  \footnotesize \bfseries \textit{BSD}};
\node[circle,draw,text width=1cm, fill=gray!40, text centered] (l0103) at (2,6.5)
{  \footnotesize \bfseries \textit{MIT}};
\node[circle,draw,text width=1cm, fill=gray!40, text centered] (l0104) at (3.5,6.5)
{  \scriptsize \bfseries \textit{?}};
\node[circle,draw,text width=1cm, fill=gray!40, text centered] (l0105) at (2,5)
{  \footnotesize \bfseries \textit{?}};
\node[circle,draw,text width=1cm, fill=gray!40, text centered] (l0106) at (3.5,5)
{  \footnotesize \bfseries \textit{?}};

\node[ellipse,minimum height=6cm,minimum width=8.5cm,draw,fill=gray!20] (l0200) at (9.2,6.5)
{  };

\draw [-,dotted,line width=0pt,white,
    decoration={text along path,
              text align={center},
              text={|\itshape| with patent clauses}},
              postaction={decorate}] (7.5,8.5) arc (120:60:4cm);


\node[ellipse,minimum height=4.5cm,minimum width=6cm,draw,fill=gray!30] (l0220) at (10.4,6.5)
{  };
\node[ellipse,minimum height=4.5cm,minimum width=5cm,draw,fill=gray!30] (l0210) at (7.45,6.5)
{  };
\node[ellipse,minimum height=4.5cm,minimum width=6cm,draw] (l0220) at (10.4,6.5)
{  };


\draw [-,dotted,line width=0pt,white,
    decoration={text along path,
              text align={center},
              text={|\itshape| granting}},
              postaction={decorate}] (5.4,6.2) arc (180:0:2cm);

\node[circle,draw,text width=1cm, fill=gray!40, text centered] (l0211) at (6.7,7)
{  \footnotesize \bfseries \textit{?}};
\node[circle,draw,text width=1cm, fill=gray!40, text centered] (l0212) at (8.4,7)
{  \footnotesize \bfseries \textit{ApL}};
\node[circle,draw,text width=1cm, fill=gray!40, text centered] (l0213) at (6.7,5.5)
{  \footnotesize \bfseries \textit{?}};
\node[circle,draw,text width=1cm, fill=gray!40, text centered] (l0214) at (8.2,5.5)
{  \footnotesize \bfseries \textit{MS-PL}};

 
% line width=0pt,white,
\draw [-,dotted,line width=0pt,white,
    decoration={text along path,
              text align={center},
              text={|\itshape| revoking}},
              postaction={decorate}] (10.4,7) arc (180:0:1cm);

\node[circle,draw,text width=1cm, fill=gray!40, text centered] (l0221) at (11.4,7)
{  \footnotesize \bfseries \textit{??}};
\node[circle,draw,text width=1cm, fill=gray!40, text centered] (l0222) at (11.4,5.5)
{  \footnotesize \bfseries \textit{?}};


\end{tikzpicture}
\end{center}

But regardless of the final textual form a license is using to express its
granting or revoking positions, in any case one has to consider some aspects: 

\begin{itemize}
  
  \item Overall, one has to keep in mind that of course no licensor, contributor
  and/or distributor can release the right to use any patents he does not own --
  even not if he tries to release them by an open source patent clause.
  Implictly touched patents of third parties not having contributed to the
  development and/or participated in the distribution can never be implicitly
  and automatically released on the base of such an (open source) patent clause:
  no rights, no right to release\footnote{This is an important aspect which is
  sometimes not considered by programmers. Inside of DTAG we had a fruitful
  discussion evoked by Mr. Stephan Altmeyer who -- as patent lawyer -- patiently
  explained this constraint to us.}. Hence: even for those open source licenses
  which try to protect the users, finally the user itself must nevertheless
  ensure that he does not violate the patents of third parties being unwillingly
  touched by the way the code works\footnote{Sometimes, this problem of
  willingly or unwillingly violated third party patents is seen as a weakness of
  open source software. But that is not true. It is a weakness of every
  software. Even a commercial licensor (developer) has only the right to license
  the use of those patents he really owns or he has 'bought' for relicensing
  them. Moreover, also commercial licensors can willingly or unwillingly violate
  patents of other persons.}.
  
  \item In the context of a granting license, one has also to consider that
  contributing to and distributing of a piece of software implictly evokes that
  all patents of the contributor and/or distributor are 'given free' which are
  necessary to use the software as whole -- including the more or less deeply
  embedded libraries. So, if one wants to check wether some of the core patents
  of one's patent portfolio are afflicted by a patent clauses (and wether one
  therefore better should not use / distribute the corresponding piece of open
  source software), one should not forget to check the embedded libraries, too.
  
  \item Finally, one has to consider in the context of a granting license that
  its patent clause only releases the use of the patents in the meaning of
  'allowed to be used for enabling the use of the distributed software'. The
  patent clause does not release the patents generally. Thus, the threat of
  (unwillingly) releasing patents by open source software is not as large as
  sometimes feared: the use of the patent is only granted in combination with
  the software. On the one hand, you may not use the software without having the
  right to use the patent because the use of the patent is inherently necessary
  for using use the software. On the other hand, you are not allowed to use the
  patent without the software because the patent clause only refers to the use
  of that specific open source software licensed by the corresponding open
  source license.

  \item Summarized, one has to consider that the granting open source licenses
  automatically and implicitly enforce you to grant all the rights which are
  necessary to use the software legally. Open source contributors and
  distributors should know that\footnote{Again: It might be debatable wether
  this is also valid for the distributors which do not contribute anything to
  the development. That's a legal discussion the OSLiC do no wish to participate
  in. From the viewpoint of an open source user who only wants to have one
  reliable and secure way to use open sopurce software compliantly, one should
  perhaps assume that there is no difference.}.

  \item With respect to the revoking licenses, one has to consider that their
  patent clauses contain a negative conditions which may be read as
  interdictions. The OSLiC will integrate these conditions into specific
  'prohibits'-sections of its to-do lists.
  
  \item Finally one should mention that in some cases, the form of the
  revocation used by the revoking license refers to the use of the software, in
  other cases to the use of the patents. But nevertheless, one can reason that
  -- from the pragmatic viewpoint of a benevolent open source software user --
  also this second case of patent revocation implictly terminates the right to
  use the software: If the use of a patent is necessary to use a piece of
  software legally, one is not allowed to use the software without having the
  right to use the patent, too; and if the use of the patent is not necessary
  for using the software, then the patent is not covered by the patent clause.
  So, in any case, this kind of patent clauses seems to terminate the right to
  use / to distribute and/or to modify the software. Hence, single users as well
  as companies or organizations should also respect such patent clauses if they
  want to be sure to use open source software compliantly.
\end{itemize}

The OSLiC wants to support its readers not only to act according to the licenses
in general, but also according to its patent clause. Thus, we now briefly cite
and summarize the meaning of particular patent clauses:

\subsection{ApL statements concerning patents}\label{subsec:ApLPatentClause}

Titled by the headline \enquote{Grant of Patent License}, the Apache License 2.0
contains a specific patent clause being comprised of two very long and condensed
sentences\footcite[cf.][\nopage wp. §3]{Apl20OsiLicense2004a}. Outside of this
patent clause, the word \emph{patent} is only used once again -- for requiring
that one \enquote{[\ldots] must retain, in the (sources) [\ldots] all [\ldots]
patent [\ldots] notices [\ldots]}\footcite[cf.][\nopage wp.
§4.3]{Apl20OsiLicense2004a}.

The one core message of the ApL patent clause is the statement that
\enquote{[\ldots] each Contributor hereby grants to You a perpetual, worldwide,
non-exclusive, no-charge, royalty-free, irrevocable [\ldots] patent license to
make, have made, use, offer to sell, sell, import, and otherwise transfer the
Work [\ldots]}\footcite[cf.][\nopage wp. §3. \enquote{Contributor},
\enquote{Work} and \enquote{You} are defined §1: \emph{Contributor} refers to
the original licensor and to all others whose contributions have been
incorporated into the Work. The \emph{Work} denotes the result of the
development process regardless of its form. \emph{You} denote the
licensees.]{Apl20OsiLicense2004a}.

The second core message of the ApL patent clause is the statement that
\enquote{if You institute patent litigation against any entity [\ldots] alleging
that the Work [\ldots] constitutes [\ldots] patent infringement, then any patent
licenses granted to You [\ldots] shall terminate [\ldots]}\footcite[cf.][\nopage
wp. §3]{Apl20OsiLicense2004a}.

The third message of the ApL patent clause is the statement, that the
\enquote{[\ldots] license applies only to those patent claims licensable by such
Contributor that are necessarily infringed by their Contribution(s) alone or by
combination of their Contribution(s) with the Work to which such Contribution(s)
was submitted}\footcite[cf.][\nopage wp. §3]{Apl20OsiLicense2004a}.

Thus, the ApL is - as we are using to say in this chapter - a granting and a
revoking license: At first you are granted to use all patents of all
contributors which are necessary to use the software legally. But if you -- with
respect to the software -- install any litigation concerning an infringement of
patents, then the rights granted to you are revoked.


\subsection{EPL statements concerning patents [tbd]}

\subsection{EUPL statements concerning patents [tbd]}

\subsection{GPL statements concerning patents [tbd]}

\subsection{LGPL statements concerning patents [tbd]}
 
\subsection{MPL statements concerning patents [tbd]}         

\subsection{MS-PL statements concerning patents}\label{subsec:MsplPatentClause}

First, the MS-PL contains a statement, by which \enquote{[\ldots] each
contributor grants (the software users) a non-exclusive, worldwide, royalty-free
license under its licensed patents to make, have made, use, sell, offer for
sale, import, and/or otherwise dispose of its contribution in the software or
derivative works of the contribution in the software}\footcite[cf.][\nopage wp.
§2.B]{MsplOsiLicense2013a}. Second, the MS-PL says that \enquote{if you bring a
patent claim against any contributor[\ldots] your patent license from such
contributor to the software ends automatically}\footcite[cf.][\nopage wp.
§3.B]{MsplOsiLicense2013a}.

Thus, the MS-PL is - as we are using to say in this chapter - a granting and a
revoking license: At first you are granted to use all patents of all
contributors which are necessary to use the software legally. But if you -- with
respect to the software -- install any litigation concerning an infringement of
patents, then the rights granted to you are revoked.
  
  
\subsection{PGL statements concerning patents [tbd]}

\subsection{PHP statements concerning patents [tbd]}



% \bibliography{../../../bibfiles/oscResourcesEn}

% Telekom osCompendium 'for being included' snippet template
%
% (c) Karsten Reincke, Deutsche Telekom AG, Darmstadt 2011
%
% This LaTeX-File is licensed under the Creative Commons Attribution-ShareAlike
% 3.0 Germany License (http://creativecommons.org/licenses/by-sa/3.0/de/): Feel
% free 'to share (to copy, distribute and transmit)' or 'to remix (to adapt)'
% it, if you '... distribute the resulting work under the same or similar
% license to this one' and if you respect how 'you must attribute the work in
% the manner specified by the author ...':
%
% In an internet based reuse please link the reused parts to www.telekom.com and
% mention the original authors and Deutsche Telekom AG in a suitable manner. In
% a paper-like reuse please insert a short hint to www.telekom.com and to the
% original authors and Deutsche Telekom AG into your preface. For normal
% quotations please use the scientific standard to cite.
%
% [ Framework derived from 'mind your Scholar Research Framework' 
%   mycsrf (c) K. Reincke 2012 CC BY 3.0  http://mycsrf.fodina.de/ ]
%


%% use all entries of the bibliography
%\nocite{*}

\section{Excursion: What is a 'Derivative Work' - the basic idea of open source}
\footnotesize
\begin{quote}\itshape
This chapter briefly discusses existing attempts to define the derivated works of
technical aspects, like dynamical or statical linking or not. We will
prove that linking can not deliver a definite criteria: 1) modules are only
unzipped libraries. 2) you can distribute software as modules added by a script,
which statically(sic!) links all modules before executing the program. 3) The
criteria of pipe-communication is good, but not sufficient. 4) All these
attempts do not match the constituting features of script languages. Therefore we
will follow Moglen(?) and will argue from the viewpoint of a developer: it is
only a question of a function, method or anything else which calls (jumps into)
a piece of code which has been licensed by a license protecting
on-top-developments and you have a derivated work.
\end{quote}
\normalsize
\ldots

As first version, we present an outline of the arguing structure this chapter will
later present in a more narrative way.

\begin{description}
  \item[The meaning 'derivative work' must be known!] Many open source licenses
  are using the the term 'derivative work'  [cite the sources], either direct or
  indirect in form of the work 'modification' [Write a table as survey]. And
  nearly all licenses, which are using the term 'derivative work' etc., are
  linking tasks which must be executed to comply with the corresponding license,
  to the precondition, that something is a derivative work.
  [tabke survey]. \textbf{Hence, for acting in accordance to such a license, it
  has to be known, what a derivate work is}
  \item[Unfortunately the meaning is not clearly fixed]. The exist some
  different readings of the term 'derivative work' [specify the differences and
  cite the sources] \textbf{Hence, it is not as clear as wished what a derivtive
  work is}
  \item[Hence lets argue from the viewpoint of a benevolent developer]: Open
  source licenses are written for software developers, mostly to preserve their
  freedom, to develop software. And sometimes these licenses are also written by
  software developers -- or at least by the assistence of. So, one should be
  able to answer the question under which circumstances a piece of software is a
  'deverivate work' of another piece of software on the base of two principles:
  \begin{itemize}
  \item Let us argue on the base of a benevolent neutral software developer
  without hidden interests or a hidden agenda.
  \item In case of doubts let us preferably assume that the two pieces
  interrelate as source and derivative work -- so that the OSLiC rather recommends
  to execute the required tasks than to put them away.
\end{itemize}
\end{description}

This are our rules by which the OSLiC decides to take something as derivative Work:
\label{sec:BenevolentDerivativeWorkUnderstanding}

\begin{description}
  \item[Copy-Case] Copying a piece of code from a source file and pasting it
  into a target file makes the target file a derivatve work of the source
  file\footnote{ Be careful: this case must still be distinguished from the case
  of an automatically inclusion (header files, script libraries) during the
  compilation / execution: Header files allon should not evekoe a derivative work.}.
  \item[Modify-Case] Inserting any new content or deleting any existing content
  of a source file makes the resulting target file being a derivate work of the
  wource file.
  \item[Call-Case] Inserting into a target file the call of function which is
  defined inside of and delivered by a sourcefile makes the target file depending
  on the source file and therefore a derivative work of the delivering source file.
\end{description}

And here are some applications of the rules

\begin{itemize}
  \item Combining the copy-case and the modify-case on the base of two
  different source files make the resulting target file being a derivative work
  of the two source files (follows from copy-case and modify-case)
  \item \ldots
\end{itemize}


%\bibliography{../../../bibfiles/oscResourcesEn}

% Telekom osCompendium 'for beeing included' snippet template
%
% (c) Karsten Reincke, Deutsche Telekom AG, Darmstadt 2011
%
% This LaTeX-File is licensed under the Creative Commons Attribution-ShareAlike
% 3.0 Germany License (http://creativecommons.org/licenses/by-sa/3.0/de/): Feel
% free 'to share (to copy, distribute and transmit)' or 'to remix (to adapt)'
% it, if you '... distribute the resulting work under the same or similar
% license to this one' and if you respect how 'you must attribute the work in
% the manner specified by the author ...':
%
% In an internet based reuse please link the reused parts to www.telekom.com and
% mention the original authors and Deutsche Telekom AG in a suitable manner. In
% a paper-like reuse please insert a short hint to www.telekom.com and to the
% original authors and Deutsche Telekom AG into your preface. For normal
% quotations please use the scientific standard to cite.
%
% [ Framework derived from 'mind your Scholar Research Framework' 
%   mycsrf (c) K. Reincke 2012 CC BY 3.0  http://mycsrf.fodina.de/ ]
%


%% use all entries of the bibliography
%\nocite{*}

\section{Excursion: The Problem of License Compatibility}
\footnotesize
\begin{quote}\itshape
Here we discuss the often neglected or only loosely touched problem of combining
differently licensed software. We will hint to the Exclusion-List of the Free
software foundation; we will hint to the eclipse / GPL-plugin problem; we will
mention the recent discussion whether the kernel requires to license the
complete Android as GPL; and finally we will discuss the just now published, short
analysis of Jaeger and Metzger presenting a combining matrix which seems to fall
into their lap. We ourselves will argue that question can simply be answered:
only if you embed two libraries which both are licensed by an on
on-top-development protecting license and if these both license require the
licensing of the derivated work by different licenses then you have a problem.
In all other case which we will describe an list there is no problem.
\end{quote}
\normalsize
\ldots

%\bibliography{../../../bibfiles/oscResourcesEn}

% Telekom osCompendium 'for being included' snippet template
%
% (c) Karsten Reincke, Deutsche Telekom AG, Darmstadt 2011
%
% This LaTeX-File is licensed under the Creative Commons Attribution-ShareAlike
% 3.0 Germany License (http://creativecommons.org/licenses/by-sa/3.0/de/): Feel
% free 'to share (to copy, distribute and transmit)' or 'to remix (to adapt)'
% it, if you '... distribute the resulting work under the same or similar
% license to this one' and if you respect how 'you must attribute the work in
% the manner specified by the author ...':
%
% In an internet based reuse please link the reused parts to www.telekom.com and
% mention the original authors and Deutsche Telekom AG in a suitable manner. In
% a paper-like reuse please insert a short hint to www.telekom.com and to the
% original authors and Deutsche Telekom AG into your preface. For normal
% quotations please use the scientific standard to cite.
%
% [ Framework derived from 'mind your Scholar Research Framework' 
%   mycsrf (c) K. Reincke 2012 CC BY 3.0  http://mycsrf.fodina.de/ ]
%


%% use all entries of the bibliography
%\nocite{*}

\section{Excursion: Open Source Software and Money [tbd]}
\footnotesize
\begin{quote}\itshape
Here we will shortly discuss ways in which money and Open Source is no problem.
\end{quote}
\normalsize
\ldots


%\bibliography{../../../bibfiles/oscResourcesEn}



%%%%%%%%%%%%%%%
% Telekom osCompendium 'for being included' snippet template
%
% (c) Karsten Reincke, Deutsche Telekom AG, Darmstadt 2011
%
% This LaTeX-File is licensed under the Creative Commons Attribution-ShareAlike
% 3.0 Germany License (http://creativecommons.org/licenses/by-sa/3.0/de/): Feel
% free 'to share (to copy, distribute and transmit)' or 'to remix (to adapt)'
% it, if you '... distribute the resulting work under the same or similar
% license to this one' and if you respect how 'you must attribute the work in
% the manner specified by the author ...':
%
% In an internet based reuse please link the reused parts to www.telekom.com and
% mention the original authors and Deutsche Telekom AG in a suitable manner. In
% a paper-like reuse please insert a short hint to www.telekom.com and to the
% original authors and Deutsche Telekom AG into your preface. For normal
% quotations please use the scientific standard to cite.
%
% [ Framework derived from 'mind your Scholar Research Framework' 
%   mycsrf (c) K. Reincke 2012 CC BY 3.0  http://mycsrf.fodina.de/ ]
%


%% use all entries of the bibliography
%\nocite{*}

\chapter{Open Source Use Cases: Concept and Taxonomy}\label{sec:OSUCdeduction}

\footnotesize \begin{quote}\itshape This chapter establishes our concept of
\emph{open source use cases} as a classification system for to-do lists. The
conditions of a specific license, in the context of a par\-ti\-cu\-lar
\emph{open source use case}, shall be satisfiable by following the corresponding
to-do list. Additionally this chapter introduces a taxonomy for these \emph{open
source use cases}. Later on, this taxonomy will organize the \emph{Open Source
Use Case Finder}.
\end{quote}
\normalsize{}

After all these introductory remarks, we can summarize our idea. We know that
the right to use open source software depends on the tasks required by the open
source licenses. As opposed to commercial licenses, you can not buy the right to
use a piece of open source software by paying money. It is embedded into the
\emph{Open Source Definition} that the right to use the software may not be
sold. The OSD states first that an open source license may \enquote{[\ldots]
not restrict any party from selling or giving away the software as a component
of (any) aggregate software distribution}, and adds second in the same context
that an open source license \enquote{[\ldots] shall not require a royalty or
other fee for such sale}\footcite[cf.][\nopage wp §1]{OSI2012a}.

However, it would be wrong to conclude that you are automatically allowed to use
open source software without any service in return: generally you have to do
something to gain the right to use the software. In other words: open source
software is covered by the idea of ’paying by doing’. Accordingly, open source
li\-cen\-ses describe specific circumstances under which the user must execute
some tasks in order to be compliant with the licenses. So, if we want to offer
to-do lists for fulfilling license conditions, we must consider these tasks and
circumstances.

In practice, such circumstances are not linear and simple. They contain
combinations of (sometimes context sensitive) conditions which can be grouped
into classes of tokens. Such a class of tokens might denote a feature of the
software itself---such as being an application or a library. Or it can refer to
the circumstances of using the software, such as 'using the software only for
yourself' or 'distributing the software also to third parties'.

At the end, we want to determine a set of specific OSUCs---the \emph{open source
use cases}. And we want to deliver for each of these OSUCs and for each of the
considered open source licenses one list of actions which fulfills the license
in that context\footnote{Fortunately, sometimes one task list fulfills the
conditions of more than one use case---a welcome reduction of complexity}.

Such an \emph{open source use case} shall be considered as a set of tokens
describing the circumstances of a specific usage. Hence, to begin, we must
specify the relevant classes of tokens, before we can determine the valid
combinations of these tokens---our \emph{open source use cases}. Finally, based
on the tokens, we generate a taxonomy in the form of a tree. This tree will
become the base of the \emph{Open Source Use Case Finder} which will be offered
in the next chapter, and which leads you to your specific OSUC by evaluating
just a few questions and answers.

There are only a handful of tokens which are relevant to the circumstances of
open source software licenses:

\label{OsucTokens}
\begin{itemize}
  \item The \textbf{\underline{type} of the open source software}: On the one
  hand, we regard code snippets, modules, libraries and plugins, and on the
  other hand, autonomous applications, programs and servers. We will take the
  word ’snimolis’ for the first set, and ’proapses’ for the second. This is
  necessary, as we are not only talking about libraries and applications in the
  everyday sense, but rather in the broadest sense\footnote{Of course, our newly
  introduced concepts of 'snimoli' and 'proapse' are not absolutely one of the
  most elegant words. So, initially we tried to talk about 'applications' and
  'libraries', although in our context these words should denote more, than they
  traditionally do. But we couldn't minimize the irritations of our
  interlocutors. Too often we had to remind them that we were not talking about
  applications and libraries in the strict sense of the words. Finally we
  decided to find our own words---and to stay open for better proposals ;-) }.
  More specifically, we will ask you, whether the open source software you want
  to use, is an includable code snippet, a linkable module or library, or a
  loadable plugin, or whether it is an autonomous application or server which
  can be executed or processed. In the first case, the answer should be 'it is a
  \underline{snimoli}', in the second 'it is a \underline{proapse}'.

  \item The \textbf{\underline{state} of the open source software}: It might be
  used exactly as one has received it. Or it can be modified, before being used.
  More specifically, we will ask you, whether you want to leave the open source
  software as you have received it, or whether you want to modify it before
  using and/or distributing it to 3rd parties. In the first case, the answer
  should be '\underline{unmodified}', in the second '\underline{modified}'.
  
  \item The \textbf{usage \underline{context} of the open source
  software}: On the one hand you might use the received open source software as a
  readily prepared application. On the other hand you might embed the received
  open source into a larger application as one of its components. More
  specifically, we will ask you, whether you are using the open source
  software as an autonomous piece of software, or whether you are using it as an
  embedded part of a larger, more complex piece of software. In the first case,
  the answer should be '\underline{independent}', in the second
  '\underline{embedded}'.
  
  \item The \textbf{\underline{recipient} of the open source software}:
  Sometimes you might wish to use the received open source software only for
  yourself. In other cases you might intend to hand over the software (also) to
  other people. More specifically, we will ask you, whether you are going to use
  the open source software only for yourself, or whether you plan to
  (re)distribute it (also) to third parties. In the first case, the answer
  should be '\underline{4yourself}', in the second '\underline{2others}'.
 
  \item The \textbf{\underline{form} of the distributed files}: Many licenses
  also draw a distinction between distributing the software as sources and
  distributing the files as binaries. In this case, we will ask you, whether you
  want to distribute the software in the form of binaries or as source code. In
  the first case, the answer should be '\underline{binaries}', in the second
  '\underline{sources}'
  
  \item The kind of the \textbf{\underline{ioAccess} of the executed program}:
  At least one license draws a distinction between an open source based work
  offering only local access to its io data and an open source based work
  distributing its io data via internet. In the first case, the answer should be
  '\underline{onlyLocally}', in the second '\underline{viaInternet}'
\end{itemize}

From a more programmatic point-of-view, we can summarize these tokens as
follows:

\begin{itemize}
  \item \texttt{type::snimoli} \emph{or} \texttt{type::proapse}
  \item \texttt{state::unmodified} \emph{or} \texttt{state::modified}
  \item \texttt{context::independent} \emph{or} \texttt{context::embedded}
  \item \texttt{recipient::4yourself} \emph{or} \texttt{recipient::2others}
  \item \texttt{form::binaries} \emph{or} \texttt{form::sources}
  \item \texttt{ioAccess::onlyLocally} \emph{or} \texttt{ioAccess::viaInternet}
\end{itemize}

We have already defined the \emph{open source use case} as the combination of
these tokens. If we simply combine all these tokens of all these classes with
all the tokens of the other classes\footnote{in the sense of the cross product
TYPE $\times$ STATE $\times$ CONTEXT $\times$ RECIPIENT $\times$ FORM. In some
earlier versions of the \oslic{}, we also asked whether you are going to combine
or to embed the open source software with other software components by linking
them statically or dynamically, or by textually including (parts of) the open
source software into your larger product. Meanwhile, we clearly discovered that
it is unnecessary to increase the complexity by the results of this question.
For Details $\rightarrow$ \oslic{} p.\ \pageref{sec:LinkingSecondary}}, we get
$2 \cdot 2 \cdot 2 \cdot 2 \cdot 2 \cdot 2 = 62$ sets of tokens---or 62
\emph{open source use cases}. Fortunately, some of the generated sets are
invalid from an empirical or logical view, and some of these sets are context
sensitive:

\begin{enumerate}
  \label{InvalidFinderTokenCombinations}
  \item If you already have specified that the used open source software is a
  \emph{proapse}---an autonomous program, an application, or a server---then
  your answer implies that the software is used independently and is not
  embedded with other components into a larger unit. But if you have specified
  that the used open source software is a \emph{snimoli}---a snippet of
  code, a module, a plugin, or a library---then it can indeed be used as an
  embedded component of a constructed larger application or server, or it can be
  used independently in case you 'only' re-distribute it to 3rd. parties.
  
  \item If you already have specified that the used open source software is a
  \emph{snimoli}---a snippet of code, a module, a plugin, or a library---and
  that this \emph{snimoli} shall be used only by yourself (not distributed to
  other 3rd.\ parties) then your answer must also imply that this \emph{snimoli}
  is used in combination, as an embedded part of a larger unit. A library can
  not be used autonomously, without using it as a component of another
  application. In this case, it would simply sit on the disk and would do
  nothing more than occupying space.
  
  \item To enquire the \emph{form} of the distributed files is only relevant if you
  have decided to distribute the software to other recipients \emph{2others}.
  
  \item With respect to the one license using the type of ioAccess as a
  discriminator, it is only relevant to enquire the type of the ioAccess if you
  either locally executes a \emph{modified} open source program \emph{4yourself}
  or if you locally executes a program \emph{4yourself}, which uses an
  \emph{embedded} open source component, regardless whether it has been modified
  or not.
  
\end{enumerate}

Does this sound complex? We thought so, too. We spent much time explaining these
constraints to ourselves, and only when we had transposed all the combinations
and rules into a tree, the situation became clearer. The following diagram
summarizes the main results of our investigation\footnote{ Each of the invalid
use cases (= sets of tokens) [for details s. p.\
\pageref{InvalidFinderTokenCombinations}] is marked by an \lightning{} and leads
to an empty set (= $\varnothing$). We are using the word 'invalid' a little
ambigiuosly: A combination of values is invalid, if it is empirically
impossible, to combine the features or if it is irrelavant to subclassify a
concept by the added features. Particularly:
\begin{itemize}
  \item A proapse can not be embedded into another software unit, also
  containing a main-function.
  \item Using a software library only for yourself and independently (not in
  combination with larger software unit), is like having an unused heap of bytes
  on your disc.
  \item To discriminate between sources and binaries is only valid in case of
  distributing software.
  \item To discriminate between an executed program with an only locally based
  io access and that with an internet based io access is only relevant, if you
  are using the software for yourself what implies to execute it.
\end{itemize}
 . 

}::

\tikzstyle{StartNode} = [font=\tiny, ellipse, draw, fill=gray!5,
text width=1em, text centered, minimum height=1em]

\tikzstyle{TypeNode} = [font=\tiny, rectangle, draw, fill=gray!10,
text width=1cm, text centered, rounded corners, minimum height=1em]

\tikzstyle{VLabelNode} = [font=\tiny, draw, rectangle,  text
width=1.4cm, fill=white]

\tikzstyle{VNibelNode} = [font=\tiny, draw, rectangle,  text
width=1cm, ]

\tikzstyle{VdTupelNode} = [font=\tiny, draw, rectangle, dotted, text
width=1.4cm, ]

\tikzstyle{VmTupelNode} = [font=\tiny, draw, rectangle, dotted, text
width=2.3cm, ]

\tikzstyle{VQuadrupelNode} = [font=\tiny, draw, rectangle, dotted, text
width=3cm, ]

\tikzstyle{VsQuadrupelNode} = [font=\tiny, draw, rectangle, dotted, text
width=2.7cm, ]


\tikzstyle{VlTupelNode} = [font=\tiny, draw, rectangle, dotted, text
width=3.8cm, ]


\tikzstyle{VmTupelLeaf} = [font=\tiny, draw, rectangle, dashed, text
width=2.7cm, ]

\tikzstyle{VlTupelLeaf} = [font=\tiny, draw, rectangle, dashed, text
width=3.8cm, ]

\tikzstyle{OsucNode} = [font=\tiny, rectangle, draw, fill=gray!20,
text width=1.4cm, text centered, rounded corners, minimum height=1em]



\tikzstyle{arrow} = [draw, -latex']
\tikzstyle{edge} = [draw]


\begin{tikzpicture}

% drwa lines at first:




% classification types and their values

\node[TypeNode] (tForm) at (2,7.4) {\textit{form?}};

\node[VNibelNode] (vSources) at (3,6.7) {sources};
\node[VNibelNode] (vBinaries) at (2.5,6.2) {binaries};

\node[TypeNode] (tType) at (0,15.6) {\textit{type?}};
\node[VNibelNode] (vProapse) at (2.8,15.9) {proapse};
\node[VNibelNode] (vSnimoli) at (2.8,15.4) {snimoli};

\node[TypeNode] (tState) at (0.8,12.6) {\textit{state?}};
\node[VLabelNode] (vUnmodified) at (2.6,12.9) {unmodified};
\node[VLabelNode] (vModified) at (2.6,12.4) {modified};

\node[TypeNode] (tContext) at (1.0,10.6) {\textit{context?}};
\node[VLabelNode] (vIndependent) at (2.7,10.9) {independent};
\node[VLabelNode] (vEmbedded) at (2.7,10.4) {embedded};

\node[TypeNode] (tRecipient) at (0.8,8.6) {\textit{recipient?}};
\node[VNibelNode] (v4yourself) at (2.3,8.9) {4yourself};
\node[VNibelNode] (v2others) at (2.3,8.4) {2others};

\node[TypeNode] (tIoAccess) at (0,1.6) {\textit{ioAccess?}};
\node[VLabelNode] (vViaInternet) at (2,1.9) {viaInternet};
\node[VLabelNode] (vOnlyLocal) at (2,1.4) {onlyLocal};

\node[StartNode] (vStart) at (0,10) {\textbf{\#}};

% value collictions defining the osucs

\node[OsucNode] (o1) at (12,19.8) {OSUC-01};
\node[VlTupelLeaf, color=blue] (vo1) at (15,19.8)
{\{proapse, independent,\\4yourself, unmodified\}};

\node[OsucNode] (o2s) at (12,19) {OSUC-02S};
\node[VlTupelLeaf, color=blue] (vo2s) at (15,19)
{\{proapse, independent,\\2others, unmodified, sources\}};

\node[OsucNode] (o2b) at (12,18.2) {OSUC-02B};
\node[VlTupelLeaf, color=blue] (vo2b) at (15,18.2)
{\{proapse, independent,\\2others, unmodified, binaries\}};

\node[OsucNode] (o3l) at (12,17.4) {OSUC-03L};
\node[VlTupelLeaf, color=blue] (vo3l) at (15,17.4)
{\{proapse, independent,\\4yourself, modified, onlyLocal\}};

\node[OsucNode] (o3n) at (12,16.6) {OSUC-03N};
\node[VlTupelLeaf, color=blue] (vo3n) at (15,16.6)
{\{proapse, independent,\\4yourself, modified, viaInternet\}};

\node[OsucNode] (o4s) at (12,15.8) {OSUC-04S};
\node[VlTupelLeaf, color=blue] (vo4s) at (15,15.8)
{\{proapse, independent,\\2others, modified, sources\}};

\node[OsucNode] (o4b) at (12,15) {OSUC-04B};
\node[VlTupelLeaf, color=blue] (vo4b) at (15,15)
{\{proapse, independent,\\2others, modified, binaries\}};

\node[OsucNode] (o5s) at (13,12.2) {OSUC-05S};
\node[VmTupelLeaf, color=blue] (vo5s) at (15.5,12.2)
{\{snimoli, independent,\\2others, unmodified,\\sources\}};

\node[OsucNode] (o5b) at (13,11.1) {OSUC-05B};
\node[VmTupelLeaf, color=blue] (vo5b) at (15.5,11.1)
{\{snimoli, independent,\\2others, unmodified,\\binaries\}};

\node[OsucNode] (o6l) at (13,10) {OSUC-06L};
\node[VmTupelLeaf, color=blue] (vo6l) at (15.5,10)
{\{snimoli, embedded,\\4yourself, unmodified,\\onlyLocal\}};

\node[OsucNode] (o6n) at (13,8.9) {OSUC-06N};
\node[VmTupelLeaf, color=blue] (vo6l) at (15.5,8.9)
{\{snimoli, embedded,\\4yourself, unmodified,\\viaInternet\}};

\node[OsucNode] (o7s) at (13,7.8) {OSUC-07S};
\node[VmTupelLeaf, color=blue] (vo7s) at (15.5,7.8)
{\{snimoli, embedded,\\2others, unmodified,\\sources\}};

\node[OsucNode] (o7b) at (13,6.7) {OSUC-07B};
\node[VmTupelLeaf, color=blue] (vo7b) at (15.5,6.7)
{\{snimoli, embedded,\\2others, unmodified,\\binaries\}};

\node[OsucNode] (o8s) at (13,5.6) {OSUC-08S};
\node[VmTupelLeaf, color=blue] (vo8s) at (15.5,5.6)
{\{snimoli, independent,\\2others, modified,\\sources\}};

\node[OsucNode] (o8b) at (13,4.4) {OSUC-08B};
\node[VmTupelLeaf, color=blue] (vo8b) at (15.5,4.4)
{\{snimoli, independent,\\2others, modified,\\binaries\}};

\node[OsucNode] (o9l) at (13,3.3) {OSUC-09L};
\node[VmTupelLeaf, color=blue] (vo9l) at (15.5,3.3)
{\{snimoli, embedded,\\4yourself, modified,\\onlyLocal\}};

\node[OsucNode] (o9n) at (13,2.2) {OSUC-09N};
\node[VmTupelLeaf, color=blue] (vo9n) at (15.5,2.2)
{\{snimoli, embedded,\\4yourself, modified,\\viaInternet\}};

\node[OsucNode] (o10s) at (13,1.1) {OSUC-010S};
\node[VmTupelLeaf, color=blue] (vo10s) at (15.5,1.1)
{\{snimoli, embedded,\\2others, modified,\\sources\}};

\node[OsucNode] (o10b) at (13,0) {OSUC-010B};
\node[VmTupelLeaf, color=blue] (vo10b) at (15.5,0)
{\{snimoli, embedded,\\2others, modified,\\binaries\}};

% concepts directly referring osucs

\node[VQuadrupelNode] (v4osuc1) at (5.4,19.8)
{\{proapse, 4yourself,\\ independent, unmodified\}};

\node[VQuadrupelNode] (v4osuc2) at (7,18.7)
{\{proapse, 2others,\\ independent, unmodified\}};

\node[VsQuadrupelNode] (v4osuc3) at (8.3,17.6)
{\{proapse, 4yourself,\\ independent, modified\}};

\node[VsQuadrupelNode] (v4osuc4) at (9.6,16.5)
{\{proapse, 2others,\\ independent, modified\}};

\node[VlTupelLeaf, color=red] (vLightning1) at (10.4,14.3)
{\{proapse, 4yourself, embedded,\\ \{unmodified, modified\}\} \lightning{}};

\node[VlTupelLeaf, color=red] (vLightning2) at (10.4,13.6)
{\{proapse, 2others, embedded,\\ \{unmodified, modified\}\} \lightning{}};

\node[VlTupelLeaf, color=red] (vLightning3) at (10.4,12.9)
{\{snimoli, 4yourself, independent,\\ \{unmodified, modified\}\} \lightning{}};

\node[VQuadrupelNode] (v4osuc5) at (10.2,11.8)
{\{snimoli, 2others,\\ independent, unmodified\}};

\node[VsQuadrupelNode] (v4osuc6) at (9.5,10.8)
{\{snimoli, 4yourself,\\ embedded, unmodified\}};

\node[VsQuadrupelNode] (v4osuc7) at (9.2,9.8)
{\{snimoli, 2others,\\ embedded, unmodified\}};

\node[VsQuadrupelNode] (v4osuc8) at (8.2,8.8)
{\{snimoli, 4yourself,\\ embedded, modified\}};

\node[VsQuadrupelNode] (v4osuc9) at (7.3,7.6)
{\{snimoli, 2others,\\ independent, modified\}};

\node[VsQuadrupelNode] (v4osuc10) at (6.2,6.4)
{\{snimoli, 2others,\\ embedded, modified\}};

%meta nodes reffering concepts reffering osucs


\node[VmTupelNode] (v2otherSources) at (6,5.4) {\{2others, sources\}};
\node[VmTupelNode] (v2otherBinaries) at (6,4.8) {\{2others, binaries\}};

\node[VlTupelNode] (v4proapseViaInternet) at (7.2,4.0)
  {\{proapse, 4yourself,\\ modified, viaInternet\}};
\node[VlTupelNode] (v4proapseOnlyLocal) at (7.2,3.2)
  {\{proapse, 4yourself,\\ modified, onlyLocal\}};

\node[VlTupelLeaf, color=red] (vLightning4) at (7.2,2.4)
  {\{proapse, 4yourself, unmodified,\\ \{viaInternet, onlyLocal\} \} \lightning{}};

\node[VlTupelNode] (v4snimoliViaInternet) at (7.2,1.6)
  {\{snimoli, 4yourself,\\ embedded, viaInternet\}};
\node[VlTupelNode] (v4snimoliOnlyLocal) at (7.2,0.8)
  {\{snimoli, 4yourself,\\ embedded, onlyLocal\}};

\node[VlTupelLeaf, color=red] (vLightning5) at (7.2,0)
  {\{snimoli, 4yourself, independent,\\ \{viaInternet, onlyLocal\} \} \lightning{}};

% entry paths

\path [edge] (vStart) -- (tType);
\path [edge] (tType) -- (vProapse);
\path [edge] (tType) -- (vSnimoli);

\path [edge] (vStart) -- (tState);
\path [edge] (tState) -- (vUnmodified);
\path [edge] (tState) -- (vModified);

\path [edge] (vStart) -- (tContext);
\path [edge] (tContext) -- (vIndependent);
\path [edge] (tContext) -- (vEmbedded);

\path [edge] (vStart) -- (tRecipient);
\path [edge] (tRecipient) -- (v4yourself);
\path [edge] (tRecipient) -- (v2others);

\path [edge] (v2others) -- (tForm);
\path [edge] (tForm) -- (vSources);
\path [edge] (tForm) -- (vBinaries);

\path [edge] (vStart) -- (tIoAccess);
\path [edge] (tIoAccess) -- (vViaInternet);
\path [edge] (tIoAccess) -- (vOnlyLocal);


% middle paths
\path [edge] (vProapse) -- (v4osuc1);
\path [edge] (vProapse) -- (v4osuc2);
\path [edge] (vProapse) -- (v4osuc3);
\path [edge] (vProapse) -- (v4osuc4);
\path [edge] (vProapse) -- (vLightning1);
\path [edge] (vProapse) -- (vLightning2);
\path [edge] (vSnimoli) -- (vLightning3);
\path [edge] (vSnimoli) -- (v4osuc5);
\path [edge] (vSnimoli) -- (v4osuc6);
\path [edge] (vSnimoli) -- (v4osuc7);
\path [edge] (vSnimoli) -- (v4osuc8);
\path [edge] (vSnimoli) -- (v4osuc9);
\path [edge] (vSnimoli) -- (v4osuc10);

\path [edge] (vUnmodified) -- (v4osuc1);
\path [edge] (vUnmodified) -- (v4osuc2);
\path [edge] (vModified) -- (v4osuc3);
\path [edge] (vModified) -- (v4osuc4);
\path [edge] (vUnmodified) -- (vLightning1);
\path [edge] (vUnmodified) -- (vLightning2);
\path [edge] (vUnmodified) -- (vLightning3);
\path [edge] (vUnmodified) -- (v4osuc5);
\path [edge] (vUnmodified) -- (v4osuc6);
\path [edge] (vUnmodified) -- (v4osuc7);
\path [edge] (vModified) -- (v4osuc8);
\path [edge] (vModified) -- (v4osuc9);
\path [edge] (vModified) -- (v4osuc10);

\path [edge] (vIndependent) -- (v4osuc1);
\path [edge] (vIndependent) -- (v4osuc2);
\path [edge] (vIndependent) -- (v4osuc3);
\path [edge] (vIndependent) -- (v4osuc4);
\path [edge] (vIndependent) -- (vLightning3);
\path [edge] (vEmbedded) -- (vLightning1);
\path [edge] (vEmbedded) -- (vLightning2);
\path [edge] (vIndependent) -- (v4osuc5);
\path [edge] (vEmbedded) -- (v4osuc6);
\path [edge] (vEmbedded) -- (v4osuc7);
\path [edge] (vEmbedded) -- (v4osuc8);
\path [edge] (vIndependent) -- (v4osuc9);
\path [edge] (vEmbedded) -- (v4osuc10);

\path [edge] (v4yourself) -- (v4osuc1);
\path [edge] (v2others) -- (v4osuc2);
\path [edge] (v4yourself) -- (v4osuc3);
\path [edge] (v2others) -- (v4osuc4);
\path [edge] (v2others) -- (v4osuc5);
\path [edge] (v4yourself) -- (v4osuc6);
\path [edge] (v2others) -- (v4osuc7);
\path [edge] (v4yourself) -- (v4osuc8);
\path [edge] (v2others) -- (v4osuc9);
\path [edge] (v4yourself) -- (v4osuc10);


\draw[->] (vSources) to [out=0,in=180] (v2otherSources);
\draw[->] (vBinaries) to [out=0,in=180] (v2otherBinaries);

\draw[->] (vViaInternet) to [out=0,in=180] (v4proapseViaInternet);
\draw[->] (vViaInternet) to [out=0,in=180] (vLightning4);
\draw[->] (vViaInternet) to [out=0,in=180] (v4snimoliViaInternet);
\draw[->] (vViaInternet) to [out=0,in=180] (vLightning5);

\draw[->] (vOnlyLocal) to [out=0,in=180] (v4proapseOnlyLocal);
\draw[->] (vOnlyLocal) to [out=0,in=180] (vLightning4);
\draw[->] (vOnlyLocal) to [out=0,in=180] (v4snimoliOnlyLocal);
\draw[->] (vOnlyLocal) to [out=0,in=180] (vLightning5);

\draw[->] (v4yourself) to [out=0,in=180] (v4proapseViaInternet);
\draw[->] (v4yourself) to [out=0,in=180] (v4proapseOnlyLocal);
\draw[->] (v4yourself) to [out=0,in=180] (vLightning4);
\draw[->] (v4yourself) to [out=0,in=180] (v4snimoliViaInternet);
\draw[->] (v4yourself) to [out=0,in=180] (v4snimoliOnlyLocal);
\draw[->] (v4yourself) to [out=0,in=180] (vLightning5);

\draw[->] (vModified) to [out=0,in=180] (v4proapseViaInternet);
\draw[->] (vModified) to [out=0,in=180] (v4proapseOnlyLocal);
\draw[->] (vUnmodified) to [out=0,in=180] (vLightning4);
\draw[->] (vEmbedded) to [out=0,in=180] (v4snimoliViaInternet);
\draw[->] (vEmbedded) to [out=0,in=180] (v4snimoliOnlyLocal);
\draw[->] (vIndependent) to [out=0,in=180] (vLightning5);





\path [arrow, color=magenta] (v4osuc1) -- (o1);
\path [arrow, color=magenta] (v4osuc2) -- (o2s);
\path [arrow, color=magenta] (v4osuc2) -- (o2b);
\path [arrow, color=magenta] (v4osuc3) -- (o3l);
\path [arrow, color=magenta] (v4osuc3) -- (o3n);
\path [arrow, color=magenta] (v4osuc4) -- (o4s);
\path [arrow, color=magenta] (v4osuc4) -- (o4b);
\path [arrow, color=magenta] (v4osuc5) -- (o5s);
\path [arrow, color=magenta] (v4osuc5) -- (o5b);
\path [arrow, color=magenta] (v4osuc6) -- (o6l);
\path [arrow, color=magenta] (v4osuc6) -- (o6n);
\path [arrow, color=magenta] (v4osuc7) -- (o7s);
\path [arrow, color=magenta] (v4osuc7) -- (o7b);
\path [arrow, color=magenta] (v4osuc8) -- (o8s);
\path [arrow, color=magenta] (v4osuc8) -- (o8b);
\path [arrow, color=magenta] (v4osuc9) -- (o9l);
\path [arrow, color=magenta] (v4osuc9) -- (o9n);
\path [arrow, color=magenta] (v4osuc10) -- (o10s);
\path [arrow, color=magenta] (v4osuc10) -- (o10b);

\draw[->, color=violet] (v4snimoliOnlyLocal) to [out=0,in=180] (o9l);
\draw[->, color=violet] (v4snimoliViaInternet) to [out=0,in=180] (o9n);
\draw[->, color=violet] (v4snimoliOnlyLocal) to [out=0,in=180] (o6l);
\draw[->, color=violet] (v4snimoliViaInternet) to [out=0,in=180] (o6n);

\draw[->, color=violet] (v4proapseOnlyLocal) to [out=0,in=180] (o3l);
\draw[->, color=violet] (v4proapseViaInternet) to [out=0,in=180] (o3n);


   \draw[->, color=blue] (v2otherSources) to [out=0,in=224] (o2s);
   \draw[->, color=blue] (v2otherBinaries) to [out=0,in=220] (o2b);
   \draw[->, color=blue] (v2otherSources) to [out=0,in=216] (o4s);
   \draw[->, color=blue] (v2otherBinaries) to [out=0,in=212] (o4b);
   \draw[->, color=blue] (v2otherSources) to [out=0,in=208] (o5s);
   \draw[->, color=blue] (v2otherBinaries) to [out=0,in=204] (o5b);
   \draw[->, color=blue] (v2otherSources) to [out=0,in=200] (o7s);
   \draw[->, color=blue] (v2otherBinaries) to [out=0,in=196] (o7b);
   \draw[->, color=blue] (v2otherSources) to [out=0,in=192] (o8s);
   \draw[->, color=blue] (v2otherBinaries) to [out=0,in=188] (o8b);
   \draw[->, color=blue] (v2otherSources) to [out=0,in=184] (o10s);
   \draw[->, color=blue] (v2otherBinaries) to [out=0,in=180] (o10b);

% 


\end{tikzpicture}





%%%%%%%%%%%%%%%
% Telekom osCompendium 'for being included' snippet template
%
% (c) Karsten Reincke, Deutsche Telekom AG, Darmstadt 2011
%
% This LaTeX-File is licensed under the Creative Commons Attribution-ShareAlike
% 3.0 Germany License (http://creativecommons.org/licenses/by-sa/3.0/de/): Feel
% free 'to share (to copy, distribute and transmit)' or 'to remix (to adapt)'
% it, if you '... distribute the resulting work under the same or similar
% license to this one' and if you respect how 'you must attribute the work in
% the manner specified by the author ...':
%
% In an internet based reuse please link the reused parts to www.telekom.com and
% mention the original authors and Deutsche Telekom AG in a suitable manner. In
% a paper-like reuse please insert a short hint to www.telekom.com and to the
% original authors and Deutsche Telekom AG into your preface. For normal
% quotations please use the scientific standard to cite.
%
% [ Framework derived from 'mind your Scholar Research Framework' 
%   mycsrf (c) K. Reincke 2012 CC BY 3.0  http://mycsrf.fodina.de/ ]
%


%% use all entries of the bibliography
%\nocite{*}

\chapter{Open Source Use Cases: Find the License Fulfilling To-do Lists}\label{sec:OSUCfinder}

\footnotesize
\begin{quote}\itshape
This chapter offers the \emph{Open Source Use Case Finder}: First, it presents
a form to gather the specifying information. Second, it offers a tree which
can easily be transversed by the gathered information. And finally it contains
the list of \emph{open source use cases}. Each leaf of the tree leads to one
\emph{open source use case} which itself then refers to the specific license
fulfilling to-do lists.
\end{quote}
\normalsize{}

\section{A standard form for gathering the relevant information}
\label{OSLiCStandardFormForGatheringInformation}
 
\begin{small}
\begin{tabular}[h]{|l|l|l|l|}
\hline 
Class & Questions & Answers\\
\hline 
  Type
  & \parbox[c][2.6cm][c]{9.4cm}{
    \textit{Is the open source software you want to use a software library
    in the broadest sense (an includable code snippet, a linkable module or
    library, or a loadable plugin) [=snimoli], or is it an autonomous
    program, application or server which can be executed or processed
    [=proapse]?}} & \parbox{10em}{ 
      $\square$\hspace{1em}proapse\\ 
      $\square$\hspace{1em}snimoli}
    \\
\hline 
  State & 
  \parbox[c][1.6cm][c]{9.4cm}{
  \textit{Do you want to leave your open source software as you have
  got it, or do you want to modify it before using and/or distributing it to 3rd
  parties?}} &
  \parbox{10em}{
    $\square$\hspace{1em}unmodified\\
    $\square$\hspace{1em}modified} \\
\hline 
  Context & 
  \parbox[c][2cm][c]{9.4cm}{
  \textit{Are you using your open source software as an au\-to\-no\-mous piece
  of software [=independent], or are you using it as an embedded part or component
  of a larger, more complex piece of software [=embedded]?}} &
  \parbox{10em}{ $\square$\hspace{1em}independent\\
    $\square$\hspace{1em}embedded}\\
\hline 
  Recipient & 
  \parbox[c][1.6cm][c]{9.4cm}{
  \textit{Are you going to use the received open source software only for
  yourself [=4yourself], or do you plan to (re)distribute it (also) to third
  parties [=4others]?}}
  & \parbox{10em}{
    $\square$\hspace{1em}4yourself\\
    $\square$\hspace{1em}4others}\\
\hline 
  Form & 
  \parbox[c][1.6cm][c]{9.4cm}{
  \textit{If you plan to (re)distribute an open source based work [=4others], do
  you want to distribute only the binaries or (also) the source code?}}
  & \parbox{10em}{
    $\square$\hspace{1em}only binaries\\
    $\square$\hspace{1em}(also) sources}\\
\hline 
  Mode & 
  \parbox[c][2.6cm][c]{9.4cm}{
  \textit{Are you going to combine the received open source software with other
  software components by linking all together statically, by linking them
  dynamically, or by textually including (parts of) the open source software
  into your larger unit?}} &
  \parbox{10em}{
    $\square$\hspace{1em}statically linked\\   
    $\square$\hspace{1em}dynamically linked\\
    $\square$\hspace{1em}textually included}\\
\hline 
\hline
\end{tabular}
\end{small}

As discussed earlier, there are of course some invalid
combinations\footnote{type::proapse excludes state::embedded;
recipient::4yourself excludes the combination with state::independent and
type::snimoli; any value of class 'mode' implies state::embedded [for details
see page \pageref{InvalidFinderTokenCombinations}]. If you have gathered one of
these invalid combinations, please check the corresponding explanations}. Here
are some extra explanations about each class:

\begin{description}
\item[Type:] A piece of (open source) software shall be viewed as a program, an
application, or a server, if you can start its binary form with your normal
program launcher, or (in case of a text file which still must be interpreted by
an interpreter like php, perl, bash etc.) if you can start an interpreter taking
the file as one of its arguments. \item[State:] You modify open source software
if you expand, reduce or modify at least one of the received software files, and
-- in case of dealing with binary object code -- if you (re)compile and (re)link
the modified software to a new binary file. If you only modify configuration
files, you do not modify the open source software.
\item[Context:] You use open source software embedded into a larger unit, if one
of your files of the larger unit contains a verbatim or modified copy (i.e.\ a
snippet) of the received open source software, or if the larger unit contains an
include statement referring to a file of the received open source software, or
if your development environment contains a compiler or linker directive
referring to the received open source software.
\item[Recipient:] You use the received open source software only for yourself, if
you as person do not pass it to other persons, or if you -- as a member of a
specific development group -- pass it only to the other members of your
development group. But if you store open source software on any device such as a
mobile phone, an USB stick, etc.\ or if you attach it to any transport
medium like email etc.\ and if you then sell, give away, or simply send this
device or transport medium to anyone (other than a direct member
of your development group) then you indeed handover the open source software to
third parties\footnote{Please remember that -- at least in Germany -- there are
opinions that even handing over software to another legal entity or department
of the same company is also a kind of distribution. It is always safest
to take the broadest possible meaning of distributing or handing over.}.
\item[Form:] Often it is up to you to decide wether you want to distribute the
source code, too. But if you do so, you have to respect special
conditions\footnote{For details concerning a necessary refinement of the open
source use case taxonomy, please see $\rightarrow$ OSLiC, p.\
\pageref{sec:SourceBinaryDifference}}.
\item[Mode:] The definition follows the corresponding standard terms of the
software development.
\end{description}

\section{The taxonomic Open Source Use Case Finder}

Now, after having gathered the necessary information, determine your specific
open source use case by traversing the following tree and its corresponding
branches:

\begin{footnotesize}

\pstree[treemode=R,levelsep=*0.2,treesep=0.2]{\Toval{OSS}}{
  \pstree{
    \Tr{\Ovalbox{\shortstack{type:\\\textbf{\textit{proapse}}\\
      \tiny (= Program,\\\tiny Application,\\\tiny Server)      
      }   
    }}
    
  }{
    \pstree{
      \Tr{\Ovalbox{\shortstack{state:\\\textbf{\textit{unmodified}}}}}
    }{
      \pstree{
        \Tr{\Ovalbox{\shortstack{context:\\\textbf{\textit{independent}}}}}
      }{
        
        \pstree{
          \Tr{\Ovalbox{\shortstack{recipient:\\\textbf{\textit{4yourself}}}}}
        }{
          \Tr[edge=none]{\begin{minipage}[b][2em][c]{6em} 
              $\Rightarrow$ OSUC-01\\
              \textit{(see p.\ \pageref{OSUC-01-DEF})}\end{minipage} } 
          }
        
        \pstree{
          \Tr{\Ovalbox{\shortstack{recipient:\\\textbf{\textit{4others}}}}}
        }{
          \Tr[edge=none]{\begin{minipage}[b][2em][c]{6em} 
              $\Rightarrow$ OSUC-02\\
              \textit{(see p.\ \pageref{OSUC-02-DEF})}\end{minipage} } 
          }        
      }
    }
    \pstree{
      \Tr{\Ovalbox{\shortstack{state:\\\textbf{\textit{modified}}}}}
    }{
      \pstree{
        \Tr{\Ovalbox{\shortstack{context:\\\textbf{\textit{independent}}}}}
      }{
        \pstree{
          \Tr{\Ovalbox{\shortstack{recipient:\\\textbf{\textit{4yourself}}}}}
        }{
          \Tr[edge=none]{\begin{minipage}[b][2em][c]{6em} 
              $\Rightarrow$ OSUC-03\\
              \textit{(see p.\ \pageref{OSUC-03-DEF})}\end{minipage} } 
          }
        
        \pstree{
          \Tr{\Ovalbox{\shortstack{recipient:\\\textbf{\textit{4others}}}}}
        }{
          \Tr[edge=none]{\begin{minipage}[b][2em][c]{6em} 
              $\Rightarrow$ OSUC-04\\
              \textit{(see p.\ \pageref{OSUC-04-DEF})}\end{minipage} } 
          }        
      }
    }
  }
  \pstree{
    \Tr{\Ovalbox{
      \shortstack{type:\\\textbf{\textit{snimoli}}\\
      \tiny (= Snippet,\\\tiny Module,\\\tiny Plugin,\\\tiny Library)            
      }
    }}
  }{
    \pstree{
      \Tr{\Ovalbox{\shortstack{state:\\\textbf{\textit{unmodified}}}}}
    }{
      \pstree{
        \Tr{\Ovalbox{\shortstack{context:\\\textbf{\textit{independent}}}}}
      }{
      
        \pstree{
          \Tr{\Ovalbox{\shortstack{recipient:\\\textbf{\textit{4others}}}}}
        }{
          \Tr[edge=none]{\begin{minipage}[b][2em][c]{6em} 
              $\Rightarrow$ OSUC-05\\
              \textit{(see p.\ \pageref{OSUC-05-DEF})}\end{minipage} } 
          }        
      
      }
      \pstree{
        \Tr{\Ovalbox{\shortstack{context:\\\textbf{\textit{embedded}}}}}
      }{
        \pstree{
          \Tr{
            \Ovalbox{\shortstack{recipient:\\\textbf{\textit{4yourself}}}}
           }
        }{
          \pstree{
            \Tr{
              \begin{tiny}
              \Ovalbox{\shortstack{mode:\\\textit{statically linked}}}
              \end{tiny}
            }
          }{
            \Tr[edge=none]{\begin{minipage}[b][2em][c]{7em} 
              $\Rightarrow$ OSUC-06a\\
              \textit{(see p.\ \pageref{OSUC-06-DEF})}\end{minipage} } 
          }        
        
          \pstree{
            \Tr{
              \begin{tiny}
              \Ovalbox{\shortstack{mode:\\\textit{dynamically linked}}}
              \end{tiny}
            }
          }{
            \Tr[edge=none]{\begin{minipage}[b][2em][c]{7em} 
              $\Rightarrow$ OSUC-06b\\
              \textit{(see p.\ \pageref{OSUC-06-DEF})}\end{minipage} } 
          }        
 
          \pstree{
            \Tr{
              \begin{tiny}
              \Ovalbox{\shortstack{mode:\\\textit{textually included}}}
              \end{tiny}
            }
          }{
            \Tr[edge=none]{\begin{minipage}[b][2em][c]{7em} 
              $\Rightarrow$ OSUC-06c\\
              \textit{(see p.\ \pageref{OSUC-06-DEF})}\end{minipage} } 
          }        
        
        }
        \pstree{
          \Tr{\Ovalbox{\shortstack{recipient:\\\textbf{\textit{4others}}}}}
        }{
          \pstree{
            \Tr{
              \begin{tiny}
              \Ovalbox{\shortstack{mode:\\\textit{statically linked}}}
              \end{tiny}
            }
          }{
            \Tr[edge=none]{\begin{minipage}[b][2em][c]{7em} 
              $\Rightarrow$ OSUC-07a\\
              \textit{(see p.\ \pageref{OSUC-07-DEF})}\end{minipage} } 
          }        
        
          \pstree{
            \Tr{
              \begin{tiny}
              \Ovalbox{\shortstack{mode:\\\textit{dynamically linked}}}
              \end{tiny}
            }
          }{
            \Tr[edge=none]{\begin{minipage}[b][2em][c]{7em} 
              $\Rightarrow$ OSUC-07b\\
              \textit{(see p.\ \pageref{OSUC-07-DEF})}\end{minipage} } 
          }        
 
          \pstree{
            \Tr{
              \begin{tiny}
              \Ovalbox{\shortstack{mode:\\\textit{textually included}}}
              \end{tiny}
             }
          }{
            \Tr[edge=none]{\begin{minipage}[b][2em][c]{7em} 
              $\Rightarrow$ OSUC-07c\\
              \textit{(see p.\ \pageref{OSUC-07-DEF})}\end{minipage} } 
          }        
        }
      }
    }
    \pstree{
      \Tr{\Ovalbox{\shortstack{state:\\\textbf{\textit{modified}}}}}
    }{
      \pstree{
        \Tr{\Ovalbox{\shortstack{context:\\\textbf{\textit{independent}}}}}
      }{
        \pstree{
          \Tr{\Ovalbox{\shortstack{recipient:\\\textbf{\textit{4others}}}}}
        }{
          \Tr[edge=none]{\begin{minipage}[b][2em][c]{6em} 
              $\Rightarrow$ OSUC-08\\
              \textit{(see p.\ \pageref{OSUC-08-DEF})}\end{minipage} } 
          }        
      }
      \pstree{
        \Tr{\Ovalbox{\shortstack{context:\\\textbf{\textit{embedded}}}}}
      }{
        \pstree{
          \Tr{\Ovalbox{\shortstack{recipient:\\\textbf{\textit{4yourself}}}}}
        }{
          \pstree{
            \Tr{
              \begin{tiny}
              \Ovalbox{\shortstack{mode:\\\textit{statically linked}}}
              \end{tiny}
            }
          }{
            \Tr[edge=none]{\begin{minipage}[b][2em][c]{7em} 
              $\Rightarrow$ OSUC-09a\\
              \textit{(see p.\ \pageref{OSUC-09-DEF})}\end{minipage} } 
          }        
        
          \pstree{
            \Tr{
              \begin{tiny}
              \Ovalbox{\shortstack{mode:\\\textit{dynamically linked}}}
              \end{tiny}
            }
          }{
            \Tr[edge=none]{\begin{minipage}[b][2em][c]{7em} 
              $\Rightarrow$ OSUC-09b\\
              \textit{(see p.\ \pageref{OSUC-09-DEF})}\end{minipage} } 
          }        
 
          \pstree{
            \Tr{
              \begin{tiny}
              \Ovalbox{\shortstack{mode:\\\textit{textually included}}}
              \end{tiny}
             }
          }{
            \Tr[edge=none]{\begin{minipage}[b][2em][c]{7em} 
              $\Rightarrow$ OSUC-09c\\
              \textit{(see p.\ \pageref{OSUC-09-DEF})}\end{minipage} } 
          }        
        }
        \pstree{
          \Tr{\Ovalbox{\shortstack{recipient:\\\textbf{\textit{4others}}}}}
        }{
          \pstree{
            \Tr{
              \begin{tiny}
              \Ovalbox{\shortstack{mode:\\\textit{\tiny statically linked}}}
              \end{tiny}
            }
          }{
            \Tr[edge=none]{\begin{minipage}[b][2em][c]{7em} 
              $\Rightarrow$ OSUC-10a\\
              \textit{(see p.\ \pageref{OSUC-10-DEF})}\end{minipage} } 
          }        
        
          \pstree{
            \Tr{
              \begin{tiny}
              \Ovalbox{\shortstack{mode:\\\textit{dynamically linked}}}
              \end{tiny}              
            }
          }{
            \Tr[edge=none]{\begin{minipage}[b][2em][c]{7em} 
              $\Rightarrow$ OSUC-10b\\
              \textit{(see p.\ \pageref{OSUC-10-DEF})}\end{minipage} } 
          }        
 
          \pstree{
            \Tr{
              \begin{tiny}
              \Ovalbox{\shortstack{mode:\\\textit{textually included}}} 
              \end{tiny}
             }
          }{
            \Tr[edge=none]{\begin{minipage}[b][2em][c]{7em} 
              $\Rightarrow$ OSUC-10c\\
              \textit{(see p.\ \pageref{OSUC-10-DEF})}\end{minipage} } 
          }        
        }
      }
    }
  }
}
\end{footnotesize}
\label{OSLiCUseCaseFinder}

Please keep in mind, that for reducing the complexity of this general taxonomy,
we suppressed the differentiation between distributing binaries and distributing
sources\footnote{For details concerning a necessary refinement of the taxonomy
$\rightarrow$ OSLiC, p.\ \pageref{sec:SourceBinaryDifference}}. Nevertheless,
the final to-do lists respect this difference whereever it is necessary. It is
up to you to select the correct to-do list on the basis of its titles.

\section{The open source use cases and its to-do list references}

On the following pages, each \textbf{O}pen \textbf{S}ource \textbf{U}se
\textbf{C}ase is textually specified one more time and added by a list of page
numbers. Each of these pages hints to that license-specific to-do list whose
items together offer a processable way for acting according to the license under
the circumstances of the described \textbf{O}pen \textbf{S}ource \textbf{U}se
\textbf{C}ase.

\begin{description}
\label{OSUCList}
\item[OSUC-01:]\label{OSUC-01-DEF}
Only for yourself, you are using an unmodified open source program, application,
or server -- just as you received it. You are not going to combine it with other
components in the sense of software development (= \textit{proapse, unmodified,
independent, 4yourself}). 
To see the \textit{specific, license fulfilling to-do lists} jump to the
following pages:
  \begin{itemize}
    \item p.\ \pageref{OSUC-01-AGPL} for the \textbf{AGPL}
      \textit{(= Affero GNU Public License)} 
    \item p.\ \pageref{OSUC-01-Apache20} for the \textbf{ApL}
      \textit{(= Apache License)}
    \item p.\ \pageref{OSUC-01-BSD} for the \textbf{BSD} License
      \textit{(= Berkeley Software Distribution)}
    \item p.\ \pageref{OSUC-01-EPL} for the \textbf{EPL}
      \textit{(= Eclipse Public License)}     
    \item p.\ \pageref{OSUC-01-EUPL} for the \textbf{EUPL}
      \textit{(= European Public License)} 
    \item p.\ \pageref{OSUC-01-GPL} for the \textbf{GPL}
       \textit{(= GNU Public License)} 
    \item p.\ \pageref{OSUC-01-LGPL} for the \textbf{LGPL}
      \textit{(= Lesser GNU Public License)}           
    \item p.\ \pageref{OSUC-01-MIT} for the \textbf{MIT} License
       \textit{(= Massachusetts Institute of Technology)} 
    \item p.\ \pageref{OSUC-01-MPL} for \textbf{MPL}
      \textit{(= Mozilla Public License)}     
    \item p.\ \pageref{OSUC-01-MsPL} for the \textbf{MSPL}
      \textit{(= Microsoft Public License)} 
    \item p.\ \pageref{OSUC-01-PGL} for the \textbf{PGL}
      \textit{(= Postgres License)} 
    \item p.\ \pageref{OSUC-01-PHP} for the \textbf{PHP} license 
  \end{itemize}

\item[OSUC-02:]\label{OSUC-02-DEF} Just as you received it, you are going to
distribute an unmodified open source program, application, or server to 3rd
parties. In this act of distribution, you do not combine this program,
application, or server with other software components in the sense of software
development (= \textit{proapse, unmodified, independent, 4others})\footnote{For
differentiating between distributing sources and binaries $\rightarrow$ OSLiC,
p.\ \pageref{sec:SourceBinaryDifference}}. To see the \textit{specific, license
fulfilling to-do lists} jump to the following pages:
   \begin{itemize}
    \item p.\ \pageref{OSUC-02-AGPL} for the \textbf{AGPL}
      \textit{(= Affero GNU Public License)} 
    \item p.\ \pageref{OSUC-02-Apache20} for the \textbf{ApL}
      \textit{(= Apache License)}
    \item p.\ \pageref{OSUC-02-BSD} for the \textbf{BSD} License
      \textit{(= Berkeley Software Distribution)}
    \item p.\ \pageref{OSUC-02-EPL} for the \textbf{EPL}
      \textit{(= Eclipse Public License)}     
    \item p.\ \pageref{OSUC-02-EUPL} for the \textbf{EUPL}
      \textit{(= European Public License)} 
    \item p.\ \pageref{OSUC-02-GPL} for the \textbf{GPL}
       \textit{(= GNU Public License)} 
    \item p.\ \pageref{OSUC-02-LGPL} for the \textbf{LGPL}
      \textit{(= Lesser GNU Public License)}           
    \item p.\ \pageref{OSUC-02-MIT} for the \textbf{MIT} License
       \textit{(= Massachusetts Institute of Technology)} 
    \item p.\ \pageref{OSUC-02-MPL} for \textbf{MPL}
      \textit{(= Mozilla Public License)}     
    \item p.\ \pageref{OSUC-02-MsPL} for the \textbf{MSPL}
      \textit{(= Microsoft Public License)} 
    \item p.\ \pageref{OSUC-02-PGL} for the \textbf{PGL}
      \textit{(= Postgres License)} 
    \item p.\ \pageref{OSUC-02-PHP} for the \textbf{PHP} license 
  \end{itemize}

\item[OSUC-03:]\label{OSUC-03-DEF} Only for yourself, you are modifying a
received open source program, application, or server, before you are using it.
But you do not combine it with other components in the sense of software
development (= \textit{proapse, modified, independent, 4yourself}).
To see the \textit{specific, license fulfilling to-do lists} jump to the
following pages:
   \begin{itemize}
    \item p.\ \pageref{OSUC-03-AGPL} for the \textbf{AGPL}
      \textit{(= Affero GNU Public License)} 
    \item p.\ \pageref{OSUC-03-Apache20} for the \textbf{ApL}
      \textit{(= Apache License)}
    \item p.\ \pageref{OSUC-03-BSD} for the \textbf{BSD} License
      \textit{(= Berkeley Software Distribution)}
    \item p.\ \pageref{OSUC-03-EPL} for the \textbf{EPL}
      \textit{(= Eclipse Public License)}     
    \item p.\ \pageref{OSUC-03-EUPL} for the \textbf{EUPL}
      \textit{(= European Public License)} 
    \item p.\ \pageref{OSUC-03-GPL} for the \textbf{GPL}
       \textit{(= GNU Public License)} 
    \item p.\ \pageref{OSUC-03-LGPL} for the \textbf{LGPL}
      \textit{(= Lesser GNU Public License)}           
    \item p.\ \pageref{OSUC-03-MIT} for the \textbf{MIT} License
       \textit{(= Massachusetts Institute of Technology)} 
    \item p.\ \pageref{OSUC-03-MPL} for \textbf{MPL}
      \textit{(= Mozilla Public License)}     
    \item p.\ \pageref{OSUC-03-MsPL} for the \textbf{MSPL}
      \textit{(= Microsoft Public License)} 
    \item p.\ \pageref{OSUC-03-PGL} for the \textbf{PGL}
      \textit{(= Postgres License)} 
    \item p.\ \pageref{OSUC-03-PHP} for the \textbf{PHP} license 
  \end{itemize}

\item[OSUC-04:]\label{OSUC-04-DEF} You are going to modify a received open
source program, application, or server, before you distribute it to 3rd parties.
But you do not combine this modified program, application, or server with other
software components in the sense of software development (= \textit{proapse,
modified, independent, 4others})\footnote{For differentiating between
distributing sources and binaries $\rightarrow$ OSLiC, p.\
\pageref{sec:SourceBinaryDifference}}.
To see the \textit{specific, license fulfilling to-do lists} jump to the
following pages:
  \begin{itemize}
    \item p.\ \pageref{OSUC-04-AGPL} for the \textbf{AGPL}
      \textit{(= Affero GNU Public License)} 
    \item p.\ \pageref{OSUC-04-Apache20} for the \textbf{ApL}
      \textit{(= Apache License)}
    \item p.\ \pageref{OSUC-04-BSD} for the \textbf{BSD} License
      \textit{(= Berkeley Software Distribution)}
    \item p.\ \pageref{OSUC-04-EPL} for the \textbf{EPL}
      \textit{(= Eclipse Public License)}     
    \item p.\ \pageref{OSUC-04-EUPL} for the \textbf{EUPL}
      \textit{(= European Public License)} 
    \item p.\ \pageref{OSUC-04-GPL} for the \textbf{GPL}
       \textit{(= GNU Public License)} 
    \item p.\ \pageref{OSUC-04-LGPL} for the \textbf{LGPL}
      \textit{(= Lesser GNU Public License)}           
    \item p.\ \pageref{OSUC-04-MIT} for the \textbf{MIT} License
       \textit{(= Massachusetts Institute of Technology)} 
    \item p.\ \pageref{OSUC-04-MPL} for \textbf{MPL}
      \textit{(= Mozilla Public License)}     
    \item p.\ \pageref{OSUC-04-MsPL} for the \textbf{MSPL}
      \textit{(= Microsoft Public License)} 
    \item p.\ \pageref{OSUC-04-PGL} for the \textbf{PGL}
      \textit{(= Postgres License)} 
    \item p.\ \pageref{OSUC-04-PHP} for the \textbf{PHP} license 
  \end{itemize}

\item[OSUC-05:]\label{OSUC-05-DEF} Just as you received it, you are going to
distribute an unmodified open source library, code snippet, module, or plugin to
3rd parties. In this act of distribution, you do not combine this library, code
snippet, module, or plugin with other software components in the sense of
software development (= \textit{snimoli, unmodified, independent, 4others}).
To see the \textit{specific, license fulfilling to-do lists} jump to the
following pages:
  \begin{itemize}
    \item p.\ \pageref{OSUC-05-AGPL} for the \textbf{AGPL}
      \textit{(= Affero GNU Public License)} 
    \item p.\ \pageref{OSUC-05-Apache20} for the \textbf{ApL}
      \textit{(= Apache License)}
    \item p.\ \pageref{OSUC-05-BSD} for the \textbf{BSD} License
      \textit{(= Berkeley Software Distribution)}
    \item p.\ \pageref{OSUC-05-EPL} for the \textbf{EPL}
      \textit{(= Eclipse Public License)}     
    \item p.\ \pageref{OSUC-05-EUPL} for the \textbf{EUPL}
      \textit{(= European Public License)} 
    \item p.\ \pageref{OSUC-05-GPL} for the \textbf{GPL}
       \textit{(= GNU Public License)} 
    \item p.\ \pageref{OSUC-05-LGPL} for the \textbf{LGPL}
      \textit{(= Lesser GNU Public License)}           
    \item p.\ \pageref{OSUC-05-MIT} for the \textbf{MIT} License
       \textit{(= Massachusetts Institute of Technology)} 
    \item p.\ \pageref{OSUC-05-MPL} for \textbf{MPL}
      \textit{(= Mozilla Public License)}     
    \item p.\ \pageref{OSUC-05-MsPL} for the \textbf{MSPL}
      \textit{(= Microsoft Public License)} 
    \item p.\ \pageref{OSUC-05-PGL} for the \textbf{PGL}
      \textit{(= Postgres License)} 
    \item p.\ \pageref{OSUC-05-PHP} for the \textbf{PHP} license 
  \end{itemize}

\item[OSUC-06:]\label{OSUC-06-DEF} Only for yourself and just as you received
it, you are going to combine an unmodified open source library, code snippet,
module, or plugin into a larger software unit as one of its parts. (=
\textit{snimoli, unmodified, embedded, 4yourself}).
To see the \textit{specific, license fulfilling to-do lists} jump to the
following pages:
   \begin{itemize}
    \item p.\ \pageref{OSUC-06-AGPL} for the \textbf{AGPL}
      \textit{(= Affero GNU Public License)} 
    \item p.\ \pageref{OSUC-06-Apache20} for the \textbf{ApL}
      \textit{(= Apache License)}
    \item p.\ \pageref{OSUC-06-BSD} for the \textbf{BSD} License
      \textit{(= Berkeley Software Distribution)}
    \item p.\ \pageref{OSUC-06-EPL} for the \textbf{EPL}
      \textit{(= Eclipse Public License)}     
    \item p.\ \pageref{OSUC-06-EUPL} for the \textbf{EUPL}
      \textit{(= European Public License)} 
    \item p.\ \pageref{OSUC-06-GPL} for the \textbf{GPL}
       \textit{(= GNU Public License)} 
    \item p.\ \pageref{OSUC-06-LGPL} for the \textbf{LGPL}
      \textit{(= Lesser GNU Public License)}           
    \item p.\ \pageref{OSUC-06-MIT} for the \textbf{MIT} License
       \textit{(= Massachusetts Institute of Technology)} 
    \item p.\ \pageref{OSUC-06-MPL} for \textbf{MPL}
      \textit{(= Mozilla Public License)}     
    \item p.\ \pageref{OSUC-06-MsPL} for the \textbf{MSPL}
      \textit{(= Microsoft Public License)} 
    \item p.\ \pageref{OSUC-06-PGL} for the \textbf{PGL}
      \textit{(= Postgres License)} 
    \item p.\ \pageref{OSUC-06-PHP} for the \textbf{PHP} license 
  \end{itemize}

\item[OSUC-07:]\label{OSUC-07-DEF} Just as you received it and before you will
distribute it to 3rd parties together with the larger software unit, you combine
an unmodified open source library, code snippet, module, or plugin into a larger
software unit in the sense of software development (= \textit{snimoli,
unmodified, embedded, 4others})\footnote{For differentiating between
distributing sources and binaries $\rightarrow$ OSLiC, p.\
\pageref{sec:SourceBinaryDifference}}.
To see the \textit{specific, license fulfilling to-do lists} jump to the
following pages:
   \begin{itemize}
    \item p.\ \pageref{OSUC-07-AGPL} for the \textbf{AGPL}
      \textit{(= Affero GNU Public License)} 
    \item p.\ \pageref{OSUC-07-Apache20} for the \textbf{ApL}
      \textit{(= Apache License)}
    \item p.\ \pageref{OSUC-07-BSD} for the \textbf{BSD} License
      \textit{(= Berkeley Software Distribution)}
    \item p.\ \pageref{OSUC-07-EPL} for the \textbf{EPL}
      \textit{(= Eclipse Public License)}     
    \item p.\ \pageref{OSUC-07-EUPL} for the \textbf{EUPL}
      \textit{(= European Public License)} 
    \item p.\ \pageref{OSUC-07-GPL} for the \textbf{GPL}
       \textit{(= GNU Public License)} 
    \item p.\ \pageref{OSUC-07-LGPL} for the \textbf{LGPL}
      \textit{(= Lesser GNU Public License)}           
    \item p.\ \pageref{OSUC-07-MIT} for the \textbf{MIT} License
       \textit{(= Massachusetts Institute of Technology)} 
    \item p.\ \pageref{OSUC-07-MPL} for \textbf{MPL}
      \textit{(= Mozilla Public License)}     
    \item p.\ \pageref{OSUC-07-MsPL} for the \textbf{MSPL}
      \textit{(= Microsoft Public License)} 
    \item p.\ \pageref{OSUC-07-PGL} for the \textbf{PGL}
      \textit{(= Postgres License)} 
    \item p.\ \pageref{OSUC-07-PHP} for the \textbf{PHP} license 
  \end{itemize}

\item[OSUC-08:]\label{OSUC-08-DEF} Before you will distribute it, you are going
to modify an open source library, code snippet, module, or plugin to 3rd
parties, but you do not combine it with other software components in the sense
of software development (= \textit{snimoli, modified, independent,
4others})\footnote{For differentiating between distributing sources and binaries
$\rightarrow$ OSLiC, p.\ \pageref{sec:SourceBinaryDifference}}. To see the
\textit{specific, license fulfilling to-do lists} jump to the following pages:
  \begin{itemize}
    \item p.\ \pageref{OSUC-08-AGPL} for the \textbf{AGPL}
      \textit{(= Affero GNU Public License)} 
    \item p.\ \pageref{OSUC-08-Apache20} for the \textbf{ApL}
      \textit{(= Apache License)}
    \item p.\ \pageref{OSUC-08-BSD} for the \textbf{BSD} License
      \textit{(= Berkeley Software Distribution)}
    \item p.\ \pageref{OSUC-08-EPL} for the \textbf{EPL}
      \textit{(= Eclipse Public License)}     
    \item p.\ \pageref{OSUC-08-EUPL} for the \textbf{EUPL}
      \textit{(= European Public License)} 
    \item p.\ \pageref{OSUC-08-GPL} for the \textbf{GPL}
       \textit{(= GNU Public License)} 
    \item p.\ \pageref{OSUC-08-LGPL} for the \textbf{LGPL}
      \textit{(= Lesser GNU Public License)}           
    \item p.\ \pageref{OSUC-08-MIT} for the \textbf{MIT} License
       \textit{(= Massachusetts Institute of Technology)} 
    \item p.\ \pageref{OSUC-08-MPL} for \textbf{MPL}
      \textit{(= Mozilla Public License)}     
    \item p.\ \pageref{OSUC-08-MsPL} for the \textbf{MSPL}
      \textit{(= Microsoft Public License)} 
    \item p.\ \pageref{OSUC-08-PGL} for the \textbf{PGL}
      \textit{(= Postgres License)} 
    \item p.\ \pageref{OSUC-08-PHP} for the \textbf{PHP} license 
  \end{itemize}


\item[OSUC-09:]\label{OSUC-09-DEF} Only for yourself, you are going to modify an
open source library, code snippet, module, or plugin, before you will combine it
-- in the sense of software development -- into a larger software unit as one of
its parts. (= \textit{snimoli, modified, embedded, 4yourself}). 
To see the \textit{specific, license fulfilling to-do lists} jump to the
following pages:
  \begin{itemize}
    \item p.\ \pageref{OSUC-09-AGPL} for the \textbf{AGPL}
      \textit{(= Affero GNU Public License)} 
    \item p.\ \pageref{OSUC-09-Apache20} for the \textbf{ApL}
      \textit{(= Apache License)}
    \item p.\ \pageref{OSUC-09-BSD} for the \textbf{BSD} License
      \textit{(= Berkeley Software Distribution)}
    \item p.\ \pageref{OSUC-09-EPL} for the \textbf{EPL}
      \textit{(= Eclipse Public License)}     
    \item p.\ \pageref{OSUC-09-EUPL} for the \textbf{EUPL}
      \textit{(= European Public License)} 
    \item p.\ \pageref{OSUC-09-GPL} for the \textbf{GPL}
       \textit{(= GNU Public License)} 
    \item p.\ \pageref{OSUC-09-LGPL} for the \textbf{LGPL}
      \textit{(= Lesser GNU Public License)}           
    \item p.\ \pageref{OSUC-09-MIT} for the \textbf{MIT} License
       \textit{(= Massachusetts Institute of Technology)} 
    \item p.\ \pageref{OSUC-09-MPL} for \textbf{MPL}
      \textit{(= Mozilla Public License)}     
    \item p.\ \pageref{OSUC-09-MsPL} for the \textbf{MSPL}
      \textit{(= Microsoft Public License)} 
    \item p.\ \pageref{OSUC-09-PGL} for the \textbf{PGL}
      \textit{(= Postgres License)} 
    \item p.\ \pageref{OSUC-09-PHP} for the \textbf{PHP} license 
  \end{itemize}


\item[OSUC-10:]\label{OSUC-10-DEF} Before you will distribute it to 3rd parties,
you are going to modify an open source library, code snippet, module, or plugin,
which you then combine with other software components in the sense of software
development (= \textit{snimoli, modified, independent, 4others})\footnote{For
differentiating between distributing sources and binaries $\rightarrow$ OSLiC,
p.\ \pageref{sec:SourceBinaryDifference}}. To see the \textit{specific, license
fulfilling to-do lists} jump to the following pages:
  \begin{itemize}
    \item p.\ \pageref{OSUC-10-AGPL} for the \textbf{AGPL}
      \textit{(= Affero GNU Public License)} 
    \item p.\ \pageref{OSUC-10-Apache20} for the \textbf{ApL}
      \textit{(= Apache License)}
    \item p.\ \pageref{OSUC-10-BSD} for the \textbf{BSD} License
      \textit{(= Berkeley Software Distribution)}
    \item p.\ \pageref{OSUC-10-EPL} for the \textbf{EPL}
      \textit{(= Eclipse Public License)}     
    \item p.\ \pageref{OSUC-10-EUPL} for the \textbf{EUPL}
      \textit{(= European Public License)} 
    \item p.\ \pageref{OSUC-10-GPL} for the \textbf{GPL}
       \textit{(= GNU Public License)} 
    \item p.\ \pageref{OSUC-10-LGPL} for the \textbf{LGPL}
      \textit{(= Lesser GNU Public License)}           
    \item p.\ \pageref{OSUC-10-MIT} for the \textbf{MIT} License
       \textit{(= Massachusetts Institute of Technology)} 
    \item p.\ \pageref{OSUC-10-MPL} for \textbf{MPL}
      \textit{(= Mozilla Public License)}     
    \item p.\ \pageref{OSUC-10-MsPL} for the \textbf{MSPL}
      \textit{(= Microsoft Public License)} 
    \item p.\ \pageref{OSUC-10-PGL} for the \textbf{PGL}
      \textit{(= Postgres License)} 
    \item p.\ \pageref{OSUC-10-PHP} for the \textbf{PHP} license 
  \end{itemize}

\end{description}

%\bibliography{../../../bibfiles/oscResourcesEn}



%%%%%%%%%%%%%%%
  
% Telekom osCompendium 'for being included' snippet template
%
% (c) Karsten Reincke, Deutsche Telekom AG, Darmstadt 2011
%
% This LaTeX-File is licensed under the Creative Commons Attribution-ShareAlike
% 3.0 Germany License (http://creativecommons.org/licenses/by-sa/3.0/de/): Feel
% free 'to share (to copy, distribute and transmit)' or 'to remix (to adapt)'
% it, if you '... distribute the resulting work under the same or similar
% license to this one' and if you respect how 'you must attribute the work in
% the manner specified by the author ...':
%
% In an internet based reuse please link the reused parts to www.telekom.com and
% mention the original authors and Deutsche Telekom AG in a suitable manner. In
% a paper-like reuse please insert a short hint to www.telekom.com and to the
% original authors and Deutsche Telekom AG into your preface. For normal
% quotations please use the scientific standard to cite.
%
% [ File structure derived from 'mind your Scholar Research Framework' 
%   mycsrf (c) K. Reincke CC BY 3.0  http://mycsrf.fodina.de/ ]
%

% Chapter Abstract
% ----------------

\chapter{Open Source License Compliance: To-Do Lists}

\footnotesize
\begin{quote}\itshape
With respect to the defined open source use cases, this chapter lists what one
has to do for acting in accordance with the specific open source licenses
\end{quote}
\normalsize{}

\section{Some general remarks on 'giving' someone a file}

This chapter has to be started with some general hints being relevant for many
to-do lists. Thus, for not repeating these remarks to often, we are starting
with these general remarks and will later on refer to these remarks:

\label{DistributingFilesHint}
\begin{itemize}
  \item On the one hand, sometimes you want to distribute a bi\-na\-ry package
  containing open source software as components or being open software as whole.
  Moreover, in some cases, you probably want to distribute it on a medium which
  doesn't allow the user, to see the package files directly -- some mobile
  devices don't give their users the full access to all stored files. On the
  other hand, open source licenses often require 'to give' someone copies of
  text files, like the license itself, copyright notes, specific notice files or
  anything else. The safe interpretation of 'giving someone a text' means that
  the receiver must be able to read it\footnote{To give someone anything he
  can't touch, feel, see etc., is like not giving him the object ;-)}. Hence, on
  systems which offer a file browser and a suitable reader, it is sufficient, to
  put these file onto the files system. On the other systems, you \emph{must}
  present the content of the files by your application -- for example in a
  specific copyright dialog\footnote{Additionally, in the open source community,
  it is a good tradition, to present these reference data voluntarily.}. The
  OSLiC does not want to refine the taxonomies down to the level of operating
  systems. So, it is up to the reader to transfer these conditions into the
  to-do lists.
  
  \item Sometimes a product which uses and distributes open source software
  tries to fulfill the requirement 'to give the recipients the license etc.' by
  presenting links to general versions of these licensing files hosted somewhere
  on the internet. But be aware: Although it is a good tradition -- especially
  if you link to the homepages of the projects for being totally transparent --
  it is not sufficient to offer only the links. If you are required by the open
  source licenses to handover something to your users, \emph{you} must do it. It
  is not safe to delegate the task to anyone hoping that he will offer the files
  all the time your product is distributed\footnote{Moreover, the advantage
  of doing the job oneself is that one has not to struggle with uncommunicated
  implicite modfications of the link targets.}. But even it would be save, the
  point is: you have to fulfill the license, no one else.
\end{itemize}

\label{OSUCToDoLists}
% Telekom osCompendium 'for being included' snippet template
%
% (c) Karsten Reincke, Deutsche Telekom AG, Darmstadt 2011
%
% This LaTeX-File is licensed under the Creative Commons Attribution-ShareAlike
% 3.0 Germany License (http://creativecommons.org/licenses/by-sa/3.0/de/): Feel
% free 'to share (to copy, distribute and transmit)' or 'to remix (to adapt)'
% it, if you '... distribute the resulting work under the same or similar
% license to this one' and if you respect how 'you must attribute the work in
% the manner specified by the author ...':
%
% In an internet based reuse please link the reused parts to www.telekom.com and
% mention the original authors and Deutsche Telekom AG in a suitable manner. In
% a paper-like reuse please insert a short hint to www.telekom.com and to the
% original authors and Deutsche Telekom AG into your preface. For normal
% quotations please use the scientific standard to cite.
%
% [ Framework derived from 'mind your Scholar Research Framework' 
%   mycsrf (c) K. Reincke 2012 CC BY 3.0  http://mycsrf.fodina.de/ ]
%


%% use all entries of the bibliography
%\nocite{*}

\section{Apache licensed software}

Officially, only the Apache License, Version 2.0 is an approved open source
license\footnote{For details $\rightarrow$ OSLiC, pp.\
\pageref{sec:ProtectPowerOfApL}}.
Explicitly, it 'only' focuses on the
\enquote{redistribution}\footcite[cf.][\nopage wp. §4]{OSI2012b}. So, you can
also use the folloing simplified Apache specific open source use case
finder\footnote{For details of the general OSUC finder $\rightarrow$ OSLiC, pp.\
\pageref{OsucTokens} and \pageref{OsucDefinitionTree}}:
 
\tikzstyle{nodv} = [font=\small, ellipse, draw, fill=gray!10, 
    text width=2cm, text centered, minimum height=2em]

\tikzstyle{nods} = [font=\footnotesize, rectangle, draw, fill=gray!20, 
    text width=1.2cm, text centered, rounded corners, minimum height=3em]

\tikzstyle{nodb} = [font=\footnotesize, rectangle, draw, fill=gray!20, 
    text width=2.2cm, text centered, rounded corners, minimum height=3em]
    
\tikzstyle{leaf} = [font=\tiny, rectangle, draw, fill=gray!30, 
    text width=1.2cm, text centered, minimum height=6em]

\tikzstyle{edge} = [draw, -latex']

\begin{tikzpicture}[]

\node[nodv] (l71) at (4,10) {ApL};

\node[nodb] (l61) at (0,8.6) {\textit{recipient:} \\ \textbf{4yourself}};
\node[nodb] (l62) at (6.5,8.6) {\textit{recipient:} \\ \textbf{4others}};

\node[nodb] (l51) at (2.5,7) {\textit{state:} \\ \textbf{unmodified}};
\node[nodb] (l52) at (9.3,7) {\textit{state:} \\ \textbf{modified}};

\node[nods] (l41) at (1.8,5.4) {\textit{form:} \textbf{source}};
\node[nods] (l42) at (3.6,5.4) {\textit{form:} \textbf{binary}};
\node[nodb] (l43) at (6.5,5.4) {\textit{type:} \\ \textbf{proapse}};
\node[nodb] (l44) at (12,5.4) {\textit{type:} \\ \textbf{snimoli}};


\node[nods] (l31) at (5.4,3.8) {\textit{form:} \textbf{source}};
\node[nods] (l32) at (7.2,3.8) {\textit{form:} \textbf{binary}};
\node[nodb] (l33) at (10,3.8) {\textit{context:} \\ \textbf{independent}};
\node[nodb] (l34) at (13.5,3.8) {\textit{context:} \\ \textbf{embedded}};

\node[nods] (l21) at (9,2.2) {\textit{form:} \textbf{source}};
\node[nods] (l22) at (10.8,2.2) {\textit{form:} \textbf{binary}};
\node[nods] (l23) at (12.6,2.2) {\textit{form:} \textbf{source}};
\node[nods] (l24) at (14.4,2.2) {\textit{form:} \textbf{binary}};

\node[leaf] (l11) at (0,0) {\textbf{ApL-1} \textit{using software only
for yourself}};

\node[leaf] (l12) at (1.8,0) { \textbf{ApL-2} \textit{ distributing unmodified
software as sources}};

\node[leaf] (l13) at (3.6,0) { \textbf{ApL-3}  \textit{ distributing unmodified
software as binaries}};

\node[leaf] (l14) at (5.4,0) { \textbf{ApL-4}  \textit{ distributing modified
program as sources}};

\node[leaf] (l15) at (7.2,0) { \textbf{ApL-5}  \textit{ distributing modified
program as binaries}};

\node[leaf] (l16) at (9,0) { \textbf{ApL-6}  \textit{ distributing modified
library as independent sources}};

\node[leaf] (l17) at (10.8,0) { \textbf{ApL-7} \textit{distributing modified
library as independent binaries}};

\node[leaf] (l18) at (12.6,0) { \textbf{ApL-8}  \textit{distributing
modified library as embedded sources}};

\node[leaf] (l19) at (14.4,0) { \textbf{ApL-9}  \textit{ distributing modified
library as embedded binaries}};


\path [edge] (l71) -- (l61);
\path [edge] (l71) -- (l62);
\path [edge] (l61) -- (l11);
\path [edge] (l62) -- (l51);
\path [edge] (l62) -- (l52);
\path [edge] (l51) -- (l41);
\path [edge] (l51) -- (l42);
\path [edge] (l52) -- (l43);
\path [edge] (l52) -- (l44);
\path [edge] (l41) -- (l12);
\path [edge] (l42) -- (l13);
\path [edge] (l43) -- (l31);
\path [edge] (l43) -- (l32);
\path [edge] (l44) -- (l33);
\path [edge] (l44) -- (l34);
\path [edge] (l31) -- (l14);
\path [edge] (l32) -- (l15);
\path [edge] (l33) -- (l21);
\path [edge] (l33) -- (l22);
\path [edge] (l34) -- (l23);
\path [edge] (l34) -- (l24);
\path [edge] (l21) -- (l16);
\path [edge] (l22) -- (l17);
\path [edge] (l23) -- (l18);
\path [edge] (l24) -- (l19);

\end{tikzpicture}


\subsection{ApL-1: Using the software only for yourself}
\label{OSUC-01-Apache20} \label{OSUC-03-Apache20} 
\label{OSUC-06-Apache20} \label{OSUC-09-Apache20}

\begin{description}
\item[means] that you are going to use a received Apache licensed software only
for yourself and that you do not handover it to any 3rd party in any sense.
\item[covers] OSUC-01, OSUC-03, OSUC-06, and OSUC-09\footnote{For details see
pp. \pageref{OSUC-01-DEF} - \pageref{OSUC-09-DEF}}
\item[requires] no tasks in order to fulfill the conditions of the Apache 2.0
license with respect to this use case:
  \begin{itemize}
    \item You are allowed to use any kind of Apache software in any sense and in
    any context without any obligations premised you do not handover the
    software to 3rd parties.
  \end{itemize}
\item[prohibits] nothing explicitly.
\end{description}

\subsection{ApL-2: Passing the unmodified software as source code}
\label{OSUC-02-Apache20} \label{OSUC-05-Apache20} \label{OSUC-07-Apache20} 

\begin{description}
\item[means] that you are going to distribute an unmodified version of the
received Apache software to 3rd parties in form of a set of source code files or
an integrated source code package\footnote{In this case it doesn't matter
whether you  distribute a program, an application, a server, a snippet, a
module, a library, or a plugin as an independent or an embedded unit}

\item[covers] OSUC-02, OSUC-05, OSUC-07\footnote{For details see pp.\ 
\pageref{OSUC-02-DEF} - \pageref{OSUC-07-DEF}}

\item[requires] the following tasks in order to fulfill the license conditions
\begin{itemize}
  \item \textbf{[mandatory:]} Give the recipient a copy of the Apache 2.0
  license. If it is still not incorporated into the software package, add
  it\footnote{For implementing the handover of files correctly $\rightarrow$
  OSLiC \pageref{DistributingFilesHint}}.
  \item \textbf{[mandatory:]} Ensure that the licensing elements -- esp.\ the
  specific copyright notice of the original author(s) -- are retained in your
  package in the form you have received them.
  \item \textbf{[mandatory:]} Ensure that a NOTICE text file is retained in
  your package in the form you have received it.
  \item \textbf{[voluntary:]} Let the documentation of your distribution
  and/or your additional material also reproduce the content of the NOTICE text
  file, a hint to the software name, a link to its homepage, and a link to the
  Apache 2.0 license.
\end{itemize}

\item[prohibits] to institute any patent litigation against anyone alleging that
the software constitutes patent infringement.

\end{description}


\subsection{ApL-3: Passing the unmodified software as binaries} 

\begin{description}
\item[means] that you are going to distribute an unmodified version of the
received Apache software to 3rd parties in form of a set of binary files or an
integrated bi\-na\-ry package\footnote{In this case it doesn't matter
whether you  distribute a program, an application, a server, a snippet, a
module, a library, or a plugin as an independent or an embedded unit}

\item[covers] OSUC-02, OSUC-05, OSUC-07\footnote{For details see pp.
\pageref{OSUC-02-DEF} - \pageref{OSUC-07-DEF}}

\item[requires] the following tasks in order to fulfill the license conditions
\begin{itemize}
  \item \textbf{[mandatory:]} Give the recipient a copy of the Apache 2.0
  license. If it is still not incorporated into your binary package, add
  it\footnote{For implementing the handover of files correctly $\rightarrow$
  OSLiC \pageref{DistributingFilesHint}}.
  
  \item \textbf{[mandatory:]} Ensure that the licensing elements -- esp.\ the
  specific copyright notice of the original author(s) -- are retained in your
  package in the form you have received them. If you compile the binary from the
  sources, ensure that all the licensing elements are also incorprated into the
  package.
  \item \textbf{[mandatory:]} Ensure that the NOTICE text file is retained or
  integrated into your binary package in the form you have initially received
  it.
  \item \textbf{[mandatory:]} Ensure that the NOTICE text file is also
  reproduced if and whereever such third-party notices normally appear --
  especially, if you are distributing an unmodified, Apache licensed library as
  embedded component of your own work which displays its own copyright notice.
  
  \item \textbf{[voluntary:]} Let the documentation of your distribution
  and/or your additional material also reproduce the content of the NOTICE text
  file, a hint to the software name, a link to its homepage, and a link to the
  Apache 2.0 license -- especially as subsection of your own copyright notice.
\end{itemize}

\item[prohibits] to institute any patent litigation against anyone alleging that
the software constitutes patent infringement.

\end{description}

\subsection{ApL-4: Passing a modified program as source code}
\label{OSUC-04-Apache20} 

\begin{description}
\item[means] that you are going to distribute a modified version of the received
Apache licensed program, application, or server (proapse) to 3rd parties in form
of a set of source code files or an integrated source code package.
\item[covers] OSUC-04\footnote{For details see pp.\ \pageref{OSUC-04-DEF}}
\item[requires] the tasks in order to fulfill the license conditions
\begin{itemize}
  
  \item \textbf{[mandatory:]} Give the recipient a copy of the Apache 2.0
  license. If it is still not incorporated into the source code package, add
  it\footnote{For implementing the handover of files correctly $\rightarrow$
  OSLiC \pageref{DistributingFilesHint}}.

  \item \textbf{[mandatory:]} Ensure that the licensing elements -- esp.\ the
  specific copyright notice of the original author(s) -- are retained in your
  package in the form you have received them.
  
  \item \textbf{[mandatory:]} Ensure that the NOTICE text file contains at least
  all the information of that NOTICE text file you have received.

  \item \textbf{[mandatory:]} Ensure that the NOTICE text file is also
  reproduced if and whereever such third-party notices normally appear. If the
  program already displays a copyright dialog, update it in an appropriate
  manner.
  
  \item \textbf{[voluntary:]} Inside of the source code, mark all your
  modifications thoroughly. Generate a NOTICE text file, if it still does not
  exist. Expand (sic!) the NOTICE text file by a description of your
  modifications.
   
  \item \textbf{[voluntary:]} Let the documentation of your distribution
  and/or your additional material also reproduce the content of the NOTICE text
  file, a hint to the software name, a link to its homepage, and a link to the
  Apache 2.0 license.
  
 \end{itemize}
 
\item[prohibits] to institute any patent litigation against anyone alleging that
the software constitutes patent infringement.

\end{description}

\subsection{ApL-5: Passing a modified program as binary}

\begin{description}
\item[means] that you are going to distribute a modified version of the received
Apache licensed pro\-gram, application, or server (proapse) to 3rd parties in
form of a set of binary files or an integrated binary package.
\item[covers] OSUC-04\footnote{For details see pp.\ \pageref{OSUC-04-DEF}}
\item[requires] the tasks in order to fulfill the license conditions
\begin{itemize}

 \item \textbf{[mandatory:]} Give the recipient a copy of the Apache 2.0
  license. If it is still not incorporated into your binary package, add
  it\footnote{For implementing the handover of files correctly $\rightarrow$
  OSLiC \pageref{DistributingFilesHint}}.
  
  \item \textbf{[mandatory:]} Ensure that the licensing elements -- esp.\ the
  specific copyright notice of the original author(s) -- are retained in your
  package in the form you have received them. If you compile the binary from the
  sources, ensure that all the licensing elements are also incorprated into the
  package.
  
  \item \textbf{[mandatory:]} Ensure that the NOTICE text file contains at least
  all the information of that NOTICE text file you have received.
  
  \item \textbf{[mandatory:]} Ensure that the NOTICE text file is also
  reproduced if and whereever such third-party notices normally appear. If the
  program already displays a copyright dialog, update it in an appropriate
  manner.
 
  \item \textbf{[voluntary:]} Inside of the source code, mark all your
  modifications thoroughly. Generate a NOTICE text file, if it still does not
  exist. Expand (sic!) the NOTICE text file by a description of your
  modifications.
 
  \item \textbf{[voluntary:]} Let the documentation of your distribution
  and/or your additional material also reproduce the content of the NOTICE text
  file, a hint to the software name, a link to its homepage, and a link to the
  Apache 2.0 license -- especially as a subsection of your own copyright notice.

\end{itemize}

\item[prohibits] to institute any patent litigation against anyone alleging that
the software constitutes patent infringement.

\end{description}

\subsection{ApL-6: Passing a modified library as independent source code}
\label{OSUC-08-Apache20}

\begin{description}
\item[means] that you are going to distribute a modified version of the received
Apache licensed code snippet, module, library, or plugin (snimoli) to 3rd
parties in form of a set of source code files or an integrated source code
package, but without embedding it into another larger software unit.
\item[covers] OSUC-08\footnote{For details see pp.\ \pageref{OSUC-08-DEF}}
\item[requires] the tasks in order to fulfill the license conditions
\begin{itemize}
  
  \item \textbf{[mandatory:]} Give the recipient a copy of the Apache 2.0
  license. If it is still not incorporated into the software package, add
  it\footnote{For implementing the handover of files correctly $\rightarrow$
  OSLiC \pageref{DistributingFilesHint}}.

  \item \textbf{[mandatory:]} Ensure that the licensing elements -- esp.\ the
  specific copyright notice of the original author(s) -- are retained in your
  package in the form you have received them.
  
  \item \textbf{[mandatory:]} Ensure that the NOTICE text file contains at least
  all the information of that NOTICE text file you have received.
 
  \item \textbf{[voluntary:]} Inside of the source code, mark all your
  modifications thoroughly. Generate a NOTICE text file, if it still does not
  exist. Expand (sic!) the NOTICE text file by a description of your
  modifications.
   
  \item \textbf{[voluntary:]} Let the documentation of your distribution
  and/or your additional material also reproduce the content of the NOTICE text
  file, a hint to the software name, a link to its homepage, and a link to the
  Apache 2.0 license.

\end{itemize}

\item[prohibits] to institute any patent litigation against anyone alleging that
the software constitutes patent infringement.

\end{description}


\subsection{ApL-7: Passing a modified library as independent binary}

\begin{description}
\item[means] that you are going to distribute a modified version of the received
Appache licensed code snippet, module, library, or plugin (snimoli) to 3rd
parties in form of a set of binary files or an integrated binary package but
without embedding it into another larger software unit.
\item[covers] OSUC-08\footnote{For details see pp.\ \pageref{OSUC-08-DEF}}
\item[requires] the tasks in order to fulfill the license conditions
\begin{itemize}
  
 \item \textbf{[mandatory:]} Give the recipient a copy of the Apache 2.0
  license. If it is still not incorporated into your binary package, add
  it\footnote{For implementing the handover of files correctly $\rightarrow$
  OSLiC \pageref{DistributingFilesHint}}.
  
  \item \textbf{[mandatory:]} Ensure that the licensing elements -- esp.\ the
  specific copyright notice of the original author(s) -- are retained in your
  package in the form you have received them. If you compile the binary from the
  sources, ensure that all the licensing elements are also incorprated into the
  package.
  
  \item \textbf{[mandatory:]} Ensure that the NOTICE text file contains at least
  all the information of that NOTICE text file you have received.
   
  \item \textbf{[voluntary:]} Inside of the source code, mark all your
  modifications thoroughly. Generate a NOTICE text file, if it still does not
  exist. Expand (sic!) the NOTICE text file by a description of your
  modifications.
 
  \item \textbf{[voluntary:]} Let the documentation of your distribution
  and/or your additional material also reproduce the content of the NOTICE text
  file, a hint to the software name, a link to its homepage, and a link to the
  Apache 2.0 license -- especially as a subsection of your own copyright notice.
  
\end{itemize}

\item[prohibits] to institute any patent litigation against anyone alleging that
the software constitutes patent infringement.

\end{description}

\subsection{ApL-8: Passing a modified library as embedded source code}
\label{OSUC-10-Apache20}

\begin{description}
\item[means] that you are going to distribute a modified version of the received
Apache licensed code snippet, module, library, or plugin (snimoli) to 3rd
parties in form of a set of source code files or an integrated source code
package together with another larger software unit which contains this code
snippet, module, library, or plugin as an embedded component.
\item[covers] OSUC-10\footnote{For details see pp.\ \pageref{OSUC-10-DEF}}
\item[requires] the tasks in order to fulfill the license conditions
\begin{itemize}
  
  \item \textbf{[mandatory:]} Give the recipient a copy of the Apache 2.0
  license. If it is still not incorporated into the software package, add
  it\footnote{For implementing the handover of files correctly $\rightarrow$
  OSLiC \pageref{DistributingFilesHint}}.

  \item \textbf{[mandatory:]} Ensure that the licensing elements -- esp.\ the
  specific copyright notice of the original author(s) -- are retained in your
  package in the form you have received them.
  
  \item \textbf{[mandatory:]} Ensure that the NOTICE text file contains at least
  all the information of that NOTICE text file you have received.
 
  \item \textbf{[mandatory:]} Ensure that the NOTICE text file is also
  reproduced if and whereever such third-party notices normally appear. If your
  overarching program displays an own copyright dialog, insert these
  information.
 
  \item \textbf{[voluntary:]} Inside of the source code, mark all your
  modifications thoroughly. Generate a NOTICE text file, if it still does not
  exist. Expand (sic!) the NOTICE text file by a description of your
  modifications.
  
  \item \textbf{[voluntary:]} Let the documentation of your distribution
  and/or your additional material also reproduce the content of the NOTICE text
  file, a hint to the software name, a link to its homepage, and a link to the
  Apache 2.0 license.

  \item \textbf{[voluntary:]} Arrange your source code distribution so that the
  integrated Apache license and the NOTICE text file clearly refers only to the
  embedded library and does not disturb the licensing of your own overarching
  work. It's a good tradition to keep the embedded components like libraries,
  modules, snippets, or plugins in specific directory which contains also all
  additional licensing elements.
 
\end{itemize}

\item[prohibits] to institute any patent litigation against anyone alleging that
the software constitutes patent infringement.

\end{description}


\subsection{ApL-9: Passing a modified library as embedded binary}

\begin{description}
\item[means] that you are going to distribute a modified version of the received
Apache licensed code snippet, module, library, or plugin to 3rd parties in form
of a set of binary files or an integrated binary package together with another
larger software unit which contains this code snippet, module, library, or
plugin as an embedded component.
\item[covers] OSUC-10\footnote{For details see pp.\ \pageref{OSUC-10-DEF}}
\item[requires] the tasks in order to fulfill the license conditions
\begin{itemize}
  
  \item \textbf{[mandatory:]} Give the recipient a copy of the Apache 2.0
  license. If it is still not incorporated into your binary package, add
  it\footnote{For implementing the handover of files correctly $\rightarrow$
  OSLiC \pageref{DistributingFilesHint}}.
  
  \item \textbf{[mandatory:]} Ensure that the licensing elements -- esp.\ the
  specific copyright notice of the original author(s) -- are retained in your
  package in the form you have received them. If you compile the binary from the
  sources, ensure that all the licensing elements are also incorprated into the
  package.
  
  \item \textbf{[mandatory:]} Ensure that the NOTICE text file contains at least
  all the information of that NOTICE text file you have received.
 
  \item \textbf{[mandatory:]} Ensure that the NOTICE text file is also
  reproduced if and whereever such third-party notices normally appear. If your
  overarching program displays an own copyright dialog, insert these
  information.
     
  \item \textbf{[voluntary:]} Inside of the source code, mark all your
  modifications thoroughly. Generate a NOTICE text file, if it still does not
  exist. Expand (sic!) the NOTICE text file by a description of your
  modifications.
 
  \item \textbf{[voluntary:]} Let the documentation of your distribution
  and/or your additional material also reproduce the content of the NOTICE text
  file, a hint to the software name, a link to its homepage, and a link to the
  Apache 2.0 license -- especially as subsection of your own copyright notice.
  

 \item \textbf{[voluntary:]} Arrange your binary distribution so that the
  integrated Apache license and the NOTICE text file clearly refers only to the
  embedded library and does not disturb the licensing of your own overarching
  work. It's a good tradition to keep the librabries, modules, snippet, or
  plugins in specific directiers which contain also all licensing elements.
  
\end{itemize}

\item[prohibits] to institute any patent litigation against anyone alleging that
the software constitutes patent infringement.

\end{description}

\subsection{Discussions and Explanations [tbd]}







%\bibliography{../../../bibfiles/oscResourcesEn}

% Telekom osCompendium 'for being included' snippet template
%
% (c) Karsten Reincke, Deutsche Telekom AG, Darmstadt 2011
%
% This LaTeX-File is licensed under the Creative Commons Attribution-ShareAlike
% 3.0 Germany License (http://creativecommons.org/licenses/by-sa/3.0/de/): Feel
% free 'to share (to copy, distribute and transmit)' or 'to remix (to adapt)'
% it, if you '... distribute the resulting work under the same or similar
% license to this one' and if you respect how 'you must attribute the work in
% the manner specified by the author ...':
%
% In an internet based reuse please link the reused parts to www.telekom.com and
% mention the original authors and Deutsche Telekom AG in a suitable manner. In
% a paper-like reuse please insert a short hint to www.telekom.com and to the
% original authors and Deutsche Telekom AG into your preface. For normal
% quotations please use the scientific standard to cite.
%
% [ Framework derived from 'mind your Scholar Research Framework' 
%   mycsrf (c) K. Reincke 2012 CC BY 3.0  http://mycsrf.fodina.de/ ]
%


%% use all entries of the bibliography
%\nocite{*}

\section{BSD Licensed Software \ldots}

As an approved open source license, the BSD license exists in two
versions\footcite[Following the OSI, there is another 'ancient' 
BSD license -- containing a fourth clause known as advertising clause -- which
\enquote{(\ldots) officially was rescinded by the Director of the Office of
Technology Licensing of the University of California on July 22nd, 1999}.
 Cf.][\nopage wp. Because of that cancellation you can simply act according the
 \textit{BSD 3-Clause license} if you have to fulfill the eldest BSD
 license]{BsdLicense3Clause}. The latest release is the \textit{BSD 2-Clause
 license}\footcite[cf.][\nopage wp]{BsdLicense2Clause}, the elder release is the
 \textit{BSD 3-Clause license}\footcite[cf.][\nopage wp]{BsdLicense3Clause}.
 The very little differences between the two versions have to be respected
 exactly. Nevertheless, we could integrate the requirements into one to-do list
 per use case.

Explicitly, all BSD open source licenses 'only' focus on the (re-)distribution
\textit{open source use cases} which we have specified by our token
\textit{4others}. Conditions for the other use cases specified by the token
\textit{4yourself} can be derived\footnote{For details of the \textit{open
source use case tokens} see p.\ \pageref{OsucTokens}. For Details of the
\textit{open source use cases} based on these token see p.
\pageref{OsucDefinitionTree} }. Additionally the BSD license considers the form
of the distribution, esp.\ whether the work is distributed as a (set of) source
code file(s) or as a (set of) the binary file(s). Use the following tree to find
the BSD license fulfilling to-do lists.

\tikzstyle{nodv} = [font=\small, ellipse, draw, fill=gray!10, 
    text width=2cm, text centered, minimum height=2em]


\tikzstyle{nods} = [font=\footnotesize, rectangle, draw, fill=gray!20, 
    text width=1.2cm, text centered, rounded corners, minimum height=3em]

\tikzstyle{nodb} = [font=\footnotesize, rectangle, draw, fill=gray!20, 
    text width=2.2cm, text centered, rounded corners, minimum height=3em]
    
\tikzstyle{leaf} = [font=\tiny, rectangle, draw, fill=gray!30, 
    text width=1.2cm, text centered, minimum height=6em]

\tikzstyle{edge} = [draw, -latex']

\begin{tikzpicture}[]

\node[nodv] (l81) at (4,11.8) {BSD};

\node[nodv] (l71) at (0,10.2) {3-Clause License};
\node[nodv] (l72) at (6.5,10.2) {2-Clause License};


\node[nodb] (l61) at (0,8.6) {\textit{recipient:} \\ \textbf{4yourself}};
\node[nodb] (l62) at (6.5,8.6) {\textit{recipient:} \\ \textbf{4others}};

\node[nodb] (l51) at (2.5,7) {\textit{state:} \\ \textbf{unmodified}};
\node[nodb] (l52) at (9.3,7) {\textit{state:} \\ \textbf{modified}};

\node[nods] (l41) at (1.8,5.4) {\textit{form:} \textbf{source}};
\node[nods] (l42) at (3.6,5.4) {\textit{form:} \textbf{binary}};
\node[nodb] (l43) at (6.5,5.4) {\textit{type:} \\ \textbf{proapse}};
\node[nodb] (l44) at (12,5.4) {\textit{type:} \\ \textbf{snimoli}};


\node[nods] (l31) at (5.4,3.8) {\textit{form:} \textbf{source}};
\node[nods] (l32) at (7.2,3.8) {\textit{form:} \textbf{binary}};
\node[nodb] (l33) at (10,3.8) {\textit{context:} \\ \textbf{independent}};
\node[nodb] (l34) at (13.5,3.8) {\textit{context:} \\ \textbf{embedded}};

\node[nods] (l21) at (9,2.2) {\textit{form:} \textbf{source}};
\node[nods] (l22) at (10.8,2.2) {\textit{form:} \textbf{binary}};
\node[nods] (l23) at (12.6,2.2) {\textit{form:} \textbf{source}};
\node[nods] (l24) at (14.4,2.2) {\textit{form:} \textbf{binary}};

\node[leaf] (l11) at (0,0) {\textbf{BSD-1} \textit{using software only
for yourself}};

\node[leaf] (l12) at (1.8,0) { \textbf{BSD-2} \textit{ distributing unmodified
software as sources}};

\node[leaf] (l13) at (3.6,0) { \textbf{BSD-3}  \textit{ distributing unmodified
software as binaries}};

\node[leaf] (l14) at (5.4,0) { \textbf{BSD-4}  \textit{ distributing modified
program as sources}};

\node[leaf] (l15) at (7.2,0) { \textbf{BSD-5}  \textit{ distributing modified
program as binaries}};

\node[leaf] (l16) at (9,0) { \textbf{BSD-6}  \textit{ distributing modified
library as independent sources}};

\node[leaf] (l17) at (10.8,0) { \textbf{BSD-7} \textit{distributing modified
library as independent binaries}};

\node[leaf] (l18) at (12.6,0) { \textbf{BSD-8}  \textit{distributing
modified library as embedded sources}};

\node[leaf] (l19) at (14.4,0) { \textbf{BSD-9}  \textit{ distributing modified
library as embedded binaries}};

\path [edge] (l81) -- (l71);
\path [edge] (l81) -- (l72);
\path [edge] (l71) -- (l61);
\path [edge] (l71) -- (l62);
\path [edge] (l72) -- (l61);
\path [edge] (l72) -- (l62);
\path [edge] (l61) -- (l11);
\path [edge] (l62) -- (l51);
\path [edge] (l62) -- (l52);
\path [edge] (l51) -- (l41);
\path [edge] (l51) -- (l42);
\path [edge] (l52) -- (l43);
\path [edge] (l52) -- (l44);
\path [edge] (l41) -- (l12);
\path [edge] (l42) -- (l13);
\path [edge] (l43) -- (l31);
\path [edge] (l43) -- (l32);
\path [edge] (l44) -- (l33);
\path [edge] (l44) -- (l34);
\path [edge] (l31) -- (l14);
\path [edge] (l32) -- (l15);
\path [edge] (l33) -- (l21);
\path [edge] (l33) -- (l22);
\path [edge] (l34) -- (l23);
\path [edge] (l34) -- (l24);
\path [edge] (l21) -- (l16);
\path [edge] (l22) -- (l17);
\path [edge] (l23) -- (l18);
\path [edge] (l24) -- (l19);

\end{tikzpicture}

\subsection{BSD-1: Using the software only for yourself}
\label{OSUC-01-BSD} 
\label{OSUC-03-BSD} 
\label{OSUC-06-BSD}
\label{OSUC-09-BSD}
  
\begin{description}
\item[means] that you are going to use a received BSD software only for yourself
and that you do not handover it to any 3rd party in any sense.
\item[covers] OSUC-01, OSUC-03, OSUC-06, and OSUC-09\footnote{For details see pp.
  \pageref{OSUC-01-DEF} - \pageref{OSUC-09-DEF}}
\item[requires] no tasks in order to fulfill the conditions of the BSD license
with respect to this use case:
  \begin{itemize}
    \item You are allowed to use any kind of BSD software in any sense and in
    any context without any obligations if you do not handover the software to
    3rd parties.
  \end{itemize}
\item[prohibits] nothing explicitly.
\end{description}


\subsection{BSD-2: Passing the unmodified software as source code}
\label{OSUC-02-BSD} \label{OSUC-05-BSD} \label{OSUC-07-BSD} 

\begin{description}
\item[means] that you are going to distribute an unmodified version of the received
BSD software to 3rd parties in form of a set of source code files or an
integrated source code package\footnote{In this case it doesn't matter whether
you  distribute a program, an application, a server, a snippet, a module, a
library, or a plugin as an independent or an embedded unit} 

\item[covers] OSUC-02, OSUC-05, OSUC-07\footnote{For details see pp.
\pageref{OSUC-02-DEF} - \pageref{OSUC-07-DEF}}

\item[requires] the following tasks in order to fulfill the license conditions
\begin{itemize}
  \item \textbf{[mandatorily:]} Ensure that the licensing elements -- esp.\
  the BSD license text, the specific copyright notice of the original author(s),
  and the BSD disclaimer -- are retained in your package in the form you have got
  them.
  \item \textbf{[voluntarily:]} Let the documentation of your distribution
  and/or your additional material also contain the original copyright notice, the
  BSD conditions, and the BSD disclaimer.
\end{itemize}

\item[prohibits] nothing explictly, if you are using the \emph{BSD 2 Clause
License}. But the \emph{BSD 3 Clause License} prohibits the following doings in
order to fulfill the license
\begin{itemize}
  \item \textbf{[explicitly:]} Do not use the name of the licensing organization
  or the names of the licensing distributors to promote your own work
\end{itemize}

\end{description}

\subsection{BSD-3: Passing the unmodified software as binary}

\begin{description}
\item[means] that you are going to distribute an unmodified version of the BSD
received software to 3rd parties in form of a set of binary files or an
integrated bi\-na\-ry package\footnote{In this case it doesn't matter whether
you distribute a program, an application, a server, a snippet, a module, a library,
or a plugin as an independent or an embedded unit}
\item[covers] OSUC-02, OSUC-05, OSUC-07\footnote{For details see pp.
\pageref{OSUC-02-DEF} - \pageref{OSUC-07-DEF}}
\item[requires] the tasks in order to fulfill the license conditions
\begin{itemize}
  \item  \textbf{[mandatory:]} Ensure that your distribution contains the
  original copyright notice, the BSD license, and the BSD disclaimer in the form
  you have got them. If you compile the binary file on the base of the source
  code package and if this compilation does not also generate and integrate the
  licensing files then create the copyright notice, the BSD conditions, and the
  BSD disclaimer according to the form of the source code package and insert
  these files into your distribution manually.
  \item  \textbf{[mandatory:]} Ensure that the documentation of your
  distribution and/or your additional material also contain the author specific
  copyright notice, the BSD conditions, and the BSD disclaimer.
\end{itemize}
\item[prohibits] nothing explictly, if you are using the \emph{BSD 2 Clause
License}. But the \emph{BSD 3 Clause License} prohibits the following doings in
order to fulfill the license
\begin{itemize}
  \item \textbf{[explicitly:]} Do not use the name of the licensing organization
  or the names of the licensing distributors to promote your own work
\end{itemize}
\end{description}

\begin{itshape}
\emph{\textbf{General remark for all binary distributions}:} 
\label{MobileDeviceHint} Sometimes you probably want to distributing a BSD
bi\-na\-ry package on a medium which doesn't allow the user, to see package
files directly -- some mobile devices don't give their users the full access to
all stored elements. But the requirement, 'to give some one the BSD license and
the copyright notes' includes that he must be able to read it\footnote{to give
someone anything he can't touch, feel, see etc., is like not giving him the
object ;-)}. Hence, on systems which offer a file browser, it is sufficient, to
put these file onto the files system. On the other systems, you \emph{must}
present the content of the files by your application -- maybe in a specific
copyright dialog. But in the open source community, it is a good tradition, to
present these reference data voluntarily.

This is often done by presenting links to general versions of these licensing
files etc.\ Be aware: Although it is a good tradition -- especially iy you link
to the homepages of the projects for being totally transparent -- it is not
sufficient to offer only links. If you are required by the open source licenses
to handover something to your users, you (sic!) must do it\footnote{The
advantage of doing the job oneself is that one has not to struggle with
uncommunicated implicite modfications of the link targets.}. The point is: you
have to fulfill the license.
\end{itshape}

\subsection{BSD-4: Passing a modified program as source code}
\label{OSUC-04-BSD}

\begin{description}
\item[means] that you are going to distribute a modified version of the received
BSD program, application, or server (proapse) to 3rd parties in form of a set
of source code files or an integrated source code package.
\item[covers] OSUC-04\footnote{For details see pp.\ \pageref{OSUC-04-DEF}}
\item[requires] the tasks in order to fulfill the license conditions
\begin{itemize}
  \item \textbf{[mandatory:]} Ensure that the licensing elements -- esp.\
  the BSD license text, the specific copyright notice of the original author(s),
  and the BSD disclaimer -- are retained in your package in the form you have got
  them. 
  \item \textbf{[voluntary:]} Let the documentation of your distribution
  and/or your additional material also contain the original copyright notice, the
  BSD conditions, and the BSD disclaimer.
  
  \item \textbf{[voluntary:]} It is a good practice of the open source
  community, to let the copyright notice which is shown by the running program
  also state that the program is licensed under the BSD license. Because you are
  already modifying the program you can also add such a hint if the presented
  original copyright notice lacks such a statement.
\end{itemize}
\item[prohibits] nothing explictly, if you are using the \emph{BSD 2 Clause
License}. But the \emph{BSD 3 Clause License} prohibits the following doings in
order to fulfill the license
\begin{itemize}
  \item \textbf{[explicitly:]} Do not use the name of the licensing organization
  or the names of the licensing distributors to promote your own work
\end{itemize}
\end{description}

\subsection{BSD-5: Passing a modified program as binary}

\begin{description}
\item[means] that you are going to distribute a modified version of the received
BSD pro\-gram, application, or server (proapse) to 3rd parties in form of a set
of binary files or an integrated binary package.
\item[covers] OSUC-04\footnote{For details see pp.\ \pageref{OSUC-04-DEF}}
\item[requires] the tasks in order to fulfill the license conditions
\begin{itemize}

  \item  \textbf{[mandatory:]} Ensure that your distribution contains the
  original copyright notice, the BSD license, and the BSD disclaimer in the form
  you have got them. If you compile the binary file on the base of the source
  code package and if this compilation does not also generate and integrate the
  licensing files then create the copyright notice, the BSD conditions, and the
  BSD disclaimer according to the form of the source code package and insert
  these files into your distribution manually\footnote{see also our 'Mobile
  Device Hint' on p.\ \pageref{MobileDeviceHint}}.

  \item  \textbf{[mandatory:]} Ensure that the documentation of your
  distribution and/or your additional material also contain the author specific
  copyright notice, the BSD conditions, and the BSD disclaimer.
  
  \item \textbf{[voluntary:]} It is a good practice of the open source
  community, to let the copyright notice which is shown by the running program
  also state that the program is licensed under the BSD license. Because you are
  already modifying the program you can also add such a hint if the presented
  original copyright notice lacks such a statement.
\end{itemize}
\item[prohibits] nothing explictly, if you are using the \emph{BSD 2 Clause
License}. But the \emph{BSD 3 Clause License} prohibits the following doings in
order to fulfill the license
\begin{itemize}
  \item \textbf{[explicitly:]} Do not use the name of the licensing organization
  or the names of the licensing distributors to promote your own work
\end{itemize}
\end{description}

\subsection{BSD-6: Passing a modified library as independent source code}
\label{OSUC-08-BSD}
\begin{description}
\item[means] that you are going to distribute a modified version of the received
BSD code snippet, module, library, or plugin (snimoli) to 3rd parties in form
of a set of source code files or an integrated source code package, but without
embedding it into another larger software unit.
\item[covers] OSUC-08\footnote{For details see pp.\ \pageref{OSUC-08-DEF}}
\item[requires] the tasks in order to fulfill the license conditions
\begin{itemize}
  \item \textbf{[mandatory:]} Ensure that the licensing elements -- esp.\
  the BSD license text, the specific copyright notice of the original author(s),
  and the BSD disclaimer -- are retained in your package in the form you have got
  them.
  \item \textbf{[voluntary:]} Let the documentation of your distribution
  and/or your additional material also contain the original copyright notice, the
  BSD conditions, and the BSD disclaimer.
\end{itemize}
\item[prohibits] nothing explictly, if you are using the \emph{BSD 2 Clause
License}. But the \emph{BSD 3 Clause License} prohibits the following doings in
order to fulfill the license
\begin{itemize}
  \item \textbf{[explicitly:]} Do not use the name of the licensing organization
  or the names of the licensing distributors to promote your own work
\end{itemize}
\end{description}


\subsection{BSD-7: Passing a modified library as independent binary}

\begin{description}
\item[means] that you are going to distribute a modified version of the received
BSD code snippet, module, library, or plugin (snimoli) to 3rd parties in form
of a set of binary files or an integrated binary package but without embedding
it into another larger software unit.
\item[covers] OSUC-08\footnote{For details see pp.\ \pageref{OSUC-08-DEF}}
\item[requires] the tasks in order to fulfill the license conditions
\begin{itemize}
   \item  \textbf{[mandatory:]} Ensure that your distribution contains the
  original copyright notice, the BSD license, and the BSD disclaimer in the form
  you have got them. If you compile the binary file on the base of the source
  code package and if this compilation does not also generate and integrate the
  licensing files, then create the copyright notice, the BSD conditions, and the
  BSD disclaimer according to the form of the source code package and insert
  these files into your distribution manually\footnote{see also our 'Mobile
  Device Hint' on p.\ \pageref{MobileDeviceHint}}.
  \item  \textbf{[mandatory:]} Ensure that the documentation of your
  distribution and/or your additional material also contain the author specific
  copyright notice, the BSD conditions, and the BSD disclaimer.
\end{itemize}
\item[prohibits] nothing explictly, if you are using the \emph{BSD 2 Clause
License}. But the \emph{BSD 3 Clause License} prohibits the following doings in
order to fulfill the license
\begin{itemize}
  \item \textbf{[explicitly:]} Do not use the name of the licensing organization
  or the names of the licensing distributors to promote your own work
\end{itemize}
\end{description}

\subsection{BSD-8: Passing a modified library as embedded source code}
\label{OSUC-10-BSD}
\begin{description}
\item[means] that you are going to distribute a modified version of the received
BSD code snippet, module, library, or plugin (snimoli) to 3rd parties in form
of a set of source code files or an integrated source code package together with
another larger software unit which contains this code snippet, module, library,
or plugin as an embedded component.
\item[covers] OSUC-10\footnote{For details see pp.\ \pageref{OSUC-10-DEF}}
\item[requires] the tasks in order to fulfill the license conditions
\begin{itemize}
  \item \textbf{[mandatory:]} Ensure that the licensing elements -- esp.\
  the BSD license text, the specific copyright notice of the original author(s),
  and the BSD disclaimer -- are retained in your package in the form you have got
  them.
  \item \textbf{[voluntary:]} Let the documentation of your distribution
  and/or your additional material also contain the original copyright notice, the
  BSD conditions, and the BSD disclaimer.
 \item \textbf{[voluntary:]} It is a good practice of the open source
  community, to let the copyright notice which is shown by the running program
  also state that it contains components licensed under the BSD license. Because
  you are embedding this snimoli into a larger software unit, you are
  developing this larger unit. Hence, you can also expand the copyright notice
  of this larger unit by such a hint to its BSD components.
\end{itemize}
\item[prohibits] nothing explictly, if you are using the \emph{BSD 2 Clause
License}. But the \emph{BSD 3 Clause License} prohibits the following doings in
order to fulfill the license
\begin{itemize}
  \item \textbf{[explicitly:]} Do not use the name of the licensing organization
  or the names of the licensing distributors to promote your own work
\end{itemize}
\end{description}


\subsection{BSD-9: Passing a modified library as embedded binary}

\begin{description}
\item[means] that you are going to distribute a modified version of the received
BSD code snippet, module, library, or plugin to 3rd parties in form of a set of
binary files or an integrated binary package together with another larger
software unit which contains this code snippet, module, library, or plugin as
an embedded component.
\item[covers] OSUC-10\footnote{For details see pp.\ \pageref{OSUC-10-DEF}}
\item[requires] the tasks in order to fulfill the license conditions
\begin{itemize}
  \item  \textbf{[mandatory:]} Ensure that your distribution contains the
  original copyright notice, the BSD license, and the BSD disclaimer in the form
  you have got them. If you compile the binary file on the base of the source
  code package and if this compilation does not also generate and integrate the
  licensing files, then create the copyright notice, the BSD conditions, and the
  BSD disclaimer according to the form of the source code package and insert
  these files into your distribution manually\footnote{see also our 'Mobile
  Device Hint' on p.\ \pageref{MobileDeviceHint}}.
  \item  \textbf{[mandatory:]} Ensure that the documentation of your
  distribution and/or your additional material also contain the author specific
  copyright notice, the BSD conditions, and the BSD disclaimer.
 \item \textbf{[voluntary:]} It is a good practice of the open source
  community, to let the copyright notice which is shown by the running program
  also state that it contains components licensed under the BSD license. Because
  you are embedding this snimoli into a larger software unit, you are
  developing this larger unit. Hence, you can also expand the copyright notice
  of this larger unit by such a hint to its BSD components.
\end{itemize}
\item[prohibits] nothing explictly, if you are using the \emph{BSD 2 Clause
License}. But the \emph{BSD 3 Clause License} prohibits the following doings in
order to fulfill the license
\begin{itemize}
  \item \textbf{[explicitly:]} Do not use the name of the licensing organization
  or the names of the licensing distributors to promote your own work
\end{itemize}
\end{description}



\subsection{Discussions and Explanations}

The \textit{BSD 2-Clause license} has a simple textual structure: In the
beginning, it generally \enquote{(permits) [the] redistribution and [the] use in
source and binary forms, with or without modification, [\ldots]}, if one
fulfills the two rules of the license\footcite[cf.][\nopage
wp]{BsdLicense2Clause}. The first rule concerns the (re)distribution in form of
source code, the second the (re)distribution of binary packages. Here are some
explanations why we translated the rules into which sets of executable tasks:

\begin{itemize}
\item For the \enquote{redistribution of source code} the license requires,
that the package must \enquote{ [\ldots] retain the above copyright notice, this
list of conditions and the following disclaimer}\footcite[cf.][\nopage
wp]{BsdLicense2Clause}. Hence, you are not allowed, to modify any of the
copyright notes which are already embedded in the received (source) files. And
from a logical point of view, there must exist an explicit or implicit
assertion that the software is licensed under the \textit{BSD 2-Clause
license}\footcite[The BSD license requires that a re-distributed software
package must contain the (package specific) copyright notice, the (license
specific) conditions and the BSD disclaimer. (cf.][\nopage wp.) You might ask
what you should do, if these elements are missed in the package you got. If so,
the package you got had not been licensed adequately. Hence, you do not know
reliably whether you have received it under a BSD license. In other words: If
you have received a BSD licensed software package, it must contain sufficient
license fulfilling elements, or it is not a BSD licensed
software]{BsdLicense2Clause}. This is often implemented by simply adding a copy
of the license into the package. Hence, you are furthermore not allowed to
modify these files or corresponding text snippets. For our purposes, we
translated the bans into the following executable task:

\begin{quote}\textit{Ensure that the licensing elements -- esp.\ the BSD license
text, the specific copyright notice of the original author(s), and the BSD disclaimer
-- are retained in your package in the form you have got them.}\end{quote}

\item For the redistribution in form of binary files, the license requires,
that the licensing elements must be \enquote{[\ldots] (reproduced) in the documentation
and/or other materials provided with the
distribution}\footcite[cf.][\nopage wp]{BsdLicense2Clause}. Hence, this is
not required as a necessary condition for the (re)distribution as source code
package. But nevertheless, even for a distribution in form of source code, it
is often possible to fulfill this rule too -- e.g.\ if you offer an own download
site for source code packages. In such cases, it is a sign of respect, to
mention the licensing not only inside of the packages, but also in the text of
your site. Because of that, we added the following voluntary task for all BSD
open source use cases which deal with the redistribution in form of source
code:

\begin{quote}\textit{Let the documentation of your distribution and/or your
additional material also contain the original copyright notice, the BSD
conditions, and the BSD disclaimer.}\end{quote}

\item Naturally, because the reproduction of the licensing elements \enquote{in
the documentation and/or other materials provided with the distribution}
is explicitly required for the \enquote{redistribution in binary
form}\footcite[cf.][\nopage wp]{BsdLicense2Clause}, we had to rewrite the
facultative task for a distribution in form of source code as a mandatory task
for all BSD open source use cases which deals with the redistribution in binary
form:

\begin{quote}\textit{Ensure that the documentation of your distribution and/or
your additional material also contains the author specific copyright notice, the
BSD conditions, and the BSD disclaimer.}\end{quote}

\item In case of (re)distributing the program in form of binary files, it is
sometimes not enough, to pass the licensing elements as one has got them. If you
compile the binary package from the source code, it is not necessarily true,
that the licensing elements are also automatically generated and embedded into
the 'binary package'. But nevertheless, you have to add the copyright notice,
the conditions and the disclaimer to this package for acting according to the
BSD license. Therefore we chose the following form of an executable, license
fulfilling task for all binary oriented distributions:

\begin{quote}\textit{Ensure that your distribution contains the original
copyright notice, the BSD license, and the BSD disclaimer in the form you have
got them. If you compile the binary file on the base of the source code package
and if this compilation does not also generate and integrate the licensing
files, then create the copyright notice the BSD conditions, and the BSD
disclaimer according to the form of the source code package and insert these
files into your distribution manually.}\end{quote}

\item Finally, we wished to insert a hint to the general (open source)
tradition, to mention the used open source software and their licenses as a
remark of the 'copyright widget' of an application. This is not required by the
BSD license. But it is a general, good tradition. Naturally, because of the
freedom to use and modify open source software and to redistribute a modified
version of it, you are also allowed to insert such references, even if they are
missing. Therefore we added a third voluntary license tradition fulfilling
task for all relevant open source use cases.

\end{itemize}




%\bibliography{../../../bibfiles/oscResourcesEn}

% Telekom osCompendium 'for being included' snippet template
%
% (c) Karsten Reincke, Deutsche Telekom AG, Darmstadt 2011
%
% This LaTeX-File is licensed under the Creative Commons Attribution-ShareAlike
% 3.0 Germany License (http://creativecommons.org/licenses/by-sa/3.0/de/): Feel
% free 'to share (to copy, distribute and transmit)' or 'to remix (to adapt)'
% it, if you '... distribute the resulting work under the same or similar
% license to this one' and if you respect how 'you must attribute the work in
% the manner specified by the author ...':
%
% In an internet based reuse please link the reused parts to www.telekom.com and
% mention the original authors and Deutsche Telekom AG in a suitable manner. In
% a paper-like reuse please insert a short hint to www.telekom.com and to the
% original authors and Deutsche Telekom AG into your preface. For normal
% quotations please use the scientific standard to cite.
%
% [ Framework derived from 'mind your Scholar Research Framework' 
%   mycsrf (c) K. Reincke 2012 CC BY 3.0  http://mycsrf.fodina.de/ ]
%


%% use all entries of the bibliography
%\nocite{*}

\section{MIT licensed software}

The MIT license is known as one of the most permissive licenses. Thus, the
MIT specific finder can be simplified:

\begin{center}
\begin{footnotesize}
\pstree[levelsep=*1,treesep=0.2]{\Toval{MIT}}{
  \pstree{
    \Tr{\Ovalbox{\shortstack{recipient: \textit{4yourself}\\
    \textbf{\textit{used by yourself}}}}} 
  }{
    \Tr{\doublebox{\shortstack{\tiny{\textbf{MIT-1}:}\\
    \tiny{\textit{using the}}\\\tiny{\textit{software}}\\
    \tiny{\textit{only for}}\\\tiny{\textit{yourself}} }}} 
  }
  \pstree[levelsep=*0.2,treesep=0.2]{
    \Tr{\Ovalbox{\shortstack{recipient: \textit{4others}\\
      \textbf{\textit{distributed to 3rd parties}}}}} 
  }{ 
    \pstree[levelsep=*0.2,treesep=0.2]{
      \Tr{\Ovalbox{\shortstack{state:\\\textbf{\textit{unmodified}}}}}
    }{
        \Tr{\doublebox{\shortstack{\tiny{\textbf{MIT-2}:}\\
        \tiny{\textit{distributing an}}\\
        \tiny{\textit{unmodified pkg}} }}}

    }

    \pstree[levelsep=*0.2,treesep=0.2]{
      \Tr{\Ovalbox{\shortstack{state:\\\textbf{\textit{modified}}}}}
    }{ 
      \pstree{
        \Tr{\Ovalbox{\shortstack{type:\\\textbf{\textit{proapse}}}}}
      }{
          \Tr{\doublebox{\shortstack{\tiny{\textbf{MIT-3}:}\\
          \tiny{\textit{distributing}}\\\tiny{\textit{a modified}}\\
          \tiny{\textit{program}} }}} 
           
      }
      \pstree{
        \Tr{\Ovalbox{\shortstack{type:\\\textbf{\textit{snimoli}}}}}
      }{
        \pstree{
          \Tr{\Ovalbox{\shortstack{context:\\\textbf{\textit{independent}}}}}
        }{        
            \Tr{\doublebox{\shortstack{\tiny{\textbf{MIT-4}:}\\
            \tiny{\textit{distributing}}\\\tiny{\textit{a modified}}\\
            \tiny{\textit{library as}}\\\tiny{\textit{independent}}
            \tiny{\textit{pkg}} }}} 
           
        }
        \pstree{
          \Tr{\Ovalbox{\shortstack{context:\\\textbf{\textit{embedded}}}}}
        }{        
            \Tr{\doublebox{\shortstack{\tiny{\textbf{MIT-5}:}\\
            \tiny{\textit{distributing}}\\\tiny{\textit{a modified}}\\
            \tiny{\textit{library as}}\\\tiny{\textit{embedded}}
            \tiny{\textit{pkg}} }}} 
           
        }


      }
 
    }
   }   
}
\end{footnotesize}
\end{center}

\subsection{MIT-1: Using the software only for yourself}
\label{OSUC-01-MIT} 
\label{OSUC-03-MIT} 
\label{OSUC-06-MIT}
\label{OSUC-09-MIT}
  
\begin{description}
\item[means] that you are going to use a received MIT software only for yourself
and that you do not handover it to any 3rd party in any sense.
\item[covers] OSUC-01, OSUC-03, OSUC-06, and OSUC-09\footnote{For details see pp.
  \pageref{OSUC-01-DEF} - \pageref{OSUC-09-DEF}}
\item[requires] the tasks in order to fulfill the conditions
    of the MIT license:
  \begin{itemize}
    \item You are allowed to use any kind of MIT licensed software in any sense
    and in any context without any other obligations if you do not handover the
    software to 3rd parties and if you do not modify the existing copyright
    notes and the existing permission notice.
  \end{itemize}
\item[prohibits] nothing explicitly.
\end{description}

\subsection{MIT-2: Passing the unmodified software}
\label{OSUC-02-MIT} \label{OSUC-05-MIT} \label{OSUC-07-MIT} 

\begin{description}
\item[means] that you are going to distribute an unmodified version of the
received MIT software to 3rd parties -- regardless whether you distribute it in
form of binaries or of source code files\footnote{In this case it also doesn't
matter whether you distribute a program, an application, a server, a snippet, a
module, a library, or a plugin as an independent package}

\item[covers] OSUC-02, OSUC-05, OSUC-07\footnote{For details see pp.
\pageref{OSUC-02-DEF} - \pageref{OSUC-07-DEF}}

\item[requires] the tasks in order to fulfill the license conditions
\begin{itemize}
  \item \textbf{[mandatory:]} Ensure that the licensing elements -- esp.\
  the MIT license text containing the specific copyright notices of the original
  author(s), the permission notices and the MIT disclaimer -- are retained in
  your package in the form you have received them.
  \item \textbf{[voluntary:]} It's a good tradition to let the documentation of
  your distribution and/or your additional material also contain a link to the
  original software (project) and its homepage.
\end{itemize}
\item[prohibits] nothing explicitly.
\end{description}

\subsection{MIT-3: Passing a modified program}
\label{OSUC-04-MIT}

\begin{description}
\item[means] that you are going to distribute a modified version of the received
MIT program, application, or server (proapse) to 3rd parties\footnote{In this
case it doesn't matter whether you are going to distribute it in form of a set
of source code files or as an integrated source code package.}.
\item[covers] OSUC-04\footnote{For details see pp. \pageref{OSUC-04-DEF}}
\item[requires] the tasks in order to fulfill the license conditions
\begin{itemize}
  \item \textbf{[mandatory:]} Ensure that the original licensing elements -- esp.\
  the MIT license text containing the specific copyright notices of the original
  author(s), the permission notices and the MIT disclaimer -- are retained in
  your package in the form you have received them.
  \item \textbf{[voluntary:]} Mark your modifications in the sourcecode,
  regardless whether you want to distribute the code or not.
  \item \textbf{[voluntary:]} It's a good tradition to let the documentation of
  your distribution and/or your additional material also contain a link to the
  original software (project) and its homepage.
  \item \textbf{[voluntary:]} You are allowed to expand an existing copyright
  notice presented by the program during its user interaction by a hint to your
  own work or part.
  \item \textbf{[voluntary:]} It is a good practice of the open source
  community, to let the copyright notice which is shown by the program also
  state that it is based on a version originally licensed under the MIT license.
  Because you are already modifying the program, you can also add such a hint,
  if the presented original copyright notice lacks such a statement.
\end{itemize}
\item[prohibits] nothing explicitly.
\end{description}

\subsection{MIT-4: Passing a modified library independently}
\label{OSUC-08-MIT}
\begin{description}
\item[means] that you are going to distribute a modified version of the received
MIT code snippet, module, library, or plugin (snimoli) to 3rd parties without
embedding it into another larger software unit.
\item[covers] OSUC-08\footnote{For details see pp. \pageref{OSUC-08-DEF}}
\item[requires] the tasks in order to fulfill the license conditions
\begin{itemize}
  \item \textbf{[mandatory:]} Ensure that the original licensing elements -- esp.\
  the MIT license text containing the specific copyright notices of the original
  author(s), the permission notices and the MIT disclaimer -- are retained in
  your package in the form you have received them.
  \item \textbf{[voluntary:]} Mark your modifications in the sourcecode,
  regardless whether you want to distribute this source code or not.
  \item \textbf{[voluntary:]} It's a good tradition to let the documentation of
  your distribution and/or your additional material also contain a link to the
  original software (project) and its homepage.
\end{itemize}
\item[prohibits] nothing explicitly.
\end{description}


\subsection{MIT-5: Passing a modified library as embedded component}
\label{OSUC-10-MIT}
\begin{description}
\item[means] that you are going to distribute a modified version of the received
MIT code snippet, module, library, or plugin (snimoli) to 3rd parties together
with another larger software unit which contains this code snippet, module,
library, or plugin as an embedded component.
\item[covers] OSUC-10\footnote{For details see pp. \pageref{OSUC-10-DEF}}
\item[requires] the tasks in order to fulfill the license conditions
\begin{itemize}
  \item \textbf{[mandatory:]} Ensure that the original licensing elements -- esp.\
  the MIT license text containing the specific copyright notices of the original
  author(s), the permission notices and the MIT disclaimer -- are retained in
  your package in the form you have received them.
  \item \textbf{[voluntary:]} Mark your modifications in the sourcecode,
  regardless whether you want to distribute this source code or not.
  
  \item \textbf{[voluntary:]} It is a good practice of the open source
  community, to let the copyright notice which is shown by the running program
  also state that the program uses a component being licensed under the MIT
  license. And it is a good tradition to insert links to the homepage / download
  page of this used component.

  \item \textbf{[voluntary:]} It's also a good tradition to let the
  documentation of your program and/or your additional material also mention
  that you have used this component added by a link to the original software
  component and its homepage.
  
  \item \textbf{[voluntary:]} Arrange your distribution so that the the original
  licensing elements -- esp.\ the MIT license text containing the specific
  copyright notices of the original author(s), the permission notices and the
  MIT disclaimer --  clearly refer only to the embedded library and do not
  disturb the licensing of your own overarching work. It's a good tradition to
  keep the libraries, modules, snippet, or plugins in specific directiers which
  contain also all licensing elements.
  
\end{itemize}
\item[prohibits] nothing explicitly.
\end{description}

\subsection{Discussions and Explanations}

The MIT-License is known as one of the most permissive licenses. It is a very
short license containing (1) a paragraph saying that you are allowed to do
almost anything you want, followed (2) by the condition that you have to
\enquote{include} the existing copyright notes and the permission notes
\enquote{[\ldots] in all copies or substantial portions of the software}, and
(3) closed by the well known disclaimer\footcite[cf.][\nopage
wp]{MitLicense2012a}. But the license doesn't talk about the difference of
source code and object code. So, you have to find the right way by yourself.
Here are our interpretations:

\begin{itemize}
  \item If you do not modify the received MIT licensed application, neither for
  your own purposes, nor for handing over the program to 3rd parties, you can
  conclude that all copyright notices and permission notices are already
  correct.
  \item Nevertheless, we added the hint not to modify these licensing elements
  in the context of the use case \emph{used by yourself}. This is evoked by the
  MIT license itself. It requires explicitly that \enquote{ the above copyright
  notice and this permission notice shall be included in all[sic!] copies or
  substantial portions of the Software}\footcite[cf.][\nopage
  wp]{MitLicense2012a} -- thus also into those copies you make for your own
  purposes own your own machines, and even if this is probably not so often
  reviewed.
  \item If you modify the received MIT licensed application, regardless for
  which purposes, you are simply not allowed to erase or modify existing
  copyright notes and permission notices. You may add your own modifications
  under new conditions, but the old base must survive.
\end{itemize}

%\bibliography{../../../bibfiles/oscResourcesEn}

% Telekom osCompendium 'for being included' snippet template
%
% (c) Karsten Reincke, Deutsche Telekom AG, Darmstadt 2011
%
% This LaTeX-File is licensed under the Creative Commons Attribution-ShareAlike
% 3.0 Germany License (http://creativecommons.org/licenses/by-sa/3.0/de/): Feel
% free 'to share (to copy, distribute and transmit)' or 'to remix (to adapt)'
% it, if you '... distribute the resulting work under the same or similar
% license to this one' and if you respect how 'you must attribute the work in
% the manner specified by the author ...':
%
% In an internet based reuse please link the reused parts to www.telekom.com and
% mention the original authors and Deutsche Telekom AG in a suitable manner. In
% a paper-like reuse please insert a short hint to www.telekom.com and to the
% original authors and Deutsche Telekom AG into your preface. For normal
% quotations please use the scientific standard to cite.
%
% [ Framework derived from 'mind your Scholar Research Framework' 
%   mycsrf (c) K. Reincke 2012 CC BY 3.0  http://mycsrf.fodina.de/ ]
%


%% use all entries of the bibliography
%\nocite{*}

\section{Microsoft Public License \ldots [tbd]}
\label{OSUC-01-MS-PL} \label{OSUC-03-MS-PL} 
\label{OSUC-06-MS-PL} \label{OSUC-09-MS-PL}

\label{OSUC-02-MS-PL} \label{OSUC-04-MS-PL} \label{OSUC-05-MS-PL}
\label{OSUC-07-MS-PL} \label{OSUC-08-MS-PL} \label{OSUC-10-MS-PL}

% TODO insert MS-PL specifc mini finder

\subsection{MS-PL specific use case 1}
(covers OSUC-X - OSUC-Z)
% \label{OSUC-10-MS-PL}
\begin{description}
\item[means] \ldots

\item[covers] OSUC-?? \ldots

\item[requires] the following tasks in order to fulfill the license conditions
\begin{itemize}
  \item \textbf{[mandatorily:]} Ensure \ldots
  \item \textbf{[voluntarily:]} Let \ldots
\end{itemize}

% \item[prohibits] nothing explicitly.
\item[prohibits] the following doings in order to fulfill the license conditions
\begin{itemize}
  \item \textbf{[directly:]} 
  \item \textbf{[indirectly:]}
\end{itemize}
\end{description}

\subsection{MS-PL specific use case n}
(covers OSUC-x - OSUC-z)
% \label{OSUC-10-MS-PL}
\begin{description}
\item[means] \ldots

\item[covers] OSUC-?? \ldots

\item[requires] the following tasks in order to fulfill the license conditions
\begin{itemize}
  \item \textbf{[mandatorily:]} Ensure \ldots
  \item \textbf{[voluntarily:]} Let \ldots
\end{itemize}

% \item[prohibits] nothing explicitly.
\item[prohibits] the following doings in order to fulfill the license conditions
\begin{itemize}
  \item \textbf{[directly:]} 
  \item \textbf{[indirectly:]}
\end{itemize}
\end{description}


%\bibliography{../../../bibfiles/oscResourcesEn}

% Telekom osCompendium 'for being included' snippet template
%
% (c) Karsten Reincke, Deutsche Telekom AG, Darmstadt 2011
%
% This LaTeX-File is licensed under the Creative Commons Attribution-ShareAlike
% 3.0 Germany License (http://creativecommons.org/licenses/by-sa/3.0/de/): Feel
% free 'to share (to copy, distribute and transmit)' or 'to remix (to adapt)'
% it, if you '... distribute the resulting work under the same or similar
% license to this one' and if you respect how 'you must attribute the work in
% the manner specified by the author ...':
%
% In an internet based reuse please link the reused parts to www.telekom.com and
% mention the original authors and Deutsche Telekom AG in a suitable manner. In
% a paper-like reuse please insert a short hint to www.telekom.com and to the
% original authors and Deutsche Telekom AG into your preface. For normal
% quotations please use the scientific standard to cite.
%
% [ Framework derived from 'mind your Scholar Research Framework' 
%   mycsrf (c) K. Reincke 2012 CC BY 3.0  http://mycsrf.fodina.de/ ]
%


%% use all entries of the bibliography
%\nocite{*}

\section{Postgres Licensed Software in the usage context of \ldots}

\label{OSUC-01-PGL} \label{OSUC-03-PGL} 
\label{OSUC-06-PGL} \label{OSUC-09-PGL}

\label{OSUC-02-PGL} \label{OSUC-04-PGL} \label{OSUC-05-PGL}
\label{OSUC-07-PGL} \label{OSUC-08-PGL} \label{OSUC-10-PGL}

% TODO insert PGL specifc mini finder

\subsection{PGL specific use case 1}
(covers OSUC-X - OSUC-Z)
% \label{OSUC-10-PGL}
\begin{description}
\item[means] \ldots

\item[covers] OSUC-?? \ldots

\item[requires] the following tasks in order to fulfill the license conditions
\begin{itemize}
  \item \textbf{[mandatorily:]} Ensure \ldots
  \item \textbf{[voluntarily:]} Let \ldots
\end{itemize}

% \item[prohibits] nothing explicitly.
\item[prohibits] the following doings in order to fulfill the license conditions
\begin{itemize}
  \item \textbf{[directly:]} 
  \item \textbf{[indirectly:]}
\end{itemize}
\end{description}

\subsection{PGL specific use case n}
(covers OSUC-x - OSUC-z)
% \label{OSUC-10-PGL}
\begin{description}
\item[means] \ldots

\item[covers] OSUC-?? \ldots

\item[requires] the following tasks in order to fulfill the license conditions
\begin{itemize}
  \item \textbf{[mandatorily:]} Ensure \ldots
  \item \textbf{[voluntarily:]} Let \ldots
\end{itemize}

% \item[prohibits] nothing explicitly.
\item[prohibits] the following doings in order to fulfill the license conditions
\begin{itemize}
  \item \textbf{[directly:]} 
  \item \textbf{[indirectly:]}
\end{itemize}
\end{description}

%\bibliography{../../../bibfiles/oscResourcesEn}

% Telekom osCompendium 'for being included' snippet template
%
% (c) Karsten Reincke, Deutsche Telekom AG, Darmstadt 2011
%
% This LaTeX-File is licensed under the Creative Commons Attribution-ShareAlike
% 3.0 Germany License (http://creativecommons.org/licenses/by-sa/3.0/de/): Feel
% free 'to share (to copy, distribute and transmit)' or 'to remix (to adapt)'
% it, if you '... distribute the resulting work under the same or similar
% license to this one' and if you respect how 'you must attribute the work in
% the manner specified by the author ...':
%
% In an internet based reuse please link the reused parts to www.telekom.com and
% mention the original authors and Deutsche Telekom AG in a suitable manner. In
% a paper-like reuse please insert a short hint to www.telekom.com and to the
% original authors and Deutsche Telekom AG into your preface. For normal
% quotations please use the scientific standard to cite.
%
% [ Framework derived from 'mind your Scholar Research Framework' 
%   mycsrf (c) K. Reincke 2012 CC BY 3.0  http://mycsrf.fodina.de/ ]
%


%% use all entries of the bibliography
%\nocite{*}

\section{PHP Licensed Software in the usage context of \ldots}
\label{OSUC-01-PHP} \label{OSUC-03-PHP} 
\label{OSUC-06-PHP} \label{OSUC-09-PHP}

\label{OSUC-02-PHP} \label{OSUC-04-PHP} \label{OSUC-05-PHP}
\label{OSUC-07-PHP} \label{OSUC-08-PHP} \label{OSUC-10-PHP}

% TODO insert php license specifc mini finder

\subsection{PHP specific use case 1}
(covers OSUC-X - OSUC-Z)

\subsection{PHP specific use case n}
(covers OSUC-x - OSUC-z)


%\bibliography{../../../bibfiles/oscResourcesEn}

% Telekom osCompendium 'for being included' snippet template
%
% (c) Karsten Reincke, Deutsche Telekom AG, Darmstadt 2011
%
% This LaTeX-File is licensed under the Creative Commons Attribution-ShareAlike
% 3.0 Germany License (http://creativecommons.org/licenses/by-sa/3.0/de/): Feel
% free 'to share (to copy, distribute and transmit)' or 'to remix (to adapt)'
% it, if you '... distribute the resulting work under the same or similar
% license to this one' and if you respect how 'you must attribute the work in
% the manner specified by the author ...':
%
% In an internet based reuse please link the reused parts to www.telekom.com and
% mention the original authors and Deutsche Telekom AG in a suitable manner. In
% a paper-like reuse please insert a short hint to www.telekom.com and to the
% original authors and Deutsche Telekom AG into your preface. For normal
% quotations please use the scientific standard to cite.
%
% [ Framework derived from 'mind your Scholar Research Framework' 
%   mycsrf (c) K. Reincke 2012 CC BY 3.0  http://mycsrf.fodina.de/ ]
%


%% use all entries of the bibliography
%\nocite{*}

\section{Eclipse Licensed Software in the usage context of \ldots}
\label{OSUC-01-EPL} \label{OSUC-03-EPL} 
\label{OSUC-06-EPL} \label{OSUC-09-EPL}

\label{OSUC-02-EPL} \label{OSUC-04-EPL} \label{OSUC-05-EPL}
\label{OSUC-07-EPL} \label{OSUC-08-EPL} \label{OSUC-10-EPL}

% TODO insert EPL specifc mini finder

\subsection{EPL specific use case 1}
(covers OSUC-X - OSUC-Z)

\subsection{EPL specific use case n}
(covers OSUC-x - OSUC-z)


%\bibliography{../../../bibfiles/oscResourcesEn}

% Telekom osCompendium 'for being included' snippet template
%
% (c) Karsten Reincke, Deutsche Telekom AG, Darmstadt 2011
%
% This LaTeX-File is licensed under the Creative Commons Attribution-ShareAlike
% 3.0 Germany License (http://creativecommons.org/licenses/by-sa/3.0/de/): Feel
% free 'to share (to copy, distribute and transmit)' or 'to remix (to adapt)'
% it, if you '... distribute the resulting work under the same or similar
% license to this one' and if you respect how 'you must attribute the work in
% the manner specified by the author ...':
%
% In an internet based reuse please link the reused parts to www.telekom.com and
% mention the original authors and Deutsche Telekom AG in a suitable manner. In
% a paper-like reuse please insert a short hint to www.telekom.com and to the
% original authors and Deutsche Telekom AG into your preface. For normal
% quotations please use the scientific standard to cite.
%
% [ Framework derived from 'mind your Scholar Research Framework' 
%   mycsrf (c) K. Reincke 2012 CC BY 3.0  http://mycsrf.fodina.de/ ]
%


%% use all entries of the bibliography
%\nocite{*}

\section{European Public Licensed Software in the usage context of \ldots}
\label{OSUC-01-EUPL} \label{OSUC-03-EUPL} 
\label{OSUC-06-EUPL} \label{OSUC-09-EUPL}

\label{OSUC-02-EUPL} \label{OSUC-04-EUPL} \label{OSUC-05-EUPL}
\label{OSUC-07-EUPL} \label{OSUC-08-EUPL} \label{OSUC-10-EUPL}

% TODO insert EUPL specifc mini finder

\subsection{EUPL specific use case 1}
(covers OSUC-X - OSUC-Z)
% \label{OSUC-10-EUPL}
\begin{description}
\item[means] \ldots

\item[covers] OSUC-?? \ldots

\item[requires] the following tasks in order to fulfill the license conditions
\begin{itemize}
  \item \textbf{[mandatorily:]} Ensure \ldots
  \item \textbf{[voluntarily:]} Let \ldots
\end{itemize}

% \item[prohibits] nothing explicitly.
\item[prohibits] the following doings in order to fulfill the license conditions
\begin{itemize}
  \item \textbf{[directly:]} 
  \item \textbf{[indirectly:]}
\end{itemize}
\end{description}

\subsection{EUPL specific use case n}
(covers OSUC-x - OSUC-z)
% \label{OSUC-10-EUPL}
\begin{description}
\item[means] \ldots

\item[covers] OSUC-?? \ldots

\item[requires] the following tasks in order to fulfill the license conditions
\begin{itemize}
  \item \textbf{[mandatorily:]} Ensure \ldots
  \item \textbf{[voluntarily:]} Let \ldots
\end{itemize}

% \item[prohibits] nothing explicitly.
\item[prohibits] the following doings in order to fulfill the license conditions
\begin{itemize}
  \item \textbf{[directly:]} 
  \item \textbf{[indirectly:]}
\end{itemize}
\end{description}

%\bibliography{../../../bibfiles/oscResourcesEn}

% Telekom osCompendium 'for being included' snippet template
%
% (c) Karsten Reincke, Deutsche Telekom AG, Darmstadt 2011
%
% This LaTeX-File is licensed under the Creative Commons Attribution-ShareAlike
% 3.0 Germany License (http://creativecommons.org/licenses/by-sa/3.0/de/): Feel
% free 'to share (to copy, distribute and transmit)' or 'to remix (to adapt)'
% it, if you '... distribute the resulting work under the same or similar
% license to this one' and if you respect how 'you must attribute the work in
% the manner specified by the author ...':
%
% In an internet based reuse please link the reused parts to www.telekom.com and
% mention the original authors and Deutsche Telekom AG in a suitable manner. In
% a paper-like reuse please insert a short hint to www.telekom.com and to the
% original authors and Deutsche Telekom AG into your preface. For normal
% quotations please use the scientific standard to cite.
%
% [ Framework derived from 'mind your Scholar Research Framework' 
%   mycsrf (c) K. Reincke 2012 CC BY 3.0  http://mycsrf.fodina.de/ ]
%


%% use all entries of the bibliography
%\nocite{*}

\section{MPL licensed software}

Also, the Mozilla Public License clearly distinguishes the distribution in the
form of source code from that in the form of binaries: First, it allows the
\enquote{Distribution of Source Form}\footcite[cf.][\nopage wp.\
§3.1]{Mpl20OsiLicense2013a}. Then, it specifies the conditions for a
\enquote{Distribution of Executable Form}\footcite[cf.][\nopage wp.\
§3.2]{Mpl20OsiLicense2013a}. Additionally, the MPL confronts the
\enquote{distribution of Covered Software} with the \enquote{distribution of a
Larger Work}\footcite[cf.][\nopage wp.\ §3.3]{Mpl20OsiLicense2013a}. So, taken
as whole, the MPL mainly focusses on the distribution of software. Thus, for
finding the relevant, simply processable task lists, also the following MPL
specific open source use case structure\footnote{For details of the general OSUC
finder $\rightarrow$ OSLiC, pp.\ \pageref{OsucTokens} and
\pageref{OsucDefinitionTree}} can be used:
 
\tikzstyle{nodv} = [font=\small, ellipse, draw, fill=gray!10, 
    text width=2cm, text centered, minimum height=2em]

\tikzstyle{nods} = [font=\footnotesize, rectangle, draw, fill=gray!20, 
    text width=1.2cm, text centered, rounded corners, minimum height=3em]

\tikzstyle{nodb} = [font=\footnotesize, rectangle, draw, fill=gray!20, 
    text width=2.2cm, text centered, rounded corners, minimum height=3em]
    
\tikzstyle{leaf} = [font=\tiny, rectangle, draw, fill=gray!30, 
    text width=1.2cm, text centered, minimum height=6em]

\tikzstyle{edge} = [draw, -latex']

\begin{tikzpicture}[]

\node[nodv] (l71) at (4,10) {MPL};

\node[nodb] (l61) at (0,8.6) {\textit{recipient:} \\ \textbf{4yourself}};
\node[nodb] (l62) at (6.5,8.6) {\textit{recipient:} \\ \textbf{2others}};

\node[nodb] (l51) at (2.5,7) {\textit{state:} \\ \textbf{unmodified}};
\node[nodb] (l52) at (9.3,7) {\textit{state:} \\ \textbf{modified}};

\node[nods] (l41) at (1.8,5.4) {\textit{form:} \textbf{source}};
\node[nods] (l42) at (3.6,5.4) {\textit{form:} \textbf{binary}};
\node[nodb] (l43) at (6.5,5.4) {\textit{type:} \\ \textbf{proapse}};
\node[nodb] (l44) at (12,5.4) {\textit{type:} \\ \textbf{snimoli}};


\node[nods] (l31) at (5.4,3.8) {\textit{form:} \textbf{source}};
\node[nods] (l32) at (7.2,3.8) {\textit{form:} \textbf{binary}};
\node[nodb] (l33) at (10,3.8) {\textit{context:} \\ \textbf{independent}};
\node[nodb] (l34) at (13.5,3.8) {\textit{context:} \\ \textbf{embedded}};

\node[nods] (l21) at (9,2.2) {\textit{form:} \textbf{source}};
\node[nods] (l22) at (10.8,2.2) {\textit{form:} \textbf{binary}};
\node[nods] (l23) at (12.6,2.2) {\textit{form:} \textbf{source}};
\node[nods] (l24) at (14.4,2.2) {\textit{form:} \textbf{binary}};

\node[leaf] (l11) at (0,0) {\textbf{MPL-C1} \textit{using software only
for yourself}};

\node[leaf] (l12) at (1.8,0) { \textbf{MPL-C2} \textit{ distributing unmodified
software as sources}};

\node[leaf] (l13) at (3.6,0) { \textbf{MPL-C3}  \textit{ distributing unmodified
software as binaries}};

\node[leaf] (l14) at (5.4,0) { \textbf{MPL-C4}  \textit{ distributing modified
program as sources}};

\node[leaf] (l15) at (7.2,0) { \textbf{MPL-C5}  \textit{ distributing modified
program as binaries}};

\node[leaf] (l16) at (9,0) { \textbf{MPL-C6}  \textit{ distributing modified
library as independent sources}};

\node[leaf] (l17) at (10.8,0) { \textbf{MPL-C7} \textit{distributing modified
library as independent binaries}};

\node[leaf] (l18) at (12.6,0) { \textbf{MPL-C8}  \textit{distributing
modified library as embedded sources}};

\node[leaf] (l19) at (14.4,0) { \textbf{MPL-C9}  \textit{ distributing modified
library as embedded binaries}};


\path [edge] (l71) -- (l61);
\path [edge] (l71) -- (l62);
\path [edge] (l61) -- (l11);
\path [edge] (l62) -- (l51);
\path [edge] (l62) -- (l52);
\path [edge] (l51) -- (l41);
\path [edge] (l51) -- (l42);
\path [edge] (l52) -- (l43);
\path [edge] (l52) -- (l44);
\path [edge] (l41) -- (l12);
\path [edge] (l42) -- (l13);
\path [edge] (l43) -- (l31);
\path [edge] (l43) -- (l32);
\path [edge] (l44) -- (l33);
\path [edge] (l44) -- (l34);
\path [edge] (l31) -- (l14);
\path [edge] (l32) -- (l15);
\path [edge] (l33) -- (l21);
\path [edge] (l33) -- (l22);
\path [edge] (l34) -- (l23);
\path [edge] (l34) -- (l24);
\path [edge] (l21) -- (l16);
\path [edge] (l22) -- (l17);
\path [edge] (l23) -- (l18);
\path [edge] (l24) -- (l19);

\end{tikzpicture}

\subsection{MPL-C1: Using the software only for yourself}
\label{OSUC-01-MPL} \label{OSUC-03-MPL} 
\label{OSUC-06-MPL} \label{OSUC-09-MPL}

\begin{description}

\item[means] that you are going to use a received MPL licensed software only
for yourself and that you do not hand it over to any 3rd party in any sense.

\item[covers] OSUC-01, OSUC-03, OSUC-06, and OSUC-09\footnote{For details 
$\rightarrow$ OSLiC, pp.\ \pageref{OSUC-01-DEF} - \pageref{OSUC-09-DEF}}

\item[requires] no tasks in order to fulfill the conditions of the MPL 2.0
license with respect to this use case:
  \begin{itemize}
    \item You are allowed to use any kind of MPL software in any sense and in
    any context without being obliged to do anything as long as you do not
    give the software to 3rd parties.
  \end{itemize}

\item[prohibits] \ldots
\begin{itemize}
  \item to remove or to alter any license notices -- including copyright
  notices, patent notices, disclaimers of warranty, or limitations of liablility
  -- contained within the software package you have received.
  \item to promote any of your services -- based on the this software -- by
  trademarks, service marks, or logos linked to this MPL software, except as
  required for unpartially describing the used software and reproducing the
  copyright notice.
\end{itemize}

\end{description}

\subsection{MPL-C2: Passing the unmodified software as source code}
\label{OSUC-02S-MPL} \label{OSUC-05S-MPL} \label{OSUC-07S-MPL} 

\begin{description}

\item[means] that you are going to distribute an unmodified version of the
received MPL software to 3rd parties - in the form of source code files or as a
source code package. In this case it is not discriminating to distribute a
program, an application, a server, a snippet, a module, a library, or a plugin
as an independent or an embedded unit

\item[covers] OSUC-02S, OSUC-05S, OSUC-07S\footnote{For details $\rightarrow$
OSLiC, pp.\ \pageref{OSUC-02S-DEF} - \pageref{OSUC-07S-DEF}}

\item[requires] the following tasks in order to fulfill the license conditions:
\begin{itemize}
  
  \item \textbf{[mandatory:]} Ensure that the licensing elements -- esp.\ all
  copyright notices, patent notices, disclaimers of warranty, or limitations of
  liability -- are retained in your package in exact the form you have received
  them.

  \item \textbf{[mandatory:]} Give the recipient a copy of the MPL 2.0 license.
  If it is not already part of the software package, add it\footnote{For
  implementing the handover of files correctly $\rightarrow$ OSLiC, p.
  \pageref{DistributingFilesHint}}. If the licensing statement in the licensing
  file of the package does still not clearly state that the package is licensed
  under the MPL, additionally insert your own correct MPL licensing file
  containing the sentence: \emph{This Source Code Form is subject to the terms
  of the Mozilla Public License, v. 2.0. If a copy of the MPL was not
  distributed with this file, You can obtain one at
  http://mozilla.org/MPL/2.0/}.

  \item \textbf{[voluntary:]} Let the documentation of your distribution and/or
  your additional material also reproduce the content of the existing
  \emph{copyright notice text files}, a hint to the software name, a link to its
  homepage, and a link to the MPL 2.0 license.
\end{itemize}

\item[prohibits] \ldots
\begin{itemize}
  \item to remove or to alter any license notices -- including copyright
  notices, patent notices, disclaimers of warranty, or limitations of liablility
  -- contained within the software package you have received.
  \item to promote any of your products -- based on the this software -- by
  trademarks, service marks, or logos linked to this MPL software, except as
  required for unpartially describing the used software and reproducing the
  copyright notice.
\end{itemize}
\end{description}


\subsection{MPL-C3: Passing the unmodified software as binaries} 
\label{OSUC-02B-MPL} \label{OSUC-05B-MPL} \label{OSUC-07B-MPL}

\begin{description}
\item[means] that you are going to distribute an unmodified version of the
received MPL software to 3rd parties -- in the form of binary files or as a
bi\-na\-ry package. In this case it is not discriminating to distribute a
program, an application, a server, a snippet, a module, a library, or a plugin
as an independent or an embedded unit.

\item[covers] OSUC-02B, OSUC-05B, OSUC-07B\footnote{For details $\rightarrow$
OSLiC, pp.\ \pageref{OSUC-02B-DEF} - \pageref{OSUC-07B-DEF}}

\item[requires] the following tasks in order to fulfill the license conditions:
\begin{itemize}
  
  \item \textbf{[mandatory:]} Ensure that the licensing elements -- esp.\ all
  copyright notices, patent notices, disclaimers of warranty, or limitations of
  liability -- are retained in your package in exact the form you have received
  them. If you compile the binary from the sources, ensure that all these
  licensing elements are also incorporated into the package.
  
  \item \textbf{[mandatory:]} Make the source code of the software accessible
  via a repository under your own control: Push the source code package into an
  internet repository and enable its download function without requiring any fee
  from the downloading user. Integrate an easily to find description into your
  distribution package which explains how the code can be received from where.
  Ensure, that this repository is usable reasonably long enough.
  
  \item \textbf{[mandatory:]} Insert a prominent hint to the download repository
  into your distribution and/or your additional material.
  
  \item \textbf{[mandatory:]} Execute the to-do list of use case MPL-C2\footnote{
  Making the code accessible via a repository means distributing the software in
  the form of source code. Hence, you must also fulfill all tasks of the
  corresponding use case.}.
  
  \item \textbf{[voluntary:]} Give the recipient a copy of the MPL 2.0 license.
  If it is not already part of the software package, add it\footnote{For
  implementing the handover of files correctly $\rightarrow$ OSLiC, p.
  \pageref{DistributingFilesHint}}. If the licensing statement in the licensing
  file of the package does still not clearly state that the package is licensed
  under the MPL, additionally insert your own correct MPL licensing file
  containing the sentence: \emph{This Source Code Form is subject to the terms
  of the Mozilla Public License, v. 2.0. If a copy of the MPL was not
  distributed with this file, You can obtain one at
  http://mozilla.org/MPL/2.0/}.
  
  \item \textbf{[voluntary:]} Let the documentation of your distribution and/or
  your additional material also reproduce the content of the existing
  \emph{copyright notice text files}, a hint to the software name, a link to its
  homepage, and a link to the MPL 2.0 license.
    
\end{itemize}

\item[prohibits] \ldots
\begin{itemize}
  \item to remove or to alter any license notices -- including copyright
  notices, patent notices, disclaimers of warranty, or limitations of liablility
  -- contained within the software package you have received.
  \item to promote any of your products -- based on the this software -- by
  trademarks, service marks, or logos linked to this MPL software, except as
  required for unpartially describing the used software and reproducing the
  copyright notice.
\end{itemize}

\end{description}

\subsection{MPL-C4: Passing a modified program as source code}
\label{OSUC-04S-MPL} 

\begin{description}
\item[means] that you are going to distribute a modified version of the received
MPL licensed program, application, or server (proapse) to 3rd parties -- in the
form of source code files or a source code package.
\item[covers] OSUC-04S\footnote{For details $\rightarrow$ OSLiC, pp.\
\pageref{OSUC-04S-DEF}}
\item[requires] the following tasks in order to fulfill the license conditions:
\begin{itemize}
  
  \item \textbf{[mandatory:]} Ensure that the licensing elements -- esp.\ all
  copyright notices, patent notices, disclaimers of warranty, or limitations of
  liability -- are retained in your package in exact the form you have received
  them.

  \item \textbf{[mandatory:]} Give the recipient a copy of the MPL 2.0 license.
  If it is not already part of the software package, add it\footnote{For
  implementing the handover of files correctly $\rightarrow$ OSLiC, p.
  \pageref{DistributingFilesHint}}. If the licensing statement in the licensing
  file of the package does still not clearly state that the package is licensed
  under the MPL, additionally insert your own correct MPL licensing file
  containing the sentence: \emph{This Source Code Form is subject to the terms
  of the Mozilla Public License, v. 2.0. If a copy of the MPL was not
  distributed with this file, You can obtain one at
  http://mozilla.org/MPL/2.0/}.  
  
  \item \textbf{[mandatory:]} Organize your modifications in a way that they are
  covered by the existing MPL licensing statements. If you add new source code
  files, insert a header containing your copyright line and an MPL adequate
  licensing the statement.
  
  \item \textbf{[voluntary:]} Create a \emph{modification text file}, if such a
  notice file still does not exist. \emph{Expand} the \emph{modification text
  file} by a more general description of your modifications. Incorporate it into
  your distribution package.
  
  \item \textbf{[voluntary:]} Mark all modifications of the source code of the
  program (proapse) thoroughly -- namely within the modfied source code.

  \item \textbf{[voluntary:]} Let the documentation of your distribution and/or
  your additional material also reproduce the content of the existing
  \emph{copyright notice text files}, a hint to the software name, a link to its
  homepage, and a link to the MPL 2.0 license.
  
 \end{itemize}
 
\item[prohibits] \ldots
\begin{itemize}
  \item to remove or to alter any license notices -- including copyright
  notices, patent notices, disclaimers of warranty, or limitations of liablility
  -- contained within the software package you have received.
  \item to promote any of your products -- based on the this software -- by
  trademarks, service marks, or logos linked to this MPL software, except as
  required for unpartially describing the used software and reproducing the
  copyright notice.
\end{itemize}

\end{description}

\subsection{MPL-C5: Passing a modified program as binary}
\label{OSUC-04B-MPL} 

\begin{description}
\item[means] that you are going to distribute a modified version of the received
MPL licensed pro\-gram, application, or server (proapse) to 3rd parties -- in
the form of binary files or as a binary package.
\item[covers] OSUC-04B\footnote{For details $\rightarrow$ OSLiC, pp.\
\pageref{OSUC-04B-DEF}}
\item[requires] the following tasks in order to fulfill the license conditions:
\begin{itemize}

  \item \textbf{[mandatory:]} Ensure that the licensing elements -- esp.\ all
  copyright notices, patent notices, disclaimers of warranty, or limitations of
  liability -- are retained in your package in exact the form you have received
  them. If you compile the binary from the sources, ensure that all these
  licensing elements are also incorporated into the package.

  \item \textbf{[mandatory:]} Make the source code of the software accessible
  via a repository under your own control: Push the source code package into an
  internet repository and enable its download function without requiring any fee
  from the downloading user. Integrate an easily to find description into your
  distribution package which explains how the code can be received from where.
  Ensure, that this repository is usable reasonably long enough.
  
  \item \textbf{[mandatory:]} Insert a prominent hint to the download repository
  into your distribution and/or your additional material. 

  \item \textbf{[mandatory:]} Execute the to-do list of use case MPL-C4\footnote{
  Making the code accessible via a repository means distributing the software in
  the form of source code. Hence, you must also fulfill all tasks of the
  corresponding use case.}.

  \item \textbf{[mandatory:]} Organize your modifications in a way that they are
  covered by the existing MPL licensing statements.
  
  \item \textbf{[voluntary:]} Create a \emph{modification text file}, if such a
  notice file still does not exist. \emph{Expand} the \emph{modification text
  file} by a more general description of your modifications. Incorporate it into
  your distribution package.
  
  \item \textbf{[voluntary:]} Give the recipient a copy of the MPL 2.0 license.
  If it is not already part of the software package, add it\footnote{For
  implementing the handover of files correctly $\rightarrow$ OSLiC, p.
  \pageref{DistributingFilesHint}}. If the licensing statement in the licensing
  file of the package does still not clearly state that the package is licensed
  under the MPL, additionally insert your own correct MPL licensing file
  containing the sentence: \emph{This Source Code Form is subject to the terms
  of the Mozilla Public License, v. 2.0. If a copy of the MPL was not
  distributed with this file, You can obtain one at
  http://mozilla.org/MPL/2.0/}.
 
  \item \textbf{[voluntary:]} Let the documentation of your distribution and/or
  your additional material  also reproduce the content of the existing
  \emph{copyright notice text files}, a hint to the software name, a link to its
  homepage, and a link to the MPL 2.0 license -- especially as a subsection of
  your own copyright notice.

\end{itemize}  

\item[prohibits] \ldots
\begin{itemize}
  \item to remove or to alter any license notices -- including copyright
  notices, patent notices, disclaimers of warranty, or limitations of liablility
  -- contained within the software package you have received.
  \item to promote any of your products -- based on the this software -- by
  trademarks, service marks, or logos linked to this MPL software, except as
  required for unpartially describing the used software and reproducing the
  copyright notice.
\end{itemize}

\end{description}

\subsection{MPL-C6: Passing a modified library as independent source code}
\label{OSUC-08S-MPL}

\begin{description}
\item[means] that you are going to distribute a modified version of the received
MPL licensed code snippet, module, library, or plugin (snimoli) to 3rd parties
-- in the form of source code files or as a source code package, but without
embedding it into another larger software unit.
\item[covers] OSUC-08S\footnote{For details $\rightarrow$ OSLiC, pp.\
\pageref{OSUC-08S-DEF}}
\item[requires] the following tasks in order to fulfill the license conditions:
\begin{itemize}

  \item \textbf{[mandatory:]} Ensure that the licensing elements -- esp.\ all
  copyright notices, patent notices, disclaimers of warranty, or limitations of
  liability -- are retained in your package in exact the form you have received
  them.
  
  \item \textbf{[mandatory:]} Give the recipient a copy of the MPL 2.0 license.
  If it is not already part of the software package, add it\footnote{For
  implementing the handover of files correctly $\rightarrow$ OSLiC, p.
  \pageref{DistributingFilesHint}}. If the licensing statement in the licensing
  file of the package does still not clearly state that the package is licensed
  under the MPL, additionally insert your own correct MPL licensing file
  containing the sentence: \emph{This Source Code Form is subject to the terms
  of the Mozilla Public License, v. 2.0. If a copy of the MPL was not
  distributed with this file, You can obtain one at
  http://mozilla.org/MPL/2.0/}.
  
  \item \textbf{[mandatory:]} Organize your modifications in a way that they are
  covered by the existing MPL licensing statements. If you add new source code
  files, insert a header containing your copyright line and an MPL adequate
  licensing the statement.
  
  \item \textbf{[voluntary:]} Create a \emph{modification text file}, if such a
  notice file still does not exist. \emph{Expand} the \emph{modification text
  file} by a more general description of your modifications. Incorporate it into
  your distribution package.

  \item \textbf{[voluntary:]} Mark all modifications of the source code of the
  library (snimoli) thoroughly -- namely within
  the modfied source code.
  
  \item \textbf{[voluntary:]} Let the documentation of your distribution and/or
  your additional material  also reproduce the content of the existing
  \emph{copyright notice text files}, a hint to the software name, a link to its
  homepage, and a link to the MPL 2.0 license.

\end{itemize}

\item[prohibits] \ldots
\begin{itemize}
  \item to remove or to alter any license notices -- including copyright
  notices, patent notices, disclaimers of warranty, or limitations of liablility
  -- contained within the software package you have received.
  \item to promote any of your products -- based on the this software -- by
  trademarks, service marks, or logos linked to this MPL software, except as
  required for unpartially describing the used software and reproducing the
  copyright notice.
\end{itemize}

\end{description}


\subsection{MPL-C7: Passing a modified library as independent binary}
\label{OSUC-08B-MPL}

\begin{description}
\item[means] that you are going to distribute a modified version of the received
MPL licensed code snippet, module, library, or plugin (snimoli) to 3rd parties
-- in the form of binary files or as a binary package but without embedding it
into another larger software unit.
\item[covers] OSUC-08B\footnote{For details $\rightarrow$ OSLiC, pp.\
\pageref{OSUC-08B-DEF}}
\item[requires] the following tasks in order to fulfill the license conditions:
\begin{itemize}

  \item \textbf{[mandatory:]} Ensure that the licensing elements -- esp.\ all
  copyright notices, patent notices, disclaimers of warranty, or limitations of
  liability -- are retained in your package in exact the form you have received
  them. If you compile the binary from the sources, ensure that all these
  licensing elements are also incorporated into the package.

  \item \textbf{[mandatory:]} Make the source code of the software accessible
  via a repository under your own control: Push the source code package into an
  internet repository and enable its download function without requiring any fee
  from the downloading user. Integrate an easily to find description into your
  distribution package which explains how the code can be received from where.
  Ensure, that this repository is usable reasonably long enough.
  
  \item \textbf{[mandatory:]} Insert a prominent hint to the download repository
  into your distribution and/or your additional material.

  \item \textbf{[mandatory:]} Execute the to-do list of use case MPL-6\footnote{
  Making the code accessible via a repository means distributing the software in
  the form of source code. Hence, you must also fulfill all tasks of the
  corresponding use case.}.
  
  \item \textbf{[mandatory:]} Organize your modifications in a way that they are
  covered by the existing MPL licensing statements.
  
  \item \textbf{[voluntary:]} Create a \emph{modification text file}, if such a
  notice file still does not exist. \emph{Expand} the \emph{modification text
  file} by a more general description of your modifications. Incorporate it into
  your distribution.
  
  \item \textbf{[voluntary:]} Give the recipient a copy of the MPL 2.0 license.
  If it is not already part of the software package, add it\footnote{For
  implementing the handover of files correctly $\rightarrow$ OSLiC, p.
  \pageref{DistributingFilesHint}}. If the licensing statement in the licensing
  file of the package does still not clearly state that the package is licensed
  under the MPL, additionally insert your own correct MPL licensing file
  containing the sentence: \emph{This Source Code Form is subject to the terms
  of the Mozilla Public License, v. 2.0. If a copy of the MPL was not
  distributed with this file, You can obtain one at
  http://mozilla.org/MPL/2.0/}.
  
  \item \textbf{[voluntary:]} Let the documentation of your distribution and/or
  your additional material  also reproduce the content of the existing
  \emph{copyright notice text files}, a hint to the software name, a link to its
  homepage, and a link to the MPL 2.0 license -- especially as a subsection of
  your own copyright notice.
  
\end{itemize}

\item[prohibits] \ldots
\begin{itemize}
  \item to remove or to alter any license notices -- including copyright
  notices, patent notices, disclaimers of warranty, or limitations of liablility
  -- contained within the software package you have received.
  \item to promote any of your products -- based on the this software -- by
  trademarks, service marks, or logos linked to this MPL software, except as
  required for unpartially describing the used software and reproducing the
  copyright notice.
\end{itemize}

\end{description}

\subsection{MPL-C8: Passing a modified library as embedded source code}
\label{OSUC-10S-MPL}

\begin{description}
\item[means] that you are going to distribute a modified version of the received
MPL licensed code snippet, module, library, or plugin (snimoli) to 3rd parties
-- in the form of source code files or as a source code package together with
another larger software unit which contains this code snippet, module, library,
or plugin as an embedded component.
\item[covers] OSUC-10S\footnote{For details $\rightarrow$ OSLiC, pp.\
\pageref{OSUC-10S-DEF}}
\item[requires] the following tasks in order to fulfill the license conditions:
\begin{itemize}

  \item \textbf{[mandatory:]} Ensure that the licensing elements -- esp.\ all
  copyright notices, patent notices, disclaimers of warranty, or limitations of
  liability -- are retained in your package in exact the form you have received
  them.
  
  \item \textbf{[mandatory:]} Give the recipient a copy of the MPL 2.0 license.
  If it is not already part of the software package, add it\footnote{For
  implementing the handover of files correctly $\rightarrow$ OSLiC, p.
  \pageref{DistributingFilesHint}}. If the licensing statement in the licensing
  file of the package does still not clearly state that the package is licensed
  under the MPL, additionally insert your own correct MPL licensing file
  containing the sentence: \emph{This Source Code Form is subject to the terms
  of the Mozilla Public License, v. 2.0. If a copy of the MPL was not
  distributed with this file, You can obtain one at
  http://mozilla.org/MPL/2.0/}.

  \item \textbf{[mandatory:]} Organize your modifications of the embedded
  library in a way that they are covered by the existing MPL licensing
  statements. If you add new source code files to the library itself, insert a
  header containing your copyright line and an MPL adequate licensing the
  statement.
  
  \item \textbf{[voluntary:]} Arrange your source code distribution so that the
  integrated MPL and the \emph{licensing files} clearly refer only to the
  embedded library and do not disturb the licensing of your own overarching
  work. It's a good tradition to keep the embedded components like libraries,
  modules, snippets, or plugins in specific directory which contains also all
  additional licensing elements.


  \item \textbf{[voluntary:]} Create a \emph{modification text file}, if such a
  notice file still does not exist. \emph{Expand} the \emph{modification text
  file} by a more general description of your modifications. Incorporate it into
  your distribution package.
  
  \item \textbf{[voluntary:]} Mark all modifications of the source code of the
  embedded library (snimoli) thoroughly -- namely within the source code.
      
  \item \textbf{[voluntary:]} Let the documentation of your distribution and/or
  your additional material also reproduce the content of the existing
  \emph{copyright notice text files}, a hint to the name of the used MPL
  licensed component, a link to its homepage, and a link to the MPL 2.0 license
  -- especially as subsection of your own copyright notice.
 
\end{itemize}

\item[prohibits] \ldots
\begin{itemize}
  \item to remove or to alter any license notices -- including copyright
  notices, patent notices, disclaimers of warranty, or limitations of liablility
  -- contained within the software package you have received.
  \item to promote any of your products -- based on the this software -- by
  trademarks, service marks, or logos linked to this MPL software, except as
  required for unpartially describing the used software and reproducing the
  copyright notice.
\end{itemize}

\end{description}


\subsection{MPL-C9: Passing a modified library as embedded binary}
\label{OSUC-10B-MPL}

\begin{description}
\item[means] that you are going to distribute a modified version of the received
MPL licensed code snippet, module, library, or plugin to 3rd parties -- in the
form of binary files or as a binary package together with another larger
software unit which contains this code snippet, module, library, or plugin as an
embedded component.
\item[covers] OSUC-10B\footnote{For details $\rightarrow$ OSLiC, pp.\
\pageref{OSUC-10B-DEF}}
\item[requires] the following tasks in order to fulfill the license conditions:
\begin{itemize}

  \item \textbf{[mandatory:]} Ensure that the licensing elements -- esp.\ all
  copyright notices, patent notices, disclaimers of warranty, or limitations of
  liability -- are retained in your package in exact the form you have received
  them. If you compile the binary from the sources, ensure that all these
  licensing elements are also incorporated into the package.

  \item \textbf{[mandatory:]} Make the source code of the embedded library
  accessible via a repository under your own control: Push the source code
  package into an internet repository and enable its download function without
  requiring any fee from the downloading user. Integrate an easily to find
  description into your distribution package which explains how the code can be
  received from where. Ensure, that this repository is usable reasonably long
  enough.
  
  \item \textbf{[mandatory:]} Insert a prominent hint to the download repository
  into your distribution and/or your additional material.

  \item \textbf{[mandatory:]} Execute the to-do list of use case MPL-C8\footnote{
  Making the code accessible via a repository means distributing the software in
  the form of source code. Hence, you must also fulfill all tasks of the
  corresponding use case.}.

  \item \textbf{[mandatory:]} Organize your modifications of the embedded
  library in a way that they are covered by the existing MPL licensing
  statements. 
    
  \item \textbf{[voluntary:]} Create a \emph{modification text file}, if such a
  notice file still does not exist. \emph{Expand} the \emph{modification text
  file} by a more general description of your modifications. Incorporate it into
  your distribution package.
  
  \item \textbf{[voluntary:]} Give the recipient a copy of the MPL 2.0 license.
  If it is not already part of the software package, add it\footnote{For
  implementing the handover of files correctly $\rightarrow$ OSLiC, p.
  \pageref{DistributingFilesHint}}. If the licensing statement in the licensing
  file of the package does still not clearly state that the package is licensed
  under the MPL, additionally insert your own correct MPL licensing file
  containing the sentence: \emph{This Source Code Form is subject to the terms
  of the Mozilla Public License, v. 2.0. If a copy of the MPL was not
  distributed with this file, You can obtain one at
  http://mozilla.org/MPL/2.0/}.

  \item \textbf{[voluntary:]} Arrange your binary distribution so that the
  integrated MPL and the \emph{licensing files} clearly refer only to the
  embedded library and do not disturb the licensing of your own overarching
  work. It's a good tradition to keep the embedded components like libraries,
  modules, snippets, or plugins in specific directory which contains also all
  additional licensing elements.
  
  
  \item \textbf{[voluntary:]} Let the documentation of your distribution and/or
  your additional material  also reproduce the content of the existing
  \emph{copyright notice text files}, a hint to the name of the used MPL
  licensed component, a link to its homepage, and a link to the MPL 2.0 license
  -- especially as subsection of your own copyright notice.
  
\end{itemize}

\item[prohibits] \ldots
\begin{itemize}
  \item to remove or to alter any license notices -- including copyright
  notices, patent notices, disclaimers of warranty, or limitations of liablility
  -- contained within the software package you have received.
  \item to promote any of your products -- based on the this software -- by
  trademarks, service marks, or logos linked to this MPL software, except as
  required for unpartially describing the used software and reproducing the
  copyright notice.
\end{itemize}

\end{description}

\subsection{Discussions and Explanations}

The MPL offers a section \enquote{Responsibilities} which contains nearly all
requirements\footcite[cf.][\nopage wp.\ §3]{Mpl20OsiLicense2013a}. Only for some
subordinated aspects, one has also to reflect other paragraphs\footcite[pars
pro to cf.][\nopage wp.\ §3 - concerning the trademarks]{Mpl20OsiLicense2013a}.
With respect to this structure, we can detect the following tasks:

\begin{itemize}

  \item In a more general attitude, the MPL states that it \enquote{[\ldots]
  does not grant any rights in the trademarks, service marks, or logos of any
  Contributor} -- except as it may be necessary \enquote{to comply with} other
  requirements of the license\footcite[cf.][\nopage wp.\
  §2.3]{Mpl20OsiLicense2013a}. The OSLiC rewrites the message as the
  interdiction to promote own services and products by and with such elements.
  
  \item The MPL also generally prescribes that \enquote{you may not remove or
  alter the substance of any license notice (including copyright notices, patent
  notices, disclaimer of warranties, or limitations of liabiliy) contained
  within the Source Code Form [\ldots]}\footcite[cf.][\nopage wp.\
  §3.4]{Mpl20OsiLicense2013a}. This focussing to the \enquote{substance of any
  license notice} refers to the allowance to \enquote{[\ldots] alter any license
  notices to the extent required to remedy known factual
  innacuracies}\footcite[cf.][\nopage wp.\ §3.4]{Mpl20OsiLicense2013a}.
  Following its principle to offer one reliable way and to ignore variants of
  secondary importance, the OSLiC simplifies this condition to the general
  proscription to modify any licensing material -- namely for all use cases
  [MPL-C1 - MPL-C9]. But for emphasizing that this is a job which must be activily
  done, the OSLiC additionally rewrites this interdiction into all
  \emph{2others} use cases [MPL-C2 - MPL-C9] as the task to retain the licensing
  notices in the form one has obtained them.
  
  \item Moreover, the MPL requires for all \enquote{distributions of [the]
  source [code] form} that all modifications of the software \enquote{[\ldots]
  must be under the terms of (the MPL)} and that the distributor
  \enquote{[\ldots] must inform} all \enquote{recipients} that the software
  \enquote{[\ldots] is governed by the terms of (the MPL), and how (the
  recipients) can obtain a copy of this license}\footcite[cf.][\nopage wp.\
  §3.1]{Mpl20OsiLicense2013a}. For the respective use case (MPL-C2, MPL-C4, MPL-C6,
  MPL-C8), the OSLiC rewrites these conditions so that each MPL source code
  package must mandatorily contain the MPL itself as textfile and an additional
  licensing file or statement strictly following the text given by the addendum
  of the MPL\footcite[cf.][\nopage wp.\ Exhibit A]{Mpl20OsiLicense2013a}. Because
  the MPL is 'only' a license with weak copyleft, the OSLiC proposes to separate
  the MPL licensed, embedded component from the overarching program (MPL-C8).
  
  \item But the MPL does not explicitly require to mark all modifications.
  Nevertheless, this is state of the art in computer emgineering. Therefore,
  with respect to the cases of distributing modified source code (MPL-C4, MPL-C6
  and MPL-C8), the OSLiC proposes to mark all modifications inside of the source
  code and to update the description of the functional changes. In case of
  distributing the modified software in the form of binaries, it should be
  sufficient, to describe the modifications only on the functional level.
  
  \item Furthermore, the MPL requires that the \enquote{Covered Software} -- in
  all cases of distributing it in an \enquote{Executable Form} (MPL-C3, MPL-C5,
  MPL-C7, MPL-C9) -- \enquote{[\ldots] must also be made available in Source Code
  Form [\ldots]} and that the distributor \enquote{[\ldots] must inform
  recipients of the Executable Form how they can obtain a copy of such Source
  Code Form by reasonable means in a timely manner, at a charge no more than the
  cost of distribution to the recipient}\footcite[cf.][\nopage wp.\
  §3.2.a]{Mpl20OsiLicense2013a}. The OSLiC rewrites these conditions as the
  obligation to offer a download service at no charge and to point towards this
  services inside of the distributed package.
  
  \item In this context, the MPL allows to distribute the binaries under terms
  of another license \enquote{[\ldots] provided that that the license for the
  Executable Form does not attempt to limit or alter the recipients’ rights in
  the Source Code Form under this License}\footcite[cf.][\nopage
  wp.\ §3.2.b]{Mpl20OsiLicense2013a}. This possibility might become important for
  those cases where the license compatibility must explicitly be managed.
  Normally, it should be sufficient also to distribute the binaries under the
  MPL. Thus, in case of distributing binaries (MPL-C3, MPL-C5, MPL-C7, MPL-C9), the
  OSLiC proposes to insert into the distribution packages the MPL itself and an
  additional licensing file or statement strictly following the text given by
  the addendum of the MPL\footcite[cf.][\nopage wp.\ Exhibit
  A]{Mpl20OsiLicense2013a}. But again, because the MPL is 'only' a license with
  weak copyleft, the OSLiC proposes to separate the MPL licensed embedded
  component from the overarching program (MPL-C9).
  
  
  \item Finally, one clearly has to state that this rule above evokes a real
  source code distribution which therefore must follow the rules of distributing
  the software. Thus, the OSLiC requires in all cases of a binary distribution
  to execute also the task-lists of the respective source code use cases.

\end{itemize}

%\bibliography{../../../bibfiles/oscResourcesEn}

% Local Variables:
% mode: latex
% fill-column: 80
% End:

% Telekom osCompendium 'for being included' snippet template
%
% (c) Karsten Reincke, Deutsche Telekom AG, Darmstadt 2011
%
% This LaTeX-File is licensed under the Creative Commons Attribution-ShareAlike
% 3.0 Germany License (http://creativecommons.org/licenses/by-sa/3.0/de/): Feel
% free 'to share (to copy, distribute and transmit)' or 'to remix (to adapt)'
% it, if you '... distribute the resulting work under the same or similar
% license to this one' and if you respect how 'you must attribute the work in
% the manner specified by the author ...':
%
% In an internet based reuse please link the reused parts to www.telekom.com and
% mention the original authors and Deutsche Telekom AG in a suitable manner. In
% a paper-like reuse please insert a short hint to www.telekom.com and to the
% original authors and Deutsche Telekom AG into your preface. For normal
% quotations please use the scientific standard to cite.
%
% [ Framework derived from 'mind your Scholar Research Framework' 
%   mycsrf (c) K. Reincke 2012 CC BY 3.0  http://mycsrf.fodina.de/ ]
%


%% use all entries of the bibliography
%\nocite{*}

\section{LGPL Licensed Software in the usage context of \ldots [tbd]}
\label{OSUC-01-LGPL} \label{OSUC-03-LGPL} 
\label{OSUC-06-LGPL} \label{OSUC-09-LGPL}

\label{OSUC-02-LGPL} \label{OSUC-04-LGPL} \label{OSUC-05-LGPL}
\label{OSUC-07-LGPL} \label{OSUC-08-LGPL} \label{OSUC-10-LGPL}

% TODO insert LGPL specifc finder

\subsection{LGPL specific use case 1}
(covers OSUC-X - OSUC-Z)
% \label{OSUC-10-LGPL}
\begin{description}
\item[means] \ldots

\item[covers] OSUC-?? \ldots

\item[requires] the following tasks in order to fulfill the license conditions
\begin{itemize}
  \item \textbf{[mandatorily:]} Ensure \ldots
  \item \textbf{[voluntarily:]} Let \ldots
\end{itemize}

% \item[prohibits] nothing explicitly.
\item[prohibits] the following doings in order to fulfill the license conditions
\begin{itemize}
  \item \textbf{[directly:]} 
  \item \textbf{[indirectly:]}
\end{itemize}
\end{description}

\subsection{LGPL specific use case n}
(covers OSUC-x - OSUC-z)
% \label{OSUC-10-LGPL}
\begin{description}
\item[means] \ldots

\item[covers] OSUC-?? \ldots

\item[requires] the following tasks in order to fulfill the license conditions
\begin{itemize}
  \item \textbf{[mandatorily:]} Ensure \ldots
  \item \textbf{[voluntarily:]} Let \ldots
\end{itemize}

% \item[prohibits] nothing explicitly.
\item[prohibits] the following doings in order to fulfill the license conditions
\begin{itemize}
  \item \textbf{[directly:]} 
  \item \textbf{[indirectly:]}
\end{itemize}
\end{description}

%\bibliography{../../../bibfiles/oscResourcesEn}

% Telekom osCompendium 'for being included' snippet template
%
% (c) Karsten Reincke, Deutsche Telekom AG, Darmstadt 2011
%
% This LaTeX-File is licensed under the Creative Commons Attribution-ShareAlike
% 3.0 Germany License (http://creativecommons.org/licenses/by-sa/3.0/de/): Feel
% free 'to share (to copy, distribute and transmit)' or 'to remix (to adapt)'
% it, if you '... distribute the resulting work under the same or similar
% license to this one' and if you respect how 'you must attribute the work in
% the manner specified by the author ...':
%
% In an internet based reuse please link the reused parts to www.telekom.com and
% mention the original authors and Deutsche Telekom AG in a suitable manner. In
% a paper-like reuse please insert a short hint to www.telekom.com and to the
% original authors and Deutsche Telekom AG into your preface. For normal
% quotations please use the scientific standard to cite.
%
% [ Framework derived from 'mind your Scholar Research Framework' 
%   mycsrf (c) K. Reincke 2012 CC BY 3.0  http://mycsrf.fodina.de/ ]
%


%% use all entries of the bibliography
%\nocite{*}

\section{AGPL Licensed Software in the usage context of \ldots}
\label{OSUC-01-AGPL} \label{OSUC-03-AGPL} 
\label{OSUC-06-AGPL} \label{OSUC-09-AGPL}

\label{OSUC-02-AGPL} \label{OSUC-04-AGPL} \label{OSUC-05-AGPL}
\label{OSUC-07-AGPL} \label{OSUC-08-AGPL} \label{OSUC-10-AGPL}

% TODO insert AGPL specifc mini finder

\subsection{AGPL specific use case 1}
(covers OSUC-X - OSUC-Z)
% \label{OSUC-10-AGPL}
\begin{description}
\item[means] \ldots

\item[covers] OSUC-?? \ldots

\item[requires] the following tasks in order to fulfill the license conditions
\begin{itemize}
  \item \textbf{[mandatorily:]} Ensure \ldots
  \item \textbf{[voluntarily:]} Let \ldots
\end{itemize}

% \item[prohibits] nothing explicitly.
\item[prohibits] the following doings in order to fulfill the license conditions
\begin{itemize}
  \item \textbf{[directly:]} 
  \item \textbf{[indirectly:]}
\end{itemize}
\end{description}

\subsection{AGPL specific use case n}
(covers OSUC-x - OSUC-z)
% \label{OSUC-10-AGPL}
\begin{description}
\item[means] \ldots

\item[covers] OSUC-?? \ldots

\item[requires] the following tasks in order to fulfill the license conditions
\begin{itemize}
  \item \textbf{[mandatorily:]} Ensure \ldots
  \item \textbf{[voluntarily:]} Let \ldots
\end{itemize}

% \item[prohibits] nothing explicitly.
\item[prohibits] the following doings in order to fulfill the license conditions
\begin{itemize}
  \item \textbf{[directly:]} 
  \item \textbf{[indirectly:]}
\end{itemize}
\end{description}

%\bibliography{../../../bibfiles/oscResourcesEn}

% Telekom osCompendium 'for being included' snippet template
%
% (c) Karsten Reincke, Deutsche Telekom AG, Darmstadt 2011
%
% This LaTeX-File is licensed under the Creative Commons Attribution-ShareAlike
% 3.0 Germany License (http://creativecommons.org/licenses/by-sa/3.0/de/): Feel
% free 'to share (to copy, distribute and transmit)' or 'to remix (to adapt)'
% it, if you '... distribute the resulting work under the same or similar
% license to this one' and if you respect how 'you must attribute the work in
% the manner specified by the author ...':
%
% In an internet based reuse please link the reused parts to www.telekom.com and
% mention the original authors and Deutsche Telekom AG in a suitable manner. In
% a paper-like reuse please insert a short hint to www.telekom.com and to the
% original authors and Deutsche Telekom AG into your preface. For normal
% quotations please use the scientific standard to cite.
%
% [ Framework derived from 'mind your Scholar Research Framework' 
%   mycsrf (c) K. Reincke 2012 CC BY 3.0  http://mycsrf.fodina.de/ ]
%


%% use all entries of the bibliography
%\nocite{*}

\section{GPL Licensed Software in the usage context of \ldots}
\label{OSUC-01-GPL} \label{OSUC-03-GPL} 
\label{OSUC-06-GPL} \label{OSUC-09-GPL}

\label{OSUC-02-GPL} \label{OSUC-04-GPL} \label{OSUC-05-GPL}
\label{OSUC-07-GPL} \label{OSUC-08-GPL} \label{OSUC-10-GPL}

% TODO insert GPL specifc mini finder

\subsection{GPL specific use case 1}
(covers OSUC-X - OSUC-Z)
% \label{OSUC-10-GPL}
\begin{description}
\item[means] \ldots

\item[covers] OSUC-?? \ldots

\item[requires] the following tasks in order to fulfill the license conditions
\begin{itemize}
  \item \textbf{[mandatorily:]} Ensure \ldots
  \item \textbf{[voluntarily:]} Let \ldots
\end{itemize}

% \item[prohibits] nothing explicitly.
\item[prohibits] the following doings in order to fulfill the license conditions
\begin{itemize}
  \item \textbf{[directly:]} 
  \item \textbf{[indirectly:]}
\end{itemize}
\end{description}

\subsection{GPL specific use case n}
(covers OSUC-x - OSUC-z)
% \label{OSUC-10-GPL}
\begin{description}
\item[means] \ldots

\item[covers] OSUC-?? \ldots

\item[requires] the following tasks in order to fulfill the license conditions
\begin{itemize}
  \item \textbf{[mandatorily:]} Ensure \ldots
  \item \textbf{[voluntarily:]} Let \ldots
\end{itemize}

% \item[prohibits] nothing explicitly.
\item[prohibits] the following doings in order to fulfill the license conditions
\begin{itemize}
  \item \textbf{[directly:]} 
  \item \textbf{[indirectly:]}
\end{itemize}
\end{description}
%\bibliography{../../../bibfiles/oscResourcesEn}


%%%%%%%%%%%%%%%
% Telekom osCompendium 'for being included' snippet template
%
% (c) Karsten Reincke, Deutsche Telekom AG, Darmstadt 2011
%
% This LaTeX-File is licensed under the Creative Commons Attribution-ShareAlike
% 3.0 Germany License (http://creativecommons.org/licenses/by-sa/3.0/de/): Feel
% free 'to share (to copy, distribute and transmit)' or 'to remix (to adapt)'
% it, if you '... distribute the resulting work under the same or similar
% license to this one' and if you respect how 'you must attribute the work in
% the manner specified by the author ...':
%
% In an internet based reuse please link the reused parts to www.telekom.com and
% mention the original authors and Deutsche Telekom AG in a suitable manner. In
% a paper-like reuse please insert a short hint to www.telekom.com and to the
% original authors and Deutsche Telekom AG into your preface. For normal
% quotations please use the scientific standard to cite.
%
% [ File structure derived from 'mind your Scholar Research Framework' 
%   mycsrf (c) K. Reincke CC BY 3.0  http://mycsrf.fodina.de/ ]
%

% Chapter Abstract
% ----------------
\chapter{Open Source Licenses and Their Legal Environments}

\footnotesize
\begin{quote}\itshape
In this chapter we analyze why to know a license alone is not enough. At the end
you will know that Open Source Licenses are embedded into the legal environment
of a state. And you will know in which sense the German legal environment
predetermines your readings of Open Source Licenses.
\end{quote}
\normalsize{}



%%%%%%%%%%%%%%%
% Telekom osCompendium 'for being included' snippet template
%
% (c) Karsten Reincke, Deutsche Telekom AG, Darmstadt 2011
%
% This LaTeX-File is licensed under the Creative Commons Attribution-ShareAlike
% 3.0 Germany License (http://creativecommons.org/licenses/by-sa/3.0/de/): Feel
% free 'to share (to copy, distribute and transmit)' or 'to remix (to adapt)'
% it, if you '... distribute the resulting work under the same or similar
% license to this one' and if you respect how 'you must attribute the work in
% the manner specified by the author ...':
%
% In an internet based reuse please link the reused parts to www.telekom.com and
% mention the original authors and Deutsche Telekom AG in a suitable manner. In
% a paper-like reuse please insert a short hint to www.telekom.com and to the
% original authors and Deutsche Telekom AG into your preface. For normal
% quotations please use the scientific standard to cite.
%
% [ File structure derived from 'mind your Scholar Research Framework' 
%   mycsrf (c) K. Reincke CC BY 3.0  http://mycsrf.fodina.de/ ]
%

% Chapter Abstract
% ----------------
\chapter{Conclusion}
\footnotesize
\begin{quote}\itshape
This chapter shortly describes what the OSLiC is, how it should be used, and how
it can be read. It shall be written as top-down explanation.
\end{quote}
\normalsize{}



%%%%%%%%%%%%%%%
\chapter{Appendices}

% Telekom osCompendium 'for being included' snippet template
%
% (c) Karsten Reincke, Deutsche Telekom AG, Darmstadt 2011
%
% This LaTeX-File is licensed under the Creative Commons Attribution-ShareAlike
% 3.0 Germany License (http://creativecommons.org/licenses/by-sa/3.0/de/): Feel
% free 'to share (to copy, distribute and transmit)' or 'to remix (to adapt)'
% it, if you '... distribute the resulting work under the same or similar
% license to this one' and if you respect how 'you must attribute the work in
% the manner specified by the author ...':
%
% In an internet based reuse please link the reused parts to www.telekom.com and
% mention the original authors and Deutsche Telekom AG in a suitable manner. In
% a paper-like reuse please insert a short hint to www.telekom.com and to the
% original authors and Deutsche Telekom AG into your preface. For normal
% quotations please use the scientific standard to cite.
%
% [ Framework derived from 'mind your Scholar Research Framework' 
%   mycsrf (c) K. Reincke 2012 CC BY 3.0  http://mycsrf.fodina.de/ ]
%

\section{Some Additional Remarks on the OSLiC Quotation Style}\label{sec:QuotationAppendix}

We have already characterized the general tone of our
footnotes\footnote{$\rightarrow$ p.\ \pageref{QuotationPrinciple} }. Let us now
briefly explain a little peculiarity of our bibliography:

Modern times have also changed the humanities. Formerly a book or an article
must be printed for being ripe to be quoted. Our statements relied on static,
readily prepared works. Nowadays even university libraries sometimes offer those
books and articles as PDF files which are printed in the original. As a scholar,
now you must rely on the equality of the printed version and the PDF file -- at
least with respect to the page numbers and the appearance. You can not verify the
equivalence -- at least to a certain degree.

Moreover: in case of such 'e-books' and 'e-articles' the libraries often do not
offer the pdf files themselves but links to the download pages of the publisher.
Formerly as a scholar you could trust that your readers would be able to
retrieve the quoted work if they want to verify your citations. It's one task of
our libraries to hold available our scientific sources. But now they do not buy
any longer the books, but the right to download files over the university net.
In this case these PDF files are not stored on the serves of the university
library. By using the link provided by the publisher each student or each reader
downloads his own file -- case by case. Therefore -- as a scholar -- you now have
to trust that the publisher, who provides the link, will not change that pdf
file that you have cited.

But it gets even worse: While it might be that publishers modify their work
secretly (even it is not very likely that they do it), it's a definite feature
of the web that its pages are fre\-quen\-tly changed. Hence we must ask
ourselves: Can we seriously argue on the basis of statements and documents which
might disappear? Can we quote such possibly volatile sources? The problem is: we
must do it, especially if we write about an internet topic -- and even if we want
to write a really reliable compendium.

So, what can we do? First, we must confide in our readers, that they either
will retrieve our sources or -- if they can not find them -- that they
believe that we really have found and read what we have written and
quoted. Second, we store all these e-wares\footnote{Take this little word as
(new) generalization of 'e-book', 'e-article', 'e-paper' and so on.} we
read\footnote{But because of the copyright we ourselves are naturally not
allowed to offer a download link for them or to send a copy of it to those who
want to verify our quotes.}. And thirdly we should lay open to our readers the
different levels of reliableness of our sources. Therefore we use
the following markers in our bibliographic data\footnote{And another hint: Nowadays sometimes
even scientific libraries don't offer exact 'e-copies' of the original. In
some cases one can only get html-versions of articles which formerly were
printed as part of journals. In these case the scholar has to use sources which
lost their original page-numbers. The same can happen to articles of proceedings
etc.\ which are now only offered as autonomous pdf files with an internal paging.
If we quote such kind of articles we try to specify the number of the quoted
article in the original row of articles, added -- if possible -- by an internal
page number. But naturally we also try to follow the bibliographic data
delivered by that organization which distributes these kind of copies.}:

\begin{itemize}
  \item Print / Copy:- The source is printed and we saw either the printed work
  really or we get an official copy by our library. Hence you should also be able
  to get the work in a library, at least in those we used (UB Frankfurt or ULB
  Darmstadt).
  \item BibWeb/[PDF/\ldots] :- The source might be printed, but we read only the
  electronic version (PDF or other type of format), offered by and over the
  net of our university libraries (UB Frankfurt or ULB Darmstadt).
  \item FreeWeb/[PDF/\ldots] :- We read the electronic version offered by the
  free web. In this case we add the url\footnote{Please note: Long urls often
  destroy the pleasing appearance of a text because it's difficult to wrap the
  lines acceptably. Hence we wished to make it easier for LaTeX to do this job.
  Therefor we sometimes split the urls and inserted blanks. So you have to erase
  all blanks if you want to verify our urls.} and the date when we downloaded /
  saw the text.
\end{itemize}


%\bibliography{../../../bibfiles/oscResourcesEn}


% Telekom osCompendium 'for being included' snippet template
%
% (c) Karsten Reincke, Deutsche Telekom AG, Darmstadt 2011
%
% This LaTeX-File is licensed under the Creative Commons Attribution-ShareAlike
% 3.0 Germany License (http://creativecommons.org/licenses/by-sa/3.0/de/): Feel
% free 'to share (to copy, distribute and transmit)' or 'to remix (to adapt)'
% it, if you '... distribute the resulting work under the same or similar
% license to this one' and if you respect how 'you must attribute the work in
% the manner specified by the author ...':
%
% In an internet based reuse please link the reused parts to www.telekom.com and
% mention the original authors and Deutsche Telekom AG in a suitable manner. In
% a paper-like reuse please insert a short hint to www.telekom.com and to the
% original authors and Deutsche Telekom AG into your preface. For normal
% quotations please use the scientific standard to cite.
%
% [ File structure derived from 'mind your Scholar Research Framework' 
%   mycsrf (c) K. Reincke CC BY 3.0  http://mycsrf.fodina.de/ ]

%


%% use all entries of the bibliography
%\nocite{*}


\section{Some Widespread Open Source Myths}

From the viewpoint of an internet student we have to consider that the web
offers a mass of rumors concerning the nature of open source software
(Licenses). Here are some of the myths\footcite[At least one time even a
scientific legally discussing book is talking about the \enquote{myth around open
source licenses} - although only as part of  the title: cf][1ff,
especially 209ff]{GuiOvd2006a} we met:
 

\begin{description}
  \item[open source tries to improve the world ethically] :- No, there's a clear
  ban to exclude persons, groups, purposes. Thus, there is no chance to exclude
  anyone from using open source software because he is an ethical or moralic
  malefactor.
  \item[Changed open source software must be re-published] :- No, in a double
  sense! There are OS licenses which allow the proprietarization of the
  modified code. And even the LGPL and the GPL, which clearly try to prevent
  the proprietarization, do not require generally that a modified code must be
  (re-)published. Only if you give your modfied (L)GPL licensed application as
  binary to anyboday, then you have to handover the modified code too.
  \item[Modified open source software must be given back to the whole community]
  :- No. Again, there are OS licenses which allow the proprietarization of the
  modified code. And even the LGPL and the GPL - which clearly require, that you
  also publish the modified code, if you give the modified binary to anybody --
  do not require that you distribute your modification around the world. LGPL and
  GPL clearly say that you have to hand over the code to those persons which you
  give the binary. And if you only give your improvement only one person or a
  group of person, then you must handover your code only to that persons or
  only to all members of that group.
  \item[Published open source software is open for ever] :- No, if this myth
  says that also all future versions will have to be distributed under an open
  source license. The copyright holder ever holds the copyright. They can change
  the licence of next release of its software -- but only for the following
  release, not for the current or for former versions. Those releases, which
  already have been distribuited under an open source license, indeed remain
  open.
  \item[Software can either be open source software or proprietary software] :-
  No. The copyright holders themselves can additionally distribute the code
  under other conditions when ever they want to do it. That's not a question of
  the licence, but of the copyright.  
  \item[The opposite of open source software is commercial Software] :- No.
  First, you are also allowed to use the open source software in any commercial
  purpose. There's only one point which is excluded in OSS: you are not allowed
  to ask for a licence fee if you distribute 'open source software'. Second,
  there are many other forms like freeware, public domain software or anything
  else which is neither open source software nor Commercial Software. It's
  pointless to take the question of money as a criterion for distinguish open
  source software and its opposite. Moreover: Proprietary Software as opposite
  of open source software should be defined ex negativo: all kind of software,
  which does not fit the OSD is proprietary.
  \item[open source software prohibits to earn money] :- No,
  you are allowed to invent each business model you want. There's only one
  exception: you are not allowed to ask for a licence fee if you distribute
  'open source software [Achtung: sollte eigentlich nur für GPL gelten].
  Historically this mistake might be evoked by Debian: The GNU project missed
  its kernel while the Linux kernel was already distributed as part of
  collections which also include GNU software. Then, in 1983? Ian Murdock was
  supported by RMS and its FSF to build a really free distribution (Debian)
  containg GNU software and the Linux kernel. But Ian Murdock states also, that
  Debian does not want to earn money.
% TODO find sources for indirect citations
% TODO: check, wether OSD requires license fee free distribution
  \item[Modifications of open source software must be marked] :- No. This is not
  a defining postulation of the OSD. The OSD allows licenses to require the mark
  of modifications. But it does not require from all licenses to require the mark
  modifications for being an open source license.
  \item[Modifications of open source software must be marked by your personal
  data] :- No, it is only required to mark modifications so that a reader could
  distinguish the modifications from the original code. It's required for saving
  the integrity of the original author. And therefore it is not required as a
  constitutiv criterion by the OSD. It might be that a license additionally
  requires your name. But that is not feature of open source software in general.
  And at least the licenses discussed by us do not require to insert your name.
% TODO: check wether any of our licenses reuire that you mark modifications by
% your personal data / real name  
  \item[The open source Definition determines the conditions to use open source
  software] :- No. The \emph{Open Source Definition} determines which licenses
  are open source licenses, nothing more. The OSD is a set of necessary
  conditions to be an open source license. It determines the freedom and the
  responsibilities of a user as a set of more or less abstract rules. But it
  does not constitute a set of sufficient tasks which a user has to perform for
  fulfilling any open source license. Open source licenses may differ by
  instantiating the OSD criteria. So, if you want to know what you have to do to
  fulfill a license, you have to go back to the real license of that software
  you are using.
\end{description}

%\bibliography{../bibfiles/oscResourcesEn}


% Telekom osCompendium 'for being included' snippet template
%
% (c) Karsten Reincke, Deutsche Telekom AG, Darmstadt 2011
%
% This LaTeX-File is licensed under the Creative Commons Attribution-ShareAlike
% 3.0 Germany License (http://creativecommons.org/licenses/by-sa/3.0/de/): Feel
% free 'to share (to copy, distribute and transmit)' or 'to remix (to adapt)'
% it, if you '... distribute the resulting work under the same or similar
% license to this one' and if you respect how 'you must attribute the work in
% the manner specified by the author ...':
%
% In an internet based reuse please link the reused parts to www.telekom.com and
% mention the original authors and Deutsche Telekom AG in a suitable manner. In
% a paper-like reuse please insert a short hint to www.telekom.com and to the
% original authors and Deutsche Telekom AG into your preface. For normal
% quotations please use the scientific standard to cite.
%
% [ File structure derived from 'mind your Scholar Research Framework' 
%   mycsrf (c) K. Reincke CC BY 3.0  http://mycsrf.fodina.de/ ]
%

% Chapter Abstract
% ----------------

\footnotesize \begin{quote}\itshape This section outlines reflections by which
we initially focused ourselves on the question why we need an OSLiC and how its
content and form should be derivated from these needs.
\end{quote}
\normalsize{}

\subsection{Why}

Do we need another book about open source? Do \emph{you} need another book about
open source software? Let us address this question from the viewpoint of what we
already know, what we instinctively believe and what we may have heard. For
example you may presume one or more of the following statements are correct. Or
you may even have experienced similar perceptions from your peers or managers.
Or you have been told they describe 'open source':

\begin{itemize}
  \item The \emph{Open Source Definition} offers rules to use open source software.
  \item Modified open source software must be published.
  \item Modified open source software must be given back to the community.
  \item All generations of open source software will remain open for ever.
  \item Software can either be open source software or proprietary software.
  \item The opposite of open source software is commercial software.
  \item open source software prohibits to earn money.
  \item Modifications of open source software must be marked explicitly.
  \item Modifiers of open source software must identify themselves.
  \item When distributing an open source binary it’s enough point to a download
  page to obtain the source code.
  \item The aim of open source software is to improve the world ethically.
  \item open source software is viral and infectious.
\end{itemize}

Do these conceptions sound familiar to you? Unfortunately, whatever we might
believe or wish for, these concepts are incorrect. Naturally we will discuss
this issue later on. For the moment let us assume they are indeed
incorrect\footnote{For those who want directly verify our argumentation, we have
generated a condensed summary of the arguments and citations. You can find this
summary in our appendices.}.

So, again: Do \emph{we} need another book about open source software? \emph{We},
that is -- in this case and at least initially -- the large German company
\textit{Deutsche Telekom AG}. Arguing from the perspective of a large company
requires not only identifying the common misconceptions, but catering for the
unique needs of a large Enterprise. And indeed the very size of the company
brings its own problems.

Large companies use more open source software in more varied contexts than small
companies. There is an important question that every company should ask:
\emph{'Are we sure that we respect all those requirements of open source
software we have to respect?'}. But large companies cannot answer this question
as easily as small companies: the large number of diverse open source
deployments in different contexts mean that case by case governance, a model
that may work in small concerns, is far from appropriate for our needs. This
leads to wasting both time and money. Further, the chances of success are small:
training at least one employee in each software team as an open source software
License expert is unrealistic in terms of cost-efficiency and reliability.

Nevertheless even large companies want to and try to fulfill the rules of open
source software thoroughly -- especially \emph{Deutsche Telekom AG}. When this
company realized that the question \textit{Are we sure that we respect all those
rules of open source software correctly which we have to respect} could be
problematic, it directly asked some of its employees known as open source
enthusiasts to establish a service and a process for answering this question.

So, it is no surprise that we, the initial authors of this \textit{Open Source
License Compendium}, were asked by our employer \emph{Deutsche Telekom AG}.
Naturally we were proud to work on an open source topic officially. But while we
were doing our job we had to ask ourselves if \emph{we} perhaps needed another
book on open source. Our answer was \textit{Yes, we do!} Let us shortly explain,
why:

First, we already knew that there exists supporting software. These
meta-pro\-grams take the code of any other application and try to list those
open source components being 'covered' by that application\footnote{As general
examples let us mention Palamida (\texttt{http://www.palamida.com/}) and
BlackDuck (\texttt{http://www.blackducksoftware.com/}).}. But we had also
already realised that this supporting software did not always match the way we
thought the problem should be solved. Second, we recognized fairly quickly that
we need a reliable guide. We personally were asked to give the \emph{ok} for
projects of our company. We could not answer such requests on the base of
\textit{'Oh yes, I read this in the \emph{Heise-Ticker} a few days ago'} -- even
if the \emph{Heise-Ticker} had described the situation completely correctly. We
ourselves had to be more reliable than this\footnote{But of course, we have to
do ourselves the honor of conceding that we -- like many many other German open
source enthusiasts -- love using the \emph{Heise-Ticker} as main IT information
source. Unfortunately, its reputation is stil not high enough that its news can
directly be cited.}. Naturally we already knew a great deal about open source
software. Even so, our knowledge was not as systematic as necessary. We looked
for an open source compendium which adequately described what a project or
product development team had to do to fulfill the criteria of its open source
licenses. We wanted to use that compendium to the basis of our recommendations.

We were very thorough but we did not find what we were looking for. Our 'little'
bibliography attest our seriousness. What we found was a lot of information
releated to individual issues spread over many sources. We did not find answers
to our question even in the specific literature. Let us describe three little
steps to increase the understanding of the issue:

Without open source licenses there is no open source movement. Nevertheless in
dealing with open source licenses, this is sometimes neglected. Take the
\emph{Apache Web Server} as an example: No doubt, it is one of the most important
pieces of open source software\footnote{To prove that the \textit{Apache} is
really a piece of open source software one must execute a set of steps: First,
you have to note, that \emph{Apache} is something like a meta project, covered
by the \emph{Apache Software Foundation}, also known as \emph{ASF} (cf.
\texttt{http://www.apache.org/}, wp). Thus, you can not directly jump into
the \emph{Apache License}. First of all you have to visit the project site (cf.
\texttt{http://httpd.apache.org/}, wp) even if at the end its license link
leads you back to the general \emph{Apache License sub site} (cf.
\texttt{http://www.apache.org/licenses/}, wp) which announces, that \enquote{all
software produced by The Apache Software Foundation or any of its projects or
subjects is licensed according to the terms of the documents listed
below}. Only now you can use the offered link for switching to the
\emph{Apache License}, Version 2.0, if you want to check your rights and duties.
But that is difficult. There does not exist any simple list what you have to do
for fulfilling the license. Even the faq (cf.
\texttt{http://httpd.apache.org/docs/2.2/faq/}, wp) -- meanwhile being moved to
a wiki -- only says that the server \enquote{[\ldots] comes with an unrestrictive
license} and that you are allowed to put the code on a CD (cf.
\texttt{http://wiki.apache.org/httpd/FAQ}, wp). Hence, from the viewpoint of
the ASF the license itself shall answer all questions. [Reference download for
all urls: 2011-08-31] } with a specific license\footcite[cf.][\nopage
wp]{AsfApacheLicense20a}. Moreover: the success of the open source movement
in the commercial world depends directly on the decision of IBM to replace its
corresponding own component in the \textit{IBM WebSphere Application Server}
with the free \textit{Apache Web Server}\footcite[cf.][287ff]{Moody2001a}.
Meanwhile many companies use the \textit{Apache Web Server} to act as a web
provider. Currently the \emph{Apache http server} -- as it has to be named
correctly -- is used more than twice as much as all the other http server
software together\footcite[cf.][\nopage wp]{Netcraft2011a}. Hence many business
models depend on the Apache License. Another aspect is that even the famous
\emph{Apache Cookbook}, which explains the installation, the configuration, and
the maintainance of an Apache Web Server in details\footcite[cf.][\nopage et
passim]{CoaBow2004a}, does not mention anything about the license which allows
for installation, configuration and maintenance. Neither the index lists the
word 'license'\footcite[cf.][245ff, esp.\ p.\ 250]{CoaBow2004a}, nor the chapters
'Installation'\footcite[cf.][1ff]{CoaBow2004a} or the chapter
'Miscellaneous'\footcite[cf.][219ff]{CoaBow2004a} mentions the license question
in a serious way. There's only one short hint as to the advantage of open source
software, i.e.\ that everybody is allowed to install it\footcite[cf.][1: \enquote{%
\ldots einer der Vorzüge von open source software besteht darin, dass
je\-der\-mann die Erlaubnis zur Erzeugung eines eigenen Installationskits hat
}]{CoaBow2004a}. Can you be sure that you are allowed to do what you are
doing on the base of such a phrase?

Naturally, the \emph{Apache Cookbook} is not a book for lawyers, it is a book for
administrators and developers. They do not want to get bogged down by
legalities, they want to set up an Apache Web Server as fast as possible and get
down to work. Indeed, the Apache Cookbook offers a good support. But not only as
a company you have to ask yourself whether you are really allowed to do what you
are doing. Can you find the answer in the \emph{Apache Cookbook}? No. Can you
find it in the license itself? Yes, but it is difficult\footnote{And do we
really want our developers and maintainers to read the original licenses? Do we
really want them to discover that they also have to check the licenses of the
used modules?}. So again: Can you find your answer in another book, which is
\emph{Amazon's} current top recommendation for the search term \emph{'apache
server'}\footnote{Tested on \texttt{http://www.amazon.de/} at 2011-08-31.}? Not
really: Sascha Kersken's Apache 2.2 Handbook offers a license chapter, but it is
only two pages long\footcite[cf.][111f]{Kersken2009a}. Moreover, the rights and
duties are condensed into just 5 bullet points which taken together do not
explain when the software and the license have to be handed over to a customer
and when you are allowed to hide your
improvements\footcite[cf.][112]{Kersken2009a}.

This brings us to the question of what prevents us from using something like a
\emph{'general license cookbook'} which explains all the necessary details and
which offers  quick access to the relevant points:

Of course we also browsed the internet. At least for German speaking people
there is an excellent site concerning the topic \emph{open source licenses}.
offered by \textit{iffross}, which, loosely translated, means an
\textit{Institute for Legal Aspects of the Free and open source
software}\footnote{originally: \enquote{Institut für Rechtsfragen der Freien und
open source software}. Main entry point for its site is the URL
\texttt{http://www.ifross.org/}.}, founded in 2000 as a private institute to
track the phenomenon 'free software' from the viewpoint of (German)
lawyers\footcite[cf.][\nopage wp]{ifross2011b}. Besides many other
aspects this site offers a very well and thoroughly elaborated
FAQ\footcite[cf.][\nopage wp]{ifross2011c} and a large list of open
source licenses and other related licenses: moreover, evidently it is
classifying the open source licenses in those 'without copyleft-effect' (BSD),
in those with 'strict copyleft-effect' (GPL) and in those with 'restricted
copyleft-effect' (LGPL)\footcite[cf.][\nopage wp]{ifross2011a}.

However, even this excellent site does not fulfill our needs. It does not offer
those context specific to-do lists which companies, developers or project
managers can use to ensure their open source software is used in a regular
manner.

We therefore evaluated that standard book which is listed in the most legal
bibliographies\footnote{at least in that German judicial literature dealing with
open source}: the book of Jaeger and Metzger which concerns -- loosely translated
-- \textit{the judicial framework requirement for open source
software}\footcite[cf.][V -- It can not be any surprise that both authors,
Mr. Jaeger and Mr. Metzger are members of ifross (cf.
\texttt{http://www.ifross.org/personen/}, wp)]{JaeMet2002a}. Even the most
earliest edition of this book already had a clear structure in its chapter
'copyright': For each license mentioned (or at least for each license cluster)
it offered a subchapter for the rights and a subchapter for the
duties\footcite[cf.][30ff]{JaeMet2002a} of the software user\footcite[For
getting a good survey of the structure and the line of thought see the contents
cf.][VIIIf]{JaeMet2002a}. Many other important aspects of the topic
\textit{open source} are discussed, too\footcite[pars pro toto: have a
look at the chapter concerning the liability: cf.][137ff]{JaeMet2002a}.

But we needed more than this. Despite the quality of the book we were certain
that we could not hand over this book to our programmers with the recommendation
\textit{check your touched licenses and follow the instructions of the relevant
subchapters\ldots}. This book did not contain simply checkable to-do lists,
neither in the first edition\footcite[cf.][VIff]{JaeMet2002a} and in the second
edition\footcite[cf.][VIIff]{JaeMet2006a} nor in the recently published third
edition\footcite[cf.][VIIIff. Naturally we use this latest edition for adopting
or discussing systematical aspects]{JaeMet2011a}. So, how can a company or a
developer or a project manager be sure of fulfilling the requirements of the
open source licenses sufficiently if he/she does not have a verified list
telling him \textit{'do this, and in case of that, do that, and finally do also
this'}? Why should he himself implicitly become an open source licenses expert
who has to extract the necessary steps out of the literature?

While we were searching for an existing open source compendium, we found an
article with the title 'Compendium for the Publication of open source
software'\footnote{approximately translated}. It aims to be a 'pragmatic
guidebook' and an 'assistance' for 'publishing software under the conditions of
an open source license'\footcite[cf.][166f (originally: ein
\enquote{pragmatischer Ratgeber} zur \enquote{Veröffentlichung einer Software
unter den Rahmenbedingungen einer Open-Source-Lizenz}) ]{BreGlaGra2008a}.
Moreover, at the end of this article, its authors formulate ambitiously that
their 'guide' should be carried out, section by section -- for getting a legally
water tight process of publishing open source software\footcite[cf.][186
(originally: ein \enquote{Ratgeber}, der es erlaubt \enquote{ (\ldots) die zu
berücksichtigende Aspekte (strukturiert abzuarbeiten) (\ldots) } und einen
\enquote{rechtlich nicht angreifbaren Veröffentlichungsprozess} zu
ermöglichen) ]{BreGlaGra2008a}.

The authors of this article describe something close to what we were looking
for. Indeed, the article lists important aspects which have to be taken in
consideration if you want to deal open source software correctly: It announces
that no obligation exists to publish code either if you embed GPL code into your
proprietary code or if you modify the GPL code. It is only if you hand over your
binary to other persons that you have to distribute the code too, but only to
them and not to the general public\footcite[cf.][170 and 181]{BreGlaGra2008a}.
Additionally the articles explains exactly that software -- at least in Germany --
can only be acknowledged as open source software by transferring the rights to
use -- the \emph{'Nutzungsrechte'} -- to other people, while the copyright itself
-- the \emph{'Urheberpersönlichkeitsrecht'} -- is not transferable and belongs to
the author\footcite[cf.][173]{BreGlaGra2008a}. Moreover, besides other aspects
the articles briefly and deeply discusses the problem of the No-Warranty-Clauses
which are not valid in Germany and which will therefore automatically be
replaced by the liability rules for a
donation\footcite[cf.][177]{BreGlaGra2008a}. And last but not least this article
actually summarizes the idea of Copyleft and the differences between LGPL and
GPL\footcite[cf.][181]{BreGlaGra2008a}.

However some gaps remain. The article does not analyze in which cases a
University or a company perhaps \emph{must} publish its developments based
on open source software. It does not discern between different licenses
and conditions. It also does not discuss what Universities or companies,
which (re-)use and/or distribute open source software (internally), must do to
fulfill the touched open source licenses. And finally this article
does not offer the step by step list as promised.

We did, however, feel supported by this article, in two ways. First, it was a
well written summary of some main problems. Second, it stated the necessity to
have a compendium for being able to establish a legally 'water-tight' process of
publishing open source software\footcite[cf.][186]{BreGlaGra2008a}. We
seemed to be justified in our assumptions. But the open source compendium we
were looking for had to be more practical, more processable, more distinguishing
and more elaborated.

So again: Did we need a new book about open source software? We had looked for a
reliable integrated open source compendium. But we found separate pieces of
information and -- as we know today -- some rumors. Our answer was clear:
naturally we did not need a new general book about open source. But what was
lacking was a description of what responsible developers, project managers or
product developers require to fulfill open source licenses. We needed an
\textit{Open Source License Compendium}.

At the best such an \textit{Open Source License Compendium} would contain a set
of simply to process \textit{'For-Fulfilling-The-License-To-Do-Lists'}.
Additionally it should offer an intuitively user-friendly search option for
these lists. In any case, it should share developers and project managers the
effort of having to become open source license experts. For the other users, it
should also clearly explain why one has to do this and not that. Hence a
reliable \textit{Open Source License Compendium} should not only list what one
has to do, but should offer both, thoroughly verified reliable details and
clearly condensed guidance.

Although we did not find such an open source compendium we were familiar with
the spirit of the open source community. Hence we followed one of its most
simple rules: \emph{'what you miss you must develop on your own'}. Some
principles should help us to achieve our targets:

\begin{description}
  \item[To-do lists as the core, discussions around them]: Our work should be
  split into two parts. As its core we wanted to offer a
  set of to-do-Lists. Each of these lists should be relevant to one specific
  open source license and should be clustered by the open source specific use
  cases. Around this all those aspects of open source software which influence the
  interpretation of the licenses and the rules core should be precisely
  characterized. Nevertheless, the users should be able to skip
  details and go directly to the section they require.
  \item[Quotations with thoroughly specified sources]: Even if our users should
  not be obliged to read every part of the compendium they should not be
  required to believe us. We wanted to be revisable. Because our sources and our
  conclusions should be easily verifiable, we decided to use the academic
  citations and list bibliographic data extensively on the basis that our task
  should be to collect information, not to invent new 'facts'.
  \item[Not the internet alone, also books and articles]: We wanted to go back
  to the originals even if the internet was full of more or less modified
  copies. We wished to get reliable facts and descriptions. Therefore we decided
  to evaluate not only the internet but also scientific sources -- for example --
  offered by university libraries.
  \item[Not clearing out the forest land, but cutting out a swathe]: Even if we
  had to deal with licenses and their legal aspects we did not want to get lost
  in detailed discussions. It should not be our task to find out whether a
  specific kind of handling would still be legal or already forbidden.
  We did not want to fight against the licenses. We did not want to stretch
  their ambit or to test their boundary. We wished to accept open source
  licenses as they are: rules written from developers for developers. And even
  if some parts of these licenses would not be valid with respect to a legal
  system\footcite[And indeed for example for the GPL one can argue in this way:
  Even if you take the GPL as a contract of the type 'donation' respectively
  \enquote{Schenkung}, it is presented in the form of AGBs respectively
  \enquote{Allgemeine Geschäftsbedingungen} and must therefore follow the
  general AGB rules.'Regrettably' in Germany these general AGB rules do not
  allow to exclude each type of warranty. If we follow Oberhem, §11 and §12 of
  the GPL must be invalid in Germany because of these general AGB rules.
  Moreover, for Oberhem even §5 -- the important clause of the GPL by which you
  can only get the right to use and to distribute GPL software if you respect
  the rules of the GPL -- seems also to be invalid respectively
  \enquote{unwirksam}. But the good message is that the GPL as whole is not
  invalid even if it contains invalid clauses.][128, 133ff, 150ff, esp.\ 146,
  159]{Oberhem2008a}, we wanted to take them as our guideline -- at least while
  they do not violate more general laws\footnote{what they clearly do not do!}.
  We simply wanted to \emph{find one proven way} to cross the maybe slightly
  unsure forest of open source licenses. Even if indeed some clauses of the
  licenses finally were not enforceable against us we wanted to respect them
  'voluntarily'. We wanted to deliver a set of rules which support users and
  remove the possibility of becoming involved in license disputes with open
  source developers or the Free Software Foundation.
  \item[Take the text seriously]: On the other side we wanted to take our
  license texts as they were. If they lacked anything\footcite[The systematical
  underdetermination of licenses is a problem being also known in the open
  source respectively Free Software movement. Following the biography of RMS his
  main judicial counselor Moglen has stated, that \enquote{there is uncertainty
  in every legal process (\ldots) } and that it seemed to be silly to try
  \enquote{(\ldots) to take out all the bugs (\ldots)}. Nevertheless -- so
  Moglen resp.\ Williams -- the goal of Richard Stallman was \enquote{the complete
  opposite}: He tried \enquote{(\ldots) to remove uncertainty which is
  inherently impossible}. But -- and that's the nub of this analysis --
  Moglen had to follow Stallmann because of RMS character. And he had to
  summarize their work so, that \enquote{(\ldots) the resulting elegance (of the
  GPL; KR.), the resulting simplicity (of the GPL; KR.) in design almost
  achieves what it has to achieve}. Hence we are asked to take the license
  texts themselves seriously. cf.][177f]{Williams2002a}, we would interpret the
  open issues in the spirit of the open source idea. But where the text was
  clear and definite we wanted to take its propositions as a definite decision --
  even if that meaning stood against well known open source 'facts'.
  \item[Trust the swarm]: We did not want to use our own research alone as a
  basis. We knew that the swarm is ever stronger than a set of some randomly
  selected experts. Therefore we decided to publish our text as a still
  unfinished work, starting with an early release 0.2. And then we wanted to
  invite the community to complete the compendium together with us. We would
  elaborate our open source compendium as a set of LaTeX- and BibTeX files which
  could be developed and managed in GIT or any other version control system. And
  finally we would publish our text under a Creative Commons Attribution-Share
  Alike German 3.0 license, to allow other people to correct us, to help us or
  even to take our results for their own purposes.
\end{description}

And so we did. Here is the result. Feel free to use it -- according to our
licensing.

\subsection{What}

Now we can briefly explain how one should be able to use the compendium:

% TODO adopt real chapter structure into the prolegomena 
\begin{description}
  \item[The Same Idea, Different Licenses] :- Here you will find background
  information to help you interpret open source licenses in the sense of the
  \emph{Free Software movement}\footcite[At least at this place you are perhaps
  expecting that we use the logograms FLOSS, F/OSS, F/LOSS, or whatever. As you
  will read later on the word \textit{Free} is ambiguous and has strained the
  use of the concept \textit{Free Software}. Later on we will also talk about
  the invention of the concept \textit{open source} designed as a 'replacement'
  and acting as a 'splitter'. The mentioned logograms are introduced to
  re-establish or -- at least -- to underline the common history and the common
  center of 'both' movements, whereby the word \textit{Libre} shall resolve the
  ambiguity of the word \textit{Free}. For a first survey cf.] [\nopage
  wp]{wpFloss2011a}, the \emph{open source software movement}\footcite[For
  another brief and informative introduction cf.][231ff esp.\ p.\ 
  232f]{Fogel2006a}, or the GNU-Project\footnote{ We ourselves will stay with the
  concept \textit{open source} because the OSD specifies the scope of our
  analysis. But we do it with a deep obeisance to Stallmann and the FSF -- even
  if we know that this will not protect us from the thunderbolt of RMS.}. We discuss
  different ways to cluster open source licenses. Finally we present our own
  taxonomy based on the labels 'protecting the developer', 'protecting the
  licensed code' and 'protecting the on-top-developments'. If you are familiar
  with the methods of grouping different open source licenses and particular
  if you know that you can not authorize your doings on the base of descriptions
  of such license groups, then it is enough, in order to understand our line of
  thought, to briefly note our taxonomy and its wording.
  \item[The Problem of Derivated Works] :- This chapter is important. In the
  spirit of software developers we try to explain which kinds of programming
  evoke a derivated work and which not. Our to-do lists will refer to this
  analysis.
  \item[The Problem of Combining Different Licenses] :- You should
  not ignore this chapter. We will explain why and how combining software
  of different licenses is not as dangerous as it is often told. The results of
  this chapter influence the structure of our to-do lists.
  \item[open source software and Money] :- Here we will shortly
  discuss ways in which money is no problem. If you already know that it is only
  prohibited to require payment for the act of licensing a piece of open source
  software to second or third parties and if you already know that this is only
  forbidden by some licenses, and not by all, than you can postpone the reading
  of this chapter.
  \item[The Problem of Implicitly Freeing Patents] :- Here we
  will illuminate some aspects of software patents and how the are handled by
  some open source licenses. You should know what licenses implicitly do with
  your patents. But it is not our intention to write a software patent
  compendium.
  \item[Open Source Use Cases as Principle of Classification] :- This is an
  important chapter. We explain our categories 'Use as it is', 'Modify the
  Code', 'With Redistribution', 'Without Redistribution', 'Isolated Initial
  Development', 'On-Top-Development': we develop and discuss our taxonomy with
  respect to the side effects of 'combining different licenses' and 'generating
  derivated works'. This taxonomy will determine the following chapters.
  \item[open source licenses: Find Your Specific To-do Lists] :- This is a kind
  of summary which joins the relevant aspects and elaborates the 'finder
  for your to-do lists'. This is the chapter which you probably will reuse
  frequently, even if you do not want to read any of our explanations.
  \item[open source license Fulfillment: Classified To-do Lists] :- This chapter
  offers all classified to-do lists. The structure of its subchapters will
  match the structure of our finder and the structure of our taxonomy.
  \item[open source licenses and Their Legal Environments] :- Here we discuss
  why using open source software in a regular manner is not only a question of
  the licenses themselves but of the kind of the surrounding legal system.
  \item[Appendices: Some Widespread Open Source Myths] :- Here we make good on
  our promise to explain why all the propositions mentioned at the beginning of
  this chapter are wrong. You might read this chapter as a special introduction
  or a reminder epilogue whenever you want to do.
\end{description}


%\bibliography{../../../bibfiles/oscResourcesEn}

% Local Variables:
% mode: latex
% fill-column: 80
% End:



\small
%\theendnotes


\footnotesize
% Telekom osCompendium English Nomenclation Tokens Include Module 
%
% (c) Karsten Reincke, Deutsche Telekom AG, Darmstadt 2011
%
% This LaTeX-File is licensed under the Creative Commons Attribution-ShareAlike
% 3.0 Germany License (http://creativecommons.org/licenses/by-sa/3.0/de/): Feel
% free 'to share (to copy, distribute and transmit)' or 'to remix (to adapt)'
% it, if you '... distribute the resulting work under the same or similar
% license to this one' and if you respect how 'you must attribute the work in
% the manner specified by the author ...':
%
% In an internet based reuse please link the reused parts to www.telekom.com and
% mention the original authors and Deutsche Telekom AG in a suitable manner. In
% a paper-like reuse please insert a short hint to www.telekom.com and to the
% original authors and Deutsche Telekom AG into your preface. For normal
% quotations please use the scientific standard to cite.
%
% [ File structure derived from 'mind your Scholar Research Framework' 
%   mycsrf (c) K. Reincke CC BY 3.0  http://mycsrf.fodina.de/ ]


%\abbr[aaO]{a.a.O.}{am angegebenen Ort}
%\abbr[ds]{ds.}{kollektiv für ders., dies., \ldots}
\abbr[etseqq]{et seqq.}{and the following ones}
\abbr[id]{id.}{idem = latin for 'the same', be it a man, woman or a group\ldots}
\abbr[ibid]{ibid.}{ibidem = latin for 'at the same place'}
\abbr[ifross]{ifross}{Institut für Rechtsfragen der Freien und Open Source
Software}
\abbr[lc]{l.c.}{loco citato = latin for 'in the place cited'}
\abbr[np]{np.}{no page numbering}
\abbr[wp]{wp.}{webpage / webdocument without any internal (page)numbering}
\abbr[nst]{n.st.}{not stated}
\abbr[njear]{n.y.}{year not stated / no year}
\abbr[nlocation]{n.l.}{location not stated / no location}
\abbr[ub]{UB}{'Universitätsbibliothek' = library of university X}
\abbr[ulb]{ULB}{'Universitäts- \& Landesbibliothek' = library of university and state X}
\abbr[apl]{ApL}{Apache License}
\abbr[bsd]{BSD}{Berkeley Software Distrobution (License)}
\abbr[mit]{MIT}{Massachusetts Institute of Technology (License)}
\abbr[mspl]{Ms-PL}{Microsoft Public License}
\abbr[pgl]{PgL}{Postgres License}
\abbr[php]{PHP}{PHP (License)}
\abbr[epl]{EPL}{Eclipse Public License}
\abbr[eupl]{EUPL}{European Union Public License}
\abbr[lgpl]{LGPL}{GNU Lesser General Public License}
\abbr[mpl]{MPL}{Mozilla Public License}
\abbr[gpl]{GPL}{GNU General Public License}
\abbr[agpl]{AGPL}{GNU Affero General Public License}
\abbr[nabbr]{n.abbr.}{no abbreviation (known)}
% Telekom osCompendium English Nomenclation Tokens Include Module 
%
% (c) Karsten Reincke, Deutsche Telekom AG, Darmstadt 2011
%
% This LaTeX-File is licensed under the Creative Commons Attribution-ShareAlike
% 3.0 Germany License (http://creativecommons.org/licenses/by-sa/3.0/de/): Feel
% free 'to share (to copy, distribute and transmit)' or 'to remix (to adapt)'
% it, if you '... distribute the resulting work under the same or similar
% license to this one' and if you respect how 'you must attribute the work in
% the manner specified by the author ...':
%
% In an internet based reuse please link the reused parts to www.telekom.com and
% mention the original authors and Deutsche Telekom AG in a suitable manner. In
% a paper-like reuse please insert a short hint to www.telekom.com and to the
% original authors and Deutsche Telekom AG into your preface. For normal
% quotations please use the scientific standard to cite.
%
% [ Derived from 'mykeds Scholar Research Framework' 
%   mykeds-CSR-framework (c) K. Reincke CC BY 3.0  http://www.mykeds.net/ ]

%\abbr[]{[n.abbr.]}{ }
\abbr[zge]{ZGE / IPJ}{Zeitschrift für geistiges Eigentum [ISSN: 1867-237x]}
\abbr[itrb]{ITRB}{Der IT-Rechtsberater [ISSN: 1617-1527]}
\abbr[cri]{CRi}{Computer Law Review international [ISSN: 1610-7608]}
\abbr[btlj]{[n.abbr.]}{Berkeley Technology Law Journal}
\abbr[eclr]{E.C.L.R.}{European Competition Law Review}
\abbr[iesw]{[n.abbr.]}{IEEE Software [ISSN: 0740-7459]}
\abbr[cuitj]{[n.abbr.]}{Cutter IT Journal [ISSN: 1048-5600]}
\abbr[uoclr]{[n.abbr.]}{University of Chicago Law Review}
\abbr[uoilr]{[n.abbr.]}{University of Illinois Law Review}
\abbr[uoplr]{[n.abbr.]}{University of Pittsburgh Law Review}
\abbr[ddt]{DDT}{Drug Discovery Today [ISSN: 1359-6446]}
\abbr[rdm]{[n.abbr.]}{R\&D Management [ISSN: 1467-9310]}
\abbr[jleo]{JLEO}{Journal of Law, Economics, \& Organization [ISSN: 1465-7341]}
\abbr[ijomi]{[n.abbr.]}{International Journal of Medical Informatics [ISSN: 1386-5056]}
\abbr[slr]{[n.abbr.]}{Stanford Law Review [ISSN: 00389765]}
\abbr[bise]{BISE}{Business \& Information Systems Engineering [ISSN: 1867-0202]}
\abbr[joals]{[n.abbr.]}{Journal of Academic Librarianship [ISSN: 0099-1333]}
\abbr[eait]{[n.abbr.]}{Ethics and Information Technology [ISSN: 1388-1957]}
\abbr[jais]{JAIS}{Journal of the Association for Information Systems [ISSN:
1536-9323]}
\abbr[josas]{[n.abbr.]}{Journal of Systems and Software [ISSN: 0164-1212]}
\abbr[iialr]{[n.abbr.]}{International Information and Library Review [ISSN: 1057-2317]}
\abbr[sthv]{STHV}{Science, Technology \& Human Values [ISSN: 0162-2439]}
\abbr[cue]{[n.abbr.]}{Computers \& Education [ISSN: 0360-1315]}
\abbr[eer]{EER}{European Economic Review [ISSN: 0014-2921]}
\abbr[icc]{ICC}{Industrial and Corporate Change [ISSN: 0960-6491]}
\abbr[ca]{[n.abbr.]}{Cultural Anthropology [ISSN: 1548-1360]}
\abbr[sqj]{[n.abbr.]}{Software Qualilty Journal [ISSN: 0963-9314]}
\abbr[jmir]{JMIR}{Journal of Medical Information Research [ISSN: 1438-8871]}
\abbr[joce]{[n.abbr.]}{Journal of Comparative Economics [ISSN: 0147-5967]}
\abbr[orgsci]{[n.abbr.]}{Organization Science [ISSN: 1047-7039]}
\abbr[iam]{[n.abbr.]}{Information \& Management [ISSN: 0378-7206]}
\abbr[rp]{RP}{Research Policy [ISSN: 0048-7333]}
\abbr[jsis]{JSIS}{Journal of Strategic Information Systems [ISSN: 0963-8687]}
\abbr[isj]{ISJ}{Information Systems Journal [ISSN: 1365-2575]}
\abbr[jise]{JISE}{Journal of Information Science and Engineering [ISSN:
1016-2364]}
\abbr[dss]{DSS}{Decision Support Systems [ISSN: 0167-9236]}
\abbr[cihp]{CiHB}{Computers in Human Behavior [ISSN: 0747-5632]}
\abbr[iep]{IEaP}{Information Economics and Policy [ISSN: 0167-6245]}
\abbr[tosem]{ToSEM}{Transactions on Software Engineering Methodology [ISSN:
1049-331X]}
\abbr[commacm]{CotACM}{Communications of the ACM [ISSN: 0001-0782]}
\abbr[interactions]{[n.abbr.]}{interactions[ISSN: 1072-5520]}
\abbr[jcsc]{JCSC}{Journal of Computing Sciences in [Small] Colleges [ISSN:
1937-4771]}
\abbr[linuxjournal]{LJ}{Linux Journal [ISSN: 1075-3583]}
\abbr[networker]{[n.abbr.]}{netWorker [ISSN: 1091-3556]}
\abbr[queue]{[n.abbr.]}{Queue [ISSN: 1542-7730]}
\abbr[sigmisdb]{SIGMIS Database}{ACM SIGMIS - The Data Base for Advances in
Information Systems [ISSN: 0095-0033]}
\abbr[sigcas]{SIGCAS}{ACM SIGCAS Computers and Society [ISSN: 0095-2737]}
\abbr[sigsoft]{SIGSOFT SEN}{SIGSOFT Software Engineering Notes [ISSN:
0163-5948]}
\abbr[toit]{ToIT}{Transaction on Internet Technology [ISSN: 1533-5399]}
\abbr[sigbul]{SIGCSE Bulletin}{SIGCSE Bulletin [ISSN: 0097-8418]}
\abbr[ubiquity]{Ubiquity}{Ubiquity - The ACM IT Magazine and Forum [ISSN:
1530-2180]}
\abbr[bwv]{BWV}{Berliner Wissenschafts-Verlag GmbH}
\abbr[cr]{CR}{Computer und Recht. Zeitschrift für die Praxis des Rechts der
Informationstechnologien}


\printnomenclature

\bibliography{bibfiles/oscResourcesEn}


\end{document}
