% Telekom osCompendium cloak file English text
%
% (c) Karsten Reincke, Deutsche Telekom AG, Darmstadt 2011
%
% This LaTeX-File is licensed under the Creative Commons Attribution-ShareAlike
% 3.0 Germany License (http://creativecommons.org/licenses/by-sa/3.0/de/): Feel
% free 'to share (to copy, distribute and transmit)' or 'to remix (to adapt)'
% it, if you '... distribute the resulting work under the same or similar
% license to this one' and if you respect how 'you must attribute the work in
% the manner specified by the author ...':
%
% In an internet based reuse please link the reused parts to www.telekom.com and
% mention the original authors and Deutsche Telekom AG in a suitable manner. In
% a paper-like reuse please insert a short hint to www.telekom.com and to the
% original authors and Deutsche Telekom AG into your preface. For normal
% quotations please use the scientific standard to cite.
%
% [ File structure derived from 'mind your Scholar Research Framework' 
%   mycsrf (c) K. Reincke CC BY 3.0  http://mycsrf.fodina.de/ ]

\documentclass[DIV=calc,BCOR=5mm,12pt,headings=small,oneside,abstract=true,
toc=bib]{scrbook}

%%% (1) general configurations %%%
\usepackage[utf8]{inputenc}

%%% (2) language specific configurations %%%
\usepackage[]{a4,ngerman}
\usepackage[ngerman, english]{babel}
\selectlanguage{english}

% jurabib configuration
\usepackage[see]{jurabib}
\bibliographystyle{jurabib}
% Telekom osCompendium English Jurabib Configuration Include Module 
%
% (c) Karsten Reincke, Deutsche Telekom AG, Darmstadt 2011
%
% This LaTeX-File is licensed under the Creative Commons Attribution-ShareAlike
% 3.0 Germany License (http://creativecommons.org/licenses/by-sa/3.0/de/): Feel
% free 'to share (to copy, distribute and transmit)' or 'to remix (to adapt)'
% it, if you '... distribute the resulting work under the same or similar
% license to this one' and if you respect how 'you must attribute the work in
% the manner specified by the author ...':
%
% In an internet based reuse please link the reused parts to www.telekom.com and
% mention the original authors and Deutsche Telekom AG in a suitable manner. In
% a paper-like reuse please insert a short hint to www.telekom.com and to the
% original authors and Deutsche Telekom AG into your preface. For normal
% quotations please use the scientific standard to cite.
%
% [ File structure derived from 'mind your Scholar Research Framework' 
%   mycsrf (c) K. Reincke CC BY 3.0  http://mycsrf.fodina.de/ ]

% the first time cite with all data, later with shorttitle
\jurabibsetup{citefull=first}

%%% (1) author / editor list configuration
%\jurabibsetup{authorformat=and} % uses 'und' instead of 'u.'
% therefore define your own abbreviated conjunction: 
% an 'and before last author explicetly written conjunction

% for authors in citations
\renewcommand*{\jbbtasep}{ a.\ } % bta = between two authors sep
\renewcommand*{\jbbfsasep}{, } % bfsa = between first and second author sep
\renewcommand*{\jbbstasep}{, a.\ }% bsta = between second and third author sep
% for editors in citations
\renewcommand*{\jbbtesep}{ a.\ } % bta = between two authors sep
\renewcommand*{\jbbfsesep}{, } % bfsa = between first and second author sep
\renewcommand*{\jbbstesep}{, a.\ }% bsta = between second and third author sep

% for authors in literature list
\renewcommand*{\bibbtasep}{ a.\ } % bta = between two authors sep
\renewcommand*{\bibbfsasep}{, } % bfsa = between first and second author sep
\renewcommand*{\bibbstasep}{, a.\ }% bsta = between second and third author sep
% for editors  in literature list
\renewcommand*{\bibbtesep}{ a.\ } % bte = between two editors sep
\renewcommand*{\bibbfsesep}{, } % bfse = between first and second editor sep
\renewcommand*{\bibbstesep}{, a.\ }% bste = between second and third editor sep

% use: name, forname, forname lastname u. forname lastname
\jurabibsetup{authorformat=firstnotreversed}
\jurabibsetup{authorformat=italic}

%%% (2) title configuration
% in every case print the title, let it be seperated from the 
% author by a colon and use the slanted font
\jurabibsetup{titleformat={all,colonsep}}
%\renewcommand*{\jbtitlefont}{\textit}

%%% (3) seperators in bib data
% separate bibliographical hints and page hints by a comma
\jurabibsetup{commabeforerest}

%%% (4) specific configuration of bibdata in quotes / footnote
% use a.a.O if possible
\jurabibsetup{ibidem=strict}
% replace ugly a.a.O. by translation of ders., a.a.O.
\AddTo\bibsgerman{
  \renewcommand*{\ibidemname}{Id., l.c.}
  \renewcommand*{\ibidemmidname}{id., l.c.}
}
\renewcommand*{\samepageibidemname}{Id., ibid.}
\renewcommand*{\samepageibidemmidname}{id., ibid.}


%%% (5) specific configuration of bibdata in bibliography
% ever an in: before journal and collection/book-tiltes 
\renewcommand*{\bibbtsep}{in: }
\renewcommand*{\bibjtsep}{in: }
% ever a colon after author names 
\renewcommand*{\bibansep}{: }
% ever a semi colon after the title
% \AddTo\bibsgerman{\renewcommand*{\urldatecomment}{Referenzdownload: }}
\renewcommand*{\bibatsep}{; }
% ever a comma before date/year
\renewcommand*{\bibbdsep}{, }

% let jurabib insert the S. and p. information
% no S. necessary in bib-files and in cites/footcites
\jurabibsetup{pages=format}

% use a compressed literature-list using a small line indent
\jurabibsetup{bibformat=compress}
\setlength{\jbbibhang}{1em}

% which follows the design of the cites and offers comments
\jurabibsetup{biblikecite}

% print annotations into bibliography
\jurabibsetup{annote}
\renewcommand*{\jbannoteformat}[1]{{ \itshape #1 }}

%refine the prefix of url download
\AddTo\bibsgerman{\renewcommand*{\urldatecomment}{reference download: }}

% we want to have the year of articles in brackets
\renewcommand*{\bibaldelim}{(}
\renewcommand*{\bibardelim}{)}

% in english version Nr. must be replaced by No.
\renewcommand*{\artnumberformat}[1]{\unskip,\space No.~#1}
\renewcommand*{\pernumberformat}[1]{\unskip\space No.~#1}%
\renewcommand*{\revnumberformat}[1]{\unskip\space No.~#1}%


%Reformatierung Seriestitels and Seriesnumber
\DeclareRobustCommand{\numberandseries}[2]{%
\unskip\unskip%,
\space\bibsnfont{(=~#2}%
\ifthenelse{\equal{#1}{}}{)}{, [Vol./No.]~#1)}%
}%


% Local Variables:
% mode: latex
% fill-column: 80
% End:


% language specific hyphenation
% Telekom osCompendium osHyphenation Include Module
%
% (c) Karsten Reincke, Deutsche Telekom AG, Darmstadt 2011
%
% This LaTeX-File is licensed under the Creative Commons Attribution-ShareAlike
% 3.0 Germany License (http://creativecommons.org/licenses/by-sa/3.0/de/): Feel
% free 'to share (to copy, distribute and transmit)' or 'to remix (to adapt)'
% it, if you '... distribute the resulting work under the same or similar
% license to this one' and if you respect how 'you must attribute the work in
% the manner specified by the author ...':
%
% In an internet based reuse please link the reused parts to www.telekom.com and
% mention the original authors and Deutsche Telekom AG in a suitable manner. In
% a paper-like reuse please insert a short hint to www.telekom.com and to the
% original authors and Deutsche Telekom AG into your preface. For normal
% quotations please use the scientific standard to cite.
%
% [ File structure derived from 'mind your Scholar Research Framework' 
%   mycsrf (c) K. Reincke CC BY 3.0  http://mycsrf.fodina.de/ ]
%


\hyphenation{rein-cke}
\hyphenation{OS-LiC}
\hyphenation{ori-gi-nal}


%%% (3) layout page configuration %%%

% select the visible parts of a page
% S.31: { plain|empty|headings|myheadings }
%\pagestyle{myheadings}
\pagestyle{headings}

% select the wished style of page-numbering
% S.32: { arabic,roman,Roman,alph,Alph }
\pagenumbering{arabic}
\setcounter{page}{1}

% select the wished distances using the general setlength order:
% S.34 { baselineskip| parskip | parindent }
% - general no indent for paragraphs
\setlength{\parindent}{0pt}
\setlength{\parskip}{1.2ex plus 0.2ex minus 0.2ex}


%%% (4) general package activation %%%
%\usepackage{utopia}
%\usepackage{courier}
%\usepackage{avant}
\usepackage[dvips]{epsfig}

% graphic
\usepackage{graphicx,color}
\usepackage{array}
\usepackage{shadow}
\usepackage{fancybox}

%- start(footnote-configuration)
%  flush the cite numbers out of the vertical line and let
%  the footnote text directly start in the left vertical line
\usepackage[marginal]{footmisc}
%- end(footnote-configuration)


% %- start(endnote-configuration) uncomment to activate
% % Let all notes being marked with \endnote instead of \footnote
% % become endnotes. This set of endnotes replaces the next 
% % arising command \theendnotes - even if it is not located
% % at the end of the text.
% 
% \usepackage{endnotes}
% 
% % Format endnotes as Block with indention - Solution 1
% %\renewcommand\enoteformat{%
% %   \noindent\theenmark.) \ \hangindent .7\parindent%
% %}
% 
% % Format endnotes as Block with indention - Solution 2
% \makeatletter
% \def\enoteformat{\rightskip\z@ \leftskip0em \parindent=0em \parskip=0em
% \leavevmode\llap{\hbox{\@theenmark.~}}}
% \makeatother
% 
% \renewcommand\notesname{Annotations}
% % additionally we shall active a special jurabib option
% % if we want to get all jurabib footnotes as endnotes
% \jurabibsetup{citetoend=true}
% %- end(footnote-configuration)

% - additional packages
\usepackage{amssymb}
\usepackage{wasysym}
\usepackage{pstricks, pst-node, pst-tree}
\usepackage{chngcntr}
\counterwithout{footnote}{chapter}

\usepackage[intoc]{nomencl}
\let\abbr\nomenclature
% Modify Section Title of nomenclature
\renewcommand{\nomname}{Periodicals, Shortcuts, and Overlapping Abbreviations}
%\renewcommand{\nomname}{Periodika, ihre Kurzformen und generelle Abkürzungen}

% insert point between abbrewviation and explanation
\setlength{\nomlabelwidth}{.24\hsize}
\renewcommand{\nomlabel}[1]{#1 \dotfill}
% reduce the line distance
\setlength{\nomitemsep}{-\parsep}
\makenomenclature

% depth of contents
\setcounter{secnumdepth}{5}
\setcounter{tocdepth}{5}
%%%%%%%%%%%%%%
\begin{document}

%% use all entries of the bliography
\nocite{*}

%%-- start(titlepage)
\titlehead{Release 0.5.0
}
\subject{\small \itshape A Practical Guide for Developers, Managers, Companies,
and OS Experts}
\title{Open Source License Compendium}
\subtitle{How to to Use Open Source Software in a Regular Manner}
\author{K. Reincke, J. Dobson, G. Sharpe, M. Kern% Telekom osCompendium License Include Module
%
% (c) Karsten Reincke, Deutsche Telekom AG, Darmstadt 2011
%
% This LaTeX-File is licensed under the Creative Commons Attribution-ShareAlike
% 3.0 Germany License (http://creativecommons.org/licenses/by-sa/3.0/de/): Feel
% free 'to share (to copy, distribute and transmit)' or 'to remix (to adapt)'
% it, if you '... distribute the resulting work under the same or similar
% license to this one' and if you respect how 'you must attribute the work in
% the manner specified by the author ...':
%
% In an internet based reuse please link the reused parts to www.telekom.com and
% mention the original authors and Deutsche Telekom AG in a suitable manner. In
% a paper-like reuse please insert a short hint to www.telekom.com and to the
% original authors and Deutsche Telekom AG into your preface. For normal
% quotations please use the scientific standard to cite.
%
% [ File structure derived from 'mind your Scholar Research Framework' 
%   mycsrf (c) K. Reincke CC BY 3.0  http://mycsrf.fodina.de/ ]
%
\footnote{
This text is licensed under the Creative Commons Attribution-ShareAlike 3.0 Germany
License (http://creativecommons.org/licenses/by-sa/3.0/de/): Feel free \enquote{to
share (to copy, distribute and transmit)} or \enquote{to remix (to
adapt)} it, if you \enquote{[\ldots] distribute the resulting work under the
same or similar license to this one} and if you respect how \enquote{you
must attribute the work in the manner specified by the author(s)
[\ldots]}):
\newline
In an internet based reuse please mention the initial authors in a suitable
manner, name their sponsor \textit{Deutsche Telekom AG} and link it to
\texttt{http://www.telekom.com}. In a paper-like reuse please insert a short
hint to \texttt{http://www.telekom.com}, to the initial authors, and to their
sponsor \textit{Deutsche Telekom AG} into your preface. For normal quotations
please use the scientific standard to cite.
\newline
{ \tiny \itshape [derived from myCsrf (= 'mind your Scholar Research Framework') 
\copyright K. Reincke CC BY 3.0  http://mycsrf.fodina.de/)] }}}
\maketitle
%%-- end(titlepage)

\footnotesize
\begin{flushright} 

\parbox{100mm}{\itshape
The Open Source Community is a swarm: it is stronger than a set of some
accidentally selected experts. We are proud to get its' help. Gladly we thank
(in alphabetical order):
}

\parbox{50mm}{
\tiny
\begin{flushright}
Eitan Adler,\\
Thomas Quiehl,\\
Peter Schichl,\\
Helene Tamer,\\
and all the \\
other\ldots
\end{flushright}
}
\end{flushright}
\normalsize
\newpage

\footnotesize
\tableofcontents
\newpage
% Telekom osCompendium 'for being included' snippet template
%
% (c) Karsten Reincke, Deutsche Telekom AG, Darmstadt 2011
%
% This LaTeX-File is licensed under the Creative Commons Attribution-ShareAlike
% 3.0 Germany License (http://creativecommons.org/licenses/by-sa/3.0/de/): Feel
% free 'to share (to copy, distribute and transmit)' or 'to remix (to adapt)'
% it, if you '... distribute the resulting work under the same or similar
% license to this one' and if you respect how 'you must attribute the work in
% the manner specified by the author ...':
%
% In an internet based reuse please link the reused parts to www.telekom.com and
% mention the original authors and Deutsche Telekom AG in a suitable manner. In
% a paper-like reuse please insert a short hint to www.telekom.com and to the
% original authors and Deutsche Telekom AG into your preface. For normal
% quotations please use the scientific standard to cite.
%
% [ File structure derived from 'mind your Scholar Research Framework' 
%   mycsrf (c) K. Reincke CC BY 3.0  http://mycsrf.fodina.de/ ]

%


%% use all entries of the bibliography


\begin{table}
\footnotesize
\caption{History of the Open Source License Compendium}
\begin{center}
\begin{tabular}{|r|c|p{10cm}|}
\hline
\hline
    \texttt{2013-03-15}
  & \texttt{0.92.1} 
  & CeBIT release\newline
    $\vartriangleright$ to-do lists for the many important licenses added\newline
    $\vartriangleright$ branches merged and new master published\\
\hline
    \texttt{2013-03-08}
  & \texttt{0.90.1} 
  & CeBIT release\newline
    $\vartriangleright$ to-do lists for the some important licenses added\newline
    $\vartriangleright$ branches merged and new master published\\
\hline
    \texttt{2013-02-16}
  & \texttt{0.8.90} 
  & CeBIT pre release\newline
    $\vartriangleright$ new arguing structure focused on the topic license fulfillment\newline
    $\vartriangleright$ new classifying license review\newline   
    $\vartriangleright$ new top down introduction\\
\hline
    \texttt{2012-12-28}
  & \texttt{0.8.0} 
  & inmternal EOY release\newline
    $\vartriangleright$ many distributed improvents unified in branch kreinck\\
\hline
    \texttt{2012-08-25}
  & \texttt{0.5.2} 
  & expanded break through release\newline
    $\vartriangleright$ MIT license fulfilling to-do lists\newline
    $\vartriangleright$ using integrated Eclipse spell checking methods\\
\hline
    \texttt{2012-07-06}
  & \texttt{0.4.0} 
  & break through release\newline
    $\vartriangleright$ open source use case definition and taxonomy\newline 
    $\vartriangleright$ open source use case based general finder\newline 
    $\vartriangleright$ corresponding BSD specific mini finder\newline 
    $\vartriangleright$ BSD license fulfilling to-do lists\\
\hline
    \texttt{2012-03-22}
  & \texttt{0.2.1} 
  & $\vartriangleright$ framework published as first community edition\\
\hline
    \texttt{2012-01-31}
  & \texttt{0.1.8} 
  & $\vartriangleright$ renamed existing introduction as prolegomena\newline
    $\vartriangleright$ inserted a shorter top-down written introduction\newline
    $\vartriangleright$ inserted an OSLiC disclaimer\\
\hline
    \texttt{2012-01-21}
  & \texttt{0.1.7} 
  & $\vartriangleright$ oscCopiedButNotRead.bib expanded\newline 
  $\vartriangleright$ list of periodicals and shortcuts added\\
\hline
    \texttt{2011-12-29}
  & \texttt{0.1.6} 
  & $\vartriangleright$ many bibliographic data added\\
\hline
    \texttt{2011-10-17}
  & \texttt{0.1.5} 
  & $\vartriangleright$ bibliographic data updated\\
\hline
    \texttt{2011-09-29}
  & \texttt{0.1.4} 
  & $\vartriangleright$ document history integrated\newline
    $\vartriangleright$ typos erased\\
\hline
    \texttt{2011-09-28}
  & \texttt{0.1.3} 
  & $\vartriangleright$ review of english teacher integrated \\
\hline
    \texttt{2011-09-19}
  & \texttt{0.1.2} 
  & $\vartriangleright$ first comments of english teacher integrated \\
\hline
    \texttt{2011-09-15}
  & \texttt{0.1.1} 
  & $\vartriangleright$ improvements of John integrated\\
\hline
    \texttt{2011-09-12}
  & \texttt{0.1.0} 
  & $\vartriangleright$ introduction completed: purpose and methods \\
\hline
\hline 
\end{tabular}
\end{center}
\end{table}

\begin{footnotesize}
\begin{itemize}
  \item Insert taks lists for AGPL, CDDL, MS-RL \ldots
  \item Improve / complete the expositions of the concept of a derivative work
  in the conxtext of software development
  \item Discuss the license compatibility
  \item Explain the relationship between open source and earning money
  \item Complete the discussion of the dynamically and statically linked open
  source software by a summary of the respective secondary literature
  \item Improve / expand the integration of the used secondary literature
  \item 
  \item 
\end{itemize}
\end{footnotesize}

%\bibliography{../bibfiles/oscResourcesEn}

\normalsize

\chapter*{Disclaimer}
% Telekom osCompendium 'for beeing included' snippet template
%
% (c) Karsten Reincke, Deutsche Telekom AG, Darmstadt 2011
%
% This LaTeX-File is licensed under the Creative Commons Attribution-ShareAlike
% 3.0 Germany License (http://creativecommons.org/licenses/by-sa/3.0/de/): Feel
% free 'to share (to copy, distribute and transmit)' or 'to remix (to adapt)'
% it, if you '... distribute the resulting work under the same or similar
% license to this one' and if you respect how 'you must attribute the work in
% the manner specified by the author ...':
%
% In an internet based reuse please link the reused parts to www.telekom.com and
% mention the original authors and Deutsche Telekom AG in a suitable manner. In
% a paper-like reuse please insert a short hint to www.telekom.com and to the
% original authors and Deutsche Telekom AG into your preface. For normal
% quotations please use the scientific standard to cite.
%
% [ File structure derived from 'mind your Scholar Research Framework' 
%   mycsrf (c) K. Reincke CC BY 3.0  http://mycsrf.fodina.de/ ]

%

%%% \chapter*{Disclaimer} %%%

This book shall be thoroughly developed -- together with the open source
community. Finally, it shall deliver reliable information with respect to the
rule that the swarm knows more than the single fish.

But nevertheless it can not offer more than the opinion(s) of its authors and
contributors. It is only one voice of chorus discussing the topic of open source
licenses. For protecting the authors and contributors from charges and claims of
idemnification we adopt the lightly modified GPL3 disclaimer:

THERE IS NO WARRANTY FOR THE OSLiC, TO THE EXTENT PERMITTED BY APPLICABLE LAW.
THE COPYRIGHT HOLDERS AND/OR OTHER PARTIES PROVIDE THE TEXT “AS IS” WITHOUT
WARRANTY OF ANY KIND, EITHER EXPRESSED OR IMPLIED, INCLUDING, BUT NOT LIMITED
TO, THE IMPLIED WARRANTIES OF MERCHANTABILITY AND FITNESS FOR A PARTICULAR
PURPOSE. THE ENTIRE RISK AS TO THE QUALITY AND PERFORMANCE OF THE OSLiC IS
WITH YOU. SHOULD THE OSLiC PROVE DEFECTIVE, YOU ASSUME THE COST OF ALL
NECESSARY SERVICING, REPAIR OR CORRECTION.

IN NO EVENT UNLESS REQUIRED BY APPLICABLE LAW OR AGREED TO IN WRITING WILL ANY
COPYRIGHT HOLDER, OR ANY OTHER PARTY WHO MODIFIES AND/OR CONVEYS THE OSLiC AS
PERMITTED ABOVE, BE LIABLE TO YOU FOR DAMAGES, INCLUDING ANY GENERAL, SPECIAL,
INCIDENTAL OR CONSEQUENTIAL DAMAGES ARISING OUT OF THE USE OR INABILITY TO USE
THE PROGRAM (INCLUDING BUT NOT LIMITED TO LOSS OF DATA OR DATA BEING RENDERED
INACCURATE OR LOSSES SUSTAINED BY YOU OR THIRD PARTIES OR A FAILURE OF THE
PROGRAM TO OPERATE WITH ANY OTHER PROGRAMS), EVEN IF SUCH HOLDER OR OTHER PARTY
HAS BEEN ADVISED OF THE POSSIBILITY OF SUCH DAMAGES.

%%%%%%%%%%%%
\chapter{Introduction}
% Telekom osCompendium 'for beeing included' snippet template
%
% (c) Karsten Reincke, Deutsche Telekom AG, Darmstadt 2011
%
% This LaTeX-File is licensed under the Creative Commons Attribution-ShareAlike
% 3.0 Germany License (http://creativecommons.org/licenses/by-sa/3.0/de/): Feel
% free 'to share (to copy, distribute and transmit)' or 'to remix (to adapt)'
% it, if you '... distribute the resulting work under the same or similar
% license to this one' and if you respect how 'you must attribute the work in
% the manner specified by the author ...':
%
% In an internet based reuse please link the reused parts to www.telekom.com and
% mention the original authors and Deutsche Telekom AG in a suitable manner. In
% a paper-like reuse please insert a short hint to www.telekom.com and to the
% original authors and Deutsche Telekom AG into your preface. For normal
% quotations please use the scientific standard to cite.
%
% [ Framework derived from 'mind your Scholar Research Framework' 
%   mycsrf (c) K. Reincke 2012 CC BY 3.0  http://mycsrf.fodina.de/ ]
%

% Chapter 1 Abstract
% ------------------

\footnotesize
\begin{quote}\itshape
This chapter shortly describes what the OSLiC is, how it should be used, and how
it can be read. It shall be written as top-down explanation.
\end{quote}
\normalsize{}


% Telekom osCompendium 'for beeing included' snippet template
%
% (c) Karsten Reincke, Deutsche Telekom AG, Darmstadt 2011
%
% This LaTeX-File is licensed under the Creative Commons Attribution-ShareAlike
% 3.0 Germany License (http://creativecommons.org/licenses/by-sa/3.0/de/): Feel
% free 'to share (to copy, distribute and transmit)' or 'to remix (to adapt)'
% it, if you '... distribute the resulting work under the same or similar
% license to this one' and if you respect how 'you must attribute the work in
% the manner specified by the author ...':
%
% In an internet based reuse please link the reused parts to www.telekom.com and
% mention the original authors and Deutsche Telekom AG in a suitable manner. In
% a paper-like reuse please insert a short hint to www.telekom.com and to the
% original authors and Deutsche Telekom AG into your preface. For normal
% quotations please use the scientific standard to cite.
%
% [ Framework derived from 'mind your Scholar Research Framework' 
%   mycsrf (c) K. Reincke 2012 CC BY 3.0  http://mycsrf.fodina.de/ ]
%


%% use all entries of the bibliography
%\nocite{*}

[TDB \ldots]

%\bibliography{../../../bibfiles/oscResourcesEn}


%%%%%%%%%%%%%%%
\chapter{Prolegomena}
% Telekom osCompendium 'for being included' snippet template
%
% (c) Karsten Reincke, Deutsche Telekom AG, Darmstadt 2011
%
% This LaTeX-File is licensed under the Creative Commons Attribution-ShareAlike
% 3.0 Germany License (http://creativecommons.org/licenses/by-sa/3.0/de/): Feel
% free 'to share (to copy, distribute and transmit)' or 'to remix (to adapt)'
% it, if you '... distribute the resulting work under the same or similar
% license to this one' and if you respect how 'you must attribute the work in
% the manner specified by the author ...':
%
% In an internet based reuse please link the reused parts to www.telekom.com and
% mention the original authors and Deutsche Telekom AG in a suitable manner. In
% a paper-like reuse please insert a short hint to www.telekom.com and to the
% original authors and Deutsche Telekom AG into your preface. For normal
% quotations please use the scientific standard to cite.
%
% [ File structure derived from 'mind your Scholar Research Framework' 
%   mycsrf (c) K. Reincke CC BY 3.0  http://mycsrf.fodina.de/ ]
%

% Chapter Abstract
% ----------------

\footnotesize
\begin{quote}\itshape
This chapter describes why we need an OSLiC and how its content and form has
been derivated from the need. It's written as bottom-up explanation and shall
deliver deeper insights.
\end{quote}
\normalsize{}


% Telekom osCompendium 'for beeing included' snippet template
%
% (c) Karsten Reincke, Deutsche Telekom AG, Darmstadt 2011
%
% This LaTeX-File is licensed under the Creative Commons Attribution-ShareAlike
% 3.0 Germany License (http://creativecommons.org/licenses/by-sa/3.0/de/): Feel
% free 'to share (to copy, distribute and transmit)' or 'to remix (to adapt)'
% it, if you '... distribute the resulting work under the same or similar
% license to this one' and if you respect how 'you must attribute the work in
% the manner specified by the author ...':
%
% In an internet based reuse please link the reused parts to www.telekom.com and
% mention the original authors and Deutsche Telekom AG in a suitable manner. In
% a paper-like reuse please insert a short hint to www.telekom.com and to the
% original authors and Deutsche Telekom AG into your preface. For normal
% quotations please use the scientific standard to cite.
%
% [ Framework derived from 'mind your Scholar Research Framework' 
%   mycsrf (c) K. Reincke 2012 CC BY 3.0  http://mycsrf.fodina.de/ ]
%


%% use all entries of the bibliography
%\nocite{*}

\section{Why}

Do we need another book about Open Source? Do \emph{you} need another book about
Open Source Software? Let us address this question from the viewpoint of what we
already know, what we instinctively believe and what we may have heard. For
example you may presume one or more of the following statements is correct. Or
you may even have experienced similar perceptions from your peers or managers.
Or you have been told they describe 'Open Source':

\begin{itemize}
  \item The Open Source Definition offers rules to use Open Source Software.
  \item Modified Open Source Software must be published.
  \item Modified Open Source Software must be given back to the community.
  \item All generations of Open Source Software will remain open for ever.
  \item Software can either be Open Source Software or proprietary software.
  \item The opposite of Open Source Software is commercial software.
  \item Open Source Software prohibits to earn money.
  \item Modifications of Open Source Software must be marked explicitly.
  \item Modifiers of Open Source Software must identify themselves.
  \item When distributing an Open Source binary it’s enough point to a download
  page to obtain the source code.
  \item The aim of Open Source Software is to improve the world ethically.
  \item Open Source Software is viral and infectious.
\end{itemize}

Do these conceptions sound familiar to you? Unfortunately, whatever we might
believe or wish for, these concepts are incorrect. Naturally we will discuss
this issue later on. For the moment let us assume they are indeed
incorrect\endnote{For those who want directly verify our argumentation, we have
generated a condensed summary of the arguments and citations. You can find this
summary in our appendices.}.

So, again: Do \emph{we} need another book about Open Source Software? \emph{We},
that is - in this case and at least initially - the large German company
\textit{Deutsche Telekom AG}. Arguing from the perspective of a large company
requires not only identifying the common misconceptions, but catering for the
unique needs of a large Enterprise. And indeed the very size of the company
brings its own problems.

Large companies use more Open Source Software in more varied contexts than small
companies. There is an important question that every company should ask:
\emph{'Are we sure that we respect all those requirements of Open Source
Software we have to respect?'}. But large companies can not answer this question
as easily as small companies: the large number of diverse Open Source
deployments in different contexts mean that case by case governance, a model
that may work in small concerns, is far from appropriate for our needs. This
leads to wasting both time and money. Further, the chances of success are small:
training at least one employee in each software team as an Open Source Software
License expert is unrealistic in terms of cost-efficiency and reliability.

Nevertheless even large companies want to and try to fulfill the rules of Open
Source Software thoroughly - especially \emph{Deutsche Telekom AG}. When this
company realized that the question \textit{Are we sure that we respect all those
rules of Open Source Software correctly which we have to respect} could be
problematic, it directly asked some of its' employees who were known as Open
Source enthusiasts - to establish a service and a process for answering this
question.

So, it is no surprising that we, the initial authors of this \textit{Open Source
License Compendium}, were asked by our employer \emph{Deutsche Telekom AG}.
Naturally we were proud to work on an Open Source topic officially. But while we
were doing our job we had to ask ourselves if \emph{we} perhaps needed another
book on Open Source. Our answer was \textit{Yes, we do!} Let us shortly explain,
why:

Firstly, we already knew that there exist supporting software. These
meta-pro\-grams take the code of any other application and try to list those
Open Source Software being 'covered' by that application\footnote{As general
examples let us mention Palamida (\texttt{http://www.palamida.com/}) and
BlackDuck (\texttt{http://www.blackducksoftware.com/}).}. But we had also
already realised that this supporting software did not always match the way we
thought the problem should be solved. Secondly we recognized fairly quickly that
we need a reliable guide. We personally were asked to give the \emph{ok} for
projects of our company. We could not answer such requests on the base of
\textit{'Oh yes, I read this in the \emph{Heise-Ticker} a few days ago'} - even
if the \emph{Heise-Ticker} had described the situation completely correctly. We
ourselves had to be reliable than this. Naturally we already knew a great deal
about Open Source Software. Even so, our knowledge was not as systematic as
necessary. We looked for an Open Source Compendium which adequately described
what a project or product development team had to do to fulfill the criteria of
its Open Source Licenses. We wanted to use that compendium to the basis of our
recommendations.

We were very thorough but we did not find what we were looking for. Our 'little'
bibliography attest our seriousness. What we found was a lot of information
releating to individual issues spread over many sources. We did not find answers
for our question even in the specific literature. Let us describe three little
steps to increase the understanding of the issue:


%\bibliography{../../../bibfiles/oscResourcesEn}

%

% Telekom osCompendium 'for being included' snippet
%
% (c) Karsten Reincke, Deutsche Telekom AG, Darmstadt 2011
%
% This LaTeX-File is licensed under the Creative Commons Attribution-ShareAlike
% 3.0 Germany License (http://creativecommons.org/licenses/by-sa/3.0/de/): Feel
% free 'to share (to copy, distribute and transmit)' or 'to remix (to adapt)'
% it, if you '... distribute the resulting work under the same or similar
% license to this one' and if you respect how 'you must attribute the work in
% the manner specified by the author ...':
%
% In an internet based reuse please link the reused parts to www.telekom.com and
% mention the original authors and Deutsche Telekom AG in a suitable manner. In
% a paper-like reuse please insert a short hint to www.telekom.com and to the
% original authors and Deutsche Telekom AG into your preface. For normal
% quotations please use the scientific standard to cite.
%
% [ File structure derived from 'mind your Scholar Research Framework' 
%   mycsrf (c) K. Reincke CC BY 3.0  http://mycsrf.fodina.de/ ]

%
Without Open Source Licenses there is no Open Source movement. Nevertheless in
dealing with Open Source Licenses, this is sometimes neglected. Take the
\emph{Apache Web Server} as an example: No doubt, it's one of the most important
pieces of Open Source Software\footnote{To prove that the \textit{Apache} is
really a piece of Open Source Software one must execute a set of steps: Firstly
you have to note, that \emph{Apache} is something like a meta project, covered
by the \emph{Apache Software Foundation}, also known as \emph{ASF} (cf.
\texttt{http://www.apache.org/}, wp.). Therefor you can not directly jump into
the \emph{Apache License}. First of all you have to visit the project site (cf.
\texttt{http://httpd.apache.org/}, wp.) even if at the end its' license link
leads you back to the general \emph{Apache License sub site} (cf.
\texttt{http://www.apache.org/licenses/}, wp.) which announces, that \enquote{all
software produced by The Apache Software Foundation or any of its projects or
subjects is licensed according to the terms of the documents listed
below}. Only now you can use the offered link for switching to the
\emph{Apache License}, Version 2.0, if you want to check your rights and duties.
But that is difficult. There does not exist any simple list what you have to do
for fulfilling the license. Even the faq (cf.
\texttt{http://httpd.apache.org/docs/2.2/faq/}, wp.) - meanwhile being moved to
a wiki - only says that the server \enquote{[\ldots] comes with an unrestrictive
license} and that you are allowed to put the code on a CD (cf.
\texttt{http://wiki.apache.org/httpd/FAQ}, wp.). Hence, from the viewpoint of
the ASF the license itself shall answer all questions. [Reference download for
all urls: 2011-08-31] } with a specific license\footcite[cf.][\nopage
wp.]{AsfApacheLicense20a}. Moreover: the success of the Open Source movement
in the commercial world depends directly on the decision of IBM to replace its
corresponding own component in the \textit{IBM WebSphere Application Server}
with the free \textit{Apache Web Server}\footcite[cf.][287ff]{Moody2001a}.
Meanwhile many companies use the \textit{Apache Web Server} to act as a web
provider. Currently the \emph{Apache http server} - as it has to be named
correctly - is used more than twice as much as all the other http server
software together\footcite[cf.][\nopage wp]{Netcraft2011a}. Hence many business
models depends on the Apache License. Another aspect is that even the famous
\emph{Apache Cookbook}, which explains the installation, the configuration and
the maintaining of an Apache Web Server in details\footcite[cf.][\nopage et
passim]{CoaBow2004a}, does not mention anything about the license which allows
for installation, configuration and maintenance. Neither the index lists the
word 'license'\footcite[cf.][245ff, esp. p. 250]{CoaBow2004a}, nor the chapters
'Installation'\footcite[cf.][1ff]{CoaBow2004a} or the chapter
'Miscellaneous'\footcite[cf.][219ff]{CoaBow2004a} mentions the license question
in a serious way. There's only one short hint as to the advantage of Open Source
Software, i.e. that everybody is allowed to install it\footcite[cf.][1: \enquote{
\ldots einer der Vorzüge von Open Source Software besteht darin, dass
je\-der\-mann die Erlaubnis zur Erzeugung eines eigenen Installationskits hat
}]{CoaBow2004a}. Can you be sure that you are allowed to do what you are
doing on the base of such a phrase?

Naturally, the \emph{Apache Cookbook} is not a book for lawyers, it's a book for
administrators and developers, They do not want to get bogged down by
legalities, they want to set up an Apache Web Server as fast as possible and get
down to work. Indeed, the Apache Cookbook offers a good support. But not only as
a company you have to ask yourself whether you are really allowed to do what you
are doing. Can you find the answer in the \emph{Apache Cookbook}? No. Can you
find it in the license itself? Yes, but it is difficult\footnote{And do we really
want our developers and maintainers to read the original licenses? Do we really
want them to discover that they also have to check the licenses of the used
modules?}. So again: Can you find your answer in another book, which is
\emph{Amazon's} current top recommendation for the request \emph{'apache
server'}\footnote{Tested on \texttt{http://www.amazon.de/} at 2011-08-31.}? Not
really: Sascha Kersken's Apache 2.2 Handbook offers a license chapter, but only
two pages long\footcite[cf.][111f]{Kersken2009a}. Moreover the rights and duties
are condensed into just 5 bullet points which taken together do not explain when
the software and the license has to be handed over to a customer and when you
are allowed to hide your improvements\footcite[cf.][112]{Kersken2009a}.

This brings us to the question of what prevents us from using something like a
\emph{'general license cookbook'} which explains all the necessary details and which
offers  quick access to the relevant points:

%\bibliography{../bibfiles/oscResourcesEn}

% Telekom osCompendium 'for beeing included' snippet
%
% (c) Karsten Reincke, Deutsche Telekom AG, Darmstadt 2011
%
% This LaTeX-File is licensed under the Creative Commons Attribution-ShareAlike
% 3.0 Germany License (http://creativecommons.org/licenses/by-sa/3.0/de/): Feel
% free 'to share (to copy, distribute and transmit)' or 'to remix (to adapt)'
% it, if you '... distribute the resulting work under the same or similar
% license to this one' and if you respect how 'you must attribute the work in
% the manner specified by the author ...':
%
% In an internet based reuse please link the reused parts to www.telekom.com and
% mention the original authors and Deutsche Telekom AG in a suitable manner. In
% a paper-like reuse please insert a short hint to www.telekom.com and to the
% original authors and Deutsche Telekom AG into your preface. For normal
% quotations please use the scientific standard to cite.
%
% [ File structure derived from 'mind your Scholar Research Framework' 
%   mycsrf (c) K. Reincke CC BY 3.0  http://mycsrf.fodina.de/ ]

%
Of course we also browsed the internet. At least for German speaking people
there is an excellent site concerning the topic \emph{Open Source Licenses}.
offered by \textit{iffross}, which, loosely translated, means an
\textit{Institute for Legal Aspects of the Free and Open Source
Software}\footnote{originally: \enquote{Institut für Rechtsfragen der Freien und
Open Source Software}. Main entry point for its' site is the URL
\texttt{http://www.ifross.org/}.}, founded in 2000 as a private institute to
tracke the phenomenon 'free software' from the viewpoint of (German)
lawyers\footcite[cf.][\nopage wp]{ifross2011b}. Besides many other
aspects this site offers a very well and thoroughly elaborated
FAQ\footcite[cf.][\nopage wp]{ifross2011c} and a large list of Open
Source Licenses and other related licenses: moreover, evidently it is
classifying the Open Source Licenses in those 'without copyleft-effect' (BSD),
in those with 'strict copyleft-effect' (GPL)) and in those with 'restricted
copyleft-effect' (LGPL)\footcite[cf.][\nopage wp]{ifross2011a}.

However, even this excellent site does not fulfill our needs. It does not offer
those context specific to-do lists which companies, developers or project
managers can use to ensure their Open Source Software is used in a regular
manner.

We therefore evaluated that standard book which is listed in the most legal
bibliographies\footnote{at least in that German judicial literature dealing with
Open Source}: the book of Jaeger and Metzger which concerns - loosely translated
- \textit{the judicial framework requirement for Open Source
Software}\footcite[cf.][V - It can not be any surprise that both authors,
Mr. Jaeger and Mr. Metzger are members of ifross (cf.
\texttt{http://www.ifross.org/personen/}, wp.)]{JaeMet2002a}. Even the most
earliest edition of this book already had a clear structure in its' chapter
'copyright': For each license mentioned (or at least for each license cluster)
it offered a subchapter for the rights and a subchapter for the
duties\footcite[cf.][30ff]{JaeMet2002a} of the software user\footcite[For
getting a good survey of the structure and the line of thought see the contents
cf.][VIIIf]{JaeMet2002a}. Many other important aspects of the topic
\textit{Open Source} are discussed, too\footcite[pars pro toto: have a
look at the chapter concerning the liability: cf.][137ff]{JaeMet2002a}.

But we needed more than this. Despite the quality of the book we were certain
that we could not hand over this book to our programmers with the recommendation
\textit{check your touched licenses and follow the instructions of the relevant
subchapters\ldots}. This book did not contain simply checkable to-do lists,
either in the first edition\footcite[cf.][VIff]{JaeMet2002a} and in the
second edition\footcite[cf.][VIIff]{JaeMet2006a} or in the recently
published third edition\footcite[cf.][VIIIff. Naturally we use this latest
edition for adopting or discussing systematical aspects]{JaeMet2011a}. So, how
can a company or a developer or a project manager be sure of fulfilling the
requirements of the Open Source Licenses sufficiently if he/she does not have a
verified list telling him \textit{'do this' and 'in case of that, do that', and
'then do this'}? Why should he himself implicitly become an Open Source Licenses
expert which has to extract the necessary steps out of the literature?

%\bibliography{../bibfiles/oscResourcesEn}

% Telekom osCompendium 'for beeing included' snippet template
%
% (c) Karsten Reincke, Deutsche Telekom AG, Darmstadt 2011
%
% This LaTeX-File is licensed under the Creative Commons Attribution-ShareAlike
% 3.0 Germany License (http://creativecommons.org/licenses/by-sa/3.0/de/): Feel
% free 'to share (to copy, distribute and transmit)' or 'to remix (to adapt)'
% it, if you '... distribute the resulting work under the same or similar
% license to this one' and if you respect how 'you must attribute the work in
% the manner specified by the author ...':
%
% In an internet based reuse please link the reused parts to www.telekom.com and
% mention the original authors and Deutsche Telekom AG in a suitable manner. In
% a paper-like reuse please insert a short hint to www.telekom.com and to the
% original authors and Deutsche Telekom AG into your preface. For normal
% quotations please use the scientific standard to cite.
%
% [ File structure derived from 'mind your Scholar Research Framework' 
%   mycsrf (c) K. Reincke CC BY 3.0  http://mycsrf.fodina.de/ ]

%
While we were searching for an existing Open Source compendium we found an
article with the title 'Compendium for the Publication of Open Source
Software'\footnote{approximately translated}. It aims to be a 'pragmatic
guidebook' and an 'assistance' for 'publishing software under the conditions of
an Open Source License'\footcite[cf.][166f (originally: ein
\glqq{}pragmatischer Ratgeber\grqq{} zur \glqq{}Veröffentlichung einer Software
unter den Rahmenbedingungen einer Open-Source-Lizenz\grqq{}) ]{BreGlaGra2008a}.
Moreover, at the end of this article its' authors formulate ambitiously that
their 'guide' should be carried out, section by section - for getting a legally
water tight process of publishing Open Source software\footcite[cf.][186
(originally: ein \glqq{}Ratgeber\grqq{}, der es erlaubt \glqq{} (\ldots) die zu
berücksichtigende Aspekte (strukturiert abzuarbeiten) (\ldots) \glqq{} und einen
\glqq{}rechtlich nicht angreifbaren Veröffentlichungsprozess\grqq{} zu
ermöglichen) ]{BreGlaGra2008a}.

The authors of this article describe something close to what we were looking
for. Indeed, the article lists important aspects which have to be taken in
consideration if you want to deal Open Source Software correctly: It announces
that no obligation exists to publish code either if you embed GPL code into your
proprietary code or if you modify the GPL code. It is only if you hand over your
binary to other persons that you have to distribute the code too, but only to
them and not to the general public\footcite[cf.][170 and
181]{BreGlaGra2008a}. Additionally the articles explains exactly that software
- at least in Germany - can only be acknowledged as Open Source Software by
transferring the rights to use - the 'Nutzungsrechte' - to other people, while
the copyright itself - the 'Urheberpersönlichkeitsrecht' - is not transferable
and belongs to the author\footcite[cf.][173]{BreGlaGra2008a}. Moreover,
besides other aspects the articles discusses briefly and deeply the problem of
the No-Warranty-Clauses which are not valid in Germany and which will therefore
automatically be replaced by the liability rules for a
donation\footcite[cf.][177]{BreGlaGra2008a}. And last but not least this
article actually summarizes the idea of Copyleft and the differences between
LGPL and GPL\footcite[cf.][181]{BreGlaGra2008a}.

However some gaps remain. The article does not analyze in which cases a
University or a company perhaps \emph{must} publish its' developments based
upon Open Source Software. It does not discern between different licenses
and conditions. It also does not discuss what Universities or companies,
which (re-)use and/or distribute Open Source Software (internally), must do to
fulfill the touched Open Source Licenses. And finally this article
does not offer the step by step list as promised.

We did, however, feel supported by this article, in two ways. Firstly it was a
well written summary of some main problems. Secondly it stated the necessity to
have a compendium for being able to establish a legally 'water-tight' process of
publishing Open Source software\footcite[cf.][186]{BreGlaGra2008a}. We
seemed to be justified in our assumptions. But the Open Source Compendium we
were looking for had to be more practical, more processable, more distinguishing
and more elaborated.

%\bibliography{../bibfiles/oscResourcesEn}

%  
% Telekom osCompendium 'for beeing included' snippet template
%
% (c) Karsten Reincke, Deutsche Telekom AG, Darmstadt 2011
%
% This LaTeX-File is licensed under the Creative Commons Attribution-ShareAlike
% 3.0 Germany License (http://creativecommons.org/licenses/by-sa/3.0/de/): Feel
% free 'to share (to copy, distribute and transmit)' or 'to remix (to adapt)'
% it, if you '... distribute the resulting work under the same or similar
% license to this one' and if you respect how 'you must attribute the work in
% the manner specified by the author ...':
%
% In an internet based reuse please link the reused parts to www.telekom.com and
% mention the original authors and Deutsche Telekom AG in a suitable manner. In
% a paper-like reuse please insert a short hint to www.telekom.com and to the
% original authors and Deutsche Telekom AG into your preface. For normal
% quotations please use the scientific standard to cite.
%
% [ Framework derived from 'mind your Scholar Research Framework' 
%   mycsrf (c) K. Reincke 2012 CC BY 3.0  http://mycsrf.fodina.de/ ]
%


%% use all entries of the bibliography
%\nocite{*}

So again: Did we need a new book about Open Source Software? We had looked for a
reliable integrated Open Source Compendium. But we found seperate pieces of
information and - as we know today - some rumors. Our answer was clear:
naturally we did not need a new general book about Open Source. But what was
lacking was a description what responsible developers, project managers or
product developers require to fulfill Open Source Licenses. We needed an
\textit{Open Source License Compendium}.

At the best such an \textit{Open Source License Compendium} would contain a set
of simply to process \textit{'For-Fulfilling-The-Licence-To-Do-Lists'}.
Additionally it should offer an intuitively user-freindly search option for
these lists. In any case, it should share developers and project managers the
effort of having to become Open Source License experts. For the other users, it
should also clearly explain why one has to do this and not that. Hence a
reliable \textit{Open Source License Compendium} should not only list what one
has to do, but should offer both, thoroughly verified reliable details and
clearly condensed guidance.

Although we did not find such an Open Source Compendium we were familiar with
the spirit of the Open Source Community. Hence we followed one of its' most
simple rules: \emph{'what you miss you must develop ion your own'}. Some
principles should help us to achieve our targets:

\begin{description}
  \item[To-do lists as the core, discussions around them]: Our work should be
  split into two parts. As it core we wanted to offer a
  set of To-Do-Lists. Each of these lists should be relevant to one specific
  Open Source License and should be clustered by the Open Source specific use
  cases. Around this all those aspects of Open Source Software which influence the
  interpretation of the licenses and the rules core should be precisely
  characterized. Nevertheless, the users should be able to skip
  details and go directly to the section they require.
  \item[Quotations with thoroughly specified sources]: Even if our users should
  not be obliged to read every part of the compendium they should not be
  required to believe us without question. We wanted to be revisable. Because
  our sources and our conclusions should be easily verifiable, we decided to use
  the academic citations and list bibliographic data extensively on the basis
  that our task should be to collect information, not to invent new 'facts'.
  \item[Not the internet alone, also books and articles]: We wanted to go back
  to the originals even if the internet was full of more or less modified
  copies. We wished to get reliable facts and descriptions. Therefore we decided
  to evaluate not only the internet but also scientific sources - for example -
  offered by university libraries.
  \item[Not clearing out the forest land, but cutting out a swathe]: Even if we
  had to deal with licenses and their legal aspects we did not want to get lost
  in detailed discussions. It should not be our task to find out whether a
  specific kind of handling would still be legal or already forbidden.
  We did not want to fight against the licenses. We did not want to stretch
  their ambit or to test their boundary. We wished to accept Open Source
  Licenses as they are: rules written from developers for developers. And even
  if some parts of these licenses would not be valid with respect to a legal
  system\footcite[And indeed for example for the GPL one can argue in this way:
  Even if you take the GPL as a contract of the type 'donation' respectively
  \glqq{}Schenkung\grqq{}, it is presented in the form of AGBs respectively
  \glqq{}Allgemeine Geschäftsbedingungen\grqq{} and must therefore follow the
  general AGB rules.'Regrettably' in Germany these general AGB rules do not
  allow to exclude each type of warranty. If we follow Oberhem, §11 and §12 of
  the GPL must be invalid in Germany because of these general AGB rules.
  Moreover, for Oberhem even §5 - the important clause of the GPL by which you
  can only get the right to use and to distribute GPL software if you respect
  the rules of the GPL - seems also to be invalid respectively
  \glqq{}unwirksam\grqq{}. But the good message is that the GPL as whole is not
  invalid even if it contains invalid clauses.][128, 133ff, 150ff, esp. 146,
  159]{Oberhem2008a}), we wanted to take them as our guideline - at least while
  they do not violate more general laws\endnote{what they clearly do not do!}.
  We simply wanted to \emph{find one proven way} to cross the maybe slightly
  unsure forest of Open Source Licenses. Even if indeed some clauses of the
  licenses finally were not enforceable against us we wanted to respect them
  'voluntarily'. We wanted to deliver a set of rules which support users and
  remove the possibility of becoming involved in license disputes with Open
  Source developers or the Free Software Foundation.
  \item[Take the text seriously]: On the other side we wanted to take our
  license texts as they were. If they lacked anything\footcite[The systematical
  underdetermination of licenses is a problem being also known in the Open
  Source respectively Free Software movement. Following the biography of RMS his
  main judicial councelor Moglen has stated, that \glqq{}there is uncertainty in
  every legal process (\ldots) \grqq{} and that it seemed to be silly to try
  \glqq{}(\ldots) to take out all the bugs (\ldots)\grqq{}. Nevertheless - so
  Moglen resp. Williams - the goal of Richard Stallman was \glqq{}the complete
  opposite\grqq{}: He tried \glqq{}(\ldots) to remove uncertainty which is
  inherently impossible\grqq{}. But - and that's the nub of this analysis -
  Moglen had to follow Stallmann because of RMS character. And he had to
  summarize their work so, that \glqq{}(\ldots) the resulting elegance (of the
  GPL; KR.), the resulting simplicity (of the GPL; KR.) in design almost
  achieves what it has to achieve\grqq{}. Hence we are asked to take the license
  texts themselves seriously. cf.][177f]{Williams2002a}, we would interpret the
  open issues in the spirit of the Open Source idea. But where the text was
  clear and definite we wanted to take its propositions as a definite decision -
  even if that meaning stood against well known Open Source 'facts'.
  \item[Trust the swarm]: We did not want to use our own research alone as a
  basis. We knew that the swarm is ever stronger than a set of some randomly
  selected experts. Therefore we decided to publish our text as a still
  unfinished work, as a 0.5 release. And then we wanted to invite the community
  to complete the compendium together with us. We would elaborate our Open
  Source Compendium as a set of LaTeX- and BibTeX files which could be developed
  and managed in GIT or any other version control system. And finally we would
  publish our text under a Creative Commons Attribution-Share Alike German 3.0
  license% Telekom osCompendium License Include Module
%
% (c) Karsten Reincke, Deutsche Telekom AG, Darmstadt 2011
%
% This LaTeX-File is licensed under the Creative Commons Attribution-ShareAlike
% 3.0 Germany License (http://creativecommons.org/licenses/by-sa/3.0/de/): Feel
% free 'to share (to copy, distribute and transmit)' or 'to remix (to adapt)'
% it, if you '... distribute the resulting work under the same or similar
% license to this one' and if you respect how 'you must attribute the work in
% the manner specified by the author ...':
%
% In an internet based reuse please link the reused parts to www.telekom.com and
% mention the original authors and Deutsche Telekom AG in a suitable manner. In
% a paper-like reuse please insert a short hint to www.telekom.com and to the
% original authors and Deutsche Telekom AG into your preface. For normal
% quotations please use the scientific standard to cite.
%
% [ File structure derived from 'mind your Scholar Research Framework' 
%   mycsrf (c) K. Reincke CC BY 3.0  http://mycsrf.fodina.de/ ]
%
\footnote{
This text is licensed under the Creative Commons Attribution-ShareAlike 3.0 Germany
License (http://creativecommons.org/licenses/by-sa/3.0/de/): Feel free \enquote{to
share (to copy, distribute and transmit)} or \enquote{to remix (to
adapt)} it, if you \enquote{[\ldots] distribute the resulting work under the
same or similar license to this one} and if you respect how \enquote{you
must attribute the work in the manner specified by the author(s)
[\ldots]}):
\newline
In an internet based reuse please mention the initial authors in a suitable
manner, name their sponsor \textit{Deutsche Telekom AG} and link it to
\texttt{http://www.telekom.com}. In a paper-like reuse please insert a short
hint to \texttt{http://www.telekom.com}, to the initial authors, and to their
sponsor \textit{Deutsche Telekom AG} into your preface. For normal quotations
please use the scientific standard to cite.
\newline
{ \tiny \itshape [derived from myCsrf (= 'mind your Scholar Research Framework') 
\copyright K. Reincke CC BY 3.0  http://mycsrf.fodina.de/)] }}, to allow other people to correct
  us, to help us or even to take our results for their own purposes.
\end{description}

And so we did. Here is the result. Feel free to use it - according to our
licensing.

%\bibliography{../../../bibfiles/oscResourcesEn}

% Telekom osCompendium 'for beeing included' snippet template
%
% (c) Karsten Reincke, Deutsche Telekom AG, Darmstadt 2011
%
% This LaTeX-File is licensed under the Creative Commons Attribution-ShareAlike
% 3.0 Germany License (http://creativecommons.org/licenses/by-sa/3.0/de/): Feel
% free 'to share (to copy, distribute and transmit)' or 'to remix (to adapt)'
% it, if you '... distribute the resulting work under the same or similar
% license to this one' and if you respect how 'you must attribute the work in
% the manner specified by the author ...':
%
% In an internet based reuse please link the reused parts to www.telekom.com and
% mention the original authors and Deutsche Telekom AG in a suitable manner. In
% a paper-like reuse please insert a short hint to www.telekom.com and to the
% original authors and Deutsche Telekom AG into your preface. For normal
% quotations please use the scientific standard to cite.
%
% [ Framework derived from 'mind your Scholar Research Framework' 
%   mycsrf (c) K. Reincke 2012 CC BY 3.0  http://mycsrf.fodina.de/ ]
%


%% use all entries of the bibliography
%\nocite{*}

\section{What}

Now we can briefly explain how you can use this compendium:

\begin{description}
  \item[Open Source: Idea and Concepts] :- Here you will find background
  information to help you interpret Open Source Licenses in the sense of the
  \emph{Free / Libre Open Source movement}\footnote{At least at this place you
  are perhaps expecting that we use the logograms FLOSS, F/OSS, F/LOSS, or
  whatever. As you will read later on the word \textit{Free} is ambiguous and
  has strained the use of the concept \textit{Free Software}. Later on we will
  also talk about the invention of the concept \textit{Open Source} designed as
  a 'replacement' and acting as a 'splitter'. The mentioned logograms are
  introduced to re-establish or - at least - to underline the common history and
  the common center of 'both' movements, whereby the word \textit{Libre} shall
  resolve the abiguity of the word \textit{Free}. For a first survey
  \cite[cf.][\nopage wp.]{wpFloss2011a} For another brief and informative
  introduction \cite[cf.][231ff esp. p. 232f.]{Fogel2006a} We ourselves will
  stay with the concept \textit{Open Source} because the OSD specifies the scope
  of our analysis. But we do it with a deep obeisance to Stallmann and the FSF -
  even if we know that this will not protect us from the thunderbolt of RMS.}. If
  you are familiar with the evolution of the Open Source Initiative, with the
  character of the Open Source Definition as a set of necessary but insufficient
  criteria and if you know about the history and meaning of free software as
  older and broader concept then you can ignore this chapter.
  \item[The Same Idea, Different Licenses] :- In this chapter we discuss
  different ways to cluster Open Source Licenses. Finally we present our own
  taxonomy based on the labels 'protecting the developer', 'protecting the
  licensed code' and 'protecting the on-top-developments'. If you are familiar
  with the methods of grouping different Open Source Licenses and particular
  if you know that you can not authorize your doings on the base of descriptions
  of such license groups than it's enough, in order to understand our line of
  thought, to briefly note our taxonomy and its wording.
  \item[The Problem of Derivated Works] :- This chapter is important. In the
  spirit of software developer we try to explain which kinds of programming
  evoke a derivated work and which not. Our to-do lists will refer to this
  analysis.
  \item[The Problem of Combining Different Licenses] :- You should
  not ignore this chapter. We will explain why and how combining software
  of different licenses is not as dangerous as it's often told. The results of
  this chapter influence the structure of our to-do lists.
  \item[Open Source Software and Money] :- Here we will shortly
  discuss ways in which money is no problem. If you already know that it is only
  prohibited to require payment for the act of licensing a piece of Open Source
  Software to second or third parties and if you already know that this is only
  forbidden by some licenses, and not by all, than you can postpone the reading
  of this chapter.
  \item[The Problem of Implicitly Freeing Patents] :- Here we
  will illuminate some aspects of software patents and how the are handled by
  some Open Source Licenses. You should know what licenses implicitly do with
  your patents. But it's not our intention to write a software patent
  compendium.
  \item[Open Source: Use Cases as Principle of Classification] :- This is an
  important chapter. We explain our categories 'Use as it is', 'Modify the
  Code', 'With Redistribution', 'Without Redistribution', 'Isolated Initial
  Development', 'On-Top-Development': we develop and discuss our taxonomy with
  respect to the side effects of 'combining different licenses' and 'generating
  derivated works'. This taxonomy will determine the following chapters.
  \item[Open Source Licenses: Find Your Specific To-do Lists] :- This is a kind
  of summary which joins the relevant aspects and elaborates the 'finder
  for your to-do lists'. This is that chapter which you probably will reuse
  multiply, even if you do not want to read any of our explanations.
  \item[Open Source License Fulfillment: Classified To-do Lists] :- This chapter
  offers all classified to-do lists. The structure of its' subchapters will
  match the structure of our finder and the structure of our taxonomy.
  \item[Open Source Licenses and Their Legal Environments] :- Here we discuss
  that using Open Source Software in a regular manner is not only a question of
  the licenses themselves but of the kind of the surrounding legal system.
  \item[Appendices: Some Widespread Open Source Myths] :- Here we make good on
  our promise to explain why all the propositions mentioned at the beginning of
  this chapter are wrong. You might read this chapter as a special introduction
  or a reminder epilogue whenever you want to do.
\end{description}


%\bibliography{../../../bibfiles/oscResourcesEn}

% 
% Telekom osCompendium 'for beeing included' snippet
%
% (c) Karsten Reincke, Deutsche Telekom AG, Darmstadt 2011
%
% This LaTeX-File is licensed under the Creative Commons Attribution-ShareAlike
% 3.0 Germany License (http://creativecommons.org/licenses/by-sa/3.0/de/): Feel
% free 'to share (to copy, distribute and transmit)' or 'to remix (to adapt)'
% it, if you '... distribute the resulting work under the same or similar
% license to this one' and if you respect how 'you must attribute the work in
% the manner specified by the author ...':
%
% In an internet based reuse please link the reused parts to www.telekom.com and
% mention the original authors and Deutsche Telekom AG in a suitable manner. In
% a paper-like reuse please insert a short hint to www.telekom.com and to the
% original authors and Deutsche Telekom AG into your preface. For normal
% quotations please use the scientific standard to cite.
%
% [ File structure derived from 'mind your Scholar Research Framework' 
%   mycsrf (c) K. Reincke CC BY 3.0  http://mycsrf.fodina.de/ ]

%
%% use all entries of the bibliography
%\nocite{*}
A final remark: We have already characterized the tone of our footnotes. Let us
now briefly explain a little peculiarity of our bibliography:

Modern times have also changed the humanities. Formerly a book or an article
must be printed for being ripe to be quoted. Our statements relied on static,
readily prepared works. Nowadays even university libraries sometimes offer those
books and articles as PDF files which are printed in the original. As a scholar,
now you must rely on the equality of the printed version and the PDF file - at
least with respect to the page numbers and the appearance. You can not verify the
equivalence - at least to a certain degree.

Moreover: in case of such 'e-books' and 'e-articles' the libraries often do not
offer the pdf files themselves but links to the download pages of the publisher.
Formerly as a scholar you could trust that your readers would be able to
retrieve the quoted work if they want to verify your citations. It's one task of
our libraries to hold available our scientific sources. But now they do not buy
any longer the books, but the right to download files over the university net.
In this case these PDF files are not stored on the serves of the university
library. By using the link provided by the publisher each student or each reader
downloads his own file - case by case. Therefore - as a scholar - you now have
to trust that the publisher, who provides the link, will not change that pdf
file that you have cited.

But it gets even worse: While it might be that publishers modify their work
secretly (even it is not very likely that they do it), it's a definite feature
of the web that its' pages are fre\-quen\-tly changed. Hence we must ask
ourselves: Can we seriously argue on the basis of statements and documents which
might disappear? Can we quote such possibly volatile sources? The problem is: we
must do it, especially if we write about an internet topic - and even if we want
to write a really reliable compendium.

So, what can we do? Firstly we must confide in our readers, that they either
will retrieve our sources or - if they can not find them - that they
believe that we really have found and read what we have written and
quoted. Secondly we store all these e-wares\footnote{Take this little word as
(new) generalization of 'e-book', 'e-article', 'e-paper' and so on.} we
read\footnote{But because of the copyright we ourselves are naturally not
allowed to offer a download link for them or to send a copy of it to those who
want to verify our quotes.}. And thirdly we should lay open to our readers the
different levels of reliableness of our sources. Therefore we use
the following markers in our bibliographic data\footnote{And another hint: Nowadays sometimes
even scientific libraries doesn't offer exact 'e-copies' of the original. In
some cases one can get only html-versions of articles which formerly were
printed as part of journals. In these case the scholar has to use sources which
lost their original page-numbers. The same can happen to articles of proceedings
etc. which are now only offered as autonomous pdf files with an internal paging.
If we quote such kind of articles we try to specify the number of the quoted
article in the original row of articles, added - if possible - by an internal
page number. But naturally we also try to follow the bibliographic data
delivered by that organization which distributes these kind of copies.}:

\begin{itemize}
	\item Print / Copy:- The source is printed and we saw either the printed work
	really or we get an official copy by our library. Hence you should also be able
	to get the work in a library, at least in those we used (UB Frankfurt or ULB
	Darmstadt).
	\item BibWeb/[PDF/\ldots] :- The source might be printed, but we read only the
	electronic version (PDF or other type of format), offered by and over the
	net of our university libraries (UB Frankfurt or ULB Darmstadt).
  \item FreeWeb/[PDF/\ldots] :- We read the electronic version offered by the
  free web. In this case we add the url\footnote{Please note: Long urls often
  destroy the pleasing appearance of a text because it's difficult to wrap the
  lines acceptably. Hence we wished to make it easier for LaTeX to do this job.
  Therefor we sometime split the urls and inserted blanks. So you have to erase
  all blanks if you want to verify our urls.} and the date when we downloaded /
  saw the text.
\end{itemize}

% 
%% Telekom osCompendium 'for beeing included' snippet template
%
% (c) Karsten Reincke, Deutsche Telekom AG, Darmstadt 2011
%
% This LaTeX-File is licensed under the Creative Commons Attribution-ShareAlike
% 3.0 Germany License (http://creativecommons.org/licenses/by-sa/3.0/de/): Feel
% free 'to share (to copy, distribute and transmit)' or 'to remix (to adapt)'
% it, if you '... distribute the resulting work under the same or similar
% license to this one' and if you respect how 'you must attribute the work in
% the manner specified by the author ...':
%
% In an internet based reuse please link the reused parts to www.telekom.com and
% mention the original authors and Deutsche Telekom AG in a suitable manner. In
% a paper-like reuse please insert a short hint to www.telekom.com and to the
% original authors and Deutsche Telekom AG into your preface. For normal
% quotations please use the scientific standard to cite.
%
% [ File structure derived from 'mind your Scholar Research Framework' 
%   mycsrf (c) K. Reincke CC BY 3.0  http://mycsrf.fodina.de/ ]

%
\newpage
\section{Form [only to demo our lib style. will be replaced]}
\begin{itemize}
  \item first initially quoted book\footnote{\cite[cf.][123ff]{Grassmuck2002a}
  (expected: complete bibl. data)} using LaTeX \texttt{$\backslash$footnote}
  \item second initially quoted book\footcite[cf.][120 (expected: complete bibl.
  data)]{Fogel2006a} using jurabib \texttt{$\backslash$footcite} (same
  appereance)
  \item initially mentioned collection /
  proceedings\footnote{\cite[cf.][123ff]{DjoGehGraKreSpi2008a} (expected: complete
  bibl. data)}
  \item first initially mentioned article in an initially mentioned collection /
  proceedings\footnote{\cite[cf.][123ff]{Spielkamp2008a} (expected: complete
  bibl. data of article, short title data of collection)} using LaTeX
  \texttt{$\backslash$footnote}
  \item second initially mentioned article in an already mentioned collection
  / proceedings\footcite[cf.][123ff (expected: complete
  bibl. data of article, short title data of collection)]{Kreutzer2008a} using
  jurabib \texttt{$\backslash$footcite} 
  \item rementioned book\footnote{\cite[cf.][120]{Fogel2006a} (expected: short
  title)}
  \item directly rementioned same book same
  page\footnote{\cite[cf.][120]{Fogel2006a} (expected: id., ibid, / ders.,
  ebda.,)}
  \item directly rementioned same book different
  page\footnote{\cite[cf.][121]{Fogel2006a} (expected: id., lc., / ders.,
  a.a.O. \& page)}
  \item rementioned collection article\footnote{\cite[cf.][120 ]{Kreutzer2008a} (expected: short
  title)}
  \item directly rementioned collection article same
  page\footnote{\cite[cf.][120]{Kreutzer2008a} (expected: id., ibid, / ders.,
  ebda.,)}
  \item directly rementioned collection article different
  page\footnote{\cite[cf.][121]{Kreutzer2008a} (expected: id., lc., / ders.,
  a.a.O. \& page)}
\end{itemize}

%\bibliography{../bibfiles/oscResourcesEn}


%%%%%%%%%%%%%%%
\chapter{Open Source: Idea and Concepts}
% Telekom osCompendium 'for being included' snippet template
%
% (c) Karsten Reincke, Deutsche Telekom AG, Darmstadt 2011
%
% This LaTeX-File is licensed under the Creative Commons Attribution-ShareAlike
% 3.0 Germany License (http://creativecommons.org/licenses/by-sa/3.0/de/): Feel
% free 'to share (to copy, distribute and transmit)' or 'to remix (to adapt)'
% it, if you '... distribute the resulting work under the same or similar
% license to this one' and if you respect how 'you must attribute the work in
% the manner specified by the author ...':
%
% In an internet based reuse please link the reused parts to www.telekom.com and
% mention the original authors and Deutsche Telekom AG in a suitable manner. In
% a paper-like reuse please insert a short hint to www.telekom.com and to the
% original authors and Deutsche Telekom AG into your preface. For normal
% quotations please use the scientific standard to cite.
%
% [ File structure derived from 'mind your Scholar Research Framework' 
%   mycsrf (c) K. Reincke CC BY 3.0  http://mycsrf.fodina.de/ ]
%

% Chapter Abstract
% ----------------

\footnotesize
\begin{quote}\itshape In this chapter we discuss the meaning of \emph{Open
Source} and the common features of all Open Source Software. We scan historical
and con\-ceptual aspects. It's the chapter of background knowledge. At the end
you will know that software is only Open Source Software if you as customer gets the
right to use, to modify, and to redistribute the code without any limitations,
but with some obligations. And these obligations implicitly refer to different
historical contexts which might influence the understanding of your licenses.
\end{quote}
\normalsize{}


% Telekom osCompendium 'for beeing included' snippet template
%
% (c) Karsten Reincke, Deutsche Telekom AG, Darmstadt 2011
%
% This LaTeX-File is licensed under the Creative Commons Attribution-ShareAlike
% 3.0 Germany License (http://creativecommons.org/licenses/by-sa/3.0/de/): Feel
% free 'to share (to copy, distribute and transmit)' or 'to remix (to adapt)'
% it, if you '... distribute the resulting work under the same or similar
% license to this one' and if you respect how 'you must attribute the work in
% the manner specified by the author ...':
%
% In an internet based reuse please link the reused parts to www.telekom.com and
% mention the original authors and Deutsche Telekom AG in a suitable manner. In
% a paper-like reuse please insert a short hint to www.telekom.com and to the
% original authors and Deutsche Telekom AG into your preface. For normal
% quotations please use the scientific standard to cite.
%
% [ Framework derived from 'mind your Scholar Research Framework' 
%   mycsrf (c) K. Reincke 2012 CC BY 3.0  http://mycsrf.fodina.de/ ]
%


%% use all entries of the bibliography
%\nocite{*}

[TDB \ldots]

%\bibliography{../../../bibfiles/oscResourcesEn}

% Telekom osCompendium 'for being included' snippet template
%
% (c) Karsten Reincke, Deutsche Telekom AG, Darmstadt 2011
%
% This LaTeX-File is licensed under the Creative Commons Attribution-ShareAlike
% 3.0 Germany License (http://creativecommons.org/licenses/by-sa/3.0/de/): Feel
% free 'to share (to copy, distribute and transmit)' or 'to remix (to adapt)'
% it, if you '... distribute the resulting work under the same or similar
% license to this one' and if you respect how 'you must attribute the work in
% the manner specified by the author ...':
%
% In an internet based reuse please link the reused parts to www.telekom.com and
% mention the original authors and Deutsche Telekom AG in a suitable manner. In
% a paper-like reuse please insert a short hint to www.telekom.com and to the
% original authors and Deutsche Telekom AG into your preface. For normal
% quotations please use the scientific standard to cite.
%
% [ Framework derived from 'mind your Scholar Research Framework' 
%   mycsrf (c) K. Reincke 2012 CC BY 3.0  http://mycsrf.fodina.de/ ]
%


%% use all entries of the bibliography
%\nocite{*}
\section{Open Source, OSI, and OSD}
\footnotesize
\begin{quote}\itshape
Here we describe the meaning of Open Source. At the end of this chapter you
should know that Open Source Software is defined by a set of necessary criteria
which together determine the common basic features of Open Source Licenses.
Additionally you will have understood that the opposite of Open Source Software
can not only by defined ex negativo. But you should also know that these features
can differently be implemented. Therefore the OSD can not be read as a set of
rules describing what we have to do if we want to fulfill the Open Source
Licenses. You should know that you have to go back to the license itself.
\end{quote}
\normalsize
\ldots

%\bibliography{../../../bibfiles/oscResourcesEn}

% Telekom osCompendium 'for beeing included' snippet template
%
% (c) Karsten Reincke, Deutsche Telekom AG, Darmstadt 2011
%
% This LaTeX-File is licensed under the Creative Commons Attribution-ShareAlike
% 3.0 Germany License (http://creativecommons.org/licenses/by-sa/3.0/de/): Feel
% free 'to share (to copy, distribute and transmit)' or 'to remix (to adapt)'
% it, if you '... distribute the resulting work under the same or similar
% license to this one' and if you respect how 'you must attribute the work in
% the manner specified by the author ...':
%
% In an internet based reuse please link the reused parts to www.telekom.com and
% mention the original authors and Deutsche Telekom AG in a suitable manner. In
% a paper-like reuse please insert a short hint to www.telekom.com and to the
% original authors and Deutsche Telekom AG into your preface. For normal
% quotations please use the scientific standard to cite.
%
% [ Framework derived from 'mind your Scholar Research Framework' 
%   mycsrf (c) K. Reincke 2012 CC BY 3.0  http://mycsrf.fodina.de/ ]
%


%% use all entries of the bibliography
%\nocite{*}

\section{Open Source and its history: some hints}
\footnotesize
\begin{quote}\itshape
Here we present main lines of the Open Source genesis: The start with the
bundling of hardware and software in the beginning on the one side and the
monopol of AT\&T and the free distribution of unix in the universities on the
other side - which together established the free hacker culture. We will shortly
describe the increase of the value of software evoked by the IBM unbundling
strategy and the antitrust suit against AT\&T which let become the software a
value itself worthful of protection and which destroyed the free exchange within
the early hacker community. Naturally we will illuminate the answer of RMS, the
GNU project, the founding of the FSF and the GPL. Then we will highlight the
introduction of the concept Open Source invented for dissolving the troubles to
talk about Free Software with managers of companies. We will hint to the Linux
kernel as an unwelcome completion of the GNU system. Finally we will outline the
convergency of business and Open Source, not only by Netscape/Mozilla, IBM
apache, Redhat, SUN/OpenOffice but also by IBM/eclipse, Sun/Java and so on. And
naturally we will highlight the meaning of 'the Cathedral and the Bazar', which
had not been written to contrast the working style of the Open Source Developemt
and the proprietary 'in company' development by for example microsoft, but for
dicern the working and leading style of RMS and Linus.
\end{quote}
\normalsize
\ldots

%\bibliography{../../../bibfiles/oscResourcesEn}

% Telekom osCompendium 'for beeing included' snippet template
%
% (c) Karsten Reincke, Deutsche Telekom AG, Darmstadt 2011
%
% This LaTeX-File is licensed under the Creative Commons Attribution-ShareAlike
% 3.0 Germany License (http://creativecommons.org/licenses/by-sa/3.0/de/): Feel
% free 'to share (to copy, distribute and transmit)' or 'to remix (to adapt)'
% it, if you '... distribute the resulting work under the same or similar
% license to this one' and if you respect how 'you must attribute the work in
% the manner specified by the author ...':
%
% In an internet based reuse please link the reused parts to www.telekom.com and
% mention the original authors and Deutsche Telekom AG in a suitable manner. In
% a paper-like reuse please insert a short hint to www.telekom.com and to the
% original authors and Deutsche Telekom AG into your preface. For normal
% quotations please use the scientific standard to cite.
%
% [ Framework derived from 'mind your Scholar Research Framework' 
%   mycsrf (c) K. Reincke 2012 CC BY 3.0  http://mycsrf.fodina.de/ ]
%


%% use all entries of the bibliography
%\nocite{*}

\section{\emph{Free Software} versus \emph{Open Sourc}e: some hints}
\footnotesize
\begin{quote}\itshape
Here we will illuminate the differences between the ideas of the Free Software
Movement and the Open Source Movement.
\end{quote}
\normalsize
\ldots

%\bibliography{../../../bibfiles/oscResourcesEn}


%%%%%%%%%%%%%%%
\chapter{The Same Idea, Different Models of License}
% Telekom osCompendium 'for being included' snippet template
%
% (c) Karsten Reincke, Deutsche Telekom AG, Darmstadt 2011
%
% This LaTeX-File is licensed under the Creative Commons Attribution-ShareAlike
% 3.0 Germany License (http://creativecommons.org/licenses/by-sa/3.0/de/): Feel
% free 'to share (to copy, distribute and transmit)' or 'to remix (to adapt)'
% it, if you '... distribute the resulting work under the same or similar
% license to this one' and if you respect how 'you must attribute the work in
% the manner specified by the author ...':
%
% In an internet based reuse please link the reused parts to www.telekom.com and
% mention the original authors and Deutsche Telekom AG in a suitable manner. In
% a paper-like reuse please insert a short hint to www.telekom.com and to the
% original authors and Deutsche Telekom AG into your preface. For normal
% quotations please use the scientific standard to cite.
%
% [ File structure derived from 'mind your Scholar Research Framework' 
%   mycsrf (c) K. Reincke CC BY 3.0  http://mycsrf.fodina.de/ ]
%

% Chapter Abstract
% ----------------

\footnotesize \begin{quote}\itshape In this chapter we describe different
license models which meet the common idea of being a piece of Free Open Source
Software. We want to discuss existing types of grouping licenses to underline
the limits of building such clusters: These groups are often used as 'virtual
prototypic licenses' which shall deliver a simplified view onto the conditions
how to act according to the referred real license instances. But one has to
fulfill the requirements of a specific license, not one's own generalized idea
of a set of licenses. Nevertheless, we wish to offer structuring view into the
world of the Open Source Licenses too. We will use a new set of grouping
criteria by referring to the common intended purpose of each license: each
license wants to protect something or someone against something or someone.
Following this pattern, we can indeed summarize the essence of each license in a
comparable way.
\end{quote}
\normalsize{}


% Telekom osCompendium 'for being included' snippet template
%
% (c) Karsten Reincke, Deutsche Telekom AG, Darmstadt 2011
%
% This LaTeX-File is licensed under the Creative Commons Attribution-ShareAlike
% 3.0 Germany License (http://creativecommons.org/licenses/by-sa/3.0/de/): Feel
% free 'to share (to copy, distribute and transmit)' or 'to remix (to adapt)'
% it, if you '... distribute the resulting work under the same or similar
% license to this one' and if you respect how 'you must attribute the work in
% the manner specified by the author ...':
%
% In an internet based reuse please link the reused parts to www.telekom.com and
% mention the original authors and Deutsche Telekom AG in a suitable manner. In
% a paper-like reuse please insert a short hint to www.telekom.com and to the
% original authors and Deutsche Telekom AG into your preface. For normal
% quotations please use the scientific standard to cite.
%
% [ Framework derived from 'mind your Scholar Research Framework' 
%   mycsrf (c) K. Reincke 2012 CC BY 3.0  http://mycsrf.fodina.de/ ]
%


%% use all entries of the bibliography
%\nocite{*}

Grouping Open Source licenses is often used. Even the set of \emph{Open Source
Li\-cen\-ses}\footcite[cf.][\nopage wp]{OSI2012b} itself is already a cluster
being established by a set of grouping criteria: The \enquote{distribution
terms} of each software license which wants to be an Open Source License,
\enquote{[\ldots] must comply with the [\ldots] criteria} of the \emph{Open
Source Definition}\footcite[cf.][\nopage wp]{OSI2012a}, maintained by the
\emph{Open Source Initiative}\footcite[cf.][\nopage wp]{OSI2012c} and often
abbreviated as \emph{OSD}. So, this \emph{OSD} demarcates 'the group of
[potential] Open Source Licenses' against 'the group of not Open Sources
Licenses'\footnote{For stating it more precisely: to meet the OSD, is only a
necessary condition for being an \emph{Open Source License}. The sufficient
condition for becoming an \emph{Open Source License}, is the approval by the
OSI, which offers a process for becoming an officially approved \emph{Open
Source License} (\cite[cf.][\nopage wp.]{OSI2012d}).}.

Another way to cluster the \emph{Free Software Licenses} is specified by the
\enquote{Free Software Definition}. This \emph{FSD} contains four conditions
which must be met by any free software license: an FSD compliant license must
asign the \enquote{the freedom to run a program, for any purpose [\ldots]},
\enquote{the freedom to study how it works, and adapt it to your needs
[\ldots]}, \enquote{the freedom to redistribute copies [\ldots]}, and finally
\enquote{the freedom to improve the program, and release your improvements
[\ldots]}\footcite[cf.][41]{Stallman1996a}. Surprisingly this definition
implies, that the requirement \emph{the sourcecode must be openly accessible},
is 'only' a derived condition. If the \enquote{freedom to make changes and the
freedom to publish improved versions} shall be \enquote{meaningful}, then the
\enquote{access to the source code of the program} is a prerequisite.
\enquote{Therefore, accessibility of source code is a necessary condition for
free software.}\footcite[cf.][41]{Stallman1996a}

The difference between these the OSD and the FSD has often been described as a
difference of emphasizing\footnote{This is also the viewpoint of Richard M.
Stallman: On the one hand, he clearly states that the \enquote{Free Software
movement} and the \enquote{Open Source movement} overall \enquote{[\ldots]
disagree on the basic principles, but agree more or less on the practical
recommendations} and that he \enquote{[\ldots] (does) not think of the Open
Source movement as an enemy}.  On the other hand, he deliniates the two
movements by stating, that \enquote{for the Open Source movement, the issue of
whether software should be open source is a practical question, not an ethical
one}, while \enquote{for the Free Software movement, non-free software is a
social problem and free software is the solution}
(\cite[cf.][55]{Stallman1998a}). \label{RmsFsPriority} As consequence, Richard
M. Stallman summarizes the positions in a simple way: \enquote{[\ldots] 'open
source' was designed not to raise [\ldots] the point that users deserve
freedom}. But he and his friends want \enquote{to spread the idea of freedom}
and therefore \enquote{[\ldots] stick to the term 'free software'}
(\cite[][59]{Stallman1998a}).}: Although both definitions \enquote{[\ldots]
(cover) almost exactly the same range of software}, the \emph{Free Software
Foundation} - as it is said - \enquote{prefers [\ldots] (to emphazise) the idea
of freedom [\ldots]} while the \emph{OSI} wants to underline the philosophically
indifferent \enquote{development methodolgy}\footcite[pars pro toto:
cf.][232]{Fogel2006a}.

A third method to collect a special group of free software and free software
licenses is specified by the \enquote{Debian Free Software Guideline}, which is
embedded into the \enquote{Debian Social Contract}. This \enquote{DFSG} contains
nine defining criteria which - as Debian itself says - have been
\enquote{[\ldots] adopted by the free[sic!] software community as the basis of
the Open Source Definition}\footcite[cf.][wp]{DFSG2013a}.

A rough understanding of these methods might allow to conclude that these three
definitions are extensionally equal and only differ intensionally.
But that's not true. To unveil the differences, let us compare the clusters
\emph{OSI approved licenses}, \emph{OSD compliant licenses}, \emph{DFSG
compliant licenses}, and \emph{FSD compliant licenses} extensionally, by asking
whether they \emph{could} establish different sets of licenses\footnote{Indeed,
for analyzing the extensional power of the definition we have to regard all
potentially covered licenses, not only the already existing licenses, because
the subset of really existing licenses still could be expanded be developing new
licenses which fit the definition.}.

Firstly, the most easily to determine difference is that of an unidirectional
inclusion: By definition, the \emph{OSI approved licenses} and the \emph{OSD
compliant licenses} meet the requirements of the OSD\footcite[cf.][\nopage
wp]{OSI2012a}. But only the \emph{OSI approved licenses} have successfully
passed the OSI process\footcite[cf.][\nopage wp]{OSI2012a} and therefore are
officially listed as \emph{Open Source Licenses}\footcite[cf.][\nopage
wp]{OSI2012b}. Hence, on the one hand, \emph{OSI approved licenses} are
\emph{Open Source Licenses} and vice versa. On the other hand, both - the
\emph{OSI approved licenses} and the \emph{Open Source Licenses} - are \emph{OSD
compliant licenses}, but not vice versa.

Secondly, a similar argumentation leeds to the differences between the
\emph{DFSG compliant licenses} and the \emph{OSI approved licenses}. As it is
stated, the OSD \enquote{[\ldots] is based on the Debian Free Software
Guildeline and any license that meets one definition almost meets the
other}\footcite[cf.][233]{Fogel2006a}. But again, meeting the definition is not
enough for being an official Open Source License: the license has to be approved
by the OSI\footcite[cf.][\nopage wp]{OSI2012b}. So, one can analogically say,
that all \emph{OSI approved licenses} are also \emph{DFSG compliant licenses},
but not vice versa.

Thirdly - by ignoring the \enquote{few exceptions}, which have been appeared
\enquote{over the years}\footcite[cf.][233]{Fogel2006a} - one can say that,
because of their 'kinsmanlike' relation, at least the \emph{OSD compliant
licenses} are also \emph{DFSG compliant licenses} and vice versa.

Last but not least one has to state, that the (potential) set of free software
licenses must be greater than all the other three sets: On the one side, the FSD
only requires that the license of a software must allow to read and to use the
software, to modify and to distribute it\footcite[cf.][41]{Stallman1996a}. These
conditions are covered by at least the first three paragraphs of the OSD
concerning the topics \enquote{Free Redistribution}, \enquote{Source Code}, and
\enquote{Derived Works}\footcite[cf.][\nopage wp]{OSI2012a}. But the OSD
contains at least some requirements, which are not mentioned by the FSD and
which nevertheless must be met by a license for being an OSD compliant
license\footnote{For example, see the condition that \enquote{the license must
be technology-neutral} (\cite[cf.][\nopage wp]{OSI2012a}).}. Hence, logically
regarded, there might exist licenses, which fulfill all conditions of the FSD
and nevertheless do not fulfill at least some conditions of the OSD\footnote{
Let us repeat: we must consider the extensional potential of the definitions,
not the set of really existing licenses. In this context, it is irrelevant, that
actually all existing Free Software Licenses like GPL, LGPL or AGPL indeed are
also classfied as Open Source Licenses. We are referring to the fact, that there
might be generated licenses which fulfill the FSD, but not the OSD.}. So, the
set of all (potential) \emph{Free Software Licenses} must be greater than the
set of all (potential) \emph{Open Source Licenses} and greater than the set of
\emph{OSD compliant licenses}.

All in all, we can visualize the situation by a picture like this:

\begin{center}

\begin{tikzpicture}
\label{LICTAX}
\small

\node[ellipse,minimum height=5.8cm,minimum width=11.6cm,draw,fill=gray!10] (l0210) at (5,5)
{ };

\draw [-,dotted,line width=0pt,white,
    decoration={text along path,
              text align={center},
              text={|\itshape|All Software Licenses}},
              postaction={decorate}] (0,6.1) arc (120:60:10cm);

\node[ellipse,minimum height=4.4cm,minimum width=10cm,draw,fill=gray!20] (l0210) at (5,5)
{ };

\draw [-,dotted,line width=0pt,white,
    decoration={text along path,
              text align={center},
              text={|\itshape|FSD Compliant Licenses}},
              postaction={decorate}] (0,5.4) arc (120:60:10cm);
              

\node[ellipse,minimum height=3cm,minimum width=8.4cm,draw,fill=gray!30] (l0210) at (5,5)
{ };


             
\draw [-,dotted,line width=0pt,white,
    decoration={text along path,
              text align={center},
              text={|\itshape|OSD Compliant Licenses}},
              postaction={decorate}] (0,4.7) arc (120:60:10cm);
              
\draw [-,dotted,line width=0pt,white,
    decoration={text along path,
              text align={center},
              text={|\itshape|DFSG Compliant Licenses}},
              postaction={decorate}] (0,5) arc (240:300:10cm);
          

\node[ellipse,text width=4.4cm, text centered,minimum height=1.6cm,minimum width=6cm,draw,fill=gray!40] (l0210) at (5,5)
{ \textit{OSI approved licenses} = \\ \textit{\textbf{Open Source Licenses}}
};

\end{tikzpicture}
\end{center}

It should be clear without longer explanations, that these clusters don't allow
to derive a correct compliant behaviour according to the \emph{Open Source
Licenses}: On the one hand, all larger clusters do not talk about the \emph{Open
Source Licenses}. On the other hand, the \emph{Open Source License cluster}
itself only collects his elements on the base of the OSD, which does not
stipulates concrete license fulfilling actions of the licensee.

The next level of clustering \emph{Open Source Licenses} concerns the inner
structure of these \emph{OSI approved licenses}. Even the OSI itself has recently
discussed whether a better kind of grouping the listed licenses would better fit
the needs of the visitors of the OSI site\footcite[cf.][\nopage wp]{OSI2013a}.
And finally the OSI ends up in the categories \enquote{popular and widely used
(licenses) or with strong communities}, \enquote{special purpose licenses},
\enquote{other/miscellaneous licenses}, \enquote{licenses that are redundant
with more popular licenses}, \enquote{non-reusable licenses},\enquote{superseded
licenses}, \enquote{licenses that have been voluntarily retired}, and \enquote{
uncategorized Licenses}\footcite[cf.][\nopage wp]{OSI2013b}.

Another way to structure the field of Open Source Licenses is to think in
\enquote{types of Open Source Licenses} by grouping the \enquote{\emph{academic
licenses}, so named because they were originally created by academic
institutions}\footcite[cf.][69]{Rosen2005a}, the \enquote{\emph{reciprocal
licenses}}, so named because they \enquote{[\ldots] require the distributors of
derivative works to dis\-tri\-bu\-te those works under same license including the
requirement that the source code of those derivative works be
published}\footcite[cf.][70]{Rosen2005a}, the \enquote{\emph{standard
licenses}}, so named because the refer to the reusability of \enquote{industry
standards}\footcite[cf.][70]{Rosen2005a}, and the \enquote{\emph{content
licenses}}, so named because they refer to
\enquote{[\ldots] other than software, such as music art, film, literary works}
and so on\footcite[cf.][71]{Rosen2005a}.

Both kind of taxonomies directly help to find the relevant licenses which should
be used for new (software) projects. But again: none of these categories does
allow to infer the license compliant behaviour, because the categories are
mostly defined on the base of license external criteria: answers to the
questions, whether a license is published by a specific kind of organizations or
whether license concerns industry standards or other kind of works than
software, inherently do not evoke a license fulfilling behaviour.

Only the act of grouping into the \enquote{\emph{academic licenses}} and the
\enquote{\emph{reciprocal licenses}} touches the idea of license fulfilling
doings, if one - as it has been done - expands the definition of the
\enquote{\emph{academic licenses}} by the specification, that these licenses
\enquote{[\ldots] allow the software to be used for any purpose whatsoever with
no obligation on the part of the licensee to distribute the source code of
derivative works}\footcite[cf.][71]{Rosen2005a}. With respect to this additional
specification, the clusters \enquote{\emph{academic licenses}} and the
\enquote{\emph{reciprocal licenses}} indeed might be referred as the
\enquote{main categories} of (Open Source)
licenses\footcite[cf.][179]{Rosen2005a}: By definition, they are constituting
not only a contrary, but contradictory opposite. But, one has also to mind that
they build an antinomy inside of the set of Open Source Licenses\footnote{Hence,
it is at least a little confusing to say, that \enquote{the Open Source License
(OSL) is a reciprocal license} and \enquote{the Academic Free License (AFL) is
the exact same license without the reciprocity provisions}
(\cite[cf.][180]{Rosen2005a}): If the BSD license is an AFL and if an AFL can't
be an OSL and if the OSI approves only OSLs, then the BSD license can't be an
approved Open Source License. But in fact, it is (\cite[cf.][\nopage
wp]{OSI2012b}).}.

Connatural to the clustering into \emph{academic licenses} and \emph{reciprocal
licenses} is the grouping into \emph{permissive licenses}, \emph{weak copyleft
licenses}, and \emph{strong copyleft licenses}: Even Wikipedia already uses the
term \enquote{permissive free software licence} in the meaning of \enquote{a
class of free software licence[s] with minimal requirements about how the
software can be redistributed} and \enquote{contrasts} them with the
\enquote{copyleft licences} as those \enquote{with reciprocity / share-alike
requirements}\footcite[cf.][\nopage wp]{wpPermLic2013a}. Some other authors name
the set of \emph{academic licenses} the \enquote{permissive licenses} and
specify the \emph{reciprocal licenses} as \enquote{restrictive licenses},
because in this case - as consequence of the embedded \enquote{copyleft} effect
- the source code must be published in case of modifications. And they
additionally introduce the subset of \enquote{strong restrictive licenses} which
additionally require that an (overarching) derivative work must be published
under the same license\footcite[pars pro toto cf.][57]{Buchtala2007a}. The next
refinement of such clustering concepts directly uses the categories
\enquote{[Open Source] licenses with a strict copyleft
clause}\footcite[Originally stated as \enquote{Lizenzen mit einer strengen
Copyleft-Klausel}. Cf.][24]{JaeMet2011a}, \enquote{[Open Source] licenses with a
restricted copyleft clause}\footcite[Originally stated as \enquote{Lizenzen mit
einer beschränkten Copyleft-Klausel}. Cf.][71]{JaeMet2011a}, and \enquote{[Open
Source] licenses without any copyleft clause}\footcite[Originally stated as
\enquote{Lizenzen ohne Copyleft-Klausel}. Cf.][83]{JaeMet2011a}. Finally, this
viewpoint can directly be mapped to the categories \emph{strong copyleft} and
\emph{weak copyleft}: While on the one hand, \enquote{only changes to the
weak-copylefted software itself become subject to the copyleft provisions of
such a license, [and] not changes to the software that links to it}, on the
other hand, the \enquote{strong copyleft} states \enquote{[\ldots] that the
copyleft provisions can be efficiently imposed on all kinds of derived
works}\footcite[cf.][\nopage wp]{wpCopyleft2013a}.

Based on these approach to an adequate clustering and labeling, we can develop
the following picture:

\begin{center}

\begin{tikzpicture}
\label{LICTAX}
\small



\node[ellipse,minimum height=8.5cm,minimum width=14cm,draw,fill=gray!10] (l0100) at (6.8,6.8)
{  };

\draw [-,dotted,line width=0pt,white,
    decoration={text along path,
              text align={center},
              text={|\itshape| OSI approved licenses}},
              postaction={decorate}] (-0.8,6.5) arc (142:38:9.5cm);

\draw [-,dotted,line width=0pt,white,
    decoration={text along path,
              text align={center},
              text={|\itshape|Open Source Licenses}},
              postaction={decorate}] (-0.8,6.5) arc (218:322:9.5cm);
              
\node[ellipse,minimum height=6.2cm,minimum width=4cm,draw,fill=gray!20] (l0100) at (2.75,6.8)
{  };

\draw [-,dotted,line width=0pt,white,
    decoration={text along path,
              text align={center},
              text={|\itshape| permissive licenses}},
              postaction={decorate}] (0.9,7.4) arc (180:0:1.8cm);

\node[circle,draw,text width=1cm, fill=gray!40, text centered] (l0101) at (2,8)
{  \footnotesize \bfseries \textit{APL}};
\node[circle,draw,text width=1cm, fill=gray!40, text centered] (l0102) at (3.5,8)
{  \footnotesize \bfseries \textit{BSD}};
\node[circle,draw,text width=1cm, fill=gray!40, text centered] (l0103) at (2,6.5)
{  \footnotesize \bfseries \textit{MIT}};
\node[circle,draw,text width=1cm, fill=gray!40, text centered] (l0104) at (3.5,6.5)
{  \footnotesize \bfseries \textit{MsPL}};
\node[circle,draw,text width=1cm, fill=gray!40, text centered] (l0105) at (2,5)
{  \footnotesize \bfseries \textit{PgL}};
\node[circle,draw,text width=1cm, fill=gray!40, text centered] (l0106) at (3.5,5)
{  \footnotesize \bfseries \textit{PHP}};

\node[ellipse,minimum height=6cm,minimum width=8.5cm,draw,fill=gray!20] (l0200) at (9.2,6.5)
{  };

\draw [-,dotted,line width=0pt,white,
    decoration={text along path,
              text align={center},
              text={|\itshape| copyleft licenses}},
              postaction={decorate}] (7.5,8.5) arc (120:60:4cm);


\node[ellipse,minimum height=4.5cm,minimum width=4.2cm,draw,fill=gray!30] (l0210) at (7.45,6.5)
{  };

\draw [-,dotted,line width=0pt,white,
    decoration={text along path,
              text align={center},
              text={|\itshape| weak copyleft licenses}},
              postaction={decorate}] (5.4,6.2) arc (180:0:2cm);

\node[circle,draw,text width=1cm, fill=gray!40, text centered] (l0211) at (6.7,7)
{  \footnotesize \bfseries \textit{EPL}};
\node[circle,draw,text width=1cm, fill=gray!40, text centered] (l0212) at (8.2,7)
{  \footnotesize \bfseries \textit{EUPL}};
\node[circle,draw,text width=1cm, fill=gray!40, text centered] (l0213) at (6.7,5.5)
{  \footnotesize \bfseries \textit{LGPL}};
\node[circle,draw,text width=1cm, fill=gray!40, text centered] (l0214) at (8.2,5.5)
{  \footnotesize \bfseries \textit{MPL}};

\node[ellipse,minimum height=4.5cm,minimum width=3cm,draw,fill=gray!30] (l0220) at (11.4,6.5)
{  };
 
% line width=0pt,white,
\draw [-,dotted,line width=0pt,white,
    decoration={text along path,
              text align={center},
              text={|\itshape| strong copyleft}},
              postaction={decorate}] (10.4,7) arc (180:0:1cm);

\draw [-,dotted,line width=0pt,white,
    decoration={text along path,
              text align={center},
              text={|\itshape| licenses}},
              postaction={decorate}] (10.4,5.4) arc (180:360:1cm);        

\node[circle,draw,text width=1cm, fill=gray!40, text centered] (l0221) at (11.4,7)
{  \footnotesize \bfseries \textit{GPL}};
\node[circle,draw,text width=1cm, fill=gray!40, text centered] (l0222) at (11.4,5.5)
{  \footnotesize \bfseries \textit{AGPL}};


\end{tikzpicture}
\end{center}

This extensionally based clarification of a possible Open Source License
taxonomy is probably well known and often - more or less explicitly -
referred\footnote{Even the FSF itself uses the term 'permissive non-copyleft
free software license' (\cite[pars pro toto: cf.][\nopage wp/section 'Original BSD
license']{FsfLicenseList2013a}) and contrasts it with the terms 'weak copyleft'
and 'strong copyleft' (\cite[pars pro toto: cf.][\nopage wp/section 'European
Union Public License']{FsfLicenseList2013a})}. Unfortunately, this taxonomy
still contains some misleading underlying messages:

\emph{Permissive} is a very positively connoted word. So, the antinomy of
\emph{permissive licenses} versus \emph{copyleft licenses} implicitly signals,
that the \emph{permissive licenses} are in any meaning better, than the
\emph{copyleft licenses}. Naturally, this 'conclusion' is evoked by
confusing the extensionally definition and the intensional power of the labels.
But, that's the way we - the human beings - like to think. 

Anyway, this underlying message is not necessarily 'wrong'. It might be
convenient for those people or companies, who only want to use Open Source
software without being restricted by the \emph{giving back obligation} as it has
been introduced by the 'copyleft'. But there might be other people and companies
which emphasize the protecting effect of the copyleft licenses. And indeed, at
least the Open Source license\footnote{Although RMS naturally prefers to specify
it as a \emph{Free Software License} (s. p. \pageref{RmsFsPriority}) }
\emph{GPL}\footnote{As the original source \cite[cf.][\nopage
wp]{Gpl20FsfLicense1991a}. Inside of the OSLiC, we constantly refer to the
license versions which are published by the OSI, because we are dealing with
officially approved Open Source Licenses. For the 'OSI-GPL' \cite[cf.][\nopage
wp]{Gpl20OsiLicense1991a}} has initially been generated to protect the freedom,
to enable the developers to help their \enquote{neighbours} and to get the
modifications back\footnote{The history of the GNU project is multiply told. For
the GNU project and its' initiator \cite[cf. pars pro toto][\nopage
passim]{Williams2002a}. For a broader survey \cite[cf. pars pro toto][\nopage
passim]{Moody2001a}. A very short version is delivered by Richard M.
Stallman himself where he states, that - in the years while the early free
community were destroyed - he saw the \enquote{nondisclosure agreement}, which
must be signed , \enquote{[\ldots] even to get an executable copy} as a clear
\enquote{[\ldots] promise not to help your neighbour}: \enquote{A cooperating
community was forbidden.} (\cite[cf.][16]{Stallman1999a}).}: So,
\enquote{Copyleft} is defined as a \enquote{[\ldots] method for making a
programm free software and requiring all modified and extended versions of the
program to be free software as well}\footcite[cf.][89]{Stallman1996c}. It is a
method\footnote{Based on the American legal copyright system, this method uses
two steps: firstly one states, \enquote{[\ldots] that it is copyrighted
[\ldots]} and secondly one adds those \enquote{[\ldots] distribution terms,
which are a legal instrument that gives everyone the rights to use, modify, and
redistribute the program's code or any program derived from it but only if the
distribution terms are unchanged} (\cite[cf.][89]{Stallman1996c}).} by which
\enquote{[\ldots] the code and the freedoms become legally
inseparable}\footcite[cf.][89]{Stallman1996c}. Because of these disparate
interests of hoping not to be restricted and hoping to be protected, it could be
helpful to find a better, because impartial label for the cluster of
\emph{permisive licenses}. But up to that time, we should at least know, that
this taxonomy still contains an underlying declassing message.

The other misleading interpretation is - counter-intuitively - evoked by using
the concept 'copyleft licenses'. If one refers to a cluster of \emph{copyleft
licenses} as the opposite of the \emph{permissive licenses}, one implicitly also
sends two kind of messages: Firstly, that republishing one's own modifications
is sufficient to fulfill the \emph{copyleft licenses}. And secondly, that the
\emph{permissive licenses} do not require anything which has to be done for
getting the right to use the software. Even if one does not wish to evoke such
an interpretation, we - the human beings - tend to take the things as simple as
possible\footnote{And indeed, it's the experience of the authors that -
sometimes - on the management level, such simplifications gain their independent
existence and determine decisions. But that's not the fault of the managers.
It's their task, to aggregate, generalize and simplify information. It's the
task of the experts, to offer better viewpoints without overwhelming the others
with details.}. But because of several aspects, this understanding of the
antinomy of \emph{copyleft licenses} and \emph{permissive licenses} is to
misleading for taking it as a serious generalization:

On the hand, even the 'strongly copylefted' GPL requires also other license
fulfilling tasks than only republishing derivative works. For example, it
additionally demands to \enquote{[\ldots] give any other recipients of the [GPL
licensed] Program a copy of this License along with the
Program}\footcite[cf.][\nopage wp §1]{Gpl20OsiLicense1991a}. Furthermore, the
'weakly copylefted' licenses require also more and different criteria, which has
to be fulfilled for acting according these license. For example, the EUPL
requires, that the licensor who does not directly deliver the binaries together
with the sourcecode, must offer a sourcecode version of his work free of
charge\footnote{The German version of the EUPL uses the phrase
\enquote{problemlos und unentgeltlich(sic!) auf den Quellcode (zugreifen
können)} (\cite[cf.][3, section 3]{EuplLicense2007de}) while the English version
contains the specification \enquote{the Source Code is easily and freely
accessible} (\cite[cf.][2, section 3]{EuplLicense2007en})}, while the MPL
requires, that under the same circumstances a recipient \enquote{[\ldots] can
obtain a copy of such Source Code Form [\ldots] at a charge no more than the
cost of distribution to the recipient [\ldots]}\footcite[cf.][\nopage section
3.2.a]{Mpl20OsiLicense2013a}. And last but not least, also the \emph{permissive
licenses} require tasks which must be fulfilled for a license compliant usage -
moreover, they also require different things. For example, the BSD demands that
\enquote{the (r)edistributions [\ldots] must (retain [and/or]) reproduce the
above copyright notice [\ldots]}. Because of the structure of the
\enquote{copyright notice}, this required announcement implies that the authors
/ copyright holders of the software must be publicly named\footcite[cf.][\nopage
wp]{BsdLicense2Clause}. As opposed to this, the Apache License requires, that
\enquote{if the Work includes a "NOTICE" text file as part of its distribution,
then any Derivative Works that You distribute must include a readable copy of
the attribution notices contained within such NOTICE file} what often means that
you have to present central parts of such file publicly\footcite[cf.][\nopage
wp. section 4.4]{Apl20OsiLicense2004a} . parts, which can contain many more
information than only the names of the authors / copyright holders.

So, no doubt - and against the intuitive interpretation of this taxonomy - each
\emph{Open Source License} must be fulfilled by some actions, even the most
permissive. And for ascertain these tasks, one has to review these licenses
themselves, not the generalized concepts of licenses taxonomies. Hence again, we
have to state that even this well known kind of grouping \emph{Open Source
Licenses} does not allow to conclude to the license compliant behavior: The
taxonomy might be appropiate, if one wants to live with the implicite messages
and generalizations of some of its concepts. But the taxonomy is not adequate
tool to determine, what one has to do for fulfilling an \emph{Open Source
License}. A license compliant behaviour for getting the right to use a specific
piece of \emph{Open Source Software} must refer to the concrete \emph{Open
Source License} by which the licensor has licensed the software. There doesn't
exist any shortcut.

Nevertheless, human beings need generalizing and structuring viewpoints for
enabling themselves to talk about a domain - even if they finally have to regard
the single objects of the domain for specific purposes. We think, that there is
a subtler method to regard and to structure the domain of \emph{Open Source
Licenses}. So, we want to offer this other possibility to cluster the \emph{Open
Source licenses}\footnote{even if finally also we have to concede that at the
end one has to look into the license itself.}:

We think, that in general, licenses have a common purpose: they should protect
someone or something against something. The structure of this tasks is based on
the nature of the word 'protect', which is a 3 valent verb: it links someone or
something, who protects, to someone or something, who is protected and both
together to something against the protector protects and against the otherone is
protected. Licenses in general do so. Therefore, it's also the purpose of Open
Source Licenses to protect: They can protect the user (receiver) of the
software, its' contributor resp. developer and/or distributor, and the software
itself. And they can protect them against different threats. With respect to
this viewpoint, we should specify the \emph{Open Source Licenses} in a specific,
purpose orientied way:

\section{The protecting power of Apache License (APL)}
\begin{itemize} 
  \item The Apache License protects \ldots
  \item But the Apache License does not protect \ldots
\end{itemize}

\section{The protecting power of BSD licenses}
\begin{itemize}
  \item As approved \emph{Open Source Licenses}\footcite[cf.][\nopage
  wp]{OSI2012b}, the BSD Licenses\footnote{BSD has to be resolved as
  \emph{Berkely Software Distribution}. For details of the BSD license release
  and namings \cite[cf.][\nopage wp. editorial]{BsdLicense3Clause}} protect the
  user against the loss of the right to use, to modify and/or to distribute the
  received copy of the source code or the binary\footcite[cf.][\nopage wp
  §1ff]{OSI2012a}. Additionally, they protect the contributors and/or
  distributors against warranty claims of the software users, because these
  licenses contain a 'No Warranty Clause'\footcite[one for all version
  cf.][\nopage wp]{BsdLicense2Clause}. And finally they protect the distributed
  sources against a change of the license which closes the sources, because each
  modification and \enquote{redistributions of [the] source code must retain the
  [\ldots] copyright notice, this list of conditions and the [\ldots]
  disclaimer}\footcite[cf.][\nopage wp]{BsdLicense2Clause}: Therefore it is
  uncorrect to distribute a BSD licensed code under another license - regardles,
  wether it closes the sources or not\footnote{In common sense based discussions
  you may have heard that BSD licenses allow to republish the work under
  another, an own license. Taking the words of the BSD License seriously, that's
  not valid under all circumstances: Yes, it is true, you are not required to 
  redistribute the sourcecode of a modified (derivative) work. You are allowed 
  to modify a received version and to distribute the results only as binary code 
  and to keep your improvements closed. But if you distribute the source code of 
  your modifications, you have retain the licensing, because 
  \enquote{Redistribution [\ldots] in source [\ldots], with or without 
  modification, are permitted provided that [\ldots] (the) redistributions of 
  source code [\ldots] retain the above copyright notice, this list of 
  conditions and the following disclaimer} 
  (\cite[cf.][\nopage wp]{BsdLicense2Clause}}.
  
  \item But the BSD Licenses protect neither the users nor the contributors
  and/or distributors against patent disputes (because they do not contain any
  patent clause). They do not protect the contributors against the loss of
  feedback (because they do not 'copyleft' the software). And they do not
  protect the undistributed software or the distributed binaries against
  re-closings, neither in unmodified nor in modified form (because they allow to
  redistribute only the binaries)\footnote{see both, the BSD-2CL License
  (\cite[cf.][\nopage wp]{BsdLicense2Clause}), and the BSD-3CL License
  (\cite[cf.][\nopage wp]{BsdLicense3Clause})}
  
\end{itemize}

\section{The protecting power of the MIT license}
\begin{itemize}
  \item As an approved \emph{Open Source Licenses}\footcite[cf.][\nopage
  wp]{OSI2012b}, the MIT License\footcite[MIT has to be resolved as
  \enquote{Massachusetts Institute of Technology} 
  (cf.][\nopage wp).]{wpMitLic2011a} protects the user against the loss of the
  right to use, to modify and/or to distribute the received copy of the source
  code or the binary\footcite[cf.][\nopage wp 1ff]{OSI2012a}. Additionally, it
  protects the contributors and/or distributors against warranty claims of the
  software users, because it contains a 'No Warranty
  Clause'\footcite[cf.][\nopage wp]{MitLicense2012a}. And finally it protects
  the distributed sources against a change of the license which would close the
  sources, because the \enquote{permission [\ldots] to use, copy, modify,
  [\ldots] distribute, [\ldots] (is granted) subject to the [\ldots] conditions,
  [that] the [\ldots] copyright notice and this permission notice shall be
  included in all copies or substantial portions of the
  Software}\footnote{\cite[cf.][\nopage wp]{MitLicense2012a}. The argumentation
  why the source code is protected, but not the binary form follows that of the
  BSD licenses: By these requirements, one is not obliged to redistribute the
  sourcecode of a modified (derivative) work. One is allowed to modify a
  received version and to distribute the results only in binary form and to keep
  one's improvements closed. But if one distribute the source code of the
  modifications, the licensing is retained, simply because the MIT
  \enquote{[\ldots] permission note shall be included in all copies or
  substantial portions of the software}.}
 
  \item But the MIT License protect neither the users nor the contributors
  and/or distributors against patent disputes (because it does not contain any
  patent clause). It does not protect the contributors against the loss of
  feedback (because it does not 'copyleft' the software). And it does do not
  protect the undistributed software or the distributed binaries against
  re-closings, neither in unmodified nor in modified form (because it allows to
  redistribute only the binaries)\footcite[cf.][\nopage wp]{MitLicense2012a}
  
\end{itemize}

\section{The protecting power of the Microsoft Public License (MsPL)}
\begin{itemize} 
  \item The Microsoft Public License protects \ldots
  \item But the Microsoft Public License does not protect \ldots
\end{itemize}

\section{The protecting power of the Postgres License (PgL)}
\begin{itemize} 
  \item The Postgres License protects \ldots
  \item But the Postgres License does not protect \ldots
\end{itemize}

\section{The protecting power of the PHP License}
\begin{itemize} 
  \item The PHP License protects \ldots
  \item But the PHP License does not protect \ldots
\end{itemize}  
 
\ldots

All these specifications can also be covered by a table:

\begin{table}
\footnotesize
\caption{Open Source Licenses as Protectors}
\begin{center}

\begin{tabular}{|c|c|c|c|c|c|c|c|c|c|c|c|c|c|c|c|}
\hline
  \multicolumn{2}{|c|}{\textit{Open}} &
  \multicolumn{12}{c|}{\textit{are protecting}}\\
\cline{3-14}
  \multicolumn{2}{|c|}{\textit{Source}} &
  \multicolumn{4}{c|}{ \textbf{Users}} &
  \multicolumn{3}{c|}{\textbf{Contributors}} &
  \multicolumn{5}{c|}{\textbf{Software}} \\
\cline{10-14}
  \multicolumn{2}{|c|}{\textit{Licenses}} &
  \multicolumn{4}{c|}{} &
  \multicolumn{3}{c|}{\tiny{(Distributors)}} &  
  not &
  \multicolumn{4}{c|}{distributed as} \\
\cline{3-9}\cline{11-14}
  \multicolumn{2}{|c|}{} &
  \multicolumn{4}{c|}{\scriptsize{\textit{who have already got}}} &
  \multicolumn{3}{c|}{\scriptsize{\textit{who spread Open}}} & 
  distri- &
  \multicolumn{2}{c|}{modified} &
  \multicolumn{2}{c|}{unmodified} \\
  \cline{11-14}
  \multicolumn{2}{|c|}{} &
  \multicolumn{4}{c|}{\scriptsize{\textit{sources or binaries}}} &
  \multicolumn{3}{c|}{\scriptsize{\textit{Source software}}} & 
  buted & 
 \footnotesize{sources} &
 \footnotesize{binaries} &
 \footnotesize{sources} &
 \footnotesize{binaries} \\
\cline{3-14}
  \multicolumn{2}{|c|}{} &
  \multicolumn{12}{c|}{\textit{against}}\\
\cline{3-14}
  \multicolumn{2}{|c|}{} &
  \multicolumn{3}{c|}{the loss of} & 
  \multirow{3}{*}{\rotatebox{270}{Patent Disputes}} &
  \multirow{3}{*}{\rotatebox{270}{Loss of Feedback}} & 
  \multirow{3}{*}{\rotatebox{270}{Warranty Claims}} & 
  \multirow{3}{*}{\rotatebox{270}{Patent Disputes}} & 
  \multicolumn{5}{c|}{}\\
% no seperator line 
  \multicolumn{2}{|c|}{} &
  \multicolumn{3}{c|}{the right to} &
  & & & &
  \multicolumn{5}{c|}{\footnotesize{Re-Closings of}}\\
\cline{3-5}
  \multicolumn{2}{|c|}{} & 
  \rotatebox{270}{use it} & 
  \rotatebox{270}{modify it} & 
  \rotatebox{270}{redistribute it} &
  &  &  &  &
  \multicolumn{5}{c|}{already Opened Software}\\
\hline
\hline
  APL & 2.0 & \checkmark  & \checkmark  & \checkmark & 
  - & - & - & - & - & - & - & - & - \\
\hline
  \multirow{2}{*}{BSD}\footnotemark & 3-Cl & \checkmark & \checkmark  & \checkmark  & 
    $\neg$ & $\neg$ & \checkmark & $\neg$  &
    $\neg$ & \checkmark  & $\neg$ & \checkmark & $\neg$ \\
\cline{2-14}
   & 2-Cl & \checkmark  & \checkmark  & \checkmark  & 
    $\neg$ & $\neg$ & \checkmark & $\neg$  &
    $\neg$ & \checkmark  & $\neg$ & \checkmark & $\neg$ \\
\hline
  MIT\footnotemark & ~ & \checkmark  & \checkmark  & \checkmark  &
  - & - & - & - & - & - & - & - & - \\
\hline
  MsPL & ~ & \checkmark  & \checkmark  & \checkmark  & 
  - & - & - & - & - & - & - & - & - \\
\hline
  PgL & ~ & \checkmark  & \checkmark  & \checkmark  &
  - & - & - & - & - & - & - & - & - \\
\hline
  PHP & 3.0 & \checkmark  & \checkmark  & \checkmark  &
  - & - & - & - & - & - & - & - & - \\
\hline
\hline
  EPL & 1.0 & \checkmark & \checkmark & \checkmark &
  - & - & - & - & - & - & - & - & - \\
\hline
  EUPL & 1.1 & \checkmark & \checkmark & \checkmark &
  - & - & - & - & - & - & - & - & - \\
\hline
  \multirow{2}{*}{LGPL} & 2.1 & \checkmark & \checkmark & \checkmark &
  - & - & - & - & - & - & - & - & - \\
\cline{2-14}
   & 3.0 & \checkmark & \checkmark & \checkmark &
   - & - & - & - & - & - & - & - & - \\
\hline
  \multirow{2}{*}{MPL} & 1.1 & \checkmark & \checkmark & \checkmark &
  - & - & - & - & - & - & - & - & - \\
\cline{2-14}
  & 2.0 & \checkmark & \checkmark & \checkmark &
  - & - & - & - & - & - & - & - & - \\
\hline
\hline
  \multirow{2}{*}{AGPL} & 2.1 & \checkmark & \checkmark & \checkmark &
   - & - & - & - & - & - & - & - & - \\
\cline{2-14}
   & 3.0 & \checkmark & \checkmark & \checkmark &
    - & - & - & - & - & - & - & - & - \\
\hline
  \multirow{2}{*}{GPL} & 2.1 & \checkmark & \checkmark & \checkmark &
   - & - & - & - & - & - & - & - & - \\
\cline{2-14}
  & 3.0 & \checkmark & \checkmark & \checkmark &
   - & - & - & - & - & - & - & - & - \\
\hline
\hline

\end{tabular}
\end{center}
\end{table}
\addtocounter{footnote}{-1}
\footnotetext{Berkeley Software Distribution [License]}
\stepcounter{footnote}
\footnotetext{Massuchat Institute of Technique [License]}

And the content of this classifying table can also be transfered into a mindmap:

\begin{tikzpicture}
\label{LICTAX}
\footnotesize

% (1.A) list of all licenses and their release numbers Level 5/6
\node[rectangle,draw,text width=1.4cm] (l0100) at (10,0.5)
{ \textit{BSD License} };
\node[text width=1.4cm] (l0101) at (12,0)
{ \scriptsize{3-Clauses} };
\node[text width=1.4cm] (l0102) at (12,1)
{ \scriptsize{2-Clauses} };
  
\node[rectangle,draw,text width=1.4cm] (l0200) at (11,2)
{ \textit{MIT License} }; 
\node[text width=0.4cm] (l0201) at (12.5,2) {\scriptsize{1.1}};
  
\node[rectangle,draw,text width=1.4cm] (l0300) at (12.5,3)
{ \textit{\textbf{AP}ache \textbf{L}icense}};
   \node[text width=0.4cm] (l0301) at (14,3) {\scriptsize{2.1}};

\node[rectangle,draw,text width=1.4cm] (l0400) at (13,4.5)
{ \textit{\textbf{M}icro\textbf{s}oft \textbf{P}ublic \textbf{L}icense} };
\node[text width=0.4cm] (l0401) at (14.5,4){\scriptsize{1.0}};
\node[text width=0.4cm,style=dotted] (l0402) at (14.5,5){\scriptsize{\textit{1.1}}};
  
\node[rectangle,draw,text width=1.4cm] (l0500) at (13,6)
{\textit{\textbf{P}ost\textbf{g}res \textbf{L}icense}};
\node[text width=0.4cm] (l0501) at (14.5,6){ \scriptsize{1.1}};
  
\node[rectangle,draw,text width=1.4cm] (l0600) at (13,7)
{\textit{PHP License}};
\node[text width=0.4cm] (l0601) at (14.5,7){\scriptsize{1.1}};
  
\node[rectangle,draw,text width=1.4cm] (l0700) at (13,8)
{ \textit{XXX License}};
\node[text width=0.4cm] (l0701) at (14.5,8){\scriptsize{1.1}};

\node[rectangle,draw,text width=1.4cm] (l0800) at (13,9.5)
{ \textit{\textbf{M}ozilla \textbf{P}ublic \textbf{L}icense}};
\node[text width=0.4cm] (l0801) at (14.5,9){\scriptsize{1.1}};
\node[text width=0.4cm] (l0802) at (14.5,10){\scriptsize{2.0}};

\node[rectangle,draw,text width=1.4cm] (l0900) at (13,11.5)
{\textit{\textbf{E}clipse \textbf{P}ublic \textbf{L}icense}};
\node[text width=0.4cm] (l0901) at (14.5,11) {\scriptsize{1.0}};
\node[text width=0.4cm,style=dotted] (l0902) at (14.5,12){\scriptsize{\textit{1.1}}};
 
\node[rectangle,draw,text width=1.5cm] (l1000) at (13,13.5)
{\textit{\textbf{E}uropean \textbf{P}ublic \textbf{L}icense}}; 
\node[text width=0.4cm] (l1001) at (14.5,13){\scriptsize{1.0}};
\node[text width=0.4cm,style=dotted] (l1002) at (14.5,14){\scriptsize{\textit{1.1}}};
  
\node[rectangle,draw,text width=1.4cm] (l1100) at (13,15.5)
{\textit{\textbf{L}esser \textbf{G}NU \textbf{P}ublic \textbf{L}icense}};

\node[text width=0.4cm] (l1101) at (14.5,15){\scriptsize{2.1}};
\node[text width=0.4cm] (l1102) at (14.5,16){\scriptsize{3.0} };

\node[rectangle,draw,text width=1.4cm] (l1200) at (13,17.5)
{\textit{\textbf{G}NU \textbf{P}ublic \textbf{L}icense}};

\node[text width=0.4cm] (l1201) at (14.5,17){\scriptsize{2.1}};
\node[text width=0.4cm] (l1202) at (14.5,18){\scriptsize{3.0} };

\node[rectangle,draw,text width=1.4cm] (l1300) at (13,19.5)
{ \textit{\textbf{A}ffero \textbf{G}NU \textbf{P}ublic \textbf{L}icense}};
\node[text width=0.4cm, style=dotted] (l1301) at (14.5,19){\scriptsize{2.1}};
\node[text width=0.4cm] (l1302) at (14.5,20){\scriptsize{3.0}};

% 2. the clustering concepts of licenses (level 4)
\node[rectangle,draw,text width=2.3cm] (n0100) at (10,8)
 { \textit{protecting the user, the con\-tri\-butor \& the initial code}\\
   \tiny{Permissive Licenses}      
 };

\node[rectangle,draw,text width=2.3cm] (n0200) at (10,11.5)
{ \textit{protecting the user, the con\-tri\-butor, the
  initial code, \& all di\-rect de\-ri\-va\-tions}\\
  \tiny{Weak Copyleft}        
};

\node[rectangle,draw,text width=2.3cm] (n0300) at (10,16.5)
{ \textit{protecting the user, the con\-tri\-bu\-tor, the 
  initial code, all di\-rect de\-ri\-va\-tions \& the 
  (in\-di\-rect\-ly de\-ri\-ved) on-top-deve\-lop\-ments}\\ 
  \tiny{Strong Copyleft}    
 };

% 3. the threats (level 3)
\node[ellipse,draw,text width=1.6cm] (c110000) at (4.5,0)
{ \textbf{\textit{Patent Disputes}}};

\node[ellipse,draw,text width=1.6cm] (c120000) at (4.5,2)
{ \textbf{\textit{Loss of Rights}} };

\node[ellipse,draw,text width=1.6cm] (c210000) at (4.5,4)
{ \textbf{\textit{Warranty Claims}} };
 
\node[ellipse,draw,text width=1.6cm] (c220000) at (4.5,6)
{ \textbf{\textit{Loss of Feeback}}};

\node[ellipse,draw,text width=0.4cm] (c311000) at (6.2,8)
{ \tiny{\textit{\textbf{clos\-ings}}}};

\node[ellipse,,draw,text width=0.4cm] (c321000) at (6.2,10)
{ \tiny{\textit{\textbf{clos\-ings}}} };

\node[ellipse,,draw,text width=0.4cm] (c331000) at (6.2,12)
{ \tiny{\textit{\textbf{clos\-ings}}} };

\node[ellipse,,draw,text width=0.4cm] (c341000) at (6.2,14)
{ \tiny{\textit{\textbf{clos\-ings}}} };

\node[ellipse,,draw,text width=0.4cm] (c351000) at (6.2,16)
{ \tiny{\textit{\textbf{clos\-ings}}} };

\node[ellipse,,draw,text width=1.6cm] (c411000) at (6.2,19)
{ \textit{\textbf{clos\-ings}} };


% 4. the subtypes of protected entities (level 2)
\node[ellipse,draw,text width=1.5cm] (c310000) at (3,8)
 { \scriptsize{un\-modified} \textbf{Sources}};

\node[ellipse,draw,text width=1.5cm] (c320000) at (3.25,10)
 { \scriptsize{un\-modified} \textbf{Binaries}};

\node[ellipse,draw,text width=1.2cm] (c330000) at (3.5,12)
 { \scriptsize{modified} \textbf{Sources}};

\node[ellipse,draw,text width=1.4cm] (c340000) at (3.25,14)
 { \scriptsize{modified} \textbf{Binaries}};

\node[ellipse,draw,text width=1.4cm] (c350000) at (3,16)
 { \scriptsize{part of \textbf{On-Top-Develop\-ments}}};


% 5. the protected entities (level 1)
\node[ellipse,draw,text width=1cm] (c100000) at (1,1)
 { \textbf{Users} };

\node[ellipse,draw,text width=0.8cm] (c200000) at (1,5)
 { \textbf{Con\-tribu\-tors}};

\node[ellipse,draw,text width=0.8cm] (c300000) at (1,12)
 { distri\-buted \textbf{Soft\-ware}};
 
\node[ellipse,draw,text width=2.2cm] (c400000) at (1,19)
 { un\-distri\-buted \textbf{Soft\-ware}}; 

% 6. main node (leve 0)
\node[ellipse,draw,text width=1.3cm] (c000000) at (0,8)
{ \textbf{Open Source License}};

% a linking Licenses to their release numbers (Linking level 5 to 6)
\foreach \father/\daughter in {
  l0100/l0101/,
  l0100/l0102/,
  l0200/l0201/,   
  l0300/l0301/,
  l0400/l0401/,
  l0400/l0402/,
  l0500/l0501/,
  l0600/l0601/,
  l0700/l0701/,
  l0800/l0801/,
  l0800/l0802/,
  l0900/l0901/,
  l0900/l0902/,
  l1000/l1001/,
  l1000/l1002/,
  l1100/l1101/,
  l1100/l1102/,
  l1200/l1201/,
  l1200/l1202/,
  l1300/l1301/,
  l1300/l1302/
  }
  \draw[dashed] (\father) to  (\daughter) ;

% b) linking Licenses to license concepts (Linking level 5 to 4)
\foreach \father/\daughter/\outangle/\inangle in {
  n0100/l0100/270/150,       
  n0100/l0200/280/155,
  n0100/l0300/290/160,
  n0100/l0400/300/165,
  n0100/l0500/310/150,
  n0100/l0600/340/160,
  n0100/l0700/360/180,
  n0200/l0800/300/160,
  n0200/l0900/340/170,
  n0200/l1000/20/190,
  n0200/l1100/60/200,
  n0300/l1200/40/180,
  n0300/l1300/80/180 
  }
  %\draw[dashed] (\father) to [out=\outangle,in=\inangle] (\daughter) ;
  \draw[dashed] (\father) to  (\daughter) ;

% c) linking license concepts to the threats against they protect
% c.1) strong copyleft licenses
\foreach \father/\daughter/\outangle/\inangle in {
  c351000/n0300/0/180,
  c341000/n0300/45/190,
  c331000/n0300/50/200,
  c321000/n0300/55/210,
  c311000/n0300/60/220,
  c220000/n0300/25/225,
  c210000/n0300/25/230,
  c120000/n0300/25/235
  }
  \draw[<-,color=blue] (\father) to [out=\outangle,in=\inangle] (\daughter) ;
% c.2) weak copyleft licenses
\foreach \father/\daughter/\outangle/\inangle in {
  c341000/n0200/330/170,
  c331000/n0200/0/180,
  c321000/n0200/0/180,
  c311000/n0200/20/190,
  c220000/n0200/15/220,
  c210000/n0200/15/230,
  c120000/n0200/15/235
  }
  \draw[<-,color=cyan] (\father) to [out=\outangle,in=\inangle] (\daughter) ;
% c.3) permissive licenses
\foreach \father/\daughter/\outangle/\inangle in {
  c331000/n0100/355/150,
  c311000/n0100/0/180,
  c210000/n0100/5/210,
  c120000/n0100/10/230
  }
  \draw[<-,color=red] (\father) to [out=\outangle,in=\inangle] (\daughter) ;
%c.4 agpl license
\foreach \father/\daughter/\outangle/\inangle in {
  c411000/l1300/0/180    
}
  \draw[<-,color=green] (\father) to [out=\outangle,in=\inangle] (\daughter) ;


%d linking protected entities, their subtypes and the the relations
\foreach \father/\daughter/\edgetext/\outangle/\inangle in {
  c000000/c100000/protecting/260/120,
  c100000/c110000/against/360/180,
  c100000/c120000/against/360/180,
  c000000/c200000/protecting/270/180,
  c200000/c210000/against/0/180,
  c200000/c220000/against/0/180,
  c000000/c300000/protecting/90/230,
  c300000/c310000/as/300/180,
  c300000/c320000/as/330/180,
  c300000/c330000/as/0/180,
  c300000/c340000/as/30/180,
  c300000/c350000/as/60/180,
  c000000/c400000/protecting/100/240,
  c400000/c411000/against/0/180        
}
  \draw[->,dotted,
    decoration={text along path,
              text align={center},
              text={|\itshape|\edgetext}},
              postaction={decorate},] (\father) to [out=\outangle,in=\inangle] (\daughter) ;

\foreach \father/\daughter/\edgetext/\outangle/\inangle in {
  c310000/c311000/against/0/180,
  c320000/c321000/against/0/180,
  c330000/c331000/against/0/180,
  c340000/c341000/against/00/180,
  c350000/c351000/against/0/180        
}
  \draw[->,dotted,
    decoration={text along path,
              text align={center},
              text={|\itshape \tiny|\edgetext}},
              postaction={decorate},] (\father) to [out=\outangle,in=\inangle] (\daughter) ;

%f linking the patent clauses
\foreach \father/\daughter/\outangle/\inangle in {
  c110000/l1302/0/295,
  c110000/l0401/0/360,
  c110000/l0301/0/360   
}
  \draw[<-,color=gray] (\father) to [out=\outangle,in=\inangle] (\daughter) ;

\end{tikzpicture}


Finally, one could generate new groups of Open Source license, new classes, like
'user protecting licenses'\footnote{all of them because all of them have to
fulfill the OSD}, 'patent disputes abwehrende licenses', 'modified \ldots'.

But one has to know: all of these grouping viewpoints do not allow to conclude,
that all members of a group can be respected by fulfilling the same
requirements. This would only be possible, if the grouping criteria would
directly refer to the fulfilling tasks. And indeed, nearly all Open Source
licenses do differ with respect to these criteria, even if the differences are
very small, they can't be neglected\footnote{Pars pro toto: Both, the BSD
license and the Apache license require, that you give hint to the developers of
the application. But in case of the BSD license you xýou have to \ldots. In case
of the Apache license you have exactly to present the content of the notice file
distributed together with the application.}. So: reflecting on possible classes
of Open Source licenses is a got method to become familiar with the area of Open
Source licenses. But it is not a method to determine, what one has to do for
getting the right to use the software. For getting these information, one has to
consider each single license.


%\bibliography{../../../bibfiles/oscResourcesEn}


%%%%%%%%%%%%%%%
\chapter{Open Source: Important Minor Points}
% Telekom osCompendium 'for beeing included' snippet template
%
% (c) Karsten Reincke, Deutsche Telekom AG, Darmstadt 2011
%
% This LaTeX-File is licensed under the Creative Commons Attribution-ShareAlike
% 3.0 Germany License (http://creativecommons.org/licenses/by-sa/3.0/de/): Feel
% free 'to share (to copy, distribute and transmit)' or 'to remix (to adapt)'
% it, if you '... distribute the resulting work under the same or similar
% license to this one' and if you respect how 'you must attribute the work in
% the manner specified by the author ...':
%
% In an internet based reuse please link the reused parts to www.telekom.com and
% mention the original authors and Deutsche Telekom AG in a suitable manner. In
% a paper-like reuse please insert a short hint to www.telekom.com and to the
% original authors and Deutsche Telekom AG into your preface. For normal
% quotations please use the scientific standard to cite.
%
% [ File structure derived from 'mind your Scholar Research Framework' 
%   mycsrf (c) K. Reincke CC BY 3.0  http://mycsrf.fodina.de/ ]
%

% Chapter Abstract
% ----------------

\footnotesize
\begin{quote}\itshape
This chapter we shortly discusses some minor but although important issues.
\end{quote}
\normalsize{}


% Telekom osCompendium 'for beeing included' snippet template
%
% (c) Karsten Reincke, Deutsche Telekom AG, Darmstadt 2011
%
% This LaTeX-File is licensed under the Creative Commons Attribution-ShareAlike
% 3.0 Germany License (http://creativecommons.org/licenses/by-sa/3.0/de/): Feel
% free 'to share (to copy, distribute and transmit)' or 'to remix (to adapt)'
% it, if you '... distribute the resulting work under the same or similar
% license to this one' and if you respect how 'you must attribute the work in
% the manner specified by the author ...':
%
% In an internet based reuse please link the reused parts to www.telekom.com and
% mention the original authors and Deutsche Telekom AG in a suitable manner. In
% a paper-like reuse please insert a short hint to www.telekom.com and to the
% original authors and Deutsche Telekom AG into your preface. For normal
% quotations please use the scientific standard to cite.
%
% [ Framework derived from 'mind your Scholar Research Framework' 
%   mycsrf (c) K. Reincke 2012 CC BY 3.0  http://mycsrf.fodina.de/ ]
%


%% use all entries of the bibliography
%\nocite{*}

[TDB \ldots]

%\bibliography{../../../bibfiles/oscResourcesEn}

% Telekom osCompendium 'for beeing included' snippet template
%
% (c) Karsten Reincke, Deutsche Telekom AG, Darmstadt 2011
%
% This LaTeX-File is licensed under the Creative Commons Attribution-ShareAlike
% 3.0 Germany License (http://creativecommons.org/licenses/by-sa/3.0/de/): Feel
% free 'to share (to copy, distribute and transmit)' or 'to remix (to adapt)'
% it, if you '... distribute the resulting work under the same or similar
% license to this one' and if you respect how 'you must attribute the work in
% the manner specified by the author ...':
%
% In an internet based reuse please link the reused parts to www.telekom.com and
% mention the original authors and Deutsche Telekom AG in a suitable manner. In
% a paper-like reuse please insert a short hint to www.telekom.com and to the
% original authors and Deutsche Telekom AG into your preface. For normal
% quotations please use the scientific standard to cite.
%
% [ Framework derived from 'mind your Scholar Research Framework' 
%   mycsrf (c) K. Reincke 2012 CC BY 3.0  http://mycsrf.fodina.de/ ]
%


%% use all entries of the bibliography
%\nocite{*}

\section{Excursion: The Problem of Derivated Works}
\footnotesize
\begin{quote}\itshape
We will shortly discuss exsting attempts to define the derivated works of
technical aspects, like dynamical or statical linking or not. We will
prove that linking can not deliver a definite criteria: 1) modules are only
unzipped libraries. 2) you can distribute software as modules added by a script,
which statically(sic!) links all modules before executing the program. 3) The
criteria of pipe-communication is good, but not sufficient. 4) All these
attempts do not match the constituting features of script languages. Therefore we
will follow Moglen(?) and will argue from the viewpoint of a developer: it's
only a question of a function, method or anything else which calls (jumps into)
a piece of code which has been licenced by a license protecting
on-top-developements and you have a derivated work.
\end{quote}
\normalsize
\ldots


%\bibliography{../../../bibfiles/oscResourcesEn}

% Telekom osCompendium 'for beeing included' snippet template
%
% (c) Karsten Reincke, Deutsche Telekom AG, Darmstadt 2011
%
% This LaTeX-File is licensed under the Creative Commons Attribution-ShareAlike
% 3.0 Germany License (http://creativecommons.org/licenses/by-sa/3.0/de/): Feel
% free 'to share (to copy, distribute and transmit)' or 'to remix (to adapt)'
% it, if you '... distribute the resulting work under the same or similar
% license to this one' and if you respect how 'you must attribute the work in
% the manner specified by the author ...':
%
% In an internet based reuse please link the reused parts to www.telekom.com and
% mention the original authors and Deutsche Telekom AG in a suitable manner. In
% a paper-like reuse please insert a short hint to www.telekom.com and to the
% original authors and Deutsche Telekom AG into your preface. For normal
% quotations please use the scientific standard to cite.
%
% [ Framework derived from 'mind your Scholar Research Framework' 
%   mycsrf (c) K. Reincke 2012 CC BY 3.0  http://mycsrf.fodina.de/ ]
%


%% use all entries of the bibliography
%\nocite{*}

\section{Excursion: The Problem of Combining Different Licensemodels}
\footnotesize
\begin{quote}\itshape
Here we discuss the often neglected or only loosely touched problem of combining
differently licensed software. We will hint to the Exclusion-List of the Free
software foundation; we will hint to the eclipse / GPL-plugin problem; we will
mention the recent discussion whether the kernel requires to license the
complete Android as GPL; and finally we will discuss the just now published, short
analysis of Jaeger and Metzger presenting a combining matrix which seems to fall
into their lap. We ourselves will argue that question can simply be answered:
only if you embed two libraries which both are licensed by an on
on-top-development protecting license and if these both license require the
licensing of the derivated work by different licenses then you have a problem.
In all other case which we will describe an list there is no problem.
\end{quote}
\normalsize
\ldots

%\bibliography{../../../bibfiles/oscResourcesEn}

% Telekom osCompendium 'for beeing included' snippet template
%
% (c) Karsten Reincke, Deutsche Telekom AG, Darmstadt 2011
%
% This LaTeX-File is licensed under the Creative Commons Attribution-ShareAlike
% 3.0 Germany License (http://creativecommons.org/licenses/by-sa/3.0/de/): Feel
% free 'to share (to copy, distribute and transmit)' or 'to remix (to adapt)'
% it, if you '... distribute the resulting work under the same or similar
% license to this one' and if you respect how 'you must attribute the work in
% the manner specified by the author ...':
%
% In an internet based reuse please link the reused parts to www.telekom.com and
% mention the original authors and Deutsche Telekom AG in a suitable manner. In
% a paper-like reuse please insert a short hint to www.telekom.com and to the
% original authors and Deutsche Telekom AG into your preface. For normal
% quotations please use the scientific standard to cite.
%
% [ Framework derived from 'mind your Scholar Research Framework' 
%   mycsrf (c) K. Reincke 2012 CC BY 3.0  http://mycsrf.fodina.de/ ]
%


%% use all entries of the bibliography
%\nocite{*}

\section{Excursion: Open Source Software and Money}
\footnotesize
\begin{quote}\itshape
Here we will shortly discuss ways in which money and Open Source is no problem.
\end{quote}
\normalsize
\ldots


%\bibliography{../../../bibfiles/oscResourcesEn}

% Telekom osCompendium 'for beeing included' snippet template
%
% (c) Karsten Reincke, Deutsche Telekom AG, Darmstadt 2011
%
% This LaTeX-File is licensed under the Creative Commons Attribution-ShareAlike
% 3.0 Germany License (http://creativecommons.org/licenses/by-sa/3.0/de/): Feel
% free 'to share (to copy, distribute and transmit)' or 'to remix (to adapt)'
% it, if you '... distribute the resulting work under the same or similar
% license to this one' and if you respect how 'you must attribute the work in
% the manner specified by the author ...':
%
% In an internet based reuse please link the reused parts to www.telekom.com and
% mention the original authors and Deutsche Telekom AG in a suitable manner. In
% a paper-like reuse please insert a short hint to www.telekom.com and to the
% original authors and Deutsche Telekom AG into your preface. For normal
% quotations please use the scientific standard to cite.
%
% [ Framework derived from 'mind your Scholar Research Framework' 
%   mycsrf (c) K. Reincke 2012 CC BY 3.0  http://mycsrf.fodina.de/ ]
%


%% use all entries of the bibliography
%\nocite{*}

\section{Excursion: The Problem of Implicitly Freeing Patents}
\footnotesize
\begin{quote}\itshape
Here we analyze the implicitly freeing of patents by licensing code as Open
Source Software.
\end{quote}
\normalsize
\ldots

%\bibliography{../../../bibfiles/oscResourcesEn}


%%%%%%%%%%%%%%%
\chapter{Open Source Use Cases: Concept and Taxonomy}
% Telekom osCompendium 'for beeing included' snippet template
%
% (c) Karsten Reincke, Deutsche Telekom AG, Darmstadt 2011
%
% This LaTeX-File is licensed under the Creative Commons Attribution-ShareAlike
% 3.0 Germany License (http://creativecommons.org/licenses/by-sa/3.0/de/): Feel
% free 'to share (to copy, distribute and transmit)' or 'to remix (to adapt)'
% it, if you '... distribute the resulting work under the same or similar
% license to this one' and if you respect how 'you must attribute the work in
% the manner specified by the author ...':
%
% In an internet based reuse please link the reused parts to www.telekom.com and
% mention the original authors and Deutsche Telekom AG in a suitable manner. In
% a paper-like reuse please insert a short hint to www.telekom.com and to the
% original authors and Deutsche Telekom AG into your preface. For normal
% quotations please use the scientific standard to cite.
%
% [ File structure derived from 'mind your Scholar Research Framework' 
%   mycsrf (c) K. Reincke CC BY 3.0  http://mycsrf.fodina.de/ ]
%

% Chapter Abstract
% ----------------

\footnotesize
\begin{quote}\itshape
This chapter establishes the concept of Open Source Use Cases
as a sorting system for to-do lists, which fulfill specific licenses in the
context of such an Open Source Use Case. Additionally it introduces a taxonomy
for these Open Source Use Cases whose structure later on will organize the Open
Sourse Use Case Finder.
\end{quote}
\normalsize{}


% Telekom osCompendium 'for being included' snippet template
%
% (c) Karsten Reincke, Deutsche Telekom AG, Darmstadt 2011
%
% This LaTeX-File is licensed under the Creative Commons Attribution-ShareAlike
% 3.0 Germany License (http://creativecommons.org/licenses/by-sa/3.0/de/): Feel
% free 'to share (to copy, distribute and transmit)' or 'to remix (to adapt)'
% it, if you '... distribute the resulting work under the same or similar
% license to this one' and if you respect how 'you must attribute the work in
% the manner specified by the author ...':
%
% In an internet based reuse please link the reused parts to www.telekom.com and
% mention the original authors and Deutsche Telekom AG in a suitable manner. In
% a paper-like reuse please insert a short hint to www.telekom.com and to the
% original authors and Deutsche Telekom AG into your preface. For normal
% quotations please use the scientific standard to cite.
%
% [ Framework derived from 'mind your Scholar Research Framework' 
%   mycsrf (c) K. Reincke 2012 CC BY 3.0  http://mycsrf.fodina.de/ ]
%


%% use all entries of the bibliography
%\nocite{*}

%[todo: \ldots (insert all contexts concerning the idea of use cases)]

After all these introductory remarks, we can summarize our idea. We know that
the right to use Open Source Software depends on the tasks required by the Open
Source Licenses. As opposed to commercial licenses, you can not buy the right to
use a piece of Open Source Software using money. It is embedded into the Open
Source Definition that the right to use the software may not be sold. The OSD
states firstly that an Open Source License may \enquote{[\ldots] not restrict any
party from selling or giving away the software as a component of (any) aggregate
software distribution}, and adds secondly in the same context that an Open
Source License \enquote{[\ldots] shall not require a royalty or other fee for such
sale}\footcite[cf.][\nopage wp. §1]{OSI2012a}.

However, it would be wrong to conclude that you are automatically allowed to use
Open Source Software without any service in return: generally you have to do
something to gain the right to use the software. In other words: Open Source
Software is covered by the idea of ’paying by doing’. As such, Open Source
Li\-cen\-ses describe specific circumstances under which the user must execute
some tasks in order to be compliant with the licenses. So, if we want to offer
to-do lists for fulfilling license conditions, we must consider these tasks and
circumstances.

In practice, such circumstances are not linear and simple. They contain
combinations of (sometimes context sensitive) conditions, which can be grouped
together into classes of tokens. Such a class of tokens might denote a feature
of the software itself - such as being an application or a library. Or it can
refer to the circumstances of using the software, such as 'using the software
only for yourself' or 'distributing the software also to third parties'.

At the end, we want to determine a set of specific OSUCs - the \emph{Open Source
Use Cases}. And we want to deliver for each of these OSUCs and for each of the
considered Open Source Licenses one list of actions, which fulfills the license
in that context\footnote{Fortunately, sometimes one task list fulfills the
conditions of more than one use case - a welcome reduction of complexity}.

Such an \emph{Open Source Use Case} shall be considered as a set of tokens
describing the circumstances of a specific usage. Hence, to begin, we must
specify the relevant classes of tokens, before we can determine the valid
combinations of these tokens - our \emph{Open Source Use Cases}. Finally, based
on the tokens, we generate a taxonomy in form of a tree. This tree will become
the base of the \emph{Open Source Use Case Finder} which will be offered by the
next chapter, and which leads you to your specific OSUC by evaluating just a few
questions and answers.

There are only a handful of tokens which are relevant to the circumstances of
Open Source Software Licenses:

\label{OsucTokens}
\begin{itemize}
  \item The \textbf{\underline{type} of the Open Source Software}: On one hand
  there are code snippets, modules, libraries and plugins, and on the other
  hand, autonomous applications, programs and servers. We’ll coin the word
  ’snimolis’ for the first set, and ’proapses’ for the second. This is
  necessary, as we are not only talking about libraries and applications in the
  everyday sense, but rather in the broadest sense\footnote{Of course, our newly
  introduced concepts 'snimoli' and 'proapse' are not absolutely one of the most
  elegant words. So, initially we tried to talk about 'applications' and
  'libraries', although in our context these words should denote more, than they
  traditionally do. But we couldn't minimize the irritations of our
  interlocutors. Too often we had to amend that we were not only talking about
  applications and libraries in the strict sense of the words. Finally we
  decided to find our own words - and to stay open for better proposals ;-) }.
  More specifically, we will ask you, whether the Open Source Software, you
  want to use, is an includable code snippet, a linkable module or library, or a
  loadable plugin, or whether it is an autonomous application or server which
  can be executed or processed. In the first case, the answer should be 'it's a
  \underline{snimoli}', in the second 'it's a \underline{proapse}'.

  \item The \textbf{\underline{state} of the of the Open Source Software}: It
  might be used, as one has got it. Or it can be modified, before being used.
  More specifically, we will ask you, whether you want to leave the Open Source
  Software as you have got it, or whether you want to modify it before using
  and/or distributing it to 3rd parties. In the first case, the answer should be
  '\underline{unmodified}', in the second '\underline{modified}'.
  
  \item The \textbf{usage \underline{context} of the of the Open Source
  Software}: On the hand you might use the received Open Source Software as a
  readily prepared application. On the other hand you might embed the received
  Open Source into a larger application as one of its' components. More
  specifically, we will ask you, whether you are you using the Open Source
  Software as an autonomous piece of software, or whether you are using it as an
  embedded part of a larger, more complex piece of software. In the first case,
  the answer should be '\underline{independent}', in the second
  '\underline{embedded}'.
  
  \item The \textbf{\underline{recipient} of the of the Open Source Software}:
  Sometimes you might wish to use the received Open Source Software only for
  yourself. In other cases you might intend to hand over the software (also) to
  other people. More specifically, we will ask you, whether you are going to use
  the Open Source Software only for yourself, or whether you plan to
  (re)distribute it (also) to third parties. In the first case, the answer
  should be '\underline{4yourself}', in the second '\underline{4others}'.
 
  \item The \textbf{\underline{mode} of combination}: In this case, we will ask
  you, whether you are going to combine or to embed the Open Source Software
  with other software components by linking them statically, by linking them
  dynamically, or by textually including (parts of) the Open Source Software
  into your larger product. In the first case, the answer should be
  '\underline{statically linked}', in the second '\underline{dynamically
  linked}', in the third '\underline{textually included}'
  
\end{itemize}

From a more programmatic point-of-view, we can summarize these tokens as
follows:

\begin{itemize}
  \item \texttt{type::snimoli} \emph{or} \texttt{type::proapse}
  \item \texttt{state::unmodified} \emph{or} \texttt{state::modified}
  \item \texttt{context::independent} \emph{or} \texttt{context::embedded}
  \item \texttt{recipient::4yourself} \emph{or} \texttt{ecipient::4others}
  \item \texttt{mode::statically-linked} \emph{or} \texttt{mode::dynamically-linked}
   \emph{or} \\ \texttt{mode::textually-included}
\end{itemize}

We already defined an Open Source Use Case as a combination of these tokens. If
we simply combine all these tokens of all these classes with all the tokens of
the other classes\footnote{in the sense of the cross product TYPE $\times$ STATE
$\times$ CONTEXT $\times$ RECIPIENT $\times$ MODE}, we get 2*2*2*2*3 = 48 sets
of tokens - or 48 \emph{Open Source Use Cases}. Fortunately, some of the
generated sets are invalid from an empirical or logical view, and some of these
sets are context sensitive:
\label{InvalidFinderTokenCombinations}

\begin{enumerate}
  
  \item It makes no sense to ask you
  whether you are going to combine the received software with other software
  components by linking them statically or dynamically, or by including it
  textually into a larger unit, if you already have answered, that the received
  Open Source Software is a \emph{proapse} or that it shall be used
  \emph{independently}: A readily prepared application or server can't be linked
  to another application or server, which also contains a
  \texttt{main}-function. And using a \emph{proapse} or \emph{snimoli}
  \emph{independently} includes that it is used \emph{not in combination} with
  other units, simply because they are tokens of the same class.
  
  \item If you already have specified that the used Open Source Software is a
  \emph{proapse} - hence an autonomous program, an application, or a server -,
  then your answer includes, that the software is used independently and is not
  embedded with other components into a larger unit - simply because of the
  nature of all \emph{proapses}. But if you have specified that the used Open
  Source Software is a \emph{snimoli} - hence a snippet of code, a module, a
  plugin, or a library -, then it can indeed be used as an embedded component of
  a constructed larger application or server, or it can be used independently in
  case you 'only' re-distribute it to 3rd parties.
  
  \item If you already have specified that the used Open Source Software is a
  \emph{snimoli} - hence a snippet of code, a module, a plugin, or a library -,
  and that this \emph{snimoli} shall be used only by yourself (not distributed
  to other 3rd. parties), then your answer must also imply that this
  \emph{snimoli} is used in combination, as an embedded part of a larger unit.
  It makes no sense to 'try' to use a library autonomously, without using it
  as component of another application. In this case, it would simply sit on the
  disk and would do nothing more than occupying space.

\end{enumerate}

Does this sound complex? We thought so too. We spent much
time explaining these constraints to ourselves, and only when we had transposed
all the combinations and rules into a tree, did the situation become clearer.
The following diagrams shall summarize this way of clarification:

%\bibliography{../../../bibfiles/oscResourcesEn}

% Telekom osCompendium 'for beeing included' snippet template
%
% (c) Karsten Reincke, Deutsche Telekom AG, Darmstadt 2011
%
% This LaTeX-File is licensed under the Creative Commons Attribution-ShareAlike
% 3.0 Germany License (http://creativecommons.org/licenses/by-sa/3.0/de/): Feel
% free 'to share (to copy, distribute and transmit)' or 'to remix (to adapt)'
% it, if you '... distribute the resulting work under the same or similar
% license to this one' and if you respect how 'you must attribute the work in
% the manner specified by the author ...':
%
% In an internet based reuse please link the reused parts to www.telekom.com and
% mention the original authors and Deutsche Telekom AG in a suitable manner. In
% a paper-like reuse please insert a short hint to www.telekom.com and to the
% original authors and Deutsche Telekom AG into your preface. For normal
% quotations please use the scientific standard to cite.
%
% [ Framework derived from 'mind your Scholar Research Framework' 
%   mycsrf (c) K. Reincke 2012 CC BY 3.0  http://mycsrf.fodina.de/ ]
%


%% use all entries of the bibliography
%\nocite{*}

After all these introducing remarks let us summarize our idea: We know that the
right to use Open Source Software depends on doings required by the Open Source
Licenses. In opposite to the commercial licenses you can not buy the right to
use a piece of Open Source software - never. It's part of the Open Source
Definition that the right to use the software may not be sold. The Open Source
Definition states firstly that an Open Source License may \glqq{}\ldots not
restrict any party from selling or giving away the software as a component of (any)
aggregate software distribution\grqq{} and adds secondly that an Open Source
License \glqq{}\ldots shall not require a royalty or other fee for such
sale\grqq{}\footcite[cf.][\nopage wp. §1]{OSI2012a}.

On the other hand it is wrong to simply conclude that you are allowed to use
Open Source Software without any service in return: generally you have to do
something for getting the right to use the software. In other words: Open Source
Software is specified by the idea of 'paying by doing'.

Therefore these Open Source Licenses describe specific circumstances under
which the user must execute some tasks. These set of conditions may be viewed as
triggers for a compliant behaviour. So, if we want to offer license fulfilling
to-do lists, we have to consider these triggers. 

The real challenge is, that such circumstances are not linear and simple. They
contain combinations of (sometimes) context sensitive conditions. These
conditions refer to different class of tokens. Such a class of token might refer
to a feature of the software itself - like being an application or a library. Or
such class of tokens might refer to the circumstances of using the software -
like 'using the software only for yourself' or 'distributing the software also
to third parties'.

In the end we want to determine a set of specific use cases and deliver for each
of these use cases and for each of the considered Open Source Licenses one list
of actions, which fulfill the license in the context of this use case. A use
case is a set of tokens describing the circumstances of the usage. So, in the
beginning we have to specify the relevant classes of tokens. Then we build the
valid combinations of tokens, our use cases. And finally - based on these
specifying tokens - we generate a taxonomy of our use cases for being able to
offer an easily to use reference to the relevant use cases and the
corresponding, the license fulfilling to-do list. 

So, let's now start with the classes of tokens by which the circumstances of
using a piece of Open Source Software can be specified:

\begin{itemize}
  \item Firstly we will consider the type of the licensed software. We will
  discriminate between code snippets, modules, libraries and plugins on the one
  hand and autonomous, processable applications, programs, servers on the other
  hand. Let's name the first set the 'snipmolibs' and the second the
  'progappsers' for indicating, that we are not only talking about libraries and
  applications in the strict, sense. More detailedly spoken, we will ask you,
  whether the Open Source Software, you want to use, is a software library in
  the broadest sense (an includable code snippet, a linkable module or library,
  or a loadable plugin) or whether it is an autonomous application or server
  which can be executed or processed. In the first case, the answer should be
  'it's a snipmolib', in the second 'it's a progappser'.
  \item Secondly we will consider the state of the software: it might be used,
  as one has got it, or one can modify it, before using it. More detailedly
  spoken, we will ask you whether you want to leave the evaluated Open Source
  Software as you have got it or whether you want to modify it before using
  and/or distributing it to 3rd parties? In the firstcase, the answer should be
  'unmodified', in the second 'modified'.
  \item Thirdly we will consider the 
\end{itemize}




 Based on this information we can no derive and define the use cases by
which the OSLiC classifies its license fulfilling to-do lists:

Firstly we have to discriminate the usage of Open Source software by the nature
of the software itself:

On the one side you can use an application intended to support the work of an
end-user. It takes input data and generates output data, mostly by using a more
or less elaborated end-user interface. On the other side you can use a computer
librabry intended to support the work of a software developer. It mostly offers
functions and/or objects and is embedded into an overarching work like an
application.

The use of an application is different from the use of a software library
because the use of a software library implies the act of developing a new
overarching piece of software.

Secondly we have to discriminate the usage of Open Source software by the
addressee:

On the one side you can intend to use the software directly for your own
purpose. On the other side you can intend to distribute the software to a third
party.

And thirdly we have to discriminate the usage of Open Source software by nature of
the usage itself:

One the one side you can intend to use the software as it is, respectively as
you got it. On the other side you can intend to modify the software before using
it.

Let's form the corresponding dichotomies

use-it-as-you-got-it <> modify-it
use-it-for-yourself <> distribute-it
application <> library

Now we have 8 possibilities to combine these attributes:

Use an application as you got it only for yourself
Use a library as you got it only for your self
Distribute an application as you got it to a third party
Distribute a library as you got it to a third party




%\bibliography{../../../bibfiles/oscResourcesEn}


%%%%%%%%%%%%%%%
\chapter{Open Source Use Cases: Find the License Fulfilling To-do Lists}
% Telekom osCompendium 'for being included' snippet template
%
% (c) Karsten Reincke, Deutsche Telekom AG, Darmstadt 2011
%
% This LaTeX-File is licensed under the Creative Commons Attribution-ShareAlike
% 3.0 Germany License (http://creativecommons.org/licenses/by-sa/3.0/de/): Feel
% free 'to share (to copy, distribute and transmit)' or 'to remix (to adapt)'
% it, if you '... distribute the resulting work under the same or similar
% license to this one' and if you respect how 'you must attribute the work in
% the manner specified by the author ...':
%
% In an internet based reuse please link the reused parts to www.telekom.com and
% mention the original authors and Deutsche Telekom AG in a suitable manner. In
% a paper-like reuse please insert a short hint to www.telekom.com and to the
% original authors and Deutsche Telekom AG into your preface. For normal
% quotations please use the scientific standard to cite.
%
% [ File structure derived from 'mind your Scholar Research Framework' 
%   mycsrf (c) K. Reincke CC BY 3.0  http://mycsrf.fodina.de/ ]
%

% Chapter Abstract
% ----------------

\footnotesize
\begin{quote}\itshape
This chapter offers the \emph{Open Source Use Case Finder}: Firstly, it presents
a form to gather the specifying information. Secondly it offers a tree, which
can easily be transversed by the gathered information. And finally it contains
the list of \emph{Open Source Use Cases}. Each leaf of the tree leads to one
\emph{Open Source Use Case} which itself then refers to the specific license
fulfilling to-do lists.
\end{quote}
\normalsize{}


% Telekom osCompendium 'for being included' snippet template
%
% (c) Karsten Reincke, Deutsche Telekom AG, Darmstadt 2011
%
% This LaTeX-File is licensed under the Creative Commons Attribution-ShareAlike
% 3.0 Germany License (http://creativecommons.org/licenses/by-sa/3.0/de/): Feel
% free 'to share (to copy, distribute and transmit)' or 'to remix (to adapt)'
% it, if you '... distribute the resulting work under the same or similar
% license to this one' and if you respect how 'you must attribute the work in
% the manner specified by the author ...':
%
% In an internet based reuse please link the reused parts to www.telekom.com and
% mention the original authors and Deutsche Telekom AG in a suitable manner. In
% a paper-like reuse please insert a short hint to www.telekom.com and to the
% original authors and Deutsche Telekom AG into your preface. For normal
% quotations please use the scientific standard to cite.
%
% [ Framework derived from 'mind your Scholar Research Framework' 
%   mycsrf (c) K. Reincke 2012 CC BY 3.0  http://mycsrf.fodina.de/ ]
%


%% use all entries of the bibliography
%\nocite{*}

\section{A standard form for gathering the relevant information}
 
\begin{small}
\begin{tabular}[h]{|l|l|l|l|}
\hline 
Class & Questions & Answers\\
\hline 
  Type
  & \parbox[c][2.6cm][c]{9.4cm}{
    \textit{Is the Open Source Software you want to use a software library
    in the broadest sense (an includable code snippet, a linkable module or
    library, or a loadable plugin) [=snimoli], or is it an autonomous
    program, application or server which can be executed or processed
    [=proapse]?}} & \parbox{10em}{ 
      $\square$\hspace{1em}proapse\\ 
      $\square$\hspace{1em}snimoli}
    \\
\hline 
  State & 
  \parbox[c][1.6cm][c]{9.4cm}{
  \textit{Do you want to leave your Open Source Software as you have
  got it, or do you want to modify it before using and/or distributing it to 3rd
  parties?}} &
  \parbox{10em}{
    $\square$\hspace{1em}unmodified\\
    $\square$\hspace{1em}modified} \\
\hline 
  Context & 
  \parbox[c][2cm][c]{9.4cm}{
  \textit{Are you using your Open Source Software as an au\-to\-no\-mous piece
  of software [=independent], or are you using it as an embedded part or component
  of a larger, more complex piece of software [=embedded]?}} &
  \parbox{10em}{ $\square$\hspace{1em}independent\\
    $\square$\hspace{1em}embedded}\\
\hline 
  Recipient & 
  \parbox[c][1.6cm][c]{9.4cm}{
  \textit{Are you are going to use the received Open Source Software only for
  yourself [=4yourself], or do you plan to (re)distribute it (also) to third
  parties [=4others]?}}
  & \parbox{10em}{
    $\square$\hspace{1em}4yourself\\
    $\square$\hspace{1em}4others}\\
\hline 
  Mode & 
  \parbox[c][2.6cm][c]{9.4cm}{
  \textit{Are you going to combine the received Open Source Software with other
  software components by linking all together statically, by linking them
  dynamically, or by textually including (parts of) the Open Source Software
  into your larger unit?}} &
  \parbox{10em}{
    $\square$\hspace{1em}statically linked\\   
    $\square$\hspace{1em}dynamically linked\\
    $\square$\hspace{1em}textually included}\\
\hline 
\hline
\end{tabular}
\end{small}

As discussed earlier, there are of course some invalid
combinations\footnote{type::proapse excludes state::embedded;
recipient::4yourself excludes the combination with state::independent and
type::snimoli; any value of class 'mode' implies state::embedded [for details
see page \pageref{InvalidFinderTokenCombinations}]. If you have gathered one of
these invalid combinations, please check the corresponding explanations}. Here
are some extra explanations about each class:

\begin{description}
\item[Type:] A piece of (Open Source) Software shall be viewed as a program, an
application, or a server, if you can start its binary form with your normal
program launcher, or (in case of an text file which still must be interpreted by
an interpreter like php, perl, bash etc.) if you can start an interpreter taking
the file as one of its' arguments. \item[State:] You modify Open Source Software
if you expand, reduce or modify at least one of the received software files, and
- in case of dealing with binary object code - if you (re)compile and (re)link
the modified software to a new binary file. If you only modify configuration
files, you do not modify the Open Source Software.
\item[Context:] You use Open Source Software embedded into a larger unit, if one
of your files of the larger unit contains a verbatim or modified copy (i.e. a
snippet) of the received Open Source Software, or if the larger unit contains an
include statement referring to a file of the received Open Source Software, or
if your development environment contains a compiler or linker directive
referring to the received Open Source Software.
\item[Recipient] You use the received Open Source Software only for yourself, if
you as person do not pass it to other persons, or if you as a member of a
specific development group pass it only to the other members of your
development group. But if you store Open Source Software on any device such as a
mobile phone, an USB stick, etc. or if you attach it to any transport
medium like email etc. and if you then sell, give away, or simply send this
device or transport medium to anyone (other than a direct member
of your development group), then you indeed handover the Open Source Software to
third parties\footnote{Please remember that - at least in Germany - there are
opinions that even handing over software to another legal entity or department
of the same company is also a kind of distribution. It is always safest
to take the broadest possible meaning of distributing or handing over.}.
\item[Mode] This follows the standard terminology of software development.
this question.
\end{description}

\section{The taxonomic Open Source Use Case Finder}

Now, after having gathered the necessary information, determine your specific
Open Source Use Case by traversing the following tree and its corresponding
branches:

\begin{footnotesize}

\pstree[treemode=R,levelsep=*0.2,treesep=0.2]{\Toval{OSS}}{
  \pstree{
    \Tr{\Ovalbox{\shortstack{type:\\\textbf{\textit{proapse}}\\
      \tiny (= Program,\\\tiny Application,\\\tiny Server)      
      }   
    }}
    
  }{
    \pstree{
      \Tr{\Ovalbox{\shortstack{state:\\\textbf{\textit{unmodified}}}}}
    }{
      \pstree{
        \Tr{\Ovalbox{\shortstack{context:\\\textbf{\textit{independent}}}}}
      }{
        
        \pstree{
          \Tr{\Ovalbox{\shortstack{recipient:\\\textbf{\textit{4yourself}}}}}
        }{
          \Tr[edge=none]{\begin{minipage}[b][2em][c]{6em} 
              $\Rightarrow$ OSUC-01\\
              \textit{(see p. \pageref{OSUC-01-DEF})}\end{minipage} } 
          }
        
        \pstree{
          \Tr{\Ovalbox{\shortstack{recipient:\\\textbf{\textit{4others}}}}}
        }{
          \Tr[edge=none]{\begin{minipage}[b][2em][c]{6em} 
              $\Rightarrow$ OSUC-02\\
              \textit{(see p. \pageref{OSUC-02-DEF})}\end{minipage} } 
          }        
      }
    }
    \pstree{
      \Tr{\Ovalbox{\shortstack{state:\\\textbf{\textit{modified}}}}}
    }{
      \pstree{
        \Tr{\Ovalbox{\shortstack{context:\\\textbf{\textit{independent}}}}}
      }{
        \pstree{
          \Tr{\Ovalbox{\shortstack{recipient:\\\textbf{\textit{4yourself}}}}}
        }{
          \Tr[edge=none]{\begin{minipage}[b][2em][c]{6em} 
              $\Rightarrow$ OSUC-03\\
              \textit{(see p. \pageref{OSUC-03-DEF})}\end{minipage} } 
          }
        
        \pstree{
          \Tr{\Ovalbox{\shortstack{recipient:\\\textbf{\textit{4others}}}}}
        }{
          \Tr[edge=none]{\begin{minipage}[b][2em][c]{6em} 
              $\Rightarrow$ OSUC-04\\
              \textit{(see p. \pageref{OSUC-04-DEF})}\end{minipage} } 
          }        
      }
    }
  }
  \pstree{
    \Tr{\Ovalbox{
      \shortstack{type:\\\textbf{\textit{snimoli}}\\
      \tiny (= Snippet,\\\tiny Module,\\\tiny Plugin,\\\tiny Library)            
      }
    }}
  }{
    \pstree{
      \Tr{\Ovalbox{\shortstack{state:\\\textbf{\textit{unmodified}}}}}
    }{
      \pstree{
        \Tr{\Ovalbox{\shortstack{context:\\\textbf{\textit{independent}}}}}
      }{
      
        \pstree{
          \Tr{\Ovalbox{\shortstack{recipient:\\\textbf{\textit{4others}}}}}
        }{
          \Tr[edge=none]{\begin{minipage}[b][2em][c]{6em} 
              $\Rightarrow$ OSUC-05\\
              \textit{(see p. \pageref{OSUC-05-DEF})}\end{minipage} } 
          }        
      
      }
      \pstree{
        \Tr{\Ovalbox{\shortstack{context:\\\textbf{\textit{embedded}}}}}
      }{
        \pstree{
          \Tr{
            \Ovalbox{\shortstack{recipient:\\\textbf{\textit{4yourself}}}}
           }
        }{
          \pstree{
            \Tr{
              \begin{tiny}
              \Ovalbox{\shortstack{mode:\\\textit{statically linked}}}
              \end{tiny}
            }
          }{
            \Tr[edge=none]{\begin{minipage}[b][2em][c]{7em} 
              $\Rightarrow$ OSUC-06a\\
              \textit{(see p. \pageref{OSUC-06-DEF})}\end{minipage} } 
          }        
        
          \pstree{
            \Tr{
              \begin{tiny}
              \Ovalbox{\shortstack{mode:\\\textit{dynamically linked}}}
              \end{tiny}
            }
          }{
            \Tr[edge=none]{\begin{minipage}[b][2em][c]{7em} 
              $\Rightarrow$ OSUC-06b\\
              \textit{(see p. \pageref{OSUC-06-DEF})}\end{minipage} } 
          }        
 
          \pstree{
            \Tr{
              \begin{tiny}
              \Ovalbox{\shortstack{mode:\\\textit{textually included}}}
              \end{tiny}
            }
          }{
            \Tr[edge=none]{\begin{minipage}[b][2em][c]{7em} 
              $\Rightarrow$ OSUC-06c\\
              \textit{(see p. \pageref{OSUC-06-DEF})}\end{minipage} } 
          }        
        
        }
        \pstree{
          \Tr{\Ovalbox{\shortstack{recipient:\\\textbf{\textit{4others}}}}}
        }{
          \pstree{
            \Tr{
              \begin{tiny}
              \Ovalbox{\shortstack{mode:\\\textit{statically linked}}}
              \end{tiny}
            }
          }{
            \Tr[edge=none]{\begin{minipage}[b][2em][c]{7em} 
              $\Rightarrow$ OSUC-07a\\
              \textit{(see p. \pageref{OSUC-07-DEF})}\end{minipage} } 
          }        
        
          \pstree{
            \Tr{
              \begin{tiny}
              \Ovalbox{\shortstack{mode:\\\textit{dynamically linked}}}
              \end{tiny}
            }
          }{
            \Tr[edge=none]{\begin{minipage}[b][2em][c]{7em} 
              $\Rightarrow$ OSUC-07b\\
              \textit{(see p. \pageref{OSUC-07-DEF})}\end{minipage} } 
          }        
 
          \pstree{
            \Tr{
              \begin{tiny}
              \Ovalbox{\shortstack{mode:\\\textit{textually included}}}
              \end{tiny}
             }
          }{
            \Tr[edge=none]{\begin{minipage}[b][2em][c]{7em} 
              $\Rightarrow$ OSUC-07c\\
              \textit{(see p. \pageref{OSUC-07-DEF})}\end{minipage} } 
          }        
        }
      }
    }
    \pstree{
      \Tr{\Ovalbox{\shortstack{state:\\\textbf{\textit{modified}}}}}
    }{
      \pstree{
        \Tr{\Ovalbox{\shortstack{context:\\\textbf{\textit{independent}}}}}
      }{
        \pstree{
          \Tr{\Ovalbox{\shortstack{recipient:\\\textbf{\textit{4others}}}}}
        }{
          \Tr[edge=none]{\begin{minipage}[b][2em][c]{6em} 
              $\Rightarrow$ OSUC-08\\
              \textit{(see p. \pageref{OSUC-08-DEF})}\end{minipage} } 
          }        
      }
      \pstree{
        \Tr{\Ovalbox{\shortstack{context:\\\textbf{\textit{embedded}}}}}
      }{
        \pstree{
          \Tr{\Ovalbox{\shortstack{recipient:\\\textbf{\textit{4yourself}}}}}
        }{
          \pstree{
            \Tr{
              \begin{tiny}
              \Ovalbox{\shortstack{mode:\\\textit{statically linked}}}
              \end{tiny}
            }
          }{
            \Tr[edge=none]{\begin{minipage}[b][2em][c]{7em} 
              $\Rightarrow$ OSUC-09a\\
              \textit{(see p. \pageref{OSUC-09-DEF})}\end{minipage} } 
          }        
        
          \pstree{
            \Tr{
              \begin{tiny}
              \Ovalbox{\shortstack{mode:\\\textit{dynamically linked}}}
              \end{tiny}
            }
          }{
            \Tr[edge=none]{\begin{minipage}[b][2em][c]{7em} 
              $\Rightarrow$ OSUC-09b\\
              \textit{(see p. \pageref{OSUC-09-DEF})}\end{minipage} } 
          }        
 
          \pstree{
            \Tr{
              \begin{tiny}
              \Ovalbox{\shortstack{mode:\\\textit{textually included}}}
              \end{tiny}
             }
          }{
            \Tr[edge=none]{\begin{minipage}[b][2em][c]{7em} 
              $\Rightarrow$ OSUC-09c\\
              \textit{(see p. \pageref{OSUC-09-DEF})}\end{minipage} } 
          }        
        }
        \pstree{
          \Tr{\Ovalbox{\shortstack{recipient:\\\textbf{\textit{4others}}}}}
        }{
          \pstree{
            \Tr{
              \begin{tiny}
              \Ovalbox{\shortstack{mode:\\\textit{\tiny statically linked}}}
              \end{tiny}
            }
          }{
            \Tr[edge=none]{\begin{minipage}[b][2em][c]{7em} 
              $\Rightarrow$ OSUC-10a\\
              \textit{(see p. \pageref{OSUC-10-DEF})}\end{minipage} } 
          }        
        
          \pstree{
            \Tr{
              \begin{tiny}
              \Ovalbox{\shortstack{mode:\\\textit{dynamically linked}}}
              \end{tiny}              
            }
          }{
            \Tr[edge=none]{\begin{minipage}[b][2em][c]{7em} 
              $\Rightarrow$ OSUC-10b\\
              \textit{(see p. \pageref{OSUC-10-DEF})}\end{minipage} } 
          }        
 
          \pstree{
            \Tr{
              \begin{tiny}
              \Ovalbox{\shortstack{mode:\\\textit{textually included}}} 
              \end{tiny}
             }
          }{
            \Tr[edge=none]{\begin{minipage}[b][2em][c]{7em} 
              $\Rightarrow$ OSUC-10c\\
              \textit{(see p. \pageref{OSUC-10-DEF})}\end{minipage} } 
          }        
        }
      }
    }
  }
}
\end{footnotesize}



\section{The Open Source Use Cases and their links to the to-do lists}

\begin{description}

\item[OSUC-01:]\label{OSUC-01-DEF}
Only for yourself, you are using an unmodified Open Source program, application,
or server - just as you received it. You are not going to combine it with other
components in the sense of software development (= \textit{proapse, unmodified,
independent, 4yourself}). 
To see the \textit{specific, license fulfilling to-do lists} jump to the
following pages:
  \begin{itemize}
    \item p. \pageref{OSUC-01-Apache20} for license \textbf{Apache 2.0}
    \item p. \pageref{OSUC-01-BSD} for license \textbf{BSD}
    \item p. \pageref{OSUC-01-ECL} for license \textbf{ECL}
    \item p. \pageref{OSUC-01-GPL2X} for license \textbf{GPL 2.x}
    \item p. \pageref{OSUC-01-MIT} for license \textbf{MIT}
  \end{itemize}

\item[OSUC-02:]\label{OSUC-02-DEF} Just as you received it, you are going to
distribute an unmodified Open Source program, application, or server to 3rd
parties. In this act of distribution, you do not combine this program,
application, or server with other software components in the sense of software
development (= \textit{proapse, unmodified, independent, 4others}). 
To see the \textit{specific, license fulfilling to-do lists} jump to the
following pages:
  \begin{itemize}
    \item p. \pageref{OSUC-02-Apache20} for license \textbf{Apache 2.0}
    \item p. \pageref{OSUC-02-BSD} for license \textbf{BSD}
    \item p. \pageref{OSUC-02-ECL} for license \textbf{ECL}
    \item p. \pageref{OSUC-02-GPL2X} for license \textbf{GPL}, release 2
    \item p. \pageref{OSUC-02-MIT} for license \textbf{MIT}
  \end{itemize}

\item[OSUC-03:]\label{OSUC-03-DEF} Only for yourself, you are modifying a
received Open Source program, application, or server, before you are using it.
But you do not combine it with other components in the sense of software
development (= \textit{proapse, modified, independent, 4yourself}).
To see the \textit{specific, license fulfilling to-do lists} jump to the
following pages:
  \begin{itemize}
    \item p. \pageref{OSUC-03-Apache20} for license \textbf{Apache 2.0}
    \item p. \pageref{OSUC-03-BSD} for license \textbf{BSD}
    \item p. \pageref{OSUC-03-ECL} for license \textbf{ECL}
    \item p. \pageref{OSUC-03-GPL2X} for license \textbf{GPL}, release 2
    \item p. \pageref{OSUC-03-MIT} for license \textbf{MIT}
  \end{itemize}

\item[OSUC-04:]\label{OSUC-04-DEF} You are going to modify a received Open
Source program, application, or server, before you distribute it to 3rd parties.
But you do not combine this modified program, application, or server with other
software components in the sense of software development (= \textit{proapse,
modified, independent, 4others}).
To see the \textit{specific, license fulfilling to-do lists} jump to the
following pages:
  \begin{itemize}
    \item p. \pageref{OSUC-04-Apache20} for license \textbf{Apache 2.0}
    \item p. \pageref{OSUC-04-BSD} for license \textbf{BSD}
    \item p. \pageref{OSUC-04-ECL} for license \textbf{ECL}
    \item p. \pageref{OSUC-04-GPL2X} for license \textbf{GPL}, release 2
    \item p. \pageref{OSUC-04-MIT} for license \textbf{MIT}
  \end{itemize}

\item[OSUC-05:]\label{OSUC-05-DEF} Just as you received it, you are going to
distribute an unmodified Open Source library, code snippet, module, or plugin to
3rd parties. In this act of distribution, you do not combine this library, code
snippet, module, or plugin with other software components in the sense of
software development (= \textit{snimoli, unmodified, independent, 4others}).
To see the \textit{specific, license fulfilling to-do lists} jump to the
following pages:
  \begin{itemize}
    \item p. \pageref{OSUC-05-Apache20} for license \textbf{Apache 2.0}
    \item p. \pageref{OSUC-05-BSD} for license \textbf{BSD}
    \item p. \pageref{OSUC-05-ECL} for license \textbf{ECL}
    \item p. \pageref{OSUC-05-GPL2X} for license \textbf{GPL}, release 2
    \item p. \pageref{OSUC-05-MIT} for license \textbf{MIT}
  \end{itemize}

\item[OSUC-06:]\label{OSUC-06-DEF} Only for yourself and just as you received
it, you are going to combine an unmodified Open Source library, code snippet,
module, or plugin into a larger software unit as one of its parts. (=
\textit{snimoli, unmodified, embedded, 4yourself}).
To see the \textit{specific, license fulfilling to-do lists} jump to the
following pages:
  \begin{itemize}
    \item p. \pageref{OSUC-06-Apache20} for license \textbf{Apache 2.0}
    \item p. \pageref{OSUC-06-BSD} for license \textbf{BSD}
    \item p. \pageref{OSUC-06-ECL} for license \textbf{ECL}
    \item p. \pageref{OSUC-06-GPL2X} for license \textbf{GPL}, release 2
    \item p. \pageref{OSUC-06-MIT} for license \textbf{MIT}
  \end{itemize}

\item[OSUC-07:]\label{OSUC-07-DEF} Just as you received it and before you will
distribute it to 3rd parties together with the larger software unit, you
combine an unmodified Open Source library, code snippet, module, or plugin into
a larger software unit in the sense of software development (= \textit{snimoli,
unmodified, embedded, 4others}). 
To see the \textit{specific, license fulfilling to-do lists} jump to the
following pages:
  \begin{itemize}
    \item p. \pageref{OSUC-07-Apache20} for license \textbf{Apache 2.0}
    \item p. \pageref{OSUC-07-BSD} for license \textbf{BSD}
    \item p. \pageref{OSUC-07-ECL} for license \textbf{ECL}
    \item p. \pageref{OSUC-07-GPL2X} for license \textbf{GPL}, release 2
    \item p. \pageref{OSUC-07-MIT} for license \textbf{MIT}
  \end{itemize}

\item[OSUC-08:]\label{OSUC-08-DEF} Before you will distribute it, you are going
to modify an Open Source library, code snippet, module, or plugin to 3rd
parties, but you do not combine it with other software components in the sense of
software development (= \textit{snimoli, modified, independent, 4others}). 
To see the \textit{specific, license fulfilling to-do lists} jump to the
following pages:
  \begin{itemize}
    \item p. \pageref{OSUC-08-Apache20} for license \textbf{Apache 2.0}
    \item p. \pageref{OSUC-08-BSD} for license \textbf{BSD}
    \item p. \pageref{OSUC-08-ECL} for license \textbf{ECL}
    \item p. \pageref{OSUC-08-GPL2X} for license \textbf{GPL}, release 2
    \item p. \pageref{OSUC-08-MIT} for license \textbf{MIT}
  \end{itemize}

\item[OSUC-09:]\label{OSUC-09-DEF} Only for yourself, you are going to modify an
Open Source library, code snippet, module, or plugin, before you will combine it
- in the sense of software development - into a larger software unit as one of
its parts . (= \textit{snimoli, modified, embedded, 4yourself}). 
To see the \textit{specific, license fulfilling to-do lists} jump to the
following pages:
  \begin{itemize}
    \item p. \pageref{OSUC-09-Apache20} for license \textbf{Apache 2.0}
    \item p. \pageref{OSUC-09-BSD} for license \textbf{BSD}
    \item p. \pageref{OSUC-09-ECL} for license \textbf{ECL}
    \item p. \pageref{OSUC-09-GPL2X} for license \textbf{GPL}, release 2
    \item p. \pageref{OSUC-09-MIT} for license \textbf{MIT}
  \end{itemize}

\item[OSUC-10:]\label{OSUC-10-DEF} Before you will distribute it to 3rd parties,
you are going to modify an Open Source library, code snippet, module, or plugin,
and to combine it with other software components in the sense of
software development (= \textit{snimoli, modified, independent, 4others}). 
To see the \textit{specific, license fulfilling to-do lists} jump to the
following pages:
  \begin{itemize}
    \item p. \pageref{OSUC-10-Apache20} for license \textbf{Apache 2.0}
    \item p. \pageref{OSUC-10-BSD} for license \textbf{BSD}
    \item p. \pageref{OSUC-10-ECL} for license \textbf{ECL}
    \item p. \pageref{OSUC-10-GPL2X} for license \textbf{GPL}, release 2
    \item p. \pageref{OSUC-10-MIT} for license \textbf{MIT}
  \end{itemize}
\end{description}

%\bibliography{../../../bibfiles/oscResourcesEn}


%%%%%%%%%%%%%%%
\chapter{Open Source License Fulfillment: Classified To-do Lists for \ldots}
% Telekom osCompendium 'for beeing included' snippet template
%
% (c) Karsten Reincke, Deutsche Telekom AG, Darmstadt 2011
%
% This LaTeX-File is licensed under the Creative Commons Attribution-ShareAlike
% 3.0 Germany License (http://creativecommons.org/licenses/by-sa/3.0/de/): Feel
% free 'to share (to copy, distribute and transmit)' or 'to remix (to adapt)'
% it, if you '... distribute the resulting work under the same or similar
% license to this one' and if you respect how 'you must attribute the work in
% the manner specified by the author ...':
%
% In an internet based reuse please link the reused parts to www.telekom.com and
% mention the original authors and Deutsche Telekom AG in a suitable manner. In
% a paper-like reuse please insert a short hint to www.telekom.com and to the
% original authors and Deutsche Telekom AG into your preface. For normal
% quotations please use the scientific standard to cite.
%
% [ File structure derived from 'mind your Scholar Research Framework' 
%   mycsrf (c) K. Reincke CC BY 3.0  http://mycsrf.fodina.de/ ]
%

% Chapter Abstract
% ----------------

\footnotesize
\begin{quote}\itshape
This chapter lists for the main Open Source Licenses and for the main use cases
what one has to do for fulfilling a specific license. And it explains in which
cases the combination with another Open Source License must be avoided. You
should be able to jump into the license specific chapter and find all relevant
information - though without proving details,
\end{quote}
\normalsize{}


% Telekom osCompendium 'for beeing included' snippet template
%
% (c) Karsten Reincke, Deutsche Telekom AG, Darmstadt 2011
%
% This LaTeX-File is licensed under the Creative Commons Attribution-ShareAlike
% 3.0 Germany License (http://creativecommons.org/licenses/by-sa/3.0/de/): Feel
% free 'to share (to copy, distribute and transmit)' or 'to remix (to adapt)'
% it, if you '... distribute the resulting work under the same or similar
% license to this one' and if you respect how 'you must attribute the work in
% the manner specified by the author ...':
%
% In an internet based reuse please link the reused parts to www.telekom.com and
% mention the original authors and Deutsche Telekom AG in a suitable manner. In
% a paper-like reuse please insert a short hint to www.telekom.com and to the
% original authors and Deutsche Telekom AG into your preface. For normal
% quotations please use the scientific standard to cite.
%
% [ Framework derived from 'mind your Scholar Research Framework' 
%   mycsrf (c) K. Reincke 2012 CC BY 3.0  http://mycsrf.fodina.de/ ]
%


%% use all entries of the bibliography
%\nocite{*}

[TDB \ldots]

%\bibliography{../../../bibfiles/oscResourcesEn}

% Telekom osCompendium 'for beeing included' snippet template
%
% (c) Karsten Reincke, Deutsche Telekom AG, Darmstadt 2011
%
% This LaTeX-File is licensed under the Creative Commons Attribution-ShareAlike
% 3.0 Germany License (http://creativecommons.org/licenses/by-sa/3.0/de/): Feel
% free 'to share (to copy, distribute and transmit)' or 'to remix (to adapt)'
% it, if you '... distribute the resulting work under the same or similar
% license to this one' and if you respect how 'you must attribute the work in
% the manner specified by the author ...':
%
% In an internet based reuse please link the reused parts to www.telekom.com and
% mention the original authors and Deutsche Telekom AG in a suitable manner. In
% a paper-like reuse please insert a short hint to www.telekom.com and to the
% original authors and Deutsche Telekom AG into your preface. For normal
% quotations please use the scientific standard to cite.
%
% [ Framework derived from 'mind your Scholar Research Framework' 
%   mycsrf (c) K. Reincke 2012 CC BY 3.0  http://mycsrf.fodina.de/ ]
%


%% use all entries of the bibliography
%\nocite{*}

\section{Apache Licensed Software \ldots}

\label{OSUC-01-Apache20} \label{OSUC-03-Apache20} 
\label{OSUC-06-Apache20} \label{OSUC-09-Apache20}

\label{OSUC-02-Apache20} \label{OSUC-04-Apache20} \label{OSUC-05-Apache20}
\label{OSUC-07-Apache20} \label{OSUC-08-Apache20} \label{OSUC-10-Apache20}

\subsection{Apache specific mini finder}

\subsection{Apache specific use case 1}
(covers OSUC-X - OSUC-Z)

\subsection{Apache specific use case n}
(covers OSUC-x - OSUC-z)




%\bibliography{../../../bibfiles/oscResourcesEn}

% Telekom osCompendium 'for being included' snippet template
%
% (c) Karsten Reincke, Deutsche Telekom AG, Darmstadt 2011
%
% This LaTeX-File is licensed under the Creative Commons Attribution-ShareAlike
% 3.0 Germany License (http://creativecommons.org/licenses/by-sa/3.0/de/): Feel
% free 'to share (to copy, distribute and transmit)' or 'to remix (to adapt)'
% it, if you '... distribute the resulting work under the same or similar
% license to this one' and if you respect how 'you must attribute the work in
% the manner specified by the author ...':
%
% In an internet based reuse please link the reused parts to www.telekom.com and
% mention the original authors and Deutsche Telekom AG in a suitable manner. In
% a paper-like reuse please insert a short hint to www.telekom.com and to the
% original authors and Deutsche Telekom AG into your preface. For normal
% quotations please use the scientific standard to cite.
%
% [ Framework derived from 'mind your Scholar Research Framework' 
%   mycsrf (c) K. Reincke 2012 CC BY 3.0  http://mycsrf.fodina.de/ ]
%


%% use all entries of the bibliography
%\nocite{*}

\section{BSD Licensed Software \ldots}

\subsection{The BSD specific mini finder}

As an approved Open Source license, the BSD license exists in two
versions\footcite[Following the Open Source Initiative, initially a not approved
BSD license contained a fourth clause also known as advertising clause which
\enquote{(\ldots) officially was rescinded by the Director of the Office of
Technology Licensing of the University of California on July 22nd, 1999}.
 Cf.][\nopage wp. Because of the cancellation you can simply act according the
 \textit{BSD 3-Clause license} if you have to fulfill the eldest BSD
 license]{BsdLicense3Clause}. The latest, most modern release is the \textit{BSD
 2-Clause license}\footcite[cf.][\nopage wp]{BsdLicense2Clause}, the elder
 release is the \textit{BSD 3-Clause license}\footcite[cf.][\nopage
 wp]{BsdLicense3Clause}. Nevertheless, the little differences between the
 two versions have to be respected.

All BSD Open Source Licenses explicitly focuses 'only' on the (re-)distribution
\textit{Open Source Use Cases}, which we have specified by our token
\textit{4others}. Conditions for the use cases specified by the token
\textit{4yourself} can be derived\footnote{For details of the \textit{Open
Source Use Cases tokens} see p. \pageref{OsucTokens}. For Details of the
\textit{Open Source Use Cases} based on these token see p.
\pageref{OsucDefinitionTree} }. Additionally the BSD license considers the form
of the distribution, eg. whether the work is distributed as a (set of) source
code file(s) or as a (set of) the binary file(s). Use the following tree to find
the BSD license fulfilling to-do lists.

\begin{center}
\begin{footnotesize}
\pstree[levelsep=*1,treesep=0.2]{\Toval{BSD}}{
  \pstree[levelsep=*0.2,treesep=1]{
    \Toval{2 Clause License}
  }{
  \pstree{
    \Tr{\Ovalbox{\shortstack{recipient: \textit{4yourself}\\
    \textbf{\textit{used by yourself}}}}} 
  }{
    \Tr{\doublebox{\shortstack{\tiny{\textbf{BSD-1}:}\\
    \tiny{\textit{using the}}\\\tiny{\textit{software}}\\
    \tiny{\textit{only for}}\\\tiny{\textit{yourself}} }}} 
  }
  \pstree[levelsep=*0.2,treesep=0.2]{
    \Tr{\Ovalbox{\shortstack{recipient: \textit{4others}\\
      \textbf{\textit{distributed to 3rd. parties}}}}} 
  }{ 
    \pstree[levelsep=*0.2,treesep=0.2]{
      \Tr{\Ovalbox{\shortstack{state:\\\textbf{\textit{unmodified}}}}}
    }{        
      \pstree{
        \Tr{\Ovalbox{\shortstack{form:\\\textbf{\textit{source}}}}}
      }{        
        \Tr{\doublebox{\shortstack{\tiny{\textbf{BSD-2}:}\\
        \tiny{\textit{distributing}}\\\tiny{\textit{unmodified}}\\
        \tiny{\textit{software as}}\\\tiny{\textit{source code}} }}} 
      }
      \pstree{
        \Tr{\Ovalbox{\shortstack{form:\\\textbf{\textit{binary}}}}}
      }{        
        \Tr{\doublebox{\shortstack{\tiny{\textbf{BSD-3}:}\\
        \tiny{\textit{distributing}}\\\tiny{\textit{unmodified}}\\
        \tiny{\textit{software as}}\\\tiny{\textit{binary pkg}} }}} 
      }

    }

    \pstree[levelsep=*0.2,treesep=0.2]{
      \Tr{\Ovalbox{\shortstack{state:\\\textbf{\textit{modified}}}}}
    }{ 
      \pstree{
        \Tr{\Ovalbox{\shortstack{type:\\\textbf{\textit{proapse}}}}}
      }{
           
        \pstree{
          \Tr{\Ovalbox{\shortstack{form:\\\textbf{\textit{source}}}}}
        }{        
          \Tr{\doublebox{\shortstack{\tiny{\textbf{BSD-4}:}\\
          \tiny{\textit{distributing}}\\\tiny{\textit{a modified}}\\
          \tiny{\textit{program as}}\\\tiny{\textit{source code}} }}} 
        }
        \pstree{
          \Tr{\Ovalbox{\shortstack{form:\\\textbf{\textit{binary}}}}}
        }{        
          \Tr{\doublebox{\shortstack{\tiny{\textbf{BSD-5}:}\\
          \tiny{\textit{distributing}}\\\tiny{\textit{a modified}}\\
          \tiny{\textit{program as}}\\\tiny{\textit{binary pkg}} }}} 
        }
      }
      \pstree{
        \Tr{\Ovalbox{\shortstack{type:\\\textbf{\textit{snimoli}}}}}
      }{
        \pstree{
          \Tr{\Ovalbox{\shortstack{context:\\\textbf{\textit{independent}}}}}
        }{        
           
          \pstree{
            \Tr{\Ovalbox{\shortstack{form:\\\textbf{\textit{source}}}}}
          }{        
            \Tr{\doublebox{\shortstack{\tiny{\textbf{BSD-6}:}\\
            \tiny{\textit{distributing}}\\\tiny{\textit{a modified}}\\
            \tiny{\textit{library as}}\\\tiny{\textit{independent}}\\
            \tiny{\textit{source pkg}} }}} 
          }
          \pstree{
            \Tr{\Ovalbox{\shortstack{form:\\\textbf{\textit{binary}}}}}
          }{        
            \Tr{\doublebox{\shortstack{\tiny{\textbf{BSD-7}:}\\
            \tiny{\textit{distributing}}\\\tiny{\textit{a modified}}\\
            \tiny{\textit{library as}}\\\tiny{\textit{independent}}\\
            \tiny{\textit{binary pkg}} }}} 
          }
        }
        \pstree{
          \Tr{\Ovalbox{\shortstack{context:\\\textbf{\textit{embedded}}}}}
        }{        
           
           \pstree{
            \Tr{\Ovalbox{\shortstack{form:\\\textbf{\textit{source}}}}}
          }{        
            \Tr{\doublebox{\shortstack{\tiny{\textbf{BSD-8}:}\\
            \tiny{\textit{distributing}}\\\tiny{\textit{a modified}}\\
            \tiny{\textit{library as}}\\\tiny{\textit{embedded}}\\
            \tiny{\textit{source pkg}} }}} 
          }
          \pstree{
            \Tr{\Ovalbox{\shortstack{form:\\\textbf{\textit{binary}}}}}
          }{        
            \Tr{\doublebox{\shortstack{\tiny{\textbf{BSD-9}:}\\
            \tiny{\textit{distributing}}\\\tiny{\textit{a modified}}\\
            \tiny{\textit{library as}}\\\tiny{\textit{embedded}}\\
            \tiny{\textit{binary pkg}} }}} 
          }
        }


      }
 
    }
   }   

  }
  \pstree[levelsep=*0.2,treesep=0.2]{
    \Toval{3 Clause License}
  }{
    \Ttri{}
  }
}
\end{footnotesize}
\end{center}

\subsection{Software licensed by the \emph{BSD 2-Clause License}}

\subsubsection{BSD-1: using the software only for yourself}
\label{OSUC-01-BSD} 
\label{OSUC-03-BSD} 
\label{OSUC-06-BSD}
\label{OSUC-09-BSD}
  
\begin{description}
\item[means] that you are going to use a received BSD software only for yourself
and that you do not handover it to any 3rd. party in any sense.
\item[covers] OSUC-01, OSUC-03, OSUC-06, and OSUC-09\footnote{For details see pp.
  \pageref{OSUC-01-DEF} - \pageref{OSUC-09-DEF}}
\item[requires] the tasks in order to fulfill the conditions
    of the BSD license:
  \begin{itemize}
    \item You are allowed to use any kind of BSD software in any sense and in
    any context without any obligations if you do not handover the software to
    3rd parties.
  \end{itemize}
\end{description}


\subsubsection{BSD-2: Passing the unmodified software as a source code package}
\label{OSUC-02-BSD} \label{OSUC-05-BSD} \label{OSUC-07-BSD} 

\begin{description}
\item[means] that you are going distribute an unmodified version of the received
BSD software to 3rd. parties in form of a set of source code files or an
integrated source code package\footnote{In this case it doesn't matter whether
you  distribute a program, an application, a server, a snippet, a module, a
library, or a plugin as an independent or an embedded unit} 

\item[covers] OSUC-02, OSUC-05, OSUC-07\footnote{For details see pp.
\pageref{OSUC-02-DEF} - \pageref{OSUC-07-DEF}}

\item[requires] the following tasks in order to fulfill the license conditions
\begin{itemize}
  \item \textbf{[mandatorily:]} Ensure, that the licensing elements - eg.
  the BSD license text, the specific copyright notice of the original author(s),
  and the BSD disclaimer - are retained in your package in the form you have got
  them.
  \item \textbf{[voluntarily:]} Let the documentation of your distribution
  and/or your additional material also contain the original copyright notice, the
  BSD conditions, and the BSD disclaimer.
\end{itemize}

\item[forbids] the following doings in order to fulfill the license conditions
\begin{itemize}
  \item \textbf{[directly:]} 
  \item \textbf{[indirectly:]}

\end{itemize}
\end{description}

\subsubsection{BSD-3: Passing the unmodified software as a binary package}

\begin{description}
\item[means] that you are going distribute an unmodified version of the BSD
received software to 3rd. parties in form of a set of binary files or an
integrated bi\-na\-ry package\footnote{In this case it doesn't matter whether
you distribute a program, an application, a server, a snippet, a module, a library,
or a plugin as an independent or an embedded unit}
\item[covers] OSUC-02, OSUC-05, OSUC-07\footnote{For details see pp.
\pageref{OSUC-02-DEF} - \pageref{OSUC-07-DEF}}
\item[requires] the tasks in order to fulfill the license conditions
\begin{itemize}
  \item  \textbf{[mandatory:]} Ensure, that your distribution contains the
  original copyright notice, the BSD license, and the BSD disclaimer in the form
  you have got them. If you compile the binary file on the base of the source
  code package and if this compilation does not also generate and integrate the
  licensing files, then create the copyright notice, the BSD conditions, and the
  BSD disclaimer according to the form of the source code package and insert
  these files into your distribution manually.
  \item  \textbf{[mandatory:]} Ensure, that the documentation of your
  distribution and/or your additional material also contain the author specific
  copyright notice, the BSD conditions, and the BSD disclaimer.
\end{itemize}
\end{description}

\begin{itshape}
\emph{\textbf{General remark for all binary distributions}:} 
\label{MobileDeviceHint} Even if you are distributing a BSD bi\-na\-ry package
on a medium, which doesn't allow the user, to see package files directly - some
mobile devices don't give their users the full access to all stored elements -
you have to fulfill the mandatory requirements. So, sometimes, it is necessary,
to let simply the BSD license and the copyright notice become an invisible part
of the binary package and to deploy the files together with the package. The
point is: you have fulfilled the license.
\end{itshape}

\subsubsection{BSD-4: Passing a modified program as a source code package}
\label{OSUC-04-BSD}

\begin{description}
\item[means] that you are going distribute a modified version of the received
BSD program, application, or server (proapse) to 3rd. parties in form of a set
of source code files or an integrated source code package.
\item[covers] OSUC-04\footnote{For details see pp. \pageref{OSUC-04-DEF}}
\item[requires] the tasks in order to fulfill the license conditions
\begin{itemize}
  \item \textbf{[mandatory:]} Ensure, that the licensing elements - e.g.
  the BSD license text, the specific copyright notice of the original author(s),
  and the BSD disclaimer - are retained in your package in the form you have got
  them. 
  \item \textbf{[voluntary:]} Let the documentation of your distribution
  and/or your additional material also contain the original copyright notice, the
  BSD conditions, and the BSD disclaimer.
  
  \item \textbf{[voluntary:]} It is a good practice of the Open Source
  community, to let the copyright notice which is shown by the running program
  also state that the program is licensed under the BSD license. Because you are
  already modifying the program, you can also add such a hint, if the presented
  original copyright notice lacks such a statement.
\end{itemize}
\end{description}

\subsubsection{BSD-5: Passing a modified program as a binary package}

\begin{description}
\item[means] that you are going distribute a modified version of the received
BSD pro\-gram, application, or server (proapse) to 3rd. parties in form of a set
of binary files or an integrated binary package.
\item[covers] OSUC-04\footnote{For details see pp. \pageref{OSUC-04-DEF}}
\item[requires] the tasks in order to fulfill the license conditions
\begin{itemize}

  \item  \textbf{[mandatory:]} Ensure, that your distribution contains the
  original copyright notice, the BSD license, and the BSD disclaimer in the form
  you have got them. If you compile the binary file on the base of the source
  code package and if this compilation does not also generate and integrate the
  licensing files, then create the copyright notice, the BSD conditions, and the
  BSD disclaimer according to the form of the source code package and insert
  these files into your distribution manually\footnote{see also our 'Mobile
  Device Hint' on p. \pageref{MobileDeviceHint}}.

  \item  \textbf{[mandatory:]} Ensure, that the documentation of your
  distribution and/or your additional material also contain the author specific
  copyright notice, the BSD conditions, and the BSD disclaimer.
  
  \item \textbf{[voluntary:]} It is a good practice of the Open Source
  community, to let the copyright notice which is shown by the running program
  also state that the program is licensed under the BSD license. Because you are
  already modifying the program, you can also add such a hint, if the presented
  original copyright notice lacks such a statement.
\end{itemize}
\end{description}

\subsubsection{BSD-6: Passing a modified library as independent source code
package}
\label{OSUC-08-BSD}
\begin{description}
\item[means] that you are going distribute a modified version of the received
BSD code snippet, module, library, or plugin (snimoli) to 3rd. parties in form
of a set of source code files or an integrated source code package, but without
embedding it into another larger software unit.
\item[covers] OSUC-08\footnote{For details see pp. \pageref{OSUC-08-DEF}}
\item[requires] the tasks in order to fulfill the license conditions
\begin{itemize}
  \item \textbf{[mandatory:]} Ensure, that the licensing elements - e.g.
  the BSD license text, the specific copyright notice of the original author(s),
  and the BSD disclaimer - are retained in your package in the form you have got
  them.
  \item \textbf{[voluntary:]} Let the documentation of your distribution
  and/or your additional material also contain the original copyright notice, the
  BSD conditions, and the BSD disclaimer.
\end{itemize}
\end{description}


\subsubsection{BSD-7: Passing a modified library as independent binary
package}

\begin{description}
\item[means] that you are going distribute a modified version of the received
BSD code snippet, module, library, or plugin (snimoli) to 3rd. parties in form
of a set of binary files or an integrated binary package, but without embedding
it into another larger software unit.
\item[covers] OSUC-08\footnote{For details see pp. \pageref{OSUC-08-DEF}}
\item[requires] the tasks in order to fulfill the license conditions
\begin{itemize}
   \item  \textbf{[mandatory:]} Ensure, that your distribution contains the
  original copyright notice, the BSD license, and the BSD disclaimer in the form
  you have got them. If you compile the binary file on the base of the source
  code package and if this compilation does not also generate and integrate the
  licensing files, then create the copyright notice, the BSD conditions, and the
  BSD disclaimer according to the form of the source code package and insert
  these files into your distribution manually\footnote{see also our 'Mobile
  Device Hint' on p. \pageref{MobileDeviceHint}}.
  \item  \textbf{[mandatory:]} Ensure, that the documentation of your
  distribution and/or your additional material also contain the author specific
  copyright notice, the BSD conditions, and the BSD disclaimer.
\end{itemize}
\end{description}

\subsubsection{BSD-8: Passing a modified library as an embedded source code
package}
\label{OSUC-10-BSD}
\begin{description}
\item[means] that you are going distribute a modified version of the received
BSD code snippet, module, library, or plugin (snimoli) to 3rd. parties in form
of a set of source code files or an integrated source code package together with
another larger software unit, which contains this code snippet, module, library,
or plugin as an embedded component.
\item[covers] OSUC-10\footnote{For details see pp. \pageref{OSUC-10-DEF}}
\item[requires] the tasks in order to fulfill the license conditions
\begin{itemize}
  \item \textbf{[mandatory:]} Ensure, that the licensing elements - e.g.
  the BSD license text, the specific copyright notice of the original author(s),
  and the BSD disclaimer - are retained in your package in the form you have got
  them.
  \item \textbf{[voluntary:]} Let the documentation of your distribution
  and/or your additional material also contain the original copyright notice, the
  BSD conditions, and the BSD disclaimer.
 \item \textbf{[voluntary:]} It is a good practice of the Open Source
  community, to let the copyright notice which is shown by the running program
  also state that it contains components licensed under the BSD license. Because
  you are embedding this snimoli into a larger software unit, you are
  developing this larger unit. Hence, you can also expand the copyright notice
  of this larger unit by such a hint to its BSD components.
\end{itemize}
\end{description}


\subsubsection{BSD-9: Passing a modified library as an embedded binary
package}

\begin{description}
\item[means] that you are going distribute a modified version of the received
BSD code snippet, module, library, or plugin to 3rd. parties in form of a set of
binary files or an integrated binary package together with another larger
software unit, which contains this code snippet, module, library, or plugin as
an embedded component.
\item[covers] OSUC-10\footnote{For details see pp. \pageref{OSUC-10-DEF}}
\item[requires] the tasks in order to fulfill the license conditions
\begin{itemize}
  \item  \textbf{[mandatory:]} Ensure, that your distribution contains the
  original copyright notice, the BSD license, and the BSD disclaimer in the form
  you have got them. If you compile the binary file on the base of the source
  code package and if this compilation does not also generate and integrate the
  licensing files, then create the copyright notice, the BSD conditions, and the
  BSD disclaimer according to the form of the source code package and insert
  these files into your distribution manually\footnote{see also our 'Mobile
  Device Hint' on p. \pageref{MobileDeviceHint}}.
  \item  \textbf{[mandatory:]} Ensure, that the documentation of your
  distribution and/or your additional material also contain the author specific
  copyright notice, the BSD conditions, and the BSD disclaimer.
 \item \textbf{[voluntary:]} It is a good practice of the Open Source
  community, to let the copyright notice which is shown by the running program
  also state that it contains components licensed under the BSD license. Because
  you are embedding this snimoli into a larger software unit, you are
  developing this larger unit. Hence, you can also expand the copyright notice
  of this larger unit by such a hint to its BSD components.
\end{itemize}
\end{description}

\subsection{Software licensed by the \emph{BSD 3-Clause License}}

Compared with the \textit{BSD 2-Clause license}, the \textit{BSD 3-Clause
license} only contains one additional 3rd. clause, the rest is the same. So, for
acting according the \textit{BSD 3-Clause license}, do that, what you would have
to do for fulfilling the 2 clause license. And additionally do not use the name
of licensing organization or the names of the licensing distributors to promote
your own work.

\subsection{Discussions and Explanations}

The \textit{BSD 2-Clause license} has a simple textual structure: In the
beginning, it generally \enquote{(permits) [the] redistribution and [the] use in
source and binary forms, with or without modification, [\ldots]}, if one
fulfills the two rules of the license\footcite[cf.][\nopage
wp]{BsdLicense2Clause}. The first rule concerns the (re)distribution in form of
source code, the second the (re)distribution of binary packages. Here are some
explanations why we translated the rules into which sets of executable tasks:

\begin{itemize}
\item For the \enquote{redistribution of source code} the license requires,
that the package must \enquote{ [\ldots] retain the above copyright notice, this
list of conditions and the following disclaimer}\footcite[cf.][\nopage
wp]{BsdLicense2Clause}. Hence, you are not allowed, to modify any of the
copyright notes, which are already embedded in the received (source) files. And
from a logical point of view, there must exist an explicit or implicit
assertion, that the software is licensed under the \textit{BSD 2-Clause
license}\footcite[The BSD license requires, that a re-distributed software
package must contain the (package specific) copyright notice, the (license
specific) conditions and the BSD disclaimer. (cf.][\nopage wp.) You might ask
what you should do, if these elements are missed in the package you got. If so,
the package you got had not been licensed adequately. Hence, you do not know
reliably whether you have received it under a BSD license. In other words: If
you have received a BSD licensed software package, it must contain sufficient
license fulfilling elements, or it is not a BSD licensed
software]{BsdLicense2Clause}. This is often implemented by simply adding a copy
of the license into the package. Hence, you are furthermore not allowed to
modify these files or corresponding text snippets. For our purposes, we
translated the bans into the following executable task:

\begin{quote}\textit{Ensure, that the licensing elements - eg. the BSD license
text, the specific copyright notice of the original author(s), and the BSD disclaimer
- are retained in your package in the form you have got them.}\end{quote}

\item For the redistribution in form of binary files, the license requires,
that the licensing elements must be \enquote{[\ldots] (reproduced) in the documentation
and/or other materials provided with the
distribution}\footcite[cf.][\nopage wp]{BsdLicense2Clause}. Hence, this is
not required as necessary condition for the (re)distribution as source code
package . But nevertheless, even for a distribution in form of source code, it
is often possible, to fulfill this rule too - eg., if you offer an own download
site for source code packages. In such cases, it is a sign of respect, to
mention the licensing not only inside of the packages, but also in the text of
your site. Because of that, we added the following voluntary task for all BSD
Open Source Use Cases, which deal with the redistribution in form of source
code'

\begin{quote}\textit{Let the documentation of your distribution and/or your
additional material also contain the original copyright notice, the BSD
conditions, and the BSD disclaimer.}\end{quote}

\item Naturally, because the reproduction of the licensing elements \enquote{in
the documentation and/or other materials provided with the distribution}
is explicitly required for the \enquote{redistribution in binary
form}\footcite[cf.][\nopage wp]{BsdLicense2Clause}, we had to rewrite the
facultative task for a distribution in form of source code as a mandatory task
for all BSD Open Source Use Cases, which deal with the redistribution in binary
form':

\begin{quote}\textit{Ensure, that the documentation of your distribution and/or
your additional material also contain the author specific copyright notice, the
BSD conditions, and the BSD disclaimer.}\end{quote}

\item In case of (re)distributing the program in form of binary files, it is
sometimes not enough, to pass the licensing elements as one has got them. If you
compile the binary package from the source code, it is not necessarily true,
that the licensing elements are also automatically generated and embedded into
the 'binary package'. But nevertheless, you have to add the copyright notice,
the conditions and the disclaimer to this package for acting according to the
BSD license. Therefore we chose the following form of an executable, license
fulfilling task for all binary oriented distributions:

\begin{quote}\textit{Ensure, that your distribution contains the original
copyright notice, the BSD license, and the BSD disclaimer in the form you have
got them. If you compile the binary file on the base of the source code package
and if this compilation does not also generate and integrate the licensing
files, then create the copyright notice, the BSD conditions, and the BSD
disclaimer according to the form of the source code package and insert these
files into your distribution manually.}\end{quote}

\item Finally, we wished to insert a hint to the general (Open Source)
tradition, to mention the used Open Source Software and their licenses as a
remark of the 'copyright widget' of an application. This is not required by the
BSD license. But it is a general, good tradition. Naturally, because of the
freedom, to use and modify Open Source Software, and to redistribute a modified
version of it, you are also allowed to insert such references, even if they are
missed. Therefore we added a third voluntary, license tradition fulfilling task
for all relevant Open Source Use Cases.

\end{itemize}




%\bibliography{../../../bibfiles/oscResourcesEn}

% Telekom osCompendium 'for being included' snippet template
%
% (c) Karsten Reincke, Deutsche Telekom AG, Darmstadt 2011
%
% This LaTeX-File is licensed under the Creative Commons Attribution-ShareAlike
% 3.0 Germany License (http://creativecommons.org/licenses/by-sa/3.0/de/): Feel
% free 'to share (to copy, distribute and transmit)' or 'to remix (to adapt)'
% it, if you '... distribute the resulting work under the same or similar
% license to this one' and if you respect how 'you must attribute the work in
% the manner specified by the author ...':
%
% In an internet based reuse please link the reused parts to www.telekom.com and
% mention the original authors and Deutsche Telekom AG in a suitable manner. In
% a paper-like reuse please insert a short hint to www.telekom.com and to the
% original authors and Deutsche Telekom AG into your preface. For normal
% quotations please use the scientific standard to cite.
%
% [ Framework derived from 'mind your Scholar Research Framework' 
%   mycsrf (c) K. Reincke 2012 CC BY 3.0  http://mycsrf.fodina.de/ ]
%


%% use all entries of the bibliography
%\nocite{*}

\section{GPL V2 Licensed Software in the usage context of \ldots}
\label{OSUC-01-GPL2X} \label{OSUC-03-GPL2X} 
\label{OSUC-06-GPL2X} \label{OSUC-09-GPL2X}

\label{OSUC-02-GPL2X} \label{OSUC-04-GPL2X} \label{OSUC-05-GPL2X}
\label{OSUC-07-GPL2X} \label{OSUC-08-GPL2X} \label{OSUC-10-GPL2X}

\subsection{GPL-2 specific mini finder}

\subsection{GPL-2 specific use case 1}
(covers OSUC-X - OSUC-Z)

\subsection{GPL-2 specific use case n}
(covers OSUC-x - OSUC-z)


%\bibliography{../../../bibfiles/oscResourcesEn}

% Telekom osCompendium 'for beeing included' snippet template
%
% (c) Karsten Reincke, Deutsche Telekom AG, Darmstadt 2011
%
% This LaTeX-File is licensed under the Creative Commons Attribution-ShareAlike
% 3.0 Germany License (http://creativecommons.org/licenses/by-sa/3.0/de/): Feel
% free 'to share (to copy, distribute and transmit)' or 'to remix (to adapt)'
% it, if you '... distribute the resulting work under the same or similar
% license to this one' and if you respect how 'you must attribute the work in
% the manner specified by the author ...':
%
% In an internet based reuse please link the reused parts to www.telekom.com and
% mention the original authors and Deutsche Telekom AG in a suitable manner. In
% a paper-like reuse please insert a short hint to www.telekom.com and to the
% original authors and Deutsche Telekom AG into your preface. For normal
% quotations please use the scientific standard to cite.
%
% [ Framework derived from 'mind your Scholar Research Framework' 
%   mycsrf (c) K. Reincke 2012 CC BY 3.0  http://mycsrf.fodina.de/ ]
%


%% use all entries of the bibliography
%\nocite{*}

\section{MIT Licensed Software in the usage context of \ldots}
\label{OSUC-01-MIT} \label{OSUC-03-MIT} 
\label{OSUC-06-MIT} \label{OSUC-09-MIT}

\label{OSUC-02-MIT} \label{OSUC-04-MIT} \label{OSUC-05-MIT}
\label{OSUC-07-MIT} \label{OSUC-08-MIT} \label{OSUC-10-MIT}

\subsection{MIT specific mini finder}

\subsection{MIT specific use case 1}
(covers OSUC-X - OSUC-Z)

\subsection{MIT specific use case n}
(covers OSUC-x - OSUC-z)

%\bibliography{../../../bibfiles/oscResourcesEn}


%%%%%%%%%%%%%%%
\chapter{Open Source Licenses and Their Legal Environments}
% Telekom osCompendium 'for being included' snippet template
%
% (c) Karsten Reincke, Deutsche Telekom AG, Darmstadt 2011
%
% This LaTeX-File is licensed under the Creative Commons Attribution-ShareAlike
% 3.0 Germany License (http://creativecommons.org/licenses/by-sa/3.0/de/): Feel
% free 'to share (to copy, distribute and transmit)' or 'to remix (to adapt)'
% it, if you '... distribute the resulting work under the same or similar
% license to this one' and if you respect how 'you must attribute the work in
% the manner specified by the author ...':
%
% In an internet based reuse please link the reused parts to www.telekom.com and
% mention the original authors and Deutsche Telekom AG in a suitable manner. In
% a paper-like reuse please insert a short hint to www.telekom.com and to the
% original authors and Deutsche Telekom AG into your preface. For normal
% quotations please use the scientific standard to cite.
%
% [ File structure derived from 'mind your Scholar Research Framework' 
%   mycsrf (c) K. Reincke CC BY 3.0  http://mycsrf.fodina.de/ ]
%

% Chapter Abstract
% ----------------

\footnotesize
\begin{quote}\itshape
In this chapter we analyze why to know a license alone is not enough. At the end
you will know that Open Source Licenses are embedded into the legal environment
of a state. And you will know in which sense the German legal environment
predetermines your readings of Open Source Licenses.
\end{quote}
\normalsize{}


% Telekom osCompendium 'for beeing included' snippet template
%
% (c) Karsten Reincke, Deutsche Telekom AG, Darmstadt 2011
%
% This LaTeX-File is licensed under the Creative Commons Attribution-ShareAlike
% 3.0 Germany License (http://creativecommons.org/licenses/by-sa/3.0/de/): Feel
% free 'to share (to copy, distribute and transmit)' or 'to remix (to adapt)'
% it, if you '... distribute the resulting work under the same or similar
% license to this one' and if you respect how 'you must attribute the work in
% the manner specified by the author ...':
%
% In an internet based reuse please link the reused parts to www.telekom.com and
% mention the original authors and Deutsche Telekom AG in a suitable manner. In
% a paper-like reuse please insert a short hint to www.telekom.com and to the
% original authors and Deutsche Telekom AG into your preface. For normal
% quotations please use the scientific standard to cite.
%
% [ Framework derived from 'mind your Scholar Research Framework' 
%   mycsrf (c) K. Reincke 2012 CC BY 3.0  http://mycsrf.fodina.de/ ]
%


%% use all entries of the bibliography
%\nocite{*}

[TDB \ldots]

%\bibliography{../../../bibfiles/oscResourcesEn}


%%%%%%%%%%%%%%%
\chapter{Conclusion}
% Telekom osCompendium 'for beeing included' snippet template
%
% (c) Karsten Reincke, Deutsche Telekom AG, Darmstadt 2011
%
% This LaTeX-File is licensed under the Creative Commons Attribution-ShareAlike
% 3.0 Germany License (http://creativecommons.org/licenses/by-sa/3.0/de/): Feel
% free 'to share (to copy, distribute and transmit)' or 'to remix (to adapt)'
% it, if you '... distribute the resulting work under the same or similar
% license to this one' and if you respect how 'you must attribute the work in
% the manner specified by the author ...':
%
% In an internet based reuse please link the reused parts to www.telekom.com and
% mention the original authors and Deutsche Telekom AG in a suitable manner. In
% a paper-like reuse please insert a short hint to www.telekom.com and to the
% original authors and Deutsche Telekom AG into your preface. For normal
% quotations please use the scientific standard to cite.
%
% [ File structure derived from 'mind your Scholar Research Framework' 
%   mycsrf (c) K. Reincke CC BY 3.0  http://mycsrf.fodina.de/ ]
%

% Chapter Abstract
% ----------------

\footnotesize
\begin{quote}\itshape
This chapter shortly describes what the OSLiC is, how it should be used, and how
it can be read. It shall be written as top-down explanation.
\end{quote}
\normalsize{}


% Telekom osCompendium 'for beeing included' snippet template
%
% (c) Karsten Reincke, Deutsche Telekom AG, Darmstadt 2011
%
% This LaTeX-File is licensed under the Creative Commons Attribution-ShareAlike
% 3.0 Germany License (http://creativecommons.org/licenses/by-sa/3.0/de/): Feel
% free 'to share (to copy, distribute and transmit)' or 'to remix (to adapt)'
% it, if you '... distribute the resulting work under the same or similar
% license to this one' and if you respect how 'you must attribute the work in
% the manner specified by the author ...':
%
% In an internet based reuse please link the reused parts to www.telekom.com and
% mention the original authors and Deutsche Telekom AG in a suitable manner. In
% a paper-like reuse please insert a short hint to www.telekom.com and to the
% original authors and Deutsche Telekom AG into your preface. For normal
% quotations please use the scientific standard to cite.
%
% [ Framework derived from 'mind your Scholar Research Framework' 
%   mycsrf (c) K. Reincke 2012 CC BY 3.0  http://mycsrf.fodina.de/ ]
%


%% use all entries of the bibliography
%\nocite{*}

[TDB \ldots]

%\bibliography{../../../bibfiles/oscResourcesEn}


%%%%%%%%%%%%%%%
\chapter{Appendices}
% Telekom osCompendium 'for beeing included' snippet template
%
% (c) Karsten Reincke, Deutsche Telekom AG, Darmstadt 2011
%
% This LaTeX-File is licensed under the Creative Commons Attribution-ShareAlike
% 3.0 Germany License (http://creativecommons.org/licenses/by-sa/3.0/de/): Feel
% free 'to share (to copy, distribute and transmit)' or 'to remix (to adapt)'
% it, if you '... distribute the resulting work under the same or similar
% license to this one' and if you respect how 'you must attribute the work in
% the manner specified by the author ...':
%
% In an internet based reuse please link the reused parts to www.telekom.com and
% mention the original authors and Deutsche Telekom AG in a suitable manner. In
% a paper-like reuse please insert a short hint to www.telekom.com and to the
% original authors and Deutsche Telekom AG into your preface. For normal
% quotations please use the scientific standard to cite.
%
% [ File structure derived from 'mind your Scholar Research Framework' 
%   mycsrf (c) K. Reincke CC BY 3.0  http://mycsrf.fodina.de/ ]
%

% Chapter Abstract
% ----------------

\footnotesize
\begin{quote}\itshape
This chapter shortly describes what the OSLiC is, how it should be used, and how
it can be read. It shall be written as top-down explanation.
\end{quote}
\normalsize{}


% Telekom osCompendium 'for beeing included' snippet template
%
% (c) Karsten Reincke, Deutsche Telekom AG, Darmstadt 2011
%
% This LaTeX-File is licensed under the Creative Commons Attribution-ShareAlike
% 3.0 Germany License (http://creativecommons.org/licenses/by-sa/3.0/de/): Feel
% free 'to share (to copy, distribute and transmit)' or 'to remix (to adapt)'
% it, if you '... distribute the resulting work under the same or similar
% license to this one' and if you respect how 'you must attribute the work in
% the manner specified by the author ...':
%
% In an internet based reuse please link the reused parts to www.telekom.com and
% mention the original authors and Deutsche Telekom AG in a suitable manner. In
% a paper-like reuse please insert a short hint to www.telekom.com and to the
% original authors and Deutsche Telekom AG into your preface. For normal
% quotations please use the scientific standard to cite.
%
% [ Framework derived from 'mind your Scholar Research Framework' 
%   mycsrf (c) K. Reincke 2012 CC BY 3.0  http://mycsrf.fodina.de/ ]
%


%% use all entries of the bibliography
%\nocite{*}

[TDB \ldots]

%\bibliography{../../../bibfiles/oscResourcesEn}


% Telekom osCompendium 'for beeing included' snippet template
%
% (c) Karsten Reincke, Deutsche Telekom AG, Darmstadt 2011
%
% This LaTeX-File is licensed under the Creative Commons Attribution-ShareAlike
% 3.0 Germany License (http://creativecommons.org/licenses/by-sa/3.0/de/): Feel
% free 'to share (to copy, distribute and transmit)' or 'to remix (to adapt)'
% it, if you '... distribute the resulting work under the same or similar
% license to this one' and if you respect how 'you must attribute the work in
% the manner specified by the author ...':
%
% In an internet based reuse please link the reused parts to www.telekom.com and
% mention the original authors and Deutsche Telekom AG in a suitable manner. In
% a paper-like reuse please insert a short hint to www.telekom.com and to the
% original authors and Deutsche Telekom AG into your preface. For normal
% quotations please use the scientific standard to cite.
%
% [ File structure derived from 'mind your Scholar Research Framework' 
%   mycsrf (c) K. Reincke CC BY 3.0  http://mycsrf.fodina.de/ ]

%


%% use all entries of the bibliography
%\nocite{*}


\section{Some Widespread Open Source Myths}

From the viewpoint of an internet student we have to consider that the web
offers a mass of rumors concerning the nature of Open Source Software
(Licenses). Here are some of the myths\footcite[At least one time even a
scientific legally discussing book is talking about the \enquote{Myth around Open
Source Licenses} - although only as part of  the title: cf][1ff,
especially 209ff]{GuiOvd2006a}.
we met:
 

\begin{description}
  \item[Open Source tries to improve the world ethically] :- no, there's a clear
  ban to exclude persons, groups, purposes
  \item[Changed Open Source Software must be re-published] :- no, in a double
  sense! There are OS licenses which a allow the proprietarization of the
  modified code. And even the LGPL and the GPL, which clearly try to prevent
  the proprietarization, do not require generally that a modified code must be
  (re-)published. Only if you give your modfied (L)GPL licensed application as
  binary to anyboday then you have to handover the modified code too.
  \item[Modified Open Source Software must be given back to the whole community]
  :- No. Again, there are OS licenses which a allow the proprietarization of the
  modified code. And even the LGPL and the GPL - which clearly require, that you
  also publish the modified code, if you give the modified binary to anybody -
  do not require that you distribute your modification around the world. LGPL and
  GPL clearly say that you have to hand over the code to those persons which you
  give the binary. And if you only give your improvement only one person or a
  group of person, then you must handover your the code only to that persons or
  only to all members of that group.
  \item[Published Open Source Software is open for ever] :- No. The Copyright
  holder ever holds the Copyright. The can change the licence of next release of
  its software.
  \item[Software can either be Open Source Software or proprietary software] :-
  The Copyright holders can ever distribute the code under other conditions, een
  additionally. That's not a question of the licence, but of the Copyright.
  \item[The opposite of Open Source Software is commercial Software] :- No.
  Firstly you are also allowed to use the Open Source software in any commercial
  purpose. There's only one point which is excluded in OSS: you are not allowed
  to ask for a licence fee if you distribute 'Open Source Software'. Secondly
  there are many other forms like Freeware, Public Domain Software or anything
  else which is neither Open Source Software nor Commercial Software. It's
  senseless to take the question of money as mark for distinguish Open Source
  Software and its opposite. Moreover: Proprietary Software as opposite of Open
  Source Software should be defined ex negativo: all kind of software, which
  does not fit the OSD is proprietary.
  \item[Open Source Software prohibits to earn money] :- No,
  you are allowed to invent each business model you want. There's only one
  exception: you are not allowed to ask for a licence fee if you distribute
  'Open Source Software [Achtung: sollte eigentlich nur für GPL gelten].
  Historically this mistake might be evoked by Debian: The GNU project missed
  its kernel while the Linux kernel was already distributed as part of
  collections which also include GNU software. Then, in 1983? Ian Murdock was
  supported by RMS and its FSF to build a really free distribution (Debian)
  containg GNU software and the Linux kernel. But Ian Murdock states also, that
  debian does not want to earn money. (clear details)
% TODO: check, wether OSD requires license fee free distribution
  \item[Modifications of Open Source Software must be marked] :- No. This is not
  a defining postulation of the OSD. The OSD allows licenses to require the mark
  of modifacations. But it does not require from all licenses to rquire the mark
  modifications for being an Open Source License.
  \item[Modifications of Open Source Software must be marked by your personal
  data] :- No, it's only required to mark modifications so that a reader could
  distinguish the modifications from the original code. It's required for saving
  the integrity of the original author. And therefor it's not required as a
  constitutiv criteria by the OSD. It might be, that a license additionally
  requires your name. But's not featue of Open Source Software in general. And
  at least the licenses discussed by us do not require to insert your name.
% TODO: check wether any of our licenses reuire that you mark modifications by
% your personal data / real name  
  \item[The Open Source Definition determines the conditions to use Open Source
  Software] :- No. The Open Source Definition determines which licenses are Open
  Source Licenses, nothing more. The OSD is a set of necessary conditions to be
  an Open Source License. It determines the freedom and the responsibilities of
  a user as a set of more or less abstract rules. But it does not constitute a
  set of sufficient tasks which a user has to do for fulfilling any Open Source
  License. Open Source Licenses may differ by instatiating the OSD criteria.
  So, if you want to know what you have to do to fulfill a license you have to
  go back to the real license of that software you are using.
\end{description}

%\bibliography{../bibfiles/oscResourcesEn}


\small
%\theendnotes


\footnotesize
% Telekom osCompendium English Nomenclation Tokens Include Module 
%
% (c) Karsten Reincke, Deutsche Telekom AG, Darmstadt 2011
%
% This LaTeX-File is licensed under the Creative Commons Attribution-ShareAlike
% 3.0 Germany License (http://creativecommons.org/licenses/by-sa/3.0/de/): Feel
% free 'to share (to copy, distribute and transmit)' or 'to remix (to adapt)'
% it, if you '... distribute the resulting work under the same or similar
% license to this one' and if you respect how 'you must attribute the work in
% the manner specified by the author ...':
%
% In an internet based reuse please link the reused parts to www.telekom.com and
% mention the original authors and Deutsche Telekom AG in a suitable manner. In
% a paper-like reuse please insert a short hint to www.telekom.com and to the
% original authors and Deutsche Telekom AG into your preface. For normal
% quotations please use the scientific standard to cite.
%
% [ File structure derived from 'mind your Scholar Research Framework' 
%   mycsrf (c) K. Reincke CC BY 3.0  http://mycsrf.fodina.de/ ]


%\abbr[aaO]{a.a.O.}{am angegebenen Ort}
%\abbr[ds]{ds.}{kollektiv für ders., dies., \ldots}
\abbr[etseqq]{et seqq.}{and the following ones}
\abbr[id]{id.}{idem = latin for 'the same', be it a man, woman or a group\ldots}
\abbr[ibid]{ibid.}{ibidem = latin for 'at the same place'}
\abbr[ifross]{ifross}{Institut für Rechtsfragen der Freien und Open Source
Software}
\abbr[lc]{l.c.}{loco citato = latin for 'in the place cited'}
\abbr[np]{np.}{no page numbering}
\abbr[wp]{wp.}{webpage / webdocument without any internal (page)numbering}
\abbr[nst]{n.st.}{not stated}
\abbr[njear]{n.y.}{year not stated / no year}
\abbr[nlocation]{n.l.}{location not stated / no location}
\abbr[ub]{UB}{'Universitätsbibliothek' = library of university X}
\abbr[ulb]{ULB}{'Universitäts- \& Landesbibliothek' = library of university and state X}
\abbr[apl]{ApL}{Apache License}
\abbr[bsd]{BSD}{Berkeley Software Distrobution (License)}
\abbr[mit]{MIT}{Massachusetts Institute of Technology (License)}
\abbr[mspl]{Ms-PL}{Microsoft Public License}
\abbr[pgl]{PgL}{Postgres License}
\abbr[php]{PHP}{PHP (License)}
\abbr[epl]{EPL}{Eclipse Public License}
\abbr[eupl]{EUPL}{European Union Public License}
\abbr[lgpl]{LGPL}{GNU Lesser General Public License}
\abbr[mpl]{MPL}{Mozilla Public License}
\abbr[gpl]{GPL}{GNU General Public License}
\abbr[agpl]{AGPL}{GNU Affero General Public License}
\abbr[nabbr]{n.abbr.}{no abbreviation (known)}
% Telekom osCompendium English Nomenclation Tokens Include Module 
%
% (c) Karsten Reincke, Deutsche Telekom AG, Darmstadt 2011
%
% This LaTeX-File is licensed under the Creative Commons Attribution-ShareAlike
% 3.0 Germany License (http://creativecommons.org/licenses/by-sa/3.0/de/): Feel
% free 'to share (to copy, distribute and transmit)' or 'to remix (to adapt)'
% it, if you '... distribute the resulting work under the same or similar
% license to this one' and if you respect how 'you must attribute the work in
% the manner specified by the author ...':
%
% In an internet based reuse please link the reused parts to www.telekom.com and
% mention the original authors and Deutsche Telekom AG in a suitable manner. In
% a paper-like reuse please insert a short hint to www.telekom.com and to the
% original authors and Deutsche Telekom AG into your preface. For normal
% quotations please use the scientific standard to cite.
%
% [ Derived from 'mykeds Scholar Research Framework' 
%   mykeds-CSR-framework (c) K. Reincke CC BY 3.0  http://www.mykeds.net/ ]

%\abbr[]{[n.abbr.]}{ }
\abbr[zge]{ZGE / IPJ}{Zeitschrift für geistiges Eigentum [ISSN: 1867-237x]}
\abbr[itrb]{ITRB}{Der IT-Rechtsberater [ISSN: 1617-1527]}
\abbr[cri]{CRi}{Computer Law Review international [ISSN: 1610-7608]}
\abbr[btlj]{[n.abbr.]}{Berkeley Technology Law Journal}
\abbr[eclr]{E.C.L.R.}{European Competition Law Review}
\abbr[iesw]{[n.abbr.]}{IEEE Software [ISSN: 0740-7459]}
\abbr[cuitj]{[n.abbr.]}{Cutter IT Journal [ISSN: 1048-5600]}
\abbr[uoclr]{[n.abbr.]}{University of Chicago Law Review}
\abbr[uoilr]{[n.abbr.]}{University of Illinois Law Review}
\abbr[uoplr]{[n.abbr.]}{University of Pittsburgh Law Review}
\abbr[ddt]{DDT}{Drug Discovery Today [ISSN: 1359-6446]}
\abbr[rdm]{[n.abbr.]}{R\&D Management [ISSN: 1467-9310]}
\abbr[jleo]{JLEO}{Journal of Law, Economics, \& Organization [ISSN: 1465-7341]}
\abbr[ijomi]{[n.abbr.]}{International Journal of Medical Informatics [ISSN: 1386-5056]}
\abbr[slr]{[n.abbr.]}{Stanford Law Review [ISSN: 00389765]}
\abbr[bise]{BISE}{Business \& Information Systems Engineering [ISSN: 1867-0202]}
\abbr[joals]{[n.abbr.]}{Journal of Academic Librarianship [ISSN: 0099-1333]}
\abbr[eait]{[n.abbr.]}{Ethics and Information Technology [ISSN: 1388-1957]}
\abbr[jais]{JAIS}{Journal of the Association for Information Systems [ISSN:
1536-9323]}
\abbr[josas]{[n.abbr.]}{Journal of Systems and Software [ISSN: 0164-1212]}
\abbr[iialr]{[n.abbr.]}{International Information and Library Review [ISSN: 1057-2317]}
\abbr[sthv]{STHV}{Science, Technology \& Human Values [ISSN: 0162-2439]}
\abbr[cue]{[n.abbr.]}{Computers \& Education [ISSN: 0360-1315]}
\abbr[eer]{EER}{European Economic Review [ISSN: 0014-2921]}
\abbr[icc]{ICC}{Industrial and Corporate Change [ISSN: 0960-6491]}
\abbr[ca]{[n.abbr.]}{Cultural Anthropology [ISSN: 1548-1360]}
\abbr[sqj]{[n.abbr.]}{Software Qualilty Journal [ISSN: 0963-9314]}
\abbr[jmir]{JMIR}{Journal of Medical Information Research [ISSN: 1438-8871]}
\abbr[joce]{[n.abbr.]}{Journal of Comparative Economics [ISSN: 0147-5967]}
\abbr[orgsci]{[n.abbr.]}{Organization Science [ISSN: 1047-7039]}
\abbr[iam]{[n.abbr.]}{Information \& Management [ISSN: 0378-7206]}
\abbr[rp]{RP}{Research Policy [ISSN: 0048-7333]}
\abbr[jsis]{JSIS}{Journal of Strategic Information Systems [ISSN: 0963-8687]}
\abbr[isj]{ISJ}{Information Systems Journal [ISSN: 1365-2575]}
\abbr[jise]{JISE}{Journal of Information Science and Engineering [ISSN:
1016-2364]}
\abbr[dss]{DSS}{Decision Support Systems [ISSN: 0167-9236]}
\abbr[cihp]{CiHB}{Computers in Human Behavior [ISSN: 0747-5632]}
\abbr[iep]{IEaP}{Information Economics and Policy [ISSN: 0167-6245]}
\abbr[tosem]{ToSEM}{Transactions on Software Engineering Methodology [ISSN:
1049-331X]}
\abbr[commacm]{CotACM}{Communications of the ACM [ISSN: 0001-0782]}
\abbr[interactions]{[n.abbr.]}{interactions[ISSN: 1072-5520]}
\abbr[jcsc]{JCSC}{Journal of Computing Sciences in [Small] Colleges [ISSN:
1937-4771]}
\abbr[linuxjournal]{LJ}{Linux Journal [ISSN: 1075-3583]}
\abbr[networker]{[n.abbr.]}{netWorker [ISSN: 1091-3556]}
\abbr[queue]{[n.abbr.]}{Queue [ISSN: 1542-7730]}
\abbr[sigmisdb]{SIGMIS Database}{ACM SIGMIS - The Data Base for Advances in
Information Systems [ISSN: 0095-0033]}
\abbr[sigcas]{SIGCAS}{ACM SIGCAS Computers and Society [ISSN: 0095-2737]}
\abbr[sigsoft]{SIGSOFT SEN}{SIGSOFT Software Engineering Notes [ISSN:
0163-5948]}
\abbr[toit]{ToIT}{Transaction on Internet Technology [ISSN: 1533-5399]}
\abbr[sigbul]{SIGCSE Bulletin}{SIGCSE Bulletin [ISSN: 0097-8418]}
\abbr[ubiquity]{Ubiquity}{Ubiquity - The ACM IT Magazine and Forum [ISSN:
1530-2180]}
\abbr[bwv]{BWV}{Berliner Wissenschafts-Verlag GmbH}
\abbr[cr]{CR}{Computer und Recht. Zeitschrift für die Praxis des Rechts der
Informationstechnologien}


\printnomenclature

\bibliography{bibfiles/oscResourcesEn}


\end{document}
