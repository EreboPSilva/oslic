% Telekom osCompendium extract template
%
% (c) Karsten Reincke, Deutsche Telekom AG, Darmstadt 2011
%
% This LaTeX-File is licensed under the Creative Commons Attribution-ShareAlike
% 3.0 Germany License (http://creativecommons.org/licenses/by-sa/3.0/de/): Feel
% free 'to share (to copy, distribute and transmit)' or 'to remix (to adapt)'
% it, if you '... distribute the resulting work under the same or similar
% license to this one' and if you respect how 'you must attribute the work in
% the manner specified by the author ...':
%
% In an internet based reuse please link the reused parts to www.telekom.com and
% mention the original authors and Deutsche Telekom AG in a suitable manner. In
% a paper-like reuse please insert a short hint to www.telekom.com and to the
% original authors and Deutsche Telekom AG into your preface. For normal
% quotations please use the scientific standard to cite.
%
% [ File structure derived from 'mind your Scholar Research Framework' 
%   mycsrf (c) K. Reincke CC BY 3.0  http://mycsrf.fodina.de/ ]

%
% select the document class
% S.26: [ 10pt|11pt|12pt onecolumn|twocolumn oneside|twoside notitlepage|titlepage final|draft
%         leqno fleqn openbib a4paper|a5paper|b5paper|letterpaper|legalpaper|executivepaper openrigth ]
% S.25: { article|report|book|letter ... }
%
% oder koma-skript S.10 + 16
\documentclass[DIV=calc,BCOR=5mm,11pt,headings=small,oneside,abstract=true, toc=bib]{scrartcl}

%%% (1) general configurations %%%
\usepackage[utf8]{inputenc}

%%% (2) language specific configurations %%%
\usepackage[]{a4,ngerman}
\usepackage[ngerman, english]{babel}
\selectlanguage{english}

%language specific quoting signs
%default for language emglish is american style of quotes
\usepackage[english=british]{csquotes}

% jurabib configuration
\usepackage[see]{jurabib}
\bibliographystyle{jurabib}
% Telekom osCompendium English Jurabib Configuration Include Module 
%
% (c) Karsten Reincke, Deutsche Telekom AG, Darmstadt 2011
%
% This LaTeX-File is licensed under the Creative Commons Attribution-ShareAlike
% 3.0 Germany License (http://creativecommons.org/licenses/by-sa/3.0/de/): Feel
% free 'to share (to copy, distribute and transmit)' or 'to remix (to adapt)'
% it, if you '... distribute the resulting work under the same or similar
% license to this one' and if you respect how 'you must attribute the work in
% the manner specified by the author ...':
%
% In an internet based reuse please link the reused parts to www.telekom.com and
% mention the original authors and Deutsche Telekom AG in a suitable manner. In
% a paper-like reuse please insert a short hint to www.telekom.com and to the
% original authors and Deutsche Telekom AG into your preface. For normal
% quotations please use the scientific standard to cite.
%
% [ File structure derived from 'mind your Scholar Research Framework' 
%   mycsrf (c) K. Reincke CC BY 3.0  http://mycsrf.fodina.de/ ]

% the first time cite with all data, later with shorttitle
\jurabibsetup{citefull=first}

%%% (1) author / editor list configuration
%\jurabibsetup{authorformat=and} % uses 'und' instead of 'u.'
% therefore define your own abbreviated conjunction: 
% an 'and before last author explicetly written conjunction

% for authors in citations
\renewcommand*{\jbbtasep}{ a.\ } % bta = between two authors sep
\renewcommand*{\jbbfsasep}{, } % bfsa = between first and second author sep
\renewcommand*{\jbbstasep}{, a.\ }% bsta = between second and third author sep
% for editors in citations
\renewcommand*{\jbbtesep}{ a.\ } % bta = between two authors sep
\renewcommand*{\jbbfsesep}{, } % bfsa = between first and second author sep
\renewcommand*{\jbbstesep}{, a.\ }% bsta = between second and third author sep

% for authors in literature list
\renewcommand*{\bibbtasep}{ a.\ } % bta = between two authors sep
\renewcommand*{\bibbfsasep}{, } % bfsa = between first and second author sep
\renewcommand*{\bibbstasep}{, a.\ }% bsta = between second and third author sep
% for editors  in literature list
\renewcommand*{\bibbtesep}{ a.\ } % bte = between two editors sep
\renewcommand*{\bibbfsesep}{, } % bfse = between first and second editor sep
\renewcommand*{\bibbstesep}{, a.\ }% bste = between second and third editor sep

% use: name, forname, forname lastname u. forname lastname
\jurabibsetup{authorformat=firstnotreversed}
\jurabibsetup{authorformat=italic}

%%% (2) title configuration
% in every case print the title, let it be seperated from the 
% author by a colon and use the slanted font
\jurabibsetup{titleformat={all,colonsep}}
%\renewcommand*{\jbtitlefont}{\textit}

%%% (3) seperators in bib data
% separate bibliographical hints and page hints by a comma
\jurabibsetup{commabeforerest}

%%% (4) specific configuration of bibdata in quotes / footnote
% use a.a.O if possible
\jurabibsetup{ibidem=strict}
% replace ugly a.a.O. by translation of ders., a.a.O.
\AddTo\bibsgerman{
  \renewcommand*{\ibidemname}{Id., l.c.}
  \renewcommand*{\ibidemmidname}{id., l.c.}
}
\renewcommand*{\samepageibidemname}{Id., ibid.}
\renewcommand*{\samepageibidemmidname}{id., ibid.}


%%% (5) specific configuration of bibdata in bibliography
% ever an in: before journal and collection/book-tiltes 
\renewcommand*{\bibbtsep}{in: }
\renewcommand*{\bibjtsep}{in: }
% ever a colon after author names 
\renewcommand*{\bibansep}{: }
% ever a semi colon after the title
% \AddTo\bibsgerman{\renewcommand*{\urldatecomment}{Referenzdownload: }}
\renewcommand*{\bibatsep}{; }
% ever a comma before date/year
\renewcommand*{\bibbdsep}{, }

% let jurabib insert the S. and p. information
% no S. necessary in bib-files and in cites/footcites
\jurabibsetup{pages=format}

% use a compressed literature-list using a small line indent
\jurabibsetup{bibformat=compress}
\setlength{\jbbibhang}{1em}

% which follows the design of the cites and offers comments
\jurabibsetup{biblikecite}

% print annotations into bibliography
\jurabibsetup{annote}
\renewcommand*{\jbannoteformat}[1]{{ \itshape #1 }}

%refine the prefix of url download
\AddTo\bibsgerman{\renewcommand*{\urldatecomment}{reference download: }}

% we want to have the year of articles in brackets
\renewcommand*{\bibaldelim}{(}
\renewcommand*{\bibardelim}{)}

% in english version Nr. must be replaced by No.
\renewcommand*{\artnumberformat}[1]{\unskip,\space No.~#1}
\renewcommand*{\pernumberformat}[1]{\unskip\space No.~#1}%
\renewcommand*{\revnumberformat}[1]{\unskip\space No.~#1}%


%Reformatierung Seriestitels and Seriesnumber
\DeclareRobustCommand{\numberandseries}[2]{%
\unskip\unskip%,
\space\bibsnfont{(=~#2}%
\ifthenelse{\equal{#1}{}}{)}{, [Vol./No.]~#1)}%
}%


% Local Variables:
% mode: latex
% fill-column: 80
% End:


% language specific hyphenation
% Telekom osCompendium osHyphenation Include Module
%
% (c) Karsten Reincke, Deutsche Telekom AG, Darmstadt 2011
%
% This LaTeX-File is licensed under the Creative Commons Attribution-ShareAlike
% 3.0 Germany License (http://creativecommons.org/licenses/by-sa/3.0/de/): Feel
% free 'to share (to copy, distribute and transmit)' or 'to remix (to adapt)'
% it, if you '... distribute the resulting work under the same or similar
% license to this one' and if you respect how 'you must attribute the work in
% the manner specified by the author ...':
%
% In an internet based reuse please link the reused parts to www.telekom.com and
% mention the original authors and Deutsche Telekom AG in a suitable manner. In
% a paper-like reuse please insert a short hint to www.telekom.com and to the
% original authors and Deutsche Telekom AG into your preface. For normal
% quotations please use the scientific standard to cite.
%
% [ File structure derived from 'mind your Scholar Research Framework' 
%   mycsrf (c) K. Reincke CC BY 3.0  http://mycsrf.fodina.de/ ]
%


\hyphenation{rein-cke}
\hyphenation{OS-LiC}
\hyphenation{ori-gi-nal}


%%% (3) layout page configuration %%%

% select the visible parts of a page
% S.31: { plain|empty|headings|myheadings }
%\pagestyle{myheadings}
\pagestyle{headings}

% select the wished style of page-numbering
% S.32: { arabic,roman,Roman,alph,Alph }
\pagenumbering{arabic}
\setcounter{page}{1}

% select the wished distances using the general setlength order:
% S.34 { baselineskip| parskip | parindent }
% - general no indent for paragraphs
\setlength{\parindent}{0pt}
\setlength{\parskip}{1.2ex plus 0.2ex minus 0.2ex}


%%% (4) general package activation %%%
%\usepackage{utopia}
%\usepackage{courier}
%\usepackage{avant}
\usepackage[dvips]{epsfig}

% graphic
\usepackage{graphicx,color}
\usepackage{array}
\usepackage{shadow}
\usepackage{fancybox}

%- start(footnote-configuration)
%  flush the cite numbers out of the vertical line and let
%  the footnote text directly start in the left vertical line
\usepackage[marginal]{footmisc}
%- end(footnote-configuration)

\begin{document}

%% use all entries of the bliography

%%-- start(titlepage)
\titlehead{Literaturexzerpt}
\subject{Autor(en): Bretschneider Glaschick Gräfe}
\title{Titel: Ratgeber für die Veröffentlichung von Open-Source-Software durch
eine Hochschule}
\subtitle{Jahr: 2008 }
\author{K. Reincke% Telekom osCompendium License Include Module
%
% (c) Karsten Reincke, Deutsche Telekom AG, Darmstadt 2011
%
% This LaTeX-File is licensed under the Creative Commons Attribution-ShareAlike
% 3.0 Germany License (http://creativecommons.org/licenses/by-sa/3.0/de/): Feel
% free 'to share (to copy, distribute and transmit)' or 'to remix (to adapt)'
% it, if you '... distribute the resulting work under the same or similar
% license to this one' and if you respect how 'you must attribute the work in
% the manner specified by the author ...':
%
% In an internet based reuse please link the reused parts to www.telekom.com and
% mention the original authors and Deutsche Telekom AG in a suitable manner. In
% a paper-like reuse please insert a short hint to www.telekom.com and to the
% original authors and Deutsche Telekom AG into your preface. For normal
% quotations please use the scientific standard to cite.
%
% [ File structure derived from 'mind your Scholar Research Framework' 
%   mycsrf (c) K. Reincke CC BY 3.0  http://mycsrf.fodina.de/ ]
%
\footnote{
This text is licensed under the Creative Commons Attribution-ShareAlike 3.0 Germany
License (http://creativecommons.org/licenses/by-sa/3.0/de/): Feel free \enquote{to
share (to copy, distribute and transmit)} or \enquote{to remix (to
adapt)} it, if you \enquote{[\ldots] distribute the resulting work under the
same or similar license to this one} and if you respect how \enquote{you
must attribute the work in the manner specified by the author(s)
[\ldots]}):
\newline
In an internet based reuse please mention the initial authors in a suitable
manner, name their sponsor \textit{Deutsche Telekom AG} and link it to
\texttt{http://www.telekom.com}. In a paper-like reuse please insert a short
hint to \texttt{http://www.telekom.com}, to the initial authors, and to their
sponsor \textit{Deutsche Telekom AG} into your preface. For normal quotations
please use the scientific standard to cite.
\newline
{ \tiny \itshape [derived from myCsrf (= 'mind your Scholar Research Framework') 
\copyright K. Reincke CC BY 3.0  http://mycsrf.fodina.de/)] }}}

%\thanks{den Autoren von KOMA-Script und denen von Jurabib}
\maketitle
%%-- end(titlepage)
%\nocite{*}

\begin{abstract}
\noindent
Das Werk / The work\footcite[][]{BreGlaGra2008a} \\
\noindent \itshape
\ldots Stellt wesentliche Aspekte einer Open Source Veröffentlichung gründlich,
knapp und übersichtlich dar: das nicht übertragbare Urheberrecht wird gegen das
transferierbare Verwertungs- und Nutzungsrecht ebenso abgegrenzt, wie
Haftungsfragen diskutiert und die Wahl der Lizenz auf den intendierten Zweck hin
beleuchtet werden. Trotzdem kann (und will) der Artikel kein allgemeines 'Open
Source Compendium' sein: der Frage, ob und wann eine universitäre Entwicklung
gezwungenermaßen veröffentlicht werden muss und was Universitäten ansonsten tun
müssen, wenn sie OS Software intern nutzen und weitergeben, wird nicht
beleuchtet.\\
\noindent
\ldots This article summarizes basic aspects of an Open Source publication
succesfully. It addresses the (German) difference between copyright
('Urheberecht') and transferring the right to use ('Nutzungsrecht'). It mentions
the problem of liability. It discusses the choice of an Open Source License with
respect to the indented purpose and many things more. Nevertheless the article
can't be taken as the sought-after 'Open Source Compendium': it doesn't analyze
in which cases a University perhaps must publish its' developments or what it must
do for fulfilling the licenses of internally (re-)used and distributed Open
Source Software.
\end{abstract}
\footnotesize
%\tableofcontents
\normalsize

\section{Line of Thought}

Der Artikel stellt sich eine hohe Aufgabe: er will \enquote{[\ldots]
Fragestellungen (adressieren), die im Zusammenhang mit einer
Vörffentlichung einer Software unter den Rahmenbedingungen einer
Open-Source-Lizenz ent[!SW!]stehen}\footcite[vgl.][167]{BreGlaGra2008a}.
Und mehrnoch, er bezeichnet \enquote{die Bereitstellung eines pragmatischen
Ratgebers} als seine eigene \enquote{Zielsetzung}, ein
\enquote{[\ldots] Ratgeber, der den Verantwortlichen eine Hilfestellung für
entsprechende Vorhaben bietet
[\ldots]}\footcite[vgl.][168]{BreGlaGra2008a}.

In der Zusammenfassung wird der Anspruch sogar noch schärfer formuliert. Hier
wird gesagt, dass \enquote{der vorliegende Ratgeber [\ldots] bei der
Veröffentlichung von Open-Source-Projekten dienen [\ldots]} und dass die
\enquote{[\ldots] die zu berücksichtigende Aspekte (strukturiert)
abgearbeitet werden (könnten) [\ldots]}, sodass damit
\enquote{rechtlich nicht angreifbarer Veröffentlichungsprozess} ermöglicht
würde\footcite[vgl.][186]{BreGlaGra2008a}

 [KR: Unterstellt, dass eine Universität auf die gleichen Interessen
und Probleme stossen dürfte, wie eine (grosse) Firma, sollte dieser Artikel zur Erhellung
unser Fragen beitragen.]

\begin{itemize}
  \item Zunächst unterstreicht der Artikel, dass sich \enquote{aus den Lizenzen
  selbst [\ldots] keine Verpflichtung zur öffentlichen Bereitstellung
  (ergäbe)}. Ja schärfer noch: Es gelte, dass Software \enquote{[\ldots]
  unter eine Open Source Lizenz gestellt und lediglich bilaterial verteilt
  werden (könne) [\ldots]}. Damit gelte allerdings auch, \enquote{[\ldots]
  dass jeder Empfänger sie dann veröffentlichen
  (könne)}\footcite[vgl.][170]{BreGlaGra2008a} - und zwar, wie von außen
  zu ergänzen wäre, ohne Restriktionen und weltweit.
  \item Sodann grenzt der Artikel - sprachlich fein - das (aus deutscher Sicht)
  nicht übertragbare \enquote{Urheberpersönlichkeitsrecht} gegen das
  \enquote{in Form von Nutzungsrechten übertragbare Verwertungsrecht}
  ab\footcite[vgl.][173]{BreGlaGra2008a}. Und er erwähnt auch, dass die
  \enquote{[\ldots] (Verwertungsrechte) bei Softwareentwicklungen, die im Rahmen
  eines Angestellten- oder Dienstverhältnisses entstanden (seien), [\ldots]
  auto[!SW!]matisch auf den Arbeitgeber [\ldots] (übergehen
  würden)}\footcite[vgl.][173f]{BreGlaGra2008a}. Diese Verwertungsrechte
  seien aber eben nur ein Teil. Nach Jaeger/Metzger, 2006 S.23 verbleibe das
  'Veröffentlichungsrecht', das 'Recht auf Anerkennung der Urheberschaft' und
  der 'Schutz gegen Entstellung der Software' gemäß UrhG eben beim
  ursprünglichen Programmierer\footcite[vgl.][174]{BreGlaGra2008a}.
  \item Schließlich erwähnt der Artikel auch, dass Open Source Veröffentlichung
  nur und durch die Übertragung von Nutzungsrechten möglich
  wird\footcite[vgl.][175]{BreGlaGra2008a}, was für Universitäten [und -
  wie zu ergänzen wäre - andere Arbeitgeber] auch bedeutet, dass sie mit dem
  Autor eine gesonderte umfassende Vereinbarung abschließen müssen, sofern sie
  selbst diese Software als Open Source veröffentlichen
  wollen\footcite[vgl.][175f]{BreGlaGra2008a}
  \item Ferner referiert der Artikel den Gedankengang, demzufolge Open Source
  Software als eine \enquote{Schenkung} im Sinne einer
  \enquote{unentgeldlichen Zuwendung} zu werten sei, wobei die Übertragung
  eines \enquote{[\ldots] (Nutzungsrechtes) an einer Software [\ldots] eine
  solche Zuwendung (sei)}\footcite[vgl.][177]{BreGlaGra2008a}. Damit müsse
  aber auch die Haftung entsprechend einer Schenkung eingegrenzt werden, und
  zwar auf die \enquote{Haftung für arglistig verschwiegene Mängel}, denn
  das Gesetzt lege fest, dass man denjenigen \enquote{[\ldots] der
  etwas verschenkt, nicht über Gebühr für das Verschenkte verantwortlich
  (machen können) solle}\footcite[vgl.][177]{BreGlaGra2008a}. Allerdings
  gelte als untere Grenze eben auch, dass diese \enquote{Haftung für
  arglistige verschwiegene Mängel} - worunter sicher auch der vorsätzlich
  erzeugte zählen dürfte - eben auch \enquote{[\ldots] nicht ausgeschlossen
  werden (könne)}, sodass
  der übliche \enquote{vollständige Gewährleistungsausschluss} aus Open
  Source Lizenzen - wenigstens in Deutschland - \enquote{nichtig}
  sei\footcite[vgl.][177]{BreGlaGra2008a}.
  \item Dann geht der Artikel auf die Lizenzwahl bei und für eine
  Eigenveröffentlichung ein. Er verweist zunächst auf die OSI Standardlizenzen
  oder die vom ifross aufgelisteten\footcite[vgl.][179]{BreGlaGra2008a}. Er
  erläutert den \enquote{Copyleft-Effekt} der
  GPL\footcite[vgl.][180]{BreGlaGra2008a} und verweist darauf, dass die LGPL für
  \enquote{Middleware} besser geeignet sei, weil sie zwar Veränderungen an
  sich selbst im Sinne des Copylefts prägt, aber \enquote{[\ldots] eine
  Integration mit kommerzieller Software erlaubt, ohne dass letztere
  unter die GPL gestellt werden
  (müsse)}\footcite[vgl.][181]{BreGlaGra2008a}. 
  \item Final verweist der Artikel darauf, dass \enquote{[\ldots] (der Einfluss
  auf das weitere Schicksal [der Software; KR]]) mit einer Veröffentlichung
  unter einer Open-Source-Lizenz [\ldots] (rechtlich) weitestgehend
  aufgegeben (werde)}\footcite[vgl.][183]{BreGlaGra2008a} und dass
  eine Veröffentlichung auch ein Umfeld für die Distribution und die
  Kommunikation mit und in die Community
  erfordere\footcite[vgl.][185f]{BreGlaGra2008a}

\end{itemize}

\section{Specific Aspects}

Herausragend ist in diesem Artikel die Darstellung der Intention und des Umfangs
der GPL. Hier wird mit Mythen aufgräumt:

Dies betrifft den Mythos, dass \enquote{[\ldots] der Rechteinhaber einer
Software verpflichtet sei, die gesamte Software incl. Quellcode
öffentlich[sic!] zum Download bereitzustellen, wenn in dieser Software
Teile einer unter GPL stehenden Software enthalten sind}.

\begin{itemize}
  \item Tatsächlich sei \enquote{der Rechteinhaber [\ldots] lediglich
  verpflichtet, im Falle einer Weitergabe von Binärcode, auch den
  Quellcode für den Empfänger bereitzustellen}.
  \item Sodann müsse er den Quellcode auch nicht \enquote{öffentlich}
  bereitstellen, sondern es \enquote{[\ldots] muss der Quellcode nur denen
  (zusätzlich) gegeben werden, die auch zuvor den Binärcode erhalten
  haben}, die dann allerdings ihrerseits frei seien, das Ganze öffentlich
  weiterrzureichen
\end{itemize}

Alle Zitate hier\footcite[vgl.][181]{BreGlaGra2008a}


\small
\bibliography{../bibfiles/oscResourcesEn}

\end{document}
