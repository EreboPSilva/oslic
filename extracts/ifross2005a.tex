% Telekom osCompendium extract template
%
% (c) Karsten Reincke, Deutsche Telekom AG, Darmstadt 2011
%
% This LaTeX-File is licensed under the Creative Commons Attribution-ShareAlike
% 3.0 Germany License (http://creativecommons.org/licenses/by-sa/3.0/de/): Feel
% free 'to share (to copy, distribute and transmit)' or 'to remix (to adapt)'
% it, if you '... distribute the resulting work under the same or similar
% license to this one' and if you respect how 'you must attribute the work in
% the manner specified by the author ...':
%
% In an internet based reuse please link the reused parts to www.telekom.com and
% mention the original authors and Deutsche Telekom AG in a suitable manner. In
% a paper-like reuse please insert a short hint to www.telekom.com and to the
% original authors and Deutsche Telekom AG into your preface. For normal
% quotations please use the scientific standard to cite.
%
% [ File structure derived from 'mind your Scholar Research Framework' 
%   mycsrf (c) K. Reincke CC BY 3.0  http://mycsrf.fodina.de/ ]

%
% select the document class
% S.26: [ 10pt|11pt|12pt onecolumn|twocolumn oneside|twoside notitlepage|titlepage final|draft
%         leqno fleqn openbib a4paper|a5paper|b5paper|letterpaper|legalpaper|executivepaper openrigth ]
% S.25: { article|report|book|letter ... }
%
% oder koma-skript S.10 + 16
\documentclass[DIV=calc,BCOR=5mm,11pt,headings=small,oneside,abstract=true, toc=bib]{scrartcl}

%%% (1) general configurations %%%
\usepackage[utf8]{inputenc}

%%% (2) language specific configurations %%%
\usepackage[]{a4,ngerman}
\usepackage[english, ngerman]{babel}
\selectlanguage{ngerman}

% jurabib configuration
\usepackage[see]{jurabib}
\bibliographystyle{jurabib}
% Telekom osCompendium German Jurabib Configuration Include Module 
%
% (c) Karsten Reincke, Deutsche Telekom AG, Darmstadt 2011
%
% This LaTeX-File is licensed under the Creative Commons Attribution-ShareAlike
% 3.0 Germany License (http://creativecommons.org/licenses/by-sa/3.0/de/): Feel
% free 'to share (to copy, distribute and transmit)' or 'to remix (to adapt)'
% it, if you '... distribute the resulting work under the same or similar
% license to this one' and if you respect how 'you must attribute the work in
% the manner specified by the author ...':
%
% In an internet based reuse please link the reused parts to www.telekom.com and
% mention the original authors and Deutsche Telekom AG in a suitable manner. In
% a paper-like reuse please insert a short hint to www.telekom.com and to the
% original authors and Deutsche Telekom AG into your preface. For normal
% quotations please use the scientific standard to cite.
%
% [ File structure derived from 'mind your Scholar Research Framework' 
%   mycsrf (c) K. Reincke CC BY 3.0  http://mycsrf.fodina.de/ ]

% the first time cite with all data, later with shorttitle
\jurabibsetup{citefull=first}

%%% (1) author / editor list configuration
%\jurabibsetup{authorformat=and} % uses 'und' instead of 'u.'
% therefore define your own abbreviated conjunction: 
% an 'and before last author explicetly written conjunction

% for authors in citations
\renewcommand*{\jbbtasep}{ u. } % bta = between two authors sep
\renewcommand*{\jbbfsasep}{, } % bfsa = between first and second author sep
\renewcommand*{\jbbstasep}{ u. }% bsta = between second and third author sep
% for editors in citations
\renewcommand*{\jbbtesep}{ u. } % bta = between two authors sep
\renewcommand*{\jbbfsesep}{, } % bfsa = between first and second author sep
\renewcommand*{\jbbstesep}{ u. }% bsta = between second and third author sep

% for authors in literature list
\renewcommand*{\bibbtasep}{ u. } % bta = between two authors sep
\renewcommand*{\bibbfsasep}{, } % bfsa = between first and second author sep
\renewcommand*{\bibbstasep}{ u. }% bsta = between second and third author sep
% for editors  in literature list
\renewcommand*{\bibbtesep}{ u. } % bte = between two editors sep
\renewcommand*{\bibbfsesep}{, } % bfse = between first and second editor sep
\renewcommand*{\bibbstesep}{ u. }% bste = between second and third editor sep

% use: name, forname, forname lastname u. forname lastname
\jurabibsetup{authorformat=firstnotreversed}
\jurabibsetup{authorformat=italic}

%%% (2) title configuration
% in every case print the title, let it be seperated from the 
% author by a colon and use the slanted font
\jurabibsetup{titleformat={all,colonsep}}
%\renewcommand*{\jbtitlefont}{\textit}

%%% (3) seperators in bib data
% separate bibliographical hints and page hints by a comma
\jurabibsetup{commabeforerest}

%%% (4) specific configuration of bibdata in quotes / footnote
% use a.a.O if possible
\jurabibsetup{ibidem=strict}

% replace ugly a.a.O. by ders., a.a.O. resp. ders., ebda.
% but if there are more than one author or girl writers?
\AddTo\bibsgerman{
  \renewcommand*{\ibidemname}{Ds., a.a.O.}
  \renewcommand*{\ibidemmidname}{ds., a.a.O.}
}
\renewcommand*{\samepageibidemname}{Ds., ebda.}
\renewcommand*{\samepageibidemmidname}{ds., ebda.}

%%% (5) specific configuration of bibdata in bibliography
% ever an in: before journal and collection/book-tiltes 
\renewcommand*{\bibbtsep}{in: }
%\renewcommand*{\bibjtsep}{in: }

% ever a colon after author names 
\renewcommand*{\bibansep}{: }
% ever a semi colon after the title 
\renewcommand*{\bibatsep}{; }
% ever a comma before date/year
\renewcommand*{\bibbdsep}{, }

% let jurabib insert the S. and p. information
% no S. necessary in bib-files and in cites/footcites
\jurabibsetup{pages=format}

% use a compressed literature-list using a small line indent
\jurabibsetup{bibformat=compress}
\setlength{\jbbibhang}{1em}

% which follows the design of the cites and offers comments
\jurabibsetup{biblikecite}

% print annotations into bibliography
\jurabibsetup{annote}
\renewcommand*{\jbannoteformat}[1]{{ \itshape #1 }}

%refine the prefix of url download
\AddTo\bibsgerman{\renewcommand*{\urldatecomment}{Referenzdownload: }}

% we want to have the year of articles in brackets
\renewcommand*{\bibaldelim}{(}
\renewcommand*{\bibardelim}{)}

%Umformatierung des Reihentitels und der Reihennummer
\DeclareRobustCommand{\numberandseries}[2]{%
\unskip\unskip%,
\space\bibsnfont{(=~#2}%
\ifthenelse{\equal{#1}{}}{)}{, [Bd./Nr.]~#1)}%
}%

% Local Variables:
% mode: latex
% fill-column: 80
% End:


% language specific hyphenation
% Telekom osCompendium osHyphenation Include Module
%
% (c) Karsten Reincke, Deutsche Telekom AG, Darmstadt 2011
%
% This LaTeX-File is licensed under the Creative Commons Attribution-ShareAlike
% 3.0 Germany License (http://creativecommons.org/licenses/by-sa/3.0/de/): Feel
% free 'to share (to copy, distribute and transmit)' or 'to remix (to adapt)'
% it, if you '... distribute the resulting work under the same or similar
% license to this one' and if you respect how 'you must attribute the work in
% the manner specified by the author ...':
%
% In an internet based reuse please link the reused parts to www.telekom.com and
% mention the original authors and Deutsche Telekom AG in a suitable manner. In
% a paper-like reuse please insert a short hint to www.telekom.com and to the
% original authors and Deutsche Telekom AG into your preface. For normal
% quotations please use the scientific standard to cite.
%
% [ File structure derived from 'mind your Scholar Research Framework' 
%   mycsrf (c) K. Reincke CC BY 3.0  http://mycsrf.fodina.de/ ]
%


\hyphenation{rein-cke}




%%% (3) layout page configuration %%%

% select the visible parts of a page
% S.31: { plain|empty|headings|myheadings }
%\pagestyle{myheadings}
\pagestyle{headings}

% select the wished style of page-numbering
% S.32: { arabic,roman,Roman,alph,Alph }
\pagenumbering{arabic}
\setcounter{page}{1}

% select the wished distances using the general setlength order:
% S.34 { baselineskip| parskip | parindent }
% - general no indent for paragraphs
\setlength{\parindent}{0pt}
\setlength{\parskip}{1.2ex plus 0.2ex minus 0.2ex}


%%% (4) general package activation %%%
%\usepackage{utopia}
%\usepackage{courier}
%\usepackage{avant}
\usepackage[dvips]{epsfig}

% graphic
\usepackage{graphicx,color}
\usepackage{array}
\usepackage{shadow}
\usepackage{fancybox}

%- start(footnote-configuration)
%  flush the cite numbers out of the vertical line and let
%  the footnote text directly start in the left vertical line
\usepackage[marginal]{footmisc}
%- end(footnote-configuration)

\begin{document}

%% use all entries of the bliography

%%-- start(titlepage)
\titlehead{Literaturexzerpt}
\subject{Autor(en): \ldots}
\title{Titel: \ldots}
\subtitle{Jahr: \ldots }
\author{K. Reincke% Telekom osCompendium License Include Module
%
% (c) Karsten Reincke, Deutsche Telekom AG, Darmstadt 2011
%
% This LaTeX-File is licensed under the Creative Commons Attribution-ShareAlike
% 3.0 Germany License (http://creativecommons.org/licenses/by-sa/3.0/de/): Feel
% free 'to share (to copy, distribute and transmit)' or 'to remix (to adapt)'
% it, if you '... distribute the resulting work under the same or similar
% license to this one' and if you respect how 'you must attribute the work in
% the manner specified by the author ...':
%
% In an internet based reuse please link the reused parts to www.telekom.com and
% mention the original authors and Deutsche Telekom AG in a suitable manner. In
% a paper-like reuse please insert a short hint to www.telekom.com and to the
% original authors and Deutsche Telekom AG into your preface. For normal
% quotations please use the scientific standard to cite.
%
% [ File structure derived from 'mind your Scholar Research Framework' 
%   mycsrf (c) K. Reincke CC BY 3.0  http://mycsrf.fodina.de/ ]
%
\footnote{
This text is licensed under the Creative Commons Attribution-ShareAlike 3.0 Germany
License (http://creativecommons.org/licenses/by-sa/3.0/de/): Feel free \enquote{to
share (to copy, distribute and transmit)} or \enquote{to remix (to
adapt)} it, if you \enquote{[\ldots] distribute the resulting work under the
same or similar license to this one} and if you respect how \enquote{you
must attribute the work in the manner specified by the author(s)
[\ldots]}):
\newline
In an internet based reuse please mention the initial authors in a suitable
manner, name their sponsor \textit{Deutsche Telekom AG} and link it to
\texttt{http://www.telekom.com}. In a paper-like reuse please insert a short
hint to \texttt{http://www.telekom.com}, to the initial authors, and to their
sponsor \textit{Deutsche Telekom AG} into your preface. For normal quotations
please use the scientific standard to cite.
\newline
{ \tiny \itshape [derived from myCsrf (= 'mind your Scholar Research Framework') 
\copyright K. Reincke CC BY 3.0  http://mycsrf.fodina.de/)] }}}

%\thanks{den Autoren von KOMA-Script und denen von Jurabib}
\maketitle
%%-- end(titlepage)
%\nocite{*}

\begin{abstract}
\noindent

\cite[(ist:)][]{ifross2005a} \\
Das Werk / The work\footcite[cf.][]{ifross2005a} \\
\noindent \itshape
\ldots Erläutert die jurstischen Implikationen und die Bedeutung der GPL -
Abschnitt für Abschnitt und einschließlich der Bedeutung des Begriffs
'abgeleitetes Werk' \\
\noindent
\ldots This book explains the legal implications and meaning of each GPL section
- including the meaning of 'derivative work'
\end{abstract}
\footnotesize
%\tableofcontents
\normalsize

\section{Vertrieb gegen Entgeld}

Die Autoren unterstreichen explizit, \glqq{}[\ldots] dass man die Software
selbst zu jedem Preis anbieten und veräußern (dürfe)\grqq{} und dass es ein
davon unabhängiger Aspoekt sei, \glqq{}oib es (gelinge), einen bestimmten Preis
zu erzielen [\ldots], da ja jeder die Software weitergeben (könne) und es dehalb
einen starken Wettbewerb (gäbe)\grqq\footcite[cf.][15]{ifross2005a}.

\section{Entstehende Pflichten =  Art der TO-DO-Liste}

Sodann listen die Autoren in nuce auf, was ein NUtzer einer GPL lizenzierten
Software zu beachten, will sagen: durch sein eigenes Tun sicherzustellen habe:
So müsse der Lizenztext mitgeliefert werden, die Copyrightvermerke und der
Haftungsausschluss beim Akt des Kopierens erhalten bleiben und nach außen
durchgreicht werden, selbst wenn sie im deutschen Raum nur sehr eingteschränkt
wirksam sind, und der Quelltext müsse zugänglich gemacht
werden\footcite[cf.][16]{ifross2005a}. Außerdem gäbe es \glqq{}beim Vertrieb von
Veränderungen eines GPL-Programms\grqq{} zusätzliche Pflichten: So müssen
deutliche Änderungsvermerke aufgenommen werden, bei interaktiven Kommandos
müssen (weiterhin) der Haftungsausschluss und die Copyrightvermerke sichtbar
gemacht sein, und es gelte die zentrale Regel des 'derivative
works'\footcite[cf.][17]{ifross2005a}:

\begin{quote}
\glqq{}Wiord ein unter GPL lizenziertes Programm so verändert, dass ein
'derivative work' [\ldots] entsteht [\ldots], darf die so veränderte Software
insgesamt nur unter den Lizenzbedingungen der GPKL an Dritte weitergegeben
werden\grqq{}\footcite[cf.][17]{ifross2005a}
\end{quote}

Leider ist die Zusammenfassung als Handlungsanweisungen für Programmierer
zumindets in einer Hinsicht irreführen: Wenn man eine GPL lizenzierte Bibliothek
oder ein Modul oder ein Plugin oder ein inkludierbares File als Basis für eine
darauf aufsetzende Entwicklung nimmt, hängt dieses darauf aufsetzende Werk
sicher von der Bibliothek, dem Modul, dem Plugin oder dem inkludierbares File
ab, ohne dass diese Bibliothek, dieses Modul, dieses Plugin oder dieses
inkludierbares File dafür geändert worden sein müsste.

\section{Derivative Work}

Das zentrale Problem bei der GPL sei der \glqq{}virale
Effekt\grqq{}\footcite[cf.][64]{ifross2005a}

Initial unterstreichen die Autoren, dass \glqq{}[\ldots] in jedem Falle die
Frage (entscheidend sei), wann  ein 'derivative work', also ein 'abgeleitetes
Werk' vorliege\grqq{} und dass glqq{}die Antwort auf die Frage (entscheide), ob
der Rechtsinhaber sein Programm der GPL unterstellen (müsse), oder ob er frei in
der Lizenzwahl (sei), also eine andere Open Sourcve-Lizenz oder eine proprietäre
Lizenz verwenden (dürfe)\grqq{}\footcite[cf.][64]{ifross2005a}

\subsection{klare nicht 'derived' Fälle}
Als einen eindeutigen Fall, bei dem ein 'zugreifendes' Werk trotz technischer
Verbindung nicht abgeleitet worden sei, stellen die Autoren sodann den
Linuxkernel selbst dar: Durch eine spezielle Beudetungserklärung durch Linux
Torvalds sei festgelegt, dass der Zugriff auf 'kernel services' das zugreifende
Produkt gerade nicht zu einem vom Kernel abgeleiteten Werk mache. Deshalb wirke
die GPL2 L;izenzierung des Kernels also nicht nach
außen\footcite[cf.][66]{ifross2005a}.

Des weiteren verweisen die Autoren darauf, dass die glibc, als eine derjenigen
Bibliotheken, die die Zugriffe auf die Kernlservices kapselt, sozusagen
sicherhateilshalber selbst nochmal der LGPL unterstellt sei, die ihrerseits
gerade nicht erzwinge, dass auf sie aufsetzenden Programme ebenfalls GPL
lizenztiert werden müssen\footcite[cf.][66]{ifross2005a}

\subsection{klare 'derived' Fälle}

Hier schreiben die Autoren klar, dass ein \glqq{}[\ldots] ein eigener Code
eindeutig der GPL unterstellt werden (müsse), wenn das vorbestehende
GPL-Programm in seiner bestehenden Form geändert werden[:] Erweiterungen,
Kürzungen und Abänderungen des Codes führen stets dazu, dass das so geänderte
Programm der GPL unterstellt werden müsse [\ldots]\grqq{}, vorausgesetzt, es
werde vertrieben und nicht nur für sich selbst
genutzt\footcite[cf.][66]{ifross2005a}.

\subsection{unklare 'derived' Fälle}

Hier sagen die Autoren, dass jene Fälle \glqq{}problematisch\grqq{} seien,
\glqq{}[ldots]  in denen nicht nur eine einfache Weiterentwicklung stattfinde,
sondern ein GPL-Programm mit anderen Programmen oder Programmbestandenteilen
kombiniert (werde)\grqq{}, und dass die GPL dazu nur \glqq{}Hinweise\grqq{}
enthalte, die \glqq{}[ldots] alles andere als eindeutig
(seien)\grqq{}\footcite[cf.][67]{ifross2005a}

Nach einigen Bemerkungen fassen die Autoren folgende Regeln als Leitschnur
zusammen

\begin{quote}
\begin{enumerate}
  \item Programme oder Softwarebestandeteile, die (inhaltlich) nicht voneinander
  abgeleitet sind, können unter unterschiedlichen Lizenzen verbreitet werden,
  wenn sie auch formal getrennt vorliegen;
  \item Programme oder Softwarebestandeteile, die (inhaltlich) nicht voneinder
  abgeleitet sind, müssen aber dann insgesamt unter der GPL verbreitet werden,
  wenn sie ein 'Ganzes' bilden, weil keine formalen Trennung besteht;
  \item Programme oder Softwarebestandteile, die (inhaltlich) voneinander
  abgeleitet sind, dürfen stets nur unter der GPL gemeinsam verbreitet werden.
\end{enumerate}
\end{quote}\footcite[cf.][69]{ifross2005a}

Abgesehen, dass diese Regeln für sich genommen sprachlich viel zu scharf sind,
weil sich nicht darauf verweisen, dass immer zumindest eines der kombinierten
Teile unter GPL lizenziert sein muss, hängt die sonstige Klarheit an den
selbst unscharfen Begriffen '(inhaltlich) [nicht] voneinander abgeleitet' und
'formal getrennt'.

Insofern kann man diese Regeln nicht an Entwickler weiterreichen.

\small
\bibliography{../bibfiles/oscResourcesDe}

\end{document}
