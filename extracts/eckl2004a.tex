% Telekom osCompendium extract template
%
% (c) Karsten Reincke, Deutsche Telekom AG, Darmstadt 2011
%
% This LaTeX-File is licensed under the Creative Commons Attribution-ShareAlike
% 3.0 Germany License (http://creativecommons.org/licenses/by-sa/3.0/de/): Feel
% free 'to share (to copy, distribute and transmit)' or 'to remix (to adapt)'
% it, if you '... distribute the resulting work under the same or similar
% license to this one' and if you respect how 'you must attribute the work in
% the manner specified by the author ...':
%
% In an internet based reuse please link the reused parts to www.telekom.com and
% mention the original authors and Deutsche Telekom AG in a suitable manner. In
% a paper-like reuse please insert a short hint to www.telekom.com and to the
% original authors and Deutsche Telekom AG into your preface. For normal
% quotations please use the scientific standard to cite.
%
% [ File structure derived from 'mind your Scholar Research Framework' 
%   mycsrf (c) K. Reincke CC BY 3.0  http://mycsrf.fodina.de/ ]

%
% select the document class
% S.26: [ 10pt|11pt|12pt onecolumn|twocolumn oneside|twoside notitlepage|titlepage final|draft
%         leqno fleqn openbib a4paper|a5paper|b5paper|letterpaper|legalpaper|executivepaper openrigth ]
% S.25: { article|report|book|letter ... }
%
% oder koma-skript S.10 + 16
\documentclass[DIV=calc,BCOR=5mm,11pt,headings=small,oneside,abstract=true, toc=bib]{scrartcl}

%%% (1) general configurations %%%
\usepackage[utf8]{inputenc}

%%% (2) language specific configurations %%%
\usepackage[]{a4,ngerman}
\usepackage[english, german, ngerman]{babel}
\selectlanguage{ngerman}

%language specific quoting signs
%default for language emglish is american style of quotes
\usepackage{csquotes}

% jurabib configuration
\usepackage[see]{jurabib}
\bibliographystyle{jurabib}
% Telekom osCompendium German Jurabib Configuration Include Module 
%
% (c) Karsten Reincke, Deutsche Telekom AG, Darmstadt 2011
%
% This LaTeX-File is licensed under the Creative Commons Attribution-ShareAlike
% 3.0 Germany License (http://creativecommons.org/licenses/by-sa/3.0/de/): Feel
% free 'to share (to copy, distribute and transmit)' or 'to remix (to adapt)'
% it, if you '... distribute the resulting work under the same or similar
% license to this one' and if you respect how 'you must attribute the work in
% the manner specified by the author ...':
%
% In an internet based reuse please link the reused parts to www.telekom.com and
% mention the original authors and Deutsche Telekom AG in a suitable manner. In
% a paper-like reuse please insert a short hint to www.telekom.com and to the
% original authors and Deutsche Telekom AG into your preface. For normal
% quotations please use the scientific standard to cite.
%
% [ File structure derived from 'mind your Scholar Research Framework' 
%   mycsrf (c) K. Reincke CC BY 3.0  http://mycsrf.fodina.de/ ]

% the first time cite with all data, later with shorttitle
\jurabibsetup{citefull=first}

%%% (1) author / editor list configuration
%\jurabibsetup{authorformat=and} % uses 'und' instead of 'u.'
% therefore define your own abbreviated conjunction: 
% an 'and before last author explicetly written conjunction

% for authors in citations
\renewcommand*{\jbbtasep}{ u. } % bta = between two authors sep
\renewcommand*{\jbbfsasep}{, } % bfsa = between first and second author sep
\renewcommand*{\jbbstasep}{ u. }% bsta = between second and third author sep
% for editors in citations
\renewcommand*{\jbbtesep}{ u. } % bta = between two authors sep
\renewcommand*{\jbbfsesep}{, } % bfsa = between first and second author sep
\renewcommand*{\jbbstesep}{ u. }% bsta = between second and third author sep

% for authors in literature list
\renewcommand*{\bibbtasep}{ u. } % bta = between two authors sep
\renewcommand*{\bibbfsasep}{, } % bfsa = between first and second author sep
\renewcommand*{\bibbstasep}{ u. }% bsta = between second and third author sep
% for editors  in literature list
\renewcommand*{\bibbtesep}{ u. } % bte = between two editors sep
\renewcommand*{\bibbfsesep}{, } % bfse = between first and second editor sep
\renewcommand*{\bibbstesep}{ u. }% bste = between second and third editor sep

% use: name, forname, forname lastname u. forname lastname
\jurabibsetup{authorformat=firstnotreversed}
\jurabibsetup{authorformat=italic}

%%% (2) title configuration
% in every case print the title, let it be seperated from the 
% author by a colon and use the slanted font
\jurabibsetup{titleformat={all,colonsep}}
%\renewcommand*{\jbtitlefont}{\textit}

%%% (3) seperators in bib data
% separate bibliographical hints and page hints by a comma
\jurabibsetup{commabeforerest}

%%% (4) specific configuration of bibdata in quotes / footnote
% use a.a.O if possible
\jurabibsetup{ibidem=strict}

% replace ugly a.a.O. by ders., a.a.O. resp. ders., ebda.
% but if there are more than one author or girl writers?
\AddTo\bibsgerman{
  \renewcommand*{\ibidemname}{Ds., a.a.O.}
  \renewcommand*{\ibidemmidname}{ds., a.a.O.}
}
\renewcommand*{\samepageibidemname}{Ds., ebda.}
\renewcommand*{\samepageibidemmidname}{ds., ebda.}

%%% (5) specific configuration of bibdata in bibliography
% ever an in: before journal and collection/book-tiltes 
\renewcommand*{\bibbtsep}{in: }
%\renewcommand*{\bibjtsep}{in: }

% ever a colon after author names 
\renewcommand*{\bibansep}{: }
% ever a semi colon after the title 
\renewcommand*{\bibatsep}{; }
% ever a comma before date/year
\renewcommand*{\bibbdsep}{, }

% let jurabib insert the S. and p. information
% no S. necessary in bib-files and in cites/footcites
\jurabibsetup{pages=format}

% use a compressed literature-list using a small line indent
\jurabibsetup{bibformat=compress}
\setlength{\jbbibhang}{1em}

% which follows the design of the cites and offers comments
\jurabibsetup{biblikecite}

% print annotations into bibliography
\jurabibsetup{annote}
\renewcommand*{\jbannoteformat}[1]{{ \itshape #1 }}

%refine the prefix of url download
\AddTo\bibsgerman{\renewcommand*{\urldatecomment}{Referenzdownload: }}

% we want to have the year of articles in brackets
\renewcommand*{\bibaldelim}{(}
\renewcommand*{\bibardelim}{)}

%Umformatierung des Reihentitels und der Reihennummer
\DeclareRobustCommand{\numberandseries}[2]{%
\unskip\unskip%,
\space\bibsnfont{(=~#2}%
\ifthenelse{\equal{#1}{}}{)}{, [Bd./Nr.]~#1)}%
}%

% Local Variables:
% mode: latex
% fill-column: 80
% End:


% language specific hyphenation
% Telekom osCompendium osHyphenation Include Module
%
% (c) Karsten Reincke, Deutsche Telekom AG, Darmstadt 2011
%
% This LaTeX-File is licensed under the Creative Commons Attribution-ShareAlike
% 3.0 Germany License (http://creativecommons.org/licenses/by-sa/3.0/de/): Feel
% free 'to share (to copy, distribute and transmit)' or 'to remix (to adapt)'
% it, if you '... distribute the resulting work under the same or similar
% license to this one' and if you respect how 'you must attribute the work in
% the manner specified by the author ...':
%
% In an internet based reuse please link the reused parts to www.telekom.com and
% mention the original authors and Deutsche Telekom AG in a suitable manner. In
% a paper-like reuse please insert a short hint to www.telekom.com and to the
% original authors and Deutsche Telekom AG into your preface. For normal
% quotations please use the scientific standard to cite.
%
% [ File structure derived from 'mind your Scholar Research Framework' 
%   mycsrf (c) K. Reincke CC BY 3.0  http://mycsrf.fodina.de/ ]
%


\hyphenation{rein-cke}




%%% (3) layout page configuration %%%

% select the visible parts of a page
% S.31: { plain|empty|headings|myheadings }
%\pagestyle{myheadings}
\pagestyle{headings}

% select the wished style of page-numbering
% S.32: { arabic,roman,Roman,alph,Alph }
\pagenumbering{arabic}
\setcounter{page}{1}

% select the wished distances using the general setlength order:
% S.34 { baselineskip| parskip | parindent }
% - general no indent for paragraphs
\setlength{\parindent}{0pt}
\setlength{\parskip}{1.2ex plus 0.2ex minus 0.2ex}


%%% (4) general package activation %%%
%\usepackage{utopia}
%\usepackage{courier}
%\usepackage{avant}
\usepackage[dvips]{epsfig}

% graphic
\usepackage{graphicx,color}
\usepackage{array}
\usepackage{shadow}
\usepackage{fancybox}

%- start(footnote-configuration)
%  flush the cite numbers out of the vertical line and let
%  the footnote text directly start in the left vertical line
\usepackage[marginal]{footmisc}
%- end(footnote-configuration)

\begin{document}

%% use all entries of the bliography

%%-- start(titlepage)
\titlehead{Literaturexzerpt}
\subject{Autor(en): Julian Eckl}
\title{Titel: Die politische Ökonomie der 'Wissensgesellschaft'}
\subtitle{Jahr: 2004 }
\author{K. Reincke% Telekom osCompendium License Include Module
%
% (c) Karsten Reincke, Deutsche Telekom AG, Darmstadt 2011
%
% This LaTeX-File is licensed under the Creative Commons Attribution-ShareAlike
% 3.0 Germany License (http://creativecommons.org/licenses/by-sa/3.0/de/): Feel
% free 'to share (to copy, distribute and transmit)' or 'to remix (to adapt)'
% it, if you '... distribute the resulting work under the same or similar
% license to this one' and if you respect how 'you must attribute the work in
% the manner specified by the author ...':
%
% In an internet based reuse please link the reused parts to www.telekom.com and
% mention the original authors and Deutsche Telekom AG in a suitable manner. In
% a paper-like reuse please insert a short hint to www.telekom.com and to the
% original authors and Deutsche Telekom AG into your preface. For normal
% quotations please use the scientific standard to cite.
%
% [ File structure derived from 'mind your Scholar Research Framework' 
%   mycsrf (c) K. Reincke CC BY 3.0  http://mycsrf.fodina.de/ ]
%
\footnote{
This text is licensed under the Creative Commons Attribution-ShareAlike 3.0 Germany
License (http://creativecommons.org/licenses/by-sa/3.0/de/): Feel free \enquote{to
share (to copy, distribute and transmit)} or \enquote{to remix (to
adapt)} it, if you \enquote{[\ldots] distribute the resulting work under the
same or similar license to this one} and if you respect how \enquote{you
must attribute the work in the manner specified by the author(s)
[\ldots]}):
\newline
In an internet based reuse please mention the initial authors in a suitable
manner, name their sponsor \textit{Deutsche Telekom AG} and link it to
\texttt{http://www.telekom.com}. In a paper-like reuse please insert a short
hint to \texttt{http://www.telekom.com}, to the initial authors, and to their
sponsor \textit{Deutsche Telekom AG} into your preface. For normal quotations
please use the scientific standard to cite.
\newline
{ \tiny \itshape [derived from myCsrf (= 'mind your Scholar Research Framework') 
\copyright K. Reincke CC BY 3.0  http://mycsrf.fodina.de/)] }}}

%\thanks{den Autoren von KOMA-Script und denen von Jurabib}
\maketitle
%%-- end(titlepage)
%\nocite{*}

\begin{abstract}
\noindent
Das Werk / The work\footcite[][]{Eckl2004a} \\
\noindent \itshape
\ldots Versucht zu zeigen, dass die Idee des 'Geistigen Eigentums' von Anbeginn
an politisch umstritten gewesen ist und das Open Source die aktuelle
'State-of-the-Art'-Position daher zurecht herausfordert. Das Buch enthält eine
kurze, aber aussagekräftige Zusammenfassung der 'Open-Source'-Geschichte. \\
\noindent
\ldots This book wants to show that for ever the concept of intellectual
properties has been discussed controversially and that Open Source therefore
challanges the current 'state of the art'-position legitimately. It contains a
short but meaningful summary of the Open Source history.
\end{abstract}
\footnotesize
%\tableofcontents
\normalsize

\section{Line of Thought}

\subsection{Anspruch und Fazit}

Als Ausgangspunkt wird konstatiert, dass es \enquote{[\ldots] Akteure (gäbe), die
ein Interesse daran (hätten), erstens das Konzept des geistigen Eingentums -
\emph{analog} zu Eigentum an materiellen Gütern - auf möglichst viele Bereiche
auszuweiten, zweitens die damit einhergehenden Rechte zu intensivieren
[\ldots] und drittens den Schutz für diese immateriellen Güter zu
verbessern}\footcite[vgl.][9]{Eckl2004a}. Nimmt man dieses als die
'State-of-the-Art'-Bewegung, so gilt ebenfalls:

\begin{quote}
\enquote{Diese dominante Interpretation wird durch das Phänomen der Freien
Software herausgefordert, das die Behauptung, es würde sich bei den derzeitigen
Entwicklungen um Notwendigkeiten \emph{ohne Alternativen} handeln, nicht länger
plausibel erscheint
}\footcite[][9; herv.i.O]{Eckl2004a}
\end{quote}

So ist es denn auch ein \enquote{Ziel} der Arbeit \enquote{[\ldots] zu
zeigen, dass das Konzept des geistigen Eigentums von Anfang an ein
politisch umstrittetenes Konzept war, dessen Entwicklung nicht linar
verlief}\footcite[vgl.][12]{Eckl2004a}

Insgesamt kündigt die Arbeit denn auch gleich zu Anfang ein spezielles Fazit an:

\begin{quote}
\enquote{Am Ende wird festzuhalten sein, dass Open Source zwar nicht zu einer
Abschaffung geistiger Eigentumsrechte geführt hat, dass aber manche der
Open-Source-Lizenzen die Privatisierung von Wissen und Ideen verhindern, indem
sie geistige Eigentumsrechte als Garantien für freien Zugang nutzen.
}\footcite[][14]{Eckl2004a}
\end{quote}

Und so schließt das Buch konsequenterweise mit dem Fazit, dass sich
\enquote{Geistiges Eigentum [\ldots] in zwei zentralen Punkten vom Eigentum
an materiellen Güttern (unterscheide)}:\footcite[][139]{Eckl2004a}
\begin{itemize}
  \item Zum ersten \enquote{(begründe) Eigentum an Sachgütern ein ausschließliches
  Recht an diesen selbst [\ldots] und (schütze) vor der Intervention Dritter
  [\ldots] (während geistige Eigentumsrechte Monopole auf die Nutzung von Ideen
  (etablieren) und [\ldots] damit Handlungen
  (unterbinden)}\footcite[][139]{Eckl2004a}
  \item Zum zweiten \enquote{[\ldots] (können) Immaterialgüter beliebig
  aoft verkauft werden}, sie werden \enquote{[\ldots] nicht
  verbraucht, wenn man einzelnen Instanzen davon
  veräußert}\footcite[][139]{Eckl2004a}
\end{itemize}

Und insgesamt gelt eben:

\begin{quote}
\enquote{Die Untersuchung der historischen Debatten über geistiges Eigentum und
der Debatte über Open Source haben gezeigt, dass die Plausibilität der einzelnen
Argumente einerseits vom jeweilgen historischen Kontext und andererseits von der
spezifischen Agenda der Auseinandersetzung abhängt.
}\footcite[][140]{Eckl2004a}
\end{quote}

\subsection{Gedankengang im Einzelnen}
\begin{itemize}
  \item In dem ersten Teil \enquote{Konzeptuelle
  Grundlagen}\footcite[vgl.][15ff]{Eckl2004a} wird allgemein abgeleitet,
  dass sich \enquote{zwei Aufgaben für eine klritische Untersuchung
  geistiger Eigentumsrechte} ergäben, nämlich \enquote{erstens eine
  Suche nach im Diskurs ausgeblendeten Alternativen und zweitens eine
  Einschätzung dieser Alternativen}\footcite[vgl.][43]{Eckl2004a}
  \item Der zweite Teil mit dem Titel \enquote{Die Debatte über geistiges
  Eigentum}\footcite[vgl.][45ff]{Eckl2004a} skizziert sodann die Genese
  der Idee des 'Geistigen Eigentums' in seiner jeweiligen sozialen Gebundenheit.
  Es wendet sich gegen die Sichtweise einer von der Sache her teleologisch auf
  das moderne \enquote{Urheberrechtssystem} ausgerichteten Entwicklung,
  sodass \enquote{[\ldots] nicht der 'Wahrheitsgehalt' der vertreten Positionen
  für den Erfolg und Misserfolg verantwortlich (wären) udn (seien)}:
  Vielmehr sei aus der Geschichte zu folgern, dass \enquote{[\ldots] Wissen und
  Ideen in der jeweiligen historischen Struktur unterschiedlich organisiert
  werden (können)}\footcite[vgl.][94]{Eckl2004a}. Und wenn dies für
  historische Standpunkte gelte, dürfe das auch für die moderne
  \enquote{Vorstellung} behauptet werden, \enquote{[\ldots] die sich
  durchgesetzt (habe)} und derzufolge \enquote{[\ldots] immaterielle
  und materielle Güter \emph{analog} zu verstehen
  (seien)}\footcite[vgl.][96, herv.i.O]{Eckl2004a}:
  \begin{quote} \enquote{Der Umstand, dass Wissen und Ideen, die
  eigentlich eine Nicht-Rivalität beim Konsum aufweisen, erst durch
  ausschließliche - d.h. ausschließende - Rechte daran verknappt werden,
  wird ignoriert. Vielmehr wird der Zirkelschluss gemacht, dass es die
  Aufgabe des Marktes sei, die knappen geistigen Güter durch effioziente
  Resourcenallokation bereit zu stellen.}\footcite[vgl.][96]{Eckl2004a}
  \end{quote}
  \item Im dritten Teil über die \enquote{Freie und
  Ope-Source-Software}\footcite[vgl.][101ff]{Eckl2004a} wird sodann
  beleuchtet, in wie weit Open Source Software \enquote{[\ldots] eine
  mögliche Alternative (darstelle) und ob sie als solche wahrgenommen
  (werde)}\footcite[vgl.][99]{Eckl2004a}. Dabei bietet das Buch eine gute
  konzentrierte Darstellung der Geschichte von Open Source als
  Idee\footcite[vgl.][102]{Eckl2004a}. Die Frage nach der 'wahrgenommenen
  Alternative' wird aber nur recht kursorisch über das Referieren von
  Bundestagsdebatten und politischen Statements
  'beantwortet'\footcite[vgl.][118ff]{Eckl2004a}
\end{itemize}

So ergibt sich als allgemeiner Schluss denn auch nur das 'Fazit',
\enquote{[\ldots] dass eine kritische Haltung gegenüber den Bestrebungen
zur Ausweitung und Intensivierung geistiger Eigentumsrechte angebracht
(sei)}\footcite[vgl.][139]{Eckl2004a}, weil - wie \enquote{die
Untersuchung der historischen Debatten über geistiges Eigentum und der
Debatte über Open Source gezeigt (hätten) [\ldots]} - \enquote{[\ldots]
die Plausibilität der einzelnen Argumente einerseits vom jeweiligen
historischen Kontext und andererseits von der spezifischen Agenda der
Auseinandersetzung (abhänge)}\footcite[vgl.][140]{Eckl2004a}

\section{Specific Aspects: Zur Genese von Open Source}

\subsection{Startpunkt Open Source Idea: Ausgangspunkt 1}

Auch Eckl rekuriert auf das Entstehungsparadoxon von Open Source: der
\enquote{historische Kontext ihrer Entstehung [\ldots] (falle) im Prinzip
mit der Entstehung proprietärer Software
zusammen}\footcite[vgl.][102]{Eckl2004a}:

Eigentlicher Ausgangspunkt sei jedoch die Zeit der 'unverfassten Freiheit', in
der \enquote{bis in die 70er Jahre des vergangenen Jahrhunderts hinein [\ldots]
Software nicht als eigenständiges Produkt aufgefasst (wurde)}, sondern
vielmehr \enquote{[\ldots] entweder gemeinsam mit der Hardware ausgeliefert oder
von den Nutzern selbst geschrieben (wurde)}, wobei die Software eben auch
in Form des \enquote{Quellcodes} ausgeliefert und weitergereicht
wurde\footcite[vgl.][102]{Eckl2004a}: \enquote{Es wurde als eine
Selbstverständlichkeit empfunden, dass man den Quellkode lesen und verändern
konnte.}\footcite[][102]{Eckl2004a}. Der universitäre Nutzungskontext
führte zudem dazu, dass \enquote{insbesondere die an den Universitäten
entwickelte Software [\ldots] ebenso wie Ergebnisse wissenschaftlicher
Forschungsarbeit behandelt (wurde)}, also \enquote{[\ldots] nicht nur
innerhalb der universitären Gemeinde verbreitet, sondern auch von
anderen Wissenschaftlern und Anwendern kritisiert, getestet und
weiterentwickelt (wurde)}\footcite[vgl.][102]{Eckl2004a}

\subsection{Proprietarisierung als Herrschaftswissen: Gegenbewegung 1}

Und eben diese \enquote{Praxis} des freien Austauschs, der freien Nutzung,
Modifikation und Weitergabe \enquote{[\ldots] wandelte sich im Zuge der
aufkommenden Vorstellung, dass es sich bei Software um ein eigenes Produkt
handle}, ein Wandel, der sich \enquote{[\ldots] zum einen darin
(zeigte), dass die Verbreitung der Software in zunehmenden Maße an
restriktive Lizenzen geknüpft (wurde) [,] und zum anderen, dass es immer
weniger selbstverständlich wurde, dem Nutzer neben dem Objekt- auch den
Quellkode zu überlassen}\footcite[vgl.][102]{Eckl2004a}

Diese Bewegung hin zur Proprietarisierung manifestieren sich in zwei Bewegungen:
Über die vielfache Gründung von \enquote{'Garagen-'Firmen} durch
Programmierer entsteht eine \enquote{Abschottung von neuem Wissen} und eine
Ausdünnung der \enquote{Hackerkultur}\footcite[vgl.][103]{Eckl2004a}

\subsection{Startpunkt Open Source Idea: Ausgangspunkt 2}

Zweite Säule der freien Kultur sei eine 'Vertriebseigenschaft' von Unix gewesen,
die Eckl in Anlehnung an Grassmuck\footcite[vgl.][105 Anm 306]{Eckl2004a}
spezifziert: Obwohl von einer gewinnorientierten Firma, nämlich AT\&T
1969 entwickelt, sei Unix - ob der kartellrechtlichen Einschränkungen von AT\&T
Ende der 1960er und 1970er Jahre - zum \enquote{Selbstkostenrpeis von 50
Dollar} im universitären Bereich samt Quellcode zur Nutzung freigegeben
worde. Damit passte sich Unix in die existierende Hackerkultur ein: Der freie
Austausch von Verbesserungen kamm AT\&T ebsno zugute, wie der der \enquote{Berkely
Software Distribution}\footcite[vgl.][105f]{Eckl2004a}

\subsection{Proprietarisierung als Herrschaftswissen: Gegenbewegung 2}

Neben der Freigabe im universitären Bereichen gab es jedoch auch die
Lizenzierung von Unix an andere Firmen, die auf dieser Basis ihre
\enquote{[\ldots] eigenen Unix-Versionen entwickelten}. Nach der
\enquote{Aufspaltung} von AT\&T entfielen die kartellrechtlichen
Beschränkungen, sodass sich eine spezielle 'Sparte' - die \enquote{Unix System
Laboratories} - der \enquote{kommerziellen Verwertung von
Unix} habe widmen können. Infolgedessen wurde die Weiterentwicklung von
Unix \enquote{[\ldots] zu einem Privileg der Herrsteller bzw. war an deren
teure Quellcodelizenzen gebunden}\footcite[vgl.][106]{Eckl2004a}.

Diese Proprietarisierung schlug auf die BSD unmittelbar durch, weil es deren
Strategie angriff, \enquote{[\ldots] für Verbesserungen möglichst auf
bestehenden Quellcode zuzugreifen}\footcite[vgl.][106]{Eckl2004a}.
Konsequenterweise mussten sie jetzt bei globalen Unix Upgrades eine Lizenz
erwerden.

\subsection{Ein freies Unix als Antwort I: BSD}

Der BSD-Ausweg aus diesem Dilemma war die eine Möglichkeit: man nahm die eine
wertvolle Berkley eigene Zutat, die \enquote{TCP/IP-Netzwerkprotokolle},
stellte sie als \enquote{Networking Release I} unter BSD-Lizenz frei und
programmierte mit der Zeit alle AT\&T eigenen Teile nach. Das sei zweier
'Besonderheiten' wegen möglich gewesen: zum einen sei Unix extrem modular
angelegt, sodass einzelnen Dienstprogramme unter Beibehaltung der äußeren
Schnittstelle autonom nachprogrammiert werden konnten. Zum anderen war das
\enquote{[\ldots] rechtlich zulässig, da Copyright nicht die Idee, sondern
[nur] die Form schütze}\footcite[vgl.][107]{Eckl2004a}.

So entstand schließlich die vollständig freie BSD Unixvariante als eine Antwort
auf die Tendenz der Proprietarisierung, das 368/BSD aus denen sich dann FreeBSD,
NetBSD und OpenBSD entwickelten\footcite[vgl.][107]{Eckl2004a}

Die 1 Antwort auf die Proprietarisierung ist eine praktische Gegenreaktion: 

\begin{quote}
\enquote{Den sogenannten BSD-artigen Lizenzen ist gemein, dass die die
Vervielfältigung und Verbreitung der Software sowohl in Form des Quell- als auch
des Objektcodes erlauben. Dies gilt sowohl für veränderte als auch für
unveränderte Versionen.
}\footcite[][108]{Eckl2004a}
\end{quote}

Entscheidend bei dieser Sichtweise ist jedoch das, was zur Lizenzerfüllung getan
werden muss:

\begin{quote}
\enquote{Unveränderte Versionen müssen zu den gleichen Konditionen verbreitet
werden, während davon abgeleitete Software den Namen des Urhebers nicht
verwenden darf. Gleichzeitig dürfen abgeleitete Versionen unter eine andere
Lizenz gestellt werden.
}\footcite[][108]{Eckl2004a}
\end{quote}

Pointiert kann man also sagen, dass \enquote{Software, die unter eine
BSD-artige Lizenz gestellt wird, [\ldots] Frei Software (sei), während
davon abgeleitete Software nicht frei sein
(müsse)}\footcite[vgl.][108]{Eckl2004a}. [Und persönlich würde ich hier
ergänzen, das BSD-artige Lizenzen deutlich den ursprünglichen Urheber schützen
wollen, vor Regresseforderungen und vor irrigen Zuschreibungen möglicherweise
minderwertiger Software. Was aus BSD-Sicht nicht 'geschützt' wird, ist die
Weiterentwicklung.]

\subsection{Ein freies Unix als Antwort II: GNU}

Die zweite Antwort war keine pragmatische Antwort wie die BSD, sondern eine
\enquote{bewusste Entscheidung gegen proprietäre
Software}\footcite[vgl.][108]{Eckl2004a}: RMS hatte die Einschränkungen
durch eine Kommerzialisierung in mehrfacher Hinsicht erfahren, er hatte
sozusagen die Vertreibung aus dem
\enquote{Hackerparadies}\footcite[vgl.][109 Anm 117 - nach
Grassmuck]{Eckl2004a} erlebt: Software wurde zu einem \enquote{zu hütenden
Geheimnis}, ausgeliefert wurde der Objektcode, nicht mehr der
Programmcode, \enquote{Vertraulichkeitsvereinbarungen} mussten unterzeichnet
werden und die Wissensträger wurden aus der Community
abgeworben\footcite[vgl.][109]{Eckl2004a}. Oder anders gesagt: \enquote{Das
Gemeinschaftswissen wurde zum
Industriegeheimnis}\footcite[][109]{Eckl2004a}

Aufgrund dieser Einschränkung initialisierte Richard M Stallmann 1984 das
\enquote{GNU Projekt} mit dem Ziel einer \enquote{eigenen
Unix-Implementierung}, wobei das Akronym GNU für 'GNU is NOT UNIX' stehe
und eben nicht ausdrücken sollte, das GNU kein Unix sei, sondern \enquote{[\ldots]
zwar funktional äquivalent zum AT\&T-Unix sein würde, ober ohne dessen
Quellcode auskommen sollte}\footcite[][109]{Eckl2004a}.

Die bewusste Entscheidung mit politischer Dimension zeige sich daran, dass mit
der GPL die \enquote{Idee der Freien Software} ausgedrückt und zugleich
dafür gesorgt wurde, \enquote{[\ldots] dass die Software auch in Zukunft
frei bleiben würde}\footcite[vgl.][110]{Eckl2004a}: Die Idee der Freiheit
manifestiere sich in dem Recht, (a) Software als Quellcode zu erhalten, (b) als
Quellcode oder Objektcode weitergeben zu dürfen, (c) die Software verändern und
verändert weitergeben zu dürfen und (d) sie - wie auch immer erhalten - zu jedem
Zweck verwenden zu dürfen\footcite[vgl.][110]{Eckl2004a}. Und die Bewahrung der
Freiheit manifestiere sich in der Regel, dass \enquote{sowohl veränderte
als auch unveränderte Versionen der Software [\ldots] nur zu den gleichen
Konditionen verbreitet werden (dürfen)}\footcite[vgl.][110]{Eckl2004a}.

Die rechtliche Konstruktion der GPL ist also so, dass \enquote{die
staatlich garantiertebn geistigen Eigentumsrechte [\ldots] in der GPL
dazu verwendet (werden), den Zugang zu Freier Software zu garantieren},
nicht, wie es Lizenzen sonst tun, Nutzungsrechte einzuschränken. Und mit dieser
Verkehrung der Stossrichtung werde eben [symbolischm KR] nicht mehr von
Copyright gesprochen, sondern von
\enquote{Copyleft}\footcite[vgl.][110]{Eckl2004a} [Die Einschränkung
'symbolisch ist zu machen, weil rechtlich die Idee natürlich nur funtkioniert,
wenn das Basis legenden Coyprigth / Urheberrecht weiterhin konstitutiv gilt KR]

Historisch war es dem GNU-Projekt bis zum \enquote{Anfang der 90er Jahre
[\ldots] gelungen, fast alle relevanten Programme [des
Unix-Betriebssystems KR] neu zu
implementieren}\footcite[vgl.][111]{Eckl2004a}. Der noch fehlende Kernel
wurde von Linus Torvalds von 1991 bis zur \enquote{ersten Vollversion} 1994
beigesteuert\footcite[vgl.][112]{Eckl2004a}

\subsection{Der weniger belastete Begriff: Open Source}

Ab 1998 gab es einen weiteren Wandel: war Linux oder GNU/linux bis dahin ein
\enquote{[\ldots] 'Ausnahmephänomen', das nur für 'hackendeNerds' relevant
erschien [\ldots]}\footcite[vgl.][113]{Eckl2004a}, wurde die Idee Freier
Software und das damit verbundene Arbeitsmodell mit der Freigabe des des
\enquote{Netscape-Quellcodes} genereller
interessant\footcite[vgl.][113]{Eckl2004a}

Damit ging es auch darum, \enquote{[\ldots] sich von der mit dem Begriff
'Freie Software' verbundenen konfrontativen Haltung zu lösen} und diese
unter einem neuen Etikett \enquote{[\ldots] auch in der Geschäftswelt
akzeptabel erscheinen zu lassen[\ldots]}.\footcite[vgl.][114]{Eckl2004a}
Zu diesem Zweck wurde über Eric Raymond der neutralere Begriff
\enquote{Open Source} eingeführt.

Tatsächlich wurden und werden die beiden Begriffe \enquote{[\ldots] sowohl von den
meisten Programmierern als auch im öffentlichen Diskurs synonym
verstanden}.\footcite[vgl.][115]{Eckl2004a}. Dies liege auch daran, dass
sich \enquote{[\ldots] allmählich die Erkenntnis (durchsetze), dass
nicht-proprietäre Software nicht unbedingt anti-kommerziell sein (müsse)},
was insbesondere \enquote{[\ldots] auch von Stallman nie bestritten worden
(sei)}, hatte er doch \enquote{[\ldots] seinen Lebensunterhalt zeitweise
durch den Verkauf von Magnetbändern mit GNU-Software
verdient}\footcite[vgl.][116]{Eckl2004a}
\small
\bibliography{../bibfiles/oscResourcesDe}

\end{document}
