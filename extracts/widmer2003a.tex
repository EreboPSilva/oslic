% Telekom osCompendium extract template
%
% (c) Karsten Reincke, Deutsche Telekom AG, Darmstadt 2011
%
% This LaTeX-File is licensed under the Creative Commons Attribution-ShareAlike
% 3.0 Germany License (http://creativecommons.org/licenses/by-sa/3.0/de/): Feel
% free 'to share (to copy, distribute and transmit)' or 'to remix (to adapt)'
% it, if you '... distribute the resulting work under the same or similar
% license to this one' and if you respect how 'you must attribute the work in
% the manner specified by the author ...':
%
% In an internet based reuse please link the reused parts to www.telekom.com and
% mention the original authors and Deutsche Telekom AG in a suitable manner. In
% a paper-like reuse please insert a short hint to www.telekom.com and to the
% original authors and Deutsche Telekom AG into your preface. For normal
% quotations please use the scientific standard to cite.
%
% [ File structure derived from 'mind your Scholar Research Framework' 
%   mycsrf (c) K. Reincke CC BY 3.0  http://mycsrf.fodina.de/ ]

%
% select the document class
% S.26: [ 10pt|11pt|12pt onecolumn|twocolumn oneside|twoside notitlepage|titlepage final|draft
%         leqno fleqn openbib a4paper|a5paper|b5paper|letterpaper|legalpaper|executivepaper openrigth ]
% S.25: { article|report|book|letter ... }
%
% oder koma-skript S.10 + 16
\documentclass[DIV=calc,BCOR=5mm,11pt,headings=small,oneside,abstract=true, toc=bib]{scrartcl}

%%% (1) general configurations %%%
\usepackage[utf8]{inputenc}

%%% (2) language specific configurations %%%
\usepackage[]{a4,ngerman}
\usepackage[english, german, ngerman]{babel}
\selectlanguage{ngerman}

%language specific quoting signs
%default for language emglish is american style of quotes
\usepackage{csquotes}

% jurabib configuration
\usepackage[see]{jurabib}
\bibliographystyle{jurabib}
% Telekom osCompendium German Jurabib Configuration Include Module 
%
% (c) Karsten Reincke, Deutsche Telekom AG, Darmstadt 2011
%
% This LaTeX-File is licensed under the Creative Commons Attribution-ShareAlike
% 3.0 Germany License (http://creativecommons.org/licenses/by-sa/3.0/de/): Feel
% free 'to share (to copy, distribute and transmit)' or 'to remix (to adapt)'
% it, if you '... distribute the resulting work under the same or similar
% license to this one' and if you respect how 'you must attribute the work in
% the manner specified by the author ...':
%
% In an internet based reuse please link the reused parts to www.telekom.com and
% mention the original authors and Deutsche Telekom AG in a suitable manner. In
% a paper-like reuse please insert a short hint to www.telekom.com and to the
% original authors and Deutsche Telekom AG into your preface. For normal
% quotations please use the scientific standard to cite.
%
% [ File structure derived from 'mind your Scholar Research Framework' 
%   mycsrf (c) K. Reincke CC BY 3.0  http://mycsrf.fodina.de/ ]

% the first time cite with all data, later with shorttitle
\jurabibsetup{citefull=first}

%%% (1) author / editor list configuration
%\jurabibsetup{authorformat=and} % uses 'und' instead of 'u.'
% therefore define your own abbreviated conjunction: 
% an 'and before last author explicetly written conjunction

% for authors in citations
\renewcommand*{\jbbtasep}{ u. } % bta = between two authors sep
\renewcommand*{\jbbfsasep}{, } % bfsa = between first and second author sep
\renewcommand*{\jbbstasep}{ u. }% bsta = between second and third author sep
% for editors in citations
\renewcommand*{\jbbtesep}{ u. } % bta = between two authors sep
\renewcommand*{\jbbfsesep}{, } % bfsa = between first and second author sep
\renewcommand*{\jbbstesep}{ u. }% bsta = between second and third author sep

% for authors in literature list
\renewcommand*{\bibbtasep}{ u. } % bta = between two authors sep
\renewcommand*{\bibbfsasep}{, } % bfsa = between first and second author sep
\renewcommand*{\bibbstasep}{ u. }% bsta = between second and third author sep
% for editors  in literature list
\renewcommand*{\bibbtesep}{ u. } % bte = between two editors sep
\renewcommand*{\bibbfsesep}{, } % bfse = between first and second editor sep
\renewcommand*{\bibbstesep}{ u. }% bste = between second and third editor sep

% use: name, forname, forname lastname u. forname lastname
\jurabibsetup{authorformat=firstnotreversed}
\jurabibsetup{authorformat=italic}

%%% (2) title configuration
% in every case print the title, let it be seperated from the 
% author by a colon and use the slanted font
\jurabibsetup{titleformat={all,colonsep}}
%\renewcommand*{\jbtitlefont}{\textit}

%%% (3) seperators in bib data
% separate bibliographical hints and page hints by a comma
\jurabibsetup{commabeforerest}

%%% (4) specific configuration of bibdata in quotes / footnote
% use a.a.O if possible
\jurabibsetup{ibidem=strict}

% replace ugly a.a.O. by ders., a.a.O. resp. ders., ebda.
% but if there are more than one author or girl writers?
\AddTo\bibsgerman{
  \renewcommand*{\ibidemname}{Ds., a.a.O.}
  \renewcommand*{\ibidemmidname}{ds., a.a.O.}
}
\renewcommand*{\samepageibidemname}{Ds., ebda.}
\renewcommand*{\samepageibidemmidname}{ds., ebda.}

%%% (5) specific configuration of bibdata in bibliography
% ever an in: before journal and collection/book-tiltes 
\renewcommand*{\bibbtsep}{in: }
%\renewcommand*{\bibjtsep}{in: }

% ever a colon after author names 
\renewcommand*{\bibansep}{: }
% ever a semi colon after the title 
\renewcommand*{\bibatsep}{; }
% ever a comma before date/year
\renewcommand*{\bibbdsep}{, }

% let jurabib insert the S. and p. information
% no S. necessary in bib-files and in cites/footcites
\jurabibsetup{pages=format}

% use a compressed literature-list using a small line indent
\jurabibsetup{bibformat=compress}
\setlength{\jbbibhang}{1em}

% which follows the design of the cites and offers comments
\jurabibsetup{biblikecite}

% print annotations into bibliography
\jurabibsetup{annote}
\renewcommand*{\jbannoteformat}[1]{{ \itshape #1 }}

%refine the prefix of url download
\AddTo\bibsgerman{\renewcommand*{\urldatecomment}{Referenzdownload: }}

% we want to have the year of articles in brackets
\renewcommand*{\bibaldelim}{(}
\renewcommand*{\bibardelim}{)}

%Umformatierung des Reihentitels und der Reihennummer
\DeclareRobustCommand{\numberandseries}[2]{%
\unskip\unskip%,
\space\bibsnfont{(=~#2}%
\ifthenelse{\equal{#1}{}}{)}{, [Bd./Nr.]~#1)}%
}%

% Local Variables:
% mode: latex
% fill-column: 80
% End:


% language specific hyphenation
% Telekom osCompendium osHyphenation Include Module
%
% (c) Karsten Reincke, Deutsche Telekom AG, Darmstadt 2011
%
% This LaTeX-File is licensed under the Creative Commons Attribution-ShareAlike
% 3.0 Germany License (http://creativecommons.org/licenses/by-sa/3.0/de/): Feel
% free 'to share (to copy, distribute and transmit)' or 'to remix (to adapt)'
% it, if you '... distribute the resulting work under the same or similar
% license to this one' and if you respect how 'you must attribute the work in
% the manner specified by the author ...':
%
% In an internet based reuse please link the reused parts to www.telekom.com and
% mention the original authors and Deutsche Telekom AG in a suitable manner. In
% a paper-like reuse please insert a short hint to www.telekom.com and to the
% original authors and Deutsche Telekom AG into your preface. For normal
% quotations please use the scientific standard to cite.
%
% [ File structure derived from 'mind your Scholar Research Framework' 
%   mycsrf (c) K. Reincke CC BY 3.0  http://mycsrf.fodina.de/ ]
%


\hyphenation{rein-cke}
\hyphenation{OS-LiC}
\hyphenation{ori-gi-nal}


%%% (3) layout page configuration %%%

% select the visible parts of a page
% S.31: { plain|empty|headings|myheadings }
%\pagestyle{myheadings}
\pagestyle{headings}

% select the wished style of page-numbering
% S.32: { arabic,roman,Roman,alph,Alph }
\pagenumbering{arabic}
\setcounter{page}{1}

% select the wished distances using the general setlength order:
% S.34 { baselineskip| parskip | parindent }
% - general no indent for paragraphs
\setlength{\parindent}{0pt}
\setlength{\parskip}{1.2ex plus 0.2ex minus 0.2ex}


%%% (4) general package activation %%%
%\usepackage{utopia}
%\usepackage{courier}
%\usepackage{avant}
\usepackage[dvips]{epsfig}

% graphic
\usepackage{graphicx,color}
\usepackage{array}
\usepackage{shadow}
\usepackage{fancybox}

%- start(footnote-configuration)
%  flush the cite numbers out of the vertical line and let
%  the footnote text directly start in the left vertical line
\usepackage[marginal]{footmisc}
%- end(footnote-configuration)

\begin{document}

%% use all entries of the bliography

%%-- start(titlepage)
\titlehead{Literaturexzerpt}
\subject{Autor(en): Mike J. Widmer}
\title{Titel: Open Source Software - Urheberrechtliche Aspekte}
\subtitle{Jahr: 2003 }
\author{K. Reincke% Telekom osCompendium License Include Module
%
% (c) Karsten Reincke, Deutsche Telekom AG, Darmstadt 2011
%
% This LaTeX-File is licensed under the Creative Commons Attribution-ShareAlike
% 3.0 Germany License (http://creativecommons.org/licenses/by-sa/3.0/de/): Feel
% free 'to share (to copy, distribute and transmit)' or 'to remix (to adapt)'
% it, if you '... distribute the resulting work under the same or similar
% license to this one' and if you respect how 'you must attribute the work in
% the manner specified by the author ...':
%
% In an internet based reuse please link the reused parts to www.telekom.com and
% mention the original authors and Deutsche Telekom AG in a suitable manner. In
% a paper-like reuse please insert a short hint to www.telekom.com and to the
% original authors and Deutsche Telekom AG into your preface. For normal
% quotations please use the scientific standard to cite.
%
% [ File structure derived from 'mind your Scholar Research Framework' 
%   mycsrf (c) K. Reincke CC BY 3.0  http://mycsrf.fodina.de/ ]
%
\footnote{
This text is licensed under the Creative Commons Attribution-ShareAlike 3.0 Germany
License (http://creativecommons.org/licenses/by-sa/3.0/de/): Feel free \enquote{to
share (to copy, distribute and transmit)} or \enquote{to remix (to
adapt)} it, if you \enquote{[\ldots] distribute the resulting work under the
same or similar license to this one} and if you respect how \enquote{you
must attribute the work in the manner specified by the author(s)
[\ldots]}):
\newline
In an internet based reuse please mention the initial authors in a suitable
manner, name their sponsor \textit{Deutsche Telekom AG} and link it to
\texttt{http://www.telekom.com}. In a paper-like reuse please insert a short
hint to \texttt{http://www.telekom.com}, to the initial authors, and to their
sponsor \textit{Deutsche Telekom AG} into your preface. For normal quotations
please use the scientific standard to cite.
\newline
{ \tiny \itshape [derived from myCsrf (= 'mind your Scholar Research Framework') 
\copyright K. Reincke CC BY 3.0  http://mycsrf.fodina.de/)] }}}

%\thanks{den Autoren von KOMA-Script und denen von Jurabib}
\maketitle
%%-- end(titlepage)
%\nocite{*}

\begin{abstract}
\noindent
Das Werk / The work\footcite[][]{Widmer2003a} \\
\noindent \itshape
\ldots Spezifiziert sehr kompetent nebst Open Source Begriffen auch die Open
Source Geschichte. Grenzt kurz 'Non Copyleft Lizenzen' und 'Copyleft Lizenzen'
gegen Formen der propritären resp. kommerziellen Software etc. ab. Analysiert
sodann - von der Systematik her - die Beziehungen von Open Source Software und
(schweizer) Urheberrecht und - von der Paragrafenabfolge her - die Beziehungen
zwischen GPL und (schweizer) Urheberrecht.\\
\noindent
\ldots The book explains concepts of Open Source movement and its history
competently. It also describes 'Non Copyleft Licenses', 'Copyleft Licenses' and
other forms like proprietary or commercial licenses etc. Then it analyzes
systematical connections between Open Source Software and the (Europaen/Swiss)
'Urheberrecht' (Copyright). And finally it analyzes - clause by clause - the
links from the GPL into the 'Urheberrecht' and vice versa.
\end{abstract}

\footnotesize
\tableofcontents
\normalsize

\section{Line of Thought}

Der grobe Gedankengang von Widmer ist folgender: (a) Es gibt \enquote{Open
Source Software} im Sinne einer spezifischen
Bedeutung\footcite[vgl.][21ff]{Widmer2003a} und einer längeren
Geschichte\footcite[vgl.][8ff]{Widmer2003a}. Also wäre (b) zu fragen, wie das
ebenfalls existierende Urheberrecht die Intentionen der Open Source Idee am
besten unterstützt\footcite[vgl.][47ff]{Widmer2003a}. Und es wäre (c)
schließlich zu beleuchten, ob und wie eine Open Source Lizenz, nämlich die GPL,
das Unterstützungspotential des (schweizer) Urheberrechts zur Verwirklichung des
Open Source Konzeptes abruft\footcite[vgl.][102ff]{Widmer2003a}.

\subsection{Open Source: Definitionen und Geschichte}

\subsubsection{History}

Für Widmer beginnt die Geschichte des
\enquote{Open}\footcite[vgl.][8]{Widmer2003a} aus dem Begriff 'Open Source'
mit dem \enquote{Unbundling} von Hardware und Software, wie es ab 1969 von
IBM eingeführt worden sei. Erst dies habe den Softwaremarkt entstehen
lassen\footcite[vgl.][9]{Widmer2003a}. Dem zur Seite gestanden habe die
Unix-Entwicklung ab 1969 durch AT\&T, die als Firma - ihrer markt-beherrschenden
Stellung wegen - allerdings schon ab 1956 nur innerhalb des Telefonsektors tätig
sein durfte. Nicht verboten gewesen sei ihr aber die Verteilung dieser Software
in den Universitäten, und zwar als freie Software ohne Support. Mit dieser
Konstellation sei dann die freie Selbsthilfe innerhalb der Universitäten
entstanden\footcite[vgl.][10f jener Arbeitsstil, wie von außen zu ergänzen wäre
- in den Stallmann hineingekommen und den er so konstitutiv zu schätzen gelernt
hatte.]{Widmer2003a}

Die nächste Stufe der Entwicklung war die Proprietarisierung von Unix, die
dadurch entstanden sei, dass ab \enquote{ 1982 [\ldots] verschiedene
Hersteller darunter IBM, HP und DEC für ihre eigenen Hardwareplattformen
angepasste Varianten des Betriebssystems UNIX proprietär zu
verbreiten}\footcite[vgl.][12 original zitiert nach Müller:
Die Wurzeln freier Software]{Widmer2003a}. Diese Bewegung sei durch die
Aufhebung der kartellrechtlichen Einschränkung für AT\&T verschärft worden, die
nun die nächste Version des Unix nicht mehr frei im Sourcecode vertreiben
mussten\footcite[vgl.][12]{Widmer2003a}

In diese Zeit der Code-Unerreichbarkeit falle dann Richard Stallmanns
Gegenmaßnahme, das GNU-Projekt: Initial auf das Problem von  Programmen ohne
Codezugriff über einen nicht funktionierenden Drucker
aufmerksamgeworden\footcite[vgl.][13]{Widmer2003a}, habe er dann den Schritt in
das proprietäre Unix derselben Nachteile wegen gerade nicht getan, sondern sich
die Entwicklung eines 'eigenen' freien Unix zum Ziel
gemacht\footcite[vgl.][13]{Widmer2003a} : \enquote{Die Nähe zu Unix, aber auch die
klare Trennung von der neuen propritären Art und Weise der Verbreitung von
[Seitenwechsel] Unix wird im Namen des neuen Systems ersiichtlich: Das rekursive
Akronym 'GNU' steht nälich für GNU's not
UNIX}\footcite[][13f]{Widmer2003a}

Der nächste Schritt der Generalisierung ist die Definition der Free Software im
Sinne von Stallmann, derzufolge Frei Software dann vorliege, wenn man die
Freiheit habe, das Programm (a) \enquote{[\ldots] für jeglichen Zweck zu
benutzen}, es (b) frei \enquote{[\ldots] zu studieren und seine
Bedürfnissen anzupassen}, es (c) in Kopien uneingeschränkt
\enquote{weiterzuverteiben} [Seitenwechsel] und es (d) \enquote{[\ldots] zu
bessern (modifizieren) und diese Verbesserungen (Modifikationen) zu
veröffentlichen [\ldots]}\footcite[vgl.][14f]{Widmer2003a}. Diese
Freiheiten \enquote{[\ldots] (setzen) die Verfügbarkeit des Quellcodes
voraus} und der Begriff 'freie Software' lasse sich \enquote{[\ldots]
als Summe einzelner Freiheiten verstehen, welche die an einem
Computerprogramm bestehenden Nutzungsrechte
charakterisieren}\footcite[vgl.][15 Widmer selbst sagt, dass
nur die ersten drei Freiheiten den Quellcode voraussetzen. Ich
meine, dass das insbesondere auch für die Modifikation gilt: nur
der Code macht praktisch die Verbesserung möglich]{Widmer2003a}

Das Problem der \enquote{Free Software Definition} ist, dass sie selbst
\enquote{[\ldots] nicht darüber (aussagt), wie die von ihre erwähnten Freiheiten
mit dem Computerprogramm verbunden sind bzw. ob diese auch Dritten gewährt
werden [\ldots]}\footcite[vgl.][15]{Widmer2003a}. Dieses Dilemma werde mit
die Erfindung des \enquote{Copylefts} aufgelöst, bei dem mit den Mitteln des
Urheberrechts Nutzungs-, Modifikations- und Vertriebsrechte an Dritte übertragen
werden\footcite[vgl.][16 Referiert Stallmann, what is copyleft
http://www.fsf.org/licenses/licenses.html\#WhatIsCopyleft]{Widmer2003a}

Nach Erfindung des Coyplefts und der GPL stand mit zunehmender Verbreitung die
Gründung der FSF auf dem Plan\footcite[vgl.][16f]{Widmer2003a}.

Diese Phase der Geschichte 'gipfelt' dann in \enquote{GNU/Linux}, also mit
der 'Inauguration' eines neuen Unix-Kernels von und durch Linus Thorvald am
17.09.1991\footcite[vgl.][17]{Widmer2003a}, wobei \enquote{die ransante
Entwicklung von GNU/Linux} eben nicht nur dem Arbeitseifer von Linus
geschuldet sei, sondern insbesondere \enquote{[\ldots] auch dem Umstand zu
verdanken (sei), das Torvalds beinahe sämtliche für ein Betriebssystem
benötigten Programme in freier Form zur Verfügung standen}, sodass es
durch aus gerechtfertigt sei, wenn \enquote{[\ldots] Stallmann Wert darauf lege,
dass das gesamte Betriebssystem als 'GNU/Linux' ud nicht bloss als 'Linux'
bezeichnet wird}\footcite[vgl.][18]{Widmer2003a}

Der Schritt 'weg' vom Begriff 'Free Software' hin zum Begriff 'Open Source'
beginnt dann 1997 mit dem Artikel 'The Cathedral and the Bazaar' von Eric S.
Raymond, aufgrund dessen großer Beachtung in Wirtschaftskreisen es galt,
\enquote{[\ldots] einen weniger belasteten Begriff zu wählen}, wenn das
\enquote{enorme Potential für die weitere Verbreitung des Grundgedankens freier
Software} erfolgreich gehoben werden
sollte\footcite[vgl.][19]{Widmer2003a}.

Ausgangspunkt für die Open Source Definition seien sodann die
\enquote{Debian Free Software Guidelines} geworden, aus denen durch
Generalisierung die OSD entstanden sei\footcite[vgl.][20]{Widmer2003a}

Nun folgt ein kommentiertes Referat der 10 OSD Kriterien.

Schließlich vergleicht Widmer die Bedeutung der Open Source Definition und der
Free Software Definition\footcite[vgl.][28 - Zur Erinnerung: beiden gemeinsam
ist, dass sie notwendige Kriterien von Lizenzmodellen spezifzieren, sie selbst
geben keine Handlungsanleitungen zur Umsetzung. Das tun die Lizenzen. Deshalb
können auch nur aus den Lizenzen selbst jeweils abgeleitet werden, was konkret
zu deren Erfüllung zu tun ist.]{Widmer2003a}. Insgesamt sei - analog zur
MITRE-Studie - festzustellen, \enquote{[\ldots] dass die beiden Definitionen
grundsätzlich identisch (seien)}\footcite[vgl.][28]{Widmer2003a}: So werde
beispielsweise von Perens die OSD als Ableitung der FSD gesehen und Raymond sehe
den Unterschied in der Kommunikationsstrategie und dem Marketing. Nur Stallmann
betone den 'fundamentalen' Unterschied\footcite[vgl.][29]{Widmer2003a}
 
 Geschichtlich ging es dann 1998 mit der Gründung der Open Source Initiative
 weiter\footcite[vgl.][29]{Widmer2003a}, gefolgt von einer Konferenzfestlegung,
 derzufolge \enquote{ [\ldots] der Begriff 'Open Source' zukünftig auch in
 Entwicklerkreisen verwendet werden solle}\footcite[vgl.][30
 originalquelle http://press.oreilly.com/pub/pr/796]{Widmer2003a} Einen
 besonderen Schub erhielt die Bewegung dann noch durch die Halloween Dokumente
 (http://www.opemsource.org/halloween/halloween1.php), in denen - gross Feind,
 gross Ehr - MS die Gefahr von Open Source beschrieb und als Mittel zur
 Bekämpfung die ms-isierung von eigentlich öffentlichen Schnittstellen empfohlen
 worden sein\footcite[vgl.][30]{Widmer2003a}
 
 \subsubsection{Open Source Lizenzen Typologie}
 
 Widmer unterscheidet auf Open Source Seite nur zwischen Non Copyleft Lizenzen
 und Copyleft Lizenzen: 
 
 Erstere \enquote{[\ldots] erlauben [\ldots] die Verwendung des Quellcodes auch in
 proprietären Produkten}\footcite[vgl.][37]{Widmer2003a} - wobei
 vorausgreifend hinzuzufügen ist, dass Widmann 'proprietäre Lizenzen' resp.
 Software ex negativo definiert: sie liegt dann vor, wenn die Nutzung von der
 Lizenz her eingeschänkt oder der Quellcode nicht offengelegt
 ist\footcite[vgl.][39]{Widmer2003a}. Als Prototyp gilt die
 BSD-Lizenz\footnote{Widmer unterstreicht, dass die ältere Version der BSD
 Lizenz eine Marketing Klausel enthielt, derzufolge in jeder Werbeaktion und
 -form daraufhingewiesen werden musste, dass darin ein produkt der Universität
 Berkeley verwendet worden ist\cite[vgl.][37]{Widmer2003a} Diese Klausel sei
 nun nicht mehr enthalten. Und meines Wissens bezieht sich die Unvereinbarkeit
 von GPL und BSD - frühform auf eben diese Klausel.}, weitere Lizenzen seien die
 Artistic License 2.0, die MIT-License oder die Apache
 License\footcite[vgl.][38]{Widmer2003a}
 
 Letztere wollen die Reproprietarisierung über ein \enquote{koordiniertes
 Zusammenspiel des Urheberechts und spezieller Lizenzbestimmungen}
 verhindern. Beispiel dafür seien die GPL. die Open Software License und die
 Affero General Public License\footcite[vgl.][38]{Widmer2003a}
 
 [Kommentar: Interessanterweise wird nicht zwischen ausgreifenden Lizenezn
 GPL/AGPL und nicht ausgreifenden Lizenezn LGPL unterschieden. Könnte es sein,
 dass die European Public License ebefalls nicht ausgreifend, aber Copy Left
 ist?]
 
 Zuletzt bietet Widmer einige andere Begriffsdefinitionen an:
 \begin{description}
  \item[``Proprietäre Software''] seien solche \enquote{Computerprogramme},
  die \enquote{[\ldots] durch restriktive Lizenzbedingugen keine umfassende
  Nutzung der Software erlauben und bei welchen der Quellcode nicht frei
  verfügbar ist}\footcite[vgl.][39]{Widmer2003a} - eine ex negativo
  Definition, derzufolge man berechtigt dann auch von \enquote{Closed
  Source Software} sprechen dürfe\footcite[vgl.][39]{Widmer2003a}
  \item[``Kommerzielle Software'']  könne inhaltlich nicht recht abgegrenzt
  werden, denn es könne aus diesem Begriff nicht abgeleitet werden,
  \enquote{[\ldots] welche Nutzungsrechte dem Anwender (zustünden) [\ldots]
  und ob der Quellcode frei verfügbar
  (sei)}\footcite[vgl.][40]{Widmer2003a}
  \item[``Freeware''] meine Programme, die man \enquote{[\ldots]
  kostenlose gebrauchen, vervielfältigen und verbreiten (dürfe)}, die aber
  keinen Zugriff auf den Quellcode erlauben
  würden\footcite[vgl.][40]{Widmer2003a}. Diese Art der Software gäbe es zudem
  in verschiedenen \enquote{Untergruppen}, wie etwa
  \enquote{Postcardware, Donationsware, Ideaware oder
  Mailware}\footcite[vgl.][41]{Widmer2003a}
  \item[``Shareware''] \enquote{[\ldots] unterscheide sich von anderer
  [propritetärer; KR.] Software nur durch das Vertriebs- und
  Marketingkonzept}: das freie Ausprobieren soll Kunden anlocken, nach der
  Entscheidung für die Software sind normalerweise Lizenzgebühren etc.
  fällig\footcite[vgl.][42]{Widmer2003a}
  \item[``Public Domain''] zeichne sich dadurch aus, dass \enquote{[\ldots]
  der Hersteller auf jegliche ihm zustehden Urheberrechte (verzichte) und
  so die uneingeschränkte Nutzung (ermögliche)
  [\ldots]}\footcite[vgl.][44]{Widmer2003a}. Wenn zudem \enquote{[\ldots]
  (die Verfügbarkeit des Quellcodes) gegeben (sei), (handele) es sich um
  Open Source Software}\footcite[vgl.][44. Dies ist ein
  gefährliches Fazit insofern, als Widmer selbst die Public Domain
  Software soäter als unzureichend zur Erfüllung des Open Source
  Gedankesn klassifizieren wird]{Widmer2003a}
  \item[``Shared Source''] sei die Form (vom Microsoft), bei der der Quellcode
  ohne Rechte offengelegt werde\footcite[vgl.][4ff]{Widmer2003a}
\end{description}

\subsection{Open Source Software und Urheberrecht}

Bzgl. der Verbindung von Open Source Software und (schweizer) Urheberrecht
konstatiert Widmer folgendes:
\begin{itemize}
  \item Ob ein Streitfall vor Schweizer Gerichten verhandelt werden müsse, könne
  nur im Einzelfall entschiedenen werden, es gäbe zu viele einzelne
  \enquote{Konstellationen}, die einer Generalisierung
  entgegenstünden\footcite[vgl.][53]{Widmer2003a}
  \item Insofern \enquote{Open Source Software} auch \enquote{Open
  Source Computerprogramme} seien, handele es sich dabei auch um
  \enquote{Computerprogramme im urheberrechtlichen Sinne}, sobald damit
  \enquote{eine geistige Schöpfung mit individuellem Charakter}
  einhergehe\footcite[vgl.][70]{Widmer2003a} - womit dann der Bogen zu der
  Festlegung geschlagen wäre, derzufolge \enquote{[\ldots]
  Computerprogramme in der Schweiz (seit 1993) als Werke im Sinne des
  Uurheberrechtsgesetzes gelten}\footcite[vgl.][2]{Widmer2003a}.
  \item Die Frage der Urheberschaft konkret zu beantworten, sei bei Open Source
  Software \enquote{[\ldots] meist äusserst schwierig}: Grund dafür sei der
  konstitutive Gedanke einer Kollaboration von \enquote{sehr vielen
  Personen}, die da \enquote{[\ldots] an der Entwicklung beteiligt
  (seien)}, was auch dazuführe, dass \enquote{[\ldots] sich die
  verschiedenen Formen kollektiven Werkschaffens regelmässig und in
  beliebiger Reihenfolge durchmischen}\footcite[vgl.][86]{Widmer2003a}
  \item Der reine Akt der Veröffentlichung von Open Source Software reiche nicht
  aus, die intendierten Ziele der Veröffentlichung zu gewährleisten. Die
  \enquote{kör\-per\-liche Verbreitung von Open Source Software} dagegen,
  erreicht schon mehr: insofern sie als \enquote{Veräusserung} gelte, dürfe
  das veräußerte Gut auch \enquote{[\ldots] gebraucht, weiterveräussert und
  [\ldots] geändert werden}. Was dabei jedoch noch fehle, sei die
  \enquote{umfassende Befugnis zur Vervielfältigung}. Also müsse der
  \enquote{Rechteinhaber} zur Erreichung der Open Source Ziele
  \enquote{[\ldots] neben der Verbreitung seiner Open Source Software auch
  entsprechende Urheberrechte übertragen oder daran Nutzungsrechte
  einräumen}\footcite[vgl.][98]{Widmer2003a}.
  \item Und zuletzt zeigt Widmer, dass nur die \enquote{explizite
  Einräumung von Nutzungsrechten}\footcite[vgl.][100]{Widmer2003a} als
  adäquates Mittel angesehen werden kann:
  \begin{itemize}
    \item So ermögliche der\enquote{Verzicht von Urheberrechten} zwar
    \enquote{die Entwicklung von Open Source Software}, gleichwohl stelle
    diese Methode der Veröffentlichung als \enquote{gemeinfreie Software
    }eben nicht sicher, \enquote{[\ldots] dass jedermann in der
    Verbreitungskette die gleichen Freiheiten
    hat}\footcite[vgl.][98]{Widmer2003a}: 
    \begin{quote} \enquote{Der
    Verzicht von Urheberrechten, wie immer dieser auch begründet werden man, ist
    somit kein probates Mittel zur Entwicklung von Open Source
    Software}\footnote{\cite[][98f]{Widmer2003a} So sympathisch dieses
    Fazit auch sein mag, so wenig genug differenziert es: die BSD/MIT Lizenzen
    schließen es eben explizit nicht aus, dass die gewährten Freiheiten im
    Rahmen einer proprietären Weiterverbreitung wieder entzogen werden.}
    \end{quote}
    \item Auch das \enquote{Übertragen von Urheberrechten als Ganzes oder
    Teilen} führe nicht weiter: einerseits könne das in praxi dieselben
    Konsequenzen haben, andererseits sei eine fortlaufende sukzessive Übetragung
    an mehrere Personen eben des vorgehenden Übertrags gerade nicht mehr
    möglich\footcite[vgl.][99]{Widmer2003a}
    \item Erst die \enquote{Einräumung von Nutzungsrechten} bringt ein
    Lösung: Hier \enquote{(bleibt) der Lizenzgeber weiterhin Inhaber des
    zur Nutzung eingeräumten Urheberteilrechts [\ldots]} und kann folglich
    \enquote{[\ldots] weitere Lizenzen
    vergeben}\footcite[vgl.][99f]{Widmer2003a}. Wirklich sichergestellt
    werde die Idee von Open Source Software allerdings nicht schon bei der
    \enquote{konkludenten Einräumung von Nutzungsrechten}, sondern erst bei
    der \enquote{expliziten}\footcite[vgl.][100]{Widmer2003a}
\end{itemize}
   
Man darf grosso modi also konstatieren, dass Widmer die Notwendigkeit einer
Urheberrechts basierten Liznezierung, wie sie mindestens der Schöpfer der GPL
gesehen hat, nachträglich rechtfertig: Anders gesagt: Ohne  Lizenz geht es nicht.
\end{itemize}

\subsection{GPL und Urheberrecht}

\subsubsection{Copyright Vermerk}
Widmer unterstreicht, dass es - trotz Aufhebung ehemals geltender Bedingungen
(in den USA) - immer noch sinnvoll ist, den Copyright-Vermerk in jeder Datei
anzubringen, es erleichtert die Klärung von
Streitigkeiten\footcite[vgl.][107]{Widmer2003a}

\subsubsection{Derivative work}

Widmer klärt sodann, dass sich die GPL auch auf andere Arten von Werken
beziehen kann, also nicht nur auf \enquote{Computerprogramme}, sondern etwa
auch auf \enquote{Anwenderdokumentationen, XML oder
HTML-Dateien}\footcite[vgl.][114]{Widmer2003a}.  Dabei stelle sich dann
neu und anders die Frage nach dem 'auf dem Programm basierenden Werke'. Und zu
eben deren Klärung verweist Widmer auf §101 USC, demzufolge \enquote{A
'derivative work' is a work based upon one or more preexisting works,
such as a translation, musical arrangement, dramatization [\ldots es
folgen eine Menge an Beispielen] or any other form in which a work may be
recast, transformed, or adapted}\footcite[vgl.][114 Anm.
540]{Widmer2003a}. Zugleich ergänzt er, dass solche abgeleiteten Werke damit
unter den Begriff des \enquote{Werkes zweiter Hand} im Urheberrecht
fielen\footcite[vgl.][114f]{Widmer2003a}

Dazu gehört dann auch die Spezifikation, dass \enquote{unter 'modification'
[\ldots] alle Aktivitäten zu verstehen (seien), mit welchen (ein)
Computerprogramm in irgendeiner Art und Weise verändert
(werden)}\footcite[vgl.][123]{Widmer2003a}. Dabei spiele der Umfang der
Änderung gerade keine Rolle, sodass dieser Aspekt nicht nur unter das
\enquote{Bearbeitungsrecht}, sondern auch das \enquote{Änderungsrecht}
falle\footcite[vgl.][124]{Widmer2003a}

Ein besonderes urheberrechtliches Problem bei der Modifikation sei zudem, dass
der Urheber ein Anrecht habe, \enquote{Entstellungen} seines Werkes
entgegentreten zu könnnen, \enquote{[\ldots] welche die Ehre oder das
berufliche Ansehen des Entwicklers zu beinträchtigen
(vermöge)}\footcite[vgl.][135]{Widmer2003a}. Die von der GPL gefordert
\enquote{datierten deutlichen Änderungsvermerke} seien jedoch geeignet,
solche Befürchtungen zu entkräften\footcite[vgl.][136]{Widmer2003a}

Kern der Distribution modifizierten GPL-Codes ist sodann die Idee des Coyplefts:
\enquote{Der einmal freie Quellcode soll und wird durch diese Bestimmung,
welche eine Art 'Weitervererbung' der Lizenz verschreibt, immer frei
bleiben}\footcite[][137f]{Widmer2003a}. Und was warum als Modifikation
zu betrachten ist, wird sodann näher diskutiert:
\begin{itemize}
  \item Zum ersten gäbe es den Fall der modifizierten
  \enquote{GPL-Quellcodedatei}. Das Ergebnis - das neue \enquote{Werk} -
  \enquote{[\ldots] (enthalte dann) klarerweise Codebestandteile eines
  vorbestehenden GPL-Programs und (falle) somit unter diese Bestimmung}
  der Redistribution unter GPL\footcite[][138]{Widmer2003a}
  \item Der zweit Fall betrifft die \enquote{Übernahme von Codefragmenten
  in den eigenen Quellcode}: auch dieser Fall werden eindeutig von der
  Redistributionsregel subsumiert\footcite[][138]{Widmer2003a}.
  \item Der dritte Fall betrifft den Geltungsbereich für Module. In Anlehnung an
  die FSF (http://www.fsf.org/licenses/gpl-faq.html\#MereAggregation)
  konstatiert Widmer die technischen Kriterien der Kommuniationsart und der Informationsart.
  Danach gilt das Linken von Modulen und das \enquote{Ablaufen im gleichen
  Adressraum} [richtiger: im selben; KR.] als Kriterien, die das
  Abgeleitetsein bekräftigen, die Kommunikation über Pipes, Sockets oder
  Kommandozeilenarguemten verwiesen dagegen auf autonome
  Programme\footcite[][139f]{Widmer2003a}
  \item Der vierte Fall betrifft die \enquote{Softwarebibliotheken}: sie
  können dynamisch und statisch gelinkt sein\footcite[][141]{Widmer2003a}. Für
  beide habe die Free Software Foundation jedoch bereits festgelegt, dass
  \enquote{[\ldots] sowohl das statischen wie auch das dynamische Verbinden
  ein einheitliches Werk betrachtet (werde)}\footcite[][142
  Belegt wird das durch den Rückgriff auf die LGPL-Lizenz, die
  genau dieses in Abs. 10 der Präambel so festlege]{Widmer2003a}. Allerdings
  gäbe es dazu (2003) \enquote{noch keine gefestigte rechtliche
  Meinung}\footcite[][142]{Widmer2003a}. Deshalb habe Linus Torvalds
  zumindest für den Kernel festgelegt, \enquote{[\ldots] dass
  Benutzerprogramme, welche auf üblichem Wege die Kerneldienste
  verwenden, nicht der GPL unterstellt werden
  (müssten)}\footcite[][142 Original unter DiBona Torvalds
  Kernelmodule.]{Widmer2003a}. Allerdings gelte diese Freistellung - so
  erneut Torvald - wiederum nicht für Kernelmodule\footcite[][142
  Original
  http://www.uwsg.iu.edu/hypermail/linux/kernel/2010.2/0603.html]{Widmer2003a}
\end{itemize}

Interessant in diesem Zusammenhang ist insbesondere der Hinweis von Widmer auf
Rosen, demzufolge es erst zuletzt un in den wenigsten Fälle auf technische
Kriterien ankomme. Vielmehr gelte es, die Intentionen des ursprünglichen
Entwicklers, des Urhebers in Rechnung zu stellen (Details siehe Kap. 2.4.2
'Kriterien fürs Abgeleitet sein')

\subsubsection{Fazit}

Insgesamt konstatiert Widmer, dass sich die \enquote{[\ldots] (Bestimmungen der
GNU Public License) ohne grössere Schwierigkeiten ins schweizerische
Urheberrecht einbetten (ließen)}, und das obwohl sie \enquote{[\ldots]
insgesamt ziemlich stark von den bisher beaknnten Regelungen proprietärer
Softwarelizenzen abweichen (würden)}\footcite[vgl.][182]{Widmer2003a}: Die
GPL stelle \enquote{[\ldots] eine auflösend bedingte, einfache Gratis- bzw.
Freilizenz dar, welche umfassende Nutzungsrechte an jedermann (einräume)},
und sie \enquote{[\ldots] charakterisiere sich insbesondere durch die
uneingeschränkte Einräumung von Nutzungsrechten, das Copyleft und das
Lizenzgebührenverbot}\footcite[vgl.][182]{Widmer2003a}.

\subsubsection{Copyleft Lücke}
Interessanterweise findet Widmer eine \enquote{Lücke im Regelungsobjekt}:
Der Gedankengang geht in etwa so: Es gibt Patch-Verfahren, die \enquote{[\ldots]
keinen Quellcode der ursprünglichen Datei enthalten [\ldots]}. Sie können
somit auch nicht als \enquote{abgeleitetes Werk im Sinne der GPL} gelten.
Mithin wäre es möglich, aus einem GPL lizeniert Programm etwas Neues zu
entwickeln, dass zwar faktisch via Pathdateien und davon unabhängigem
Patchprogrann jeweils immer neu aus dem ursprünglichen GPL-Code 'abgeleitet
wird, aber diese Patchdateien können ohne unter die GPL-Pflichten zu fallen,
weitergegeb / weiterverkauft / spezial lizenziert
werden.\footcite[vgl.][144f]{Widmer2003a}

[Hint: Ich frage mich, ob das richtige ist. M.W. erscheinen zumindest im patch
Programm selbst doch die zu ersetzenden Zeilen.]


\subsubsection{Redistribution und Distribution}

Widmer konstatiert, dass die GPL in sich unheitlich mal von 'Distribution' und
mal von 'Redistribution' spreche, während andere Lizenzen einheitlich nur von
Distribution oder nur von Redistribution sprechen
würden\footcite[vgl.][117 Anm 552]{Widmer2003a}


\subsubsection{Pflichten der unveränderten Weitergabe}

Widmer konstatiert für winw GPL adäquate \enquote{Verbreitung} folgende
\enquote{drei kumulativ zu erfüllende Bedingungen}: es müsse (a)
\enquote{mit jeder Kopie [\ldots] ein deutlich sichtbarer Copyrightvermerk
und eine Enthaftungserklärung veröffentlicht werden}, es dürften aber (b)
\enquote{[\ldots] bereits bestehende Verweise auf dei GPL und das Fehlen
einer Garantie [\ldots] nicht verändert werden} und es müsse (c)
\enquote{den Empfängern der Kopie [\ldots] zusammen mit dem
Computerprogramm eine Kopie der GPL übergeben
werden}\footcite[vgl.][127f]{Widmer2003a}

\subsubsection{Fragen der Bezahlung}
Widmer unterstreicht, das der \enquote{Grundgedanke freier Software} nicht
auf \enquote{Kostenlosigkeit} ziele, \enquote{[\ldots] sondern Freiheit
kommunizieren solle}\footcite[vgl.][131]{Widmer2003a}: Konsequenterweise
sei es sehr wohl erlaubt, sich die \enquote{Verbreitung freier Software}
oder die \enquote{Einräumung einer Garantie} geldlich vergüten zu
lassen\footcite[vgl.][131]{Widmer2003a}. Was dagegen kostenlos ist und
automatisch geschieht, ist das \enquote{Einräumen der Nutzungsrecht}, die
der Urheber - sozusagen über den Kopf des \enquote{Ersterwerbers} hinweg -
dem \enquote{Zweiterwerber} immer
einräumt\footcite[vgl.][131f]{Widmer2003a}.

Dass freie Software im Übrigen zumeist günstiger ist, als proprietäre Software,
sei - so Widmer - \enquote{[\ldots] aber nicht aufgrund einer angeblich
statuioierten Preisbeschränkung der Fall, sondern dem
Lizenzgebührenverbot zuzuschreiben}\footcite[vgl.][132]{Widmer2003a}

\begin{quote} \enquote{Dies [das Lizenzgebührenverbot; KR.] erlaubt es
jedem Empfänger, kostenlos selber Werkexemplare herzustellen und danach
in beliebiger Stückzahl zu einem beliebigen Preis zu verbreiten. Das
ökonomische Prinzip von Angebot und Nachfrage lässt die Preise dann in
kürzester Zeit auf ein dem Marktwert entsprechendes Niveau
fallen.}\footcite[][132]{Widmer2003a}
\end{quote}

\section{Specific Aspects}

\subsection{Autokauf versus Softwarekauf: ein bildlicher Vergleich}

Widmer stellt in einem charmanten Vergleich den Kauf eines Autos dem eines
Softwareprogramms gegenüber. Auffällig sei es demnach, dass wir Käufer
selbstverständlich davon ausgingen, unsere Autos in einer beliebigen Werkstatt
unserer Wahl pfegen, reparieren und tunen lassen zu können und zu dürfen. Ganz
anders sei das bei proprietärer Software: \enquote{solche Softwareprodukte
ähneln vielmehr Autos, bei welchen die Kühlerhaube zugeschweisst ist und
nur der Hersteller Wartungsarbeiten durchführen kann und
darf}\footcite[vgl.][1]{Widmer2003a}.

\subsection{Relevanz der GPL}
Auch Widmer folgt der Argumentation, dass die GPL die wichtigste Open Source
Lizenz sei, weil man etwa anhand von Sourceforg.net ablesen könne, wie
herausragend oft die GPL verwendet werde\footcite[vgl.][102]{Widmer2003a}

\subsection{Problem der Zwischenspeicherung}
Im Rahmen der Distribution gäbe es, so Widmer, das Problem der
Zwischenspeicherung \enquote{lediglich zum Zweck des vorrübergehenden
Gebrauchs auf einem entfernten Computer}.
Aus Sicht des Urheberrechts \enquote{[\ldots] nicht mehr von einer
Verbreitung des Werkexemplars [\ldots] gesprochen werden}, aus Sicht der
GPL könne das aber anders beurteilt werden.\footcite[vgl.][119]{Widmer2003a}.
Und dann trifft Widmer die wichtige Festlegung:
\begin{quote} \enquote{ [\ldots] die GPL ist unabhängig vom URG aus sich
selbst auszulegen.}\footcite[][119]{Widmer2003a}
\end{quote}

\subsection{virale Software}

Widmer erwähnt, dass \enquote{Copyleft} Software - natürlich \enquote{von
gegnerischer Seite} - \enquote{[\ldots] in Anlehnung an einen Virus
als 'viral' bezeichnet (werde)}\footcite[][137]{Widmer2003a} und er
veweist dabei als Urheber auf die Firma Mircosoft, die geschrieben haben soll: 

\begin{quote} \enquote{These licenses are described as viral because they
'infect' derivative programs. [\ldots] the GPL [\ldots] is the most
infectious. It attempts to subject any work that includes GPL licensed
code to the GPL. Thus, if a government or business uses even a few lines
of GPL-licensed code in a program, and then redistributes that program to
others, it would be required to provide the program under the
GPL}\footcite[zit. nach][137. Original unter
htto://www.microsoft.com/licensing/sharedsource/ssfaq.asp]{Widmer2003a}
\end{quote}

Insgesamt bleibt dazu aber nur eines zu sagen:
\begin{quote}
\enquote{Die GNU General Public License befällt [\ldots] nicht wie ein
Virus aus heiterem Himmel fremde Computerprogramme, sondern legt
lediglich die Spielregeln fest, welche der Urheber selbst der Nutzung
seiner Software zugrunde gelegt hat}\footcite[][183]{Widmer2003a}
\end{quote}

\subsubsection{Artistic Licence}
Widmer weist daraufhin, dass zu Zeiten der OSD-Definition vielfach noch auf die
Artistic Licence rekuriert wurde, die \enquote{[\ldots] in Art. 5 die
kommerzielle Verbreitung der Software nur als Bestandteil einer grösseren
Softwaredistribution, nicht aber losgelöst davon (erlaubte)}. Damit
Programme unter der Artitistic Licence auch als Open Source Software gelten
können (so z.B. Perl), ist deshalb im Kriterium der 'Free Redistribution' die
Formulierung aufgenommen worden 'The License shall not restrict any party from
selling or giving away the software as a componant of an aggregate software
distribution [\ldots]'\footcite[vgl.][21 insbesonder auch Anm 74]{Widmer2003a}

\subsubsection{Kriterien für das Abgeleitetsein}
Details sind oben unter Derivation aufgelistet.

Aber hier ein guter anderer Hinweis: 
\begin{quote} \enquote{Nach Lawrence Rosen, Rechtsvertreter der Open Source
Initiative, bedarf es in den wenigsten Fällen der technischen Beurteilung
der in Frage stehenden Verbindung [von Software]. Vielmehr soll die
Intention des Entwicklers, welcher das vorgestehende Werk geschaffen hat,
ausschlaggebend sein. Die Intention lässt sich dabei u.a. aus der Art der
Verbreitung bzw. Vermarktung feststellen}\footcite[vgl.][140]{Widmer2003a}
\end{quote}

\small
\bibliography{../bibfiles/oscResourcesDe}
\end{document}
