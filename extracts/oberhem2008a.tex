% Telekom osCompendium extract template
%
% (c) Karsten Reincke, Deutsche Telekom AG, Darmstadt 2011
%
% This LaTeX-File is licensed under the Creative Commons Attribution-ShareAlike
% 3.0 Germany License (http://creativecommons.org/licenses/by-sa/3.0/de/): Feel
% free 'to share (to copy, distribute and transmit)' or 'to remix (to adapt)'
% it, if you '... distribute the resulting work under the same or similar
% license to this one' and if you respect how 'you must attribute the work in
% the manner specified by the author ...':
%
% In an internet based reuse please link the reused parts to www.telekom.com and
% mention the original authors and Deutsche Telekom AG in a suitable manner. In
% a paper-like reuse please insert a short hint to www.telekom.com and to the
% original authors and Deutsche Telekom AG into your preface. For normal
% quotations please use the scientific standard to cite.
%
% [ File structure derived from 'mind your Scholar Research Framework' 
%   mycsrf (c) K. Reincke CC BY 3.0  http://mycsrf.fodina.de/ ]
%
% select the document class
% S.26: [ 10pt|11pt|12pt onecolumn|twocolumn oneside|twoside notitlepage|titlepage final|draft
%         leqno fleqn openbib a4paper|a5paper|b5paper|letterpaper|legalpaper|executivepaper openrigth ]
% S.25: { article|report|book|letter ... }
%
% oder koma-skript S.10 + 16
\documentclass[DIV=calc,BCOR=5mm,11pt,headings=small,oneside,abstract=true, toc=bib]{scrartcl}

%%% (1) general configurations %%%
\usepackage[utf8]{inputenc}

%%% (2) language specific configurations %%%
\usepackage[]{a4,ngerman}
\usepackage[english, german, ngerman]{babel}
\selectlanguage{ngerman}
%language specific quoting signs
%default for language emglish is american style of quotes
\usepackage{csquotes}

% jurabib configuration
\usepackage[see]{jurabib}
\bibliographystyle{jurabib}
% Telekom osCompendium German Jurabib Configuration Include Module 
%
% (c) Karsten Reincke, Deutsche Telekom AG, Darmstadt 2011
%
% This LaTeX-File is licensed under the Creative Commons Attribution-ShareAlike
% 3.0 Germany License (http://creativecommons.org/licenses/by-sa/3.0/de/): Feel
% free 'to share (to copy, distribute and transmit)' or 'to remix (to adapt)'
% it, if you '... distribute the resulting work under the same or similar
% license to this one' and if you respect how 'you must attribute the work in
% the manner specified by the author ...':
%
% In an internet based reuse please link the reused parts to www.telekom.com and
% mention the original authors and Deutsche Telekom AG in a suitable manner. In
% a paper-like reuse please insert a short hint to www.telekom.com and to the
% original authors and Deutsche Telekom AG into your preface. For normal
% quotations please use the scientific standard to cite.
%
% [ File structure derived from 'mind your Scholar Research Framework' 
%   mycsrf (c) K. Reincke CC BY 3.0  http://mycsrf.fodina.de/ ]

% the first time cite with all data, later with shorttitle
\jurabibsetup{citefull=first}

%%% (1) author / editor list configuration
%\jurabibsetup{authorformat=and} % uses 'und' instead of 'u.'
% therefore define your own abbreviated conjunction: 
% an 'and before last author explicetly written conjunction

% for authors in citations
\renewcommand*{\jbbtasep}{ u. } % bta = between two authors sep
\renewcommand*{\jbbfsasep}{, } % bfsa = between first and second author sep
\renewcommand*{\jbbstasep}{ u. }% bsta = between second and third author sep
% for editors in citations
\renewcommand*{\jbbtesep}{ u. } % bta = between two authors sep
\renewcommand*{\jbbfsesep}{, } % bfsa = between first and second author sep
\renewcommand*{\jbbstesep}{ u. }% bsta = between second and third author sep

% for authors in literature list
\renewcommand*{\bibbtasep}{ u. } % bta = between two authors sep
\renewcommand*{\bibbfsasep}{, } % bfsa = between first and second author sep
\renewcommand*{\bibbstasep}{ u. }% bsta = between second and third author sep
% for editors  in literature list
\renewcommand*{\bibbtesep}{ u. } % bte = between two editors sep
\renewcommand*{\bibbfsesep}{, } % bfse = between first and second editor sep
\renewcommand*{\bibbstesep}{ u. }% bste = between second and third editor sep

% use: name, forname, forname lastname u. forname lastname
\jurabibsetup{authorformat=firstnotreversed}
\jurabibsetup{authorformat=italic}

%%% (2) title configuration
% in every case print the title, let it be seperated from the 
% author by a colon and use the slanted font
\jurabibsetup{titleformat={all,colonsep}}
%\renewcommand*{\jbtitlefont}{\textit}

%%% (3) seperators in bib data
% separate bibliographical hints and page hints by a comma
\jurabibsetup{commabeforerest}

%%% (4) specific configuration of bibdata in quotes / footnote
% use a.a.O if possible
\jurabibsetup{ibidem=strict}

% replace ugly a.a.O. by ders., a.a.O. resp. ders., ebda.
% but if there are more than one author or girl writers?
\AddTo\bibsgerman{
  \renewcommand*{\ibidemname}{Ds., a.a.O.}
  \renewcommand*{\ibidemmidname}{ds., a.a.O.}
}
\renewcommand*{\samepageibidemname}{Ds., ebda.}
\renewcommand*{\samepageibidemmidname}{ds., ebda.}

%%% (5) specific configuration of bibdata in bibliography
% ever an in: before journal and collection/book-tiltes 
\renewcommand*{\bibbtsep}{in: }
%\renewcommand*{\bibjtsep}{in: }

% ever a colon after author names 
\renewcommand*{\bibansep}{: }
% ever a semi colon after the title 
\renewcommand*{\bibatsep}{; }
% ever a comma before date/year
\renewcommand*{\bibbdsep}{, }

% let jurabib insert the S. and p. information
% no S. necessary in bib-files and in cites/footcites
\jurabibsetup{pages=format}

% use a compressed literature-list using a small line indent
\jurabibsetup{bibformat=compress}
\setlength{\jbbibhang}{1em}

% which follows the design of the cites and offers comments
\jurabibsetup{biblikecite}

% print annotations into bibliography
\jurabibsetup{annote}
\renewcommand*{\jbannoteformat}[1]{{ \itshape #1 }}

%refine the prefix of url download
\AddTo\bibsgerman{\renewcommand*{\urldatecomment}{Referenzdownload: }}

% we want to have the year of articles in brackets
\renewcommand*{\bibaldelim}{(}
\renewcommand*{\bibardelim}{)}

%Umformatierung des Reihentitels und der Reihennummer
\DeclareRobustCommand{\numberandseries}[2]{%
\unskip\unskip%,
\space\bibsnfont{(=~#2}%
\ifthenelse{\equal{#1}{}}{)}{, [Bd./Nr.]~#1)}%
}%

% Local Variables:
% mode: latex
% fill-column: 80
% End:


% language specific hyphenation
% Telekom osCompendium osHyphenation Include Module
%
% (c) Karsten Reincke, Deutsche Telekom AG, Darmstadt 2011
%
% This LaTeX-File is licensed under the Creative Commons Attribution-ShareAlike
% 3.0 Germany License (http://creativecommons.org/licenses/by-sa/3.0/de/): Feel
% free 'to share (to copy, distribute and transmit)' or 'to remix (to adapt)'
% it, if you '... distribute the resulting work under the same or similar
% license to this one' and if you respect how 'you must attribute the work in
% the manner specified by the author ...':
%
% In an internet based reuse please link the reused parts to www.telekom.com and
% mention the original authors and Deutsche Telekom AG in a suitable manner. In
% a paper-like reuse please insert a short hint to www.telekom.com and to the
% original authors and Deutsche Telekom AG into your preface. For normal
% quotations please use the scientific standard to cite.
%
% [ File structure derived from 'mind your Scholar Research Framework' 
%   mycsrf (c) K. Reincke CC BY 3.0  http://mycsrf.fodina.de/ ]
%


\hyphenation{rein-cke}
\hyphenation{OS-LiC}
\hyphenation{ori-gi-nal}


%%% (3) layout page configuration %%%

% select the visible parts of a page
% S.31: { plain|empty|headings|myheadings }
%\pagestyle{myheadings}
\pagestyle{headings}

% select the wished style of page-numbering
% S.32: { arabic,roman,Roman,alph,Alph }
\pagenumbering{arabic}
\setcounter{page}{1}

% select the wished distances using the general setlength order:
% S.34 { baselineskip| parskip | parindent }
% - general no indent for paragraphs
\setlength{\parindent}{0pt}
\setlength{\parskip}{1.2ex plus 0.2ex minus 0.2ex}


%%% (4) general package activation %%%
%\usepackage{utopia}
%\usepackage{courier}
%\usepackage{avant}
\usepackage[dvips]{epsfig}

% graphic
\usepackage{graphicx,color}
\usepackage{array}
\usepackage{shadow}
\usepackage{fancybox}

%- start(footnote-configuration)
%  flush the cite numbers out of the vertical line and let
%  the footnote text directly start in the left vertical line
\usepackage[marginal]{footmisc}
%- end(footnote-configuration)

\usepackage{amssymb}
\begin{document}

%% use all entries of the bliography

%%-- start(titlepage)
\titlehead{Literaturexzerpt}
\subject{Autor(en): Oberhem}
\title{Titel: Vertrags und Haftungsfragen beim Vertrieb von OPen Source
Software}
\subtitle{Jahr: 2008}
\author{K. Reincke% Telekom osCompendium License Include Module
%
% (c) Karsten Reincke, Deutsche Telekom AG, Darmstadt 2011
%
% This LaTeX-File is licensed under the Creative Commons Attribution-ShareAlike
% 3.0 Germany License (http://creativecommons.org/licenses/by-sa/3.0/de/): Feel
% free 'to share (to copy, distribute and transmit)' or 'to remix (to adapt)'
% it, if you '... distribute the resulting work under the same or similar
% license to this one' and if you respect how 'you must attribute the work in
% the manner specified by the author ...':
%
% In an internet based reuse please link the reused parts to www.telekom.com and
% mention the original authors and Deutsche Telekom AG in a suitable manner. In
% a paper-like reuse please insert a short hint to www.telekom.com and to the
% original authors and Deutsche Telekom AG into your preface. For normal
% quotations please use the scientific standard to cite.
%
% [ File structure derived from 'mind your Scholar Research Framework' 
%   mycsrf (c) K. Reincke CC BY 3.0  http://mycsrf.fodina.de/ ]
%
\footnote{
This text is licensed under the Creative Commons Attribution-ShareAlike 3.0 Germany
License (http://creativecommons.org/licenses/by-sa/3.0/de/): Feel free \enquote{to
share (to copy, distribute and transmit)} or \enquote{to remix (to
adapt)} it, if you \enquote{[\ldots] distribute the resulting work under the
same or similar license to this one} and if you respect how \enquote{you
must attribute the work in the manner specified by the author(s)
[\ldots]}):
\newline
In an internet based reuse please mention the initial authors in a suitable
manner, name their sponsor \textit{Deutsche Telekom AG} and link it to
\texttt{http://www.telekom.com}. In a paper-like reuse please insert a short
hint to \texttt{http://www.telekom.com}, to the initial authors, and to their
sponsor \textit{Deutsche Telekom AG} into your preface. For normal quotations
please use the scientific standard to cite.
\newline
{ \tiny \itshape [derived from myCsrf (= 'mind your Scholar Research Framework') 
\copyright K. Reincke CC BY 3.0  http://mycsrf.fodina.de/)] }}}

%\thanks{den Autoren von KOMA-Script und denen von Jurabib}
\maketitle
%%-- end(titlepage)
%\nocite{*}

\begin{abstract}
\noindent
Das Werk / The work\footcite[][]{Oberhem2008a} \\
\noindent \itshape
\ldots Zielt auf die Klärung der Haftung bei der Weitergabe von GPL Software.
Das bedarf der Erörterung, da der Gewährleistungsausschluss in der GPL in
Deutschland nichtig ist. Damit greift ersatzweise normales Recht, hier das der
Schenkung. Der privaten Distributor oder Urheber haftet demnach nur bei
vorsätzlichem oder grob fahrlässigen Verhalten, z.B. bei arglistigem
Verschweigen von Schadroutinen. Der professionelle Distributor muss darüber
hinaus Programme per Virenscanner auf Schadroutinen überprüfen. Davon unabhängig
zu bewerten, sind seine Zusatzleistungen. Neben AGB-haftigkeit der GPL und des
Vertragstypus wird auch detailliert die Open-Source-Begrifflichkeit verhandelt.
\\
\noindent
\ldots The book analyzes the liability of authors and distributors of GPL
licensed software. This is necessary because in Germany the NO-Warranty clauses
of the GPL are not applicable. They are substituted by the rules of 'gifts'. In
this sense private authors and distributors are only liable in case of acting
deliberately ["vorsätzlich"] or strongly negligently ["grob fahrlässig"].
Additionally commercial distributors must thouroughly check whether a program
might damage a customer. But commercial distributors have the full liability for
their additionally offered specific services. Beside the 'AGB' character and the
typus of GPL as a contract the book refers the idea of Open Source seriously.
\end{abstract}
\footnotesize
%\tableofcontents
\normalsize

\section{Line of Thought}

Der grobe Gedankengang dieser Arbeit ist - von hinten betrachtet - folgender:

Es soll ermittelt werden, welche Art der Haftung bei der Distribution von Open
Source Software anzusetzen ist\footcite[vgl.][161ff]{Oberhem2008a}. Diese
Analyse wird konzeptionel nötig, weil die Analyse der GPL als
AGB\footcite[vgl.][133ff]{Oberhem2008a} ergibt, dass die 'NO-WARRANTY'-Klauseln
der GPL im deutschen Rechtsraum unzulässig und also nichtig
sind\footcite[vgl.][149ff]{Oberhem2008a}. Der Haftungsumfang wird also
gesetztlich bestimmt. Zu dessen Herleitung muss der Charakter der GPL als
Vertrag bestimmt sein\footcite[vgl.][93ff]{Oberhem2008a}. Und dazu wiederum
müssen das Wesen von Open Source\footcite[vgl.][5ff]{Oberhem2008a} und
rechtliche Grundlagen\footcite[vgl.][49ff]{Oberhem2008a} ermittelt sein.

Umgekehrt betrachtet gibt es dazu in etwa die folgenden Ergebnisse:

\subsection{allgemeine rechtliche Rahmenbedingungen}
\begin{itemize}
  \item In der juristischen Literature werde diskutiert, ob es sich bei
  \enquote{Software} um ein \enquote{immaterielles
  Gut}\footcite[vgl.][53f]{Oberhem2008a} oder eine
  \enquote{Sache}\footcite[vgl.][55f]{Oberhem2008a} handele, wobei Oberhem
  nach tiefer Diskussion folgert, dass \enquote{[\ldots] der rechtlichen
  Qualifizierung von Computerprogrammen als immaterielle Güter eine Absage
  erteilt werde (müsse)} und dass \enquote{[\ldots] ein
  Computerprogramm (vielmehr) als  körperlicher Gegenstand und damit als
  Sache gemäß §90 BGB anzusehen (sei)}\footcite[vgl.][67f]{Oberhem2008a}
  \item Des weiteren gelte die GPL - als im Übrigen \enquote{einzige
  vetragliche Vereinbarung zwischen Urheber und
  Nutzer}\footcite[vgl.][71]{Oberhem2008a}, die allerdings pikanterweise
  das \enquote{Benutzungsrecht} nicht explizit
  einräume\footcite[vgl.][72]{Oberhem2008a}, sodass zu dessen Konstitution -
  jedenfalls im deutschen Rechtsaum -  das Urheberrecht bemüht werden müsse.
  Glücklicherweise lege dieses von sich aus fest, das der legale Verwender einer
  \enquote{Programmkopie} keine gesonderte \enquote{ Zustimmung des
  Rechteinhabers} zur bloßen \enquote{Benutzung} des Programms
  benötige\footcite[vgl.][73]{Oberhem2008a}.
  \item Für das ganze Szenario entscheidend ist auch die 'Eigenart' des
  (deutschen) Urheberrechtsgesetzes, demzufolge das \enquote{[\ldots] das
  Urheberrecht [\ldots] unübertragbar (sei)}: man könne als Urheber
  \enquote{anderen Personen} nur \enquote{[\ldots] Nutzungsrechte [\ldots]
  einräumen oder in die Nutzung seines Werkes durch sie [die anderen
  Personen; KR] einwilligen}\footcite[vgl.][75]{Oberhem2008a}
  \item Desweiteres \enquote{[\ldots] (entstehe) aufgrund der Vielzahl an
  Beteiligten und der Vielfältigkeit ihrer Bezugsmöglichkeiten ein
  komplexes Netz an vertraglichen Bindungen, deren Vertragsgegenstände je
  nach Funktion der Beteiligten im Rahmen der Entwicklung und Verbreitung
  des Programms stark variieren können}\footcite[vgl.][89]{Oberhem2008a}.
  So gebe es da den \enquote{Urheber}\footcite[vgl.][78]{Oberhem2008a} und
  möglicherweiser \enquote{Miturheber}, wobei für das Ansetzen einer
  \enquote{Miturheberschaft} nicht der Umfang, sondern die Qualität des
  Beitrages anzusetzen sei\footcite[vgl.][79 und 83f]{Oberhem2008a}. Und daneben
  stünden wiederum der \enquote{Nutzer}\footcite[vgl.][85]{Oberhem2008a} und
  verschiedene Arten der Distributoren\footcite[vgl.][85ff]{Oberhem2008a}.
  \item Für die GPL sei es angesichts der Überlagerungen aber immerhin klar,
  dass dem Nutzer die Nutzungsrechte (also hier das der Nutzung i.e.S, das der
  Weitergabe und das der Modifikation) \enquote{[\ldots] direkt vom Urheber
  eingeräumt (werden)}, auch wenn er die Kopie 'nur' mittelbar vom
  \enquote{Weitergebenden} erhält\footcite[vgl.][90]{Oberhem2008a}
\end{itemize}

\subsection{\enquote{Vertragliche Einordnung des Open Source
Software-Erwerbs}}

Das (deutsche) Urheberrecht sieht vor, dass \enquote{[\ldots] (nur der)
Inhaber des Urheberrechts berechtigt (sei) Verwertungsrechte ans seiner
Software einzuräumen}\footcite[vgl.][103]{Oberhem2008a}. Dies passe zum
Ansatz der GPL, bei der der Distributor \enquote{[\ldots] lediglich als Bote
(fungiere)} und \enquote{[\ldots] die Nutzungsrechte an der Software
immer direkt vom Urheber eingeräumt
(werden)}\footcite[vgl.][104]{Oberhem2008a}. Aus beiderlei Sicht, aus der
des Urheberechts und der der GPL \enquote{[\ldots] kommt daher ein
Rechtsgeschäft zwischen Urheber und Erwerber zustande, die die
Übertragung der Nutzungsrechte aus der GPL zum Gegenstand
hat}\footcite[vgl.][104]{Oberhem2008a}.

Zentral sei die in der rechtsliteratur herrschenden Meinung, dass die GPL
\enquote{[\ldots] als ein Angebot an jedermann auf Abschluss eines
dinglichen Vertrages anzusehen ist, dessen Inhalt die Einräumung der
genannten Nutzungsrechte ist}\footcite[vgl.][104]{Oberhem2008a}

Insofern nun der Begriff der Lizenz selbst nichts über die Art des Vertrages
aussage und da trotzdem - eben wegen des 'Rechtsgeschäftes' - \enquote{[\ldots]
nicht gänzlich auf eine schuldrechtliche Beziehung zwischen den Beteiligten
verzichten werden (könne)}, müsse der Vertragstyp \enquote{rechtlich
qualifiziert werden}\footcite[vgl.][105]{Oberhem2008a}. Zur Disposition
stpnden die \enquote{Schenkung}\footcite[vgl.][106ff]{Oberhem2008a} und der
\enquote{Vertrag sui generis}\footcite[vgl.][109]{Oberhem2008a}. Nach
umfangreicher Detaildiskussion\footcite[vgl.][112ff]{Oberhem2008a} - das
logische Problem ist hier, in wie weit (a) die Schenkung das Vermögen des
Schenkenden mindert\footcite[vgl.][118ff]{Oberhem2008a} und (b) die aus
der GPL erwachsenen Pflichten dem Charakter einer Schenkung
entgegenstehen\footcite[vgl.][122ff]{Oberhem2008a} - kommt Oberhem zu dem
Schluss, dass es sich bei dem Charakter um eine Schenkung
handelt\footcite[vgl.][128]{Oberhem2008a}.

\subsection{\enquote{GPL als AGB}}

Unabhängig von Klasse des Vetrages wird das in der GPL enthaltene
\enquote{Angebot an jedermann}\footcite[vgl.][104]{Oberhem2008a} in Form
einer AGB vermittelt, der Lizenztext selbst \enquote{[\ldots] (sei)
vollständig als AGB gestaltet}, weil \enquote{der Lizenznehmer
[\ldots] auf den Inhalt der Bestimmungen keinen Einfluss nehmen (könne)},
sodass \enquote{[\ldots] der Urheber als 'Steller' der AGB anzusehen (sei)
und zwar unabhängig davon, von wem der Nutzer den Lizenztext
(erhalte)}\footcite[vgl.][133]{Oberhem2008a}.

Als AGB müsse die GPL damit auch den Bedingungen einer AGB gerecht werden, sie
müsse (a) adäquat zur Kenntnis genommen werden können und sie dürfe, um wie vom
Urheber geplant zu funktionieren, keine Bedingungen enthalten, die dem BGB
zuwiderlaufen. Diesen beiden Aspekten geht Oberhem sodann nach:

Für das zur Kenntnisnehmen gilt, dass
\begin{itemize}
  \item der Nutzer, sofern die in von und mit der GPL vorgeschlagenen Methoden
  verwendet werden, hinreichend auf Existenz und Umfang hingewiesen
  sei\footcite[vgl.][134]{Oberhem2008a}
  \item er zumindest im Akt des Ausübens seiner Recht dieser AGB
  zustimme\footcite[vgl.][135]{Oberhem2008a}
  \item die Verwendung der englischen Sprache zumindest im Computerbereich als
  adäquat betrachtet werden könne\footcite[vgl.][135]{Oberhem2008a}
  \item die Verwendung der englischen Sprache im nicht Computer
  spezifischen Endverbraucherbereich insofern akzeptabel sei, als dieser
  \enquote{[\ldots] nur dann das Recht zur Vervielfältigung, Bearbeitung
  und Verbreitung der Software erwirbt, wenn die Vereinbarung der GPL als
  wirksam betrachtet wird}. Mithin gelte auch hier, dass \enquote{bei
  einer Auslegung zu Gunsten des Verbrauchers [\ldots] desöhab die
  Formulierung der GPL-Klauseln in englischer Sprache einer zumutbaren
  Möglichkeit der Kenntnisnahme nicht
  (entgegenstehe)}\footcite[vgl.][136]{Oberhem2008a}
\end{itemize}

Für die Gültigkeit der GPL-Klauseln als AGB-Klauseln im Hinblick auf das
übergeordnete und vorgehenden BGB kommt Oberhem zu folgendem Schluss, dass
sich die Vereinbarungen aus §1, §2, §3 und §4 der GPL jeweils - mutatis mutandis
- als \enquote{[\ldots] für den Lizenznehmer [\ldots] vorteilhaft
(darstellen}\footcite[vgl.][144]{Oberhem2008a} respektive dass er dadurch
\enquote{[\ldots] nicht unangemessen benachteiligt
(werde)}\footcite[vgl.][145]{Oberhem2008a}. Gleichartig wird für §6
argumentiert\footcite[vgl.][148]{Oberhem2008a}

Anders dagegen der §5: hier gelte schlicht, dass diese Bestimmung
\enquote{unwirksam} sei\footcite[vgl.][146]{Oberhem2008a}

Besonders präker wirkt sich aus, dass §11 und §12, wo es um den
Haftungsauschluss geht - die sogenannten NO-WARRANTY-Klauseln [KR] - ebenfalls
beide schlicht \enquote{unwirksam sind}\footcite[vgl.][150f]{Oberhem2008a}.
Und eben dies macht die Untersuchung notwendig, von welcher Art die Haftungs
dennung wirklich ist, wenn sie - anders als vom Urheber geplant - in Deutschland
nicht ausgeschlossen werden kann.

Randbemerkung: Oberhem erläutert auch, dass die GPL trotz fehlerhafter Klauseln
nicht als Ganzes unwirksam wird\footcite[vgl.][159]{Oberhem2008a}

\subsection{Haftung}

Damit ist der grosse Gedankengang einfach abzuschließen: Da es sich - zumindest
bei privaten Distributoren - um eine
Schenkung\footcite[vgl.][163ff]{Oberhem2008a} handelt (oder wenigstens auch
deren Eigenarten trägt), seien jene BGB Bestimmungen gültig, \enquote{[\ldots]
nach denen der Schenker nur für arglistig verschwiegene Rechts- bzw.
Sachmängel (hafte)}\footcite[vgl.][171]{Oberhem2008a}. Im Prinzip gelte
das auch für den kommerziellen Distributore, allerdings habe dieser die
zusätzliche Pflicht, \enquote{[\ldots] die von ihm verbreitete
Programmversion auf ihre Virenfreiheit hin (zu untersuchen) und ggf. auf
mögliche Risiken bei der Verwendung
(hinzuweisen)}\footcite[vgl.][211]{Oberhem2008a}


\section{Specific Aspects}

\subsection{Was über Open Source zu sagen ist}

Oberhem folgt dem gängigen Muster, was zum Thema Open Source zu sagen ist:

Zunächst gilt es, auf die steigende Wichtigkeit von Open Source
hinzuweisen\footcite[vgl.][1f]{Oberhem2008a}, sodann gilt es die Geschichte von
Open Source zu beleuchten\footcite[vgl.][17ff]{Oberhem2008a} und - womöglich
schon darin vermischt - Open Source
Begrifflichkeiten\footcite[vgl.][6ff]{Oberhem2008a}, um schließlich auf die Open
Source Definition einzugehen\footcite[vgl.][10ff]{Oberhem2008a}.

Neben diesem Schema skizziert Oberhem auch das, was gerichtlich in Sachen Open
Source bereits entschieden ist: So datiert sie die erste
Auseinandersetzung mit \enquote{rechtlichen Aspekten des
Open-Source-Software-Vertriebs} in das Jahr
2004\footcite[vgl.][2]{Oberhem2008a} und listet noch drei weitere Entscheidungen
auf - zwei aus 2006 und eins aus 2007, bei denen es in
\enquote{einstweiligen Verfügungsverfahren} oder um
\enquote{Hauptsacheverfahren} handelt\footcite[vgl.][2f]{Oberhem2008a}, um
dann konstatierend von einem \enquote{bisher sehr begrenzten Beitrag durch
die Rechtsprechung} zu reden\footcite[vgl.][3]{Oberhem2008a}. Natürlich
ist dies insofern ein veralteter Befund, als Oberhem explizit angibt,
\enquote{[\ldots] Rechtsprechung und Literatur bis Juni berücksichtigt (zu
haben)}\footcite[vgl.][V = Vorwort]{Oberhem2008a}

\subsubsection{Open Source Begriffe}

Zunächst bemerkt Oberhem, dass 'Open Source' als Begriff durch die
\enquote{Ausweitung} der \enquote{Open-Source-Philosophie [\ldots] auch auf
andere Bereiche des Wissens} zumindest teilweise bereits \enquote{[\ldots]
als Obergriff verwendet (werde)}\footcite[vgl.][5]{Oberhem2008a}, sodass
\enquote{[\ldots] (Open Source Software) dessen wohl bekanntester Unterbegriff
(darstelle)}\footcite[vgl.][5f]{Oberhem2008a}

Insgesamt gelte also:

\begin{quote} \enquote{Kennzeichnend für Open-Source-Software ist also,
dass der Anwender neben der einfachen Nutzung der Software das Recht hat,
diese nach seinen individuellen Bedrüfnissen zu modifizieren und in
dieser bearbeitenden Version weiterzuverbreiten. Die einfache Nutzung des
Prgramms unterliegt dabei keinerlei Restriktionen. Die Freiheit des
Nutzers im Umgang mit der Software wird einzig dadurch eingeschränkt, dass
niemand das Recht haben soll, anderen dieselben Freiheiten
vorzuenthalten.}\footcite[vgl.][7]{Oberhem2008a}
\end{quote}

Leider scheint der letzte Satz eine unzulässige Vereinfachung zu sein. Diese
Bewahrung der Freiheiten auch dritten gebenüber (andere) ist ein Spezifikum der
GPL und Abarten und trifft wenigstens auf die BSD/MIT/APACE-Lizenz nicht zu. Die
Relizenzierung ist da erlaubt. Man sieht, es ist irreführend, sich bei der
Behandlung von Open Source nur auf die GPL zu konzentrieren und von dieser
wieder auf die Allgemeinheit zurückzuschließen. Und genau das tut Oberhem, wenn
sie zuerst sagt, dass \enquote{die General Public License (GPL) [\ldots] die von
allen existierenden Open-Source-Lizenzen am weitesten verbreitete (sei)
und [\ldots] vielen Lizenzen als Vorbild (diene)}, und wenn sie dann
ergänzt, dass die GPL \enquote{[\ldots] daher mit Recht als Grundtypus
zahlreicher Open Source Lizenzen bezeichnet werden
(könne)}\footcite[vgl.][33]{Oberhem2008a}. So ist denn Gegenstand der
rechtlichen Untersuchung nur die GPL in der Version 2 und 3, wobei letztere
explizit in die Fußnoten gelegt wird\footcite[vgl.][34]{Oberhem2008a}

\paragraph{Abgrenzung von anderen Begriffen}

Oberhem setzt die umliegenden Begriffe in folgenden Zusammenhang:

\begin{description}
  \item[``Freie Software"] :- \enquote{[\ldots] fungiert als
  Sammlebegriff}\footcite[vgl.][8]{Oberhem2008a}
  \begin{description}
    \item[``Free Software"] :- sei der ursprünglichere Begriff, der
    aber ob seiner Assoziationen mit \enquote{Kostenfreiheit} zumindest
    \enquote{[\ldots] auf den ersten Blick irreführend (sei)}. Denn in
    Wirklichkeit solle das Wörtchen [Seitenwechsel] 'free' \enquote{[\ldots] die
    vom Nutzer hinzugewonnenen Freiheiten im Umgang mit ihre verdeutlichen
    (sollte)}, will sagen das Recht für jedermann, sie \enquote{[\ldots]
    benutzen, kopieren oder verteilen (zu dürfen)} und zwar
    \enquote{unverändert oder mit Modifikationen, kostenlos oder gegen 
    Bezahlung, stets aber mit
    Source-Code}\footcite[vgl.][8f]{Oberhem2008a}.
    \item[``Open-Source-Software"] :- ist der \enquote{aus
    Marketinggründen} erfolgte Ersatz für 'Free Software', sodass man
    \enquote{unter 'Open-Source-Software' [\ldots] im Wesentlichen das
    Gleiche (verstehe) wie unter 'Free
    Software'}\footcite[vgl.][9]{Oberhem2008a}.
    \item[``Public Domain Software"] :- wird pikanterweise hier nicht
    näher erläutert. (sie düfr andere Passagen)
     
  \end{description}
\end{description}

\subsubsection{Open Source Geschichte}

Hier bietet Oberhem das Muster an Geschichtsmrkmalen, das zu Recht von einer
kurzen Historie zu erwarten ist:

Die Geschichte der Open Source beginnt einerseits mit dem impliziten Leben
dieser Praxis. Insofern Software anfangs noch an Hardware gebunden war, wurde
sie mitgeliefert und als Feature der Hardware betrachtet, das von den Nutzer der
Hardware gerne verbessert werden durfte. Und eben dies geschah in freiem
Austauch - ein Tun, aus dem heraus auch die ursprüngliche
\enquote{Hacker-Ethik} entstanden
sei.\footcite[vgl.][17f]{Oberhem2008a}. Die Gefährung dieser freien
Kooperation entsteht dann mit dem ``Unbundling'' von Hardware und Software,
angestossen durch ein \enquote{Kartellverfahren gegen IBM} ab
1969\footcite[vgl.][19]{Oberhem2008a}. Damit erhält Software einen
eigenständigen Wert, der auch ein Geschäftsmodell begründen kann.

Die zweite Säule des Beginns der Open Source Geschichte betrifft das
Unix-Betriebssystem. Dieses - 1969 bei AT\&T entwickelt - durfte zu dieser Zeit
nur als Public Domain innerhlab der Universitäten vertrieben werden. AT\&T war
schon früher durch ein anderes Kartellverfahren dazu verpflichtet worden, sich
nur im Telefonsektor zu betätigen. Deshalb gab diese Firma die Software Unix
\enquote{[\ldots] zum Selbstkostenpreis von ca. 50 Dollar an interessierte
Universitäten und Institute ab. Damit stand einer universitäten
Weiterentwicklung in freier Organisation nichts mehr
entgegen}\footcite[vgl.][20]{Oberhem2008a}.

Die Situation änderte sich mit der kartellrechtlich bedingten Aufspliitung von
AT\&T im Jahren 1984, sodass auch die AT\&T-Ableger \enquote{[\ldots] von nun an
als Wettbewerber auf dem Softwaremarkt (auftreten konnten)}. Damit wurde
der Vertieb von Unix-Lizenzen zum Geschäftsmodell des neue gegründeten
'Tochterunternehmens' \enquote{Unix System Labaratories}. Die
Proprietarisierung von Unix hatte sich
etabliert.\footcite[vgl.][21]{Oberhem2008a}

Damit ist das Umfeld für den nächsten Schritt gezeichnet: es gab einmal einen
freien Umgang und später gab es ihn nicht mehr. Diese Beschränkung stiess bei R.
Stallman auf radikale Gegenmassnahmen: Er schuf das GNU Projekt mit dem großen
Ziel eines freien Unix, das sich wie ein Unix verhalten sollte, ohne bei seiner
Erstellung dafür \enquote{[\ldots] Teile des von AT\&T geschützten
Quellcodes zu verwenden}; er gründete 1983 die FSF als \enquote{gemeinnütze
Stiftung}; und er erfand 1989 die GNU General Public License, und zwar
\enquote{in Zusammenarbeit mit Eben Moglen}\footcite[vgl.][22]{Oberhem2008a}

Als \enquote{Herzstück} des neuen Betriebssystem kam dann 1991 der
Linux-Kernel von Linux Thorvalds dazu\footcite[vgl.][23]{Oberhem2008a}, wobei
auch Oberhem betont, dass \enquote{[\ldots] Thorvalds auf die bereits
entwickelten Anwendungen des GNU-Projektes zurückgreifen (konnte), deren
Verbindung mit dem von ihm entwickelten Kernel erst ein vollständiges
Betriebssystem entstehen ließen}\footcite[vgl.][24]{Oberhem2008a}

Und um die Jahre um 1998 herum drang GNU/Linux dann schließlich auch in die
Geschäftswelt ein\footcite[vgl.][25]{Oberhem2008a}

\subsubsection{Open Source Definition}

Oberhem geht die \enquote{Open-Source-Software
Definition}\footcite[vgl.][10. Die Abwandlung des
originären Namens dürfte ihrer Einteilungvon Open Source als
Oberbegriff geschuldet sein]{Oberhem2008a} durch listet dazu Merkmale /
Besonderheit auf:

\begin{description}
  \item[``Offener Quellcode''] :- \enquote{Hauptmerkmal für die
  Einordnung eines Programms als Open-Source-Software (sei) [\ldots] die
  Verpflichtung zur Offenlegen des sog.
  Quellcodes}\footcite[vgl.][11]{Oberhem2008a}. Dazu müsse zwischen
  \enquote{maschinenorientierten Programmiersprache} und
  \enquote{problemorientierten Programmiersprachen} unterschieden werden:
  erste richten sich als Ausführungsbefehle an Computer, letztere an Menschen,
  denn [seitenwechsel] \enquote{Adressat dieser Programme (sei) nicht der Computer
  selbst, sondern der Programmierer}\footcite[vgl.][12f]{Oberhem2008a}.
  Und eben ein solche in 'problemorientierten Programmiersprachen' verfasstes
  Programm werde als \enquote{[\ldots] als Quellcode [\ldots] oder auch
  Quellprogramm bezeichnet}\footcite[vgl.][13]{Oberhem2008a}. Die
  \enquote{maschinenorientierten Programmiersprache} zeichne sich durch so
  etwas wie 'Information-Hiding' aus, denn \enquote{der für den Hersteller
  erfreuliche Nebeneffekt (sei ja), dass die in seinem Programm
  gefundenen Lösungswege ohne diese Information von Dritten nicht
  nachvollzögen werden könne [\ldots]}\footcite[vgl.][12]{Oberhem2008a}.
  Und diesem Information-Hiding wolle sich die Open-Source-Software
  systematischen entgegenstellen\footcite[vgl.][13]{Oberhem2008a}
  \item[``Freie Weiterverbereitung''] :- meine, dass \enquote{[\ldots]
  eine Open-Source-Software-Lizenz keinerlei Beschärnkung oder Verbot der
  Weiterverbreitung enthalten (dürfe)}, und zwar weder
  \enquote{unmittelbare Beschränkungen} noch \enquote{mittelbare
  Beschränkungen}. So entstehe auch das \enquote{Verbot der Erhebung
  von Lizenzgebühren} und zugleich leite sich daraus ab, dass erlaubt sei,
  Open-Source-Software auch gemeinsam mit  'proprietärer Software' zu
  verbreiten\footcite[vgl.][13]{Oberhem2008a}. Unbeschadet dieser Vorgaben,
  dürfen bei gemeinsamen Vertrieb für die proprietäre Software sehr wohl
  Lizenzgebühren erheben werden und für \enquote{die Überlassung der
  Programmkpie} dürfe ebenfalls ein Entgelt verlangt werden. Nur das
  \enquote{Einräumen der Nutzungsrechte} habe unentgeltlich zu
  erfolgen\footcite[vgl.][14]{Oberhem2008a}
  \item[``Abgeleitete Programme''] :- Eine Open Source Lizenz müsse
  \enquote{[\ldots] jedem Nutzer gestatten, Veränderungen am Programm
  vorzunehmen und auf deren Grundlage auch neue Programme, sog.
  abgeleitete Programme [\ldots], zu
  erstellen}\footcite[vgl.][14]{Oberhem2008a}.
  \item[``Integritätsschutz des Originals''] :- zwecks Schutz des
  Autors vor \enquote{Entstellung seines Programms durch Bearbeitung
  Dritter} erlaubt die OSD es, \enquote{[\ldots] dass die Lizenz
  jeden Bearbeiter zu einem Hinweis auf die Änderungen verpflichten könne
  }\footcite[vgl.][14]{Oberhem2008a} - wobei \enquote{ eine Preisgabe
  der wahren Identität des jeweiligen Autors [Bearbeiters KR.] [\ldots]
  jedoch in der Regel nicht verlangt
  (werde)}\footcite[vgl.][15]{Oberhem2008a}. Wichtig hier: Der Wortlaut
  der OSD gibt in der Tat ein Kann-Bestimmung vor.
  \item[``Integritätsschutz des Originals''] :- Oberhem konstatiert,
  dass eine Open Source Lizenz \enquote{[\ldots] auf keine Diskriminierung
  gegenüber bestimmten Personen oder Gruppen enthalten
  (solle)}\footcite[vgl.][15]{Oberhem2008a}. Die Wortwahl ist ungünstig.
  Denn in der Tat legt die OSD fest, dass eine OS-Lizenz keine solche
  Ausgrenzung enthalten darf.
  \item[``Verbot der Beschränkung auf bestimmte Anwendungsarten"] :-
  Im gleichen Sinne lege die OSD fest, dass eine OS-Lizenz \enquote{[\ldots]
  die Anwendbarkeit der Software nicht von vorneherein auf bestimmte
  Gebiete beschränken (dürfe)} - wobei 'Gebiete' hier nicht geographisch
  zu verstehen ist, sondern im Sinne eines Verwendungszeckes zu verstehen ist.
  Jedenfalls dürfe als Zweck insbesondere der \enquote{[\ldots]
  gewerbliche Einsatz nicht grundsätzlich ausgeschlossen
  werden}\footcite[vgl.][15]{Oberhem2008a} - Zu ergänzen ist, dass er
  nicht nur grundsätzlich nicht ausgeschlossen werden dürfe, sondern auch nicht
  speziell, aben überhaupt nicht.
  \item[``Verbreitung der Lizenzvorschriften"] :- Zur
  Aufrechterhaltung des Systems erfordere die OSD, dass \enquote{[\ldots]
  die jeweilige Lizenz zusammen mit dem Programm vertireben
  (werde)}\footcite[vgl.][16]{Oberhem2008a}
  \item[``Verbot der Beschränkung auf ein bestimmtes Produktqrqq{}] :-
  Wichtig zu verstehen ist dieses Verbot: es besagt, dass \enquote{[\ldots]
  sich die Lizenz nicht nur auf das Programm in seiner Gesamtheit,
  sondern auf alle seine Bestandteile beziehen (müsse)}. Damit soll
  unterbunden werden, \enquote{[\ldots] dass einzelne Programmteile
  isoliert weiterverarbeitet und unter eine proprietäre Lizenz gestellt
  werden}\footcite[vgl.][16]{Oberhem2008a}. Produkt bezieht sich also
  nicht nach außen - [außen: 'darf dies Programm für Solarzellen verwenden, aber
  nicht für Atomkraftwerke'] - sondern nach innen.
  \item[``Verbot der Austrahlung auf andere Software"] :- Diese
  Regelung besage, dass die vorgenannten Bestimmungen nur für die OS-Lizenziert
  Software selbst gelten sollen und dass bei einer gemeinsamen Distribution
  durch den Akt der Distribution nicht selbst schon die mitgepackte proprietäre
  Software zu Open Source Software werden
  solle.\footcite[vgl.][16]{Oberhem2008a}.
  \item[``Neutralitätsgebot gegenüber anderen Technologien"] :-
  \enquote{Schließlich (dürfe) eine Open-Source-Software-Lizenz keine
  Bestimmung enthalten, die wertende Aussagen über andere Technologien
  (beinhalte)}\footcite[vgl.][17]{Oberhem2008a}
\end{description}

Insgesamt unterstreicht Oberhem, dass diese \enquote{[\ldots] Anforderungen
[\ldots] gleichermaßen an alle Open-Source-Software-Lizenzen gestellt
(werden)}. Einzelne Lizenzen dürfen \enquote{[\ldots] über die
geannten Regelungen (hinausgehen)}, diese definierenden Merkmal bilden nur
\enquote{Kernaussagen} ab: \enquote{Nur wenn sich eine Lizenz
inhaltlich mit dem durch die 'Open-Source-Definition' gezeichneten
Leitbild deckt, kann sie als Open-Source-Software-Lizenz qualifiziert
werden}\footcite[vgl.][17]{Oberhem2008a}

\subsubsection{Abgrenzung von anderen Softwarearten}

Oberhem nähert sich auf eine spezielle Weise dem 'Feld der freien Software',
wobei es eben drarum geht, was denn genau dieses Feld der freien Software ist
und wie es in sich zu strukturieren wäre:

Zunächst fixiert Oberhem, dass sich Open Source Software \enquote{[\ldots]
(inhaltlich) nicht [\ldots] von kommerzieller Software (unterscheide)},
sondern nur \enquote{[\ldots] durch seine besondere Herstllung und
Vertriebsart}: \enquote{wesentlichen Merkmale der Open Source Software
(seien) die Bearbeitungsmöglichkeit durch jeden User und ihre
\textit{autodistributive} Verbreitungsform}\footcite[vgl.][8 herv.
KR.]{Oberhem2008a}. Die \enquote{Kostengünstigkeit} von Open-Source-Software
sei dabei ein \enquote{Nebeneffekt}, aber eben \enquote{kein primäres
Ziel von Open-Source-Software}\footcite[vgl.][8]{Oberhem2008a}

Sodann rekapituliert Oberhem unter der Überschrift \enquote{Abgrenzung von
Open-Source-Software zu proprietärer Software}, \enquote{die
Entstehungsgeschichte von Open-Source-Software (zeige) deutliche, dass für ihre
Entwicklung vor allem ihr \textit{autodistributiver Vertrieb} von Bedeutung war
und ist}\footcite[vgl.][27]{Oberhem2008a}. Damit wird das Feld der
Software im Allgemeinen in zwei Bereiche geteilt, in den Bereich der
\enquote{proprietären Software} und in den Bereich der
\enquote{Open-Source-Software}\footcite[vgl.][27]{Oberhem2008a}, womit
Oberhem - zumindest unter der Hand - auch einer ex-Negativo Definition von
proprietärer Software das Wort redet. Leider ist es nun nicht so, dass das
Merkmal des \enquote{autodistributiven Vertriebs} auf den Bereich der Open
Source Software beschränkt, weshalb trotzdem \enquote{[\ldots] die
Unterschiede zu Kaufsoftware und vor allem zu ebenfalls autodistributiv
verbreiteter proprotärer Software herausgestellt werden
(müsse}\footcite[vgl.][27. BTW: Dieser 'Begriff' des 'autodistributiven
Vertriebs' klingt sicher gut, erhellt aber wenig. Zunächst wird er schwer den
Verdacht des Oxymorons loswerden. Distribution von Software ist irgendwie ja
auch Vertrieb - und umgekehrt. Und dann verteilt sich Software nicht selbst. Es
sind immer Menschen, die sich Software herunterladen oder gar anderen zusenden.
Es gibt schwerlich ein 'auto' in dieser Art der Distribution. Wir werden den
Begriff des 'autodistributiven Vertriebs' deshlab auch nicht weiter verfolgen,
so verführerisch er sich zunächst auch anbietet.]{Oberhem2008a}

Insgesamt zeichnet Oberhem - wie mittelbar auch an ihrer Kapitelstruktur
abzulesen ist - folgende Klassifikation

\begin{itemize}
  \item \enquote{Open-Source-Software} versus
  \enquote{Kaufsoftware}\footcite[vgl.][27f]{Oberhem2008a} :- Hier listet
  Oberhem als Unterscheidungsmerkmal den Preis, die \enquote{Möglichkeit
  der freien Bearbeitung} und die Nutzung der Community als Quelle einer
  \enquote{ständigen Weiterentwicklung und
  Verbesserung}\footcite[vgl.][28. So intutitiv passend die
  Unterschiedung in kommerzielle Software und nicht kommerzielle
  Software auch sein mag, so sehr wird noch zu überlegen sein, ob die hier
  genannten Kriterien eine Systematik wirklich tragen]{Oberhem2008a}. Eine
  andere Spezifikation von \enquote{kommerzieller Software} hat Oberhem
  bereits vorher geliefert, in dem sie sagte, dass diese \enquote{[\ldots]
  ausschließlich durch autorisierte Händler gegen Entgeld
  (Lizenzgebühren) vertrieben (werde)
  [\ldots]}\footcite[vgl.][6]{Oberhem2008a}
  \item \enquote{Open-Source-Software} versus \enquote{sonstige
  autodistributive Software}\footcite[vgl.][28ff]{Oberhem2008a}
  \begin{itemize}
    \item 'Open-Source-Software' versus ebenfalls 'autodistributiver'
    \enquote{Public-Domain-Software}\footcite[vgl.][29]{Oberhem2008a} :-
    wird selbst als ein aus der Verwendung heraus \enquote{unklarer Bergiff}
    bezeichnet. Am deutlichsten sei die aus der USA stammenden Interpretation
    als 'öffentliches Eigentum', bei der der Urheber \enquote{[\ldots]
    gänzlich auf (sein) Urheberrecht
    (verzichtet)}\footcite[vgl.][30]{Oberhem2008a}. Allerdings: \enquote{Ein
    derart umfassender Verzicht auf sämtliche Urheberrechte ist nach deutschem
    [\ldots] nicht möglich, da es sich um ein Persönlichkeitsrecht handelt, das
    der Urheber nicht übertragen kann}\footcite[][30]{Oberhem2008a}. Und
    des weiteren weist Oberhem daraufhin, dass der uns heute geläufige
    \enquote{Copyright Vermerk} sozusagen ein Relikt sei, dessen Fehlen die
    Existenz von Public Domain Software begünstigt habe: \enquote{Da nach
    früherem US-amerikanischen Recht jeder Urheber dazu verpflichtet war,
    sein Werk mit einem Copyright Vermerk zu versehen, wurde jede
    Software, bei der dieser Vermerk fehlt zu öffentlichem
    Eigentum.}\footcite[][30]{Oberhem2008a}. Allerdings habe sich diese
    Situation seit 1988 mit dem \enquote{Beitritt der USA zur revidierten
    Berner Übereinkunft (RBÜ)} geändert: nur bis dahin habe
    \enquote{[\ldots] des Fehlen des Coypright Vermerks dazu (geführt),
    dass das Werk nicht den vollen urheberrechtlichen Schutz
    genoss}\footcite[][30 Anm.152]{Oberhem2008a}
    \item 'Open-Source-Software' versus ebenfalls 'autodistributiver'
    \enquote{Freeware}\footcite[vgl.][31]{Oberhem2008a} :- Hier meint
    Oberhem, dass dieser Begriff - trotz aller seiner Unschärfe - ebenfalls
    nicht mit dem der Open Source Software gleichgetzt werden könne: zwar
    \enquote{[\ldots] verzichte (der Autor von Freeware) auf die
    wirtschaftliche Vermarktung seines Programms}, allerdings erlaube er
    zumeist nicht seine Veränderung, ja mehr noch: \enquote{[\ldots] teilweise
    (werde) die Veränderung sogar ausdrücklich
    verboten}\footcite[vgl.][31]{Oberhem2008a}
    \item 'Open-Source-Software' versus ebenfalls 'autodistributiver
    \enquote{Shareware}\footcite[vgl.][31f]{Oberhem2008a} :- hier werde
    die Software - nach Oberhem - nur \enquote{[\ldots] zunächst kostenlos
    vertrieben und weiterverbreitet}, also gewissermaßen
    vorläufig\footcite[vgl.][31]{Oberhem2008a}. Denn nach \enquote{Ablauf
    der Probezeit} werde - wie bei Kaufsoftware auch - über die
    \enquote{Registrierungsgebühr} eben doch eine Lizenzgebühr
    fälligfootcite[vgl.][32]{Oberhem2008a}.
    \item 'Open-Source-Software' versus ebenfalls 'autodistributiver'
    \enquote{Shared Source Software}\footcite[vgl.][32f]{Oberhem2008a} :-
    sei ein von der Firma Microsoft \enquote{im Jahre 2001 ins
    Leben gerufenes Lizenz-Modell}, wobei der \enquote{Lizenznehmer}
    formal \enquote{[\ldots] das Recht (habe), zu nicht kommerziellen
    Zwecken die Software zu nutzen, zu verändern und zu
    verbreiten}footcite[vgl.][32]{Oberhem2008a}. Das hört sich an wie eine
    Open Source Definition. Gleichwohl liege da sozusagen ein Hase im Pfeffer:
    Denn \enquote{bereits die Nutzung im geschäftlichen Bereich (stelle)
    dabei eine kommerzielle Zwecksetzung dar
    [\ldots]}footcite[vgl.][33]{Oberhem2008a}. Konsequenterweise kann und
    wird diese Modell nicht zu einer sich selbst tragenden Fortentwicklung durch
    eine Community führen. Konsequenterweise sei es denn auch 'nur' \enquote{Ziel
    dieser Shared Source Lizenz [\ldots], den Geschäftskunden die
    Mitarbeit zu ermöglichen, ohne jedoch das herkömmliche
    Geschäftsmodell [des Umsatz/Gewinns durch Lizenzgebühren; KR]
    aufzugegeben}\footcite[vgl.][33]{Oberhem2008a}
  \end{itemize}
\end{itemize}

Leider liefert Oberhem eine \enquote{schematisch Darstellung}, die der
skizzierten Struktur nicht entspricht. Darin wird der Briff der 'Kaufsoftware'
doch wieder durch den 'Proprietären Software' ersetzt und die Begriffe 'Public
Domain Software' und 'Open Source Software' haben die gleichen
Merkmale\footcite[vgl.][33]{Oberhem2008a}, sodass dieses Schema eher hinter den
Erkenntnisgewinn der vorheregehnden Darstellung zurückfällt.


\subsection{Geldkosten / Free Beer Software}

Dass Open Source Software kostengünstig ist und dass \enquote{[\ldots]
ohne den (Effekt der Kostengünstigkeit) eine Konkurrenzfähigkeit von
Open-Source-Software zu herkömmlicher kommerzieller Software wohl kaum
hätte erreicht werden können}, bezeichnet Oberhem allenfalls als
\enquote{Nebeneffekt} und unterstreicht, dass diese Kostengünstigkeit
\enquote{[\ldots] kein primäres Ziel von Open-Source-Software
(sei)}\footcite[vgl.][8]{Oberhem2008a}

\subsection{Wesentliche Bestimmung der GPL}

Dann folgt Oberhem dem Gedankengang der GPL und erläutert ihre Struktur und dei
Beduetung einzelner Passagen.

\subsubsection{Anwendungsbereich der GPL}

So wird unter dem Stickwort \enquote{Anwendungsbereich der
Lizenz}\footcite[vgl.][34]{Oberhem2008a} festgehalten, dass der
\enquote{Gegenstand der Lizenz [\ldots] jedes Programm oder sonstiges Werk
(sei), das einen entsprechenden Copyright-Vermerk (enthalte), der die
Geltung der GPL vorschreibt}, und dass - sozusagen 'darüberhinaus' -
\enquote{[\ldots] alle 'auf dem Programm basierenden Werk' (erfasst
werden)}. Witz dieser Formulierung ist, dass sie über das deutsche
Urheberrecht hinausgehe: werde dort \enquote{[\ldots] die Definition der
'Bearbeitung' durch die Merkmale der Wesentlichkeit und der dienenden
Funktion eingegrenzt}, so das bei der GPL nicht der Fall: sie stelle
\enquote{keine qualitativen oder quantitativen Anforderungen}, damit ein
'Abgeleitetsein' eintrete: Und so könne \enquote{[\ldots] könne bereits die
Integration unbedeutender Teile eines Open-Source-Programms die Geltung
der GPL für das Arbeitsergbenis
auslösen}\footcite[vgl.][35]{Oberhem2008a}.

Nach Oberhem stelle der vom Compiler erzeugte maschinen Code allerdings keine
Bearbeitung dar, solange er keine \enquote{[\ldots] eigenen Routinen in das
Programm (kopiere) [\ldots]}. Diese Deutung ist wichtig, weil andernfalls
auch alle mit einem unter GPL stehenden Compiler erzeugten Maschineprogramme zu
GPL-Code würden\footcite[vgl.][36]{Oberhem2008a}. [Comment: Allerdings ist die
Abgrenzung mit der Routinencopy grenzwertig. Optimierende Compiler tuen
sinnvollerweise genau das. Wenn die Kopf/End/basierte Rekursion implizit durch
eine Schleife ersetzt wird, dann steht im maschinen code etwas anderes.Gemeint
ist hier eher, dass ein Compiler nicht zusätzliche Funktionalität hinzufügen
darf.]

Der Geltungsbereich wird auch durch diemZusatzregel in §2.2 bestimmt, derzufolge
das einheitlich vertriebene Ganze dann ebenfalls unter der GPL vertrieben werden
muss\footcite[vgl.][37]{Oberhem2008a}

\subsubsection{Rechte der Nutzer}

Zunächst der in §1.1 der GPL dem Nutzer das Recht der
\enquote{Vervielfältigung und Verbreitung des ursprünglichen Quellcodes}
zugesprochen\footcite[vgl.][38]{Oberhem2008a}. Sodann erhalte er §2.1 auch das
Recht der \enquote{Veränderung, Vervielfältigung und Verbreitung des veränderten
Quellcodes}\footcite[vgl.][39]{Oberhem2008a}. Des weiteren werde
festgelegt, \enquote{[\ldots] dass die Einräumung der Nutzungsrechte an der
Software [\ldots] konstenfrei zu erfolgen
hat}\footcite[vgl.][41]{Oberhem2008a}.

Grundsätzlich bleibt zu konstatieren, dass nach deutschem Urheberrecht die
eingeräumten Nutzungsrechte abschließend sind. Das heißt, \enquote{[\ldots]
dass dem Nutzer nur solche Nutzungsrechte an der Software durch die GPL
eingeräumt werden, die zum Zeitpunkt ihrer ersten [SW] Verbreitung bekannt
sind}.\footcite[vgl.][41]{Oberhem2008a}. [Was bedeuetet das? Nun, man
könnte sich ja vorstellen, dass GNU lizenziert Computerprogramm durch ein
Wunderhilfsmittel in Zukunft hervorragend dazu sind, Despoten zu töten: das
Wunderhilfsmittel berechnet die Gödelzahl in Gold und eben diese
goldene Gödelzahl ist bei GNU Programmen immer so, dass Despoten tot vom Tisch
fallen. Das dürfte ein Verwendungszweck sein, den keiner der bisherigen GNU
Autoren bei der Veröffentlichung gekannt hat.]  Und unbekannte Verwendungszwecke
können in Deutschland - etwa durch Änderung der GPL und Auswahl der neuen
Version durch einen Nutzer - nicht nachträglich mitlizenziert sein:
\enquote{Der Urheber kannn [\ldots] nicht dazu gezwungen werden, seine
Software auch für Nutzungsarten, die er zum Zeitpunkt ihrer Lizenzierung
nicht kannte, unter der GPL freizugeben}\footcite[vgl.][43]{Oberhem2008a}.

\subsubsection{Pflichten der Nutzer}

\paragraph{Pflichten bei Verbreitung der unveränderten Software}

Bei der \enquote{Weitergabe in unverändertem Zustand} entstünden nach
Oberhem für den Weitergebenden drei Pflichten\footcite[vgl.][43]{Oberhem2008a}
und ein Verbot\footcite[vgl.][44]{Oberhem2008a}:

\begin{itemize}
  \item Jede Kopie muß\enquote{ [\ldots] mit einem Copyright-Vermerk und einem
  Haftungsausschluss veröffentlicht werden}
  \item \enquote{Zudem müssen alle Vermerke, dich sich auf die Lizenz und
  das Fehlen einer Garantie beziehen, unverändert erhalten bleiben}
  \item Jeder Empfänger einer \enquote{Programmkopie} muss auch eine
  \enquote{Kopie der Lizenz} erhalten
  \item Man darf keine Lizenzgebühren verlangen, wohl aber 
  \enquote{[\ldots] für den reinen Kopiervorgang bei der Vervielfältigung
  der Software eine Gebühr verlangen}\footcite[vgl.][44]{Oberhem2008a}
\end{itemize}

\paragraph{Pflichten bei Verbreitung der veränderter Software}

Zunächst bleiben die o.a. Pflichten für die Weitergabe unveränderter Software
auch bei der Weitergabe veränderter Software
bestehen\footcite[vgl.][44]{Oberhem2008a}

Desweiteren müssen Änderungen markiert werden, gleichwohl dürfe das auch anonym
geschehen.\footcite[vgl.][44]{Oberhem2008a}

Sodann gibt es einer Pflicht zu \enquote{Freigabe der Veränderungen}.
Allerdings zwinge die GPL den Nutzer gerade \enquote{[\ldots] nicht in
jedem Fall, die von ihm vorgemommen Änderungen wiederum unter der GPL
freizugeben}: \enquote{Nur wenn der Nutzer seine Bearbeitung
veröffentlicht und damit verbreitet, ist er dazu verpflichtet, diese
wiederum der GPL zu unterstellen}\footcite[vgl.][45]{Oberhem2008a}.

\begin{quote} Zur Weiterverbreitung an sich wird der Lizenznehmer [\ldots] an
keiner Stelle der GPL verpflichtet. Vielmehr obliegt es seiner freien
Entschiedung, ob er das Programm weiterverbreiten will oder nicht.
\end{quote}



Rechtlicht brisant ist die Frage, wann denn nun eine Veröffentlichung
anzusetzen ist:

\begin{quote} \enquote{Die GPL selbst enthält keine Anhaltspunkte zur
Beantwortung der Frage, wann eine 'Veröffentlichun' vorliegt. Zu Ihrer
Beantwortung ist auf den Öffentlichkeitsbegriff des [\ldots]
(Urheberrechtsgesetzes) abzustellen. Nach diesem ist die Wiedergabe eines
Werkes öffentlich, 'wenn sie für eine Mehrzahl von Personen bestimmt ist,
es sei denn, dass der Kreis dieser Personen bestimmt abgegrenzt ist und
sie durch gegenseitige Beziehungen oder zum Veranstalter persönlich
untereinander verbunden sind.}\footcite[vgl.][45]{Oberhem2008a}.
\end{quote}

Aus dieser Formuleriung zieht Oberhem nun radikalen Schlüsse [die
wahrschienlich nicht von allen geteilt werden]. Sie sagt, dass die
\enquote{persönliche Verbundenheit} in der Regel nicht vorliege,
\enquote{[\ldots] wenn die Mitarbeiter eines Unternehmens oder einer
Behörde lediglich auf beruflicher Ebene miteinander
verkehren}\footcite[vgl.][45]{Oberhem2008a}. Und schärfer noch: Mit
wenigen Ausnahmen - so Oberhem - \enquote{[\ldots] (sei) die Weitergabe von
Software innerhalb eines Unternehmens oder einer Behörder jedoch als
'öffentlich' i.S.d. §15 Abs. 3 UrhG anzusehen [\ldots] }, was unmittelbar
auf eine Veröffentlichung im Sinne der GPL
durchschlage\footcite[vgl.][46]{Oberhem2008a}:

\begin{quote} \enquote{Somit muss jede Änderung, auch wenn sie nur der
behörden- bzw. firmeninternen Nutzung dienen soll, bei ihrer Weitergabe
innerhalb des Unternehmens wiederum der GPL unterstellt
werden}\footcite[vgl.][46]{Oberhem2008a}
\end{quote}

[Comment: dies hört sich bedrohlich an, ist aber gar nicht. Denn es besteht
keine Plicht, seine Veränderungen weltweit freizugeben, es besteht nur die
Pflicht, seinem Empfänger Code und lizenzierte Software weiterzureichen. Wenn
also zufälligerweise die Modifkationen gar nicht aus der Firma oder der Behörde
hinauskommen, so wäre das immer noch ein Ergebnis im Sinne der GPL. Was geiß
aber nicht mit der GPL übereinstimmt, wenn man modifizierte Software innerhlab
einer Software GPL lizenziert weitergibt, verbunden mit der mündlichen
Chef-Anordnung, sie aber ja nicht and dritte weiterzugeben. Damit wäre
unstritiig das Recht der Weitergabe doch eingeschränkt, Und das gewährt ja die
GPL. Dies muss man sich als Firma auch klar machen, wenn man GPL lizenzierten
Code an 'Freelancer' weitergbt oder sie geradezu diese erarbeiten läßt: sie
haben das Recht die Verbesserung ihrerseits an dritte weiterzugeben.
Allerdings: Warum sollte eine Firma das tun? GPL-Software geheim nur für sich
so zu verbessern und das Ergebnis verschleiern?]

\paragraph{Pflichten bei Verbreitung im Objectcode}

Bei der Weitergabe als Binary entsteht das Problem, wie mit der Weitergabe des
Quelcodes zu verfahren ist. Hier bietet die GPL klare Möglichkeiten:

\begin{itemize}
  \item Zum ersten könne der Quellcode schlicht auf CD etc. mitgeliefert
  werden.\footcite[vgl.][46]{Oberhem2008a}
  \item Oder es kann die Verpflichtung der Codeweitergabe \enquote{[\ldots]
  durch ein schriftliches, mindestens drei jahre gültiges Angebot
  (nachgekommen werden), den Quelltext jedem Dritten auf einem für den
  Datenaustausch üblichen Medium zur Verfügung zu stellen
  [\ldots]}\footcite[vgl.][46]{Oberhem2008a}
  \item Und es dürfen diejenige, die die Software mit der Versicherung des 3
  Jahre gültigen Lieferzusage erhalten haben, diese schriftliche Zusage
  ebenfalls weiterreichen, sofern sie die Software \enquote{[\ldots] zu nicht
  kommerziellen Zwecken weiterverbreiten
  wollen}\footcite[vgl.][47]{Oberhem2008a}.
\end{itemize}

Ein Spezialfall ist der Download: Hier reicht es aus, an der Stelle des
Downloads der Binaries, auch den Sourcecoe zum Doqnload anzubieten,
\enquote{[\ldots] auch wenn der Nutzer nicht dazu gezwungen ist, diesen
zugleich mit dem Objectcode zu kopieren}\footcite[vgl.][47]{Oberhem2008a}

[Hint: Der Witz ist jetzt, dass es aus GPL Sicht gerade nicht hinreichend ist,
einfach nur den Link auf die Download-Seite eines anderen
Distributors weiterzugeben. Wenn man schon den Binärcode in eines seiner
Produkte verbundelt, sei unmodifiziert oder modifiziert, dann muss man den Code
auch bei sich und durch sich zum Download anbieten. Und das ist keine Schikane,
sondern ganz im Sinne der GPL: es könnte ja sein, dass dieser Fremddistributor
morgen seine Quellen verschliesst. Dann wären aber meine Kunden nicht mehr GPL
mäßig bedient. Sicher wäre es eine gute Sache, wenn man einen solche zentralen
Downloadserver als Service bestellen könnte. Das wäre sicher im Sinne der GPL.]

\small
\bibliography{../bibfiles/oscResourcesDe}

\end{document}
