% Telekom osCompendium extract template
%
% (c) Karsten Reincke, Deutsche Telekom AG, Darmstadt 2011
%
% This LaTeX-File is licensed under the Creative Commons Attribution-ShareAlike
% 3.0 Germany License (http://creativecommons.org/licenses/by-sa/3.0/de/): Feel
% free 'to share (to copy, distribute and transmit)' or 'to remix (to adapt)'
% it, if you '... distribute the resulting work under the same or similar
% license to this one' and if you respect how 'you must attribute the work in
% the manner specified by the author ...':
%
% In an internet based reuse please link the reused parts to www.telekom.com and
% mention the original authors and Deutsche Telekom AG in a suitable manner. In
% a paper-like reuse please insert a short hint to www.telekom.com and to the
% original authors and Deutsche Telekom AG into your preface. For normal
% quotations please use the scientific standard to cite.
%
% [ File structure derived from 'mind your Scholar Research Framework' 
%   mycsrf (c) K. Reincke CC BY 3.0  http://mycsrf.fodina.de/ ]

%
% select the document class
% S.26: [ 10pt|11pt|12pt onecolumn|twocolumn oneside|twoside notitlepage|titlepage final|draft
%         leqno fleqn openbib a4paper|a5paper|b5paper|letterpaper|legalpaper|executivepaper openrigth ]
% S.25: { article|report|book|letter ... }
%
% oder koma-skript S.10 + 16
\documentclass[DIV=calc,BCOR=5mm,11pt,headings=small,oneside,abstract=true, toc=bib]{scrartcl}

%%% (1) general configurations %%%
\usepackage[utf8]{inputenc}

%%% (2) language specific configurations %%%
\usepackage[]{a4,ngerman}
\usepackage[ngerman, german, english]{babel}
\selectlanguage{english}

%language specific quoting signs
%default for language emglish is american style of quotes
\usepackage{csquotes}

% jurabib configuration
\usepackage[see]{jurabib}
\bibliographystyle{jurabib}
% Telekom osCompendium English Jurabib Configuration Include Module 
%
% (c) Karsten Reincke, Deutsche Telekom AG, Darmstadt 2011
%
% This LaTeX-File is licensed under the Creative Commons Attribution-ShareAlike
% 3.0 Germany License (http://creativecommons.org/licenses/by-sa/3.0/de/): Feel
% free 'to share (to copy, distribute and transmit)' or 'to remix (to adapt)'
% it, if you '... distribute the resulting work under the same or similar
% license to this one' and if you respect how 'you must attribute the work in
% the manner specified by the author ...':
%
% In an internet based reuse please link the reused parts to www.telekom.com and
% mention the original authors and Deutsche Telekom AG in a suitable manner. In
% a paper-like reuse please insert a short hint to www.telekom.com and to the
% original authors and Deutsche Telekom AG into your preface. For normal
% quotations please use the scientific standard to cite.
%
% [ File structure derived from 'mind your Scholar Research Framework' 
%   mycsrf (c) K. Reincke CC BY 3.0  http://mycsrf.fodina.de/ ]

% the first time cite with all data, later with shorttitle
\jurabibsetup{citefull=first}

%%% (1) author / editor list configuration
%\jurabibsetup{authorformat=and} % uses 'und' instead of 'u.'
% therefore define your own abbreviated conjunction: 
% an 'and before last author explicetly written conjunction

% for authors in citations
\renewcommand*{\jbbtasep}{ a.\ } % bta = between two authors sep
\renewcommand*{\jbbfsasep}{, } % bfsa = between first and second author sep
\renewcommand*{\jbbstasep}{, a.\ }% bsta = between second and third author sep
% for editors in citations
\renewcommand*{\jbbtesep}{ a.\ } % bta = between two authors sep
\renewcommand*{\jbbfsesep}{, } % bfsa = between first and second author sep
\renewcommand*{\jbbstesep}{, a.\ }% bsta = between second and third author sep

% for authors in literature list
\renewcommand*{\bibbtasep}{ a.\ } % bta = between two authors sep
\renewcommand*{\bibbfsasep}{, } % bfsa = between first and second author sep
\renewcommand*{\bibbstasep}{, a.\ }% bsta = between second and third author sep
% for editors  in literature list
\renewcommand*{\bibbtesep}{ a.\ } % bte = between two editors sep
\renewcommand*{\bibbfsesep}{, } % bfse = between first and second editor sep
\renewcommand*{\bibbstesep}{, a.\ }% bste = between second and third editor sep

% use: name, forname, forname lastname u. forname lastname
\jurabibsetup{authorformat=firstnotreversed}
\jurabibsetup{authorformat=italic}

%%% (2) title configuration
% in every case print the title, let it be seperated from the 
% author by a colon and use the slanted font
\jurabibsetup{titleformat={all,colonsep}}
%\renewcommand*{\jbtitlefont}{\textit}

%%% (3) seperators in bib data
% separate bibliographical hints and page hints by a comma
\jurabibsetup{commabeforerest}

%%% (4) specific configuration of bibdata in quotes / footnote
% use a.a.O if possible
\jurabibsetup{ibidem=strict}
% replace ugly a.a.O. by translation of ders., a.a.O.
\AddTo\bibsgerman{
  \renewcommand*{\ibidemname}{Id., l.c.}
  \renewcommand*{\ibidemmidname}{id., l.c.}
}
\renewcommand*{\samepageibidemname}{Id., ibid.}
\renewcommand*{\samepageibidemmidname}{id., ibid.}


%%% (5) specific configuration of bibdata in bibliography
% ever an in: before journal and collection/book-tiltes 
\renewcommand*{\bibbtsep}{in: }
\renewcommand*{\bibjtsep}{in: }
% ever a colon after author names 
\renewcommand*{\bibansep}{: }
% ever a semi colon after the title
% \AddTo\bibsgerman{\renewcommand*{\urldatecomment}{Referenzdownload: }}
\renewcommand*{\bibatsep}{; }
% ever a comma before date/year
\renewcommand*{\bibbdsep}{, }

% let jurabib insert the S. and p. information
% no S. necessary in bib-files and in cites/footcites
\jurabibsetup{pages=format}

% use a compressed literature-list using a small line indent
\jurabibsetup{bibformat=compress}
\setlength{\jbbibhang}{1em}

% which follows the design of the cites and offers comments
\jurabibsetup{biblikecite}

% print annotations into bibliography
\jurabibsetup{annote}
\renewcommand*{\jbannoteformat}[1]{{ \itshape #1 }}

%refine the prefix of url download
\AddTo\bibsgerman{\renewcommand*{\urldatecomment}{reference download: }}

% we want to have the year of articles in brackets
\renewcommand*{\bibaldelim}{(}
\renewcommand*{\bibardelim}{)}

% in english version Nr. must be replaced by No.
\renewcommand*{\artnumberformat}[1]{\unskip,\space No.~#1}
\renewcommand*{\pernumberformat}[1]{\unskip\space No.~#1}%
\renewcommand*{\revnumberformat}[1]{\unskip\space No.~#1}%


%Reformatierung Seriestitels and Seriesnumber
\DeclareRobustCommand{\numberandseries}[2]{%
\unskip\unskip%,
\space\bibsnfont{(=~#2}%
\ifthenelse{\equal{#1}{}}{)}{, [Vol./No.]~#1)}%
}%


% Local Variables:
% mode: latex
% fill-column: 80
% End:


% language specific hyphenation
% Telekom osCompendium osHyphenation Include Module
%
% (c) Karsten Reincke, Deutsche Telekom AG, Darmstadt 2011
%
% This LaTeX-File is licensed under the Creative Commons Attribution-ShareAlike
% 3.0 Germany License (http://creativecommons.org/licenses/by-sa/3.0/de/): Feel
% free 'to share (to copy, distribute and transmit)' or 'to remix (to adapt)'
% it, if you '... distribute the resulting work under the same or similar
% license to this one' and if you respect how 'you must attribute the work in
% the manner specified by the author ...':
%
% In an internet based reuse please link the reused parts to www.telekom.com and
% mention the original authors and Deutsche Telekom AG in a suitable manner. In
% a paper-like reuse please insert a short hint to www.telekom.com and to the
% original authors and Deutsche Telekom AG into your preface. For normal
% quotations please use the scientific standard to cite.
%
% [ File structure derived from 'mind your Scholar Research Framework' 
%   mycsrf (c) K. Reincke CC BY 3.0  http://mycsrf.fodina.de/ ]
%


\hyphenation{rein-cke}
\hyphenation{OS-LiC}
\hyphenation{ori-gi-nal}


%%% (3) layout page configuration %%%

% select the visible parts of a page
% S.31: { plain|empty|headings|myheadings }
%\pagestyle{myheadings}
\pagestyle{headings}

% select the wished style of page-numbering
% S.32: { arabic,roman,Roman,alph,Alph }
\pagenumbering{arabic}
\setcounter{page}{1}

% select the wished distances using the general setlength order:
% S.34 { baselineskip| parskip | parindent }
% - general no indent for paragraphs
\setlength{\parindent}{0pt}
\setlength{\parskip}{1.2ex plus 0.2ex minus 0.2ex}


%%% (4) general package activation %%%
%\usepackage{utopia}
%\usepackage{courier}
%\usepackage{avant}
\usepackage[dvips]{epsfig}

% graphic
\usepackage{graphicx,color}
\usepackage{array}
\usepackage{shadow}
\usepackage{fancybox}

%- start(footnote-configuration)
%  flush the cite numbers out of the vertical line and let
%  the footnote text directly start in the left vertical line
\usepackage[marginal]{footmisc}
%- end(footnote-configuration)

\begin{document}

%% use all entries of the bliography

%%-- start(titlepage)
\titlehead{Literaturexzerpt}
\subject{Autor(en): Stallman / Stallman1999a}
\title{Titel: The GNU Project}
\subtitle{Jahr: 1999 / 2002 }
\author{K. Reincke% Telekom osCompendium License Include Module
%
% (c) Karsten Reincke, Deutsche Telekom AG, Darmstadt 2011
%
% This LaTeX-File is licensed under the Creative Commons Attribution-ShareAlike
% 3.0 Germany License (http://creativecommons.org/licenses/by-sa/3.0/de/): Feel
% free 'to share (to copy, distribute and transmit)' or 'to remix (to adapt)'
% it, if you '... distribute the resulting work under the same or similar
% license to this one' and if you respect how 'you must attribute the work in
% the manner specified by the author ...':
%
% In an internet based reuse please link the reused parts to www.telekom.com and
% mention the original authors and Deutsche Telekom AG in a suitable manner. In
% a paper-like reuse please insert a short hint to www.telekom.com and to the
% original authors and Deutsche Telekom AG into your preface. For normal
% quotations please use the scientific standard to cite.
%
% [ File structure derived from 'mind your Scholar Research Framework' 
%   mycsrf (c) K. Reincke CC BY 3.0  http://mycsrf.fodina.de/ ]
%
\footnote{
This text is licensed under the Creative Commons Attribution-ShareAlike 3.0 Germany
License (http://creativecommons.org/licenses/by-sa/3.0/de/): Feel free \enquote{to
share (to copy, distribute and transmit)} or \enquote{to remix (to
adapt)} it, if you \enquote{[\ldots] distribute the resulting work under the
same or similar license to this one} and if you respect how \enquote{you
must attribute the work in the manner specified by the author(s)
[\ldots]}):
\newline
In an internet based reuse please mention the initial authors in a suitable
manner, name their sponsor \textit{Deutsche Telekom AG} and link it to
\texttt{http://www.telekom.com}. In a paper-like reuse please insert a short
hint to \texttt{http://www.telekom.com}, to the initial authors, and to their
sponsor \textit{Deutsche Telekom AG} into your preface. For normal quotations
please use the scientific standard to cite.
\newline
{ \tiny \itshape [derived from myCsrf (= 'mind your Scholar Research Framework') 
\copyright K. Reincke CC BY 3.0  http://mycsrf.fodina.de/)] }}}

%\thanks{den Autoren von KOMA-Script und denen von Jurabib}
\maketitle
%%-- end(titlepage)
%\nocite{*}

\begin{abstract}
\noindent
\cite[(in:)][]{StaGay2002a} \\
\noindent
\cite[(ist:)][]{Stallman1999a} \\
Das Werk / The work\footcite[][]{Stallman1999a} \\
\noindent \itshape
\ldots  
\\
\noindent
\ldots
\end{abstract}
\footnotesize
%\tableofcontents
\normalsize

\section{Line of Thought}

The first line of telling is the line of telling the history of GNU and of RMS
itself. The second more systematical line is RMS’ consistent view on Free
software.

\subsection{The History of GNU as it’s told by RMS himself}

\begin{itemize}
  \item For RMS himself his first phase of using free software is the time
  starting in 1971 \enquote{at the MIT Artificial Intelligence LAB}. RMS had
  to \enquote{improve} the \enquote{[\ldots] timesharing operating system
  called ITS (the Incompatible Timesharing System) [\ldots] written in assembler
  language for the Digital PDP-10 [\ldots]}. In these nearly ten years he
  was \enquote{part of a software-sharing community} which \enquote{[\ldots] did
  not call (their) software ’free software,’ because that term did not yet
  exist; but that is what it was}\footcite[cf][15]{Stallman1999a}
  \item Following RMS this early free community was destroyed by at least two
  movements:
  \begin{itemize}
  	\item First, \enquote{in 1981, the spin-off company Symbolics hired away nearly
  	all of the hackers from the AI Lab, and the depopulated community was unable to
  	maintain itself}\footcite[cf][15]{Stallman1999a}
  	\item Second, the old machines and the known operating system ICS were
  	replaced by modern computers which \enquote{ [\ldots] had their own operating
  	systems, but none of them were free software [\ldots]}. The
  	\enquote{nondisclosure agreement}, which must be signed \enquote{[\ldots]
  	even to get an executable copy} was seen by RMS as the \enquote{[\ldots]
  	promise not to help your neighbour}: \enquote{A cooperating community was
  	forbidden.}\footcite[cf][16]{Stallman1999a}
  \end{itemize}
  \item The solution was to develop an free operating system not owned by any
  company and called GNU - \enquote{a recursive acronym for ’GNU’s Not
  Unix’}. Following the understanding of the 1970’s an operating system
  was not only kernel, but \enquote{[\ldots] included command processors,
  assemblers, compilers, interpreters, debuggers, text editors, mailers, and
  much more}. This meaning fixed the targets of the GNU
  project\footcite[cf][17]{Stallman1999a}
  \item RMS himself doesn’t mention when he found the GNU project. BUt that must
  be 1982 or 1983. [Whe shopuld refer to the corresponding announcement]
  \item \enquote{In January 1984} RMS started with the real work at the GNU
  project: he \enquote{[\ldots] quit (his) job at MIT and began writing GNU
  software}\footcite[cf][18]{Stallman1999a}. To give up the job was
  essential for RMS because he did not want to allow the MIT to cover his
  work\footcite[cf][18]{Stallman1999a}. But nevertheless he was supported by the
  MIT: they \enquote{[\ldots] kindly invited (him) to keep using the lab’s
  facilities}\footcite[cf][19]{Stallman1999a}:
  \begin{itemize}
    \item First, RMS evaluated wether he could reuse any existing compiler.
    Later on he had to write hiw own compiler, the
    GCC\footcite[cf][19]{Stallman1999a}
    \item From Sepmtember 1984 on he wrote the GNU emacs
    editor\footcite[cf][19]{Stallman1999a}
  \end{itemize}
  \item During these months while RMS had no job, he \enquote{[\ldots] was looking
  for ways to make money from free software}. The solution was to sell
  tapes with Emacs on it \enquote{[\ldots] for a fee of
  \$150}\footcite[cf][19f]{Stallman1999a}. So one can see that in the own
  eyes of RMS earning money with free software is allowed
  \item In 1985 RMS and some friends \enquote{[\ldots] created the Free Software
  Foundation, a tax-exempt charity for free software
  develeopment}\footcite[cf][21]{Stallman1999a}
\end{itemize}


\subsection{Some systematical issues}
\subsubsection{\enquote{Free as in Freedom}}
’Free software’ is defined by four features:
\begin{itemize}
  \item First, \enquote{ you have the freedom to run the program, for any
  purpose}\footcite[cf][18]{Stallman1999a}
  \item Second, \enquote{you have the freedom to modify the program to suit your
  needs}\footcite[cf][18]{Stallman1999a}
  \item Third, \enquote{you have the freedom to redistribute copies, either
  gratis or for a fee}\footcite[cf][18]{Stallman1999a}
  \item Fourthly \enquote{you have the freedom to distribute modified versions of
  the program, so that the community can benefit from your
  improvements}\footcite[cf][18]{Stallman1999a}
\end{itemize}

Again RMS highlights that \enquote{[\ldots] ’free’ refers to freedom, not to
price}, that free software \enquote{[\ldots] has nothing to do with
price}. Quite the reverse: \enquote{[\ldots] the freedom to sell copies is
crucial: collections of free software sold on CD-ROMs are important for the
community, and selling them is an important way to raise funds for free software
development.}\footcite[cf][18]{Stallman1999a} 

\subsubsection{\enquote{Copyleft}}
The problem of free, but non copylefted software is explained by the example of
the X Window System. RMS described that the goal of the X Window developer was
’success’ in the sense of ’having many users’ and that they therefore
\enquote{[\ldots] released (it) as free software with a permissive license}.
Thus the Unix companies did what they were allowed to do: \enquote{they added X to
their proprietary Unix systems, in binary form only, and covered by the same
nondisclosure agreement}\footcite[cf][20]{Stallman1999a}. 

For preventing his own software from such an use RMS invented the
\enquote{Copyleft}, which \enquote{[\ldots] uses copyright law} in
another than normal use: \enquote{instead of a means of privatizing software, it
becomes a means of keeping software free}\footcite[cf][20]{Stallman1999a}:
\begin{quote}\enquote{The central idea of copyleft is that we give everyone
permission to run the program, copy the program, modify the program, and
distribute modified versions - but not permission to add restrictions of their
own. Thus, the crucial freedoms that define ’free software’ are guaranteed to
everyone who has a copy; they become inalienable
rights.}\footcite[cf][20]{Stallman1999a}
\end{quote}

NOTE: essential is: \emph{everyone who has a copy} doesn’t mean everyone
in the world!!!!

The term Copyleft is initially invented as a kind of joke by \enquote{Don
Hopkins} who sent a letter with an envelope to RMS containing the sentence
\enquote{ Copyleft - all rights reversed} instead of ’Copyright - all rights
reserved’\footcite[cf][21]{Stallman1999a} 


\subsubsection{LGPL}

RMS mentions that the GNU C library uses \enquote{a special kind of
copyleft} fixed in the LGPL \enquote{[\ldots] which gives permission to link
proprietary software with the library}\footcite[cf][23]{Stallman1999a}.
This is done as \enquote{ a matter of
strategy}\footcite[cf][23]{Stallman1999a}: making \enquote{[\ldots] our C
library available only to free software would not have given free software any
advantage - it would only have discouraged use of our
library}\footcite[cf][23f]{Stallman1999a}

\subsubsection{Non-Free Libraries}

RMS says that \enquote{a non-free library that runs on free operating systems acts
as a trap for free software developer}: its’ attractive features seduce
developers and make the result unfree\footcite[cf][27]{Stallman1999a}, As an
example RMS mentions the situation of KDE between 1996 and 1998 which was based
upon the in those days still unfree QT. Following RMS the GNOME project was
initialized in 1997 as an answer. In the following years QT was given free and
that’s the reason why we still have two Graphical User
Interfaces\footcite[cf][27]{Stallman1999a}.

\subsubsection{Open Source}

RMS mentions that since 1998 there is another term denoting the free software,
the term ’open source’, because in those days \enquote{[\ldots] a part of the
community decided to stop using the term ’free software’ and say ’open source
software’ instead}\footcite[cf][29]{Stallman1999a}.

In the eyes of RMS \enquote{[\ldots] the rhetoric of ’open source’ focuses on the
potential to make high-quality, powerful software, but shuns the ideas of
freedom, community, and principle}\footcite[cf][30]{Stallman1999a}:

\begin{quote}\enquote{’Freesoftware’ and ’open Source’ describe the same category
of software, more or less, but say different things about the software, and
about the values. The GNU Project continues to use the term ’free software’, tp
express the idea that freedom, not just technology, is
important.}\footcite[cf][30]{Stallman1999a}.
\end{quote}

This viewpoint is underlined by the fact, that the accessibility of the open
sourcecode is ’only’ a necessary condition for the definition of free
software\footcite[cf][18]{Stallman1999a}.


\small
\bibliography{../bibfiles/oscResourcesEn}

\end{document}
