% Telekom osCompendium extract template
%
% (c) Karsten Reincke, Deutsche Telekom AG, Darmstadt 2011
%
% This LaTeX-File is licensed under the Creative Commons Attribution-ShareAlike
% 3.0 Germany License (http://creativecommons.org/licenses/by-sa/3.0/de/): Feel
% free 'to share (to copy, distribute and transmit)' or 'to remix (to adapt)'
% it, if you '... distribute the resulting work under the same or similar
% license to this one' and if you respect how 'you must attribute the work in
% the manner specified by the author ...':
%
% In an internet based reuse please link the reused parts to www.telekom.com and
% mention the original authors and Deutsche Telekom AG in a suitable manner. In
% a paper-like reuse please insert a short hint to www.telekom.com and to the
% original authors and Deutsche Telekom AG into your preface. For normal
% quotations please use the scientific standard to cite.
%
% [ File structure derived from 'mind your Scholar Research Framework' 
%   mycsrf (c) K. Reincke CC BY 3.0  http://mycsrf.fodina.de/ ]

%
% select the document class
% S.26: [ 10pt|11pt|12pt onecolumn|twocolumn oneside|twoside notitlepage|titlepage final|draft
%         leqno fleqn openbib a4paper|a5paper|b5paper|letterpaper|legalpaper|executivepaper openrigth ]
% S.25: { article|report|book|letter ... }
%
% oder koma-skript S.10 + 16
\documentclass[DIV=calc,BCOR=5mm,11pt,headings=small,oneside,abstract=true, toc=bib]{scrartcl}

%%% (1) general configurations %%%
\usepackage[utf8]{inputenc}

%%% (2) language specific configurations %%%
\usepackage[]{a4,ngerman}
\usepackage[english, german, ngerman]{babel}
\selectlanguage{ngerman}

%language specific quoting signs
%default for language emglish is american style of quotes
\usepackage{csquotes}

% jurabib configuration
\usepackage[see]{jurabib}
\bibliographystyle{jurabib}
% Telekom osCompendium German Jurabib Configuration Include Module 
%
% (c) Karsten Reincke, Deutsche Telekom AG, Darmstadt 2011
%
% This LaTeX-File is licensed under the Creative Commons Attribution-ShareAlike
% 3.0 Germany License (http://creativecommons.org/licenses/by-sa/3.0/de/): Feel
% free 'to share (to copy, distribute and transmit)' or 'to remix (to adapt)'
% it, if you '... distribute the resulting work under the same or similar
% license to this one' and if you respect how 'you must attribute the work in
% the manner specified by the author ...':
%
% In an internet based reuse please link the reused parts to www.telekom.com and
% mention the original authors and Deutsche Telekom AG in a suitable manner. In
% a paper-like reuse please insert a short hint to www.telekom.com and to the
% original authors and Deutsche Telekom AG into your preface. For normal
% quotations please use the scientific standard to cite.
%
% [ File structure derived from 'mind your Scholar Research Framework' 
%   mycsrf (c) K. Reincke CC BY 3.0  http://mycsrf.fodina.de/ ]

% the first time cite with all data, later with shorttitle
\jurabibsetup{citefull=first}

%%% (1) author / editor list configuration
%\jurabibsetup{authorformat=and} % uses 'und' instead of 'u.'
% therefore define your own abbreviated conjunction: 
% an 'and before last author explicetly written conjunction

% for authors in citations
\renewcommand*{\jbbtasep}{ u. } % bta = between two authors sep
\renewcommand*{\jbbfsasep}{, } % bfsa = between first and second author sep
\renewcommand*{\jbbstasep}{ u. }% bsta = between second and third author sep
% for editors in citations
\renewcommand*{\jbbtesep}{ u. } % bta = between two authors sep
\renewcommand*{\jbbfsesep}{, } % bfsa = between first and second author sep
\renewcommand*{\jbbstesep}{ u. }% bsta = between second and third author sep

% for authors in literature list
\renewcommand*{\bibbtasep}{ u. } % bta = between two authors sep
\renewcommand*{\bibbfsasep}{, } % bfsa = between first and second author sep
\renewcommand*{\bibbstasep}{ u. }% bsta = between second and third author sep
% for editors  in literature list
\renewcommand*{\bibbtesep}{ u. } % bte = between two editors sep
\renewcommand*{\bibbfsesep}{, } % bfse = between first and second editor sep
\renewcommand*{\bibbstesep}{ u. }% bste = between second and third editor sep

% use: name, forname, forname lastname u. forname lastname
\jurabibsetup{authorformat=firstnotreversed}
\jurabibsetup{authorformat=italic}

%%% (2) title configuration
% in every case print the title, let it be seperated from the 
% author by a colon and use the slanted font
\jurabibsetup{titleformat={all,colonsep}}
%\renewcommand*{\jbtitlefont}{\textit}

%%% (3) seperators in bib data
% separate bibliographical hints and page hints by a comma
\jurabibsetup{commabeforerest}

%%% (4) specific configuration of bibdata in quotes / footnote
% use a.a.O if possible
\jurabibsetup{ibidem=strict}

% replace ugly a.a.O. by ders., a.a.O. resp. ders., ebda.
% but if there are more than one author or girl writers?
\AddTo\bibsgerman{
  \renewcommand*{\ibidemname}{Ds., a.a.O.}
  \renewcommand*{\ibidemmidname}{ds., a.a.O.}
}
\renewcommand*{\samepageibidemname}{Ds., ebda.}
\renewcommand*{\samepageibidemmidname}{ds., ebda.}

%%% (5) specific configuration of bibdata in bibliography
% ever an in: before journal and collection/book-tiltes 
\renewcommand*{\bibbtsep}{in: }
%\renewcommand*{\bibjtsep}{in: }

% ever a colon after author names 
\renewcommand*{\bibansep}{: }
% ever a semi colon after the title 
\renewcommand*{\bibatsep}{; }
% ever a comma before date/year
\renewcommand*{\bibbdsep}{, }

% let jurabib insert the S. and p. information
% no S. necessary in bib-files and in cites/footcites
\jurabibsetup{pages=format}

% use a compressed literature-list using a small line indent
\jurabibsetup{bibformat=compress}
\setlength{\jbbibhang}{1em}

% which follows the design of the cites and offers comments
\jurabibsetup{biblikecite}

% print annotations into bibliography
\jurabibsetup{annote}
\renewcommand*{\jbannoteformat}[1]{{ \itshape #1 }}

%refine the prefix of url download
\AddTo\bibsgerman{\renewcommand*{\urldatecomment}{Referenzdownload: }}

% we want to have the year of articles in brackets
\renewcommand*{\bibaldelim}{(}
\renewcommand*{\bibardelim}{)}

%Umformatierung des Reihentitels und der Reihennummer
\DeclareRobustCommand{\numberandseries}[2]{%
\unskip\unskip%,
\space\bibsnfont{(=~#2}%
\ifthenelse{\equal{#1}{}}{)}{, [Bd./Nr.]~#1)}%
}%

% Local Variables:
% mode: latex
% fill-column: 80
% End:


% language specific hyphenation
% Telekom osCompendium osHyphenation Include Module
%
% (c) Karsten Reincke, Deutsche Telekom AG, Darmstadt 2011
%
% This LaTeX-File is licensed under the Creative Commons Attribution-ShareAlike
% 3.0 Germany License (http://creativecommons.org/licenses/by-sa/3.0/de/): Feel
% free 'to share (to copy, distribute and transmit)' or 'to remix (to adapt)'
% it, if you '... distribute the resulting work under the same or similar
% license to this one' and if you respect how 'you must attribute the work in
% the manner specified by the author ...':
%
% In an internet based reuse please link the reused parts to www.telekom.com and
% mention the original authors and Deutsche Telekom AG in a suitable manner. In
% a paper-like reuse please insert a short hint to www.telekom.com and to the
% original authors and Deutsche Telekom AG into your preface. For normal
% quotations please use the scientific standard to cite.
%
% [ File structure derived from 'mind your Scholar Research Framework' 
%   mycsrf (c) K. Reincke CC BY 3.0  http://mycsrf.fodina.de/ ]
%


\hyphenation{rein-cke}




%%% (3) layout page configuration %%%

% select the visible parts of a page
% S.31: { plain|empty|headings|myheadings }
%\pagestyle{myheadings}
\pagestyle{headings}

% select the wished style of page-numbering
% S.32: { arabic,roman,Roman,alph,Alph }
\pagenumbering{arabic}
\setcounter{page}{1}

% select the wished distances using the general setlength order:
% S.34 { baselineskip| parskip | parindent }
% - general no indent for paragraphs
\setlength{\parindent}{0pt}
\setlength{\parskip}{1.2ex plus 0.2ex minus 0.2ex}


%%% (4) general package activation %%%
%\usepackage{utopia}
%\usepackage{courier}
%\usepackage{avant}
\usepackage[dvips]{epsfig}

% graphic
\usepackage{graphicx,color}
\usepackage{array}
\usepackage{shadow}
\usepackage{fancybox}

%- start(footnote-configuration)
%  flush the cite numbers out of the vertical line and let
%  the footnote text directly start in the left vertical line
\usepackage[marginal]{footmisc}
%- end(footnote-configuration)

\begin{document}

%% use all entries of the bliography

%%-- start(titlepage)
\titlehead{Literaturexzerpt}
\subject{Autor(en): Spindler}
\title{Titel: Rechtsfragen bei Open Source}
\subtitle{Jahr: 2004 }
\author{K. Reincke% Telekom osCompendium License Include Module
%
% (c) Karsten Reincke, Deutsche Telekom AG, Darmstadt 2011
%
% This LaTeX-File is licensed under the Creative Commons Attribution-ShareAlike
% 3.0 Germany License (http://creativecommons.org/licenses/by-sa/3.0/de/): Feel
% free 'to share (to copy, distribute and transmit)' or 'to remix (to adapt)'
% it, if you '... distribute the resulting work under the same or similar
% license to this one' and if you respect how 'you must attribute the work in
% the manner specified by the author ...':
%
% In an internet based reuse please link the reused parts to www.telekom.com and
% mention the original authors and Deutsche Telekom AG in a suitable manner. In
% a paper-like reuse please insert a short hint to www.telekom.com and to the
% original authors and Deutsche Telekom AG into your preface. For normal
% quotations please use the scientific standard to cite.
%
% [ File structure derived from 'mind your Scholar Research Framework' 
%   mycsrf (c) K. Reincke CC BY 3.0  http://mycsrf.fodina.de/ ]
%
\footnote{
This text is licensed under the Creative Commons Attribution-ShareAlike 3.0 Germany
License (http://creativecommons.org/licenses/by-sa/3.0/de/): Feel free \enquote{to
share (to copy, distribute and transmit)} or \enquote{to remix (to
adapt)} it, if you \enquote{[\ldots] distribute the resulting work under the
same or similar license to this one} and if you respect how \enquote{you
must attribute the work in the manner specified by the author(s)
[\ldots]}):
\newline
In an internet based reuse please mention the initial authors in a suitable
manner, name their sponsor \textit{Deutsche Telekom AG} and link it to
\texttt{http://www.telekom.com}. In a paper-like reuse please insert a short
hint to \texttt{http://www.telekom.com}, to the initial authors, and to their
sponsor \textit{Deutsche Telekom AG} into your preface. For normal quotations
please use the scientific standard to cite.
\newline
{ \tiny \itshape [derived from myCsrf (= 'mind your Scholar Research Framework') 
\copyright K. Reincke CC BY 3.0  http://mycsrf.fodina.de/)] }}}

%\thanks{den Autoren von KOMA-Script und denen von Jurabib}
\maketitle
%%-- end(titlepage)
%\nocite{*}

\begin{abstract}
\noindent
Das Werk / The work\footcite[][]{Spindler2004a} \\
\noindent \itshape
\ldots Basiert auf einem Rechtsgutachten. Die meisten Kapitel stammen von
Spindler selbst und behandeln die Typen von Open Source Lizenzen, das
Urheberrecht, das Vertragsrecht und das Haftungsrecht. Am Ende wird zudem ein
Exkurs zum Thema BSD- und Mozilla-Lizenzen angeboten. In allen Fällen wird
juristisch detailliert analysiert und diskutiert. \\
\noindent
\ldots This book is based upon a legal opinion. Most of the chapters are written
by Spindler himself and deal with the Open Source License types, the German
Copyright ('Urheberrecht'), the German contract right ('Vertragsrecht') and the
German liability law ('Haftungsrecht'). At the end it contains an excursus
dealing with BSD and Mozilla licenses. In all cases the chapters deeply discuss
different aspects of the topic.
\end{abstract}
\footnotesize
%\tableofcontents
\normalsize

\section{Line of Thought}

Spindler gibt in seinem Vorwort an, dass die ersten 5 Kapitel Basis eines
\enquote{[\ldots] umfangreichen Rechtsgutachtens) des Herausgebers für den Verband
der Softwareindustire Deutschland (gewesen sei)}. Entsprechend nimmt sich
das Buch vor, die \enquote{[\ldots] die juristische Diskussion in Deutschland über
diese besondere Form der Liz7enzierung von Software
voranzutreiben}\footcite[vgl.][V]{Spindler2004a}.

So ergibt sich für ihn folgender Gedankengang:

\begin{itemize}
  \item Zunächst stellt er sehr kurz die Idee von Open Source
  dar\footcite[vgl.][1ff]{Spindler2004a} um dann die die \enquote{Open Source
  Software Lizenztypen} gegeneinander
  \enquote{abzugrenzen}\footcite[vgl.][9ff]{Spindler2004a}. Er selbst
  konzentriert sich in den folgenden Kapiteln jedoch auf die
  GPL\footcite[vgl.][21ff]{Spindler2004a} und überlässt es seinen Coautoren, die
  BSD- und die Mozilla-Lizenzen zu
  untersuchen\footcite[vgl.][317ff]{Spindler2004a}
  \item Erster Untersuchungspunkt ist das Urheberrecht und die Wechselwirkungen
  zur/für die GPL\footcite[vgl.][21ff]{Spindler2004a}.
  \item Sodann geht es um das Vertragsrecht\footcite[vgl.][151ff]{Spindler2004a}
  \ldots
  \item \ldots und schließlich auch um das
  Haftungsrecht\footcite[vgl.][151ff]{Spindler2004a}.
  \item Anschließend erhält der Autor Wiebe die Möglichkeit, das Patentrecht zu
  erläutern\footcite[vgl.][223ff]{Spindler2004a}, der Autor Heath das
  Kartellrecht\footcite[vgl.][267]{Spindler2004a} und der Autor Heckmann das
  Vergaberecht\footcite[vgl.][281ff]{Spindler2004a} [- alles Punkte, die ich
  nicht ausgewertet habe.]
  \item Von besonderem Interesse ist dann wieder die Darstellung von Arlt,
  Brinkel und Volkmann bzgl. der \enquote{'BSD'- und 'Mozilla'-artigen
  Lizenzen}\footcite[vgl.][317ff]{Spindler2004a} [die ich als eigenes
  Extract erfasst habe]
  \item 
\end{itemize}


\section{Spindler Thesen}

\subsection{Opensource}

\subsubsection{Grenzen der Freiheit: Rechtskonstruktion}

Wesentlich hervorgehoben wird hier, dass Open Source eben nicht in dem Sinne
frei sei, \enquote{[\ldots] dass es an Open Source keinerlei Urheberrechte
gäbe bzw. die Software und der Quellcode gemeinfrei
wären}\footcite[vgl.][2]{Spindler2004a}:
\begin{quote}\enquote{Charakteristisches Merkmal von Open Source Software
ist vielmehr die Verwendung der urheberrechtlichen Verwertungsrechte, um
den Nutzer [\ldots] gerade die Pflichten aufzuerlegen, die zur
kostenfreien Weitergabe und Offenlegung des Quellcodes
führen.}\footcite[][2]{Spindler2004a}
\end{quote}

\subsubsection{Grenzen der Freiheit: Geld}

Eingeschränkt wird das Geschäft mit Open Source nur in einer Hinsicht:
\enquote{allein Lizenzgebühren für die Spftware und den Quellcode dürfen
nicht erhoben werden}, der sonstige kommerzielle Vertrieb von Open Source
Software, bei dem etwa \enquote{[\ldots] Entgelte für Datenträger, Beratung,
Garantien oder sonstige Serviceleistungen verlangt werden} sei keineswegs
ausgeschlossen.\footcite[vgl.][5]{Spindler2004a}

Allerdings, so Spindler, dürfte der Versuch, den Gedanken der kostenlosen
Lizenzierung über 'Fantasiepreise' für die \enquote{physische Distribution}
sozusagen \enquote{unterlaufen} zu wollen, juristisch \enquote{[\ldots]
erheblichen Bedenken ausgsetzt (sein)}\footcite[vgl.][6]{Spindler2004a}

\subsubsection{Lizenzabgrenzung}

\paragraph{GPL}

Grundsätzlich könne die GPL so spezifiziert werden, dass sie dem Nutzer ein
weitgehendes Verwertungsrecht einräume, bei dem ihm \enquote{[\ldots] ihm
die Anfertigung beliebiger (unveränderter) Kopien des Quellcodes und auch
deren Veränderung gestattet wird}. Die modifzierten Werke dürfe der
\enquote{[\ldots] NUtzer wiederum vervielfältigen und verbreiten [\ldots],
sofern er sie wiederum der GPL (unterstelle) und für die Lizenzierung
kein Entgelt (erhebe)}\footcite[vgl.][9]{Spindler2004a}

\paragraph{LGPL}

Als Hauptzweck zur \enquote{Schaffung der LGPL}, der \enquote{Lesser
GPL}\footcite[vgl.][10]{Spindler2004a} führt Spindler an, dass durch die
\enquote{Inkorporation} von GPL lizenzierten \enquote{[\ldots] (Modulen) in
den Ablauf der [umfassenderen KR] Software diese ebenfalls unter die GPL
gestellt werden müsste}. Wären alle Teile eines Betriebssystems mithin
rein GPL lizenziert, wäre eine andersartige Nutzung ausgeschlossen. Um den
Nutzerkreis dennoch zu erweitern, seien eben zentrale Teile des GNU
Betriebssystems LGPL lizenziert, die auf die Auswirkung auf die übergreifende
Software unter gewissen Umständen verzichte\footcite[vgl.][11]{Spindler2004a}.

Sodann kommt eine Darstellung der LGPL, die gefährlich FALSCH ist: Zunächst wird
behauptet, dass \enquote{Programme, die zwar auf eine LGPL-Bibliothek zugreifen,
aber unabhängig hiervon und nicht zusammen mit der Bibliothek vertrieben werden
('in isolation'), [\ldots] nach §5 I LGPL nicht unter diese
Lizenz gestellt werden (müssten), sondern [\ldots] proprietär
bleiben (könnten)}\footcite[vgl.][12]{Spindler2004a} [KR: Wenn das in
aller Strenge so gelte, wäre damit jedwede Distributionen, die die Linux LGPL
Bibliotheken und püropriuetäre Linux-Programnme enthielten, schon mal
untersagt. Das aber ist gerade nicht der Sinn der LGPL! KR]

Sodann wird behauptet, dass diese Freiheit 'getrennt vertrieben, aber zusammen
genutzt' sich gerade bei der \enquote{dynamischen Verlinkung} ändere: 
\begin{quote}\enquote{Dies ändert sich jedoch bei einer Verknüpfung des
Programms mit der Bibliothek (dynamischer Verlinkung) und deren
gemeinsamer Verbreitung, da §5 II, III, LGPL dann das Ergebnis der
'Zusammenarbeit' zwischen Bibliothek und Programm als ein 'neues'
Programm bzw. Werk ansieht.}footcite[vgl.][12]{Spindler2004a}
\end{quote}

[KR: Dies kann aus technischer Sicht so niemals von der LGPL so gemeint gewesen
sein: Beim Starten eines Programms werden die von dem Programm
genutzten Bibliotheken immer dann zusammengefügt und als Einheit in den Speicher
geladen, wenn sie auf der Platte noch als getrennte Dateien abgelegt sind. Dies
ist das wesen des Dynamsichen Linkens. Zweck ist, dass intelligente Linker
mehrfach genutzte Bibliotheken nicht mehrfach in den Arbeitsspeicher laden
müssen, sondern die Einspruingsadressen mehrfach verwenden können. Und diese Art
des Zusammenfügens von Objektcode (dynmaischens Linken) hat überhaupt nichts
damit zu tun, ob Bibliothek und Programm zuvor getrennt oder zusammen auf einer
DVD/Diskette vertrieben worden sind.

Die Einschränkung der LGPL trifft aber genau dann zu, wenn Bibliothek und
Programm bei der Erstellung statisch gelinkt worden sind. Dann bilden sie als
Datei eine Einheit, die bei Ausführen auch nur als ganzes in den
Arbeitsspreicher geladen werden kann. Und ebene hier ist klar, dass es sich als
ganzes um ein abgeleitetes Werk handelt!
 ]

\paragraph{Mozilla Lizenzen}

Die Mozilla-Lizenzen werden ähnlich wie die GPL gesehen, sollen sich aber
dennoch unterscheiden: Die Rechte, die die Mozilla-Lizenzen einräumen,
\enquote{[\ldots] (seien) weitgehend der GPL vergleichbar [im Sinne von
gleichwertig; KR] [\ldots]} und betreffen die \enquote{Nutzung},
die \enquote{Vervielfältigung}, die \enquote{Veränderung}, die
\enquote{Zugänglichmachung} und die
\enquote{Verbreitung}\footcite[vgl.][14]{Spindler2004a}

Unterschiedlich seien in gewisser Hinsicht die Pflichten der Nutzer: Zunächst
sei es \enquote{im Ergebnis unzweifelhaft}, dass der \enquote{[\ldots]
contributor die Pflicht (habe), seine veränderte Software der
Mozilla-Lizenz ebenfalls zu
unterstellen}\footcite[vgl.][15]{Spindler2004a}:
\begin{quote}\enquote{Anders als bei der GPL wird jedoch nicht jede
Veränderung erfasst, sondern nur diejenige, die entweder in dem selben
Quellcode enthalten ist oder - bei mehreren Dateien - in denen Quellcode
geschireben ist. Werden jedoch die Änderungen in einer eigenen Datei
abgespeichert, müssen diese nicht der Mozilla-Lizenz unterstellt und
damit auch nicht freigegeben werden}\footcite[][15]{Spindler2004a}
\end{quote}

Anders gesagt:

\begin{quote}\enquote{Es muss daher nur ein selbständiger Code in einem
eigenständigen File vorliegen, um von der Pflicht zur unentgeltlichen
Freigabe entbunden werden.}\footcite[][15]{Spindler2004a}

[ KR: Damit ist eigentlich klar, wie das KOnstrukt funktioniert: Werden
Änderungen am ursprünglichen Code im Sinne von Dateiänderungen vorgenommen,
müssen diese mozilla-lizenizi
ert werden, sofern sie distribuiert werden. Das ist
wie bei der LGPL und der GPL auch. Werden hingegen Objektdateien dazu gestellt,
die dnamisch gelinkt auf den Mozilla-Objekt-Code zugreifen, dann brauchen sie
nicht veröffentlicht zu werden. Das entspricht der LPGL, aber nicht der GPL.
Mozilla-Lizenzen entsprechen also der LGP, nicht der GPL. ]
\end{quote}

\paragraph{Dual Licensing}


Spindler erwähnt das \enquote{Dual Licensing} als Methode des
Urheberrechtsinhabers, sich verschiedene Wege
offenzuhalten\footcite[vgl.][16]{Spindler2004a}, etwa, wenn man anderen Firmen
eine proprietäre Weiterentwicklung des eigenen Codes gegen Entgeld ermöglichen
will. Allerdings gilt: übernommener und bereits GPL lizenzierter Code kann nicht
in der Zweit- oder DRittverwertung Dual-lizenziert
werden\footcite[vgl.][17]{Spindler2004a}

\paragraph{Andere Formen}

\begin{itemize}
  \item \enquote{ Für Public Domain Software (sei) der (versuchte) vollständige
  Verzicht auf jegliche Urheberechte kennzeichnend, jedenfalls aber das
  \emph{uneingeschränkte Recht der Verwertung} für den
  Nutzer}\footcite[][17]{Spindler2004a} - im Gegnesatz zur GPL dürfe der
  Nutzer mit solch einer Software also im Prinzip verfahren, wie es ihm gefalle,
  also minsbesondere auch proprietarisieren\footcite[][17]{Spindler2004a}.
  Allerdings sei die aktive Freigabe von Public Domain Software in Deutschland
  kaum möglich, eben weil in Deutschland nicht alle Rechte übertragen werden
  können\footcite[][17]{Spindler2004a}. So bleibt etwa die Autorenschaft
  unauflöslich erhalten.
  \item Freeware zeichne sich dadurch aus, \enquote{[\ldots] dass zwar die
  Software selbst kostenlos zur Benutzung überlassen, [\ldots] \emph{der
  Quellcode aber nicht offengelegt} (werde)}
  \item Shared Sourcfe rekuriert auf das MS Modell des Einblicks für ausgewählte
  Kunden ohne das Recht der Änderung, Nutzung, oder
  Weitergabe\footcite[][18]{Spindler2004a}
\end{itemize}

\paragraph{Klassifikation}
Insgesamt ergibt sich für Spindler folgende Taxonomie:

\begin{table}
\footnotesize
\caption{Freie Internetresourcen (ggfls. über UB-Portale DA oder FaM)}
\begin{center}
\begin{tabular}[h]{|p{4cm}|p{2cm}|p{2cm}|p{2cm}|p{2cm}|}
\hline
& kostenlose Nutzung
& unbeschränkter Gebruach
& Quellcode veränderbar
& Direvate kostenlose abzugeben
\\
\hline \hline 
\emph{Shareware} & X & - & - & -\\
\hline
\emph{Freeware / Public Domain Software} & X & X & - & -\\
\hline
\emph{BSD} & X & X & X & -\\
\hline
\emph{LGPL} & X & X & X & X\\
\hline
\emph{GPL} & X & X & X & X\\
\hline

Tabelle\footcite[vgl.][19]{Spindler2004a}
[KR: Die Schwäche der Klassifikation zeuigt asich sofort qwieder daran, dass
zwi9schen GPL und LGPL sdo nicht unterschieden werden kann]
\end{tabular}
\end{center}
\end{table}

\subsection{Urheberrecht}

\begin{itemize}
  \item Grundsätzlich gilt, dass \enquote{nach §69c UrhG [\ldots] die
  Verielfältigung, die Bearbeitung und die Verbreitung einer Software nur
  mit Erlaubnis des Urhebers zulässig (ist)}\footcite[][22]{Spindler2004a}
  \item Mit der GPL geht gerade systematisch intendiert \enquote{keine
  Aufgabe des Urheberrechts} einher\footcite[][22]{Spindler2004a},
  vielmehr \enquote{[\ldots] verwendet (die Open Source Bewegung) lediglich
  die von Urheberrecht gebotenen Instrumente, um einen entsprechenden
  Zweck zu erreichen}\footcite[vgl.][26]{Spindler2004a} - daraus könne
  \enquote{[\ldots] kein allgemeiner Verstoß der GPL gegen ungeschriebene
  Grundsätze des Urheberrechts abgeleitet werden
  [\ldots]}\footcite[vgl.][25]{Spindler2004a}
  \item Wichtig ist bei der Veröffentlichung aber, dass einzelne Autoren, die
  man einem Code mitarbeiten, als \enquote{Miturheber}
  gelten\footcite[vgl.][27]{Spindler2004a}, und zwar selbst dann, wenn
  eingereichte Beiträge nur per Commiteebeschluss in das Werk
  eingehen\footcite[vgl.][28]{Spindler2004a} [worau8s zu folgern wäre, dass
  sich Open Source Projekte die Freigabe des eingereichten Codes explizit
  bestätigen lassen sollten]
\end{itemize}

\subsection{Haftungsrecht}

Ausgehend von der Deutung des \enquote{isolierten Erwerbs} von Open Source
Software als einer \enquote{Schenkung} ergibt sich für die Haftung gerade
nicht die vollständige Freistellung, \enquote{Vorsatz und grobe
Fahrlässigkeit} muss sich der Schenkende als Verursacher zurechnen lassen
und für entsprechende Schäden haften\footcite[vgl.][169f]{Spindler2004a} - daran
ändert auch nichts der §12, will sagen: die Haftungsausschlussklausel der GPL,
mit der jegliche Haftung ausgeschlossen werden soll, was aber nach BGB
grundsätzlich nicht zulässig sei\footcite[vgl.][170]{Spindler2004a}.
Ähnlich\footcite[vgl.][205]{Spindler2004a}

Interessant ist der Hinweis, dass der \enquote{Nutzer} von Open Source
Software - eben weil nur rudimentäre Gewährleistungspflichten bestehen [und
wie zu ergänzen wäre: weil er um diese Tatsache weiß] - \enquote{[\ldots]
je nach Gefahrenpotential, in dessen Zusammenhang Open Source eingesetzt
wird, verpflichtet sein (kann), entsprechende externe Beratungsleistungen
undService einzukaufen, wenn er sich nicht eines Organisationsmangels
schuldig machen möchte}\footcite[vgl.][221]{Spindler2004a}

\small


\bibliography{../bibfiles/oscResourcesDe}

\end{document}
