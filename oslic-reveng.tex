% Telekom osCompendium cloak file English text
%
% (c) Karsten Reincke, Deutsche Telekom AG, Darmstadt 2011
%
% This LaTeX-File is licensed under the Creative Commons Attribution-ShareAlike
% 3.0 Germany License (http://creativecommons.org/licenses/by-sa/3.0/de/): Feel
% free 'to share (to copy, distribute and transmit)' or 'to remix (to adapt)'
% it, if you '... distribute the resulting work under the same or similar
% license to this one' and if you respect how 'you must attribute the work in
% the manner specified by the author ...':
%
% In an internet based reuse please link the reused parts to www.telekom.com and
% mention the original authors and Deutsche Telekom AG in a suitable manner. In
% a paper-like reuse please insert a short hint to www.telekom.com and to the
% original authors and Deutsche Telekom AG into your preface. For normal
% quotations please use the scientific standard to cite.
%
% [ File structure derived from 'mind your Scholar Research Framework' 
%   mycsrf (c) K. Reincke CC BY 3.0  http://mycsrf.fodina.de/ ]

%\documentclass[DIV=calc,BCOR=5mm,12pt,headings=small,oneside,toc=bib,draft]{scrbook}
\documentclass[DIV=calc,BCOR=5mm,12pt,headings=small,oneside,toc=bib]{scrbook}

%%% (1) general configurations %%%
\usepackage[utf8]{inputenc}

%%% (2) language specific configurations %%%
\usepackage[]{a4}
\usepackage[english]{babel}
\selectlanguage{english}

\usepackage{microtype}

\usepackage{to_oscad/oscad}

%language specific quoting signs
%default for language emglish is american style of quotes
%\usepackage[english=british]{csquotes}
\usepackage[english=american]{csquotes}

% jurabib configuration
\usepackage[see]{jurabib}
\bibliographystyle{jurabib}
% do not comment the litrature any longer
% Telekom osCompendium English Jurabib Configuration Include Module 
%
% (c) Karsten Reincke, Deutsche Telekom AG, Darmstadt 2011
%
% This LaTeX-File is licensed under the Creative Commons Attribution-ShareAlike
% 3.0 Germany License (http://creativecommons.org/licenses/by-sa/3.0/de/): Feel
% free 'to share (to copy, distribute and transmit)' or 'to remix (to adapt)'
% it, if you '... distribute the resulting work under the same or similar
% license to this one' and if you respect how 'you must attribute the work in
% the manner specified by the author ...':
%
% In an internet based reuse please link the reused parts to www.telekom.com and
% mention the original authors and Deutsche Telekom AG in a suitable manner. In
% a paper-like reuse please insert a short hint to www.telekom.com and to the
% original authors and Deutsche Telekom AG into your preface. For normal
% quotations please use the scientific standard to cite.
%
% [ File structure derived from 'mind your Scholar Research Framework' 
%   mycsrf (c) K. Reincke CC BY 3.0  http://mycsrf.fodina.de/ ]

% the first time cite with all data, later with shorttitle
\jurabibsetup{citefull=first}

%%% (1) author / editor list configuration
%\jurabibsetup{authorformat=and} % uses 'und' instead of 'u.'
% therefore define your own abbreviated conjunction: 
% an 'and before last author explicetly written conjunction

% for authors in citations
\renewcommand*{\jbbtasep}{ a.\ } % bta = between two authors sep
\renewcommand*{\jbbfsasep}{, } % bfsa = between first and second author sep
\renewcommand*{\jbbstasep}{, a.\ }% bsta = between second and third author sep
% for editors in citations
\renewcommand*{\jbbtesep}{ a.\ } % bta = between two authors sep
\renewcommand*{\jbbfsesep}{, } % bfsa = between first and second author sep
\renewcommand*{\jbbstesep}{, a.\ }% bsta = between second and third author sep

% for authors in literature list
\renewcommand*{\bibbtasep}{ a.\ } % bta = between two authors sep
\renewcommand*{\bibbfsasep}{, } % bfsa = between first and second author sep
\renewcommand*{\bibbstasep}{, a.\ }% bsta = between second and third author sep
% for editors  in literature list
\renewcommand*{\bibbtesep}{ a.\ } % bte = between two editors sep
\renewcommand*{\bibbfsesep}{, } % bfse = between first and second editor sep
\renewcommand*{\bibbstesep}{, a.\ }% bste = between second and third editor sep

% use: name, forname, forname lastname u. forname lastname
\jurabibsetup{authorformat=firstnotreversed}
\jurabibsetup{authorformat=italic}

%%% (2) title configuration
% in every case print the title, let it be seperated from the 
% author by a colon and use the slanted font
\jurabibsetup{titleformat={all,colonsep}}
%\renewcommand*{\jbtitlefont}{\textit}

%%% (3) seperators in bib data
% separate bibliographical hints and page hints by a comma
\jurabibsetup{commabeforerest}

%%% (4) specific configuration of bibdata in quotes / footnote
% use a.a.O if possible
\jurabibsetup{ibidem=strict}
% replace ugly a.a.O. by translation of ders., a.a.O.
\AddTo\bibsgerman{
  \renewcommand*{\ibidemname}{Id., l.c.}
  \renewcommand*{\ibidemmidname}{id., l.c.}
}
\renewcommand*{\samepageibidemname}{Id., ibid.}
\renewcommand*{\samepageibidemmidname}{id., ibid.}


%%% (5) specific configuration of bibdata in bibliography
% ever an in: before journal and collection/book-tiltes 
\renewcommand*{\bibbtsep}{in: }
\renewcommand*{\bibjtsep}{in: }
% ever a colon after author names 
\renewcommand*{\bibansep}{: }
% ever a semi colon after the title
% \AddTo\bibsgerman{\renewcommand*{\urldatecomment}{Referenzdownload: }}
\renewcommand*{\bibatsep}{; }
% ever a comma before date/year
\renewcommand*{\bibbdsep}{, }

% let jurabib insert the S. and p. information
% no S. necessary in bib-files and in cites/footcites
\jurabibsetup{pages=format}

% use a compressed literature-list using a small line indent
\jurabibsetup{bibformat=compress}
\setlength{\jbbibhang}{1em}

% which follows the design of the cites and offers comments
\jurabibsetup{biblikecite}

% print annotations into bibliography
% \jurabibsetup{annote}
% \renewcommand*{\jbannoteformat}[1]{{ \itshape #1 }}

%refine the prefix of url download
\AddTo\bibsgerman{\renewcommand*{\urldatecomment}{reference download: }}

% we want to have the year of articles in brackets
\renewcommand*{\bibaldelim}{(}
\renewcommand*{\bibardelim}{)}

% in english version Nr. must be replaced by No.
\renewcommand*{\artnumberformat}[1]{\unskip,\space No.~#1}
\renewcommand*{\pernumberformat}[1]{\unskip\space No.~#1}%
\renewcommand*{\revnumberformat}[1]{\unskip\space No.~#1}%


%Reformatierung Seriestitels and Seriesnumber
\DeclareRobustCommand{\numberandseries}[2]{%
\unskip\unskip%,
\space\bibsnfont{(=~#2}%
\ifthenelse{\equal{#1}{}}{)}{, [Vol./No.]~#1)}%
}%


% Local Variables:
% mode: latex
% fill-column: 80
% End:
 %no annotations

% language specific hyphenation
% Telekom osCompendium osHyphenation Include Module
%
% (c) Karsten Reincke, Deutsche Telekom AG, Darmstadt 2011
%
% This LaTeX-File is licensed under the Creative Commons Attribution-ShareAlike
% 3.0 Germany License (http://creativecommons.org/licenses/by-sa/3.0/de/): Feel
% free 'to share (to copy, distribute and transmit)' or 'to remix (to adapt)'
% it, if you '... distribute the resulting work under the same or similar
% license to this one' and if you respect how 'you must attribute the work in
% the manner specified by the author ...':
%
% In an internet based reuse please link the reused parts to www.telekom.com and
% mention the original authors and Deutsche Telekom AG in a suitable manner. In
% a paper-like reuse please insert a short hint to www.telekom.com and to the
% original authors and Deutsche Telekom AG into your preface. For normal
% quotations please use the scientific standard to cite.
%
% [ File structure derived from 'mind your Scholar Research Framework' 
%   mycsrf (c) K. Reincke CC BY 3.0  http://mycsrf.fodina.de/ ]
%


\hyphenation{rein-cke}
\hyphenation{OS-LiC}
\hyphenation{ori-gi-nal}


%%% (3) layout page configuration %%%

% select the visible parts of a page
% S.31: { plain|empty|headings|myheadings }
%\pagestyle{myheadings}
\pagestyle{headings}

% select the wished style of page-numbering
% S.32: { arabic,roman,Roman,alph,Alph }
\pagenumbering{arabic}
\setcounter{page}{1}

% select the wished distances using the general setlength order:
% S.34 { baselineskip| parskip | parindent }
% - general no indent for paragraphs
\setlength{\parindent}{0pt}
\setlength{\parskip}{1.2ex plus 0.2ex minus 0.2ex}

%%% (4) general package activation %%%
%\usepackage{utopia}
%\usepackage{courier}
%\usepackage{avant}

% graphic
\usepackage{graphicx,color}
\usepackage{array}
\usepackage{shadow}
\usepackage{fancybox}
\usepackage{alltt}

%- start(footnote-configuration)
%  flush the cite numbers out of the vertical line and let
%  the footnote text directly start in the left vertical line
% \usepackage[marginal,hang]{footmisc}
% \renewcommand\footnotemargin{1.5em}

% formatting the footnote with koma script tools
% \deffootnote[1em]{1.5em}{1em}{\textsuperscript{\thefootnotemark}}
\deffootnote[1.5em]{1.5em}{1.5em}{\textsuperscript{\thefootnotemark)\ }}


%\deffootnote[0em]{1.5em}{1em}{\textsuperscript{\thefootnotemark}}
%- end(footnote-configuration)

% %- start(endnote-configuration) uncomment to activate
% % Let all notes being marked with \endnote instead of \footnote
% % become endnotes. This set of endnotes replaces the next 
% % arising command \theendnotes - even if it is not located
% % at the end of the text.
% 
% \usepackage{endnotes}
% 
% % Format endnotes as Block with indention - Solution 1
% %\renewcommand\enoteformat{%
% %   \noindent\theenmark.) \ \hangindent .7\parindent%
% %}
% 
% % Format endnotes as Block with indention - Solution 2
% \makeatletter
% \def\enoteformat{\rightskip\z@ \leftskip0em \parindent=0em \parskip=0em
% \leavevmode\llap{\hbox{\@theenmark.~}}}
% \makeatother
% 
% \renewcommand\notesname{Annotations}
% % additionally we shall active a special jurabib option
% % if we want to get all jurabib footnotes as endnotes
% \jurabibsetup{citetoend=true}
% %- end(footnote-configuration)

% - additional packages

\usepackage{tikz}
\usetikzlibrary{arrows}
\usetikzlibrary{shapes,snakes}
\usetikzlibrary{positioning}
\usetikzlibrary{decorations.text}
\usetikzlibrary{trees}

\usepackage{multirow}

%RPD%%%\usepackage{blindtext}
\usepackage{caption}

\usetikzlibrary{matrix}

\usepackage{amsmath}
\usepackage{amsfonts}
\usepackage{amssymb}
\usepackage{wasysym}
\usepackage{chngcntr}
\usepackage{nameref}

\counterwithout{footnote}{chapter}

\usepackage[intoc]{nomencl}
\let\abbr\nomenclature
% Modify Section Title of nomenclature
\renewcommand{\nomname}{Periodicals, Shortcuts, and Abbreviations}
%\renewcommand{\nomname}{Periodika, ihre Kurzformen und generelle Abkürzungen}

% insert point between abbrewviation and explanation
\setlength{\nomlabelwidth}{.24\hsize}
\renewcommand{\nomlabel}[1]{#1 \dotfill}
% reduce the line distance
\setlength{\nomitemsep}{-\parsep}
\makenomenclature

% depth of contents
\setcounter{secnumdepth}{5}
\setcounter{tocdepth}{5}

% Hyperlinks
\usepackage{hyperref}
\hypersetup{bookmarks=true,breaklinks=true,colorlinks=true,citecolor=blue,draft=false}

% Compatibility command if hyperref cannot be used
%\newcommand{\texorpdfstring}[2]{#1}

% Abbreviations
\newcommand{\oslic}{OSLiC}

% Often Cited Sources
% --------------------
% first (optional) argument is text at beginning, defaults to "cf."
% second argument is location, like "§2"; must be "wp" if no paragraph is given 
\newcommand*{\citeAGPL}[2][cf.]{\footcite[#1][\nopage wp #2]{Agpl30OsiLicense2007a}}
\newcommand*{\citeAPL}[2][cf.]{\footcite[#1][\nopage wp #2]{Apl20OsiLicense2004a}}
\newcommand*{\citeBSDnew}[2][cf.]{\footcite[#1][\nopage wp #2]{BsdLicense3Clause}}
\newcommand*{\citeBSDsimple}[2][cf.]{\footcite[#1][\nopage wp #2]{BsdLicense2Clause}}
\newcommand*{\citeCDDL}[2][cf.]{\footcite[#1][\nopage wp #2]{Cddl10OsiLicense2004a}}
\newcommand*{\citeEPL}[2][cf.]{\footcite[#1][\nopage wp #2]{Epl10OsiLicense2005a}}
\newcommand*{\citeEUPL}[2][cf.]{\footcite[#1][\nopage wp #2]{Eupl11OsiLicense2007a}}
\newcommand*{\citeGPLthree}[2][cf.]{\footcite[#1][\nopage wp #2]{Gpl30OsiLicense2007a}}
\newcommand*{\citeGPLtwo}[2][cf.]{\footcite[#1][\nopage wp #2]{Gpl20OsiLicense1991a}}
\newcommand*{\citeLGPLthree}[2][cf.]{\footcite[#1][\nopage wp #2]{Lgpl30OsiLicense2007a}}
\newcommand*{\citeLGPLtwo}[2][cf.]{\footcite[#1][\nopage wp #2]{Lgpl21OsiLicense1999a}}
\newcommand*{\citeMIT}[2][cf.]{\footcite[#1][\nopage wp #2]{MitLicense2012a}}
\newcommand*{\citeMPL}[2][cf.]{\footcite[#1][\nopage wp #2]{Mpl20OsiLicense2013a}}
\newcommand*{\citeMSPL}[2][cf.]{\footcite[#1][\nopage wp #2]{MsplOsiLicense2013a}}
\newcommand*{\citePGL}[2][cf.]{\footcite[#1][\nopage wp #2]{PglOsiLicense2013a}}
\newcommand*{\citePHP}[2][cf.]{\footcite[#1][\nopage wp #2]{Php30OsiLicense2013a}}
%%%%%%%%%%%%%%

\begin{document}

%% use all entries of the bliography
\nocite{*}

%%-- start(titlepage)
\titlehead{Version 1.0}

\subject{\small \itshape How to Achieve Open Source License Compliance} 

\title{Open Source Software and Reverse Engineering}

\subtitle{Extract of the OSLiC% Telekom osCompendium License Include Module
%
% (c) Karsten Reincke, Deutsche Telekom AG, Darmstadt 2011
%
% This LaTeX-File is licensed under the Creative Commons Attribution-ShareAlike
% 3.0 Germany License (http://creativecommons.org/licenses/by-sa/3.0/de/): Feel
% free 'to share (to copy, distribute and transmit)' or 'to remix (to adapt)'
% it, if you '... distribute the resulting work under the same or similar
% license to this one' and if you respect how 'you must attribute the work in
% the manner specified by the author ...':
%
% In an internet based reuse please link the reused parts to www.telekom.com and
% mention the original authors and Deutsche Telekom AG in a suitable manner. In
% a paper-like reuse please insert a short hint to www.telekom.com and to the
% original authors and Deutsche Telekom AG into your preface. For normal
% quotations please use the scientific standard to cite.
%
% [ File structure derived from 'mind your Scholar Research Framework' 
%   mycsrf (c) K. Reincke CC BY 3.0  http://mycsrf.fodina.de/ ]
%
\footnote{
This text is licensed under the Creative Commons Attribution-ShareAlike 3.0 Germany
License (http://creativecommons.org/licenses/by-sa/3.0/de/): Feel free \enquote{to
share (to copy, distribute and transmit)} or \enquote{to remix (to
adapt)} it, if you \enquote{[\ldots] distribute the resulting work under the
same or similar license to this one} and if you respect how \enquote{you
must attribute the work in the manner specified by the author(s)
[\ldots]}):
\newline
In an internet based reuse please mention the initial authors in a suitable
manner, name their sponsor \textit{Deutsche Telekom AG} and link it to
\texttt{http://www.telekom.com}. In a paper-like reuse please insert a short
hint to \texttt{http://www.telekom.com}, to the initial authors, and to their
sponsor \textit{Deutsche Telekom AG} into your preface. For normal quotations
please use the scientific standard to cite.
\newline
{ \tiny \itshape [derived from myCsrf (= 'mind your Scholar Research Framework') 
\copyright K. Reincke CC BY 3.0  http://mycsrf.fodina.de/)] }}} \author{
Karsten Reincke\thanks{Deutsche Telekom AG, Products \& Innovation, 
T-Online-Allee 1, 64295 Darmstadt}
}

\maketitle
%%-- end(titlepage)


\normalsize

\section*{Disclaimer}
% Telekom osCompendium 'for beeing included' snippet template
%
% (c) Karsten Reincke, Deutsche Telekom AG, Darmstadt 2011
%
% This LaTeX-File is licensed under the Creative Commons Attribution-ShareAlike
% 3.0 Germany License (http://creativecommons.org/licenses/by-sa/3.0/de/): Feel
% free 'to share (to copy, distribute and transmit)' or 'to remix (to adapt)'
% it, if you '... distribute the resulting work under the same or similar
% license to this one' and if you respect how 'you must attribute the work in
% the manner specified by the author ...':
%
% In an internet based reuse please link the reused parts to www.telekom.com and
% mention the original authors and Deutsche Telekom AG in a suitable manner. In
% a paper-like reuse please insert a short hint to www.telekom.com and to the
% original authors and Deutsche Telekom AG into your preface. For normal
% quotations please use the scientific standard to cite.
%
% [ File structure derived from 'mind your Scholar Research Framework' 
%   mycsrf (c) K. Reincke CC BY 3.0  http://mycsrf.fodina.de/ ]

%

%%% \chapter*{Disclaimer} %%%

This book shall be thoroughly developed -- together with the open source
community. Finally, it shall deliver reliable information with respect to the
rule that the swarm knows more than the single fish.

But nevertheless it can not offer more than the opinion(s) of its authors and
contributors. It is only one voice of chorus discussing the topic of open source
licenses. For protecting the authors and contributors from charges and claims of
idemnification we adopt the lightly modified GPL3 disclaimer:

THERE IS NO WARRANTY FOR THE OSLiC, TO THE EXTENT PERMITTED BY APPLICABLE LAW.
THE COPYRIGHT HOLDERS AND/OR OTHER PARTIES PROVIDE THE TEXT “AS IS” WITHOUT
WARRANTY OF ANY KIND, EITHER EXPRESSED OR IMPLIED, INCLUDING, BUT NOT LIMITED
TO, THE IMPLIED WARRANTIES OF MERCHANTABILITY AND FITNESS FOR A PARTICULAR
PURPOSE. THE ENTIRE RISK AS TO THE QUALITY AND PERFORMANCE OF THE OSLiC IS
WITH YOU. SHOULD THE OSLiC PROVE DEFECTIVE, YOU ASSUME THE COST OF ALL
NECESSARY SERVICING, REPAIR OR CORRECTION.

IN NO EVENT UNLESS REQUIRED BY APPLICABLE LAW OR AGREED TO IN WRITING WILL ANY
COPYRIGHT HOLDER, OR ANY OTHER PARTY WHO MODIFIES AND/OR CONVEYS THE OSLiC AS
PERMITTED ABOVE, BE LIABLE TO YOU FOR DAMAGES, INCLUDING ANY GENERAL, SPECIAL,
INCIDENTAL OR CONSEQUENTIAL DAMAGES ARISING OUT OF THE USE OR INABILITY TO USE
THE PROGRAM (INCLUDING BUT NOT LIMITED TO LOSS OF DATA OR DATA BEING RENDERED
INACCURATE OR LOSSES SUSTAINED BY YOU OR THIRD PARTIES OR A FAILURE OF THE
PROGRAM TO OPERATE WITH ANY OTHER PROGRAMS), EVEN IF SUCH HOLDER OR OTHER PARTY
HAS BEEN ADVISED OF THE POSSIBILITY OF SUCH DAMAGES.

%%%%%%%%%%%%

% Telekom osCompendium 'for being included' snippet template
%
% (c) Karsten Reincke, Deutsche Telekom AG, Darmstadt 2011
%
% This LaTeX-File is licensed under the Creative Commons Attribution-ShareAlike
% 3.0 Germany License (http://creativecommons.org/licenses/by-sa/3.0/de/): Feel
% free 'to share (to copy, distribute and transmit)' or 'to remix (to adapt)'
% it, if you '... distribute the resulting work under the same or similar
% license to this one' and if you respect how 'you must attribute the work in
% the manner specified by the author ...':
%
% In an internet based reuse please link the reused parts to www.telekom.com and
% mention the original authors and Deutsche Telekom AG in a suitable manner. In
% a paper-like reuse please insert a short hint to www.telekom.com and to the
% original authors and Deutsche Telekom AG into your preface. For normal
% quotations please use the scientific standard to cite.
%
% [ Framework derived from 'mind your Scholar Research Framework' 
%   mycsrf (c) K. Reincke 2012 CC BY 3.0  http://mycsrf.fodina.de/ ]
%


%% use all entries of the bibliography
%\nocite{*}

\section{Excursion: Reverse Engineering and Open Source}

Beyond any doubt, the LGPL mentions \enquote{reverse engineering}
literally\footnote{For the LGPL-v2 \cite[cf.][\nopage wp.
§6]{Lgpl21OsiLicense1999a}; for the LGPL-v3 \cite[cf.][\nopage wp.
§4]{Lgpl30OsiLicense2007a} } for indicating that reverse engineering in any
sense must be allowed to use and distribute LGPL software compliantly:

\begin{quote}\noindent\emph{\enquote{[\ldots] you may [\ldots] distribute a work
(containing portions of the Library) under terms of your choice, provided that
the terms permit [\ldots] \emph{reverse engineering} [\ldots]}
\footnote{\cite[cf.][\nopage wp, §6]{Lgpl21OsiLicense1999a}. The LGPL-v2 uses
the capitalized word \enquote{Library} for denoting a library which
\enquote{[\ldots] has been distributed under (the) terms} of the LGPL-v2.
(\cite[cf.][\nopage wp, §0]{Lgpl21OsiLicense1999a}). Inside of our LGPL
chapter(s) we follow this interpretation. } }
\end{quote}

There are three strategies for dealing with such provisions: one can try to
fully honor its meaning, one can mitigate its meaning, or one can avoid to
discuss this requirement altogether:

A first group of well known open source experts take the sentence of the LGPL-v2
as a strict rule which requires that one has to allow reverse engineering of the
whole software product if one embeds any LGPL licensed component into that
product\footnote{For example, a very trustworthy German expert states that the
LGPL-2.1 generally requires that a distributor of a program which accesses a
LGPL-2.1 licensed library, must grant his customer also the right to modify the
accessing program and hence also the right to execute reverse engineering.
Literally the German text says:
\begin{quote}\enquote{Ziffer 6 LGPLv2.1 knüpft die Erlaubnis, das zugreifende
Programm unter beliebigen Lizenzbestimmungen verbreiten zu drüfen, an eine Reihe
von Verpflichtungen, die in der Praxis oft übersehen werden: Zunächst muss dem
Kunden, dem die Software geliefert wird, die Veränderung des zugreifenden
Programms gestattet werden und zu diesem Zweck auch ein Reverse Engineering zur
Fehlerbehebung. Dies dürfte alle Formen des Debugging und das Dekompilieren des
zugreifenden Programms umfassen.} (\cite[cf.][81]{JaeMet2011a}).\end{quote}
At first glance, also \enquote{copyleft.org} -- the \enquote{[...] collaborative
project to create and disseminate useful information, tutorial material, and new
policy ideas regarding all forms of copyleft licensing} (\cite[cf.][\nopage
wp.]{CopyLeftOrg2014a}) -- could be taken as another witness for such an
attitude of strict reading: Some of its contributors elucidate in a chapter
dealing with \enquote{special topics in compliance} that \enquote{the license of
the whole work must [sic!] permit \enquote{reverse engineering for debugging
such modifications} to the library} and that one therefore \enquote{ should take
care that the EULA used for the Application does not contradict this
permission}(\cite[cf.][86]{KuhSebGin2014a}}.

A second group of well known and knowledgeable open source experts signify that
the LGPL-v2 indeed literally contains a strict rule, but that this rule actually
is not meant as it sounds: For example, two of these explain that \enquote{these
requirements on the licensed combination require that the license chosen not
prohibit the customer’s modification and reverse engineering for debugging these
modifications in the work as a whole}. But then they directly add the
limitation, that \enquote{in practice, enforcement history suggests, it means
that the license terms chosen may not prohibit modification and reverse
engineering for debugging of modification in the LGPL’d code included in the
combination}\footnote{\cite[cf.][\nopage wp., chapter LGPLv2.1, section
6]{MogCho2014a}. Such a mitigation can also be found in the tutorial of
copyleft.org: After they have summarized the LGPL-v2 sentence as a strict rule,
they directly continue, that one \enquote{[\ldots] must refrain from license
terms on works based on the licensed work that prohibit replacement of the
licensed components of the larger non-LGPL'd work, or prohibit decompilation or
reverse engineering in order to enhance or fix bugs in the LGPL'd components}
(\cite[cf.][86]{KuhSebGin2014a}). This added specification indicates, that one
only has to facilitate the modification of the library and that reverse
engineering can be ignored as long as there are other ways to improve the
embedded library.}.

Finally, a third group of experts prefers not to discuss the problem of reverse
engineering, although this technique is literally mentioned in the license and
although they want explain how to use GPL/LGPL licensed software
complaintly\footnote{An article of Terry J. Ilardi might be taken as a first
witness of this third strategy: he profoundly explains the essence of the LGPL,
he especially discusses §6, and he delivers applicable rules like \enquote{DO
NOT statically link to LGPL [\ldots] code if you wish to keep your program
proprietary}. But he does not discuss \emph{reverse engineering}
(\cite[cf.][5f]{Ilardi2010a}). Similarily argues Rosen
(\cite[cf.][121ff]{Rosen2005a}). And -- despite their comments on reverse
engineering in the specific chapter \emph{special topics in compliance} -- the
copyleft.org document can also be taken as an instance of this attitude:
Although its' authors recommend to \enquote{study chapter 10 carefully} for
establishing an adequate \enquote{compliance with LGPLv2.1}
(\cite[cf.][86]{KuhSebGin2014a}), this chapter 10 -- dedicated to the meaning of
the \enquote{Lesser GPL} -- does not deal with reverse engineering, although it
discusses the §6 of the LGPLv2.1 in depth (\cite[cf.][56ff, esp.
60f]{KuhSebGin2014a}).}.

This situation must bother companies and people who want to use open source
software compliantly and who therefore are looking for guidance. Particularly it
disturbs those who want to protect their business relevant software. At the end,
they might consider that this sentence is not consistently understood by the
open source community itself. And -- as far as we know -- at least some of these
companies preventively prohibit their developers to embed LGPL licensed
components into programs which contain business relevant techniques.
Unfortunately, this consequence does not only obstruct access to a large set of
well written free software, but it is scarcely possible to obey such an
interdiction consequently: The glibc, which enables the programms to talk with
the kernel of the GNU/Linux system\footnote{cf.
http://www.gnu.org/software/libc/}, is licensed under the LGPL\footnote{cf.
http://en.wikipedia.org/wiki/GNU\_C\_Library}. And hence, this library is
indirectly linked to or combined with any program running on the GNU/Linux
system. So, if the LGPL-v2 indeed required, that reverse engineering of every
program must be allowed, which contains any LGPL license library, then every
GNU/Linux user would be allowed to examine every GNU/Linux progam by
\emph{reverse engineering}, simply, because finally every GNU/Linux program is
linked to the glibc\footnote{This conclusion might surprise. But it is inferred
with exactly the same arguments as the conclusion, that without a licence
offering a weaker copyleft every program would have been licensed under the GPL.
The copyleft.org document explains this argumentation in great detail
(\cite[cf.][56f]{KuhSebGin2014a}).}. In other words: if the LGPL indeed required
the permission of reverse engineering, then every GNU/Linux program may be
reverse engineered.

But an exhaustive reading of the LGPL-v2 delivers a strong indicator for
another, more 'liquid' understanding of the LGPL: The preamble explains the
reason for offering another weaker license beside the GPL. It says that
\enquote{[\ldots] on rare occasions, there may be a special need to encourage
the widest possible use of a certain library, so that it becomes a de-facto
standard} and that therefore it could be strategicly necessary to \enquote{allow
[\ldots] non-free programs [\ldots] to use the library} without enforcing that
these programs become free software too\footcite[cf.][\nopage wp,
§preamble]{Lgpl21OsiLicense1999a}.

So, if the LGPL had indeed determined that every program linked or combined to
any LGPL library may be reverse engineered, then the LGPL would have an effect
contrary to its own intention. It would have introduced something like
\emph{'security by obscurity'}: First, the LGPL allows to protect the internals
of your own work against investigation because the code of the non-free programm
using the library does not necessarily have to be published as well\footnote{The
weak copyleft has been introduced for encouring the widest possible use of the
library.}. But in the end the user would also be allowed to reverse engineer the
received binary -- and hence would nevertheless be able discover all
internals\footnote{It would only cost a little more effort - as security by
obscurity indicates.}. By this means, the LGPL-v2 would have undermined its' own
raison d'$\grave{e}$tre introduced by its' inventors: under such circumstances
there probably would have been less hope that any LGPL library could have become
a defacto standard.

We know that the inventors of the GNU licenses and GNU software are very
sophisticated experts. They never would have published such an inconsistent
document. So, this dissent read in(to) the document is a strong indicator that
there must be a better way to understand the license. And thus, it is up to us,
the followers, to explicate a more adequate interpretation. Of course, such an
interpretation must be grounded on the written text. We, the scholars, are not
allowed to add our own wishes. We must read the license very strictly. We have
to deduce 'understandings' only by matching the interpretations explicitly,
strictly, and reasonably back to the license text itself.


NEU FORMULIEREN: start

Encouraged by the indication that a better understanding of the license may
exist and contrary to the other strategies, we are going to prove that, in
fact, none of the open source licenses\footnote{being discussed in the OSLiC} in
general require to allow reverse engineering of software containing a component
licensed under that open source license. In particular, we will prove, that even
the LGPL does not claim this permission generally: We want to explain, why the
LGPL only requires to permit reverse engineering if and only if the LGPL
licensed component is embedded into a statically linked and distributed piece of
software. Moreover, we want to show that in all other cases the LGPL allows 
to distribute packages without granting permission to reverse engineer the
software which uses the LGPL licensed library\footnote{By the way, our analysis
should also provide proof that the LGPL is not something like a 'poisoned'
license containing \enquote{an imprenetrable maze of technology babble} which
\enquote{[\ldots] should not be in a general-purpose software license}
(\cite[cf.][124]{Rosen2005a}). The challenge of the today's descendants is to
understand the former inventors of the GNU licenses and their way to think about
computing - including all the hassle the computing language C might provoke.}.
We hope that our analysis, grounded on the license text itself, will support
companies and people to compliantly use open source software more often and with
less scruples.

NEU FORMULIEREN: ende

Hence, let us prove our position 'bottom up'. Let us firstly show that it is
true for the LGPL-v2 -- by explicating the license text lingually, then
logically, and finally empirically, before we infer the correct understanding.
Then let us show that it is also true for the LGPL-v3. And in the end let us
show that it is true for all other licenses, analysed by the OSLiC.

\subsection{Reverse Engineering in the LGPL-v2}
The LGPL-v2.1 contains one sentence which literally refers to the issues of
\emph{reverse engineering}:

\begin{quote}\noindent\emph{\enquote{[\ldots] you may [\ldots] combine or link a
\enquote{work that uses the Library} with the Library to produce a work
containing portions of the Library, and distribute that work under terms of your
choice, provided that the terms permit modification of the work for the
customer's own use and \emph{reverse engineering} for debugging such
modifications.}\footnote{\cite[cf.][\nopage wp. §6. ]{Lgpl21OsiLicense1999a}.
The first ellipse in this citation -- notated by the string '[\ldots]' -- refers
to the phrase \enquote{As an exception to the Sections above,}, the second to
the phrase \enquote{also}. These words together want to indicate, that the LGPL
offers its §6-way-of-distribution as an exception of its default way. The kind
of way is not affected by this hint. Thus, we feel free to erase this contextual
link.}}\end{quote}

Hereinafter, we will sometimes denote these lines by
the word \emph{LGPL2-RefEng-Sentence}.

\subsubsection{Linguistical Clarification}

For fulfilling our rule, to read the text strictly and deduce our
interpretations reasonably, let us firstly only highlight the syntactical
conjunctions for simplifying the understanding:

\begin{quote}\noindent\emph{\enquote{[\ldots] you may [\ldots] combine
\textbf{or} link a \enquote{work that uses the Library} with the Library to
produce a work containing portions of the Library \textbf{and} distribute that
work under terms of your choice, \textbf{provided that} the terms permit
modification of the work for the customer's own use \textbf{and} \emph{reverse
engineering} for debugging such modifications.}\footnote{\cite[cf.][\nopage wp.
emphasis KR.]{Lgpl21OsiLicense1999a}.}}
\end{quote}

It is evident that the conjunction \emph{'provided that'} is splitting the
sentence into two parts: you are allowed to do something \emph{provided that} a
precondition is fulfilled. Additionally, both parts of the sentence --
the one before the conjunction \emph{'provided that'} and the part after it --
are syntactically condensed embedded phrases which also contain subordinated 
conjunctions and elliptical constructions\footnote{cf.
http://en.wikipedia.org/wiki/Ellipsis\_\%28linguistics\%29, wp.
}. These syntactical interconnections must be disbanded:

Let us firstly dissolve the syntactical compression before the conjunction
\emph{'provided that'}: It is established by using the two other conjunctions
\emph{and} and \emph{or} and introduced by the subordinating phrase \emph{you
may [\ldots]}. Unfortunately, from a formal point of view, one can read the
phrase \emph{you may (X or Y and Z)} as two different groupings: either as \emph{you
may ((X or Y) and Z)} or as \emph{you may (X or (Y and Z))}.

But, fortunately, we know from the semantic point of view that speaking about
\emph{\enquote{[\ldots] combining \textbf{or} linking [\ldots something] to
produce a work containing portions of the Library}} denotes two different
methods which both can \emph{join} the components \emph{\enquote{[\ldots] to
produce a work containing portions of the Library}}. So, let us -- only for a
moment\footnote{Later on we will re-insert th original phrase!} -- simply
replace the string \emph{\enquote{combine or link}} by the string
\emph{\enquote{*join}}\footnote{When the LGPL and the GPL were initially
defined, the C programming language was the predominant model of software
development. Knowing this method eases the understanding of these licenses.
Thus, it is not totally wrong to take this token *join also as a curtsey to the
C programming language.}. This reduces the syntactical structure of the sentence
back to the simple phrase \emph{you may (W and Z)} in which \emph{W} stands for
\emph{(X or Y)}.

Now, we see also that the phrase \emph{you may (W and Z)} itself is a
condensed version of the explicit phrase \emph{ (you may W) and (you may Z)}.

Finally we have to note, that the phrase before the conjunction \emph{'provided
that'} contains also a linguistic ellipsis\footnote{cf.
http://en.wikipedia.org/wiki/Ellipsis\_\%28linguistics\%29, wp.
}: It says that you may *join the components \enquote{to produce \textbf{a work
containing portions of the Library} \textbf{and} distribute \textbf{that work}
under terms of your choice}. With respect to the English grammar, we may
conclude that the second term \emph{that work} refers back to the previously
introduced specification of \emph{a work containing portions of the Library}: if
a complete phrase has just been introduced explicitly, then the English language
allows to reduce its' next occurence syntactically while its' complete meaning
is retained. Hence, conversely, we are allowed to unfold the reduced form to
restore the complete phrase.

So -- overall -- we may understand the phrase before the conjunction
\emph{'provided that'} as a phrase with the structure \emph{(you may W) and (you
may Z')}:

\begin{quote}\noindent\emph{\textbf{((}you may [\ldots] \emph{*join} a
\enquote{work that uses the Library} with the Library to produce a work
containing portions of the Library\textbf{) and (}you may [\ldots] distribute
that work containing portions of the Library under terms of your
choice\textbf{))}} \textbf{provided that} [\ldots]\end{quote}

Theoretically, a reader of the OSLiC could reject our first dissolution of the
LGPL2-RefEng-Sentence. But for reasonably denying our interpretation he has to
deliver other resolutions of the lingustic elliptical subphrases or other
dissolvations of the conjunctions. Fortunately, it seems to be evident that such
attempts must violate the English grammar.

Let us now dissolve the part after the conjunction \emph{'provided that'}:
With respect to the subordinated conjunction \emph{'and'}, the subphrase
\emph{the terms permit} syntactically refers to both, the \emph{modifcation} and
the \emph{reverse engineering}: An embedded conjunction \emph{'and'} allows to
use a more stylish grammatical compaction. So, it should be clear, that saying

\begin{quote}\noindent\textbf{provided that} \emph{the terms permit modification
of the work for the customer's own use \emph{\textbf{and}} reverse engineering
for debugging such modifications}\end{quote}

means

\begin{quote}\noindent\textbf{provided that} \emph{the terms permit
\textbf{(} modification of the work for the customer's own use \emph{\textbf{and}}
reverse engineering for debugging such modifications\textbf{)}}\end{quote}

and is totally equivalent to the sentence 

\begin{quote}\noindent[\ldots] \textbf{provided that} \emph{\textbf{((}the terms
permit modification of the work for the customer's own use\textbf{)}
\emph{\textbf{and}} \textbf{(}the terms permit reverse engineering for debugging
such modifications\textbf{))}}.
\end{quote}

We believe that there is no other possibility to understand this part of the 
LGPL2-RefEng-Sentence with respect to the rules of the English language.
Nevertheless, this is the next point where an OSLiC reader may formally disagree
with us. But if he really wants to object our dissolution, he must deliver
another valid interpreation of the scope of the conjunction \emph{and} or he
must deliver another resolutions of the linguistic ellipsis. But we reckon, that
one can not reasonably argue for such alternatives.

Finally, there are other deeply embedded ellipses, which need to be resolved
as well:

\begin{enumerate}
  \item  In the part before the splitting conjunction \emph{'provided that'} we
  already had to expand the abridging \emph{'that work'} by its intended
  explicated version \emph{'that work containing portions of the Library'}.  In
  the part after the splitting conjunction the first subphrase also contains the
  term \emph{'the work'}. Formally, this term can either refer to \emph{'the
  work that uses the library'} as one of the components which are joined, or it
  can refer to \emph{'the work containing portions of the Library'} as the
  result of joining the components. We decide to constantly dissolve the
  elliptic abridgement by the phrase \emph{'the work containing portions of the
  Library'}.
  \item The first clause of the part after the splitting conjunction
  \emph{'provided that'} talks about the purpose of \enquote{permitting
  modification of the work} which we just had to unfold to the phrase
  \emph{'permitting modification of the work containing portions of the
  Library'}. The second clause talks about the purpose of \enquote{permitting
  reverse engineering}: it shall support the \enquote{debugging [of] such
  modifications}. The pronoun \emph{'such'} indicates that the word
  \emph{'modifications'} refers back to the just unfolded phrase
  \emph{modification of the work containing portions of the Library}. So, even
  the second sentence has to be expanded to that explicit phrase.
  \item Finally and only for being complete, we also have to unfold the clause
  \enquote{the terms} to the form which is predetermined by the first referred
  instance \enquote{the terms of your choice}
\end{enumerate}

So -- overall -- we are allowed to rewrite the first sentence of the LGPL-v2, §6
in the following form, namely without having changed its
meaning\footnote{Recollect that '*join' still stands for 'combine or link'.}:

\begin{verbatim}

( ( you may 
       *join a work that uses the Library with the Library
        to produce a work containing portions of the Library )
  AND 
  ( you may 
        distribute that work containing portions of the Library
        under terms of your choice 
) )
PROVIDED THAT
( ( the terms of your choice permit 
        modification of the work containing portions of 
        the Library for the customer's own use )
  AND
  ( the terms of your choice permit
        reverse engineering for debugging modifications 
        of the work containing portions of the Library   
) )
\end{verbatim}

At this point we must recommend all our readers to verify that this 'structured
explicated presentation' does exactly mean the same as the intially quoted
LGPL2-RefEng-Sentence. We are now going to discuss some of its' logical aspects
by some formal transformations. For accepting our following operations and
linking the results back to the original LGPL2-RefEng-Sentence, it is very
helpful to know that one already has accepted the equivalence of this
explicated form and the more condensed original version. For reviewing the
equivalence the reader should ask himself which of our rewritings are wrong, why
they are wrong and which alternatives can reasonably be offered for solving the
syntactical issues which disposed us to chose our solutions. Again, we ourselves
-- of course -- are profoundly convinced that both versions are completely
equivalent.

\subsubsection{Logical Clarification}

For simplifying our discussion let us now replace the meaningful terminal
phrases of our form by some logical variables:

\begin{description}
  \item[$\Gamma$] :- (you may *join a work that uses the Library with the
  Library to produce a work containing portions of the Library) 
  \item[$\Delta$] :- (you may distribute that work containing portions of the
  Library under terms of your choice)
  \item[$\Phi$] :- (the terms of your choice permit modification of the work 
  containing portions of the Library for the customer's own use)
  \item[$\Sigma$] :- (the terms of your choice permit reverse engineering for
  debugging modifications of the work containing portions of the Library)
  \item[$\Theta$] :- \emph{$\Gamma$ and $\Delta$}
  \item[$\Omega$] :- \emph{$\Phi$ and $\Sigma$}
\end{description}

Based on these definitions, we can syntactically reduce the
LGPL2-RefEng-Sentence to the formula \emph{$(\Gamma$ and $\Delta)$ provided that
$(\Phi$ and $\Sigma)$} or -- even shorter -- to \emph{$(\Theta$ provided that
$\Omega)$}.

Now, we have to clarify the meaning of the conjunction \emph{'provided that'}:

Obviously, \emph{provided that} means something like \emph{under the
precondition that}. So, one might try to take this conjunction as another more
stylish version of the more common \emph{if(\ldots)then(\ldots)}-formula, also
known as logical implication\footnote{Actually the logical implication and the
computational if-then-construct are not equivalent. Fortunately, we later on can
show, that in the context of this discussion the difference can be ignored.}.
Thus, we have to consider the process of sequencing the linguistic form into a
logical formula: if we take the conjunction \emph{provided that} as another form
of the logical implication, it is not obviously clear, which part of the 
linguistic sentence must become the premise, and which the conclusion. Does
\emph{($\Theta$ provided that $\Omega$)} mean \emph{(if $\Theta$ then $\Omega$)}
or \emph{(if $\Omega$ then $\Theta$)}?

Obviously, \emph{provided that} wants to establish something like a
precondition. So, one might conclude that \emph{$(\Theta$ provided that
$\Omega)$} means \emph{(if $\Omega$ then $\Theta)$} or -- more logically notated
-- \emph{$((\Phi$ $\wedge$ $\Sigma)$ $\rightarrow$ $(\Gamma$ $\wedge$
$\Delta))$}. If this interpretation is adequate, it of course must fulfill the
intended purpose of the corresponding LGPL-v2-section: there, the LGPL wants to
regulate the distribution of works containing portions of LGPL libraries.

For facilitating the understanding of our argumentation, let us first check
whether our interpretation fits the purpose of the LGPL. Thus, let us unfold the
slightly reduced version \emph{$(\Sigma$ $\rightarrow$ $\Delta)$} back to the
corresponding verbal form:

\begin{quote}\noindent\emph{\textbf{if (} [\ldots] the terms permit reverse
engineering for debugging modifications of the work containing portions of the
Library, \textbf{) then (} [\ldots] you may distribute that work containing
portions of the Library under the terms of your choice.\textbf{)}}\end{quote}

Now we can better see the problem: An implication as a whole is false only if
the premise is true and the conclusion is false. In all other cases it is true.
Especially, it is true, if the premise is false: in this case the truth value of
the conclusion does not matter in any sense. Thus, if we take this implication
as a rule, which shall determine our behaviour, then this implication only
supports us, if we already have decided to permit reverse engineering. In this
case the rule tells us that we are allowed to distribute the work containing
portions of the Library. But from the converse decision, that we will not permit
reverse engineering follows nothing - because a false premise does not influence
the truth value of the conclusion. Especially, it does not follow that we may
not distribute the work containing portions of the Library. So -- from the
viewpoint of the formal logic -- this translation of the original conjunction 
\emph{'provided that'} says, that if the terms of your own license do not permit
reverse engineering for debugging modifications of the work containing portions
of the Library\footnote{The premise is false.}, then you may or may not
distribute that work containing portions of the Library under the terms of your
choice\footnote{The truth value of the conlusion is undetermined by the rule.}.
Hence, we must state that this interpretation does not fulfill the purpose of
the LGPL-V2: if reverse engineering is not allowed, the distribution of the work
containing portions of the Library is not regulated. We have to conclude, that
this sequencing the LGPL2-RefEng-Sentence as a logical implication is simply
wrong.

But we deduced this consequence from a slighty reduced form of our
LGPL2-RefEng-Sentence. Thus, we still have to ask, whether we have to derive
this conclusion also on the base of the completely unfolded formula
\emph{$((\Phi$ $\wedge$ $\Sigma)$ $\rightarrow$ $(\Gamma$ $\wedge$ $\Delta))$}?
The answer is yes: the premise \emph{$((\Phi$ $\wedge$ $\Sigma)$} contains a
logical conjunction. So the truth value of the whole premise depends on the
truth value of each of its terminal statements, particularly on that of the
statement $\Sigma$: If we decide not to permit reverse engineering, then the
premise as whole is false, regardless we forbid or allow modifications. And
hence the premise does not influence the truth value of the conclusion. So,
there is no way, to conclude that we have to allow or that we do not have to
allow reverse engineering. Hence we can transfer our result, deduced for the
slightly reduced formula to the unfolded complete formula.

Let us now test the other possibility. Let us ask, whether \emph{()$\Theta$
provided that $\Omega$)} means \emph{(if $\Theta$ then $\Omega$)} or -- more
logically notated -- \emph{$((\Gamma$ $\wedge$ $\Delta)$ $\rightarrow$ $(\Phi$
$\wedge$ $\Sigma))$}. If we again for a moment focus on the reduced version
\emph{$(\Delta$ $\rightarrow$ $\Sigma)$} and dissolve our replacements, then we
get back the rule:

\begin{quote}\noindent\emph{\textbf{if (} [\ldots] you may distribute that work
containing portions of the Library under the terms of your choice, \textbf{)
then (} [\ldots] the terms permit reverse engineering for debugging
modifications of the work containing portions of the Library.
\textbf{)}}\end{quote}

Now we can see, that this version perfectly regulates the distribution of works
containing portions of LGPL libraries: If we are allowed to do so or -- in other
words: if we are compliantly distributing works containing portions of LGPL
libraries\footnote{The premise is true.}, then we have to permit reverse
engineering\footnote{The conclusion must be true, too!}. This follows from
applying \emph{Modus Ponens} to the implication\footnote{A true premise evokes a
true conclusion based on the given truth of the implication / rule itself.}. And
if we do not permit reverse engineering\footnote{The conclusion is false.}, then
we are not allowed to distribute works containing portions of LGPL
libraries\footnote{The premise must be false, too!}. This follows from applying
\emph{Modus Tollens} to the implication\footnote{A false conclusion evokes a
false premise based on the given truth of the implication / rule itself.}

Based on this clarification, we can reasonably replace the more stylish
conjunction \emph{'provided that'} by its more known equivalent
\emph{'implication'} which we do not notate as traditional
\emph{if-then}-construction\footnote{Here we can also see, that the difference
between the if-then-command as part of a procedural computer language and the
logical implication does not influence our results: In the context of a
procedural if-then-command the truth of the premise triggers the execution of
the conclusion. In our discussion, this aspect is totally covered by the Modus
Ponens derivation of the logical interpretation. And the Modus Tollens
derivation of the logical interpretation on the other side does not play any
role in a procedural if-then-command. So, it was the right decision to
understand the LGPL2-RefEng-Sentence logically and not as procedual command.}, but
as logical implication \emph{$\rightarrow$}. So, we now get

\begin{description}
  \item[\#]  $\Theta$ provided that $\Omega$
  \item[$\equiv$] $\Theta$ $\rightarrow$ $\Omega$
  \item[$\equiv$] ($\Phi$ $\wedge$ $\Sigma$) $\rightarrow$ ($\Gamma$ $\wedge$
  $\Delta$)
  \item[$\equiv$]
\begin{alltt}   
  ( ( [\(\Phi\)] you may 
       *join a work that uses the Library with the Library
       to produce a work containing portions of the Library )
  \(\wedge\)
  ( [\(\Sigma\)] you may 
        distribute that work containing portions of the 
        Library under terms of your choice 
) )
\(\rightarrow\)
( ( [\(\Gamma\)] the terms of your choice permit 
        modification of the work containing portions of 
        the Library for the customer's own use )
  \(\wedge\)
  ( [\(\Delta\)] the terms of your choice permit
        reverse engineering for debugging modifications 
        of the work containing portions of the Library   
) )
\end{alltt}
\end{description}

\subsubsection{Empirical Clarification}

Now, we can simplify this formula once more by regarding some empirical facts
and explicating some underlying understandings:

The first sentence $\Phi$ explains that the \emph{work that uses the Library}
and the used \emph{Library} itself together are joined and thereby tansformed
into a \emph{work containing portions of the Library}. So, formally, one might
ask, whether this newly generated \emph{work containing portions of the Library}
also still \emph{uses the Library}?

Unfortunately, it is empirically possible, that a process for combining the two
components (a) copies all original portions of the library into a something like
a 'dead end section' of the program where they are never excuted, and (b)
replaces all original portions of the library by functionally equivalent
portions of another library. Thus, the resulting \emph{work containing portions
of the Library} indeed still contains portions of the Library, although it does
not use it any longer. And hence, we are not allowed to say, that every work
containing portions of a library also uses the library\footnote{\ldots even if
we think that this is a really silly way to organize the joining process!}.

But, fortunately, the normal computational process of \emph{combining and
linking a work that uses the Library with the Library to produce a work
containing portions of the Library} inherently preserves the utilization of the
joined library: It is the general purpose of a software library to offer
functions and/or data (structures) for really being used by applications. And
vice versa, software developers want to use readily prepared libraries (or
classes or anything else) for simplifying their own work. They chose a library
because they know, that the standard compiling and linking process guarantees,
that indeed the chosen library is used (and not secretly substituted by a
mysterious 'equivalent'). Based on this empirical praxis we are therefore
allowed to say that a \emph{work containing portions of the Library} which has
been \textbf{built by the normal development processes} of combining, compiling,
and linking source and object files, indeed also uses the intended library.

So, let us now consider a specific correlation between the first sentence
$\Phi$ and the second sentence $\Sigma$:

It seems to be evident, that we must already have done $\Phi$, in other words:
that we must already have \emph{*joined -- respectively: combined or linked -- a
work that uses the Library with the Library to produce a work containing
portions of the Library}, if we are going to compliantly \emph{distribute that
work containing portions of the Library under terms of your choice}. Or briefly
spoken: It seems to be conclusive that $\Sigma$ \emph{\textbf{empirically}
implies} $\Phi$\footnote{but not vice versa.}.

But is this conclusion correct? Let us check this statement by assuming the
opposite: If the contrary was true, then there had to exist a \emph{work
containing portions of the Library} which had been gained without linking or
combining the work and the Library in any sense. But from the inference above we
already know that \emph{works containing portions of the Library}, which have
been produced by the standard computational processes of \emph{combining and
linking a work that uses the Library with the Library}, indeed also \emph{use
the Library}. Thus, it would be self-contradictory to talk about a \emph{work
containing portions of the Library}, which was produced by the standard
combining and linking processes, and similarily to state, that exactly this work
is not combined with the library in any sense. And from a proof by contradiction
we may conclude to the logical opposite:

With respect to the meaning of \emph{being standardly combined or linked with},
we may say, that
\begin{itemize}
  \item it is necessarily true that a computional work, which is standardly
  produced on the base of \emph{a work that uses the Library} and \emph{the
  Library} and which therefore literally contains more or less
  \emph{portions of a library}, indeed uses the \emph{the Library} and \emph{is}
  therefore \emph{combined with the library}.
  \item  $\Sigma$\footnote{distributing \emph{a work that uses the Library and
  contains portions of a library}} empirically implies $\Phi$\footnote{A work
  that uses the Library has been *joined with the Library to produce a work
  containing portions of the Library} (in the standardized world of software
  development), because $\Phi$ must ever have been executed when $\Sigma$ is
  going to be realized.
\end{itemize}

Thus, we can now reduce the LGPL2-RefEng-Sentence to its' real core:

\begin{quote}
\begin{alltt}   
(   [\(\Sigma\)] you may
        distribute (a) work containing portions of the 
        Library\(\footnote{which previously has been prepared for being distributed by standardly combining and
linking the work that uses the Library with the Library in a way that this prepared work indeed
also uses the Library}\) under terms of your choice )   
\(\rightarrow\)
( ( [\(\Gamma\)] the terms of your choice permit 
        modification of the work containing portions of 
        the Library for the customer's own use )
  \(\wedge\)
  ( [\(\Delta\)] the terms of your choice permit
        reverse engineering for debugging modifications 
        of the work containing portions of the Library   
) )
\end{alltt}
\end{quote}

This is indeed the essence of the LGPL2-RefEng-Sentence. It logically explains
us that we have to \emph{allow reverse engineering} and modification of a
\emph{work containing portions of the Library} if we distribute it (Modus
Ponens) and that we are \emph{not allowed to distribute a work containing
portions of the Library}, if we do \emph{not allow} its modification or
\emph{reverse engineering} (Modus Tollens).

Thus, for applying this rule correctly, we now only must know whether a work
indeed contains portions of the Library or not.

\subsubsection{Final Conclusion}

Unfortunately, there are more than one software developing scenarios, which must
be considered for answering this question in detail. We see three general types
of developing computer software:

\begin{enumerate}
  \item You can produce software by using script languages. Source file s
  which contain script language commands are distributed and then directly
  executed by an interpreter without being transformed into another 'more'
  machine specific language.
  \item Or you can develop software by using languages which are designed for
  being compiled into a machine independent bytecode. Later on this independent
  bytecode is executed by a machine specific virtual machine.
  \item  Or you can write traditional software files. Sometimes, these files are
  remastered by a preprocessor before the real process starts. The traditional
  sources themselves or the output of the preprocessor is then compiled and
  linked down to machine specific binary code.
\end{enumerate}
  
You may take 'php' is an example for the first environment, 'Java' an example
for the second, and 'C/C++' an example for the third.

Fortunately, the nature of these environments simplifies the answer to the
question under which conditions the work using the Library contains portions of
the Library:

\paragraph{Manually copied code evokes copyleft effect:} 


Copying code from the sources of the Library into the the overarching work that
uses the Library is not the standard way of combining both components, neither
in the world pf script programming, nor in the world of byte code programming
nor in the world of programming machine specific code:

Normally, the work which uses the Library is joined to the intended Library by
an include statement, an input statement, an import statement, a package
statement, or anything else. These *join-statements are inserted into the code
of the work. They directly or indirectly denote the file(s) which serve(s) the
used functions, methods, classes, or data. It is an integrated feature of this
development step that these insertions of such *join-statements do not directly
augment the code of the work using the library by some code of the Library: all
the development processes are designed for offering an automatical expansion
during the standard compilation or execution. And this augmention occurs after a
development loop is terminated.

Nevertheless, developers can of course circumvent these standard methods for
using a Library. Technically, they can directly copy code of the Library into
their own work. These manually evoked extensions of the code will then later
on directly be compiled or executed. Thus, it is clear, that in this case the
work that uses the Library already contains portions of the Library, namely
before the normal *join-processes of the environment are executed.

Hence, if you are going to distribute files that contain literally copies of the
Library source code or which have been compiled (but still not linked) on the
base of such augmented files, then you have to allow reverse engineering.

But you have also to know, that the LGPL-v2 directly regulates this kind of
using the Library: It says, that \enquote{you may modify your copy or copies of
the Library or any portion of it [\ldots] provided that you also [\ldots] cause
the whole of the work to be licensed [\ldots]] under the terms of this
License}\footcite[cf.][\nopage wp.,§2, escpcially §2c]{Lgpl21OsiLicense1999a}.
Thus, in case of literally copying code from the Library into the work using the
Library, also the LGPL causes the copyleft effect: The code of the work using
the library has to be made accessible, too. 

So, overall, we might say, that copying code from an LGPL-v2 Library into a work
using that Library and distributing the result indeed requires to additionally
permit its reverse engineering -- but probably this permission is not very
important for the recipient, because he can directly get the code.

\paragraph{Scripts do not need reverse engineering}

Let us secondly consider the case of script programming. Computer programs
written in a script language are distributed as they have been developed. They
are not transformed into another kind of code\footnote{Java script is often
offered as compressed code. Roughly spoken, this means that at first all white
space signs have been replaced by blanks and then all rows of blanks have been
reduced to at most one single blank. So, even then, the code itself is directly
readable and understandable.}. The interpreter takes the script file itself and
directly executes it. Thus, there is no special technique of reverse engineering
for understanding these kind of software: you can directly read it if you know
the script language. 

So, again, we might conclude, that a script using a script Library perhaps
requires to permit its reverse engineering -- but probably this permission is
not very important for the recipient, because he can directly read the code.

\paragraph{Statically and dynamically combined byte code components}

Now, let us analyse Java as the prototype for languages which are compiled down
to machine independent byte code:

In Java, each class is compiled to an own class file. These class files have to
be stored somewhere in the classpath. A side from that, classes can also be
collected and distributed in form of packages which then can be used like
'traditional' Libraries. These packages must also be stored somewhere in the
classpath. A single class is made known to the work that want to use it, by an
import statement which contains the class name. A whole Java library is made
usable by integrating a package statement into the code. 

The code which follows these import- or package statements can then refer to the
definitions offered by the classes by using the (qualified) names of its member
variables or methods. So, from a very strict viewpoint, the code of the work
using the Library indeed contains portions of the library -- even if these
portions are only identifying names. The Java compilation process which
generates the byte code, preserves these denoting names. It does not replace the
referring names by the referred code. Only just at the end, when the java
virtual machine itself tries to execute the work using the Library, it collects
all necessary commands of all 'joined' classes.

So, one might tend to argue that answering the question whether a distributed
java byte code already contains portions of the used Library, depends on the
interpretation, whether a denoting identifier of a Library indeed is a portion
of the Library. We will discuss this case together with the corresponding
c/c++-Case. 

But there is still one specific Java case which has to be analysed separately.
As already mentioned, in Java you can also join your work that uses the Library
(by by using the denoting indentifiers) and the Library itself by building a new
package which then contains your work using the Library and the used Library
itself. One can say, that this package is quasi statically linked, because if
you distribute it, then you are distributing both parts together. And if you
distribute a work (package) containing portions of the Library, you have to
permit reverse engineering.

So, preliminarily we conclude, that with respect to Java programming you (a)
have to permit reverse engineering, if you distribute your work using the
Library and the Library itself as (statically linked) package that (b) in all
other cases your obligation to permit reverse engineering depends on the
interpretation whether used identifying names of a Library are indeed portions
of your own work using the Library.

Fortunately, we can reasonalby decide this issue soon.

\paragraph{Statically and dynamically linked binary components}

But let us previously analyse C/C++ development as the prototype of those
languages which are compiled down to machine specific code. 

Similar to Java, in C/C++ code a Library is made known to the work that want to
use it, by include statements. These include statements denote the header files
offered by and distributed with the Library. They contain the declarations of
those elements which the Library want to announce. Except some cases, the
corresponding definitions are delivered by the Library itself: the Library
contains the code, the header files its declarations. But, of course, even the 
declarations are elements of the Library. 

The C/C++ code which follows these include statements can refer to the
definitions offered by the Library by using the elements which are published by
the header files. So, again, from a very strict viewpoint, the code of the work
using the Library indeed contains portions of the library -- even if these
portions are only identifying names offered by the header files.

Beyond that conceptual relation, the C/C++ development process finally compiles
the work using the library down to an object file containing machine specific
code. Just as the Java compilation, also this process preserves the denoting
names. It does not replace the referring names by the referred code of the
Library. This C/C++ compilation process is (mostly) managed by a make file,
which is executed by the make command\footnote{Sometimes there additionally
exist a complete meta environment which generates such make files. The GNU build
system for example offers a complex set of configure scripts and make file
templates (cf. http://en.wikipedia.org/wiki/GNU\_build\_system, wp.).}. This
development system calls the compiler for each source file, makes known the
directories which contain the compiled target object files, and finally calls
the linker. The linker recursively scans the 'top' object file and its still
embedded identifiers, which refer code snippets of other object files and
libraries: So, only just at the end, the linker collects all necessary commands
of all 'joined' object files and Libraries and produces the really executable
work.

But, nevertheless, the linker can either be called as integrated component of
this process or it can seperatedly be called on another machine: In the first
case, the process generates a \emph{statically linked executable} which can be
distributed and indeed contains all necessary portions of all used Libraries. In
the second case, the process generates a \emph{dynamically linkable program}
which is distributed as set of still not executable object files and which are
linked on the executing machine just in the moment when the program shall be
executed\footnote{There is no difference whether you produces a statically
linked or dynamically linkable program or library.}.

Now, we have also to consider a little complication based on the nature of the a
C/C++ development process. In contrary to the Java development environment, a
C/C++ development process inherently uses a preprocessor engine.
This engine takes the header files delivered by the Library, verifies the
syntactically correct use of the Library and can indeed replaces some tokens of
the work using the Library by sourcecode from the Library. These snippets which
replaces strings in the source code of the work using the library are known as
inline functions or macros. They have been invented for those cases where
expanding the stack of commands by a real function is more expensive than
writing the commands of the function more than one time into the whole code. But
-- of course -- there is no technical limit. A header file could also contain
more than only some lines of the Library code. However that may be, in the C/C++
cevelopment process the compiled object files can indeed contain more than only
the referring names which denote portions of the Library: they can contain real
portions of the Library.
 
Thus -- again and similar to Java compilation --, we may nevertheless conclude,
that with respect to C/C++ programming you (a) have to permit reverse
engineering, if you distribute your work using the Library together with the
Library as a statically linked program and that (b) in all other cases your
obligation to permit reverse engineering depends on the interpretation whether
the used identifying names and inline functions being embedded into an object
file are indeed portions of your own work using the Library. 

So, it is time to answer the question wether indentifiers and inline-functions
being embedded into a work that uses the Library indeed are protions of the
Library and hence require, that we have to allow reverse engineering.

\paragraph{Object files and the embedded portions of the Library}

Of course, there is only one instance, that can definitely answer this question,
the LGPL-v2 itself. And --- fortunately -- the LGPL supports us in a very clear
way:

The LGPL simply specifies that \enquote{linking a \enquote{work that uses the
Library} with the Library creates an executable that is a derivative of the
Library (because it contains portions of the Library) [\ldots]} and that
\enquote{the executable is therefore covered by this
License}\footcite[cf.][\nopage wp. §5]{Lgpl21OsiLicense199a}. But an
\enquote{object file} -- for example that of the \enquote{work using the
Library} -- which still is not executable, because it is still not linked to the
Library and which \enquote{[\ldots] uses only numerical parameters, data
structure layouts and accessors, and small macros and small inline functions
(ten lines or less in length)} is pratically not covered by the license of the
Library because \enquote{[\ldots] the use of the object file is unrestricted
regardless of whether it is legally a derivative work}\footcite[cf.][\nopage wp.
§5]{Lgpl21OsiLicense199a}.

In other words: The answer of the LGPL to our question is this: (a) yes, such
object files containing names and snippets offered by the used Library, contain
indeed portions of the Library. But (b), these kind of 'little' portions being
embedded into the object file by the standard compilation processes do not evoke
any requirements, especially not the obligation to allow reverse engineering.


From a theoretical point of view, it is clear, that one has to count the maximum
of the lines which 

 From a practical point of view, one now could reply that this rule does not
work: for determining whether a compiled work that uses the Library contains
more than the limits of \enquote{small macros or small inline functions (ten
lines or less in length)} one has to study the header files. 



Now, we reached our target:
this interpretation can directly be applied to both open cases, to the case of
distributing Java byte code which only contains referring identifiers and to the
case of distribution C/C++ object code. And fortunately the limits given by the
LGPL-v2-§5 totally covers the normal use of C/C++ and Java libraries.

So we indeed finally may conclude that the LGPL-v2-§6 sentence

\begin{quote}\noindent\emph{\enquote{[\ldots] you may [\ldots] combine
\textbf{or} link a \enquote{work that uses the Library} with the Library to
produce a work containing portions of the Library \textbf{and} distribute that
work under terms of your choice, \textbf{provided that} the terms permit
modification of the work for the customer's own use \textbf{and} \emph{reverse
engineering} for debugging such modifications.}\footcite[cf.][\nopage wp.
emphasis KR.]{Lgpl21OsiLicense1999a}}
\end{quote}

means 'nothing more' than

\emph{
\begin{itemize}
  \item You are not required to allow reverse engineering if you compile the
  work using the Library as a discret (set of) dynamically linkable file(s) by
  using the standard compilation methods and if you distribute that produced
  (set of) object or byte code files before they are linked as an executable
  which indeed contains more than the tolerated size of portions of the Library.
  \item In all other cases, you are required to allow reverse engineering of a
  work using a Library:
  \begin{itemize}
    \item You must allow reverse engineering of a work using a Library if you
    distribute the work using the Library and the Library together as a
    statically linked program or as a collected package.
    \item You must allow reverse engineering if you manually copy code from the
    Library into the work using the Library and if you distribute the result.
    But because of the LGPL copyleft effect, which is evoked under these
    circumstances, you also have to distribute the source code of the work using
    the Library. Hence, the permission of reverse engineering will not be very
    relevant.
    \item scripts \ldots
    \item \ldots
  \end{itemize}
\end{itemize}
}

\subsubsection{Final Securing}

We have done a lot of work: At first we unfolded and dissolved stylisch
condensed formulations used in the original LGPL-v2-§6 sentence back to their
linguistically explicit form. At second we explicated the logical structure of
the sentence. At third we empirically carved out the real meaning of the
sentence. And finally we mapped the triggering part of that rule to a definitive
fact. Indeed, a lot of work for only one sentence.

So, it is a good step to verify that deduced result fits the spirit and the
goals of the LGPL-v2 perfectly.


\subsection{Reverse Engineering in the LGPL-v3}

Now we can shorten the way to understand the meaning of the LGPL-v3 concerning
\emph{reverse enigeering}. Also the LGPL-v3.0 contains only one paragraph
explicitly referring the \emph{reverse enigeering}:


\begin{quote}\emph{
\enquote{You may convey a Combined Work under terms of your choice that,
taken together, effectively \emph{do not restrict} modification of the portions
of the Library contained in the Combined Work and \emph{reverse engineering} for
debugging such modifications, if you also do each of the following
[\ldots]}\footcite[cf.][\nopage wp]{Lgpl30OsiLicense2007a}}
\end{quote}


\subsection{Reverse Engineering in the GPL}

\subsection{Reverse Engineering in the other Licenses}













%\bibliography{../../../bibfiles/oscResourcesEn}



\footnotesize
% Telekom osCompendium English Nomenclation Tokens Include Module 
%
% (c) Karsten Reincke, Deutsche Telekom AG, Darmstadt 2011
%
% This LaTeX-File is licensed under the Creative Commons Attribution-ShareAlike
% 3.0 Germany License (http://creativecommons.org/licenses/by-sa/3.0/de/): Feel
% free 'to share (to copy, distribute and transmit)' or 'to remix (to adapt)'
% it, if you '... distribute the resulting work under the same or similar
% license to this one' and if you respect how 'you must attribute the work in
% the manner specified by the author ...':
%
% In an internet based reuse please link the reused parts to www.telekom.com and
% mention the original authors and Deutsche Telekom AG in a suitable manner. In
% a paper-like reuse please insert a short hint to www.telekom.com and to the
% original authors and Deutsche Telekom AG into your preface. For normal
% quotations please use the scientific standard to cite.
%
% [ File structure derived from 'mind your Scholar Research Framework' 
%   mycsrf (c) K. Reincke CC BY 3.0  http://mycsrf.fodina.de/ ]


%\abbr[aaO]{a.a.O.}{am angegebenen Ort}
%\abbr[ds]{ds.}{kollektiv für ders., dies., \ldots}
\abbr[etseqq]{et seqq.}{and the following ones}
\abbr[id]{id.}{idem = latin for 'the same', be it a man, woman or a group\ldots}
\abbr[ibid]{ibid.}{ibidem = latin for 'at the same place'}
\abbr[ifross]{ifross}{Institut für Rechtsfragen der Freien und Open Source
Software}
\abbr[lc]{l.c.}{loco citato = latin for 'in the place cited'}
\abbr[np]{np.}{no page numbering}
\abbr[wp]{wp.}{webpage / webdocument without any internal (page)numbering}
\abbr[nst]{n.st.}{not stated}
\abbr[njear]{n.y.}{year not stated / no year}
\abbr[nlocation]{n.l.}{location not stated / no location}
\abbr[ub]{UB}{'Universitätsbibliothek' = library of university X}
\abbr[ulb]{ULB}{'Universitäts- \& Landesbibliothek' = library of university and state X}
\abbr[apl]{ApL}{Apache License}
\abbr[bsd]{BSD}{Berkeley Software Distrobution (License)}
\abbr[mit]{MIT}{Massachusetts Institute of Technology (License)}
\abbr[mspl]{Ms-PL}{Microsoft Public License}
\abbr[pgl]{PgL}{Postgres License}
\abbr[php]{PHP}{PHP (License)}
\abbr[epl]{EPL}{Eclipse Public License}
\abbr[eupl]{EUPL}{European Union Public License}
\abbr[lgpl]{LGPL}{GNU Lesser General Public License}
\abbr[mpl]{MPL}{Mozilla Public License}
\abbr[gpl]{GPL}{GNU General Public License}
\abbr[agpl]{AGPL}{GNU Affero General Public License}
\abbr[nabbr]{n.abbr.}{no abbreviation (known)}
% Telekom osCompendium English Nomenclation Tokens Include Module 
%
% (c) Karsten Reincke, Deutsche Telekom AG, Darmstadt 2011
%
% This LaTeX-File is licensed under the Creative Commons Attribution-ShareAlike
% 3.0 Germany License (http://creativecommons.org/licenses/by-sa/3.0/de/): Feel
% free 'to share (to copy, distribute and transmit)' or 'to remix (to adapt)'
% it, if you '... distribute the resulting work under the same or similar
% license to this one' and if you respect how 'you must attribute the work in
% the manner specified by the author ...':
%
% In an internet based reuse please link the reused parts to www.telekom.com and
% mention the original authors and Deutsche Telekom AG in a suitable manner. In
% a paper-like reuse please insert a short hint to www.telekom.com and to the
% original authors and Deutsche Telekom AG into your preface. For normal
% quotations please use the scientific standard to cite.
%
% [ Derived from 'mykeds Scholar Research Framework' 
%   mykeds-CSR-framework (c) K. Reincke CC BY 3.0  http://www.mykeds.net/ ]

%\abbr[]{[n.abbr.]}{ }
\abbr[zge]{ZGE / IPJ}{Zeitschrift für geistiges Eigentum [ISSN: 1867-237x]}
\abbr[itrb]{ITRB}{Der IT-Rechtsberater [ISSN: 1617-1527]}
\abbr[cri]{CRi}{Computer Law Review international [ISSN: 1610-7608]}
\abbr[btlj]{[n.abbr.]}{Berkeley Technology Law Journal}
\abbr[eclr]{E.C.L.R.}{European Competition Law Review}
\abbr[iesw]{[n.abbr.]}{IEEE Software [ISSN: 0740-7459]}
\abbr[cuitj]{[n.abbr.]}{Cutter IT Journal [ISSN: 1048-5600]}
\abbr[uoclr]{[n.abbr.]}{University of Chicago Law Review}
\abbr[uoilr]{[n.abbr.]}{University of Illinois Law Review}
\abbr[uoplr]{[n.abbr.]}{University of Pittsburgh Law Review}
\abbr[ddt]{DDT}{Drug Discovery Today [ISSN: 1359-6446]}
\abbr[rdm]{[n.abbr.]}{R\&D Management [ISSN: 1467-9310]}
\abbr[jleo]{JLEO}{Journal of Law, Economics, \& Organization [ISSN: 1465-7341]}
\abbr[ijomi]{[n.abbr.]}{International Journal of Medical Informatics [ISSN: 1386-5056]}
\abbr[slr]{[n.abbr.]}{Stanford Law Review [ISSN: 00389765]}
\abbr[bise]{BISE}{Business \& Information Systems Engineering [ISSN: 1867-0202]}
\abbr[joals]{[n.abbr.]}{Journal of Academic Librarianship [ISSN: 0099-1333]}
\abbr[eait]{[n.abbr.]}{Ethics and Information Technology [ISSN: 1388-1957]}
\abbr[jais]{JAIS}{Journal of the Association for Information Systems [ISSN:
1536-9323]}
\abbr[josas]{[n.abbr.]}{Journal of Systems and Software [ISSN: 0164-1212]}
\abbr[iialr]{[n.abbr.]}{International Information and Library Review [ISSN: 1057-2317]}
\abbr[sthv]{STHV}{Science, Technology \& Human Values [ISSN: 0162-2439]}
\abbr[cue]{[n.abbr.]}{Computers \& Education [ISSN: 0360-1315]}
\abbr[eer]{EER}{European Economic Review [ISSN: 0014-2921]}
\abbr[icc]{ICC}{Industrial and Corporate Change [ISSN: 0960-6491]}
\abbr[ca]{[n.abbr.]}{Cultural Anthropology [ISSN: 1548-1360]}
\abbr[sqj]{[n.abbr.]}{Software Qualilty Journal [ISSN: 0963-9314]}
\abbr[jmir]{JMIR}{Journal of Medical Information Research [ISSN: 1438-8871]}
\abbr[joce]{[n.abbr.]}{Journal of Comparative Economics [ISSN: 0147-5967]}
\abbr[orgsci]{[n.abbr.]}{Organization Science [ISSN: 1047-7039]}
\abbr[iam]{[n.abbr.]}{Information \& Management [ISSN: 0378-7206]}
\abbr[rp]{RP}{Research Policy [ISSN: 0048-7333]}
\abbr[jsis]{JSIS}{Journal of Strategic Information Systems [ISSN: 0963-8687]}
\abbr[isj]{ISJ}{Information Systems Journal [ISSN: 1365-2575]}
\abbr[jise]{JISE}{Journal of Information Science and Engineering [ISSN:
1016-2364]}
\abbr[dss]{DSS}{Decision Support Systems [ISSN: 0167-9236]}
\abbr[cihp]{CiHB}{Computers in Human Behavior [ISSN: 0747-5632]}
\abbr[iep]{IEaP}{Information Economics and Policy [ISSN: 0167-6245]}
\abbr[tosem]{ToSEM}{Transactions on Software Engineering Methodology [ISSN:
1049-331X]}
\abbr[commacm]{CotACM}{Communications of the ACM [ISSN: 0001-0782]}
\abbr[interactions]{[n.abbr.]}{interactions[ISSN: 1072-5520]}
\abbr[jcsc]{JCSC}{Journal of Computing Sciences in [Small] Colleges [ISSN:
1937-4771]}
\abbr[linuxjournal]{LJ}{Linux Journal [ISSN: 1075-3583]}
\abbr[networker]{[n.abbr.]}{netWorker [ISSN: 1091-3556]}
\abbr[queue]{[n.abbr.]}{Queue [ISSN: 1542-7730]}
\abbr[sigmisdb]{SIGMIS Database}{ACM SIGMIS - The Data Base for Advances in
Information Systems [ISSN: 0095-0033]}
\abbr[sigcas]{SIGCAS}{ACM SIGCAS Computers and Society [ISSN: 0095-2737]}
\abbr[sigsoft]{SIGSOFT SEN}{SIGSOFT Software Engineering Notes [ISSN:
0163-5948]}
\abbr[toit]{ToIT}{Transaction on Internet Technology [ISSN: 1533-5399]}
\abbr[sigbul]{SIGCSE Bulletin}{SIGCSE Bulletin [ISSN: 0097-8418]}
\abbr[ubiquity]{Ubiquity}{Ubiquity - The ACM IT Magazine and Forum [ISSN:
1530-2180]}
\abbr[bwv]{BWV}{Berliner Wissenschafts-Verlag GmbH}
\abbr[cr]{CR}{Computer und Recht. Zeitschrift für die Praxis des Rechts der
Informationstechnologien}


\printnomenclature

\bibliography{bibfiles/oscResourcesEn,bibfiles/oscCopiedButNotRead,bibfiles/oscNextActions}

\end{document}

